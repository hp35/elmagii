%
% File: teach/elmagii/gallery/gallery.tex [plain TeX code]
% Github: https://github.com/elmagii/gallery/
% Last change: January 13, 2026
%
% Gallery of primary physicists who formed the theory of electomagnetism
% through history. These physicists and mathematicians essentially formed
% the theory which we are describing in the course ``Elektromagnetism II,
% 1TE626'', held 2022--2026 at Uppsala University, Sweden.
%
% Copyright (C) 2026, Fredrik Jonsson, under Gnu General Public
% License (GPL) v3. See the enclosed LICENSE for details.
%
% This program is free software: you can redistribute it and/or modify
% it under the terms of the GNU General Public License as published by
% the Free Software Foundation, either version 3 of the License, or
% (at your option) any later version.
%
% This program is distributed in the hope that it will be useful,
% but WITHOUT ANY WARRANTY; without even the implied warranty of
% MERCHANTABILITY or FITNESS FOR A PARTICULAR PURPOSE.  See the
% GNU General Public License for more details.
%
% You should have received a copy of the GNU General Public License
% along with this program.  If not, see <https://www.gnu.org/licenses/>.
%
\input macros/epsf.tex
\input macros/eplain.tex

%
% Define the 'boxit' macro from D.E. Knuths "The TeXbook, Exercise 21.3.
%
\def\boxit#1{\vbox{\hrule\hbox{\vrule\kern3pt
  \vbox{\kern3pt#1\kern3pt}\kern3pt\vrule}\hrule}}

%
% Define the macro which "typographically atomizes" the individual portraits.
%
\newdimen\portraitxsize  \portraitxsize=200pt
\newdimen\portraitysize  \portraitysize=300pt
\newdimen\portraitmargin \portraitmargin=14pt

\def\portrait#1#2#3#4{%
  \boxit{%
    \vbox{%
      \boxit{%
        \hbox to \portraitxsize{\epsfysize=164pt\hfil\epsfbox{#1}\hfil}%
      }%
      \hbox{\vbox{\hsize=\portraitxsize\parindent=0pt\hfil{\bf#2}\hfil}}%
      \hbox{\vbox{\hsize=\portraitxsize\parindent=0pt\hfil{(#3)}\hfil}}%
      \hbox{\vbox{\hsize=\portraitxsize\parindent=0pt{#4}}}%
    }%
  }%
}

\def\portraitpair#1#2{%
  \hbox{%
    \boxit{\hbox{\vbox to \portraitysize{\hbox{#1}}}}%
    \hskip\portraitmargin%
    \boxit{\hbox{\vbox to \portraitysize{\hbox{#2}}}}%
  }%
}

\def\portraitpage#1#2#3#4{%
  \portraitpair{#1}{#2}%
  \vskip\portraitmargin%
  \portraitpair{#3}{#4}%
  \vfill\eject%
}

%~\vskip 5pc\goodbreak\noindent{\bf{Persongalleri}}%
%  \writenumberedtocentry{index}{\hskip24pt{Persongalleri}}
%\nobreak\vskip 48pt\noindent\ignorespaces

\noindent
Portr{\"a}ttgalleri inom klassisk elektromagnetism
% Gallery of the principal scientists behind the theory covered in the course.

%
% Jean Le Rond d'Alembert (1717--1783)
%
\def\dalembert{%
  \portrait{images/jean_le_rond_dalembert_scaled_bw.eps}%
    {Jean Le Rond d'Alembert}%
    {1717--1783}%
    {French mathematician, mechanician, physicist, philosopher, and music
    theorist. In this course we are using d'Alembert's formulation of wave
    propagation in the form $f(z-ct)+g(z+ct)$.
    \sidx{d'Alembert, Jean le Rond (1717--1783)}}%
}

%
% Andre-Marie Ampere (1775--1836)
%
\def\ampere{%
  \portrait{images/andre_ampere_scaled_bw.eps}%
    {Andr\'e-Marie Amp\`ere}%
    {1775--1836}%
    {French physicist and mathematician. One of the principal
    founders of classical electromagnetism, which he referred to as
    electrodynamics. Amp\`ere's law for static magnetic fields yields
    $\nabla\times{\bf B}=\mu_0{\bf J}$, later modified by Maxwell to
    include the displacement current.
    \sidx{Amp\`ere, Andr\'e-Marie (1775--1836)}}%
}

%
% Michael Faraday (1791--1867)
%
\def\faraday{%
  \portrait{images/michael_faraday_scaled_bw.eps}%
    {Michael Faraday}%
    {1791--1867}%
    {English chemist and physicist. Discovered the principles underlying
    electromagnetic induction, diamagnetism, and electrolysis. In this
    course, we make frequent use of Faraday's law of induction
    $$
      {\cal E}= -{{d\Phi_{\rm M}}\over{dt}}
        \quad\Leftrightarrow\quad
        \nabla\times{\bf E}=-{{\partial{\bf B}}\over{\partial t}}.
    $$
    \sidx{Faraday, Michael (1791--1867)}}%
}

%
% William Thomson, 1st Baron Kelvin (1824--1907)
%
\def\kelvin{%
  \portrait{images/lord_kelvin_scaled_bw.eps}%
    {William Thomson, Lord Kelvin}%
    {1824--1907}%
    {British mathematician, mathematical physicist and engineer.
    In this course we focus on the Kelvin--Stokes theorem,
    $\def\iint{\mathop{\int\kern-5pt\int}}\iint_V(\nabla\times{\bf a})
    \cdot d{\bf S}=\oint_{\Gamma}{\bf a}\cdot d{\bf l}$, after Lord Kelvin
    and George Stokes, the fundamental theorem for curls.
    \sidx{Kelvin, Lord (1824--1907)}[William Thomson, 1st Baron Kelvin]}%
}

%
% Rickard Wilson (1930--2000)
%
\def\wilson{%
  \portrait{images/rickard_wilson_scaled_bw.eps}%
    {Rickard Wilson}%
    {1930--2000}%
    {Svensk vetenskapsman. Upphovsman till fatilarkalkylen, presenterad 1955.}%
    \sidx{Wilson, Rickard (1930--2000)}%
}

%
% Hans Christian Oersted (1777--1851)
%
\def\oersted{%
  \portrait{images/hans_christian_oersted_scaled_bw.eps}%
    {Hans Christian {\OE}rsted}%
    {1777--1851}%
    {Danish chemist and physicist who discovered that electric currents
    create magnetic fields. The unit Oe (oersted) of the magnetic field
    strength ${\bf H}$ is named after him, and defined as $1\ {\rm Oe}=
    (4\pi)^{-1}\times10^3\ {\rm A}/{\rm m}\approx 79.58\ {\rm A}/{\rm m}$.}%
    \sidx{{\OE}rsted, Hans Christian (1777--1851)}%
}

%
% Hendrik Antoon Lorentz (1853--1928)
%
\def\lorentz{%
  \portrait{images/hendrik_lorentz_scaled_bw.eps}%
    {Hendrik Antoon Lorentz}%
    {1853--1928}%
    {Dutch theoretical physicist who shared the 1902 Nobel
    Prize in Physics with Pieter Zeeman for their discovery and theoretical
    explanation of the Zeeman effect. Derived the Lorentz transformation of
    the special theory of relativity, as well as the Lorentz force
    $
      {\bf F}=q\big({\bf E}+({\bf v}\times{\bf B})\big)
    $,
    describing the force acting upon a moving charged particle.
    \sidx{Lorentz, Hendrik Antoon (1853--1928)}}%
}

%
% Ludvig Lorenz (1829--1891)
%
\def\lorenz{%
  \portrait{images/ludvig_valentin_lorenz_scaled_bw.eps}%
    {Ludvig Lorenz}%
    {1829--1891}%
    {Danish physicist and mathematician. In 1867, Lorenz gave
    completely general integral solutions to the differential equations of
    electromagnetism, including retardation due to the finite speed of
    light, and the introduction of the Lorenz gauge
    $$
      \nabla\cdot{\bf A}+
        {{1}\over{c^2}}{{\partial\phi}\over{\partial t}} = 0.
    $$
    \sidx{Lorenz, Ludvig (1829--1891)}}%
}

%
% Charles-Augustin de Coulomb (1736--1806)
%
\def\coulomb{%
  \portrait{images/charles_de_coulomb_scaled_bw.eps}%
    {Charles-Augustin de Coulomb}%
    {1736--1806}%
    {French officer, engineer, and physicist. Discoverer of the law
    describing the electrostatic force of attraction and repulsion,
    $$
      {\bf F}={{qq'}\over{4\pi\varepsilon_0}}
        {{({\bf x}-{\bf x}')}\over{|{\bf x}-{\bf x}'|^3}}.
    $$
    \sidx{de Coulomb, Charles-Augustin (1736--1806)}\sidx{Coulombs kraftlag}}%
}

%
% Jean-Baptiste Biot (1774--1862)
%
\def\biot{%
  \portrait{images/jean-baptiste_biot_scaled_bw.eps}%
    {Jean-Baptiste Biot}%
    {1774--1862}%
    {French physicist, astronomer, and mathematician who co-discovered
    the Biot--Savart law of magnetostatics with F\'elix Savart,
    $$
      {\bf B}({\bf x})={{\mu_0 I}\over{4\pi}}\int^{{\bf x}_b}_{{\bf x}_a}
        {{d{\bf l}'\times({\bf x}-{\bf x}')}\over{|{\bf x}-{\bf x}'|^3}},
    $$
    and studied the polarization of light.}%
    \sidx{Biot, Jean-Baptiste (1774--1862)}\sidx{Biot--Savarts lag}%
}

%
% Felix Savart (1791--1841)
%
\def\savart{%
  \portrait{images/felix_savart_scaled_bw.eps}%
    {F\'elix Savart}%
    {1791--1841}%
    {French physicist and mathematician who is primarily known for the
    Biot--Savart law of electromagnetism,
    $$
      {\bf B}({\bf x})={{\mu_0 I}\over{4\pi}}\int^{{\bf x}_b}_{{\bf x}_a}
        {{d{\bf l}'\times({\bf x}-{\bf x}')}\over{|{\bf x}-{\bf x}'|^3}},
    $$
    co-discovered with Jean-Baptiste Biot.}%
    \sidx{Savart, F\'elix (1791--1841)}\sidx{Biot--Savarts lag}%
}

%
% Oliver Heaviside (1850--1925)
%
\def\heaviside{%
  \portrait{images/oliver_heaviside_scaled_bw.eps}%
    {Oliver Heaviside}%
    {1850--1925}%
    {British mathematician and electrical engineer who invented a new
    technique for solving differential equations (equivalent to the
    Laplace transform), independently developed vector calculus, and
    rewrote Maxwell's equations in the form commonly used today.}%
    \sidx{Heaviside, Oliver (1850--1925)}\sidx{Maxwells ekvationer}%
}

%
% James Clerk Maxwell (1831--1879)
%
\def\maxwell{%
  \portrait{images/james_clerk_maxwell_scaled_bw.eps}%
    {James Clerk Maxwell}%
    {1831--1879}%
    {Scottish physicist and mathematician who was responsible for the
    classical theory of electromagnetic radiation, which was the first
    theory to describe electricity, magnetism and light as different
    manifestations of the same phenomenon. Maxwell's equations for
    electromagnetism achieved the second great unification in physics,
    where the first one had been realised by Isaac Newton.}%
    \sidx{Maxwell, James Clerk (1831--1879)}\sidx{Maxwells ekvationer}%
}

%
% Richard Feynman (1918--1988)
%
\def\feynman{%
  \portrait{images/richard_feynman_scaled_bw.eps}%
    {Richard Feynman}%
    {1918--1988}%
    {American theoretical physicist. Developed the path integral formulation
    of quantum mechanics and the development of the theory of quantum
    electrodynamics. For his contributions to the development of quantum
    electrodynamics, Feynman in 1965 received the Nobel Prize in Physics
    along with Julian Schwinger and Shin'ichir{\=o} Tomonaga.}%
    \sidx{Feynman, Richard (1918--1988)}%
    \sidx{Elektrodynamik}[Kvantelektrodynamik]%
}

%
% Pierre-Simon Laplace (1749--1827)
%
\def\laplace{%
  \portrait{images/pierre-simon_de_laplace_scaled_bw.eps}%
    {Pierre-Simon Laplace}%
    {1749--1827}%
    {French polymath, a scholar whose work has been instrumental in the
    fields of physics, astronomy, mathematics, engineering, statistics,
    and philosophy. In this course we are primarily concerned with the
    Laplace equation $\nabla^2\phi=0$ for the scalar electrostatic
    potential $\phi$.}%
    \sidx{Laplace, Pierre-Simon (1749--1827)}%
    \sidx{Laplaces ekvation}\sidx{Laplace-operatorn}%
}

%
% Siméon Denis Poisson (1781--1840)
%
\def\poisson{%
  \portrait{images/simeon_denis_poisson_scaled_bw.eps}%
    {Sim\'eon Denis Poisson}%
    {1781--1840}%
    {French mathematician and physicist. Worked on complex analysis, partial
    differential equations, electricity and magnetism. Predicted the
    {\it Arago spot} in his attempt to disprove the wave theory of
    Augustin-Jean Fresnel. In this course we are primarily concerned
    with the Poisson equation $\nabla^2\phi=\rho/\varepsilon_0$ for the
    scalar electrostatic potential in presence of charges.}%
    \sidx{Poisson, Sim\'eon Denis (1781--1840)}%
    \sidx{Poissons ekvation}%
}

%
% Carl Friedrich Gauss (1777--1855)
%
\def\gauss{%
  \portrait{images/carl_friedrich_gauss_scaled_bw.eps}%
    {Carl Friedrich Gauss}%
    {1777--1855}%
    {German mathematician and astronomer. Director of the G{\"o}ttingen
    Observatory in Germany and professor of astronomy. We are in this
    course primarily concerned with Gauss' theorem
    $$
      \def\oiint{\mathop{\int\kern-8pt\int\kern-13.2pt{\bigcirc}}}
      \def\iiint{\mathop{\int\kern-8pt\int\kern-8pt\int}}
      \iiint(\nabla\cdot{\bf a})\,dV=\oiint{\bf a}\cdot d{\bf S}.
    $$}%
    \sidx{Gauss, Carl Friedrich (1777--1855)}%
    \sidx{Gauss teorem}%
}

%
% George Stokes (1819--1903)
%
\def\stokes{%
  \portrait{images/george_stokes_scaled_bw.eps}%
    {George Stokes, 1st Baronet}%
    {1819--1903}%
    {Irish mathematician and physicist. Formulated the Navier--Stokes
    equations and contributed to the theory of optical polarization
    (Stokes vector). We are in this course primarily concerned with
    Stokes' theorem
    $$
      \def\iint{\mathop{\int\kern-8pt\int}}
      \iint(\nabla\times{\bf a})\cdot d{\bf S}
        =\oint{\bf a}\cdot d{\bf l}
    $$}%
    \sidx{Stokes, George (1819--1902)}%
    \sidx{Stokes teorem}%
}

%
% Franz Ernst Neumann (1798--1895)
%
\def\neumann{%
  \portrait{images/franz_ernst_neumann_scaled_bw.eps}%
    {Franz Ernst Neumann}%
    {1798--1895}%
    {German mineralogist and physicist. Devised the first formulas to
    calculate inductance, and the purely geometrical formula for mutual
    inductance is named after him. In electromagnetism, he is credited
    for introducing the magnetic vector potential ${\bf A}$, as extensively
    used throughout this course.}%
    \sidx{Neumann, Franz Ernst (1798--1895)}%
    \sidx{Neumanns formel}[{\"O}msesidig induktans]%
}

\portraitpage{\dalembert}{\ampere}{\faraday}{\kelvin}
\portraitpage{\wilson}{\oersted}{\lorentz}{\lorenz}
\portraitpage{\coulomb}{\biot}{\savart}{\heaviside}
\portraitpage{\maxwell}{\feynman}{\laplace}{\poisson}
\portraitpage{\gauss}{\stokes}{\neumann}{}

\bye
