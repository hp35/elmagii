%
% File: teach/elmagii/gallery/gallery.tex [plain TeX code]
% Github: https://github.com/elmagii/gallery/
% Last change: January 13, 2026
%
% Gallery of primary physicists who formed the theory of electomagnetism
% through history. These physicists and mathematicians essentially formed
% the theory which we are describing in the course ``Elektromagnetism II,
% 1TE626'', held 2022--2026 at Uppsala University, Sweden.
%
% Copyright (C) 2026, Fredrik Jonsson, under Gnu General Public
% License (GPL) v3. See the enclosed LICENSE for details.
%
% This program is free software: you can redistribute it and/or modify
% it under the terms of the GNU General Public License as published by
% the Free Software Foundation, either version 3 of the License, or
% (at your option) any later version.
%
% This program is distributed in the hope that it will be useful,
% but WITHOUT ANY WARRANTY; without even the implied warranty of
% MERCHANTABILITY or FITNESS FOR A PARTICULAR PURPOSE.  See the
% GNU General Public License for more details.
%
% You should have received a copy of the GNU General Public License
% along with this program.  If not, see <https://www.gnu.org/licenses/>.
%
\input macros/epsf.tex
\input macros/eplain.tex

%
% Define the 'boxit' macro from D.E. Knuths "The TeXbook, Exercise 21.3.
%
\def\boxit#1{\vbox{\hrule\hbox{\vrule\kern3pt
  \vbox{\kern3pt#1\kern3pt}\kern3pt\vrule}\hrule}}

%
% Define the macro which "typographically atomizes" the individual portraits.
%
\newdimen\portraitxsize  \portraitxsize=200pt
\newdimen\portraitysize  \portraitysize=300pt
\newdimen\portraitmargin \portraitmargin=14pt

\def\portrait#1#2#3{%
  \boxit{%
    \vbox{%
      \boxit{%
        \hbox to \portraitxsize{\epsfysize=170pt\hfil\epsfbox{#1}\hfil}%
      }%
      \hbox{\vbox{\hsize=\portraitxsize\parindent=0pt\hfil{\bf#2}\hfil}}%
      \hbox{\vbox{\hsize=\portraitxsize\parindent=0pt{#3}}}%
    }%
  }%
}

\def\portraitpair#1#2{%
  \hbox{%
    \boxit{\hbox{\vbox to \portraitysize{\hbox{#1}}}}%
    \hskip\portraitmargin%
    \boxit{\hbox{\vbox to \portraitysize{\hbox{#2}}}}%
  }%
}

\def\portraitpage#1#2#3#4{%
  \portraitpair{#1}{#2}%
  \vskip\portraitmargin%
  \portraitpair{#3}{#4}%
  \vfill\eject%
}

%~\vskip 5pc\goodbreak\noindent{\bf{Persongalleri}}%
%  \writenumberedtocentry{index}{\hskip24pt{Persongalleri}}
%\nobreak\vskip 48pt\noindent\ignorespaces

\noindent
Persongalleri inom klassisk elektromagnetism
% Gallery of the principal scientists behind the theory covered in the course.

%
% Jean Le Rond d'Alembert (1717--1783)
%
\def\dalembert{%
  \portrait{images/jean_le_rond_dalembert_scaled_bw.eps}%
    {Jean Le Rond d'Alembert (1717--1783)}%
    {French mathematician, mechanician, physicist, philosopher, and music
    theorist. In this course we are using d'Alembert's formulation of wave
    propagation in the form $f(z-ct)+g(z+ct)$.
    \sidx{d'Alembert, Jean le Rond (1717--1783)}}%
}

%
% Andre-Marie Ampere (1775--1836)
%
\def\ampere{%
  \portrait{images/andre_ampere_scaled_bw.eps}%
    {Andr\'e-Marie Amp\`ere (1775--1836)}%
    {French physicist and mathematician. One of the principal
    founders of classical electromagnetism, which he referred to as
    electrodynamics. Amp\`ere's law for static magnetic fields yields
    $\nabla\times{\bf B}=\mu_0{\bf J}$, later modified by Maxwell to
    include the displacement current.
    \sidx{Amp\`ere, Andr\'e-Marie (1775--1836)}}%
}

%
% Michael Faraday (1791--1867)
%
\def\faraday{%
  \portrait{images/michael_faraday_scaled_bw.eps}%
    {Michael Faraday (1791--1867)}%
    {English chemist and physicist. Discovered the principles underlying
    electromagnetic induction, diamagnetism, and electrolysis. Faraday's
    law of induction yields ${\cal E}= −d\Phi_{\rm M}/dt\ \Leftrightarrow
    \ \nabla\times{\bf E}=-{{\partial{\bf B}}/{\partial t}}$.
    \sidx{Faraday, Michael (1791--1867)}}%
}

%
% William Thomson, 1st Baron Kelvin (1824--1907)
%
\def\kelvin{%
  \portrait{images/lord_kelvin_scaled_bw.eps}%
    {William Thomson, 1st Baron Kelvin (1824--1907)}%
    {British mathematician, mathematical physicist and engineer.
    In this course we focus on the Kelvin--Stokes theorem,
    $\def\iint{\mathop{\int\kern-5pt\int}}\iint_V(\nabla\times{\bf a})
    \cdot d{\bf S}=\oint_{\Gamma}{\bf a}\cdot d{\bf l}$, after Lord Kelvin
    and George Stokes, the fundamental theorem for curls.
    \sidx{Kelvin, Lord (1824--1907)}[William Thomson, 1st Baron Kelvin]}%
}

%
% Rickard Wilson (1930--2000)
%
\def\wilson{%
  \portrait{images/rickard_wilson_scaled_bw.eps}%
    {Rickard Wilson (1930--2000)}%
    {Svensk vetenskapsman. Upphovsman till fatilarkalkylen, presenterad 1955.}%
    \sidx{Wilson, Rickard (1930--2000)}%
}

%
% Hans Christian Oersted (1777--1851)
%
\def\oersted{%
  \portrait{images/hans_christian_oersted_scaled_bw.eps}%
    {Hans Christian {\OE}rsted (1777--1851)}%
    {Danish chemist and physicist who discovered that electric currents
    create magnetic fields.
    \sidx{{\OE}rsted, Hans Christian (1777--1851)}}%
}

%
% Hendrik Antoon Lorentz (1853--1928)
%
\def\lorentz{%
  \portrait{images/hendrik_lorentz_scaled_bw.eps}%
    {Hendrik Antoon Lorentz (1853--1928)}%
    {Dutch theoretical physicist who shared the 1902 Nobel
    Prize in Physics with Pieter Zeeman for their discovery and theoretical
    explanation of the Zeeman effect. Derived the Lorentz transformation of
    the special theory of relativity, as well as the Lorentz force,
    describing the force acting upon a moving charged particle,
    ${\bf F}=q\big[{\bf E}+({\bf v}\times{\bf B})\big]$.
    \sidx{Lorentz, Hendrik Antoon (1853--1928)}}%
}

%
% Ludvig Lorenz (1829--1891)
%
\def\lorenz{%
  \portrait{images/ludvig_valentin_lorenz_scaled_bw.eps}%
    {Ludvig Lorenz (1829--1891)}%
    {Danish physicist and mathematician. In 1867, Lorenz gave
    completely general integral solutions to the differential equations of
    electromagnetism, including retardation, reflecting the finite speed of
    light, and the introduction of the Lorenz gauge.
    \sidx{Lorenz, Ludvig (1829--1891)}}%
}

%
% Charles-Augustin de Coulomb (1736--1806)
%
\def\coulomb{%
  \portrait{images/charles_de_coulomb_scaled_bw.eps}%
    {Charles-Augustin de Coulomb (1736--1806)}%
    {French officer, engineer, and physicist. Discoverer of the law
    describing the electrostatic force of attraction and repulsion,
    ${\bf F}=({{qq'}/{4\pi\varepsilon_0}})
    {{({\bf x}-{\bf x}')}/{|{\bf x}-{\bf x}'|^3}}$.
    \sidx{de Coulomb, Charles-Augustin (1736--1806)}\sidx{Coulombs kraftlag}}%
}

\portraitpage{\dalembert}{\ampere}{\faraday}{\kelvin}
\portraitpage{\wilson}{\oersted}{\lorentz}{\lorenz}
\portraitpage{\coulomb}{}{}{}

\bye
