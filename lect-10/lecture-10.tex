%
% File: teach/elmagii/lect-10/lecture-10.tex [plain TeX code]
% Github: https://github.com/elmagii/lect-10/
% Last change: November 25, 2025
%
% Lecture No 10 in the course ``Elektromagnetism II, 1TE626 (2023)'',
% held November 26, 2025, at Uppsala University, Sweden.
%
% Copyright (C) 2022-2025, Fredrik Jonsson, under Gnu General Public
% License (GPL) v3. See the enclosed LICENSE for details.
%
% This program is free software: you can redistribute it and/or modify
% it under the terms of the GNU General Public License as published by
% the Free Software Foundation, either version 3 of the License, or
% (at your option) any later version.
%
% This program is distributed in the hope that it will be useful,
% but WITHOUT ANY WARRANTY; without even the implied warranty of
% MERCHANTABILITY or FITNESS FOR A PARTICULAR PURPOSE.  See the
% GNU General Public License for more details.
%
% You should have received a copy of the GNU General Public License
% along with this program.  If not, see <https://www.gnu.org/licenses/>.
%
\def\coursename{Elektromagnetism II}
\def\coursecode{1TE626}
\def\courseyear{2025}
\def\courserepo{https://github.com/hp35/elmagii/}
\def\lecturenumber{10}
\def\lecturetitle{V{\aa}gutbredning i homogena och isotropa dielektrika}
\def\lecturetitlesub{}
\def\lectureauthor{Fredrik Jonsson}
\def\lectureplace{Uppsala Universitet}
\def\lecturedate{26 november 2025}
%-------------------- BEGIN OF LOCAL MACROS --------------------
\input macros/epsf.tex
\input macros/eplain.tex
\input amssym % to get the {\Bbb E} font (strikethrough E)
\font\ninerm=cmr9
\font\tenssbx=cmssbx10
\font\twelvesc=cmcsc10 at 12 truept
\newif\ifcolors % creates \ifcolors, \colorstrue, \colorsfalse
\colorstrue     % \colorstrue turns colors on, \colorsfalse turns them off
\ifcolors\input color\fi  % to get colored text output
\def\red#1{\ifcolors{\color{red}#1}\else#1\fi\ignorespaces}
\def\ifempty#1{\ifx\relax#1\relax}
\def\initlecture{
  \hsize=150mm\hoffset=4.6mm\vsize=230mm\voffset=7mm
  \topskip=0pt\baselineskip=12pt\parskip=0pt\leftskip=0pt\parindent=15pt
  \ifcolors
    \voffset=-10.2mm\topskip=0pt
  \fi
  \headline={\ifnum\pageno>1\ifodd\pageno\rightheadline\else\leftheadline\fi
    \else\hfill\fi}
  \def\rightheadline{\tenrm{\it F\"orel\"asning \lecturenumber}
    \hfil{\it \coursename, \coursecode\ (\courseyear)}}
  \def\leftheadline{\tenrm{\it \coursename, \coursecode\ (\courseyear)}
    \hfil{\it F\"orel\"asning \lecturenumber}}
  \noindent~\vskip-60pt\hskip-40pt{\epsfbox{macros/UU_logo_color.eps}}
  \vskip-42pt\hfill\vbox{
      \hbox{{\it \coursename, \coursecode\ (\courseyear)}}
      \hbox{{\it Lecture Notes, \lectureauthor}}
      \hbox{{\it Document Revision \today}}
      \hbox{{\it \courserepo}}}\vskip 36pt
  \centerline{\twelvesc F\"orel\"asning \lecturenumber}
  \vskip 24pt\noindent
  \centerline{\twelvesc\lecturetitle}
  \expandafter\ifempty\expandafter{\lecturetitlesub}%
    \else\centerline{\twelvesc\lecturetitlesub}\fi
  \bigskip
  \centerline{\lectureauthor, \lectureplace, \lecturedate}
  \vskip24pt}
\def\section #1 {\medskip\goodbreak\noindent{\tenssbx #1}
  \par\nobreak\smallskip\noindent}
\def\subsection #1 {\medskip\goodbreak\noindent{\it #1}
  \par\nobreak\smallskip\noindent}
\def\iint{\mathop{\int\kern-8pt\int}}
\def\iiint{\mathop{\int\kern-8pt\int\kern-8pt\int}}
\def\oiint{\mathop{\int\kern-8pt\int\kern-13.2pt{\bigcirc}}}
\def\sgn{\mathop{\rm sgn}\nolimits} % sign
\def\Re{\mathop{\rm Re}\nolimits}   % real part
\def\Im{\mathop{\rm Im}\nolimits}   % imaginary part
\def\Tr{\mathop{\rm Tr}\nolimits}   % quantum mechanical trace
\def\eqq{\mathop{\vbox{\hbox{\hskip2pt?}\vskip-6pt\hbox{=}}}}
\def\quote#1{\par\leftskip=36pt\rightskip=36pt\smallskip\noindent#1\par
  \leftskip=0pt\rightskip=0pt\smallskip}
\long\def\plan#1{\par\leftskip=36pt\rightskip=36pt\bigskip
  \noindent{\it Sammanfattning}\smallskip
  \noindent{\it #1}\par\leftskip=0pt\rightskip=0pt}
\def\threepointsummary#1#2#3{\par\leftskip=36pt\rightskip=36pt\bigskip
  \noindent{\it Tre h{\aa}llpunkter i f{\"o}rel{\"a}sningen}\smallskip
  \leftskip=48pt\rightskip=36pt\hangindent=20pt
  \noindent{\it\hbox to 20pt{1. }#1}\smallskip
  \leftskip=48pt\rightskip=36pt\hangindent=20pt
  \noindent{\it\hbox to 20pt{2. }#2}\smallskip
  \leftskip=48pt\rightskip=36pt\hangindent=20pt
  \noindent{\it\hbox to 20pt{3. }#3}\par
  \leftskip=0pt\rightskip=0pt\vfill\eject}
\def\epsfig#1{\bigskip\centerline{\epsfbox{#1}}\medskip}
\def\captionwide{\advance\leftskip by 60pt
  \advance\rightskip by 60pt}
\newdimen\itemindent \itemindent=28pt
\newdimen\hangitemindent \hangitemindent=46pt
\def\litem[#1]{\smallbreak\noindent%
  \hbox to\itemindent{\hfil}\hbox to\itemindent{#1\hfill}%
  \hangindent\hangitemindent\ignorespaces}
\newif\ifshowindex
\showindextrue  % Use \showindextrue and \showindexfalse to enable/disable index
\def\index{\ifshowindex\vfill\eject\section{Index} \readindexfile{i}\fi}
%--------------------- END OF LOCAL MACROS ---------------------

\initlecture

\plan{Maxwells ekvationer och den elektromagnetiska v{\aa}gekvationen
  sammanfattas och vi introducerar fyra t{\"a}mligen allm{\"a}nt giltiga
  approximationer:
    (i)   ${\bf J}_{\rm f}={\bf 0}$ (inga fria str{\"o}mmar),
          \sidx{Str{\"o}mt{\"a}thet}[Fria laddningar]
    (ii)  $\rho=0$ (inga fria laddningar),
          \sidx{Laddningst{\"a}thet}[Fria laddningar $\rho_{\rm f}$]
    (iii) ${\bf P}=\varepsilon_0(\varepsilon_{\rm r}-1){\bf E}$
              (linj{\"a}rt medium med homogent $\varepsilon_{\rm r}({\bf x})
               =\hbox{konstant}$), samt
          \sidx{Elektrisk polarisationsdensitet ${\bf P}$}
    (iv)  ${\bf M}={\bf 0}$ (f{\"o}rsumbar magnetisering, $\mu_{\rm r}\approx1$).
          \sidx{Magnetisering ${\bf M}$}
  Med dessa approximationer omformulerar vi v{\aa}gekvationerna f{\"o}r det
  elektromagnetiska f{\"a}ltet i form av tv{\aa} identiska ekvationer, f{\"o}r
  det elektriska f{\"a}ltet ${\bf E}$ och det magnetiska f{\"a}ltet ${\bf B}$
  respektive, vilka vidare direkt kan reduceras till en skal{\"a}r
  v{\aa}gekvation som $(\partial^2/\partial z^2
  -\mu_0\varepsilon_0\varepsilon_{\rm r}\partial^2/\partial t^2) E(z,t)=0$.
  Utifr{\aa}n denna form identifierar vi att ljushastigheten i vakuum $c_0$
  ges av $c^{-2}_0=\mu_0\varepsilon_0$ samt brytningsindex
  $n=\varepsilon^{1/2}_{\rm r}$, skalandes utbredningshastigheten f{\"o}r de
  elektromagnetiska v{\aa}gorna som $c=c_0/n$.
  Med hj{\"a}lp av d'Alemberts metod tar vi fram den generella l{\"o}sningen
  f{\"o}r v{\aa}g\-ekva\-tionen och visar p{\aa} hur denna i en dimension har
  motpropagerande komponenter som alltid alltid uppfyller
  $E(z,t)=f(z-ct)+g(z+ct)$.

  F{\"o}r tidsharmoniska f{\"a}lt ans{\"a}tter vi komplexv{\"a}rda envelopper
  som ${\bf E}({\bf x},t)=\sum_{\bf k}\Re[{\bf E}_{\bf k}\exp(i{\bf k}
  \cdot{\bf x}-i\omega({\bf k}) t)]$, med samma form f{\"o}r det magnetiska
  f{\"a}ltet, och visar p{\aa} hur denna planv{\aa}gsuppdelning leder till
  en ortogonal koppling mellan det elektriska och magnetiska f{\"a}ltet som
  ${\bf B}_{\bf k}={\bf k}\times{\bf E}_{\bf k}/\omega$.
  \sidx{Planv{\aa}guppdelning}}

\threepointsummary{%
  \sidx{Elektromagnetisk v{\aa}gekvation}
  Den elektromagnetiska v{\aa}gekvationen i homogena dielektrika utan
  n{\"a}rvaro av fria laddningar, str{\"o}mmar eller magnetisering {\"a}r
  identisk f{\"o}r det elektriska och magnetiska f{\"a}ltet,
  $$
    \nabla^2{\bf E}-{{1}\over{c^2}}{{\partial^2{\bf E}}\over{\partial t^2}}=0,
    \qquad\quad
    \nabla^2{\bf B}-{{1}\over{c^2}}{{\partial^2{\bf B}}\over{\partial t^2}}=0,
  $$
  d{\"a}r $c=c_0/n$ {\"a}r den hastighet som v{\aa}gen har i mediet, med
  ljus\-hastigheten i vakuum $c_0=(\mu_0\varepsilon_0)^{-1/2}$ och
  brytnings\-index $n=\varepsilon^{1/2}_{\rm r}$.
}{%
  \sidx{d'Alembert, Jean le Rond (1717--1783)}[d'Alemberts l{\"o}sningsmetod
  (1747)]
  Med d'Alemberts l{\"o}sningsmetod ges generella l{\"o}sningar till
  v{\aa}gekvationen i en dimension som
  $$
    E(z,t) = f(z-ct) + g(z+ct).
  $$
}{%
  Planv{\aa}gsuppdelning av den elektromagnetiska v{\aa}gen med
  komplex-v{\"a}rda envelopper,\sidx{Planv{\aa}guppdelning}
  $$
    \eqalign{
      {\bf E}({\bf x},t)&=\sum_{\bf k}
         \Re[{\bf E}_{\bf k}\exp(i{\bf k}\cdot{\bf x}-i\omega({\bf k}) t)],\cr
      {\bf B}({\bf x},t)&=\sum_{\bf k}
         \Re[{\bf B}_{\bf k}\exp(i{\bf k}\cdot{\bf x}-i\omega({\bf k}) t)].\cr
    }
  $$
}
%----------------------- END OF PREAMBLE -----------------------

\section{Elektromagnetiska v{\aa}gekvationen i homogena dielektrika}
\sidx{Elektromagnetisk v{\aa}gekvation}[I homogent dielektrikum]
Under den f{\"o}rra f{\"o}rel{\"a}sningen tog vi ur Maxwells ekvationer fram
de generella v{\aa}gekvationerna f{\"o}r ${\bf E}$- och ${\bf B}$-f{\"a}lten.
Dessa visade sig ha en gemensam form p{\aa} k{\"a}lltermerna i h{\"o}gerledet
som\sidx{Homogent medium}\sidx{Elektromagnetisk v{\aa}gekvation}
$$
  \eqalign{
    \nabla\times\nabla\times{\bf E}
      +\mu_0\varepsilon_0{{\partial^2{\bf E}}\over{\partial t^2}}&=
         -\mu_0{{\partial}\over{\partial t}}
          \underbrace{
             \bigg({\bf J}_{\rm f}
	        +{{\partial{\bf P}}\over{\partial t}}
	        +\nabla\times{\bf M}\bigg)
                }_{\hbox{gemensam k{\"a}llterm}},\cr
    \nabla\times\nabla\times{\bf B}
      +\mu_0\varepsilon_0{{\partial^2{\bf B}}\over{\partial t^2}}&=
          \mu_0\nabla\times
          \underbrace{
	     \bigg({\bf J}_{\rm f}
	        +{{\partial{\bf P}}\over{\partial t}}
	        +\nabla\times{\bf M}\bigg)
                }_{\hbox{gemensam k{\"a}llterm}}.\cr
  }
$$
{\"A}ven om de generella v{\aa}gekvationerna f{\"o}r ${\bf E}$ och ${\bf B}$
enligt tidigare har en viktig betydelse i sig, i och med deras generalitet och
fr{\aa}nvaro av approximationer, s{\aa} {\"a}r de i rent praktisk mening av
begr{\"a}nsad betydelse eftersom de inte {\"a}nnu klarg{\"o}r hur effekten av
mediet f{\"o}r v{\aa}gpropagationen sl{\aa}r in. Vi kommer d{\"a}rf{\"o}r att
nu g{\"o}ra f{\"o}ljande f{\"o}renklingar:
\sidx{Laddningst{\"a}thet}[Fria laddningar $\rho_{\rm f}$]
\sidx{Str{\"o}mt{\"a}thet}[Fria laddningar]
\sidx{Elektrisk polarisationsdensitet ${\bf P}$}
\sidx{Elektrisk polarisationsdensitet ${\bf P}$}[Linearitet]
\sidx{Polarisationsdensitet {\bf P}}[Elektrisk]
\sidx{Magnetisering ${\bf M}$}
\medskip
\item{$\bullet$}{${\bf J}_{\rm f}={\bf 0}$ (inga fria str{\"o}mmar)}
\item{$\bullet$}{$\rho=0$ (inga fria laddningar)}
\item{$\bullet$}{${\bf P}=\varepsilon_0(\varepsilon_{\rm r}-1){\bf E}$
   ({\it linj{\"a}rt} medium, och vi antar {\"a}ven {\it homogent}
   $\varepsilon_{\rm r}({\bf x})=\hbox{konstant}$)}
\item{$\bullet$}{${\bf M}={\bf 0}$ (f{\"o}rsumbar magnetisering,
   $\mu_{\rm r}\approx1$)}
\medskip
\noindent
Med ``linj{\"a}rt'' medium menar vi h{\"a}r att polarisationsdensiteten
${\bf P}$ kort och gott {\"a}r en linj{\"a}r funktion av det p{\aa}lagda
elektriska f{\"a}ltet ${\bf E}$. Genom att inf{\"o}ra dessa f{\"o}renklingar,
speciellt f{\"o}r polarisationsdensiteten
${\bf P}=\varepsilon_0(1-\varepsilon_{\rm r}){\bf E}$, har vi att
$$
  \eqalign{
    \nabla\times\nabla\times{\bf E}
      +\mu_0\varepsilon_0{{\partial^2{\bf E}}\over{\partial t^2}}
        &=-\mu_0{{\partial^2{\bf P}}\over{\partial t^2}}\cr
        &=-\mu_0\varepsilon_0(\varepsilon_{\rm r}-1)
            {{\partial^2{\bf E}}\over{\partial t^2}},\cr
    \nabla\times\nabla\times{\bf B}
      +\mu_0\varepsilon_0{{\partial^2{\bf B}}\over{\partial t^2}}
        &= \mu_0\nabla\times{{\partial{\bf P}}\over{\partial t}}\cr
        &= \mu_0\varepsilon_0(\varepsilon_{\rm r}-1)
	  \nabla\times{{\partial{\bf E}}\over{\partial t}}\cr
    &= \{\ \hbox{Byt ordning p{\aa}}\ \nabla\times\ \hbox{och}
           \ \partial/\partial t\ \}\cr
    &= \mu_0\varepsilon_0(\varepsilon_{\rm r}-1)
          {{\partial}\over{\partial t}}\nabla\times{\bf E}\cr
    &= \{\ \hbox{Faradays lag}\ \}\cr
    &= -\mu_0\varepsilon_0(\varepsilon_{\rm r}-1)
          {{\partial^2{\bf B}}\over{\partial t^2}}.\cr
  }
$$
Vi ser redan h{\"a}r en fundamental egenskap i elektrodynamiken, n{\"a}mligen
att ekvationerna f{\"o}r v{\aa}gutbredningen av elektriska och magnetiska
f{\"a}lt i fr{\aa}nvaro av p{\aa}verkan av externa k{\"a}llor {\"a}r
{\it identiska}. Detta f{\aa}r till f{\"o}ljd att vi (efter att ha definierat
respektive randvillkor och initialv{\"a}rden till de partiella
differentialekvationerna) kan behandla elektriska och magnetiska f{\"a}lt
p{\aa} samma s{\"a}tt, och om f{\"a}lten har ett gemensamt ursprung, exempelvis
fr{\aa}n en antenn, laser eller liknande, s{\aa} kommer ${\bf E}$- och
${\bf B}$-f{\"a}lten att f{\"o}ljas {\aa}t d{\aa} de f{\"o}ljer {\it exakt}
samma ekvationer.

Vi kan f{\"o}renkla v{\"a}nsterledets spatiala differentialtermer n{\aa}got,
genom att f{\"o}rst konstatera att
$$
  \nabla\times\nabla\times{\bf E}
    =\nabla(\nabla\cdot{\bf E})-\nabla^2{\bf E}.
$$
F{\"o}r v{\aa}gutbredning i {\it homogena} material {\it utan fria laddningar}
blir termen $\nabla(\nabla\cdot{\bf E})$ noll, eftersom
\sidx{Elektrisk f{\"a}ltstyrka ${\bf E}$}[Divergens f{\"o}r]
$$
  \nabla\cdot{\bf E}
    =\nabla\cdot\bigg({{{\bf D}}\over{\varepsilon_0\varepsilon_{\rm r}}}\bigg)
    ={{1}\over{\varepsilon_0\varepsilon_{\rm r}}}\nabla\cdot{\bf D}
    ={{\rho}\over{\varepsilon_0\varepsilon_{\rm r}}}
    =0.
$$
Notera dock att Gauss lag f{\"o}r elektriska f{\"a}ltet formellt kommer fr{\aa}n
$\nabla\cdot{\bf D}=0$, och {\it ej} fr{\aa}n ``$\nabla\cdot{\bf E}=0$'' (som
i grund och botten bara {\"a}r ett specialfall, om {\"a}n vanligt
f{\"o}rekommande).

F{\"o}r det magnetiska f{\"a}ltet g{\"a}ller samma sak, speciellt eftersom vi
{\it alltid} har att $\nabla\cdot{\bf B}=0$. Med andra ord, f{\"o}r
{\it homogena} media utan fria laddningar kan vi ers{\"a}tta
$$
  \nabla\times\nabla\times\qquad\to\qquad-\nabla^2,
$$
och ekvationerna f{\"o}r elektromagnetisk v{\aa}gutbredning antar d{\aa} formen
$$
  \nabla^2{\bf E} - \mu_0\varepsilon_0\varepsilon_{\rm r}
    {{\partial^2{\bf E}}\over{\partial t^2}} = 0,
  \qquad\qquad
  \nabla^2{\bf B} - \mu_0\varepsilon_0\varepsilon_{\rm r}
    {{\partial^2{\bf B}}\over{\partial t^2}} = 0.
$$
Med detta i bagaget kan vi s{\aa} l{\"a}nge g{\aa} vidare med att enbart
studera det elektriska f{\"a}ltet, och konstatera att en analys av det
magnetiska f{\"a}ltet f{\"o}ljer helt analogt. Notera dock att {\"a}ven om
dessa ekvationer skenbart ger vid hand att ${\bf E}$- och ${\bf B}$-f{\"a}lten
ser ut att vara helt frikopplade fr{\aa}n varandra, s{\aa} har de fortfarande
kopplingen sinsemellan via exempelvis Faradays induktionslag. D{\"a}rf{\"o}r
{\"a}r det oftast en mer korrekt v{\"a}g, med f{\"a}rre f{\"a}llor att ramla
ner i vad g{\"a}ller korrekt formulerade randvillkor och initialv{\"a}rden,
om man f{\"o}rst l{\"o}ser f{\"a}ltproblemet f{\"o}r det elektriska f{\"a}ltet
och d{\"a}refter (om man beh{\"o}ver det magnetiska f{\"a}ltet i analysen)
anv{\"a}nda Faradays eller Amp\`eres lag f{\"o}r att direkt l{\"a}nka in
l{\"o}sningen f{\"o}r magnetf{\"a}ltet.

\section{Plana elektromagnetiska f{\"a}lt och d'Alemberts generella l{\"o}sning
         till v{\aa}gekvationen}
\sidx{Plana elektromagnetiska f{\"a}lt}
\sidx{d'Alembert, Jean le Rond (1717--1783)}[d'Alemberts l{\"o}sningsmetod
  (1747)]
Formen med ``$\nabla^2$'' f{\"o}r de elektromagnetiska f{\"a}ltekvationerna
{\"a}r viktig och kan i m{\aa}nga fall l{\"o}sas {\"a}ven analytiskt, givet
att geometrin {\"a}r tillr{\"a}ckligt enkel. Denna form kan anv{\"a}ndas
s{\aa}v{\"a}l f{\"o}r att beskriva diffraktion som tidsberoende fenomen som
dispersion och pulsbreddning. Denna generella form {\"a}r sj{\"a}lvfallet
ocks{\aa} anv{\"a}ndbar som utg{\aa}ngspunkt f{\"o}r analys i cylindriska
eller sf{\"a}riska koordinater, i vilket fall vi f{\aa}r uttrycka $\nabla^2$
i respektive system.

Vi kommer nu dock att ytterligare f{\"o}renkla beskrivningen genom att anta att
v{\aa}gutbredningen sker i form av {\it o{\"a}ndligt utstr{\"a}ckta plana
v{\aa}gor}, f{\"o}r vilka vi kan f{\"o}rsumma spatial breddning eller
fokusering tv{\"a}rs axeln l{\"a}ngs med vilken v{\aa}gorna breder ut sig.
I detta fall ers{\"a}tter vi, om vi antar v{\aa}gutbredning l{\"a}ngs med
$z$-axeln i ett kartesiskt koordinatsystem,
$$
  \nabla^2\qquad\to\qquad{{\partial^2}\over{\partial z^2}}
$$
Eftersom den elektromagnetiska v{\aa}gekvationen s{\aa} l{\aa}ngt i
approximationen f{\"o}rvisso {\"a}r p{\aa} vektoriell form, men med samtliga
koefficienter och operatorer som skal{\"a}rer ({\"a}ven $\nabla^2$!), s{\aa}
kan vi {\"a}ven v{\"a}lja att studera endast en komponent av respektive
f{\"a}lt. F{\"o}r att g{\aa} vidare endast med det elektriska f{\"a}ltet,
eftersom magnetiska f{\"a}ltet som vi sett dessutom f{\"o}ljer analogt, s{\aa}
blir den partiella differentialekvationen omformulerad som
$$
  {{\partial^2 E(z,t)}\over{\partial z^2}}
    -\mu_0\varepsilon_0\varepsilon_{\rm r}
      {{\partial^2 E(z,t)}\over{\partial t^2}} = 0.
$$
Redan h{\"a}r kan vi med viss grad av kunskap om partiella
differentialekvationer dra slutsatsen att hastigheten f{\"o}r v{\aa}gor som
beskrivs av denna ekvation m{\aa}ste ha beloppet
$(\mu_0\varepsilon_0\varepsilon_{\rm r})^{-1/2}$.
Vi kommer d{\"a}rf{\"o}r att, enbart f{\"o}r att f{\"o}renkla notationen,
{\it definiera}
$$
  c \equiv {{1}\over{(\mu_0\varepsilon_0\varepsilon_{\rm r})^{1/2}}},
$$
{\"a}ven om vi s{\aa} l{\"a}nge kommer att h{\aa}lla sj{\"a}lva tolkningen av
$c$ {\"o}ppen fram till dess att vi explicit visat hur l{\"o}sningen ser ut.
\vfill\eject

\subsection{d'Alemberts variabelsubstitution}
Homogena partiella differentialekvationer med termer d{\"a}r derivatorna med
avseende p{\aa} olika variabler {\"a}r separerade {\"a}r skolexempel p{\aa}
fall d{\"a}r vi kan till{\"a}mpa {\it variabelseparation}. Vi kommer dock
h{\"a}r att beskriva ``standards{\"a}ttet'' att angripa v{\aa}gekvationen
p{\aa}.\numberedfootnote{Griffiths v{\"a}ljer tyv{\"a}rr att inte g{\aa}
  igenom d'Alemberts f{\"o}rh{\aa}llandevis enkla f{\"o}rklaring till att
  v{\aa}gekvationen st{\"o}djer godtycklig v{\aa}gform s{\aa} l{\"a}nge som
  vi har argument av formen $z\pm ct$, utan v{\"a}ljer att bara visa att
  dessa l{\"o}sningar uppfyller v{\aa}gekvationen. Se Griffiths sid.~382--385.}
F{\"o}rst av allt, byt variabler till\numberedfootnote{Detta
   tillv{\"a}gag{\aa}ngss{\"a}tt f{\"o}ljer i stort sett den h{\"a}rledning av
   den generella l{\"o}sningen till v{\aa}g\-ekva\-tionen som gjordes 1747 av
   den franske matematikern Jean le Rond d'Alembert, fransk matematiker och
   upplysningsfilosof (1717--1783). Faktum {\"a}r att d'Alembert-operatorn
   \sidx{d'Alembert, Jean le Rond (1717--1783)}[d'Alembert-operator]
   $$
     \square\equiv{{1}\over{c^2}}{{\partial^2}\over{\partial t^2}}-\nabla^2
   $$
   {\"a}r uppkallad efter honom. F{\"o}r en detaljerad om {\"a}n n{\aa}got
   ostrukturerad beskrivning av metodiken, se Wikipedia-artikeln
   {\tt https://en.wikipedia.org/wiki/D\%27Alembert\%27s\_formula}.}
$$
  \xi = z - ct,
  \qquad\qquad
  \eta = z + ct.
$$
F{\"o}r derivatorna med avseende p{\aa} $\xi$ och $\eta$ har vi med
anv{\"a}ndande av kedjeregeln i tv{\aa} variabler att
$$
  \eqalign{
    {{\partial}\over{\partial z}}
      &=\underbrace{{{\partial\xi}\over{\partial z}}}_{=1}
           {{\partial}\over{\partial\xi}}
         +\underbrace{{{\partial\eta}\over{\partial z}}}_{=1}
	   {{\partial}\over{\partial\eta}}
       ={{\partial}\over{\partial\xi}}
         +{{\partial}\over{\partial\eta}}\cr
  &\hskip20pt\Rightarrow
    {{\partial^2 E}\over{\partial z^2}}
       =\bigg({{\partial}\over{\partial\xi}}
         +{{\partial}\over{\partial\eta}}\bigg)
        \bigg({{\partial E}\over{\partial\xi}}
         +{{\partial E}\over{\partial\eta}}\bigg)
       ={{\partial^2 E}\over{\partial\xi^2}}
	 +2{{\partial^2 E}\over{\partial\xi\partial\eta}}
         +{{\partial^2 E}\over{\partial\eta^2}},\cr
    {{\partial}\over{\partial t}}
      &=\underbrace{{{\partial\xi}\over{\partial t}}}_{=-c}
           {{\partial}\over{\partial\xi}}
         +\underbrace{{{\partial\eta}\over{\partial t}}}_{=c}
	   {{\partial}\over{\partial\eta}}
       =-c\bigg({{\partial}\over{\partial\xi}}
         -{{\partial}\over{\partial\eta}}\bigg)\cr
  &\hskip20pt\Rightarrow
    {{\partial^2 E}\over{\partial t^2}}
       =c^2\bigg({{\partial}\over{\partial\xi}}
         -{{\partial}\over{\partial\eta}}\bigg)
        \bigg({{\partial E}\over{\partial\xi}}
         -{{\partial E}\over{\partial\eta}}\bigg)
       =c^2\bigg({{\partial^2 E}\over{\partial\xi^2}}
	 -2{{\partial^2 E}\over{\partial\xi\partial\eta}}
         +{{\partial^2 E}\over{\partial\eta^2}}\bigg).\cr
  }
$$
Detta f{\"o}r {\"o}ver den partiella differentialekvationen f{\"o}r $E$ till
formen
$$
  \underbrace{
  {{\partial^2 E}\over{\partial\xi^2}}
     +2{{\partial^2 E}\over{\partial\xi\partial\eta}}
     +{{\partial^2 E}\over{\partial\eta^2}}
     }_{\displaystyle={{\partial^2E}\over{\partial z^2}}}
   -{{1}\over{c^2}}
  \underbrace{
     c^2\bigg({{\partial^2 E}\over{\partial\xi^2}}
     -2{{\partial^2 E}\over{\partial\xi\partial\eta}}
     +{{\partial^2 E}\over{\partial\eta^2}}\bigg)
     }_{\displaystyle={{\partial^2E}\over{\partial t^2}}} = 0
  \qquad\Leftrightarrow\qquad
  {{\partial^2 E}\over{\partial\xi\partial\eta}} = 0.
$$
Denna partiella differentialekvation (PDE) {\"a}r {\it linj{\"a}r}, s{\aa}
superpositionsprincipen -- som s{\"a}ger att l{\"o}sningar som individuellt
satisfierar ekvationen kan adderas till varandra och fortfarande (som summa
betraktad) bist{\aa} med en l{\"o}sning till ekvationen -- g{\"a}ller generellt.
Den speciella formen p{\aa} ekvationen s{\"a}ger oss ocks{\aa} att om vi finner
en l{\"o}sning s{\aa} kan den {\it enbart} vara en funktion av {\it antingen}
$\xi$ {\it eller} $\eta$, eftersom andraderivatan i den andra, parvisa,
variabeln garanterar att endast l{\"o}sningar av denna form kommer att
satisfiera ekvationen.
\vfill\eject

\subsection{Integration of d'Alemberts transformerade v{\aa}gekvation}
\sidx{d'Alembert, Jean le Rond (1717--1783)}[Integration of d'Alemberts
  transformerade v{\aa}gekvation]
Vi v{\"a}ljer att f{\"o}rst integrera d'Alemberts ekvation i de transformerade
variablerna med avseende p{\aa} $\xi$, vilket resulterar i ett h{\"o}gerled med
en ``integrationskonstant'' i form av en funktion $h$ som {\it endast kan bero
p{\aa} variabeln $\eta$} (eftersom partialderivatan $\partial/\partial\xi$ av
denna ``konstant'' m{\aa}ste bli identiskt noll), som
$$
  (\hbox{``$\int d\xi$`` }\to)\qquad
  {{\partial E}\over{\partial\eta}}=h(\eta).
$$
Om vi nu integrerar denna med avseende p{\aa} den kvarvarande variabeln $\eta$,
s{\aa} kommer det ing{\aa}ende h{\"o}gerledet $h(\eta)$ ist{\"a}llet att bli
den primitiva funktionen, s{\"a}g $g(\eta)=\int h(\eta)\,d\eta=H(\eta)$, samt
att en till ``integrationskonstant'' som endast beror av $\xi$ m{\aa}ste
l{\"a}ggas till, s{\"a}g en funktion $f(\xi)$, med f{\"o}ljd att
$$
  (\hbox{``$\int d\eta$`` }\to)\qquad
  E=f(\xi)+\underbrace{g(\eta)}_{=H(\eta)}.
$$
Om vi ers{\"a}tter de transformerade variablerna med de ursprungliga,
$\xi=z-ct$ och $\eta=z+ct$, s{\aa} erh{\aa}ller vi allts{\aa} den generella
l{\"o}sningen som
$$
  E = f(\underbrace{z-ct}_{=\xi}) + g(\underbrace{z+ct}_{=\eta}).
$$
Denna generella l{\"o}sning beskriver dels en v{\aa}g $f(z-ct)$ som r{\"o}r sig
i {\it positiv} $z$-led med hastigheten $c$, dels en v{\aa}g $g(z+ct)$ som
r{\"o}r sig i {\it negativ} $z$-led med samma hastighet. B{\aa}da dessa
v{\aa}gor {\"a}r generella till sin form, och beh{\"o}ver ej vara ``klassiska
harmoniska sinusv{\aa}gor''. V{\aa}gorna beh{\"o}ver ej heller vara repetitiva,
utan kan lika v{\"a}l beskriva solit{\"a}ra pulser.\sidx{Motpropagerande
v{\aa}gor}

\subsection{Tolkning av reducerad fashastighet och brytningsindex}
\sidx{Brytningsindex $n$}
\sidx{Ljushastighet}[I dielektrikum, $c$]
\sidx{Ljushastighet}[I vakuum, $c_0$]
\sidx{Magnetisk permeabilitet}[Vakuumpermeabilitet $\mu_0$]
\sidx{Elektrisk permittivitet}[Vakuumpermittivitet $\varepsilon_0$]
\sidx{Elektrisk permittivitet}[Relativ permittivitet $\varepsilon$]
Det intressanta h{\"a}r {\"a}r att vi nu har en tolkning av
$$
  c \equiv {{1}\over{(\mu_0\varepsilon_0\varepsilon_{\rm r})^{1/2}}}
$$
som varandes den hastighet som den elektromagnetiska v{\aa}gr{\"o}relsen har i
mediet, och d{\aa} ljus\-hastig\-heten i vakuum {\"a}r
$$
  c_0 \equiv {{1}\over{(\mu_0\varepsilon_0)^{1/2}}}
      = 299\,792\,458\ {\rm m}/{\rm s}\qquad(\hbox{exakt})
$$
s{\aa} inneb{\"a}r det att vi i $c$ har en v{\aa}g som propagerar
l{\aa}ngsammare med en faktor $n$ enligt
$$
  c = c_0/n,
$$
d{\"a}r
$$
  n \equiv\varepsilon^{1/2}_{\rm r}
$$
{\"a}r {\it brytningsindex} f{\"o}r mediet.

\subsection{Initialvillkor}
\sidx{Initialvillkor f{\"o}r v{\aa}gpropagation}
F{\"o}r den formella l{\"o}sningen f{\"o}r $E(z,t)$ {\"o}ver en dom{\"a}n,
s{\"a}g, $0\le z\le L$, beh{\"o}ver vi dessutom dels initialv{\"a}rdet
$$
  E(z,0) = f(z) + g(z),
$$
men {\"a}ven tidsderivatan
$$
  {{\partial E(z,t)}\over{\partial t}}\bigg|_{t=0}
    = -c{{df(z)}\over{dz}} + c{{dg(z)}\over{dz}},
$$
varefter vi kan integrera l{\"o}sningen och f{\aa} fram tidsberoendet hos de
fram{\aa}t- och bak{\aa}tpropagerande v{\aa}gorna. F{\"o}r detaljer kring
d'Alemberts l{\"o}sningsmetodik f{\"o}r dessa initialv{\"a}rdesproblem, se
praktiskt taget vilken standardbok som helst f{\"o}r en grundkurs i partiella
differentialekvationer.\numberedfootnote{Exempelvis D.~W.~Trim, {\it Applied
Partial Differential Equations} (PWS-Kent, 1990), s.~48.}

\section{Tidsharmoniska f{\"a}lt och planv{\aa}guppdelning}
\sidx{Tidsharmoniska f{\"a}lt}
\sidx{Planv{\aa}guppdelning}
I f{\"o}reg{\aa}ende sektion visade vi f{\"o}r en plan v{\aa}g att vi har en
generell l{\"o}sning i form av en fram{\aa}tg{\aa}ende och en
bak{\aa}tg{\aa}ende v{\aa}g med samma fashastighet $c$. Detta visade vi f{\"o}r
en {\it generell v{\aa}gform}, som inte beh{\"o}ver vara av harmonisk
karakt{\"a}r (med vilket vi l{\"o}st formulerat menar av typen
``sinus-l{\"o}sning''). Samtidigt vet vi fr{\aa}n teorin bakom
differentialekvationer ({\"a}ven partiella s{\aa}dana) att
superpositionsprincipen g{\"a}ller, varvid vi kan analysera l{\"o}sningar
f{\"o}r olika delar av ett elektromagnetiskt f{\"a}lt, och d{\"a}refter
sammanfoga den totala l{\"o}sningen f{\"o}r f{\"a}ltet.

\subsection{Komplexv{\"a}rda f{\"a}ltenvelopper}
\sidx{Komplexv{\"a}rda f{\"a}ltenvelopper}
\sidx{Elektrisk f{\"a}ltstyrka ${\bf E}$}[Komplexv{\"a}rd envelopp]
\sidx{Magnetisk fl{\"o}dest{\"a}thet ${\bf B}$}[Komplexv{\"a}rd envelopp]
Formen p{\aa} den homogena differentialekvationen,\numberedfootnote{Den
  f{\"o}ljande behandlingen av plana v{\aa}gor f{\"o}ljer i huvudsak
  Griffiths Kap.~9.2, sid.~393--415.}
$$
  \nabla^2{\bf E} - {{1}\over{c^2}} % \mu_0\varepsilon_0\varepsilon_{\rm r}
    {{\partial^2{\bf E}}\over{\partial t^2}} = 0,
  \qquad\qquad
  \nabla^2{\bf B} - {{1}\over{c^2}} % \mu_0\varepsilon_0\varepsilon_{\rm r}
    {{\partial^2{\bf B}}\over{\partial t^2}} = 0,
$$
ger dels vid hand att det i det h{\"a}r antagna {\it isotropa} och
{\it homogena} mediet g{\aa}r att frikoppla olika polarisationstillst{\aa}nd
fr{\aa}n varandra (eftersom ekvationerna till{\aa}ter att vi helt enkelt
skal{\"a}r±-multipli\-cerar med en enhetsvektor {\aa}t n{\aa}got h{\aa}ll
och direkt f{\aa}r en frikopplad skal{\"a}r partiell differentialekvation
f{\"o}r just det polarisationstillst{\aa}ndets f{\"a}lt), men {\"a}ven att
vi kan dela f{\"a}lten i deras respektive harmoniska frekvensinneh{\aa}ll,
enligt\numberedfootnote{Ett litet sidosp{\aa}r som kan vara v{\"a}rt att
  n{\"a}mna: Om vi i formen f{\"o}r de homogena differentialekvationerna
  stannar vid att enbart projicera ut den tidsharmoniska delen, genom
  $$
    {\bf E}({\bf x},t)=\sum_{\bf k}\Re[\tilde{\bf E}\exp(-i\omega({\bf k}) t)],
    \qquad\qquad
    {\bf B}({\bf x},t)=\sum_{\bf k}\Re[\tilde{\bf B}\exp(-i\omega({\bf k}) t)],
  $$
  s{\aa} erh{\aa}ller vi egenv{\"a}rdesproblemen f{\"o}r Laplace-operatorn som
  $$
    \nabla^2\tilde{\bf E} - {{\omega^2}\over{c^2}}\tilde{\bf E} = 0,
    \qquad\qquad
    \nabla^2\tilde{\bf B} - {{\omega^2}\over{c^2}}\tilde{\bf B} = 0,
  $$
  i form av {\it Helmholtz ekvationer} f{\"o}r de komplexv{\"a}rda
  tids-envelopperna $\tilde{\bf E}$ och $\tilde{\bf B}$. Dessa ekvationer
  {\"a}r uppkallade efter Hermann von Helmholtz, tysk fysiker (1821--1894).
  \sidx{von Helmholtz, Hermann (1821--1894)}
  \sidx{Laplace-operatorn}[Egenv{\"a}rdesproblem f{\"o}r]}
$$
  {\bf E}({\bf x},t)=\sum_{\bf k}
     \Re[{\bf E}_{\bf k}\exp(i{\bf k}\cdot{\bf x}-i\omega({\bf k}) t)],
  \qquad\qquad
  {\bf B}({\bf x},t)=\sum_{\bf k}
     \Re[{\bf B}_{\bf k}\exp(i{\bf k}\cdot{\bf x}-i\omega({\bf k}) t)],
$$
d{\"a}r ${\bf k}$ {\"a}r {\it v{\aa}gvektorn} f{\"o}r komponenten med komplex
amplitud ${\bf E}_{\bf k}$ samt d{\"a}r $\omega({\bf k})$ {\"a}r
vinkelhastigheten (rad/s) f{\"o}r oscillationen hos f{\"a}lten i tid.
Om vi har generella v{\aa}gformer som inte {\"a}r av ``sinus-typ'', s{\aa} kan
vi alltid f{\aa} fram dessa genom addition av harmoniska f{\"a}lt till en
{\it Fourier-utveckling}, med respektive amplitud hos varje frekvenskomponent
som {\it Fourier-koefficienter}.\sidx{V{\aa}gvektor ${\bf k}$}

Denna tids- och rums-harmoniska form f{\"o}r {\"o}ver de partiella
differentialekvationerna f{\"o}r f{\"a}lten p{\aa}
formen\numberedfootnote{Eftersom $\nabla\to{\bf k}\cdot$ och
  ${{\partial}\over{\partial t}}\to -i\omega$.}
$$
  \Big(k^2 - {{\omega^2}\over{c^2}}\Big) {\bf E} = 0,
  \qquad\qquad
  \Big(k^2 - {{\omega^2}\over{c^2}}\Big) {\bf B} = 0,
$$
fr{\aa}n vilket vi direkt (f{\"o}r noll-skilda f{\"a}lt) f{\"o}r beloppet
$k=|{\bf k}|$ f{\"o}r v{\aa}gvektorn f{\aa}r att
$$
  k = {{\omega}\over{c}}
    = {{\omega}\over{(c_0/n)}},
$$
d{\"a}r, f{\"o}r att rekapitulera,
$$
  \hskip100pt\hphantom{ljushastigheten}
  c_0 = {{1}\over{\sqrt{\mu_0\varepsilon_0}}},
  \hskip60pt(\hbox{ljushastigheten i vakuum})
$$
samt
$$
  \hskip69pt\hphantom{brytningsindex}
  n = \sqrt{\varepsilon_{\rm r}}.
  \hskip69pt(\hbox{brytningsindex})
$$

\subsection{Ortogonalitet mellan f{\"a}lt och v{\aa}gvektor}
\sidx{Komplexv{\"a}rda f{\"a}ltenvelopper}
\sidx{Ortogonalitet mellan f{\"a}lt och v{\aa}gvektor}
Med den tids- och rums-harmoniska formen f{\aa}r vi {\"a}ven direkt ett par
andra karakteristiska egenskaper f{\"o}r elektromagnetisk v{\aa}gutbredning
i homogena och isotropa material, exempelvis fr{\aa}n Faradays lag, att
$$
  \nabla\times{\bf E}=-{{\partial{\bf B}}\over{\partial t}}
  \qquad\Rightarrow\qquad
  i{\bf k}\times{\bf E}_{\bf k} = i\omega{\bf B}_{\bf k}
  \qquad\Leftrightarrow\qquad
  {\bf B}_{\bf k}={\bf k}\times{\bf E}_{\bf k}/\omega,
$$
fr{\aa}n vilket vi direkt har ortogonalitet mellan den elektriska
f{\"a}ltstyrkan ${\bf E}_{\bf k}$ och magnetiska fl{\"o}des\-t{\"a}t\-heten
${\bf B}_{\bf k}$ som
$$
  {\bf E}_{\bf k}\cdot{\bf B}_{\bf k}=0,
$$
men {\"a}ven ortogonatlitet mellan magnetiska fl{\"o}dest{\"a}theten och
v{\aa}gvektorn, d{\aa}\sidx{V{\aa}gvektor ${\bf k}$}
$$
  {\bf k}\cdot{\bf B}_{\bf k}=0.
$$
Fr{\aa}n Gauss lag\numberedfootnote{I fr{\aa}nvaro av fria elektriska
  laddningar, $\nabla\cdot{\bf D}=\nabla\cdot(\varepsilon_0\varepsilon_{\rm r}
  {\bf E})=\varepsilon_0\varepsilon_{\rm r}\nabla\cdot{\bf E}=0$ med
  $\varepsilon_{\rm r}$ som oberoende av rumskoordinater.}
har vi {\"a}ven motsvarande ortogonalitet mellan den elektriska f{\"a}ltstyrkan
och v{\aa}gvektorn som
$$
  \nabla\cdot{\bf E}=0
  \qquad\Rightarrow\qquad
  {\bf k}\cdot{\bf E}_{\bf k}=0.
$$
Detta sammantaget inneb{\"a}r att vektorerna $({\bf E}_{\bf k},{\bf B}_{\bf k},
{\bf k})$ bildar ett {\it h{\"o}gerhands-orienterat system}. Med andra ord,
s{\aa} kan vi sammanfatta detta med att de elektriska och magnetiska f{\"a}lten
i homogena och isotropa media alltid {\"a}r ortogonala mot varandra samt
ortogonala mot v{\aa}gvektorn (riktningen) f{\"o}r v{\aa}gutbredningen.
\epsfig{figs/ebk.1}

\section{Poynting-vektorn}
\sidx{Poynting-vektorn ${\bf S}$}[Effekt transporterat av elektromagnetiskt
  f{\"a}lt]
Som ett m{\aa}tt p{\aa} den effekt som transporteras per ytenhet i
tv{\"a}rsnitt av ett elektromagnetiskt f{\"a}lt, definieras
{\it Poynting-vektorn} ${\bf S}$ som
$$
  {\bf S}={\bf E}\times{\bf H},
$$
med enheten ${\rm W}/{\rm m}^2$.
\vfill\eject

\section{Sammanfattning av F{\"o}rel{\"a}sning~10
  -- V{\aa}gutbredning i homogena och isotropa dielektrika}
\item{$\bullet$}{Den generella elektromagnetiska v{\aa}gekvationen,
  $$
    \eqalign{
      \nabla\times\nabla\times{\bf E}
        +\mu_0\varepsilon_0{{\partial^2{\bf E}}\over{\partial t^2}}&=
           -\mu_0{{\partial}\over{\partial t}}
            \underbrace{
               \bigg({\bf J}_{\rm f}
  	        +{{\partial{\bf P}}\over{\partial t}}
  	        +\nabla\times{\bf M}\bigg)
                  }_{\hbox{gemensam k{\"a}llterm}},\cr
      \nabla\times\nabla\times{\bf B}
        +\mu_0\varepsilon_0{{\partial^2{\bf B}}\over{\partial t^2}}&=
            \mu_0\nabla\times
            \underbrace{
  	     \bigg({\bf J}_{\rm f}
  	        +{{\partial{\bf P}}\over{\partial t}}
  	        +\nabla\times{\bf M}\bigg)
                  }_{\hbox{gemensam k{\"a}llterm}}.\cr
    }
  $$}
\item{$\bullet$}{Fyra allm{\"a}nna f{\"o}renklingar,
  \litem[1.]{${\bf J}_{\rm f}={\bf 0}$ (inga fria str{\"o}mmar)}
  \litem[2.]{$\rho=0$ (inga fria laddningar)}
  \litem[3.]{${\bf P}=\varepsilon_0(\varepsilon_{\rm r}-1){\bf E}$
             ({\it linj{\"a}rt} medium, och vi antar {\"a}ven homogent
             $\varepsilon_{\rm r}({\bf x})=\hbox{konstant}$)}
  \litem[4.]{${\bf M}={\bf 0}$ (f{\"o}rsumbar magnetisering,
             $\mu_{\rm r}\approx1$)}
  \smallskip
  leder till att v{\aa}gekvationerna antar formen
  $$
    \nabla^2{\bf E}-{{1}\over{c^2}}{{\partial^2{\bf E}}\over{\partial t^2}}=0,
    \qquad\quad
    \nabla^2{\bf B}-{{1}\over{c^2}}{{\partial^2{\bf B}}\over{\partial t^2}}=0,
  $$
  d{\"a}r $c=c_0/n$ {\"a}r den hastighet som v{\aa}gen har i mediet, med
  ljus\-hastigheten i vakuum som $c_0=1/\sqrt{\mu_0\varepsilon_0}$ och
  brytnings\-index $n=\sqrt{\varepsilon_{\rm r}}$.}
\item{$\bullet$}{Med d'Alemberts l{\"o}sningsmetod ges generella l{\"o}sningar
  till v{\aa}gekvationen i en dimension som
  $$
    E(z,t) = f(z-ct) + g(z+ct).
  $$}
\item{$\bullet$}{Vi utf{\"o}r en planv{\aa}gsuppdelning av den elektromagnetiska
  v{\aa}gen med assistans av komplex-v{\"a}rda envelopper ${\bf E}_{\bf k}$ och
  ${\bf B}_{\bf k}$ som
  $$
    \eqalign{
      {\bf E}({\bf x},t)&=\sum_{\bf k}
         \Re[{\bf E}_{\bf k}\exp(i{\bf k}\cdot{\bf x}-i\omega({\bf k}) t)],\cr
      {\bf B}({\bf x},t)&=\sum_{\bf k}
         \Re[{\bf B}_{\bf k}\exp(i{\bf k}\cdot{\bf x}-i\omega({\bf k}) t)].\cr
    }
  $$}
\item{$\bullet$}{
  De komplexv{\"a}rda envelopperna i planv{\aa}gsuppdelningen {\"a}r ortogonala
  enligt
  $$
    {\bf B}_{\bf k}={\bf k}\times{\bf E}_{\bf k}/\omega,\qquad
    {\bf E}_{\bf k}\cdot{\bf B}_{\bf k}=0,\qquad
    {\bf k}\cdot{\bf B}_{\bf k}=0,\qquad
    {\bf k}\cdot{\bf E}_{\bf k}=0.
  $$}
\item{$\bullet$}{Den effekt som per ytenhet transporteras i tv{\"a}rsnitt av
  en elektromagnetisk v{\aa}g ges av {\it Poynting-vektorn},
  $$
    {\bf S}={\bf E}\times{\bf H},
  $$
  med enheten ${\rm W}/{\rm m}^2$.}
\index
\bye
