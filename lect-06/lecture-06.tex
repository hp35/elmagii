%
% File: teaching/elmagii/2024/lect-06/lecture-06.tex [plain TeX code]
% Last change: November 18, 2024
%
% Lecture No 6 in the course ``Elektromagnetism II, 1TE626 (2024)'',
% held November 19, 2024, at Uppsala University, Sweden.
%
% Copyright (C) 2024, Fredrik Jonsson, under Gnu General Public License
% (GPL) v3. See the enclosed LICENSE for details.
%
% This program is free software: you can redistribute it and/or modify
% it under the terms of the GNU General Public License as published by
% the Free Software Foundation, either version 3 of the License, or
% (at your option) any later version.
%
% This program is distributed in the hope that it will be useful,
% but WITHOUT ANY WARRANTY; without even the implied warranty of
% MERCHANTABILITY or FITNESS FOR A PARTICULAR PURPOSE.  See the
% GNU General Public License for more details.
%
% You should have received a copy of the GNU General Public License
% along with this program.  If not, see <https://www.gnu.org/licenses/>.
%
\input macros/epsf.tex
\input macros/eplain.tex
\font\ninerm=cmr9
\font\twentyrm=cmr12 at 20 truept
\font\twelvesc=cmcsc10
\input amssym % to get the {\Bbb E} font (strikethrough E)
\def\lecture #1 {\hsize=150mm\hoffset=4.6mm\vsize=230mm\voffset=7mm
  \topskip=0pt\baselineskip=12pt\parskip=0pt\leftskip=0pt\parindent=15pt
  \headline={\ifnum\pageno>1\ifodd\pageno\rightheadline\else\leftheadline\fi
    \else\hfill\fi}
  \def\rightheadline{\tenrm{\it F\"orel\"asning #1}
    \hfil{\it Elektromagnetism II, 1TE626 (2024)}}
  \def\leftheadline{\tenrm{\it Elektromagnetism II, 1TE626 (2024)}
    \hfil{\it F\"orel\"asning #1}}
  \noindent~\vskip-60pt\hskip-40pt{\epsfbox{macros/UU_logo_color.eps}}
  \vskip-42pt\hfill\vbox{\hbox{{\it Elektromagnetism II, 1TE626 (2024)}}
  \hbox{{\it Lecture Notes, Fredrik Jonsson}}}\vskip 36pt
    \centerline{\twelvesc F\"orel\"asning #1}
  \vskip 24pt\noindent}
\def\section #1 {\medskip\goodbreak\noindent{\bf #1}
  \par\nobreak\smallskip\noindent}
\def\subsection #1 {\smallskip\goodbreak\noindent{\it #1}
  \par\nobreak\smallskip\noindent}
\def\iint{\mathop{\int\kern-8pt\int}}
\def\iiint{\mathop{\int\kern-8pt\int\kern-8pt\int}}
\def\oiint{\mathop{\int\kern-8pt\int\kern-13.2pt{\bigcirc}}}
\def\Re{\mathop{\rm Re}\nolimits} % real part
\def\Im{\mathop{\rm Im}\nolimits} % imaginary part
\def\Tr{\mathop{\rm Tr}\nolimits} % quantum mechanical trace
\def\eqq{\mathop{\vbox{\hbox{\hskip2pt?}\vskip-6pt\hbox{=}}}}

\lecture{6}
\centerline{\twelvesc Elektriska f\"alt i material}
\centerline{Fredrik Jonsson, Uppsala Universitet, 19 november 2024}
\vskip24pt
\noindent
Vi kommer i denna f{\"o}rel{\"a}sning att behandla hur ett elektriskt icke
ledande material (dielektrikum) upptr{\"a}der d{\aa} det p{\aa}verkas av ett
p{\aa}lagt elektriskt f{\"a}lt, specifikt hur vi elektriskt polariserar mediet
genom att de atom{\"a}ra eller molekyl{\"a}ra elektriska dipolerna linjeras i
f{\"a}ltet.

Om man skall sammanfatta vad vi rent makroskopiskt kan uppfatta g{\"a}llande
elektrisk polarisering i material, s{\aa} kan vi konstatera att denna
polarisering {\"a}r sj{\"a}lva grunden till all form av {\it reflektion},
samt i f{\"o}rl{\"a}ngningen {\"a}ven {\it refraktion} (ljusbrytning) och
{\it diffraktion}.
I och med att allt vi {\"o}verhuvudtaget ser {\"a}r spritt ljus, m{\"o}jligen
undantaget en blick direkt in i en ljusk{\"a}lla, s{\aa} kan man d{\"a}rmed
s{\"a}ga att elektrisk polarisation hos matrial {\"a}r k{\"a}llan till praktiskt
allt vi ser med v{\aa}ra {\"o}gon. Elektriska f{\"a}lt i material och elektrisk
polarisation har med andra ord en mycket konkret effekt p{\aa} v{\aa}rt dagliga
liv.

\section{V{\"a}rldskarta}
V{\"a}xelverkan mellan elektriska f{\"a}lt och materia {\"a}r ett komplext
{\"a}mne som generellt m{\aa}ste utg{\aa} fr{\aa}n de intrinsiska (inneboende)
egenskaperna hos atomerna som utg{\"o}r mediet, hur dessa {\"a}r arrangerade
(exempelvis i gitter, som i kristallina material, eller slumpvis, som i en gas
eller ett amorft medium).
Vi kommer h{\"a}r sj{\"a}lvfallet inte att g{\aa} igenom kvantmekaniken i sig,
men det kan vara bra att ha i huvudet vad vi siktar mot att f{\"o}rs{\"o}ka
f{\aa} till en modell f{\"o}r.

Grunden f{\"o}r hur vi g{\aa}r fr{\aa}n atomer och deras kvantmekniska
v{\aa}gfuktioner upp till makro\-skopiskt observerbara effekter som relativ
permittivitet (elektrisk susceptibilitet) kan generellt sammanfattas med
f{\"o}ljande v{\"a}rldskarta:
$$
  \matrix{
  \psi_j({\bf r},t)
        &\hskip40pt\hbox to 220pt{V{\aa}gfuktion (fr{\aa}n Schr{\"o}dinger;
        beroende av ${\bf E}$)\hfill}\cr
  \downarrow&\cr
  \hat\rho=\sum_j p_j|\psi_j\rangle\langle\psi_j|
        &\hskip40pt\hbox to 220pt{Densitetsmatris/operator\hfill}\cr
  \downarrow&\cr
  {\bf p}=\langle\hat{\bf p}\rangle={\rm Tr}(\hat\rho\hat{\bf p})
        &\hskip40pt\hbox to 220pt{Elektrisk polarisation (dipol,
           ${\rm C}\cdot{\rm m}$)\hfill}\cr
  \downarrow&\cr
  \displaystyle
  {\bf P}={{\Delta{\bf p}}\over{\Delta V}}
        &\hskip40pt\hbox to 220pt{Elektrisk polarisationsdensitet
            (makrosk., ${\rm C}/{\rm m}^2$)\hfill}\cr
  \downarrow&\cr
  {\bf P}=\varepsilon_0\chi_{\rm e}{\bf E}
         =\varepsilon_0(1-\varepsilon_{\rm r}){\bf E}
        &\hskip40pt\hbox to 220pt{Elektrisk susceptibilitet $\chi_{\rm e}$
                         (enhetsl{\"o}s)\hfill}\cr
  \downarrow&\cr
  {\bf D}\equiv\varepsilon_0(1+\chi_{\rm e}){\bf E}
                    =\varepsilon_0\varepsilon_{\rm r}{\bf E}
        &\hskip40pt\hbox to 220pt{Elektrisk fl{\"o}dest{\"a}thet
                                  {\bf D} (${\rm C}/{\rm m}^2$)\hfill}\cr
  }
$$
Fr{\aa}n den elektriska susceptibiliteten $\chi_{\rm e}$ {\"o}ppnar sig
d{\"a}refter en m{\"a}ngd olika discipliner inom fysik, kemi och till{\"a}mpad
matematik. N{\aa}gra exempel:
\medskip
\item{$\bullet$}{Brytningsindex $n=\varepsilon^{1/2}_{\rm r}$: Linser, teleskop,
                 optiska fibrer, speglar, radar. ``Allt vi kan se.''}
\item{$\bullet$}{Radar, ekolod, skiktr{\"o}ntgen, radiov{\aa}gspropagation
                 {\"o}ver l{\aa}nga distanser.}
\item{$\bullet$}{Dispersion med v{\aa}gl{\"a}ngdsberoende brytningsindex,
                 $n=n(\omega)$.}
\item{$\bullet$}{Dubbelbrytning med brytningsindex beroende p{\aa} riktning
                 och polarisation hos ljus..}
\item{$\bullet$}{Solitoner och sj{\"a}lvfokuserande ljus, med
                 intensitetsberoende brytningsindex $n=n(I,\omega)$.}
\item{$\bullet$}{Optisk {\"o}vertonsgenerering med ickelinj{\"a}r optik,
                 $\omega_3=\omega_1+\omega_2$.}
\medskip
\noindent
Not: Griffiths anv{\"a}nder generellt symbolen ``$V$'' f{\"o}r att beteckna
potentialer, vilket {\"a}r lite olyckligt, d{\aa} $V$ normalt anv{\"a}nds
f{\"o}r att beteckna s{\aa}v{\"a}l volym som elektrisk sp{\"a}nning (elektrisk
potentialskillnad). Vi kommer h{\"a}r d{\"a}rf{\"o}r att ist{\"a}llet
genomg{\aa}ende anv{\"a}nda $\phi$ f{\"o}r att beteckna en potential.

\section{Klassisk modell f{\"o}r dipoler}
Den enklast m{\"o}jliga modell vi kan f{\"o}rest{\"a}lla oss f{\"o}r hur ett
medium kan polariseras {\"a}r genom att betrakta var och en av de (neutrala)
atomerna som utg{\"o}r materialet som best{\aa}ende av en positiv punktladdning
(atomk{\"a}rnan) och ett sf{\"a}risk-symmetriskt moln av homogen laddning
(elektronmolnet) omg{\"a}rdande k{\"a}rnan. D{\aa} mediet uts{\"a}tts f{\"o}r
ett externt p{\aa}lagt elektriskt f{\"a}lt kommer atomernas negativt laddade
elektronmoln att dras mot ``$+$'', medan den positivt laddade k{\"a}rnan
ist{\"a}llet dras mot ``$-$''. Detta ger upphov till en elektrisk polarisering
av atomen, vilket resulterar i att vi har en kollektiv polarisering av mediet
p{\aa} makroskopisk niv{\aa}.

Observera att den f{\"o}ljande modellen {\"a}r extremt f{\"o}renklad och bara
kan betraktas som en illustrativ modell {\"o}ver v{\"a}xelverkan mellan
f{\"a}lt och materia. I verkligheten skel v{\"a}xelverkan p{\aa} kvantniv{\aa}
d{\"a}r vi projicerar atomens eller molekylens v{\aa}gfunktion p{\aa} en
dipoloperator. Dessutom f{\"o}rsummar denna modell helt och h{\aa}llet
eventuella multipolmoment hos systemet (mer om detta i kommande
f{\"o}rel{\"a}sningar).
\bigskip
\centerline{\epsfbox{figs/dipole.1}}
\medskip
\noindent
Eftersom k{\"a}rnan {\"a}r mycket tyngre {\"a}n elektronmolnet s{\aa} kan vi
dessutom g{\"o}ra det enkelt f{\"o}r oss och helt enkelt se k{\"a}rnan som fix
i rummet och med elektronmolnet r{\"o}rligt kring k{\"a}rnan.
D{\aa} vi applicerar ett elektriskt f{\"a}lt {\"o}ver detta system, vilket vi
s{\aa} l{\aa}ngt antar vara statiskt i den m{\aa}n att f{\"a}ltets
tidsvariation {\"a}r l{\aa}ngsam j{\"a}mf{\"o}rt med naturliga
resonansfrekvenser i det atom{\"a}ra systemet, s{\aa} komer elektronmolnet att
f{\"o}rskjutas relativt atomk{\"a}rnan. Den negativa laddningen kommer att
f{\"o}rskjutas mot positiv elektrisk potential (k{\"a}llan ``$+$'' f{\"o}r
det elektriska f{\"a}ltet), och vi kommer att f{\aa} en resulterande elektrisk
dipol.
\vfill\eject

\bigskip
\centerline{\epsfbox{figs/displace.1}}
\medskip
\noindent
F{\"o}r att ta fram relationen mellan det resulterande dipolmomentet ${\bf p}$
och det externa elektriska f{\"a}ltet ${\bf E}_{\rm ext}$, s{\aa} kan vi
exempelvis betrakta krafterna som verkar p{\aa} k{\"a}rnan. F{\"o}rst av
allt ges kraften p{\aa} k{\"a}rnan fr{\aa}n det externa f{\"a}ltet som
$$
  {\bf F}_{\rm ext}=(+q){\bf E}_{\rm ext}.
$$
Den elektrostatiska kraften ${\bf F}_{\rm e}$ p{\aa} k{\"a}rnan fr{\aa}n det
omgivande elektronmolnet, som str{\"a}var mot att centrera k{\"a}rnan och
elektronmolnen mot varandra, {\"a}r
$$
  {\bf F}_{\rm e}=(+q){\bf E}_{\rm e},
$$
d{\"a}r ${\bf E}_{\rm e}$ {\"a}r det elektriska f{\"a}lt som k{\"a}rnan upplever
fr{\aa}n den negativa elektriska laddningen hos elektronmolnet.

Det elektriska f{\"a}ltet fr{\aa}n elektronmolnet kan tas fram genom att
utnyttja Gauss lag under sf{\"a}risk symmetri {\"o}ver den sf{\"a}r $\Omega$
som {\"a}r centrerad i elektronmolnet och med k{\"a}rnan p{\aa} ytan,
$$
  \iiint_V\nabla\cdot{\bf E}_{\rm e}\,dV
    =\oiint_{\Omega}{\bf E}_{\rm e}\cdot d{\bf A}
    ={{Q}\over{\varepsilon_0}},
$$
d{\"a}r $Q$ {\"a}r den av $\Omega$ inneslutna laddningen.
Med laddningsdensiteten $\rho$ f{\"o}r elektronmolnet, och med detta moln
f{\"o}rskjutet str{\"a}ckan $r$, {\"a}r med andra ord den inneslutna laddningen
$$
  Q={{4\pi r^3}\over{3}}\rho
   =\bigg\{\rho={{(-q)}\over{(4\pi a^3/3)}}\bigg\}
   =-{{r^3}\over{a^3}}q.
$$
Samtidigt har vi fr{\aa}n sf{\"a}risk symmetri hos elektronmolnet att den
radiella komponenten av elektriska f{\"a}ltet p{\aa} sf{\"a}ren med radie
$r$ {\"a}r konstant, och ytintegralen ges d{\"a}rmed som
$$
  \oiint_{\Omega}{\bf E}_{\rm e}\cdot d{\bf A} =E_{\rm e}4\pi r^2.
$$
Om vi s{\"a}tter samman dessa uttryck har vi allts{\aa} att
$$
  E_{\rm e}4\pi r^2 = -{{r^3}\over{a^3}}{{q}\over{\varepsilon_0}}
  \qquad\Leftrightarrow\qquad
  E_{\rm e} = -{{q}\over{4\pi\varepsilon_0}}{{r}\over{a^3}}.
$$
Den {\aa}terst{\"a}llande kraften som verkar p{\aa} k{\"a}rnan fr{\aa}n
elektronmolnet {\"a}r med andra ord
$$
  F_{\rm e} = -{{q^2}\over{4\pi\varepsilon_0}}{{r}\over{a^3}}.
$$
Kraftj{\"a}mvikt inneb{\"a}r d{\"a}rmed att
$$
  F_{\rm ext}+F_{\rm e} = {\bf 0}
  \qquad\Leftrightarrow\qquad
  qE_{\rm ext}-{{q^2}\over{4\pi\varepsilon_0}}{{r}\over{a^3}} = 0
  \qquad\Leftrightarrow\qquad
  r = {{4\pi\varepsilon_0 a^3}\over{q}} E_{\rm ext}.
$$
Det resulterande dipolmomentet {\"a}r d{\"a}rmed
$$
  p = qr
    = \underbrace{4\pi\varepsilon_0 a^3}_{=\alpha} E_{\rm ext}
    = \alpha E_{\rm ext},
$$
d{\"a}r $\alpha$ {\"a}r den s{\aa} kallade {\it atom{\"a}ra
polarisabiliteten}.\numberedfootnote{Exempel p{\aa} v{\"a}rden f{\"o}r denna
    {\aa}terfinns i Griffiths, Tabell 4.1 p{\aa} sidan 168.}
Utifr{\aa}n denna extremt f{\"o}renklade modell kan vi observera ett par
intressanta saker:
\medskip
\item{$\bullet$}{Dipolmomentet {\"a}r en linj{\"a}r funktion av det
                 p{\aa}lagda elektriska f{\"a}ltet. Detta brukar vi beteckna
                 med att {\it systemet {\"a}r linj{\"a}rt}.}
\item{$\bullet$}{Dipolmomentet {\"a}r {\"a}ven riktat {\it l{\"a}ngs med} det
                 p{\aa}lagda (godtyckligt riktade) elektriska f{\"a}ltet.
                 Detta {\"a}r ett beteende vi associerar vi med ett
                 {\it isotropt} medium.}
\item{$\bullet$}{Dipolmomentet beror endast p{\aa} eventuell frekvens hos
                 det p{\aa}lagda elektriska f{\"a}ltet. Detta {\"a}r ett
                 beteende vi associerar vi med ett medium fritt fr{\aa}n
                 {\it dispersion} (frekvensoberoende).}
\item{$\bullet$}{Polarisabiliteten beror p{\aa} ``atomens''
                 storlek\numberedfootnote{``Atomens'' med citat-tecken,
                 d{\aa} vi ju alla vet att atomer inte {\"a}r isolerade
                 sf{\"a}rer som l{\aa}ter sig beskrivas med klassisk
                 mekanik.} i sig.}
\item{$\bullet$}{Polarisabiliteten beror d{\"a}remot {\it inte} p{\aa} den
                 totala laddning som k{\"a}rnan uppb{\"a}r (d.v.s. atomnummer).}
\medskip
\section{Varf{\"o}r inte en modell med punktladdningar ist{\"a}llet?}
Retorisk fr{\aa}ga: {\it Varf{\"o}r konstla till det med ett elektronmoln om vi
{\"a}nd{\aa} effektivt har verkan av molnet riktat in mot (det sf{\"a}riska)
molnets centrum? Skulle vi inte lika kunna f{\aa} till en enklare men lika
kvalitativ modell genom att ist{\"a}llet ans{\"a}tta tv{\aa} punktladdningar
med olika tecken?}

Svaret p{\aa} detta {\"a}r att denna modell direkt kommer att resultera i en
direkt ofysikalisk situa\-tion redan utan ett p{\aa}lagt elektriskt f{\"a}lt,
d{\aa} den {\"o}msesidiga attraktionskraften mellan punktladdningarna i s{\aa}
fall skulle g{\aa} mot o{\"a}ndligheten, d{\aa} Coulombkraften
$$
F_{\rm c}={{q^2}\over{4\pi\varepsilon_0 r^2}}\to\infty,\qquad r\to 0.
$$
I modellen med elektronmolnet kommer kraften att n{\"a}rma sig noll d{\aa}
$r\to 0$, eftersom vi med Gauss lag s{\aa} att s{\"a}ga ``skalar av'' den
inneslutna laddningen lager f{\"o}r lager d{\aa} radien $r$, det vill s{\"a}ga
avst{\aa}ndet mellan punktladdningen $+q$ f{\"o}r k{\"a}rnan och centrum
f{\"o}r elektronmolnet, successivt minskas.


\section{Terminologi}
F{\"o}r att beskriva grundl{\"a}ggande egenskaper hos ett material, och d{\aa}
inte bara ur elektromagnetisk synvinkel, s{\aa} finns n{\aa}gra
grundl{\"a}ggande begrepp som kan vara bra att ha i bakhuvudet:
\medskip
\item{$\bullet$}{{\it Isotropt:} Invariant under godtycklig rotation.}
\item{}{{\it ``Mediet beter sig lokalt likadant oavsett i vilket riktning man
      tittar.''}}
\smallskip
\item{}{$\displaystyle \qquad\qquad\to P_i=\varepsilon_0\chi_{\rm e}E_i$\hskip20pt (med $\chi_{ij}$ diagonal).}
\smallskip
\item{$\bullet$}{{\it Anisotropt:} Egenskaperna hos mediet i ett fixt
      koordinatsystem (``labbsystem'') {\it {\"a}ndras} under rotation.}
\smallskip
\item{}{$\displaystyle \qquad\qquad\to P_i=\varepsilon_0\sum_{j}\chi_{ij}E_j$.}
\smallskip
\item{$\bullet$}{{\it Homogent:} Egenskaperna hos mediet {\"a}r samma oavsett
      vilken punkt man observerar.}
\item{}{{\it Notera att ett homogent medium fortfarande kan vara anisotropt!}}
\smallskip
\item{}{$\displaystyle \qquad\qquad\to \chi_{ij}=\hbox{konstant}$,
      oberoende av ${\bf r}$.}
\smallskip
\item{$\bullet$}{{\it Inhomogent:} Egenskaperna hos mediet varierar
      beroende p{\aa} var i rummet man observerar dem.}
\item{}{{\it Notera att ett inhomogent men i {\"o}vrigt lokalt isotropt
      medium ger effekter som kan vara beroende p{\aa} riktning!
      Ett vardagligt exempel {\"a}r polarisation hos solljuset som n{\aa}r
      oss vid solnedg{\aa}ng.}}
\smallskip
\item{}{$\displaystyle \qquad\qquad\to \chi_{ij}=\chi_{ij}({\bf r})$.}
\smallskip
\item{$\bullet$}{{\it Dispersion:} Egenskaperna hos mediet varierar beroende
      p{\aa} vilken frekvens ett drivande f{\"a}lt har.}
\item{}{{\it Vardagligt exempel: F{\"a}rger fr{\aa}n vitt ljus som bryts
      i en kristall (eller billig lins).}
\smallskip
\item{}{$\displaystyle \qquad\qquad\to \chi_{ij}=\chi_{ij}(\omega)$.}
\medskip
\section{Anisotropi}
F{\"o}r molekyler som {\"a}r l{\aa}sta i gitter (kristallina strukturer)
{\"a}r polarisabiliteten generellt en {\it tensor} som komponentvis ger hur
polarisabiliteten i en viss riktning p{\aa}verkas av en given komposant hos
det externt p{\aa}lagda elektriska f{\"a}ltet $E_k$, som
$$
  p_j=\alpha_{jk}E_k
  \qquad\Leftrightarrow\qquad
  \pmatrix{p_x\cr p_y\cr p_z}
     =\pmatrix{
         \alpha_{xx}&\alpha_{xy}&\alpha_{xz}\cr
         \alpha_{yx}&\alpha_{yy}&\alpha_{yz}\cr
         \alpha_{zx}&\alpha_{zy}&\alpha_{zz}
         }
      \pmatrix{E_x\cr E_y\cr E_z}.
$$
Denna form svarar f{\"o}r effekter som exempelvis dubbelbrytning och optisk
aktivitet, samt {\"a}ven Faraday-rotation (i vilken ett statiskt p{\aa}lagt
magnetf{\"a}lt skapar en anisotropi).

\section{Elektrisk polarisationsdensitet}
I ett dielektrikum {\"a}r laddningarna bundna till molekylerna i ett effektivt
``moln'' av elektroner kring en k{\"a}rna. Om detta moln av elektroner ligger
centrerat kring k{\"a}rnan finns inga permanenta elektriska dipoler i mediet,
och d{\"a}rmed heller ingen netto-polarisering. Om ett elektriskt f{\"a}lt
l{\"a}ggs p{\aa} {\"o}ver detta dielektrikum kommer dock molnet av elektroner
att dras mot den ``positiva'' k{\"a}llan f{\"o}r de elektriska
f{\"a}ltlinjerna, och kommer d{\"a}rmed ocks{\aa} att orsaka en elektrisk
polarisering av materialet. I den h{\"a}r modellen blir polariseringen av
mediet riktad {\it l{\"a}ngs med} f{\"a}ltlinjerna f{\"o}r ${\bf E}$-f{\"a}ltet.
\bigskip
\centerline{\epsfbox{figs/poldensity.1}}
\medskip
\noindent
Vi m{\"a}ter denna inducerade (``p{\aa}tvingade'') polarisering som en
{\it elektrisk polarisationsdensitet} ${\bf P}$, beskrivande det elektriska
dipolmomentet per volymenhet (d{\"a}rav ``densitet''). Som en f{\"o}renklad
modell kan vi se detta inducerade dipolmoment som linj{\"a}rt beroende av det
p{\aa}lagda elektriska f{\"a}ltet,
$$
  {\bf P} \equiv \Big\langle{{d{\bf p}}\over{dV}}\Big\rangle
    = \varepsilon_0\chi_{\rm e}{\bf E},
$$
d{\"a}r $\chi_{\rm e}$ {\"a}r den {\it elektriska susceptibiliteten} f{\"o}r
materialet ($\chi_{\rm e}$ {\"a}r dimensionsl{\"o}s) och d{\"a}r
$\langle\ldots\rangle$ betecknar medelv{\"a}rdesbildning.

Inom elektromagnetisk teori anv{\"a}nder vi ofta ett f{\"a}lt som beskriver
summan av den effektiva polariseringen som ges av det externt p{\aa}lagda
${\bf E}$-f{\"a}ltet tillsammans med den elektriska polariseringen av
materialet i sig, det s{\aa} kallade {\it elektriska fl{\"o}dest{\"a}theten}
(electric displacement field),
$$
  \eqalign{
    {\bf D}&\equiv\varepsilon_0{\bf E}+{\bf P}\cr
      &=\varepsilon_0(1+\chi_{\rm e}){\bf E}
       \equiv\varepsilon_0\varepsilon_{\rm r}{\bf E}.\cr
  }
$$
Vi kommer fram{\"o}ver alltsom oftast ist{\"a}llet f{\"o}r den elektriska
susceptibiliteten $\chi_{\rm e}$ att anv{\"a}nda den {\it relativa elektriska
permittiviteten} $\varepsilon_{\rm r}\equiv 1+\chi_{\rm e}$ (liksom
$\chi_{\rm e}$ {\"a}ven den dimensionsl{\"o}s). F{\"o}r material med
f{\"o}rsumbar magnetisering och i avsaknad av fria laddningar motsvarar
$n=\sqrt{\varepsilon_{\rm r}}$ brytningsindex hos materialet (mer om detta
i kommande f{\"o}rel{\"a}sning om den elektromagnetiska v{\aa}gekvationen).
I kommande analys av Maxwells ekvationer och de fr{\aa}n dessa f{\"o}ljande
elektromagnetiska v{\aa}gekvationerna kommer vi genomg{\aa}ende att anv{\"a}nda
$\varepsilon_{\rm r}$ f{\"o}r att beskriva materialegenskaperna.

\section{Gauss teorem f{\"o}r elektriska fl{\"o}dest{\"a}theten {\bf D}}
Att den eletriska fl{\"o}dest{\"a}theten ${\bf D}$ {\"a}r konstruerad som den
{\"a}r faller sig naturligt om vi betraktar, till exempel, hur en mer generell
form av Gauss lag kan formuleras i termer av fria s{\aa}v{\"a}l som bundna
laddningsdensiteter.
Genom att anv{\"a}nda att sambandet mellan divergensen av
polarisationsdensiteten ${\bf P}$ och den {\it bundna} laddningst{\"a}theten
$\rho_{\rm b}$,\numberedfootnote{Griffiths Ekv.~(4.12), sidan~174; f{\"o}r
h{\"a}rledning av denna, se Griffiths, kapitel 4.2.1, sid.~173--174.}
$$
  \qquad
  \qquad
  \qquad
  \nabla\cdot{\bf P} = -\rho_{\rm b},
  \qquad
  \bigg(\quad\Leftrightarrow\qquad
  \iiint_V \rho_{\rm b}\,dV
    = -\iiint_V\nabla\cdot{\bf P}\,dV
    = -\oiint_{\Omega}{\bf P}\,dA
    \quad\bigg)
$$
s{\aa} har vi allts{\aa} f{\"o}r divergensen f{\"o}r det elektriska f{\"a}ltet
att
$$
  \varepsilon_0\nabla\cdot{\bf E}
    =\rho_{\rm tot}
    =\rho_{\rm b}+\rho_{\rm f}
    =-\nabla\cdot{\bf P}+\rho_{\rm f}.
$$
Genom att kombinera divergenserna kan detta skrivas som
$$
  \nabla\cdot(\varepsilon_0{\bf E}+{\bf P})=\rho_{\rm f},
$$
d{\"a}r nu allts{\aa} k{\"a}lltermen i h{\"o}gerledet endast utg{\"o}rs av den
{\it fria} laddningst{\"a}theten $\rho_{\rm f}$, och om vi {\it definierar} den
elektriska fl{\"o}dest{\"a}theten som
$$
  {\bf D}\equiv\varepsilon_0{\bf E}+{\bf P},
$$
s{\aa} lyder allts{\aa} denna under Gauss lag
$$
  \nabla\cdot{\bf D}=\rho_{\rm f}.
$$
Eftersom denna nu innefattar {\"a}ven den elektriska polarisationsdensiteten
hos materialet, med andra ord v{\"a}xelverkan mellan det elektriska f{\"a}ltet
och mediet, s{\aa} {\"a}r detta en mycket anv{\"a}ndbar generalisering n{\"a}r
vi nu skall analysera gr{\"a}nsytor mellan olika medier.
\vfill\eject

\section{Randvillkor och {\"o}verg{\aa}ngar mellan olika media}
Vid {\"o}verg{\aa}ngar mellan olika media blir {\"a}ndringarna p{\aa}
${\bf E}$- och ${\bf D}$-f{\"a}ltens komposanter olika bero\-ende p{\aa} om
de {\"a}r  normala (vinkelr{\"a}ta) eller tangentiala (parallella) mot
gr{\"a}nsytan mellan media.\numberedfootnote{Se Griffiths s.~185.}
\medskip
\noindent{\it Recap p{\aa} hoppvillkor f{\"o}r elektriska f{\"a}lt i vakuum}
\smallskip
\noindent
F{\"o}rst av allt en liten recap p{\aa} randvillkor (``hoppvillkor'') {\"o}ver
gr{\"a}nsytor som uppb{\"a}r en ytladdning $\sigma$!
\bigskip
\centerline{\epsfbox{figs/esurfnorm.1}}
\medskip
\noindent
Om vi betraktar en yta med en ytladdningst{\"a}thet
$\sigma$ (${\rm C}/{\rm m}^2$), och innesl{\aa}ter denna yta med en sluten
yta\numberedfootnote{Jackson's {\it Classical Electrodynamics} kallar denna
f{\"o}r ``pillbox'' medan Griffits anv{\"a}nder det fyndiga och tr{\"a}ffande
  ``Gaussian pillbox''.}
$\Omega$ av h{\"o}jd $h\to0$, som har platta ytor ovanf{\"o}r och under,
s{\aa} har vi fr{\aa}n Gauss teorem i elektrostatik att normal-komposanterna
($\perp$) f{\"a}ltet ${\bf E}_1$ under ytan och f{\"a}ltet ${\bf E}_2$ ovan
ytan {\"a}r relaterade genom:\numberedfootnote{Sj{\"a}lvfallet med
  ytladdningsdensiteten som $\sigma=Q/A$ (C/A).}
$$
  \hbox{Gauss:}\qquad
  \displaystyle
  \iiint_V\nabla\cdot{\bf E}\,dV=
  \oiint_{\Omega}{\bf E}\cdot d{\bf A}={{Q}\over{\varepsilon_0}}
  \qquad\to\qquad
  {\bf e}_n \cdot({\bf E}_{2}-{\bf E}_{1})A
    ={{Q}\over{\varepsilon_0}}
  \quad\Leftrightarrow\quad
  E^{\perp}_2-E^{\perp}_1={{\sigma}\over{\varepsilon_0}}
$$
\bigskip
\centerline{\epsfbox{figs/esurftang.1}}
\medskip
\noindent
P{\aa} liknande s{\"a}tt kan vi analysera tangentiella komposanter ($\parallel$), parallella med ytan, genom att ist{\"a}llet anv{\"a}nda Stokes teorem i en sluten rektangul{\"a}r slinga $\Sigma$ av h{\"o}jd $h\to0$ och som innefattar gr{\"a}nsytan:
$$
  \hbox{Stokes:}\qquad
  \iint_{\Sigma}\underbrace{(\nabla\times{\bf E})}_{=-\partial{\bf B}/\partial t=0}
      \cdot d{\bf A}
    =\oint_{\Sigma}{\bf E}\cdot d{\bf s}
    =0
  \qquad\to\qquad
  {\bf e}_n \times({\bf E}_{2}-{\bf E}_{1})={\bf 0}
  \quad\Leftrightarrow\quad
  E^{\parallel}_2=E^{\parallel}_1
$$
\vfill\eject
\noindent{\it Hoppvillkor f{\"o}r elektriska f{\"a}lt mellan olika media}
\smallskip
\noindent
Vi kan nu anv{\"a}nda en analogi till ovanst{\aa}ende f{\"o}r att analysera vad som h{\"a}nder d{\aa} materialen under och ovanf{\"o}r gr{\"a}nsytan har olika relativ elektrisk permittivitet.
F{\"o}r den elektriska fl{\"o}dest{\"a}theten har Gauss lag (utifr{\aa}n $\nabla\cdot{\bf D}=\rho_{\rm f}$) samma form som f{\"o}r det elektriska f{\"a}ltet, med skillnaden att vi nu har den {\it fria} elektriska laddningst{\"a}theten $\rho_{\rm f}$ som en k{\"a}llterm i h{\"o}gerledet. Med exakt samma geometri som tidigare f{\"o}ljer d{\"a}rmed att vi f{\aa}r\numberedfootnote{{\AA}terigen, sj{\"a}lvfallet med den {\it fria} ytladdningsdensiteten som $\sigma_{\rm f}=Q_{\rm f}/A$ (C/A).}
$$
  \hbox{Gauss:}\qquad
  \displaystyle
  \iiint_V\nabla\cdot{\bf D}\,dV=
  \oiint_{\Omega}{\bf D}\cdot d{\bf A} = Q_{\rm f}
  \qquad\to\qquad
  {\bf e}_n \cdot({\bf D}_{2}-{\bf D}_{1})A = Q_{\rm f}
  \quad\Leftrightarrow\quad
  D^{\perp}_{2}-D^{\perp}_{1} = \sigma_{\rm f},
$$
d{\"a}r $\sigma_{\rm f}$ {\"a}r den {\it fria} elektriska ytladdningst{\"a}theten. Med andra ord, s{\aa} tar den normalkomposanten av den elektriska fl{\"o}dest{\"a}theten ${\bf D}$ ett hopp motsvarande den fria ytladdningst{\"a}theten $\sigma_{\rm f}$ d{\aa} vi passerar gr{\"a}nsytan.

Utifr{\aa}n detta resultat {\"a}r det mycket l{\"a}tt att f{\"o}rledas att tro att ett ``universalrecept'' f{\"o}r att behandla elektrostatik i ett medium:
\medskip
\hbox{\hskip60pt ``Vi beh{\"o}ver ju bara byta ut ${\bf E}$ mot ${\bf D}$ och ist{\"a}llet anv{\"a}nda de {\it fria}}
\hbox{\hskip60pt laddningarna, s{\aa} har vi ju precis samma resultat. Busenkelt!''}
\medskip
\noindent
Detta antagande {\"a}r dock falskt, vilket vi kan se genom att exempelvis betrakta rotationen av den elektriska fl{\"o}dest{\"a}theten,
$$
  \nabla\times{\bf D}
    =\varepsilon_0\underbrace{\nabla\times{\bf E}}_{=0}+\nabla\times{\bf P}
    =\nabla\times{\bf P},
$$
vilken generellt {\it inte} alltid {\"a}r identiskt noll.\numberedfootnote{Griffiths har en bra genombelysning av detta i kapitel 4.3.2, sidan~184.} Om vi anv{\"a}nder Stokes teorem p{\aa} $\nabla\times{\bf P}$ s{\aa} erh{\aa}ller vi ist{\"a}llet
$$
  \eqalign{
  \hbox{Stokes:}\qquad
  \iint_{\Sigma}\underbrace{(\nabla\times{\bf D})}_{\ne0}\cdot d{\bf A}
    =\oint_{\Sigma}{\bf D}\cdot d{\bf s}
    =\oint_{\Sigma}{\bf P}\cdot d{\bf s}
  \quad\to\quad&
  {\bf e}_n \times({\bf D}_{2}-{\bf D}_{1})=
  {\bf e}_n \times({\bf P}_{2}-{\bf P}_{1})\cr
  &\quad\Leftrightarrow\quad
  D^{\parallel}_2-D^{\parallel}_1 = P^{\parallel}_2-P^{\parallel}_1\cr
  }
$$
Om vi har tv{\aa} medier av olika relativ permittivitet $\varepsilon_1$ och
$\varepsilon_2$, s{\aa} kan vi 

\section{Upplagrad elektrisk energi}
D{\aa} vi applicerar ett elektriskt f{\"a}lt {\"o}ver ett dielektrikun, s{\aa}
kommer vi att lagra upp energi i det system av mikrosopiska
(atom{\"a}ra/molekyl{\"a}ra) dipoler som d{\"a}rmed kommer att linjeras upp i
det externt p{\aa}lagda f{\"a}ltet. Som an analogi kan vi se detta som
motsvarigheten till en upplagring av mekanisk energi i ett distribuerat system
av mycket sm{\aa} men m{\aa}nga fj{\"a}drar. Det som h{\"a}r tillkommer {\"a}r
att vi dessutom har fria laddningar som kommer att justera position
allteftersom j{\"a}mvikt i systemet uppn{\aa}s.

Uttrycket f{\"o}r den upplagrade energin (i Joule) hos ett dielektrikun under
ett externt p{\aa}lagt elektriskt f{\"a}lt ${\bf E}$ ges av
$$
  W = {{1}\over{2}}\iiint{\bf D}\cdot{\bf E}\,dV
$$
H{\"a}rledningen av denna form ges av Griffiths p{\aa} sidorna 197--198, och
{\"a}r en nyttig {\"o}vningsuppgift att f{\"o}lja.

\section{Dielektrisk sf{\"a}r i elektriskt f{\"a}lt}
Vi har just sett hur en extremt f{\"o}renklad modell av en atom kan anv{\"a}ndas
f{\"o}r att kvalitativt beskriva uppkomsten av dipolmoment d{\aa} systemet
uts{\"a}tts f{\"o}r ett externt p{\aa}lagt elektriskt f{\"a}lt. Om vi lyfter
blicken en aning, s{\aa} {\"a}r ett annat intressant objekt en sf{\"a}r
best{\aa}ende av ett homogent och isotropt medium, f{\"o}r enkelhets skull
placerad i ett medium med $\varepsilon_{\rm e}=1$ (h{\"a}r antar vi att ``luft
$\approx$ vakuum''). F{\"o}rutom att vara en mycket anv{\"a}ndbar modell
f{\"o}r spridning av elektromagnetiska f{\"a}lt, s{\aa} utg{\"o}r detta enkla
system {\"a}ven en intressant exercis i partiella differentialekvationer och
elektrostatik.

Om en dielektrisk sf{\"a}r av radie $R$ och relativ permittivitet
(dielektricitetskonstant) $\varepsilon_{\rm r}$ placeras i ett elektriskt
f{\"a}lt ${\bf E}$, s{\aa} kommer dipolmomenten i sf{\"a}ren att addera upp
till att ge sf{\"a}ren i sig ett netto-dipolmoment ${\bf p}$. Man kan visa att
detta dipolmoment ges som\numberedfootnote{F{\"o}r en h{\"a}rledning av det
elektriska f{\"a}ltet inuti en dielektrisk sf{\"a}r, se Griffiths Exempel~4.7
p{\aa} sidan 193 (vilket f{\"o}r {\"o}vrigt {\"a}r ett mycket bra exempel);
alternativt Jackson, s.~151. Se {\"a}ven Griffiths, Problem~4.41, sidan 208
f{\"o}r ett exempel p{\aa} relationen mellan atom{\"a}r polarisabilitet
$\alpha$ och relativ permittivitet $\varepsilon_{\rm r}$.
Clausius--Mossotti-relationen g{\"a}ller generellt f{\"o}r icke-pol{\"a}ra
media, medan uttryck enligt Langevin g{\"a}ller f{\"o}r pol{\"a}ra media.}
$$
  {\bf p}=\underbrace{4\pi\varepsilon_0
    \bigg({{\varepsilon_{\rm r}-1}\over{\varepsilon_{\rm r}+2}}\bigg)
      R^3}_{=``\alpha''}{\bf E}_{\infty}
$$
d{\"a}r ${\bf E}_{\infty}$ {\"a}r det elektriska f{\"a}ltet l{\aa}ngt ifr{\aa}n
sj{\"a}lva sf{\"a}ren. Faktorn $(\varepsilon_{\rm r}-1)/(\varepsilon_{\rm r}+1)$
kallas {\it Clausius--Mossotti}-relationen\numberedfootnote{Efter
Ottaviano-Fabrizio Mossotti (1791--1863) och Rudolf Clausius (1822--1888).}.
Ur denna relation har vi att den effektiva polarisabiliteten hos sf{\"a}ren
i sig ges som
$$
  \alpha=4\pi\varepsilon_0
    \bigg({{\varepsilon_{\rm r}-1}\over{\varepsilon_{\rm r}+2}}\bigg) R^3
$$
\bye
