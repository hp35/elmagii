%
% File: teach/elmagii/lect-09/lecture-09.tex [plain TeX code]
% Github: https://github.com/elmagii/lect-09/
% Last change: December 2, 2024
%
% Lecture No 9 in the course ``Elektromagnetism II, 1TE626 (2023)'',
% held November 25, 2025, at Uppsala University, Sweden.
%
% Copyright (C) 2022-2025, Fredrik Jonsson, under Gnu General Public
% License (GPL) v3. See the enclosed LICENSE for details.
%
% This program is free software: you can redistribute it and/or modify
% it under the terms of the GNU General Public License as published by
% the Free Software Foundation, either version 3 of the License, or
% (at your option) any later version.
%
% This program is distributed in the hope that it will be useful,
% but WITHOUT ANY WARRANTY; without even the implied warranty of
% MERCHANTABILITY or FITNESS FOR A PARTICULAR PURPOSE.  See the
% GNU General Public License for more details.
%
% You should have received a copy of the GNU General Public License
% along with this program.  If not, see <https://www.gnu.org/licenses/>.
%
\input macros/epsf.tex
\input macros/eplain.tex
\font\ninerm=cmr9
\font\twelvesc=cmcsc10 at 12 truept
\input amssym % to get the {\Bbb E} font (strikethrough E)
\def\lecture #1 {\hsize=150mm\hoffset=4.6mm\vsize=230mm\voffset=7mm
  \topskip=0pt\baselineskip=12pt\parskip=0pt\leftskip=0pt\parindent=15pt
  \headline={\ifnum\pageno>1\ifodd\pageno\rightheadline\else\leftheadline\fi
    \else\hfill\fi}
  \def\rightheadline{\tenrm{\it F\"orel\"asning #1}
    \hfil{\it Elektromagnetism II, 1TE626 (2025)}}
  \def\leftheadline{\tenrm{\it Elektromagnetism II, 1TE626 (2025)}
    \hfil{\it F\"orel\"asning #1}}
  \noindent~\vskip-60pt\hskip-40pt{\epsfbox{macros/UU_logo_color.eps}}
  \vskip-42pt\hfill\vbox{
      \hbox{{\it Elektromagnetism II, 1TE626 (2025)}}
      \hbox{{\it Lecture Notes, Fredrik Jonsson}}
      \hbox{{\it Document Revision \today}}
      \hbox{{\it https://github.com/hp35/elmagii/}}}\vskip 36pt
    \centerline{\twelvesc F\"orel\"asning #1}
  \vskip 24pt\noindent}
\def\section #1 {\medskip\goodbreak\noindent{\bf #1}
  \par\nobreak\smallskip\noindent}
\def\subsection #1 {\medskip\goodbreak\noindent{\it #1}
  \par\nobreak\smallskip\noindent}
\def\iint{\mathop{\int\kern-8pt\int}}
\def\iiint{\mathop{\int\kern-8pt\int\kern-8pt\int}}
\def\oiint{\mathop{\int\kern-8pt\int\kern-13.2pt{\bigcirc}}}
\def\Re{\mathop{\rm Re}\nolimits} % real part
\def\Im{\mathop{\rm Im}\nolimits} % imaginary part
\def\Tr{\mathop{\rm Tr}\nolimits} % quantum mechanical trace
\def\eqq{\mathop{\vbox{\hbox{\hskip2pt?}\vskip-6pt\hbox{=}}}}
\def\boxit#1{\vbox{\hrule\hbox{\vrule\kern3pt
  \vbox{\kern3pt#1\kern3pt}\kern3pt\vrule}\hrule}}
\def\quote#1{\leftskip=36pt\rightskip=36pt\smallskip\noindent#1\par
  \leftskip=0pt\rightskip=0pt\smallskip}
\def\epsfig#1{\bigskip\centerline{\epsfbox{#1}}\medskip}

\lecture{9}
\centerline{\twelvesc Maxwells ekvationer och v{\aa}gutbredning}
\centerline{Fredrik Jonsson, Uppsala Universitet, 25 november 2025}
\vskip24pt

\section{Maxwells generalisering av Amp\`eres lag}
S{\aa} l{\aa}ngt i kursen\numberedfootnote{Den f{\"o}ljande behandlingen i
  denna f{\"o}rel{\"a}sning, d{\"a}r vi till slut kommer fram till en mer
  generell form av Maxwell's ekvationer {\"a}n vad man som oftast st{\"o}ter
  p{\aa} i standard-textb{\"o}cker, f{\"o}ljer i huvudsak Griffiths kapitel
  7.3. Dock tycker jag att Griffiths saknar po{\"a}ngen att k{\"a}lltermerna
  f{\"o}r s{\aa}v{\"a}l det elektriska som magnetiska f{\"a}ltet i sj{\"a}lva
  verket har en gemensam form, best{\aa}ende av den fria str{\"o}mt{\"a}theten,
  tidsderivatan av den elektriska polarisationsdensiteten, samt rotationen av
  magnetf{\"a}ltet. Min f{\"o}rhoppning {\"a}r att analysen h{\"a}r, om {\"a}n
  n{\aa}got mer omfattande, skall belysa denna {\it fundamenta} och v{\"a}cka
  intresset f{\"o}r denna oerh{\"o}rt vackra del av elektromagnetisk teori.}
har Maxwells ekvationer baserats p{\aa} frirymds-formen
$$
  \nabla\cdot{\bf E}={{\rho}\over{\varepsilon_0}},\qquad
  \nabla\times{\bf E}=-{{\partial{\bf B}}\over{\partial t}},\qquad
  \nabla\cdot{\bf B}=0,
$$
samt med Amp\`eres lag (s{\aa} l{\aa}ngt) p{\aa} den {\it statiska} formen
$$
  \hskip70pt\nabla\times{\bf B}=\mu_0{\bf J}.\hskip30pt(\hbox{Statisk!})
$$
Utifr{\aa}n en {\it elektrostatisk} betraktelse, s{\aa} ser vi direkt att det
elektriska f{\"a}lt har sin k{\"a}lla i statiska elektriska laddningar, via
Gauss lag f{\"o}r det elektriska f{\"a}ltet, medan magnetf{\"a}ltet i sin tur
har sin k{\"a}lla i motsvarande {\it r{\"o}relse} av de elektriska laddningarna
(det vill s{\"a}ga {\it str{\"o}m}). Redan i Faradays lag ovan har vi dock en
f{\"o}rsta koppling till en {\it elektrodynamisk} koppling i och med att
tidsvariationen hos det magnetiska f{\"a}ltet kopplar till rotationen hos det
elektriska f{\"a}ltet. Denna koppling sker dessutom via ett omv{\"a}nt tecken,
vilket i grund och botten fastst{\"a}ller {\it Lenz lag} (som s{\"a}ger att en
inducerad elektromotorisk kraft alltid riktas s{\aa} att den motverkar
k{\"a}llan som inducerade den).

Som ett grundl{\"a}ggande krav f{\"o}r att dessa tre ekvationer skall vara
fysikaliskt korrekta m{\aa}ste dessutom lagen om {\it konservering av elektrisk
laddning}\numberedfootnote{Notera att man talade om elektrisk laddning
  l{\aa}ngt innan elektronen 1897 uppt{\"a}cktes experimentellt av brittiske
  fysikern och nobelpristagaren Joseph John Thomson. Elektronens existens
  framf{\"o}rdes dock som hypotes 1838 av den brittiske naturfilosofen
  Richard Laming, d.v.s. mindre {\"a}n 20 {\aa}r innan Maxwell konsoliderade
  ``sina'' ekvationer.
} vara uppfylld,\numberedfootnote{Notera att tecknet f{\"o}r
  $-\partial\rho/\partial t$ i lagen f{\"o}r konservering av elektrisk
  laddning h{\"a}nger ihop med att str{\"o}mmen ${\bf J}$ r{\"a}knas som
  positiv {\it ut} fr{\aa}n punkten f{\"o}r {\it positiv} laddningstransport.
  Med $\nabla\cdot{\bf J}({\bf x})>0$ m{\aa}ste vi d{\"a}rf{\"o}r ha en
  {\it ackumulering av negativ laddning} i punkten ${\bf x}$, och d{\"a}rmed
  att $\partial\rho({\bf x})/\partial t < 0$.}
$$
  {{\partial\rho}\over{\partial t}}+\nabla\cdot{\bf J}=0.
$$
\epsfig{figs/chargeconsv.1}\noindent
L{\aa}t oss d{\"a}rf{\"o}r en g{\aa}ng f{\"o}r alla kontrollera att den
elektriska laddningen {\"a}r konserverad utifr{\aa}n ekvationerna ovan!
Fr{\aa}n den {\it statiska} formen av Amp\`eres lag har vi dessv{\"a}rre att
$$
  \nabla\cdot{\bf J}
    = {{1}\over{\mu_0}}\nabla\cdot(\nabla\times{\bf B})
    \equiv \{\ \hbox{vektoridentitet}\ \}
    \equiv 0,
$$
{\it fast vi egentligen skulle beh{\"o}vt ett ``$-\partial\rho/\partial t$'' i
h{\"o}gerledet ist{\"a}llet f{\"o}r en nolla.} Detta visar tydligt hur Amp\`eres
lag p{\aa} formen ovan (i sin statiska form) {\it ej} kan till{\"a}mpas p{\aa}
generella tidsberoende problem.

Ett av James Clerk Maxwells viktigare bidrag till elektromagnetisk f{\"a}ltteori
var n{\"a}r han 1856 ins{\aa}g hur Amp\`eres lag b{\"o}r modifieras f{\"o}r att
dessutom uppfylla kravet p{\aa} konservering av elektrisk laddning. Argumentet
lyder i stort som f{\"o}ljer: Om vi fr{\aa}n kravet p{\aa} konservering av
laddning har att ``$-\partial\rho/\partial t$'' saknas i h{\"o}gerledet ovan,
kan vi d{\aa} inte helt enkelt l{\"a}gga till en s{\aa}dan term och se vad som
h{\"a}nder?
$$
    \nabla\cdot{\bf J}
       = {{1}\over{\mu_0}}\nabla\cdot(\nabla\times{\bf B})
       \eqq -{{\partial\rho}\over{\partial t}}
       = -{{\partial}\over{\partial t}}(\varepsilon_0\nabla\cdot{\bf E})
       = -\varepsilon_0\nabla\cdot{{\partial{\bf E}}\over{\partial t}}.
$$
Vi anv{\"a}nde i mellansteget ovan Gauss lag f{\"o}r elektrisk laddning.
Uppenbarligen s{\aa} skulle en modifiering av Amp\`eres ``statiska'' lag med
den (fria) str{\"o}mt{\"a}theten ersatt med
$$
  {\bf J}\to{\bf J}+\varepsilon_0{{\partial{\bf E}}\over{\partial t}}
$$
direkt l{\"o}sa problemet med konservering av laddning f{\"o}r tidsberoende
problem, och detta {\"a}r ocks{\aa} sj{\"a}lva k{\"a}rnan i den modifiering
som Maxwell introducerade.

I n{\"a}rvaron av fria str{\"o}mmar (men utan att befinna sig i ett medium som
kan polariseras elektriskt) {\"a}r d{\"a}rf{\"o}r den generella (tidsberoende)
formen av Amp\`eres lag
$$
  \nabla\times{\bf B}=\mu_0\bigg({\bf J}
    +\varepsilon_0{{\partial{\bf E}}\over{\partial t}}\bigg),
$$
d{\"a}r
$$
  \varepsilon_0{{\partial{\bf E}}\over{\partial t}}
    =\hbox{``f{\"o}rskjutningsstr{\"o}mmen'' (i vakuum, paradoxalt nog!)}
$$
Att det {\"a}r just $\varepsilon_0{\bf E}$ som dyker upp i derivatan, och att
denna term samtidigt finns naturligt i uttrycket f{\"o}r den elektriska
polarisationsdensiteten hos ett material (se exempelvis f{\"o}reg{\aa}ende
f{\"o}rel{\"a}sning), leder oss till att fundera p{\aa} om det i det generella
fallet (inuti ett elektriskt polariserbart medium, och inte bara i vakuum)
i sj{\"a}lva verket inte snarare {\"a}r ${\bf D}=\varepsilon_0{\bf E}+{\bf P}$
som borde ing{\aa}. I sj{\"a}lva verket {\"a}r det exakt s{\aa} som den
korrekta generella formen {\"a}r, som\numberedfootnote{Vi inf{\"o}r h{\"a}r
   beteckningen ${\bf J}_{\rm f}$ f{\"o}r den {\it fria} str{\"o}mt{\"a}theten,
   identisk med all str{\"o}mt{\"a}thet som diskuterats tidigare i kursen, bara
   f{\"o}r att vara noga med att s{\"a}rskilja denna fr{\aa}n r{\"o}relsen hos
   {\it bundna} laddningar hos mediet. Notera att f{\"o}r statiska problem
   {\"a}r ${\bf E}$ tidsoberoende och Amp\`eres statiska lag op{\aa}verkad
   av denna modifikation.}
$$
  {\bf J}_{\rm tot} = {\bf J}_{\rm f}+{{\partial{\bf D}}\over{\partial t}},
$$
d{\"a}r ${\bf J}_{\rm f}$ {\"a}r str{\"o}mmen av {\it fria laddningar}, medan
${{\partial{\bf D}}/{\partial t}}$ {\"a}r ``str{\"o}mmen'' av {\it bundna
laddningar}. Maxwell sj{\"a}lv kallade denna form f{\"o}r ``{\it The Law of
Total Currents}''.

S{\aa} varf{\"o}r kallar vi ${{\partial{\bf D}}/{\partial t}}$ f{\"o}r just
``f{\"o}rskjutningsstr{\"o}m''? Till att b{\"o}rja med dyker denna term upp
som en naturlig korrektion till str{\"o}mmen ${\bf J}_{\rm f}$ av fria
laddningar, och har sj{\"a}lvfallet samma fysikaliska dimension som
str{\"o}mt{\"a}thet (${\rm C}/{\rm m}^2)$. Ut{\"o}ver detta, s{\aa} involverar
${\bf D}$ {\"a}ven den elektriska polarisationen av mediet, med en
{\it f{\"o}rskjutning} av elektrisk laddning som en direkt f{\"o}ljd av det
p{\aa}lagda elektriska f{\"a}ltet. Tidsderivatan av denna f{\"o}rskjutning av
elektrisk laddning blir d{\aa} en effektiv elektrisk {\it str{\"o}m}
(f{\"o}rflyttning av elektrisk laddning per tidsenhet), och d{\"a}rav att
${{\partial{\bf D}}/{\partial t}}$ {\"a}r en effektiv
``f{\"o}rskjutningsstr{\"o}m''. Viktigt h{\"a}r {\"a}r att notera att en
{\it str{\"o}m} kan utg{\"o}ras {\"a}ven av {\it bundna} laddningar,
exempelvis i ett annars icke elektriskt ledande dielektrikum.
\epsfig{figs/poldensity.1}\noindent
Exempelvis {\"a}r en oscillerande molekyl{\"a}r elektrisk dipol ett exempel
p{\aa} denna f{\"o}rskjutningsstr{\"o}m, vilket vi kan se som en elektriskt
driven antenn p{\aa} mikroskopisk niv{\aa}, d{\"a}r en medelv{\"a}rdesbildning
som omfattar antennen ger vid hand att ingen netto-str{\"o}m genom volymen sker
trots att ``antennen'' lokalt uppb{\"a}r en periodiskt oscillerande effektiv
(och h{\"o}gst lokal) str{\"o}m.

\section{Maxwells ekvationer}
Maxwells fyra ekvationer\numberedfootnote{Det var faktiskt inte
   f{\"o}rr{\"a}n 1884 som Oliver Heaviside (samme Heaviside som stegfunktionen
   $H(x)$), samtidigt med liknande liknande arbeten av Josiah Willard Gibbs och
   Heinrich Hertz, f{\"o}renklade och grupperade ihop de ursprungligen 20
   ekvationerna till endast fyra, under anv{\"a}nd\-an\-de av modern
   vektornotation. Denna grupp av fyra vektor-ekvationer genom historien har
   kallats s{\aa}v{\"a}l {Hertz--Heavisides ekvationer} som {Maxwell--Hertz
   ekvationer}, men {\"a}r idag kort och gott k{\"a}nda som {\it Maxwells
   ekvationer}.}
kan uttryckas antingen i en differentiell form eller i integralform.
Ekvationerna involverar de tre elektriska f{\"a}lten (${\bf E}$, ${\bf D}$
och polarisationsdensiteten ${\bf P}$) och de tre magnetiska f{\"a}lten
(${\bf B}$, ${\bf H}$ och magnetiseringen ${\bf M}$).

F{\"o}r v{\aa}gpropagation i fri rymd (vakuum) kan vi anta att den relativa
elektriska permittiviteten $\varepsilon_{\rm r}=1$ och att den relativa
magnetiska permeabiliteten $\mu_{\rm r}=1$, f{\"o}r vilket fall vi frikopplar
ekvationerna f{\"o}r ${\bf E}$ och ${\bf B}$ fr{\aa}n konstitutiva relationer.
Vi kommer h{\"a}r att ist{\"a}llet f{\"o}r att utg{\aa} ifr{\aa}n ett
f{\"o}renklat fall anta att vi har en mer generell situation i ett material.
Maxwells ekvationer kan i sin generella form sammanfattas med
$$\hbox{~\hskip33pt}
  \matrix{
  &
    \displaystyle\oiint{\bf D}\cdot d{\bf S}=\iiint\rho\,dV\hfill&
    \quad\Leftrightarrow\quad&
    \displaystyle\nabla\cdot{\bf D}=\rho\hfill&
    &\hfill\hbox{(Gauss lag)}\cr
  &
    \displaystyle\oiint{\bf B}\cdot d{\bf S}=0\hfill&
    \quad\Leftrightarrow\quad&
    \displaystyle\nabla\cdot{\bf B}=0\hfill&
    &\hfill\hbox{(Gauss lag)}\cr
  &
    \displaystyle\oint{\bf E}\cdot d{\bf l}
       =-{{\partial}\over{\partial t}}\iint{\bf B}\cdot d{\bf S}\hfill&
    \quad\Leftrightarrow\quad&
    \displaystyle\nabla\times{\bf E}
       =-{{\partial{\bf B}}\over{\partial t}}\hfill&
    &\hfill\hbox{(Faradays lag)}\cr
  &
    \displaystyle\oint{\bf H}\cdot d{\bf l}=\iint{\bf J}_{\rm f}\cdot d{\bf S}
       +{{\partial}\over{\partial t}}\iint{\bf D}\cdot d{\bf S}\hfill&
    \quad\Leftrightarrow\quad&
    \displaystyle\nabla\times{\bf H}={\bf J}_{\rm f}
       +{{\partial{\bf D}}\over{\partial t}}\hfill&
    &\hfill\hbox{(Amp\`eres lag)}\cr
  }
$$
I dessa ekvationer {\"a}r $\rho$ den {\it fria elektriska
laddningst{\"a}theten}\numberedfootnote{Till skillnad fr{\aa}n den fria
   str{\"o}mt{\"a}theten l{\aa}ter vi h{\"a}r bli att s{\"a}tta ett explicit
   index ``f'' p{\aa} laddningsdensiteten, eftersom det alltid {\"a}r klart
   att $\rho$ relaterar just till {\it fria} laddningar, till skillnad
   fr{\aa}n ${\bf J}_{\rm f}$ som {\"a}r separerad fr{\aa}n
   f{\"o}rskjutningsstr{\"o}mmen $\partial{\bf D}/\partial t$.}
${\rm C}/{\rm m}^3$) och ${\bf J}_{\rm f}$ den {\it fria elektriska
str{\"o}mt{\"a}theten} (${\rm A}/{\rm m}^2$). Ut{\"o}ver dessa ekvationer
har vi (fr{\aa}n exempelvis f{\"o}reg{\aa}ende f{\"o}rel{\"a}sning) de
{\it konstitutiva relationerna}
$$
  \eqalign{
    {\bf D}&=\varepsilon_0{\bf E}+{\bf P}
            =\varepsilon_0\varepsilon_{\rm r}{\bf E},\cr
    {\bf B}&=\mu_0({\bf H}+{\bf M})
            =\mu_0\mu_{\rm r}{\bf H}.\cr
  }
$$
Notera att vi har en lite ``avig'' relation f{\"o}r magnetiska f{\"a}lten
om vi betraktar paret (${\bf E}$,${\bf B}$) som v{\aa}ra prim{\"a}ra
f{\"a}ltvariabler. I m{\aa}nga fall {\"a}r det en fr{\aa}ga om tycke och
smak om man v{\"a}ljer att definiera f{\"a}ltproblemet utifr{\aa}n paret
(${\bf E}$,${\bf B}$) eller (${\bf E}$,${\bf H}$), och speciellt i fallet
med propagation i fri rymd {\"a}r valet egalt, men alltsom oftast {\"a}r det
mest bekv{\"a}mt att inom elektromagnetisk f{\"a}ltteori h{\aa}lla sig till
(${\bf E}$,${\bf B}$). Vi skall strax se varf{\"o}r.

De ing{\aa}ende f{\"a}lten och deras respektive SI-enheter {\"a}r, f{\"o}r
att rekapitulera,
$$
  \eqalign{
    {\bf E} &= \hbox{Elektrisk f{\"a}ltstyrka (``elektriskt\ f{\"a}lt'')}
               \ ({\rm V}/{\rm m})\cr
    {\bf D} &= \hbox{Elektrisk fl{\"o}dest{\"a}thet}
               \ ({\rm C}/{\rm m}^2)\cr
    {\bf P} &= \hbox{Elektrisk polarisationsdensitet}
               \ ({\rm C}/{\rm m}^2)\cr
    {\bf B} &= \hbox{Magnetisk fl{\"o}dest{\"a}thet (``B-f{\"a}lt'')}
               \ ({\rm T})\cr
    {\bf H} &= \hbox{Magnetisk f{\"a}ltstyrka (``H-f{\"a}lt'')}
               \ ({\rm A}/{\rm m})\cr
    {\bf M} &= \hbox{Magnetisering}
               \ ({\rm A}/{\rm m})\cr
  }
$$

\section{Fr{\aa}n Maxwells ekvationer till elektromagnetisk v{\aa}gekvation}
{\"O}verg{\aa}ngen fr{\aa}n Maxwells ekvationer till de tv{\aa} vektoriella
elektromagnetiska v{\aa}gekvationerna {\"a}r onekligen en av de stiligaste
h{\"a}rledningarna i klassisk elektrodynamik. Det kan starkt rekommenderas att
en g{\aa}ng f{\"o}r alla g{\aa} igenom denna h{\"a}rledning med papper och
penna, om s{\aa} inte annat bara f{\"o}r att f{\"o}lja den aningen ov{\"a}ntade
kopplingen mellan klassisk induktion till elektromagnetisk v{\aa}gutbredning.
(Dessutom {\"a}r det en ganska kul och enkel exercis i vektoralgebra!)

Maxwells ekvationer i den form som de st{\aa}r i den klassiska beskrivningen
kan tolkas ganska direkt som relationer f{\"o}r induktions- och k{\"a}llagar.
De beskriver dock inte sj{\"a}lvklart v{\aa}gekvationer, {\aa}tminstone inte
vid en f{\"o}rsta anblick. Vi kommer h{\"a}r att h{\"a}rleda v{\aa}gekvationerna
f{\"o}r ${\bf E}$- och ${\bf B}$-f{\"a}lten genom att eliminera den elektriska
fl{\"o}dest{\"a}theten ${\bf D}$ och den magnetiska f{\"a}ltstyrkan ${\bf H}$,
till f{\"o}rm{\aa}n f{\"o}r en beskrivning direkt i termer av fri
str{\"o}mt{\"a}thet ${\bf J}$, elektrisk polarisationsdensitet ${\bf P}$ samt
magnetiseringen ${\bf M}$.

Vi b{\"o}rjar med den elektriska f{\"a}ltstyrkan ${\bf E}$ genom att applicera
$\nabla\times$ (``ta rotationen'') p{\aa} Faradays generella induktionslag,
$$
  \eqalign{
    \nabla\times\nabla\times{\bf E}
      &=\nabla\times\bigg(-{{\partial{\bf B}}\over{\partial t}}\bigg)\cr
      &=\{\ \hbox{Konstitutiv relation}
             \ {\bf B}=\mu_0({\bf H}+{\bf M})\ \}\cr
      &=-{{\partial}\over{\partial t}}
          \nabla\times\bigg(\mu_0({\bf H}+{\bf M})\bigg)\cr
      &=-\mu_0{{\partial}\over{\partial t}}\nabla\times{\bf H}
          -\mu_0{{\partial}\over{\partial t}}\nabla\times{\bf M}\cr
      &=\{\ \hbox{Till{\"a}mpa Amp\`eres lag}\ \}\cr
      &=-\mu_0{{\partial}\over{\partial t}}
          \bigg({\bf J}_{\rm f}+{{\partial{\bf D}}\over{\partial t}}\bigg)
          -\mu_0{{\partial}\over{\partial t}}\nabla\times{\bf M}\cr
      &=\{\ \hbox{Konstitutiv relation}
             \ {\bf D}=\varepsilon_0{\bf E}+{\bf P}\ \}\cr
      &=-\mu_0{{\partial^2}\over{\partial t^2}}
          \big(\varepsilon_0{\bf E}+{\bf P}\big)
          -\mu_0{{\partial}\over{\partial t}}
	     \big({\bf J}_{\rm f}+\nabla\times{\bf M}\big)\cr
      &=\{\ \hbox{Kombinera polarisationsdensiteten in i k{\"a}llterm}\ \}\cr
      &=-\mu_0\varepsilon_0{{\partial^2{\bf E}}\over{\partial t^2}}
          -\underbrace{
	     \mu_0{{\partial}\over{\partial t}}
	     \bigg({\bf J}_{\rm f}
	        +{{\partial{\bf P}}\over{\partial t}}
	        +\nabla\times{\bf M}\bigg)}_{\hbox{k{\"a}llterm}}\cr
  }
$$
Notera att denna ekvation f{\"o}r ${\bf E}$ g{\"a}ller {\it oavsett} eventuella
spatiala variationer hos relativa permittiviteten eller permeabiliteten, det
vill s{\"a}ga oavsett om relationen mellan de exciterande f{\"a}lten och den
resulterande elektriska polarisationsdensiteten eller magnetiseringen
{\"a}ndras. P{\aa} samma s{\"a}tt har vi f{\"o}r magnetiska
fl{\"o}dest{\"a}theten ${\bf B}=\mu_0({\bf H}+{\bf M})$ att
$$
  \eqalign{
    \nabla\times\nabla\times{\bf B}
      &=\mu_0\nabla\times\nabla\times{\bf H}
           +\mu_0\nabla\times\nabla\times{\bf M}\cr
      &=\{\ \hbox{Till{\"a}mpa Amp\`eres lag}\ \}\cr
      &=\mu_0\nabla\times\bigg({\bf J}_{\rm f}
           +{{\partial{\bf D}}\over{\partial t}}\bigg)
           +\mu_0\nabla\times\nabla\times{\bf M}\cr
      &=\mu_0{{\partial}\over{\partial t}}\nabla\times{\bf D}
           +\mu_0\nabla\times\bigg({\bf J}_{\rm f}+\nabla\times{\bf M}\bigg)\cr
      &=\{\ \hbox{Konstitutiv relation}
             \ {\bf D}=\varepsilon_0{\bf E}+{\bf P}\ \}\cr
      &=\mu_0\varepsilon_0{{\partial}\over{\partial t}}\nabla\times{\bf E}
           +\mu_0\nabla\times\bigg({\bf J}_{\rm f}
	   +{{\partial{\bf P}}\over{\partial t}}
	   +\nabla\times{\bf M}\bigg)\cr
      &=\{\ \hbox{Till{\"a}mpa Faradays lag}\ \}\cr
      &=-\mu_0\varepsilon_0{{\partial{\bf B}}^2\over{\partial t^2}}
           +\mu_0\nabla\times\bigg({\bf J}_{\rm f}
	   +{{\partial{\bf P}}\over{\partial t}}
	   +\nabla\times{\bf M}\bigg)\cr
  }
$$
{\"A}ven denna ekvation f{\"o}r ${\bf B}$ g{\"a}ller {\it oavsett} eventuella
spatiala variationer hos relativa permittiviteten eller permeabiliteten.
F{\"o}r att sammanfatta resultaten f{\"o}r den elektriska f{\"a}ltstyrkan
${\bf E}$ och den magnetiska fl{\"o}dest{\"a}theten ${\bf B}$:
$$
  \eqalign{
    \nabla\times\nabla\times{\bf E}
      +\mu_0\varepsilon_0{{\partial^2{\bf E}}\over{\partial t^2}}&=
         -\mu_0{{\partial}\over{\partial t}}
          \underbrace{
             \bigg({\bf J}_{\rm f}
	        +{{\partial{\bf P}}\over{\partial t}}
	        +\nabla\times{\bf M}\bigg)}_{\hbox{gemensam k{\"a}llterm}},\cr
    \nabla\times\nabla\times{\bf B}
      +\mu_0\varepsilon_0{{\partial^2{\bf B}}\over{\partial t^2}}&=
          \mu_0\nabla\times
          \underbrace{
	     \bigg({\bf J}_{\rm f}
	        +{{\partial{\bf P}}\over{\partial t}}
	        +\nabla\times{\bf M}\bigg)}_{\hbox{gemensam k{\"a}llterm}}.\cr
  }
$$
I dessa ekvationer svarar v{\"a}nsterleden mot de tre-dimensionella
v{\aa}gekvationerna for elektriska och magnetiska f{\"a}lt, medan
h{\"o}gerleden svarar mot k{\"a}lltermer inkluderande fria str{\"o}mmar,
f{\"o}rskjut\-nings\-str{\"o}mmens bidrag fr{\aa}n materialet i sig (via
polarisationsdensiteten ${\bf P}$, som regel en funktion beroende av elektriska
f{\"a}ltet), samt magnetiseringen ${\bf M}$ (som regel en funktion beroende av
magnetf{\"a}ltet).

V{\"a}rt att notera i ekvationerna f{\"o}r de elektromagnetiska f{\"a}lten
{\"a}r den gemensamma formen av k{\"a}lltermerna, d{\"a}r den enda skillnaden
(f{\"o}rutom omv{\"a}nt tecken) {\"a}r tidsderivatan respektive rotationen.
Dessa ekvationer har i sig ingen inneboende best{\"a}md frekvens eller liknande
f{\"o}r f{\"a}lten, utan detta best{\"a}ms av randv{\"a}rden (exempelvis en
drivande antenn eller laser) samt materialegenskaperna som modelleras via
${\bf J}_{\rm f}$, ${\bf P}$ och ${\bf M}$.

\section{Aprop{\aa} s{\"a}rskiljning av polarisationsdensitet och magnetisering}
En intressant f{\"o}ljd av den gemensamma formen p{\aa} k{\"a}lltermerna i
h{\"o}gerleden i ekvationerna f{\"o}r ${\bf E}$ och ${\bf B}$ {\"a}r att
effekterna av den fria str{\"o}mmen, polarisationsdensiteten och magnetiseringen
ej {\"a}r entydigt best{\"a}mda relativt varandra, utan har en viss grad av
godtycklighet mellan sig. Som ett exempel, om vi antar att
polarisationsdensiteten har en godtycklig term i sig som beskrivs av en
rotation, s{\"a}g
$$
  {\bf P}={\bf P}'+\nabla\times{\bf G},
$$
s{\aa} {\"a}r det egalt om termen $\nabla\times{\bf G}$ formellt ing{\aa}r i
polarisationsdensiteten ${\bf P}$ eller magnetiseringen ${\bf M}$, eftersom
$$
  {\partial\over{\partial t}}\big({\bf P}'+\nabla\times{\bf G}\big)
    +\nabla\times{\bf M}
  = {{\partial{\bf P}}'\over{\partial t}}
      +\nabla\times\big({\bf M}+{\bf G}\big)
$$
Eftersom ingen av ekvationerna f{\"o}r f{\"a}lten ${\bf E}$ eller ${\bf B}$
ovan p{\aa}verkas av om vi har antingen en modell f{\"o}r
polarisationsdensiteten som ${\bf P}={\bf P}'+\nabla\times{\bf G}$ eller
magnetiseringen som ${\bf M}={\bf M}'+{\bf G}$, s{\aa} blir fr{\aa}gan vad
som egenttligen {\"a}r polarisationsdensitet eller magnetisering till viss
del en fr{\aa}ga om godtycklighet och konvention, {\aa}tminstone vad
betr{\"a}ffar den elektromagnetiska f{\"a}ltteorin.

Ett exempel p{\aa} en polarisationsdensitet som inneh{\aa}ller just
$\nabla\times{\bf E}$ {\"a}r fallet d{\aa} ett medium {\"a}r optiskt aktivt,
under vilket det roterar polarisationstillst{\aa}ndet hos ljuset runt axeln
l{\"a}ngs vilken ljuset propagerar. I detta fall kan inte effekten av optisk
aktivitet s{\"a}rskiljas fr{\aa}n fallet med Faraday-effekt, d{\"a}r vi har
ett tatiskt magnetf{\"a}lt p{\aa}lagt l{\"a}ngs med axeln f{\"o}r ljusets
propagationsriktning. (F{\"o}r att s{\"a}rskilja dessa fall kr{\"a}vs att vi
{\"a}ven studerar motsvarande effekter vid motpropagerande f{\"a}lt, men detta
{\"a}r l{\aa}ngt utanf{\"o}r vad denna kurs t{\"a}cker.)

Ett annat s{\"a}tt att se p{\aa} k{\"a}lltermerna i h{\"o}gerleden {\"a}r som
{\it effektiva str{\"o}mt{\"a}theter} ${\bf J}_{\rm eff}$ involverande
tidsderivatan av ${\bf P}$ (f{\"o}rskjutningsstr{\"o}mmen) och rotationen
av ${\bf M}$ som till{\"a}ggstermer till den fria str{\"o}mt{\"o}theten
${\bf J}_{\rm f}$, som
$$
  {\bf J}_{\rm eff}= {\bf J}_{\rm f}
    +{{\partial{\bf P}}\over{\partial t}} + \nabla\times{\bf M},
$$
och helt enkelt reducera f{\"a}ltekvationerna till
$$
  \eqalign{
    \nabla\times\nabla\times{\bf E}
      +\mu_0\varepsilon_0{{\partial^2{\bf E}}\over{\partial t^2}}&=
         -\mu_0{{\partial{\bf J}_{\rm eff}}\over{\partial t}},\cr
    \nabla\times\nabla\times{\bf B}
      +\mu_0\varepsilon_0{{\partial^2{\bf B}}\over{\partial t^2}}&=
          \mu_0\nabla\times{\bf J}_{\rm eff}.\cr
  }
$$
\vfill\eject
\section{V{\aa}gekvation, induktion, elektrostatik, elektrodynamik, vad
         g{\"a}ller detta egentligen?}
I h{\"a}rledandet av formen p{\aa} v{\aa}gekvationerna f{\"o}r f{\"a}lten
${\bf E}$ och ${\bf B}$ ovan, s{\aa} har vi varken lagt till eller tagit bort
n{\aa}got. Allt utg{\aa}r ifr{\aa}n Faradays och Amp\`eres lagar, inklusive
Maxwells korrektion f{\"o}r att uppfylla villkoret f{\"o}r laddnings\-%
konservering, och ekvationerna beskriver i en form eller annan d{\"a}rmed
{\it samtliga} omr{\aa}den som vi hittills behandlat i kursen.

Allt handlar om att reducera v{\aa}gekvationerna till relevant situation,
exempelvis om vi har att g{\"o}ra med ett statiskt problem, om vi har avsaknad
av magnetisering eller om vi rent av har att g{\"o}ra med ett frirymdsproblem.
Allt handlar med andra ord om vilka f{\"o}ruts{\"a}ttningar vi s{\aa} att
s{\"a}ga ``matar in som antaganden'' i v{\aa}gekvationerna ovan. Om man s{\aa}
vill, s{\aa} kan man s{\"a}ga att grunden f{\"o}r samtliga elektrodynamiska
problem {\"a}r tv{\aa} v{\aa}gekvationer, f{\"o}r ${\bf E}$- och ${\bf B}$-%
f{\"a}lten, och att statiska problem helt enkelt reducerar v{\aa}gekvationerna
till enbart spatial dom{\"a}n genom att tidsderivator f{\"o}rsvinner.
\bigskip
\noindent{\it Exempel I: Elektrostatik}
\smallskip
\noindent
Om vi till exempel behandlar fallet elektrostatik i n{\"a}rvaro av konstanta
str{\"o}mmar, s{\aa} reducerar ekvationen ovan for det (statiska) elektriska
f{\"a}ltet till
$$
  \nabla\times\nabla\times{\bf E}=0,
$$
vilket under anv{\"a}ndande av vektoridentiteten
$\nabla\times\nabla\times{\bf A}=\nabla\cdot(\nabla{\bf A})-\nabla^2{\bf A}$
reducerar till
$$
  \nabla(\nabla\cdot{\bf E})-\nabla^2{\bf E}=0.
$$
Det {\"a}r h{\"a}r l{\"a}tt att f{\"o}rledas att tro att $\nabla\cdot{\bf E}=0$
om vi inte har n{\aa}gra fria laddningar. Dock, vi m{\aa}ste komma ih{\aa}g att
detta argument bara h{\aa}ller f{\"o}r frirymdsformen av Gauss lag
($\nabla\cdot{\bf E}=\rho/\varepsilon_0$). Om materialet {\"a}r {\it inhomogent}
men i {\"o}vrigt fritt fr{\aa}n fria laddningar ($\rho=0$), s{\aa} m{\aa}ste vi
g{\aa} p{\aa} den egentliga formen av Gauss lag som g{\"a}ller den elektriska
fl{\"o}dest{\"a}theten,
$$
  \nabla\cdot{\bf D}
    =\varepsilon_0\nabla\cdot(\varepsilon_{\rm r}{\bf E})=0
  \qquad\Leftrightarrow\qquad
  {\bf E}\cdot\nabla\varepsilon_{\rm r}+
    \varepsilon_{\rm r}\nabla\cdot{\bf E}=0
  \qquad\Leftrightarrow\qquad
  \nabla\cdot{\bf E}=
    -{{1}\over{\varepsilon_{\rm r}}}{\bf E}\cdot\nabla\varepsilon_{\rm r}.
$$
Allts{\aa}, elektrostatik i {\it inhomogena media} beskrivs av ekvationen
$$
  \nabla^2{\bf E}=-\nabla\bigg(
      {{1}\over{\varepsilon_{\rm r}}}{\bf E}\cdot\nabla\varepsilon_{\rm r}
    \bigg).
$$
\bigskip
\noindent{\it Exempel II: Magnetostatik}
\smallskip
\noindent
P{\aa} samma s{\"a}tt har vi f{\"o}r det magnetiska f{\"a}ltet ${\bf B}$ i
n{\"a}rvaro av en konstant str{\"o}mt{\"a}thet ${\bf J}_{\rm f}$, men i
fr{\aa}nvaro av magnetisering ${\bf M}$, att ekvationen ovan for det (statiska)
magnetiska f{\"a}ltet enligt ovan reducerar till
$$
  \nabla\times\nabla\times{\bf B}=\mu_0\nabla\times{\bf J}_{\rm f}
  \qquad\Leftrightarrow\qquad
  \nabla\times{\bf B}=\mu_0{\bf J}_{\rm f},
$$
det vill s{\"a}ga svarandes mot frirymdsformen av Amp\'eres lag, precis som
f{\"o}rv{\"a}ntat utifr{\aa}n premisserna f{\"o}r reduktionen fr{\aa}n de
generella v{\aa}gekvationerna ovan.
\bye
