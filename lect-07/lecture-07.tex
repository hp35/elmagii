%
% File: teaching/elmagii/2024/lect-07/lecture-07.tex [plain TeX code]
% Last change: November 18, 2024
%
% Lecture No 7 in the course ``Elektromagnetism II, 1TE626 (2024)'',
% held November 19, 2024, at Uppsala University, Sweden.
%
% Copyright (C) 2024, Fredrik Jonsson
%
\input macros/epsf.tex
\input macros/eplain.tex
\font\ninerm=cmr9
\font\twentyrm=cmr12 at 20 truept
\font\twelvesc=cmcsc10
\input amssym % to get the {\Bbb E} font (strikethrough E)
\def\lecture #1 {\hsize=150mm\hoffset=4.6mm\vsize=230mm\voffset=7mm
  \topskip=0pt\baselineskip=12pt\parskip=0pt\leftskip=0pt\parindent=15pt
  \headline={\ifnum\pageno>1\ifodd\pageno\rightheadline\else\leftheadline\fi
    \else\hfill\fi}
  \def\rightheadline{\tenrm{\it F\"orel\"asning #1}
    \hfil{\it Elektromagnetism II, 1TE626 (2024)}}
  \def\leftheadline{\tenrm{\it Elektromagnetism II, 1TE626 (2024)}
    \hfil{\it F\"orel\"asning #1}}
  \noindent~\vskip-60pt\hskip-40pt{\epsfbox{macros/UU_logo_color.eps}}
  \vskip-42pt\hfill\vbox{\hbox{{\it Elektromagnetism II, 1TE626 (2024)}}
  \hbox{{\it Lecture Notes, Fredrik Jonsson}}}\vskip 36pt
    \centerline{\twelvesc F\"orel\"asning #1}
  \vskip 24pt\noindent}
\def\section #1 {\medskip\goodbreak\noindent{\bf #1}
  \par\nobreak\smallskip\noindent}
\def\subsection #1 {\smallskip\goodbreak\noindent{\it #1}
  \par\nobreak\smallskip\noindent}
\def\iint{\mathop{\int\kern-8pt\int}}
\def\iiint{\mathop{\int\kern-8pt\int\kern-8pt\int}}
\def\oiint{\mathop{\int\kern-8pt\int\kern-13.2pt{\bigcirc}}}
\def\Re{\mathop{\rm Re}\nolimits} % real part
\def\Im{\mathop{\rm Im}\nolimits} % imaginary part
\def\Tr{\mathop{\rm Tr}\nolimits} % quantum mechanical trace
\def\eqq{\mathop{\vbox{\hbox{\hskip2pt?}\vskip-6pt\hbox{=}}}}

\lecture{7}
\centerline{\twelvesc Magnetiska f\"alt i material}
\centerline{Fredrik Jonsson, Uppsala Universitet, 19 november 2024}
\vskip24pt
\noindent
I denna f{\"o}rel{\"a}sning analyserar vi vad som h{\"a}nder d{\aa} det
magnetiska spinnet hos material linjeras och magnetiserar materialet, antingen
genom ett externt p{\aa}lagt magnetiskt f{\"a}lt eller genom att spinnen {\"a}r
naturligt linjerade i materialet, i s{\aa} kallade permanentmagneter.

\section{Dipolmodellen av magneter}
L{\aa}t oss anta att det fanns en magnetisk motsvarighet till elektrisk
laddning, motsvarande magnetiska monopoler. Fr{\aa}n $\nabla\cdot{\bf B}=0$
f{\"o}ljer det direkt via Gauss lag att om vi skulle f{\"o}rs{\"o}ka isolera
en s{\aa}dan magnetisk laddning $Q_{\rm m}$, s{\aa} skulle vi ha att
$$
  \iiint_V\underbrace{\nabla\cdot{\bf B}}_{=0}\,dV=Q_{\rm m}=0,
$$
det vill s{\"a}ga att den enda m{\"o}jligheten {\"a}r att laddningen {\"a}r
just noll. {\it Med andra ord s{\aa} existerar inga magnetiska monopoler.}
Trots detta, s{\aa} kan vi {\"a}nd{\aa} t{\"a}nka oss magnetisering som
best{\aa}ende av en magnetisk ``nord-monopol'' och en ``syd-monopol''.
Denna modell betecknas med ``Gilbert-modellen''.
\medskip
\centerline{\epsfbox{figs/magdipole.1}}
\medskip
\noindent
En mer fysikalisk modell {\"a}r dock att ans{\"a}tta varje magnetisk dipol som
best{\aa}ende av en str{\"o}mslinga uppb{\"a}rande en str{\"o}m, skapandes en
inducerad magnetisk dipol. Denna modell kan vi kalla ``Am\`ere-modellen'',
d{\aa} dipolmomentet i den direkt f{\"o}ljer av Amp\`eres klassiska lag.
I Gilbert-modellen {\"a}r det magnetiska moment som genereras kort och gott
$$
  {\bf m}=I\iint_A\,d{\bf A}=IA{\bf e}_{\bf n},
$$
d{\"a}r $A$ {\"a}r arean som innesluts av den str{\"o}mb{\"o}rande loopen, med
riktningen ${\bf e}_n$ normal mot loopens plan. F{\"o}r att sammanfatta, s{\aa}
f{\"o}ljer magnetism inte av n{\aa}gra magnetiska monopoler, utan kommer
fr{\aa}n {\it r{\"o}relse av elektrisk laddning}.
\vfill\eject
\section{Krafter och moment p\aa\ dipoler}
Elektriska och magnetiska dipoler i externt p{\aa}lagda elektriska respektive
magnetiska f{\"a}lt upplagrar energierna
$$
  W_{\rm e} = -{\bf p}\cdot{\bf E}_{\rm ext},\qquad\qquad
  W_{\rm m} = -{\bf m}\cdot{\bf B}_{\rm ext}
$$
J{\"a}mf{\"o}r med elektrisk dipol placerad i f{\"a}lt med sin positiva
laddning n{\"a}rmst k{\"a}llan f{\"o}r elektriska f{\"a}ltet (med linjer
fr{\aa}n ``$+$''). Den elektriska dipolen str{\"a}var efter att linjera sig
(med axeln pekande fr{\aa}n negative till positiv laddning) l{\"a}ngs med
det externa elektriska f{\"a}ltet.

Kraften p{\aa} den elektriska respektiva magnetiska dipolen ges d{\aa} av
gradienten av den upplagrade energin, som (observera tecken p{\aa} $\nabla$!)%
\numberedfootnote{Fys.~dimension:
$[\nabla({\bf p}\cdot{\bf E}_{\rm ext})]={\rm m}^{-1}{\rm C}{\rm m}{\rm V/m}
={\rm J/m}={\rm N}$;
$[\nabla({\bf m}\cdot{\bf B}_{\rm ext})]={\rm m}^{-1}{\rm A}{\rm m}^2{\rm N/Am}
={\rm N}$.}
$$
  {\bf F}_{\rm e} = -\nabla W_{\rm e}
    =\nabla({\bf p}\cdot{\bf E}_{\rm ext}),\qquad\qquad
  {\bf F}_{\rm m} = -\nabla W_{\rm m}
    =\nabla({\bf m}\cdot{\bf B}_{\rm ext}).
$$
Notera att f{\"o}r att ha en kraft p{\aa} en elektrisk dipol som saknar
netto-laddning, s{\aa} kr{\"a}vs det att det p{\aa}lagda elektriska f{\"a}ltet
${\bf E}$ har en gradient vid dipolen ${\bf p}$. Omv{\"a}nt, om det p{\aa}lagda
elektriska f{\"a}ltet {\"a}r homogent (utan gradienter), s{\aa} m{\aa}ste
dipolen ha en nettoladdning f{\"o}r att kunna p{\aa}verkas av en kraft.

F{\"o}r det magnetiska fallet har vi att det inte existerar n{\aa}gon magnetisk
nettoladdning (eftersom $\nabla\cdot{\bf B}=0$ alltid g{\"a}ller;
j{\"a}mf{\"o}r med $\nabla\cdot{\bf E}=\rho/\varepsilon_0$), s{\aa} d{\"a}r
blir kravet att det magnetiska f{\"a}ltet ${\bf B}$ ovillkorligen m{\aa}ste ha
en gradient f{\"o}r att kunna ut{\"o}va en nettokraft p{\aa} den magnetiska
dipolen ${\bf m}$.

Vridmomentet $\tau$ (SI-enhet ${\rm N}{\rm m}$) p{\aa} en elektrisk respektive
magnetis dipol i externt p{\aa}lagda f{\"a}lt {\"a}r
$$
  {\bf \tau}={\bf p}\times{\bf E}_{\rm ext},\qquad\qquad
  {\bf \tau}={\bf m}\times{\bf B}_{\rm ext}.
$$
Notera att vridmomenten blir noll d{\aa} dipolerna {\"a}r parallella med
f{\"a}lten (${\bf e}\times{\bf e}\equiv{\bf 0}$); dock {\"a}r endast fallen
d{\aa} dipolerna dessutom pekar i samma riktning som f{\"a}lten stabila (detta
f{\"o}ljer fr{\aa}n att energierna $W_{\rm e}$ och $W_{\rm m}$ d{\aa} minimeras;
att dessutom visa att energimaximum {\"a}r instabilt {\"a}r enkelt att visa
genom att uttrycka skal{\"a}rprodukten som en faktor $\cos\theta$ i sf{\"a}riska
koordinater och analysera andraderivatan av energierna med avseende p{\aa}
$\theta$).

\section{Magnetisering - ``magnetisk polarisationsdensitet''}
Liksom f{\"o}r den elektriska polarisationsdensiteten ${\bf P}$ f{\"o}ljer att
ett material som regel linjerar sina mikroskopiska magnetiska dipoler
(molekyl{\"a}ra spinn) efter ett externt p{\aa}lagt magnetiskt f{\"a}lt.
{\"A}ven om man kan t{\"a}nka sig en bild av detta som en dipol best{\aa}ende
av en moln av positiva och negativa ``magnetiska laddningar'', s{\aa} {\"a}r
detta en felaktig bild d{\aa} det inte existerar magnetiska laddningar
({\aa}terigen, fr{\aa}n $\nabla\cdot{\bf B}=0$). Ist{\"a}llet kan vi
f{\"o}rest{\"a}lla oss detta som en ensemble av mikroskopiska
str{\"o}mb{\"a}rande slutna slingor som var och en bidrar med ett magnetisk
dipolmoment.
\bigskip\centerline{\epsfbox{figs/magdensity.1}}\medskip
\noindent
Vi beskriver detta som en magnetisering ${\bf M}$ av magnetiska dipolmoment
per volymenhet (d{\"a}rav en slags ``magnetisk polarisationsdensitet'').
Som en f{\"o}renklad modell kan vi se denna inducerade magnetisering
som linj{\"a}rt beroende av det p{\aa}lagda magnetiska f{\"a}ltet,
$$
  {\bf M} \equiv \Big\langle{{d{\bf m}}\over{dV}}\Big\rangle
    = {{1}\over{\mu_0}}\Big(1-{{1}\over{\mu_{\rm r}}}\Big){\bf B}
$$
d{\"a}r $\mu_{\rm r}$ {\"a}r den {\it relativa magnetiska permeabiliteten}
f{\"o}r materialet ($\mu_{\rm r}$ {\"a}r dimensionsl{\"o}s).

Liksom f{\"o}r den elektriska polarisationsdensiteten ${\bf P}$ {\"a}r det
inom elektromagnetisk f{\"a}ltteori bekv{\"a}mt att baka ihop det magnetiska
f{\"a}ltet ${\bf B}$ och magnetiseringen ${\bf M}$ till
{\it magnetiseringsstyrkan}
$$
  {\bf H}\equiv{{\bf B}\over{\mu_0}}-{\bf M}
    ={{\bf B}\over{\mu_0\mu_{\rm r}}},
$$
en konstitutiv relation som {\"a}r lite avigt definierad d{\aa} vi ser ${\bf B}$
som den prim{\"a}ra beskrivningen av magnetf{\"a}ltet. Normalt uttrycker vi inom
elektromagnetisk f{\"a}ltteori denna p{\aa} formen\numberedfootnote{Griffiths
  betecknar $H$-f{\"a}ltet som ``auxiliary field''; se Griffiths sidan~279.}
$$
  {\bf B}=\mu_0\mu_{\rm r}{\bf H},
$$
d{\aa} detta {\"a}r den naturliga tolkningen (${\bf B}$ som funktion av
${\bf H}$ och inte tv{\"a}rtom) d{\aa} vi formulerar de elektromagnetiska
v{\aa}gekvationerna fr{\aa}n Maxwells ekvationer. Dock, om vi utg{\aa}r
fr{\aa}n Faradays induktionslag,
$$
  \nabla\times{\bf E}=-{{\partial{\bf B}}\over{\partial t}},
$$
s{\aa} faller sig paret $[{\bf E},{\bf B}]$ som det mest naturliga att
anv{\"a}nda, trots att ${\bf E}$ r{\"a}knas som en (elektrisk)
{\it f{\"a}ltstyrka} och ${\bf B}$ som en (magnetisk)
{\it fl{\"o}dest{\"a}thet}.
\medskip
\item{$\bullet$}{F{\"o}r {\it diamagnetiska material} har vi att
   $\mu_{\rm r}\le 1$, f{\"o}r vilka magnetiseringen i materialet {\"a}r riktat
   i {\it motsatt riktning} mot det p{\aa}lagda magnetf{\"a}ltet.
   Supraledare {\"a}r exempel p{\aa} perfekta diamagneter, d{\"a}r de helt
   repellerar the p{\aa}lagda f{\"a}ltet fr{\aa}n det interna magnetf{\"a}ltet
   (som i en supraledare {\"a}r noll).}
\item{$\bullet$}{F{\"o}r {\it paramagnetiska} material {\"a}r $\mu_{\rm r}>1$,
   f{\"o}r vilka magnetiseringen i materialet {\"a}r riktat i {\it samma
   riktning} som det p{\aa}lagda magnetf{\"a}ltet.}
\item{$\bullet$}{I {\it ferromagnetiska} material {\"a}r de magnetiska
   momenten (molekyl{\"a}ra spinn) permanent linjerade i en huvudsaklig
   riktning. {\it Curietemperaturen} f{\"o}r ett ferromagnetiskt material
   {\"a}r den temperatur d{\"a}r materialet {\"o}verg{\aa}r fr{\aa}n att
   vara ferromagnetiskt till att bli paramagnetiskt.}
\medskip
\noindent
I elektromagnetisk v{\aa}gpropagation och analys av v{\aa}gekvationen kan vi
s{\aa} gott som alltid anta att mediet {\"a}r ickemagnetiskt, med
$\mu_{\rm r}=1$, det vill s{\"a}ga att vi ur magnetisk synpunkt kan betrakta
materialet som om det vore vakuum. (Detta antagande g{\"a}ller dock alltsom
oftast {\it ej} f{\"o}r den elektriska polarisationsdensiteten.)

\section{Upplagrad energi i magnetf{\"a}lt}
I likhet med ett dielektrikum kommer ett p{\aa}lagt magnetiskt f{\"a}lt
${\bf B}$ d{\aa} det magnetiserar mediet att lagra upp energi. Uttrycket
f{\"o}r den upplagrade energin (i Joule) hos ett magnetiskt material under
ett externt p{\aa}lagt magnetiskt f{\"a}lt ${\bf B}$ ges av
$$
  W = {{1}\over{2}}\iiint{\bf B}\cdot{\bf H}\,dV
$$
Vid hysteres\numberedfootnote{Griffiths sid.~291.}, som {\"a}r en vanligt
f{\"o}rekommande effekt i magnetiska material, finns det ett ``minne'' av
historiken hur magnetf{\"a}ltet har varit riktat och med vilket styrka.
N{\"a}r ett externt p{\aa}lagt magnetf{\"a}lt st{\"a}ngs av kan en kvarvarande
magnetisering finnas kvar, och om vi cykliskt varierar det p{\aa}lagda
f{\"a}ltet kommer eftersl{\"a}pningen att inneb{\"a}ra en tr{\"o}ghet d{\"a}r
materialets magnetisering arbetar mot f{\"o}r{\"a}ndringar av det p{\aa}lagda
f{\"a}ltet. Med andra ord s{\aa} utf{\"o}rs vid hysteres ett internt arbete som
leder till att v{\"a}rme utvecklas.
\medskip
\centerline{\epsfbox{figs/hysteres.1}}
\medskip
\noindent
Vi kan se denna utvecklade v{\"a}rme som att vi i hystereskurvan f{\"o}r
materialet t{\"a}cker in en viss ``area'' $B\cdot H$ som har dimensionen
energi per volymsenhet, och ju st{\"o}rre area vi t{\"a}cker in d{\aa} vi
g{\aa}r runt med en kurva i hysteresdiagrammet, desto st{\"o}rre
v{\"a}rmeutveckling.

\section{Vektorpotential fr{\aa}n ett magnetiserat objekt}
Vi kommer nu att studera ett objekt som har den rumsberoende statiska
magnetiseringen ${\bf M}({\bf x})$ given. Vi har sedan tidigare att
s{\aa}v{\"a}l elektriska som magnetiska f{\"a}lt kan uttryckas i termer av
skal{\"a}r potential och vektorpotential, d{\"a}r speciellt magnetf{\"a}ltet
${\bf B}({\bf x})$ direkt kan uttryckas i
vektorpotentialen\numberedfootnote{Recap: Existensen av en
   vektorpotential ${\bf A}$ motiveras av ``teoremet om att inga magnetiska
   dipoler existerar'',
   $$
     \nabla\cdot{\bf B}=0
     \qquad\Leftrightarrow\qquad
     {\bf B}=\nabla\times{\bf A},
   $$
   detta eftersom vi har vektoridentiteten $\nabla\cdot(\nabla\times{\bf A})
   \equiv 0$ f{\"o}r {\it godtycklig} vektorfunktion ${\bf A}$.}
${\bf A}({\bf x})$ som
$$
  {\bf B}({\bf x})=\nabla\times{\bf A}({\bf x}).
$$
Det {\"a}r d{\"a}rf{\"o}r av intresse att se om vi utifr{\aa}n en magnetisering
${\bf M}({\bf x})$ kan f{\aa} fram vektorpotentialen ${\bf A}({\bf x})$ eftersom
vi via den kan extrahera {\"o}vriga f{\"a}lt.\numberedfootnote{Vi f{\"o}ljer
h{\"a}r i huvudsak Griffiths kapitel 6.2.1, sidorna 274--275.}

Som en f{\"o}rsta b{\"o}rjan i den kommande vektoralgebran, s{\aa} kan vi
konstatera att vektorpotentialen i en {\it observationspunkt} ${\bf x}$
fr{\aa}n en enda, isolerad magnetisk dipol ${\bf m}$ i {\it k{\"a}llpunkten}
${\bf x}'$ ges av\numberedfootnote{Se Griffiths Ekv.~(5.85), sidan 253.}
$$
  {\bf A}({\bf x})={{\mu_0}\over{4\pi}}{{{\bf m}\times{\bf e}_r}\over{r^2}},
$$
d{\"a}r
$$
  r=|{\bf x}-{\bf x}'|
  \qquad\hbox{och}\qquad
  {\bf e}_r=({\bf x}-{\bf x}')/|{\bf x}-{\bf x}'|
$$
{\"a}r respektive avst{\aa}ndet och enhetsvektorn fr{\aa}n k{\"a}llpunkten
${\bf x}'$ till observationspunkten ${\bf x}$. Med denna enkla relation kan
vi enkelt g{\aa} vidare med en generalisering om vi ser sm{\aa} volumselement
$dV$ som en ensemble av k{\"a}llpunkter, var och en uppb{\"a}rande en
rumsberoende magnetisering (``magnetisk polarisationsdensitet'')
${\bf M}({\bf x}')$.
\medskip
\centerline{\epsfbox{figs/vectpot.1}}
\medskip
\noindent
Bidraget $d{\bf A}({\bf x})$ till vektorpotentialen vid observationspunkten
${\bf x}$ fr{\aa}n ett enskild magnetiskt dipolmoment $d{\bf m}$ vid
k{\"a}llpunkten ${\bf x}'$ har tagits fram tidigare, och ges av
$$
  d{\bf A}({\bf x})={{\mu_0}\over{4\pi}}
     {{d{\bf m}\times({\bf x}-{\bf x}')}\over{|{\bf x}-{\bf x}'|^3}}.
$$
Varje volymselement $dV'$ av det magnetiserade mediet uppb{\"a}r ett magnetiskt
moment
$$
  d{\bf m}({\bf x}')={\bf M}({\bf x}')dV',
$$
och den totala vektorpotentialen vid observationspunkten ${\bf x}$ ges d{\aa}
genom att summera (integrera) all delbidrag $d{\bf A}({\bf x})$, enligt
$$
  {\bf A}({\bf x}) = \iiint_V d{\bf A}({\bf x})
  ={{\mu_0}\over{4\pi}}
     \iiint_V{{{\bf M}({\bf x}')\times({\bf x}-{\bf x}')}
                       \over{|{\bf x}-{\bf x}'|^3}}\,dV'.
$$
I princip {\"a}r detta uttryck en helt acceptabel slutdestination; dock finns
det ett ``trick'' som vi kan till{\"a}mpa, och som vi {\"a}ven kommer att
anv{\"a}nda senare i kursen f{\"o}r att f{\"o}renkla uttryck f{\"o}r
mutipolutveckling f{\"o}r elektriska f{\"a}lt och skal{\"a}r potential.
Detta ``trick'' g{\aa}r ut p{\aa} att utnyttja relationen\numberedfootnote{Detta
  eftersom
  $$
    \eqalign{
      \nabla'{{1}\over{|{\bf x}-{\bf x}'|}}
      &\equiv\bigg(
         {{\partial}\over{\partial x'}},
         {{\partial}\over{\partial y'}},
         {{\partial}\over{\partial z'}}
       \bigg)
       {{1}\over{[(x-x')^2+(y-y')^2+(z-z')^2]^{1/2}}}\cr
      &= -{{1}\over{2}}{{\Big(-2(x-x'),-2(y-y'),-2(z-z')\Big)}
       \over{[(x-x')^2+(y-y')^2+(z-z')^2]^{3/2}}}\cr
      &= {{{\bf x}-{\bf x}'}\over{|{\bf x}-{\bf x}'|^3}}\cr
    }
  $$}
$$
  \nabla'{{1}\over{|{\bf x}-{\bf x}'|}}
  \equiv
     \bigg(
       {{\partial}\over{\partial x'}},
       {{\partial}\over{\partial y'}},
       {{\partial}\over{\partial z'}}
     \bigg)
     {{1}\over{|{\bf x}-{\bf x}'|}}
  ={{{\bf x}-{\bf x}'}\over{|{\bf x}-{\bf x}'|^3}},
$$
vilket till{\aa}ter oss att formulera vektorpotentialen som
$$
  {\bf A}({\bf x}) = {{\mu_0}\over{4\pi}}\iiint_V
     \bigg[
       {\bf M}({\bf x}')\times\bigg(
         \nabla'{{1}\over{|{\bf x}-{\bf x}'|}}
       \bigg)
     \bigg]
     \,dV'.
$$
Vi kan h{\"a}r till{\"a}mpa partiell integration\numberedfootnote{Av n{\aa}gon
outgrundlig anledning kallar Griffiths denna f{\"o}r ``integration by parts,
using product rule 7'' p{\aa} sidan 274. Varf{\"o}r? Jo, den finns p{\aa}
insidan av p{\"a}rmen som en ``Product Rule''.} p{\aa} detta uttryck, genom
att observera att rotationen av en produkt mellan en skal{\"a}r funktion $f$
och vektor ${\bf G}$ ges som
$$
  \eqalign{
    \nabla\times(f{\bf G})=f\nabla&\times{\bf G}-{\bf G}\times(\nabla f)\cr
      &\Updownarrow\cr
    {\bf G}\times(\nabla f) = f\nabla&\times{\bf G} - \nabla\times(f{\bf G})\cr
  }
$$
vilket ger oss att vektorpotentialen kan uttryckas som
$$
  {\bf A}({\bf x}) =
     {{\mu_0}\over{4\pi}}\iiint_V
     {{\big(\nabla'\times{\bf M}({\bf x}')\big)}\over{|{\bf x}-{\bf x}'|}}\,dV'
    -{{\mu_0}\over{4\pi}}
     \underbrace{
       \iiint_V
       \nabla'\times
       \bigg(
         {{{\bf M}({\bf x}')}\over{|{\bf x}-{\bf x}'|}}
       \bigg)
       \,dV'
     }_{\hbox{Stokes! Gauss? (Eller?)}}
$$
\vfill\eject
\noindent
Den sista termen har en form som {\"a}r misst{\"a}nkt lik grunden f{\"o}r
Stokes teorem; detta {\"a}r dock en {\aa}terv{\"a}ndsgr{\"a}nd i och med att
vi h{\"a}r har att g{\"o}ra med en {\it volymsintegral} av en ing{\aa}ende
rotation (och inte en ytintegral, som annars {\"a}r integranden i Stokes
teorem). (Med andra ord varken Stokes eller Gauss teorem, i den bem{\"a}rkelsen
vi normalt anv{\"a}nder terminologin i denna kurs.) Ist{\"a}llet kan vi h{\"a}r
anv{\"a}nda en liknande variant,\numberedfootnote{Denna form finns som ett
   h{\"a}rlednings-problem 1.61 (b) i Griffiths, sidan 56. L{\"o}sningen {\"a}r
   som f{\"o}ljer, enligt Griffiths ledtr{\aa}dar: Ans{\"a}tt f{\"o}rst
   ${\bf v}\to{\bf v}\times{\bf c}$, d{\"a}r ${\bf c}$ {\"a}r en konstant
   godtycklig vektor, och anv{\"a}nd denna form i Gauss teorem:
   $$
      \iiint_V\nabla\cdot({\bf v}\times{\bf c}) dV
        =\oiint_S({\bf v}\times{\bf c})\cdot d{\bf A}.
   $$
   Genom anv{\"a}ndande av produktregeln (``product rule \#6'' i p{\"a}rmen i
   Griffiths),
   $$
     \nabla\cdot({\bf v}\times{\bf c})
       ={\bf c}\cdot(\nabla\times{\bf v})
         -{\bf v}\cdot\underbrace{(\nabla\times{\bf c})}_{=0}
       ={\bf c}\cdot(\nabla\times{\bf v}),
   $$
   samt att vi genom vektoridentiteten (1) i p{\"a}rmen i Griffiths har att
   $$
     d{\bf A}\cdot({\bf v}\times{\bf c})
       = {\bf c}\cdot(d{\bf A}\times{\bf v})
       = -{\bf c}\cdot({\bf v}\times d{\bf A}),
   $$
   s{\aa} omformuleras Gauss teorem enligt ovan till
   $$
     \iiint_V {\bf c}\cdot(\nabla\times{\bf v})\,dV
       =\oiint_S -{\bf c}\cdot({\bf v}\times d{\bf A})
     \qquad\Leftrightarrow\qquad
     \iiint_V (\nabla\times{\bf v})\,dV
       =-\oiint_S ({\bf v}\times d{\bf A}).
     \qquad\hbox{(Q.E.D.)}
   $$}
$$
  \iiint_V(\nabla\times{\bf v})\,dV = -\oiint_{\Omega} {\bf v}\times d{\bf A}.
$$
Den andra integralen i h{\"o}gerledet kan nu omformuleras med hj{\"a}lp av
detta ``kvasi-Gauss-teorem'' som
$$
  \iiint_V
  \nabla'\times\bigg({{{\bf M}({\bf x}')}\over{|{\bf x}-{\bf x}'|}}\bigg)\,dV'
  = \oiint_{\Omega}
  \bigg({{{\bf M}({\bf x}')}\over{|{\bf x}-{\bf x}'|}}\bigg)\times d{\bf A}'.
$$
Vektorpotentialen enligt ovan uttrycks d{\"a}rmed (n{\"a}stan) slutligen som
$$
  {\bf A}({\bf x}) = {{\mu_0}\over{4\pi}}\iiint_V
    {{\big(\nabla'\times{\bf M}({\bf x}')\big)}\over{|{\bf x}-{\bf x}'|}}\,dV'
  +{{\mu_0}\over{4\pi}}\oiint_V
     \bigg({{{\bf M}({\bf x}')}\over{|{\bf x}-{\bf x}'|}}\bigg)\times d{\bf A}'.
$$
Den f{\"o}rsta termen kan tolkas som ett potentialbidrag fr{\aa}n en
{\it volymsstr{\"o}m} orsakad av {\it bundna laddningar},
$$
  {\bf J}_{\rm b}=\nabla\times{\bf M},
$$
medan den andra termen ist{\"a}llet kan tolkas som ett potentialbidrag fr{\aa}n
en {\it ytstr{\"o}m}, {\"a}ven den orsakad av bundna laddningar,
$$
  {\bf K}_{\rm b}={\bf M}\times{\bf e}_n,
$$
d{\"a}r ${\bf e}_n$ {\"a}r normalvektorn ut fr{\aa}n den slutna ytan. Summa
summarum kan allts{\aa} vektorpotentialen skrivas i termer av ekvivalenta
volyms- och ytstr{\"o}mmar som
$$
  {\bf A}({\bf x}) = {{\mu_0}\over{4\pi}}\iiint_V
    {{{\bf J}_{\rm b}({\bf x}')}\over{|{\bf x}-{\bf x}'|}}\,dV'
  +{{\mu_0}\over{4\pi}}\oiint_V
    \bigg({{{\bf K}_{\rm b}({\bf x}')}\over{|{\bf x}-{\bf x}'|}}\bigg)\times d{\bf A}'.
$$
Detta betyder att vektorpotentialen ${\bf A}({\bf x})$, och d{\"a}rmed {\"a}ven
det magnetiska f{\"a}ltet ${\bf B}({\bf x})=\nabla\times{\bf A}({\bf x})$
fr{\aa}n det magnetiserade objektet ges som om objektet ist{\"a}llet hade
uppburit en motsvarande volyms-str{\"o}m ${\bf J}_{\rm b}$ och en ytstr{\"o}m
${\bf K}_{\rm b}$. Detta illustrerar den t{\"a}ta kopplingen mellan magnetiska
f{\"a}lt och ekvivalenta str{\"o}mmar.
\bye
