%
% File: teach/elmagii/lect-05/lecture-05.tex [plain TeX code]
% Github: https://github.com/elmagii/lect-04/
% Last change: November 15, 2025
%
% Lecture No 5 in the course ``Elektromagnetism II, 1TE626 (2025)'',
% held November 17, 2025, at Uppsala University, Sweden.
%
% Copyright (C) 2022-2025, Fredrik Jonsson, under Gnu General Public
% License (GPL) v3. See the enclosed LICENSE for details.
%
% This program is free software: you can redistribute it and/or modify
% it under the terms of the GNU General Public License as published by
% the Free Software Foundation, either version 3 of the License, or
% (at your option) any later version.
%
% This program is distributed in the hope that it will be useful,
% but WITHOUT ANY WARRANTY; without even the implied warranty of
% MERCHANTABILITY or FITNESS FOR A PARTICULAR PURPOSE.  See the
% GNU General Public License for more details.
%
% You should have received a copy of the GNU General Public License
% along with this program.  If not, see <https://www.gnu.org/licenses/>.
%
\input macros/epsf.tex
\input macros/eplain.tex
\font\ninerm=cmr9
\font\twelvesc=cmcsc10 at 12 truept
\input amssym % to get the {\Bbb E} font (strikethrough E)
\def\lecture #1 {\hsize=150mm\hoffset=4.6mm\vsize=230mm\voffset=7mm
  \topskip=0pt\baselineskip=12pt\parskip=0pt\leftskip=0pt\parindent=15pt
  \headline={\ifnum\pageno>1\ifodd\pageno\rightheadline\else\leftheadline\fi
    \else\hfill\fi}
  \def\rightheadline{\tenrm{\it F\"orel\"asning #1}
    \hfil{\it Elektromagnetism II, 1TE626 (2025)}}
  \def\leftheadline{\tenrm{\it Elektromagnetism II, 1TE626 (2025)}
    \hfil{\it F\"orel\"asning #1}}
  \noindent~\vskip-60pt\hskip-40pt{\epsfbox{macros/UU_logo_color.eps}}
  \vskip-42pt\hfill\vbox{
      \hbox{{\it Elektromagnetism II, 1TE626 (2025)}}
      \hbox{{\it Lecture Notes, Fredrik Jonsson}}
      \hbox{{\it Document Revision \today}}
      \hbox{{\it https://github.com/hp35/elmagii/}}}\vskip 36pt
    \centerline{\twelvesc F\"orel\"asning #1}
  \vskip 24pt\noindent}
\def\section #1 {\medskip\goodbreak\noindent{\bf #1}
  \par\nobreak\smallskip\noindent}
\def\subsection #1 {\medskip\goodbreak\noindent{\it #1}
  \par\nobreak\smallskip\noindent}
\def\iint{\mathop{\int\kern-8pt\int}}
\def\iiint{\mathop{\int\kern-8pt\int\kern-8pt\int}}
\def\oiint{\mathop{\int\kern-8pt\int\kern-13.2pt{\bigcirc}}}
\def\sgn{\mathop{\rm sgn}\nolimits} % sign
\def\Re{\mathop{\rm Re}\nolimits}   % real part
\def\Im{\mathop{\rm Im}\nolimits}   % imaginary part
\def\Tr{\mathop{\rm Tr}\nolimits}   % quantum mechanical trace
\def\eqq{\mathop{\vbox{\hbox{\hskip2pt?}\vskip-6pt\hbox{=}}}}
\def\encircle#1{\kern3pt#1\kern-7.6pt$\bigcirc$\kern4pt}
\def\boxit#1{\vbox{\hrule\hbox{\vrule\kern3pt
  \vbox{\kern3pt#1\kern3pt}\kern3pt\vrule}\hrule}}
\def\quote#1{\par\leftskip=36pt\rightskip=36pt\bigskip\noindent#1\par
  \leftskip=0pt\rightskip=0pt\bigskip}
\def\plan#1{\par\leftskip=36pt\rightskip=36pt\bigskip%
  \noindent{\it Sammanfattning av f{\"o}rel{\"a}sningen}\smallskip
  \noindent{\it #1}\par\leftskip=0pt\rightskip=0pt\vfill\eject}
\def\epsfig#1{\bigskip\centerline{\epsfbox{#1}}\medskip}

\lecture{5}
\centerline{\twelvesc Elektrodynamik -- Elektromagnetisk induktion}
\centerline{Fredrik Jonsson, Uppsala Universitet, 17 november 2025}
\vskip24pt

\plan{{\"A}mnet f{\"o}r f{\"o}rel{\"a}sningen beskrivs kort och koncist
  med Michael Faradays egna ord (1831), fritt tolkade som ``Ett varierande
  magnetf{\"a}lt inducerar ett elektriskt f{\"a}lt''. Vi g{\aa}r igenom
  definitionerna av magnetiskt fl{\"o}de $\Phi_{\rm M}$ och det historiskt
  betingade begreppet elektromotorisk ``kraft'' ${\cal E}$.
  Vi h{\"a}rleder Faradays induk\-tions\-lag ${\cal E}=-d\Phi_{\rm M}/dt$ i
  tv{\aa} separata och inb{\"o}rdes sammanh{\aa}llna fall.
  Det f{\"o}rsta fallet introduceras f{\"o}r sin enkelhet och intuitivt
  greppbara geometri, d{\"a}r vi studerar en rektangul{\"a}r str{\"o}mslinga
  som f{\"o}rs genom ett inhomogent magnetiskt f{\"a}lt och kommer p{\aa}
  s{\aa} s{\"a}tt fram till formen p{\aa} Faradays induktionslag via ett
  specialfall.
  Det andra fallet {\"a}r en formell h{\"a}rledning av Faradays induktionslag
  f{\"o}r en godtycklig r{\"o}rlig geometri i form av en slinga med godtycklig
  hastighet utefter sin trajektoria, samt med ett godtyckligt varierande
  inneslutet magnetf{\"a}lt. Vi noterar att Faradays induktionslag h{\"a}rleds
  enbart utifr{\aa}n Lorentz kraftlag och ej involverande vare sig Coulombs
  eller Biot--Savarts lag, eller n{\aa}got av deras derivat i det
  elektromagnetiska ``sl{\"a}kttr{\"a}det''; i och med detta har vi i Faradays
  induktionslag avsaknad av s{\aa}v{\"a}l den elektriska permittiviteten
  $\varepsilon_0$ som den magnetiska permeabiliteten $\mu_0$.
  Utifr{\aa}n Faradays induktionslag formulerar vi Lenz lag som slutsatsen
  att en inducerad str{\"o}m alltid har en riktning som motverkar orsaken
  till att den uppkom. Vi g{\aa}r utifr{\aa}n denna princip igenom en
  upps{\"a}ttning av tankeexperiment med b{\"a}ring p{\aa} tolkning av
  elektromagnetisk induktion.
  Med utg{\aa}ngspunkt i Faradays induktionslag h{\"a}rleder vi Faradays lag
  $\nabla\times{\bf E}=-\partial{\bf B}/\partial t$, ofta betecknad
  ``Maxwell--Faradays lag'', p{\aa} differential- och integralform.
  Vi avslutar med att g{\aa} igenom hur tv{\aa} str{\"o}mb{\"a}rande slingor
  p{\aa}verkar varandra genom {\"o}msesidig induktion, och vi h{\"a}rleder
  Neumanns formel f{\"o}r den {\"o}msesidiga induktansen. Specifikt g{\aa}r
  vi igenom tolkningen av Neumanns formel i form av elektromagnetisk
  reciprocitet mellan tv{\aa} slingor, d{\"a}r vi har det sm{\aa}tt
  f{\"o}rbluffande resultatet att det magnetiska fl{\"o}de som uppf{\aa}ngas
  av en slinga fr{\aa}n en str{\"o}m i den andra slingan exakt motsvaras av
  det magnetiska fl{\"o}de som den andra slingan skulle uppf{\aa}nga om
  ist{\"a}llet den f{\"o}rsta slingan drevs med exakt samma str{\"o}m.}

\section{Introduktion - Faradays induktionslag}
Vi kan i en enda mening sammanfatta {\"a}mnet f{\"o}r dagens
f{\"o}rel{\"a}sning\numberedfootnote{Vi kommer i denna f{\"o}rel{\"a}sning
  att i huvudsak f{\"o}lja Griffiths Kapitel~7, med h{\"a}rledningen av
  Faradays induktionslag p{\aa} Sid.~307--309.}
med Faradays egen slutsats, som fritt tolkad lyder:
\quote{\vbox{\hbox{{\it Ett varierande magnetf{\"a}lt inducerar ett
  elektriskt f{\"a}lt.}}\hbox{\hbox to 116pt{}--- Michael Faraday (1831)}}}
\noindent
F{\"o}r {\it elektrostatiska} system g{\"a}ller det, som tidigare visats i
F{\"o}rel{\"a}sning~2, att som en f{\"o}ljd av att rotationen av det elektriska
f{\"a}ltet alltid {\"a}r noll i elektrostatiska system, s{\aa} {\"a}r via Stokes
teorem integralen av det elektriska f{\"a}ltet {\"o}ver en godtycklig sluten
slinga $\Gamma$ alltid {\"a}r noll, det vill s{\"a}ga att\numberedfootnote{Recap
  fr{\aa}n F{\"o}rel{\"a}sning~3: F{\"o}r {\it statiska} elektriska f{\"a}lt
  g{\"a}ller alltid att $\nabla\times{\bf E}={\bf 0}$, s{\aa} med Stokes
  teorem ({\it Curl Theorem}) har vi i elektrostatiken trivialt att
  $$
    \oint_{\Gamma}{\bf E}\cdot d{\bf l}
      =\iint_S\underbrace{\nabla\times{\bf E}}_{={\bf 0}}\cdot d{\bf S}
      =0.
  $$}
$$
  \oint_{\Gamma}{\bf E}\cdot d{\bf l}=0.
  \qquad\hbox{(Statiskt!)}
$$
Michael Faraday observerade 1831 att i tidsberoende (icke-statiska = dynamiska)
f{\"a}lt s{\aa} g{\"a}ller inte detta, utan f{\"a}ltet driver en str{\"o}m som
lyder
$$
  \oint_{\Gamma}{\bf E}\cdot d{\bf l}=-{{d\Phi_{\rm M}}\over{dt}},
  \qquad\hbox{(Dynamiskt!)}
$$
d{\"a}r $\Phi_{\rm M}$ {\"a}r det {\it magnetiska fl{\"o}det}, som vi strax
kommer att definiera. Vi brukar i vardagligt tal kalla denna ekvation
{\it Faradays induktionslag}\numberedfootnote{Denna beteckning {\"a}r
  i sig lite olycklig, d{\aa} vi l{\"a}tt kan associera denna med
  Faradays lag p{\aa} differentialform som vi inom kort kommer att
  stifta bekantskap med,
  $$
    \nabla\times{\bf E}=-{{\partial{\bf B}}\over{\partial t}}.
  $$
  Dock kommer vi inom kort att visa hur denna kan h{\"a}rmedas ur just
  Faradays induktionslag p{\aa} formen som involverar magnetiska fl{\"o}det
  $\Phi_{\rm M}$. Vissa textb{\"o}cker inom elektromagnetism tar Faradays
  lag p{\aa} denna differentialform som ett {\it axiom} (postulat) utan
  att visa p{\aa} hur formen uppkommer, och g{\aa}r helt enkelt till att
  visa hur induktionslagen erh{\aa}lls fr{\aa}n denna form; detta {\"a}r
  dock fusk och ett cirkelresonemang som vi skall h{\aa}lla oss borta
  fr{\aa}n i m{\"o}jligaste m{\aa}n.},
eller kort och gott bara {\it induktionslagen}.

Vi skall h{\"a}r notera att Faradays induktionslag, fr{\aa}n vilken vi kommer
att h{\"a}rleda Faradays lag (eller ``Maxwell--Faradays lag'', som
engelsk-spr{\aa}kig litteratur ofta betecknar den) {\it inte} {\"a}r m{\"o}jlig
att h{\"a}rleda enbart fr{\aa}n elektrostatiska eller magnetostatiska teorem
som Coulombs eller Biot--Savarts lag.
Induktion {\"a}r ett strikt {\it dynamiskt} fenomen d{\"a}r vi v{\"a}ljer att
antingen
\medskip
\item{A.}{H{\"a}rleda den fr{\aa}n Lorentz-kraften plus information om
  hur magnetf{\"a}ltet ${\bf B}({\bf x},t)$ beror i tiden, eller}
\item{B.}{Helt enkelt godta den empiriskt verifierade ``magnetiska
  fl{\"o}desregeln'' som ett axiom fr{\aa}n vilket vi kan h{\"a}rleda
  Faradays lag.}
\medskip
\noindent
Vi kommer h{\"a}r sj{\"a}lvfallet att formellt h{\"a}rleda fram Faradays lag
fr{\aa}n Lorentz-kraften, d{\aa} det vore n{\"a}stintill moraliskt
f{\"o}rkastligt i en kurs som denna att inte ta tillf{\"a}llet i akt och en
g{\aa}ng f{\"o}r alla visa p{\aa} ursprunget f{\"o}r denna fundamentala lag
inom elektrodynamiken.\numberedfootnote{Det {\"a}r mycket vanligt att
  se ``h{\"a}rledningar'' av Faradays induktionslag (``magnetiska
  fl{\"o}des\-lagen'') ${\cal E}=-d\Phi_{\rm M}/dt$ utg{\aa}ende
  ifr{\aa}n Faradays lag $\nabla\times{\bf E}=-\partial{\bf B}/\partial t$
  som om den senare vore en axiomatisk sanning som inte beh{\"o}ver bevisas.
  (Faradays lag kommer ist{\"a}llet i denna f{\"o}rel{\"a}sning att h{\"a}rledas
  fr{\aa}n induktionslagen.) Denna approach med Faradays lag som ett axiom
  ger endast ett cirkelresonemang, d{\"a}r den ena vyn av induktion
  {\"o}msesidigt bevisar den andra. Om man fr{\aa}gar exempelvis ChatGPT,
  Grok eller n{\aa}gon annan LLM ({\it large language model}) s{\aa} finner
  man sorgligt nog direkt att svaret man f{\aa}r s{\aa} gott som uteslutande
  bygger p{\aa} just Faradays lag som ett axiom.}

\section{Grundl{\"a}ggande begrepp inf{\"o}r Faradays induktionslag}
\subsection{Definition: Magnetiskt fl{\"o}de}
I likhet med det elektriska fl{\"o}det $\Phi_{\rm E}$ som vi introducerade i
F{\"o}rel{\"a}sning~1, definierar vi det {\it magnetiska fl{\"o}det}
$\Phi_{\rm M}$ som integralen av normalkomponenten av det magnetiska f{\"a}ltet
${\bf B}$ {\"o}ver den yta $S$ som innesluts av en sluten trajektoria $\Gamma$,
som
$$
  \Phi_{\rm M}=\iint_S{\bf B}\cdot d{\bf S}.
$$
\epsfig{figs/magflow.1}\noindent
L{\aa}t oss nu f{\"o}rst se hur detta magnetiska fl{\"o}de kan t{\"a}nkas ha
ett tidsberoende. Ur definitionen av fl{\"o}det $\Phi_{\rm M}$ ser vi direkt
att om antingen
\medskip
\item{1.}{Det magnetiska f{\"a}ltet som s{\aa}dant {\"a}r tidsberoende,
  ${\bf B}={\bf B}({\bf x},t)$,}
\item{2.}{Ytan hos den slutna slingan $\Gamma$ {\"a}ndras, eller}
\item{3.}{Den slutna slingan $\Gamma$ p{\aa} n{\aa}got s{\"a}tt {\"a}ndras
  i rummet, exempelvis genom att roteras eller translateras,}
\medskip
\noindent
s{\aa} kommer det magnetiska fl{\"o}det $\Phi_{\rm M}=\Phi_{\rm M}(t)$ att f{\aa}
ett tidsberoende.

\subsection{Magnetisk fl{\"o}dest{\"a}thet och lite terminologi}
Den korrekta svenska beteckningen f{\"o}r det som vi lite l{\"o}st kallat
``magnetf{\"a}lt'' (${\bf B}$-f{\"a}ltet) {\"a}r {\it magnetisk
fl{\"o}dest{\"a}thet}. Anledningen till denna terminologi st{\aa}r ganska
klar i och med vad vi diskuterat ovan, d{\aa} det {\it magnetiska fl{\"o}det}
definieras av integralen
$$
  \Phi_{\rm M}=\iint_S{\bf B}\cdot d{\bf S},
$$
vilket direkt g{\"o}r att vi b{\"o}r associera ``magnetf{\"a}ltet'' ${\bf B}$
med en slags {\it densitet av fl{\"o}de per ytenhet}, eller kort och gott som
just en {\it magnetisk fl{\"o}dest{\"a}thet}. Vi kan p{\aa} s{\"a}tt och vis
s{\"a}ga att den magnetiska fl{\"o}dest{\"a}theten ${\bf B}$ {\"a}r ett
m{\aa}tt p{\aa} hur m{\aa}nga magnetiska f{\"a}ltlinjer som sk{\"a}r en
yta per ytenhet.
En annan beteckning f{\"o}r ${\bf B}$-f{\"a}ltet {\"a}r {\it magnetstyrka},
vilken dock {\"a}r betydligt mer intets{\"a}gande, {\"a}ven om det ger en
parallell association till dess elektriska motsvarighet ${\bf E}$-f{\"a}ltet
i form av {\it elektrisk f{\"a}ltstyrka}.

\subsection{Definition: Elektromotorisk ``kraft'' - EMK}
Vi rekapitulerar att Lorentzkraften p{\aa} en punktladdning $q$ i r{\"o}relse
med hastigheten ${\bf v}$ i ett kombinerat elektriskt och magnetiskt f{\"a}lt
{\"a}r given som
$$
  {\bf F}=q\big[{\bf E}+({\bf v}\times{\bf B})\big].
$$
Om vi t{\"a}nker oss att vi f{\"o}r en laddad partikel l{\"a}ngs en sluten
slinga $\Gamma$, och att vi under denna r{\"o}relse integrerar den kraft som
verkar p{\aa} punktladdningen och dividerar denna med laddningen i sig, s{\aa}
f{\aa}r vi den resulterande potentialskillnad som agerar f{\"o}r att skicka
laddningen som en str{\"o}m genom slingan.
Den potentialskillnad som ackumulerats runt slingan {\"a}r d{\aa}
$$
  {\cal E}=\oint_{\Gamma}\bigg({{{\bf F}}\over{q}}\bigg)\cdot d{\bf l}
    =\oint_{\Gamma}[{\bf E}+({\bf v}\times{\bf B})\big]\cdot d{\bf l},
$$
vilken brukar betecknas med den olyckligtvis t{\"a}mligen missvisande termen
``elektromotorisk kraft'', eller kort och gott EMK. Att detta inte {\"a}r en
``kraft'' i egentlig bem{\"a}rkelse {\"a}r tydligt fr{\aa}n den fysikaliska
dimensionen hos ${\cal E}$, som {\"a}r~V (volt), men termen har satt sig
fr{\aa}n dess historiska sammanhang, och {\"a}r idag den allm{\"a}nt
vedertagna.\numberedfootnote{Som ett uttryck f{\"o}r v{\aa}r allm{\"a}nna
  irritation {\"o}ver detta spr{\aa}kbruk, som g{\aa}r tv{\"a}rs emot den
  fysikaliska dimensionen, s{\aa} kommer vi fram{\"o}ver genomg{\aa}ende
  att anv{\"a}nda citattecken n{\"a}rhelst denna ``kraft'' n{\"a}mns!}
Likaledes {\"a}r symbolen ${\cal E}$ f{\"o}r den elektromotoriska ``kraften''
olycklig, d{\aa} den ju ger intryck av att vi har att g{\"o}ra med ett
elektriskt f{\"a}lt, vilket ju heller ej {\"a}r fallet, men {\aa}terigen
s{\aa} {\"a}r ``${\cal E}$'' allm{\"a}nt anv{\"a}nt inom litteraturen,
s{\aa} vi h{\aa}ller kvar vid denna.

En annan lite udda aspekt {\"a}r hur vi skall se p{\aa} denna elektromotoriska
``kraft'' f{\"o}r en {\it sluten slinga} $\Gamma$, som ju rimligen har
startpunkten f{\"o}r integralen exakt sammanfallande med slutpunkten.
Med andra ord:
\quote{{\it Hur kan vi ha en potentialskillnad i en och samma punkt i rummet?}}
\noindent
Svaret p{\aa} denna paradox {\"a}r att den elektromotoriska ``kraften'' {\"a}r
ett rent konceptuellt begrepp som inf{\"o}rts som ett {\it skal{\"a}rt m{\aa}tt
p{\aa} hur en noll-skild rotation av ett f{\"a}lt yttrar sig d{\aa} vi
traverserar en sluten trajektoria genom det}.
Vi kan se det som den sp{\"a}nning som skulle alstras i en ledare l{\"a}ngs med
trakektorian $\Gamma$, d{\"a}r vi kan t{\"a}nka oss att vi klippt av slingan
och kopplat in en voltmeter {\"o}ver de l{\"o}sa {\"a}ndarna som h{\aa}lls
mycket n{\"a}ra varandra.\numberedfootnote{Vi kan h{\"a}r p{\aa}minna
  oss om att vi i F{\"o}rel{\"a}sning~2 tog fram att vi f{\"o}r
  {\it elektrostatiska} f{\"a}lt hade att $\nabla\times{\bf E}={\bf 0}$; detta
  noll-resultat {\"a}r n{\aa}got som vi nu l{\"a}mnar bakom oss i och med att
  vi nu kommer att introducera tidsberoende, {\it elektrodynamiska} f{\"a}lt.}
\vfill\eject

\section{Faradays induktionslag h{\"a}rledd f{\"o}r en rektangul{\"a}r
  slinga med konstant hastighet}
Innan vi tar itu en {\it generell} h{\"a}rledning av Faradays induktionslag,
som tyv{\"a}rr riskerar att f{\aa} oss att fastna i vektoralgebra och tappa
fokus p{\aa} sj{\"a}lva k{\"a}rnan i den, l{\aa}t oss f{\"o}rst betrakta ett
f{\"o}renklat specialfall med en rektangul{\"a}r slinga $\Gamma$ som med
konstant hastighet ${\bf v}_0$ dras ut ur ett rektangul{\"a}rt omr{\aa}de med
ett i {\"o}vrigt homogent magnetf{\"a}lt ${\bf B}_0$, ortogonalt mot slingans
plan och ortogonalt mot den konstanta hastigheten ${\bf v}_0$.
\epsfig{figs/faradayrect.1}\noindent
L{\aa}t oss se vad detta h{\"o}gst f{\"o}renklade system kan ge i form av
elektromotorisk ``kraft''. Till att b{\"o}rja med, s{\aa} har vi f{\"o}r
samtliga fyra segment av $\Gamma$ att Lorentz-kraften, om vi antar att inga
statiska elektriska f{\"a}lt {\"a}r med i problemet, beskrivs av
$$
  \eqalign{
    {\bf F}&=q({\bf v}\times{\bf B}_0)\cr
      &=q({\bf e}_z v_0)\times({\bf e}_y B_0))\cr
      &=-q v_0 B_0 {\bf e}_x.\cr
  }
$$
\vfill\eject
N{\"a}r vi integrerar denna kraft, normaliserad med laddningen $q$, l{\"a}ngs
med $\Gamma$, s{\aa} ser vi att segmenten \encircle{2} och \encircle{4} som
{\"a}r ortogonala mot denna kraft ger noll i bidrag, eftersom
skal{\"a}rprodukten ${\bf F}\cdot d{\bf l}$ d{\"a}r {\"a}r identiskt noll.
Med andra ord {\"a}r det bara segmenten \encircle{1} och \encircle{3} som
kan ge bidrag, och eftersom segment \encircle{3} ligger utanf{\"o}r
magnetf{\"a}ltet och d{\"a}rmed {\"a}ven det ger noll i bidrag till
linjeintegralen, s{\aa} {\"a}r det endast segment \encircle{1} som ger ett
nettobidrag till v{\aa}r elektromotoriska ``kraft'',
$$
  \eqalign{
    {\cal E}&=\oint_{\Gamma}\bigg({{{\bf F}}\over{q}}\bigg)\cdot d{\bf l}
       =\oint_{\Gamma}({\bf v}\times{\bf B}_0)\cdot d{\bf l}\cr
      &=\oint_{\hbox{\encircle{1}}}({\bf v}\times{\bf B}_0)\cdot d{\bf l}
          +\oint_{\hbox{\encircle{2}}}\underbrace{
              ({\bf v}\times{\bf B}_0)\cdot d{\bf l}
           }_{=0,\ {\bf v}\parallel d{\bf l}}
          +\oint_{\hbox{\encircle{3}}}
              \underbrace{({\bf v}\times{\bf 0})\cdot d{\bf l}}_{=0}
          +\oint_{\hbox{\encircle{4}}}\underbrace{
              ({\bf v}\times{\bf B}_0)\cdot d{\bf l}
           }_{=0,\ {\bf v}\parallel d{\bf l}}\cr
      &=\int_{\hbox{\encircle{1}}}(-v_0 B_0 {\bf e}_x)\cdot({\bf e}_x dx)
       =-v_0 B_0\int_{\hbox{\encircle{1}}}dx\cr
      &=-v_0 B_0 L.\cr
  }
$$
Samtidigt har vi att det magnetiska fl{\"o}det $\Phi_{\rm M}$ ges av
ytintegralen\numberedfootnote{Vi skall h{\"a}r komma ih{\aa}g
  att integrationsriktningen som vi valt f{\"o}r den slutna linjeintegralen
  {\"o}ver slingan $\Gamma$ dessutom best{\"a}mmer vilken riktning v{\aa}ra
  ytelement $d{\bf S}$ har, med den sedvanliga ``h{\"o}gerhands\-regeln''.
  I detta fall har vi valt en integration som g{\aa}r {\it moturs} i slingans
  plan, s{\aa} som vi ser den i figuren, och ytelementen har s{\aa} en
  {\it normalriktning som pekar ut ur planet}, i negativ ${\bf e}_y$-riktning.}
{\"o}ver den yta $S$ som innesluts av slingan $\Gamma$ och har ett nollskilt
bidrag fr{\aa}n det magnetiska f{\"a}ltet. Det magnetiska fl{\"o}det har en
f{\"o}r{\"a}ndring i tiden som ges av
$$
  \eqalign{
    {{d\Phi_{\rm M}}\over{dt}}
       &={{d}\over{dt}}\iint_{S\land{\bf B}\ne{\bf 0}}{\bf B}_0\cdot d{\bf S}
       ={{d}\over{dt}}\iint_{S\land{\bf B}\ne{\bf 0}} (B_0{\bf e}_y)\cdot
         (\underbrace{-{\bf e}_ydA}_{=d{\bf S}})
       =-B_0{{d}\over{dt}}\big((h-v_0 t)L\big)\cr
      &=\underbrace{v_0 B_0 L}_{=-{\cal E}}\cr
  }
$$
Ur detta h{\"o}gst f{\"o}renklade resonemang kan vi dra slutsatsen att den i
slingan $\Gamma$ genererade elektromotoriska ``kraften'' ges som
f{\"o}r{\"a}ndringen i tid av det av slingan inneslutna magnetiska fl{\"o}det,
med omv{\"a}nt tecken, som
$$
  {\cal E}=-{{d\Phi_{\rm M}}\over{dt}}.
$$
Detta samband sammanfattar {\it Faradays induktionslag}\numberedfootnote{Ibland
  kallas denna kort och gott f{\"o}r ``Faradays lag'' eller ``magnetiska
  fl{\"o}deslagen''; vi v{\"a}ljer h{\"a}r dock att beh{\aa}lla den
  formella beteckningen f{\"o}r att inte blanda ihop denna induktionslag
  med den lag p{\aa} differentialform som vi inom kort kommer att
  h{\"a}rleda fr{\aa}n denna.},
vilken vi erinrar oss h{\"a}r har h{\"a}rletts fram f{\"o}r en specifik
geometri och med ett konstant magnetf{\"a}lt ${\bf B}_0$, i vilken det
magnetiska fl{\"o}det endast p{\aa}verkas genom att den slutna slingan
traverserar magnetf{\"a}ltet s{\aa} att fl{\"o}det successivt minskar.

Om vi hade haft det homogena magnetf{\"a}ltet t{\"a}ckande hela den r{\"o}rliga
slingans yta, s{\aa} hade segmentet \encircle{3} gett ett till beloppet lika
stort men motriktat bidrag som segmentet \encircle{1}, och den elektromotoriska
''netto-kraften'' hade blivit noll.
\vfill\eject

\subsection{Observation~I kring Faradays induktionslag
  - Avsaknad av permittivitet och permebilitet}
Notera att Faradays induktionslag enligt ovan
``h{\"a}rleddes''\numberedfootnote{I den m{\aa}n vi {\"o}verhuvud taget kan
  tala om en ``h{\"a}rledning'' med anv{\"a}ndande av ett specialfall!}
enbart under antagandet om Lorentz-kraften p{\aa} laddade partiklar i
r{\"o}relse.
Vi har i h{\"a}rledningen inte anv{\"a}nt vare sig Coulombs lag eller n{\aa}gon
av de fr{\aa}n den lagen h{\"a}rledda f{\"o}ljdsatserna, och d{\"a}rmed kan vi
konstatera att den elektriska permittiviteten $\varepsilon_0$ lyser med sin
fr{\aa}nvaro.
Vi har ej heller anv{\"a}nt Biot--Savarts lag eller n{\aa}got av de fr{\aa}n
denna h{\"a}rledda teoremen l{\"a}ngre ner i det ``elektromagnetiska
sl{\"a}kttr{\"a}det'', s{\aa} {\"a}ven den magnetiska permeabiliteten $\mu_0$
lyser med sin fr{\aa}nvaro.
Det enda som vi i h{\"a}rledningen av Faradays lag har tagit som ett axiom
{\"a}r just Lorentz kraftlag, vilket ger oss vid hand att vi {\"a}r n{\aa}got
nytt p{\aa} sp{\aa}ret.

\subsection{Observation~II kring Faradays induktionslag
  - Fysisk vs virtuell slinga}
Utifr{\aa}n diskussionen ovan kring alstrande av en elektromotorisk ``kraft''
och integralen av en normaliserad kraft ${\bf F}$ verkande p{\aa} en laddning
$q$, s{\aa} har vi tyv{\"a}rr letts in i tankarna att detta med Faradays
induktionslag i grund och botten bara skulle handla om fysiska str{\"o}mslingor
och hur vi kan generera str{\"o}m induktivt i klassiska generatorer och
liknande. H{\"a}r skall vi dock komma ih{\aa}g att den elektromotoriska
``kraften'' ju {\"a}r definierad i termer av en sluten linjeintegral som
arbetar med {\it f{\"a}lten i sig} som integrand. I grund och botten {\"a}r
ju den elektromotoriska kraften ingenting annat {\"a}n ett {\it rent
matematiskt m{\aa}tt p{\aa} rotationen av det elektriska f{\"a}lt som
{\"a}r inneslutet}, som vi strax skall se.
Med andra ord finns det ingenting som hindrar oss att evaluera detta m{\aa}tt
som en integral {\"o}ver en sluten {\it virtuell slinga} $\Gamma$ som
innesluter n{\aa}got tidsvarierande magnetiskt (eller f{\"o}r den delen
elektriskt) f{\"a}lt.

Samtidigt {\"a}r det magnetiska fl{\"o}det $\Phi_{\rm M}$ {\"a}ven det bara en
matematisk konstruktion i form av en integral {\"o}ver en yta $S$ innesluten
av slingan $\Gamma$, virtuell eller fysisk, och {\"a}ven detta m{\aa}tt kan
sj{\"a}lvfallet integreras {\"o}ver en yta som inte n{\"o}dv{\"a}ndigtvis
m{\aa}ste ringas in av just en {\it fysisk} str{\"o}mslinga.

Med andra ord visar detta p{\aa} att Faradays induktionslag handlar om n{\aa}got
djupare {\"a}n bara generatorer, med b{\"a}ring p{\aa} en tidsberoende
{\it koppling mellan elektriska och magnetiska f{\"a}lt i sig}.
P{\aa} s{\"a}tt och vis kan vi s{\"a}ga att det {\"a}r {\it precis vid denna
punkt i kursen som vi b{\"o}rjar formulera en sammanh{\aa}llen teori f{\"o}r
elektrodynamik och i f{\"o}rl{\"a}ngningen en modell f{\"o}r elektromagnetisk
v{\aa}gutbredning}.\numberedfootnote{Vilket r{\aa}kar vara just den
  f{\"o}rsta tentamensuppgiften, som under F{\"o}rel{\"a}sning~1
  l{\"a}mnades ut p{\aa}
  {\tt https://github.com/hp35/elmagii/blob/main/lect-01/extras/examprob.pdf}}

\subsection{Observation~III kring Faradays induktionslag
  - Negativt tecken och Lenz lag}
I uttrycket f{\"o}r Faradays induktionslag ser vi {\"a}ven att tidsderivatan av
det magnetiska fl{\"o}det $\Phi_{\rm M}$ f{\"o}rekommer med ett negativt tecken i
h{\"o}gerledet. Detta negativa tecken {\"a}r i grunden en signatur av att den
genererade elektromotoriska ``kraften'', eller om vi s{\aa} vill den i en fysisk
slinga genererade str{\"o}mmen, alltid kommer att genereras {\it s{\aa} att den
har en riktning som motverkar k{\"a}llan till induktionen}.
Denna princip, som brukar betecknas med {\it Lenz lag}, bist{\aa}r med ett
synnerligen kraftfullt verktyg n{\"a}r det kommer till punkten att vi skall
tolka ett resultat eller g{\"o}ra en {\it sanity check} p{\aa} att ett resultat
verkar rimligt.

Specifikt i och med h{\"a}rledningen av Faradays induktionslag enligt ovan,
s{\aa} har vi just r{\aa}kat ta fram formen p{\aa} Lenz lag f{\"o}r
elektromotorisk ``kraft'' genererad i det klassiska problemet f{\"o}r en
ledande stav som r{\"o}r sig ortogonalt mot sin egen axel och ortogonalt mot
ett omgivande konstant magnetf{\"a}lt, som\numberedfootnote{Griffiths
  problem~7.7, sid.~310.}
$$
  {\cal E}=-v_0 B_0 L.
$$
\vfill\eject

\section{Lenz lag som rimlighetsbed{\"o}mning av l{\"o}sningar till
  induktionsproblem}
{\it Lenz lag s{\"a}ger att en inducerad str{\"o}m har en riktning som
motverkar orsaken till att den uppkom.} Detta inneb{\"a}r att om
magnetf{\"a}ltet genom en ledande slinga (det magnetiska fl{\"o}det
$\Phi_{\rm M}$) {\"o}kar, s{\aa} kommer den i slingan inducerade str{\"o}mmen
att ha en riktning som skapar ett magnetf{\"a}lt som motverkar {\"o}kningen.
Motsatt g{\"a}ller att om det magnetiska fl{\"o}det minskar, s{\aa} kommer den
inducerade str{\"o}mmen att skapa ett magnetf{\"a}lt som motverkar minskningen.

Denna princip {\"a}r oerh{\"o}rt kraftfull n{\"a}r det g{\"a}ller att f{\aa}
en k{\"a}nsla f{\"o}r induktionsproblem i allm{\"a}nhet, och kan med f{\"o}rdel
anv{\"a}ndas som en rimlighetsbed{\"o}mning av l{\"o}sningar som tagits fram i
induktionsproblem. Med detta sagt, l{\aa}t oss applicera Lenz lag p{\aa} ett
antal tankeexperiment enligt f{\"o}ljande.
\epsfig{figs/lenz.1}\noindent
\vfill\eject

\section{Generell h{\"a}rledning av Faradays induktionslag}
Utifr{\aa}n en illustrativ men mycket f{\"o}renklad geometri har vi s{\aa}
l{\aa}ngt visat p{\aa} en {\it sannolik} form av Faradays induktionslag, men
det vore n{\"a}stintill skamligt av oss att bara acceptera denna som ett faktum
utan att g{\aa} in p{\aa} hur vi generellt och formellt kan h{\"a}rleda den,
om {\"a}n att det nu blir en aning st{\"o}kigt. Vi kommer nu att introducera
en generell trajektoria $\Gamma(t)$, l{\"a}ngs med vilken varje linjeelement
till{\aa}ts ha en godtycklig hastighet ${\bf v}({\bf x},t)$ som har ett
godtyckligt beroende i tid och rum. Denna generalitet g{\"o}r att {\"a}ven den
omslutna arean och riktningen av ytnormalen till{\aa}ts att variera fritt.
Vi till{\aa}ter vidare att det av trajektorian inneslutna magnetf{\"a}ltet
${\bf B}({\bf x},t)$ till{\aa}ts att variera fritt i tid och
rum.\numberedfootnote{Leibniz integralregel n{\"a}mns tyv{\"a}rr inte
  i Griffiths, oturligt nog eftersom det {\"a}r ett statement som i m{\aa}ngt
  och mycket skulle g{\"o}ra att vi bara skulle kunna referera till resultatet
  h{\"a}r.\goodbreak
  \noindent{\tt https://en.wikipedia.org/wiki/Leibniz\_integral\_rule}}
\epsfig{figs/faradayrib.1}\noindent
L{\aa}t oss nu se vad f{\"o}r{\"a}ndringen $d\Phi_{\rm M}$ i magnetiskt fl{\"o}de
fr{\aa}n tiden $t$ till $t+dt$ kan t{\"a}nkas vara f{\"o}r en s{\aa} generell
konstruktion.
D{\aa} trajektorian $\Gamma(t)$ r{\"o}r sig, fr{\aa}n tiden $t$ till tiden
$t+dt$ ``strax d{\"a}refter'', kommer f{\"o}r{\"a}ndringen i magnetiskt
fl{\"o}de att ges som alla delbidrag i den ``remsa'' som i rummet beskrivs
av zonen mellan $\Gamma(t)$ och $\Gamma(t+dt)$,
$$
  d\Phi_{\rm M}=\Phi_{\rm M}(t+dt)-\Phi_{\rm M}(t)
    =\iint_{\hbox{remsa}}{\bf B}({\bf x},t)\cdot d{\bf S},
$$
d{\"a}r areaelementen $d{\bf S}$ l{\"a}ngs med remsan ges som vektorprodukten
mellan den str{\"a}cka som elementet vid punkten ${\bf x}$ p{\aa} trajektorian
tillryggal{\"a}gger under tiden $dt$ samt linjeelementet $d{\bf l}$ l{\"a}ngs
med trajektorian vid tiden $t$,
$$
  d{\bf S}=({\bf v}({\bf x},t)dt)\times(d{\bf l})
    =({\bf v}({\bf x},t)\times d{\bf l})\,dt
$$
Utifr{\aa}n detta, med areaelementen uttryckta som en produkt mellan
linjeelement $d{\bf l}$ och infinitesimala tidssteg $dt$, ser vi direkt att
vi kan ers{\"a}tta areaintegralen {\"o}ver v{\aa}r remsa med en linjeintegral,
som
$$
  \eqalign{
    d\Phi_{\rm M}
      &=\iint_{\hbox{remsa}}{\bf B}({\bf x},t)\cdot d{\bf S}\cr
      &=\oint_{\Gamma(t)}{\bf B}({\bf x},t)
             \cdot\underbrace{
               ({\bf v}({\bf x},t)\times d{\bf l})\,dt
             }_{d{\bf S}\hbox{ l{\"a}ngs }\Gamma(t)}\cr
      &=dt\oint_{\Gamma(t)}{\bf B}({\bf x},t)
             \cdot({\bf v}({\bf x},t)\times d{\bf l}).\cr
  }
$$
Med andra ord kan vi formulera f{\"o}r{\"a}ndringen av det magnetiska fl{\"o}det
genom trajektorian $\Gamma(t)$ som den s{\aa} gott som f{\"o}rv{\"a}ntade
tidsderivatan
$$
  \eqalign{
    {{d\Phi_{\rm M}}\over{dt}}
      &=\oint_{\Gamma(t)}{\bf B}({\bf x},t)\cdot
           ({\bf v}({\bf x},t)\times d{\bf l}).
  }
$$
L{\aa}t oss med denna deriviata formulerad g{\aa} vidare med att analysera i
mer detalj hur en laddning som f{\"a}rdas l{\"a}ngs den i sig {\it r{\"o}rliga
trajektorian} $\Gamma(t)$ kommer att f{\"o}rflyttas.
\epsfig{figs/faradayelement.1}\noindent
Om vi t{\"a}nker oss att en laddning har hastigheten ${\bf u}({\bf x},t)$
l{\"a}ngs med trajektorian $\Gamma(t)$, s{\aa} {\"a}r sj{\"a}lvfallet denna
hastighet parallell med linjeelementet $d{\bf l}$ i samma punkt.
Trajektorian $\Gamma(t)$ har vid samma tidpunkt en och punkt i rummet
hastigheten ${\bf v}({\bf x},t)$, s{\aa} den t{\"a}nkta laddningens
resulterande hastighet ${\bf w}({\bf x},t)$ ges som summan av dessa tv{\aa}
komposanter som
$$
  {\bf w}({\bf x},t)={\bf u}({\bf x},t)+{\bf v}({\bf x},t).
  \qquad\hbox{(Laddningens hastighet)}
$$
\vfill\eject
Notera att eftersom ${\bf u}({\bf x},t)$ {\"a}r parallell med linjeelementet
$d{\bf l}$, s{\aa} kommer samma linjeelements kryssprodukt med
${\bf v}({\bf x},t)$ att ge en produkt som {\"a}r ortogonal mot just
${\bf u}({\bf x},t)$; med andra ord s{\aa} kan vi precis lika g{\"a}rna
ers{\"a}tta hastigheten i kryssprodukten med den totala resulterande
hastigheten ${\bf w}({\bf x},t)$ f{\"o}r en hypotetisk laddnings r{\"o}relse,
d{\aa}
$$
  \eqalign{
    {{d\Phi_{\rm M}}\over{dt}}
      &=\oint_{\Gamma(t)}{\bf B}({\bf x},t)\cdot
           \underbrace{
             \big(
               (
                 \underbrace{
                   {\bf u}({\bf x},t)
                 }_{\parallel d{\bf l}\to0}+{\bf v}({\bf x},t)
               )\times d{\bf l}
             \big)
           }_{
              {\bf w}\times d{\bf l}={\bf v}\times d{\bf l}
           }\cr
      &=\oint_{\Gamma(t)}{\bf B}({\bf x},t)\cdot
             \big({\bf w}({\bf x},t)\times d{\bf l}\big)\cr
      &=\big\{\hbox{ Griffiths {\it Triple Product (1)} }\big\}\cr
      &=\big\{\hbox{ ${\bf a}\cdot({\bf b}\times{\bf c})
                       ={\bf c}\cdot({\bf a}\times{\bf b})
                       =-({\bf b}\times{\bf a})\cdot{\bf c}$ }\big\}\cr
      &=-\oint_{\Gamma(t)}\underbrace{
        \big({\bf w}({\bf x},t)\times{\bf B}({\bf x},t)\big)
      }_{\equiv{\bf F}_{\rm M}/q}
     \cdot d{\bf l}.\cr
  }
$$
Vi ser nu po{\"a}ngen med att ers{\"a}tta den hypotetiska laddningens hastighet
${\bf v}({\bf x},t)$ l{\"a}ngs trajektorian med den totala hastigheten
${\bf w}({\bf x},t)$ i och med att kryssprodukten med det magnetiska f{\"a}ltet
exakt {\"a}r det magnetiska bidraget till Lorentz-kraften,
$$
  {\bf F}_{\rm M}=q\big({\bf w}({\bf x},t)\times{\bf B}({\bf x},t)\big),
$$
vilket i sin tur g{\"o}r att vi direkt kan tolka den sista integralen som den
elektromotoriska ``kraft'' som alstras runt trajektorian $\Gamma(t)$,
$$
    {{d\Phi_{\rm M}}\over{dt}}
      =-\oint_{\Gamma(t)}\bigg({{{\bf F}_{\rm M}}\over{q}}\bigg)\cdot d{\bf l}
      =-{\cal E}
$$
L{\aa}t oss sammanfatta detta generella resultat med att vi till slut p{\aa}
ett strikt s{\"a}tt h{\"a}rlett Faradays induktionslag f{\"o}r en godtycklig
geometri som
$$
  {\cal E}=-{{d\Phi_{\rm M}}\over{dt}}.
$$
Vi erinrar oss {\aa}terigen att denna h{\"a}rletts {\it endast utifr{\aa}n
Lorentz-kraften agerande p{\aa} en hypotetisk laddning}, och att denna lag
d{\"a}rmed dels {\"a}r i avsaknad av den elektriska permittiviteten
$\varepsilon_0$, s{\aa}v{\"a}l som den magnetiska permeabiliteten $\mu_0$.

Vi tar ocks{\aa} tillf{\"a}llet i akt att erinra oss att {\"a}ven om vi h{\"a}r
f{\"o}r enkelhets skull diskuterat Lorentzkraften s{\aa} som den skulle agerat
p{\aa} en fysisk laddning $q$ i r{\"o}relse, s{\aa} {\"a}r resultatet i form av
Faradays lag en {\it relation mellan f{\"a}lt}, specifikt visande att i
n{\"a}rvaro av tidsberoende magnetiska f{\"a}lt s{\aa} {\"a}r rotationen
f{\"o}r det elektriska f{\"a}ltet inte l{\"a}ngre noll.
Detta {\it dynamiska resultat} st{\aa}r i kontrast till det statiska fallet
vilket vi analyserade i F{\"o}rel{\"a}sning~2, d{\"a}r vi konstaterade att
$\nabla\times{\bf E}={\bf 0}$ alltid g{\"a}ller f{\"o}r {\it statiska} f{\"a}lt.

\section{Faradays induktionslag f{\"o}r station{\"a}ra slingor}
I fall d{\aa} en str{\"o}mslinga $\Gamma$ {\"a}r {\it station{\"a}r i rummet
och har en form som inte beror av tiden}, s{\aa} kan vi direkt uttrycka den
genererade elektromotoriska ``kraften'' utifr{\aa}n definitionen av det
magnetiska fl{\"o}det som\numberedfootnote{Vi betecknar h{\"a}r generellt
  en {\it yttre} verkande tidsderivata som $d/dt$, f{\"o}r p{\aa} s{\aa}
  s{\"a}tt inkludera m{\"o}jligheten att {\"a}ven sj{\"a}lva
  {\it integrationsdom{\"a}nen {\"a}r tidsberoende}.
  N{\"a}r det {\"a}r uppenbart att det endast {\"a}r magnetf{\"a}ltet i
  sig som tidsderiveras kan vi ist{\"a}llet med f{\"o}rdel anv{\"a}nda
  partialderivatan $\partial/\partial t$.}
$$
  {\cal E}=-{{d}\over{dt}}\iint_{S}{\bf B}({\bf x},t)\cdot d{\bf S}
    =-\iint_{S}{{\partial{\bf B}({\bf x},t)}\over{\partial t}}\cdot d{\bf S}
$$
Som exempel p{\aa} applikationer av Faradays induktionslag f{\"o}r
station{\"a}ra slingor kan n{\"a}mnas {\it statorn}, som {\"a}r den
station{\"a}ra, icke r{\"o}rliga komponenten i en generator, som typiskt
inneh{\aa}ller spolar av koppartr{\aa}d (lindningar) som omvandlar det
varierande magnetf{\"a}ltet fr{\aa}n magneter p{\aa} en {\it rotor} till
elektrisk sp{\"a}nning via den inducerade elektromotoriska ``kraften''.
Ett annat exempel {\"a}r pickupen p{\aa} en elgitarr eller elbas, d{\"a}r
station{\"a}ra magneter omgivna av en spole via den vibrerande metalliska
str{\"a}ngen skapar ett varierande magnetiskt f{\"a}lt, och p{\aa} liknande
s{\"a}tt inducerar en signal i form av en sp{\"a}nning.

\section{Spolar och utv{\"a}xling p{\aa} det magnetiska fl{\"o}det}
F{\"o}r en spole med $N$ varv, som vart och ett kan anses som identiskt i
sitt t{\"a}ckning av det magnetiska f{\"a}ltet och d{\"a}rmed vart och ett
upplever ett identiskt magnetiskt fl{\"o}de $\Phi_{\rm M}$ s{\aa} blir den
resulterande inducerade elektromotoriska kraften utv{\"a}xlad i samma grad,
som
$$
  {\cal E}=-N{{d\Phi_{\rm M}}\over{dt}}.
$$
F{\"o}r ett tillr{\"a}ckligt h{\"o}gt antal varv $N$ kan mycket sm{\aa}
fluktuationer i det magnetiska f{\"a}ltet detekteras, speciellt om vi
f{\"o}rst{\"a}rker B-f{\"a}ltet genom att introducera en ferrit
(j{\"a}rnk{\"a}rna) inuti spolen. Detta {\"a}r exempelvis principen som typiskt
anv{\"a}nds f{\"o}r att passivt generera signalen fr{\aa}n en elgitarrs eller
elbas pickup. Som exempel kan n{\"a}mnas att en klassisk PAF-style humbucker
p{\aa} en Gibson Les Paul typiskt har cirka $N=5\,000$ varv per spole.

\section{Faradays lag p{\aa} differentialform}
Utifr{\aa}n den generella definitionen av den elektromotoriska ``kraften''
kan vi applicera Stokes teorem\numberedfootnote{Se exempelvis innerp{\"a}rmen
p{\aa} Griffiths, {\it Curl Theorem},
  $$
    \iint_S\nabla\times{\bf A}\,d{\bf S}=\oint{\bf A}\cdot d{\bf l}.
  $$} p{\aa} den ing{\aa}ende linjeintegralen, som
$$
  {\cal E}
    \equiv\oint_{\Gamma}{\bf E}({\bf r},t)\cdot d{\bf l}
    =\underline{\underline{
        \iint_{S}(\nabla\times{\bf E}({\bf r},t))\cdot d{\bf S}
     }}
    =-{{d\Phi_{\rm M}({\bf x},t)}\over{dt}}
    =\underline{\underline{
        -\iint_S{{\partial{\bf B}({\bf x},t)}\over{\partial t}}\cdot d{\bf S}
     }}.
$$
Eftersom ytintegralen sker {\"o}ver en yta innesluten av en godtycklig
trajektoria $\Gamma$, s{\aa} inneb{\"a}r det att vi f{\"o}r integranderna
erh{\aa}ller det vektoriella sambandet\numberedfootnote{{\"A}ven h{\"a}r
  anv{\"a}nder vi partialderivatan $\partial/\partial t$ f{\"o}r att
  explicit visa att vi h{\aa}ller oss till en beskriv\-ning av en
  motsvarande station{\"a}r slinga $\Gamma$ i rummet.}
$$
  \nabla\times{\bf E}({\bf r},t)
    =-{{\partial{\bf B}({\bf x},t)}\over{\partial t}},
$$
vilket vi betecknar som {\it Faradays lag p{\aa} differentialform}.
Notera att liksom f{\"o}r Faradays induktionslag, s{\aa} {\"a}r denna ekvation
helt oberoende av den elektriska permittiviteten $\varepsilon_0$ eller
magnetiska permeabiliteten $\mu_0$, vilket vi rekapitulerar {\"a}r en effekt
av att denna lag h{\"a}rletts oberoende av Coulombs eller Biot--Savarts lagar
eller n{\aa}gon av deras derivat l{\"a}ngre ner i det elektromagnetiska
sl{\"a}kttr{\"a}det.
\vfill\eject

\section{Faradays lag p{\aa} integralform}
Vi har i princip redan fastst{\"a}llt Faradays lag p{\aa} integralform i och
med att vi tog fram att inte\-grand\-erna i ytintegralerna m{\aa}ste vara
identiska f{\"o}r en generell trajektoria $\Gamma$, men l{\aa}t oss
{\"a}nd{\aa} f{\"o}r sakens skull ta fram integralformen av Faradays lag
fr{\aa}n differentialformen.
Om vi integrerar differentialformen av Faradays lag {\"o}ver en yta $S$
innesluten av en sluten trajektoria $\Gamma$ och direkt till{\"a}mpar Stokes
teorem, s{\aa} har vi att
$$
  \iint_S(\nabla\times{\bf E}({\bf r},t))\cdot d{\bf S}
  =\underline{\underline{
     \oint_{\Gamma}{\bf E}({\bf r},t)\cdot d{\bf l}
   }}
  =-\iint_S{{\partial{\bf B}({\bf x},t)}\over{\partial t}}\cdot d{\bf S}
  =\underline{\underline{
     -{{d}\over{dt}}\iint_S{\bf B}({\bf x},t)\cdot d{\bf S}
   }},
$$
vilket vi kan sammanfatta med {\it Faradays lag p{\aa} integralform} som
$$
  \underbrace{
    \oint_{\Gamma}{\bf E}({\bf r},t)\cdot d{\bf l}
  }_{\displaystyle ={\cal E}}
    =\underbrace{
       -{{d}\over{dt}}\iint_S{\bf B}({\bf x},t)\cdot d{\bf S}
     }_{\displaystyle =-{{d\Phi_{\rm M}}\over{dt}}}.
$$
Vi kan i integralformen av Faradays lag direkt identifiera v{\"a}nsterledet som den elektromotoriska ``kraft'' ${\cal E}$ som alstras i en station{\"a}r slinga, och h{\"o}gerledet som den negativa tidsderivatan av det magnetiska fl{\"o}det, $-d\Phi_{\rm M}/dt$; med andra ord {\"a}r denna form i grunden bara {\it en manifestering av Faradays induktionslag}.

\section{Tre principiella specialfall f{\"o}r induktion}
I Faradays ursprungliga experiment 1831, som vi kan se som tidpunkten d{\"a}r
elektromagnetism och induktion f{\"o}r f{\"o}rsta g{\aa}ngen demonstrerades,
s{\aa} presenterades i huvudsak tre fundamentala
fall.\numberedfootnote{Faradays tre experiment, se Griffiths sektion~7.2,
  sid.~312.}${^{,}}$%
  \numberedfootnote{Fr{\aa}ga till l{\"a}saren: Om man till{\"a}mpar Lenz
    lag p{\aa} dessa tre fall, s{\aa} som de h{\"a}r {\"a}r illustrerade,
    st{\"a}mmer riktningarna p{\aa} de utritade str{\"o}mmarna?}

\epsfig{figs/inductioncases.1}\noindent
Vi kan principiellt s{\"a}rskilja tre grundl{\"a}ggande fall f{\"o}r
elektromagnetisk induktion med Faradays lag:
\medskip
\item{1.}{En r{\"o}rlig slinga $\Gamma$ som traverserar ett inhomogent men
  i {\"o}vrigt konstant magnetf{\"a}lt ${\bf B}_0$, d{\"a}r det magnetiska
  fl{\"o}det $\Phi_{\rm M}$ genom slingan f{\"o}r{\"a}ndras genom slingans
  r{\"o}relse.}
\item{2.}{En slinga $\Gamma$ som {\"a}r fix i rummet, d{\"a}r ett inhomogent
  men i {\"o}vrigt konstant magnetf{\"a}lt ${\bf B}_0$ r{\"o}r sig {\"o}ver
  slingan, d{\"a}r det magnetiska fl{\"o}det $\Phi_{\rm M}$ f{\"o}r{\"a}ndras
  genom att magnetf{\"a}ltet varierar genom magnetf{\"a}ltets r{\"o}relse.}
\item{3.}{En slinga $\Gamma$ som {\"a}r fix i rummet med ett tidsvarierande
  (dynamiskt) magnetf{\"a}lt ${\bf B}(t)$, d{\"a}r det magnetiska fl{\"o}det
  $\Phi_{\rm M}$ f{\"o}r{\"a}ndras genom magnetf{\"a}ltets variation i tiden.}
\vfill\eject

\section{{\"O}msesidig induktans}
Vi har under F{\"o}rel{\"a}sning~4 visat hur Biot--Savarts lag beskriver hur
magnetf{\"a}lt kan alstras genom str{\"o}m som traverserar en slinga, s{\"a}g
$\Gamma'$, och vi har nu {\"a}ven visat hur Faradays lag kopplar ett
tidsvarierande magnetiskt f{\"a}lt som genom induktion kan generera str{\"o}m
i en annan slinga, s{\"a}g $\Gamma$.
Uppenbarligen kan vi med andra ord genom att skicka str{\"o}m genom en
prim{\"a}r slinga $\Gamma'$ inducera en str{\"o}m i en sekund{\"a}r slinga
$\Gamma$ utan att dessa slingor har fysisk kontakt med varandra, och beroende
p{\aa} riktning och avst{\aa}nd f{\"o}r magnetf{\"a}ltet som alstras fr{\aa}n
den prim{\"a}ra slingan $\Gamma'$ kommer vi att ha en mer eller mindre effektiv
{\"o}verf{\"o}ring av effekt {\"o}ver till den sekund{\"a}ra slingan $\Gamma$.
\epsfig{figs/inductance.1}\noindent
Fr{\aa}gan blir d{\aa}:\numberedfootnote{Vi f{\"o}ljer h{\"a}r i huvudsak
  Griffiths sid.~321--323.}
\quote{{\it Kan vi p{\aa} n{\aa}got s{\"a}tt sy ihop dessa teorier f{\"o}r
  att extrahera hur stark den induktiva {\"o}msesidiga kopplingen, eller
  den s{\aa} kallade {\it {\"o}msesidiga induktansen}, mellan slingorna
  {\"a}r?}}
\noindent
L{\aa}t oss f{\"o}rst konstatera att det magnetiska f{\"a}ltet som genereras
fr{\aa}n den prim{\"a}ra slingan\numberedfootnote{Eftersom vi i detta problem
  kan betrakta den prim{\"a}ra slingan som en {\it k{\"a}lla}, s{\aa}
  kommer vi genomg{\aa}ende att primma relevanta storheter fr{\aa}n
  denna, f{\"o}ljande den konvention vi h{\aa}ller oss till i denna kurs.}
$\Gamma'$ beskrivs av Biot--Savarts lag, som f{\"o}r detta fall och f{\"o}r
alla observationspunkter ${\bf x}$ formuleras som
linjeintegralen\numberedfootnote{Se F{\"o}rel{\"a}sning~4 eller
  Griffiths Ekv.~(5.34), sid.~224.}
{\"o}ver den prim{\"a}ra loopen $\Gamma'$,
$$
  {\bf B}({\bf x},t)={{\mu_0}\over{4\pi}}\int_{\Gamma'}
    {{{\bf I}'({\bf x}',t)\times({\bf x}-{\bf x}')}
      \over{|{\bf x}-{\bf x}'|^3}}\,dl'
    ={{\mu_0 I'(t)}\over{4\pi}}\int_{\Gamma'}
      {{d{\bf l}'\times({\bf x}-{\bf x}')}\over{|{\bf x}-{\bf x}'|^3}}.
$$
Eftersom detta direkt bist{\aa}r oss med det magnetiska f{\"a}ltet
${\bf B}({\bf x},t)$ som i s{\aa} fall t{\"a}cks av den sekund{\"a}ra loopen,
som vi i detta fall r{\"a}knar som uppbyggd av {\it observationspunkter}, och
eftersom vi vet att ett varierande f{\"a}lt genom sekund{\"a}rloopen kommer
att ge upphov till en elektromotorisk ``kraft'' {\"o}ver denna, l{\aa}t oss
d{\"a}rf{\"o}r formulera det magnetiska fl{\"o}det $\Phi_{\rm M}$ genom just
sekund{\"a}rloopen som ytintegralen av det magnetiska f{\"a}ltet (den
magnetiska fl{\"o}dest{\"a}theten!) {\"o}ver den av sekund{\"a}rloopen
inneslutna ytan,
$$
  \eqalign{
    \Phi_{\rm M}&=\iint_{S}{\bf B}({\bf x},t)\cdot d{\bf S}
        \qquad\hbox{(Genom sekund{\"a}rloopen $\Gamma$)}\cr
      &=\iint_{S}\bigg(
            \underbrace{
              {{\mu_0 I'(t)}\over{4\pi}}\int_{\Gamma'}
                {{d{\bf l}'\times({\bf x}-{\bf x}')}\over{|{\bf x}-{\bf x}'|^3}}
            }_{={\bf B}({\bf x},t)}
        \bigg)\cdot d{\bf S}\cr
      &=I'(t)\bigg[
          {{\mu_0}\over{4\pi}}\iint_{S}\bigg(
            \int_{\Gamma'}
              {{d{\bf l}'\times({\bf x}-{\bf x}')}\over{|{\bf x}-{\bf x}'|^3}}
          \bigg)\cdot d{\bf S}
        \bigg]\cr
      &=M_{\Gamma\Gamma'} I'(t),\cr
  }
$$
d{\"a}r vi definierade den {\it {\"o}msesidiga induktansen} ({\it mutual
inductance}) mellan slingorna $\Gamma$ och $\Gamma'$ som
$$
  M_{\Gamma\Gamma'}=
    {{\mu_0}\over{4\pi}}\iint_{S}\bigg(
      \int_{\Gamma'}
        {{d{\bf l}'\times({\bf x}-{\bf x}')}\over{|{\bf x}-{\bf x}'|^3}}
    \bigg)\cdot d{\bf S}.
$$
Vi kan h{\"a}r i princip n{\"o}ja oss med detta resultat, d{\aa} vi f{\aa}tt
fram ett paketerat uttryck som relaterar str{\"o}mmen $I'(t)$ i
prim{\"a}rslingan $\Gamma'$ till det magnetiska fl{\"o}det $\Phi_{\rm M}$ i
sekund{\"a}rslingan $\Gamma$, fr{\aa}n vilket vi direkt kan erh{\aa}lla den
resulterande elektromotoriska ``kraften''. Det finns dock ett par moment vi
kan g{\aa} igenom f{\"o}r att f{\"o}ra {\"o}ver den {\"o}msesidiga induktansen
p{\aa} en form som vi enklare kan extrahera fysikaliska samband fr{\aa}n.

\section{Neumanns formel f{\"o}r den {\"o}msesidiga induktansen}
Till att b{\"o}rja med backar vi bandet ett steg, och konstaterar att om vi
formulerar den magnetiska fl{\"o}det $\Phi_{\rm M}$ genom sekund{\"a}rslingan
i termer av {\it vektorpotentialen} som
$$
  \eqalign{
    \Phi_{\rm M}&=\iint_{S}{\bf B}({\bf x},t)\cdot d{\bf S}
        \qquad\hbox{(Genom sekund{\"a}rloopen $\Gamma$)}\cr
      &=\big\{\hbox{ Vektorpotential
                     ${\bf B}\equiv\nabla\times{\bf A}$ }\big\}\cr
      &=\iint_{S}(\nabla\times{\bf A}({\bf x},t))\cdot d{\bf S}\cr
      &=\big\{\hbox{ Stokes teorem (Griffiths {\it Curl Theorem}) }\big\}\cr
      &=\oint_{\Gamma}{\bf A}({\bf x},t)\cdot d{\bf l}.\cr
  }
$$
Notera nu att vektorpotentialen ${\bf A}({\bf x},t)$ som ing{\aa}r i
integranden h{\"a}rr{\"o}r fr{\aa}n prim{\"a}rslingan $\Gamma'$, {\"a}ven om
linjeintegralen i sig sker {\"o}ver sekund{\"a}rslingan $\Gamma$.
Vi kan nu utnyttja att den explicita l{\"o}sningen\numberedfootnote{Vi erinrar
  oss att vektorpotentialen ${\bf A}$ lyder Poissons ekvation (se omslaget
  p{\aa} Griffiths!) och i och med detta har en l{\"o}sning som {\"a}r p{\aa}
  exakt samma form som l{\"o}sningen f{\"o}r den skal{\"a}ra potentialen
  $\phi$ fr{\aa}n Coulombs generella ekvation; se F{\"o}rel{\"a}sning~4,
  alternativt Griffiths Ekv.~(5.66), sid.~245.}
f{\"o}r vektorpotentialen fr{\aa}n prim{\"a}rslingan lyder
$$
  {\bf A}({\bf x},t)={{\mu_0}\over{4\pi}}\oint_{\Gamma'}
    {{{\bf I}({\bf x}',t)}\over{|{\bf x}-{\bf x}'|}}\,dl',
$$
s{\aa} det magnetiska fl{\"o}det genom sekund{\"a}rslingan beskrivs av
$$
  \eqalign{
    \Phi_{\rm M}&=\oint_{\Gamma}
      \bigg(
        {{\mu_0}\over{4\pi}}\oint_{\Gamma'}
        {{{\bf I}({\bf x}',t)}\over{|{\bf x}-{\bf x}'|}}\,dl'
      \bigg)\cdot d{\bf l}.\cr
    &=\big\{\hbox{ I linjeintegralen f{\"o}r prim{\"a}rslingan {\"a}r
                   ${\bf I}'({\bf x})\,dl'=I'(t)\,d{\bf l}'$ }\big\}\cr
    &=\oint_{\Gamma}
      \bigg(
        {{\mu_0}\over{4\pi}}\oint_{\Gamma'}I'(t)
        {{d{\bf l}'}\over{|{\bf x}-{\bf x}'|}}
      \bigg)\cdot d{\bf l}.\cr
    &=\big\{\hbox{ Str{\"o}mmen $I'(t)$ samma {\"o}ver hela
                   prim{\"a}rslingan $\Gamma'$ }\big\}\cr
    &=I'(t) {{\mu_0}\over{4\pi}}\oint_{\Gamma}\oint_{\Gamma'}
        {{d{\bf l}'\cdot d{\bf l}}\over{|{\bf x}-{\bf x}'|}}\cr
    &=M_{\Gamma\Gamma'} I'(t),\cr
  }
$$
d{\"a}r vi nu har formulerat den {\"o}msesifiga induktansen i form av
{\it Neumanns formel}\numberedfootnote{Efter Franz Ernst Neumann (1798--1895),
  tysk fysiker som inom elektromagnetism fr{\"a}mst {\"a}r k{\"a}nd f{\"o}r
  att ha formulerat vektorpotentialen ${\bf A}$; icke att f{\"o}rv{\"a}xla
  med Alfred E. Neuman, som stavar sitt namn med bara ett ``n''.
  {\tt https://en.wikipedia.org/wiki/Alfred\_E.\_Neuman}}
$$
  M_{\Gamma\Gamma'}={{\mu_0}\over{4\pi}}\oint_{\Gamma}\oint_{\Gamma'}
        {{d{\bf l}'\cdot d{\bf l}}\over{|{\bf x}-{\bf x}'|}}.
$$

\subsection{Ett par observationer kring Neumanns formel}
Utifr{\aa}n den just h{\"a}rledda formen p{\aa} Neumanns formel kan vi
observera f{\"o}ljande:
\medskip
\item{1.}{Neumanns formel f{\"o}r den {\"o}msesidiga induktansen mellan
  tv{\aa} slingor $\Gamma'$ och $\Gamma$ {\"a}r en {\it rent geometrisk
  konstruktion}, skalad med den magnetiska permeabiliteten $\mu_0/4\pi$.}
\item{2.}{I den {\"o}msesidiga induktansen kan vi fritt byta beteckning
  mellan prim{\"a}r- och sekund{\"a}rslinga, med f{\"o}ljd att
  $M_{\Gamma\Gamma'}\equiv M_{\Gamma'\Gamma}$ d{\aa} det enda som p{\aa}verkas i
  Neumanns formel {\"a}r integrationsordningen,
  $$
    M_{\Gamma\Gamma'}
       ={{\mu_0}\over{4\pi}}\oint_{\Gamma}\oint_{\Gamma'}
          {{d{\bf l}'\cdot d{\bf l}}\over{|{\bf x}-{\bf x}'|}}
       ={{\mu_0}\over{4\pi}}\oint_{\Gamma'}\oint_{\Gamma}
          {{d{\bf l}\cdot d{\bf l}'}\over{|{\bf x}'-{\bf x}|}}
       \equiv M_{\Gamma'\Gamma}.
  $$
  Detta s{\"a}ger oss att det magnetiska fl{\"o}de $\Phi_{\rm M}$ som vi
  erh{\aa}ller i sekund{\"a}rslingan $\Gamma$ d{\aa} vi driver
  prim{\"a}rslingan med en str{\"o}m $I'(t)$ {\"a}r {\it exakt lika stort}
  som det magnetiska fl{\"o}de $\Phi'_{\rm M}$ som vi skulle detektera i
  prim{\"a}rslingan $\Gamma'$ om vi ist{\"a}llet drev sekund{\"a}rslingan
  $\Gamma$ med samma str{\"o}m $I'(t)$. Detta g{\"a}ller oavsett form eller
  inb{\"o}rdes riktning hos slingorna $\Gamma'$ och  $\Gamma$.}
\item{3.}{P{\aa} ett djupare plan visar denna {\"o}msesidighet mellan
  slingorna $\Gamma'$ och $\Gamma$ p{\aa} {\it elektromagnetisk
  reciprocitet}'.\numberedfootnote{Tyv{\"a}rr g{\aa}r inte Griffiths
    igenom denna synnerligen intressanta aspekt av elektromagnetism;
    f{\"o}r en djupare behandling av detta {\"a}mne, se exempelvis
    J.~D.~Jacksons standardverk {\it Classical Electrodynamics}.}}
\vfill\eject

\section{Kontinuitetsekvationen - Teaser inf{\"o}r Maxwell's ekvationer}
Vi har nu formellt h{\"a}rlett Faradays lag, och vi kan nu fr{\aa}ga oss vad
som egentligen kvarst{\aa}r innan vi har en komplett elektrodynamisk
beskrivning av de elektriska och magnetiska f{\"a}lten. Sanningen {\"a}r att
det fortfarande finns en hel del att g{\"o}ra vad g{\"a}ller v{\"a}xelverkan
mellan f{\"a}lt och materia (de s{\aa} kallade {\it konstitutiva relationerna})
som vi {\"a}nnu inte ens nosat p{\aa} d{\aa} vi helt arbetat i
vakuumdom{\"a}nen.\numberedfootnote{F{\"o}rvisso kan vi fr{\aa}ga oss om
  vakuum i n{\"a}rvaro av str{\"o}mmar av elektroner eller andra laddade
  partiklar verkligen {\"a}r att betrakta som just {\it vakuum} i ordets
  egentliga bem{\"a}rkelse; h{\"a}r ser vi dock dessa t{\"a}theter av
  laddade partiklar som s{\aa} pass l{\aa}g att vi fortfarande helt kan
  f{\"o}rsumma dem i j{\"a}mf{\"o}relse med n{\"a}r vi inom kort kommer
  att g{\aa} in p{\aa} hur ett medium polariseras i av elektriska och
  magnetiska f{\"a}lt.}
Vi har i F{\"o}rel{\"a}sning~4 tagit fram Amp\`eres lag inom magnetostatiken,
$$
  \nabla\times{\bf B}=\mu_0{\bf J}.\qquad\hbox{(Statiskt!)}
$$
Vi har samtidigt i F{\"o}rel{\"a}sning~4 h{\"a}rlett ``lagen om att laddning
inte kan f{\"o}rsvinna'', som uttrycker sambandet mellan divergensen f{\"o}r
str{\"o}mt{\"a}thet ${\bf J}$ och tidsderivatan av laddningst{\"a}theten
$\rho$ som\numberedfootnote{{\it Continuity Equation}; se Griffiths
  Ekv.~(5.29), sid.~222 samt Griffiths Ekv.~(8.4), sid.~356.}
$$
  \nabla\cdot{\bf J}+{{\partial\rho}\over{\partial t}}=0.
$$
Om vi substituerar f{\"o}r str{\"o}mt{\"a}theten i denna lag fr{\aa}n
Amp\`eres statiska lag, s{\aa} har vi med andra ord att
$$
  \nabla\cdot{\bf J}
    ={{1}\over{\mu_0}}\nabla\cdot(\nabla\times{\bf B})
    =\big\{\hbox{ Griffiths vektoridentitet }\big\}
    \equiv 0,
$$
detta trots att vi f{\"o}r att uppfylla den fundamentala lagen om laddningens
bevarande borde ha haft ett ``$-{{\partial\rho}/{\partial t}}$'' i
h{\"o}gerledet ist{\"a}llet f{\"o}r en nolla.

Tricket som James Clerk Maxwell kom p{\aa} i l{\"o}sandet av detta problem
var helt enkelt att l{\"a}gga till en term $\varepsilon_0{{\partial{\bf E}}
/{\partial t}}$ till den fria str{\"o}mt{\"a}theten, en s{\aa} kallad
{\it f{\"o}rskjutningsstr{\"o}m}, s{\aa} att vi helt enkelt ers{\"a}tter
den statiska fria str{\"o}mmen ${\bf J}$ i Amp\`eres lag med
$$
  {\bf J}\quad\to\quad{\bf J}+\varepsilon_0{{\partial{\bf E}}\over{\partial t}},
$$
det vill s{\"a}ga med Amp\`eres lag p{\aa} formen
$$
  \nabla\times{\bf B}=\mu_0\bigg(
    {\bf J}+\varepsilon_0{{\partial{\bf E}}\over{\partial t}}
  \bigg).\qquad\hbox{(Dynamiskt!)}
$$
med f{\"o}ljd att
$$
  \eqalign{
    \nabla\cdot{\bf J}
      &=\nabla\cdot\bigg(
           {{1}\over{\mu_0}}\nabla\times{\bf B}
             -\varepsilon_0{{\partial{\bf E}}\over{\partial t}}
        \bigg)\cr 
      &={{1}\over{\mu_0}}\underbrace{\nabla\cdot(\nabla\times{\bf B})}_{\equiv0}
           -\varepsilon_0{{\partial}\over{\partial t}}
           \underbrace{\nabla\cdot{\bf E}}_{=\rho/\varepsilon_0}\cr
      &=-{{\partial\rho}\over{\partial t}},\cr
 }
$$
det vill s{\"a}ga att vi med den extra termen f{\"o}r
f{\"o}rskjutningsstr{\"o}mmen n{\"a}rvarande direkt l{\"o}ser problemet med
kontinuitetsekvationen f{\"o}r dynamiska f{\"a}lt.
Det finns en liten korrektion som fortfarande beh{\"o}ver g{\"o}ras i
tolkningen av f{\"o}rskjutningsstr{\"o}mmen, som vi kommer att {\aa}terkomma
till i F{\"o}rel{\"a}sning~9 d{\aa} vi till slut kommer att s{\"a}tta samman
alla pusselbitar och till slut formulera den slutliga formen av Maxwells
ekvationer och utifr{\aa}n dem hur elektromagnetiska v{\aa}gor beskrivs
p{\aa} differentialform.\numberedfootnote{Vilket f{\"o}r {\"o}vrigt
  r{\aa}kar vara just den {\it Tentamensuppgift~1} som delades ut
  under F{\"o}rel{\"a}sning~1!}
\vfill\eject

\section{Sammanfattning av F{\"o}rel{\"a}sning~5 -- Elektromagnetisk induktion}
\item{$\bullet$}{Faradays lag kan inte h{\"a}rledas fr{\aa}n Coulombs eller
  Biot--Savarts lag, och inneh{\aa}ller som en f{\"o}ljd av detta ej heller
  de karakteristiska sp{\aa}ren fr{\aa}n dem i form av den elektriska
  permittiviteten $\varepsilon_0$ eller den magnetiska permeabiliteten $\mu_0$.}
\item{$\bullet$}{Magnetiskt fl{\"o}de definieras som
  $$
    \Phi_{\rm M}=\iint_S{\bf B}\cdot d{\bf S}.
  $$}
\item{$\bullet$}{Den elektromotoriska ``kraften'' (en mycket missvisande term)
  runt en sluten slinga $\Gamma$ definieras som
  $$
    {\cal E}=\oint_{\Gamma}\bigg({{{\bf F}}\over{q}}\bigg)\cdot d{\bf l}
      =\oint_{\Gamma}[{\bf E}+({\bf v}\times{\bf B})\big]\cdot d{\bf l},
  $$}
\item{$\bullet$}{Faradays induktionslag (``{\it the flux law}'') relaterar
  den genererade elektromotoriska ``kraften'' i en slinga $\Gamma$ till en
  f{\"o}r{\"a}ndring av det magnetiska fl{\"o}det med omv{\"a}nt tecken som
  $$
    {\cal E}=-{{d\Phi_{\rm M}}\over{dt}}.
  $$
  Faradays induktionslag h{\"a}rleds endast utifr{\aa}n Lorentz-kraften
  (agerande p{\aa} en hypotetisk ladd\-ning), och {\"a}r d{\"a}rmed i
  avsaknad av s{\aa}v{\"a}l den elektriska permittiviteten $\varepsilon_0$
  som den magnetiska permeabiliteten $\mu_0$ d{\aa} varken Coulombs eller
  Biot--Savarts lag anv{\"a}nts.}
\item{$\bullet$}{Lenz lag s{\"a}ger att {\it en inducerad str{\"o}m har en
  riktning som motverkar orsaken till att den uppkom.} Detta inneb{\"a}r att
  om magnetf{\"a}ltet genom en ledande slinga {\it {\"o}kar}, s{\aa} kommer
  den i slingan inducerade str{\"o}mmen att ha en riktning som skapar ett
  magnetf{\"a}lt som {\it motverkar} {\"o}kningen.}
\item{$\bullet$}{F{\"o}r en spole med $N$ varv ($N$ slingor) blir den
  resulterande elektromotoriska ``kraften'' utv{\"a}xlad i motsvarande grad som
  $$
    {\cal E}=-N{{d\Phi_{\rm M}}\over{dt}}.
  $$}
\item{$\bullet$}{Faradays lag p{\aa} differentialform och integralform lyder
  $$
    \nabla\times{\bf E}({\bf r},t)
      =-{{\partial{\bf B}({\bf x},t)}\over{\partial t}}
    \qquad\Leftrightarrow\qquad
    \oint_{\Gamma}{\bf E}({\bf r},t)\cdot d{\bf l}
      =-{{d}\over{dt}}\iint_S{\bf B}({\bf x},t)\cdot d{\bf S}.
  $$}
\item{$\bullet$}{{\"O}msesidig induktans beskrivs av det magnetiska fl{\"o}de
  $\Phi_{\rm M}$ som genereras i en sekund{\"a}rslinga $\Gamma$ fr{\aa}n en
  str{\"o}m $I'(t)$ som drivs genom en prim{\"a}rslinga $\Gamma'$, som
  $$
    \Phi_{\rm M}=M_{\Gamma\Gamma'} I'(t),
  $$
  d{\"a}r den {\it {\"o}msesidiga induktansen} enligt Neumanns formel ges som
  $$
    M_{\Gamma\Gamma'}={{\mu_0}\over{4\pi}}\oint_{\Gamma}\oint_{\Gamma'}
          {{d{\bf l}'\cdot d{\bf l}}\over{|{\bf x}-{\bf x}'|}},
  $$
  vilket {\"a}r en rent geometrisk konstruktion, skalad med den magnetiska
  permeabiliteten $\mu_0/4\pi$.}
\item{$\bullet$}{I den {\"o}msesidiga induktansen kan vi godtyckligt v{\"a}lja
  vad vi v{\"a}ljer att beteckna som prim{\"a}r- eller sekund{\"a}rslinga,
  d{\aa}
  $$
    M_{\Gamma\Gamma'}=M_{\Gamma'\Gamma},
  $$
  vilket i sin tur s{\"a}ger oss att det magnetiska fl{\"o}de $\Phi_{\rm M}$
  som vi erh{\aa}ller i sekund{\"a}rslingan $\Gamma$ d{\aa} vi driver
  prim{\"a}rslingan med en str{\"o}m $I'(t)$ {\"a}r exakt lika stort
  som det maknetiska fl{\"o}de $\Phi'_{\rm M}$ som vi skulle detektera
  i prim{\"a}rslingan $\Gamma'$ om vi ist{\"a}llet drev sekund{\"a}rslingan
  $\Gamma$ med samma str{\"o}m $I'(t)$.}

\bye
