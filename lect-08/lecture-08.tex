%
% File: teach/elmagii/lect-08/lecture-08.tex [plain TeX code]
% Github: https://github.com/elmagii/lect-08/
% Last change: November 19, 2025
%
% Lecture No 8 in the course ``Elektromagnetism II, 1TE626 (2023)'',
% held November 24, 2025, at Uppsala University, Sweden.
%
% Copyright (C) 2022-2025, Fredrik Jonsson, under Gnu General Public
% License (GPL) v3. See the enclosed LICENSE for details.
%
% This program is free software: you can redistribute it and/or modify
% it under the terms of the GNU General Public License as published by
% the Free Software Foundation, either version 3 of the License, or
% (at your option) any later version.
%
% This program is distributed in the hope that it will be useful,
% but WITHOUT ANY WARRANTY; without even the implied warranty of
% MERCHANTABILITY or FITNESS FOR A PARTICULAR PURPOSE.  See the
% GNU General Public License for more details.
%
% You should have received a copy of the GNU General Public License
% along with this program.  If not, see <https://www.gnu.org/licenses/>.
%
\input macros/epsf.tex
\input macros/eplain.tex
\font\ninerm=cmr9
\font\tenssbx=cmssbx10
\font\twelvesc=cmcsc10 at 12 truept
\input amssym % to get the {\Bbb E} font (strikethrough E)
\def\captionwide{\advance\leftskip by 60pt
  \advance\rightskip by 60pt}
\def\lecture #1 {\hsize=150mm\hoffset=4.6mm\vsize=230mm\voffset=7mm
  \topskip=0pt\baselineskip=12pt\parskip=0pt\leftskip=0pt\parindent=15pt
  \headline={\ifnum\pageno>1\ifodd\pageno\rightheadline\else\leftheadline\fi
    \else\hfill\fi}
  \def\rightheadline{\tenrm{\it F\"orel\"asning #1}
    \hfil{\it Elektromagnetism II, 1TE626 (2025)}}
  \def\leftheadline{\tenrm{\it Elektromagnetism II, 1TE626 (2025)}
    \hfil{\it F\"orel\"asning #1}}
  \noindent~\vskip-60pt\hskip-40pt{\epsfbox{macros/UU_logo_color.eps}}
  \vskip-42pt\hfill\vbox{
      \hbox{{\it Elektromagnetism II, 1TE626 (2025)}}
      \hbox{{\it Lecture Notes, Fredrik Jonsson}}
      \hbox{{\it Document Revision \today}}
      \hbox{{\it https://github.com/hp35/elmagii/}}}\vskip 36pt
    \centerline{\twelvesc F\"orel\"asning #1}
  \vskip 24pt\noindent}
\def\section #1 {\medskip\goodbreak\noindent{\tenssbx #1}
  \par\nobreak\smallskip\noindent}
\def\subsection #1 {\medskip\goodbreak\noindent{\it #1}
  \par\nobreak\smallskip\noindent}
\def\iint{\mathop{\int\kern-7pt\int}}
\def\iiint{\mathop{\int\kern-7pt\int\kern-7pt\int}}
\def\Re{\mathop{\rm Re}\nolimits} % real part
\def\Im{\mathop{\rm Im}\nolimits} % imaginary part
\def\Tr{\mathop{\rm Tr}\nolimits} % quantum mechanical trace
\def\boxit#1{\vbox{\hrule\hbox{\vrule\kern3pt
  \vbox{\kern3pt#1\kern3pt}\kern3pt\vrule}\hrule}}
\def\quote#1{\leftskip=36pt\rightskip=36pt\smallskip\noindent#1\par
  \leftskip=0pt\rightskip=0pt\smallskip}
\def\plan#1{\par\leftskip=36pt\rightskip=36pt\bigskip%
  \noindent{\it Sammanfattning av f{\"o}rel{\"a}sningen}\smallskip
  \noindent{\it #1}\par\leftskip=0pt\rightskip=0pt} %\vfill\eject}
\def\threepointsummary#1#2#3{\par\leftskip=36pt\rightskip=36pt\bigskip
  \noindent{\it Sammanfattning i tre punkter}\smallskip
  \leftskip=48pt\rightskip=36pt\hangindent=20pt
  \noindent{\it\hbox to 20pt{1. }#1}\smallskip
  \leftskip=48pt\rightskip=36pt\hangindent=20pt
  \noindent{\it\hbox to 20pt{2. }#2}\smallskip
  \leftskip=48pt\rightskip=36pt\hangindent=20pt
  \noindent{\it\hbox to 20pt{3. }#3}\par%\medskip
  \leftskip=0pt\rightskip=0pt\vfill\eject}
\def\epsfig#1{\bigskip\centerline{\epsfbox{#1}}\medskip\noindent}
\newdimen\itemindent \itemindent=28pt
\newdimen\hangitemindent \hangitemindent=46pt
\def\litem[#1]{\smallbreak\noindent%
  \hbox to\itemindent{\hfil}\hbox to\itemindent{#1\hfill}%
  \hangindent\hangitemindent\ignorespaces}

\lecture{8}
\centerline{\twelvesc Multipolutvecklingen}
\centerline{Fredrik Jonsson, Uppsala Universitet, 24 november 2025}
\vskip24pt

\plan{Vi l{\"a}mnar f{\"o}r denna f{\"o}rel{\"a}sning dipolmodellen f{\"o}r ett
  tag, och visar p{\aa} att andra konstruktioner av laddningsf{\"o}rdelningar,
  exempelvis kvadrupoler och oktopoler, ger bidrag till f{\"a}lt som avklingar
  med en annan takt en det klassiska ``$1/r^2$''-upptr{\"a}dandet.
  Vi utg{\aa}r fr{\aa}n de klassiska integralerna f{\"o}r den skal{\"a}ra
  potentialen $\phi$ och vektorpotentialen ${\bf A}$ och {\"a}gnar oss {\aa}t
  en statisk modell f{\"o}r multipolutvecklingen.
  Vi g{\aa}r igenom begreppen dipol, kvadrupol, oktopol och h{\"o}gre
  ordningar; specifikt visar vi p{\aa} att {\"a}ven det enklast t{\"a}nkbara
  fallet av en elektrisk dipol har h{\"o}gre ordningars multipolmoment
  n{\"a}rvarande, men att dessa oftast f{\"o}rsummas vid betraktelse p{\aa}
  stora avst{\aa}nd.
  F{\"o}rel{\"a}sningen avslutas med att demonstrera multipolutvecklingen
  f{\"o}r generella laddningsf{\"o}rdelningar i termer av skal{\"a}r potential,
  samt fastst{\"a}llande av dipolapproximationen f{\"o}r elektrostatiska och
  magnetostatiska f{\"a}lt.
}

\threepointsummary{%
  Multipolutvecklingen {\"a}r i grund och botten en serieutveckling av
  potentialen (skal{\"a}r eller vektor) i inversa potenser av distansen
  mellan k{\"a}lla och observationspunkt.
}{%
  Multipolutvecklingen klassificeras term f{\"o}r term utifr{\aa}n hur
  beroendet av avst{\aa}ndet $r=|{\bf x}-{\bf x}'|$ mellan k{\"a}llpunkt
  ${\bf x}'$ och f{\"a}ltpunkt ${\bf x}$ ser ut:
  \litem[1.]{\hbox to110pt{Monopol (1-pol):\hfil}
    $\Phi({\bf x})\sim 1/r$}
  \litem[2.]{\hbox to110pt{Dipol (2-pol):\hfil}
    $\Phi({\bf x})\sim 1/r^2$}
  \litem[3.]{\hbox to110pt{Kvadrupol (4-pol):\hfil}
    $\Phi({\bf x})\sim 1/r^3$}
  \litem[4.]{\hbox to110pt{Oktopol (8-pol):\hfil}
    $\Phi({\bf x})\sim 1/r^4$}
  \litem[5.]{\hbox to110pt{Hexadecapol (16-pol):\hfil}
    $\Phi({\bf x})\sim 1/r^5$}
  \litem[6.]{\hbox to110pt{Dotriacontapol (32-pol):\hfil}
    $\Phi({\bf x})\sim 1/r^6$}
}{%
  Dipolapproximationen f{\"o}r elektrostatiska (${\bf E}=-\nabla\Phi$) och
  magnetostatiska (${\bf B}=\nabla\times{\bf A}$) f{\"a}lt p\aa\ l{\aa}ngt
  avst{\aa}nd $|{\bf x}|$ fr{\aa}n laddningsf{\"o}rdelningen (k{\"a}llan)
  $\rho$ lyder
  $$
    \Phi({\bf x})\approx{{1}\over{4\pi\varepsilon_0}}\bigg(
        {{q}\over{|{\bf x}|}} + {{{\bf p}\cdot{\bf x}}\over{|{\bf x}|^3}}
      \bigg),\qquad\qquad
    {\bf A}({\bf x})\approx{{\mu_0}\over{4\pi}}
        {{{\bf m}\times{\bf x}}\over{|{\bf x}|^3}},
  $$
  d\"ar $q$ {\"a}r nettoladdningen och ${\bf p}$ dipolmomentet hos en
  magnetisk dipol samt ${\bf m}$ det magnetiska dipolmomentet.
}

\section{Multipolutveckling av f\"alt}
\item{$\bullet$}{Formalism f{\"o}r att projicera ut olika bidrag (``moment'')
   till elektromagnetiska f{\"a}lt.} 
\item{$\bullet$}{Ofta f{\"o}rekommande f{\"o}r karakterisering av antenner.}
\item{$\bullet$}{Molekyler som interagerar med elekromagnetiska f{\"a}lt agerar
   som ``antenner'' vilka kan beskrivas utifr{\aa}n deras multipolutvecklingar,}
\item{$\bullet$}{Basen f\"or beskrivning av $\varepsilon_{\rm r}$ i konstitutiva
   relationen ${\bf P}=\varepsilon_0\varepsilon_{\rm r}{\bf E}$ (hur material
   polariseras av elektriska f\"alt) eller ${\bf B}=\mu_0\mu_{\rm r}{\bf H}$
   (hur material magnetiseras av magnetiska f{\"a}lt).}
\item{$\bullet$}{Oftast fokus p\aa\ elektrisk dipol-approximation, men vissa
   fenomen, som optisk aktivitet, kan bara beskrivas om h{\"o}gre ordningars
   termer tas med.}
\medskip
\noindent
Vi kommer h{\"a}r att i huvudsak anv{\"a}nda de generella uttrycken f{\"o}r
elektrisk skal{\"a}r potential och vektorpotentialen,
som\numberedfootnote{F{\"o}r formulering av skal{\"a}ra potentialen
  $\Phi({\bf x})$ samt vektorpotentialen ${\bf A}({\bf x})$ fr{\aa}n en
  generell laddningsdistribution, se Griffiths Ekv.~(10.26), s.~445.
  Notera h{\"a}r hur vektorpotentialen ${\bf A}({\bf x})$ har en form
  som {\it exakt matchar det uttryck som vi under f{\"o}rra
  f{\"o}rel{\"a}sningen tog fram f{\"o}r samma vektorpotential f{\"o}r
  ett magnetiserat objekt} med magnetiseringen ${\bf M}({\bf x})$, i
  termer av en {\it bunden volymstr{\"o}m}
  ${\bf J}_{\rm b}=\nabla\times{\bf M}$ (${\rm A}/{\rm m}^2$) och en
  {\it bunden ytstr{\"o}m} ${\bf K}_{\rm b}={\bf M}\times{\bf e}_n$
  (${\rm A}/{\rm m}$).
  Detta uttrycker p{\aa} ett mer generellt plan att vi med f{\"o}rra
  f{\"o}rel{\"a}sningens slutresultat {\it kan applicera den
  multipolutveckling som vi nu kommer att g{\aa} igenom p{\aa} ett
  godtyckligt magnetiserat objekt!}}
$$
  \Phi({\bf x})={{1}\over{4\pi\varepsilon_0}}
    \iiint_{{\Bbb R}^3}{{\rho({\bf x}')}\over{|{\bf x}-{\bf x}'|}}\,dV',
    \qquad\qquad
  {\bf A}({\bf x})={{\mu_0}\over{4\pi}}
    \iiint_{{\Bbb R}^3}{{{\bf J}({\bf x}')}\over{|{\bf x}-{\bf x}'|}}\,dV',
$$
d{\"a}r $\varepsilon_0= 8.854\times10^{-12}\ {\rm F}/{\rm m}$ {\"a}r den
{\it elektriska permittiviteten f{\"o}r vakuum} (electric permittivity of
vacuum), samt $\mu_0=1.257\times10^{-6}\ {\rm N}/{\rm A}^2$ den {\it magnetiska
permeabiliteten f{\"o}r vakuum} (vacuum magnetic permeability). I dessa uttryck
{\"a}r $\rho$ den eletriska laddningsdensiteten (${\rm C}/{\rm m}^3$) och
${\bf J}$ den eletriska str{\"o}mt{\"a}theten (${\rm A}/{\rm m}^2$).
Vi anv{\"a}nder h{\"a}r de generella, tredimensionella uttrycken f{\"o}r
potentialerna f{\"o}r att {\"o}va p{\aa} deras till{\"a}mpningen, samt f{\"o}r
att bygga upp en generell verktygsl{\aa}da f{\"o}r att probleml{\"o}sning inom
elektromagnetisk f{\"a}ltteori.

F{\"o}r att rekapitulera sj{\"a}lva vitsen med att anv{\"a}nda den skal{\"a}ra
potentialen och vektorpotentialen, s{\aa} kan vi ur dessa extrahera de
{\it statiska} (ej tidsberoende) elektriska och magnetiska f{\"a}lten
som\numberedfootnote{Recap p{\aa} ursprunget f{\"o}r vektorpotentialen:
  $\nabla\cdot{\bf B}=0\quad\Leftrightarrow
    \quad\exists{\bf A}: {\bf B}=\nabla\times{\bf A}$.}
$$
  {\bf E}=-\nabla\Phi-\underbrace{{{\partial{\bf A}}\over{\partial t}},}_{=0}
  \qquad
  {\bf B}=\nabla\times{\bf A}.
$$
Vi kan ocks{\aa} konstatera att SI-enheterna f{\"o}r den skal{\"a}ra
potentialen och vektorpotentialen {\"a}r
$$
  \eqalign{
    [\Phi]&={{[\rho][dV']}\over{[\varepsilon_0][{\bf x}]}}
        ={{({\rm C}/{\rm m}^3){\rm m}^3}\over{({\rm F}/{\rm m}){\rm m}}}
	=\{\ {\rm F}={\rm C}/{\rm V}\ \}
	={\rm V},\cr
    [{\bf A}]&={{[\mu_0][{\bf J}][dV']}\over{[{\bf x}]}}
        ={{({\rm N}/{\rm A}^2)({\rm A}/{\rm m}^2){\rm m}^3}\over{{\rm m}}}
        ={\rm N}/{\rm A}.\cr
  }
$$
\vfill\eject

\section{Multipolutveckling f\"or skal{\"a}r potential f{\"o}r en
         elektrisk dipol}
Ett av de enklaste testobjekten inom elektromagnetism {\"a}r den elektriska
dipolen,\numberedfootnote{The English prefixes {\it bi-}, derived from Latin,
  and its Greek variant {\it di-} both mean ``two''. The Latin prefix is far
  more prevalent in common words, such as {\it bilingual}, {\it biceps}, and
  {\it biped}; the more technical Greek {\it di-} appears in such words as
  {\it diphthong} and {\it dilemma}.} best{\aa}ende av en positiv och negativ
laddning separerade ett avst{\aa}nd $L$.
\"Aven f\"or en elektrisk {\it di}pol finns det (paradoxalt) termer av
{\it multi}polmoment d\aa\ vi l\"amnar approximationen att den skal\"ara
elektriska potentialen enbart ges av skal\"arprodukten mellan dipolmomentet
och ortsvektorn till observationspunkten.

Vi betraktar en klassisk elektrisk dipol med tv\aa\ laddningar $+q$ or $-q$
separerade avst{\aa}ndet $L=2b$. Dipolen kan beskrivas med en
laddningsf{\"o}rdelning enligt {\it distributionen}
$$
  \rho({\bf x})=(+q)\delta({\bf x}-b{\bf e}_z)
     +(-q)\delta({\bf x}+b{\bf e}_z)
$$
\epsfig{figs/eldipole.1}\noindent
\vfill\eject
Den {\it elektriska skal\"ara potentialen} fr{\aa}n denna specifika
distribution av tv\aa\ punktk\"allor\numberedfootnote{Vi utg{\aa}r h{\"a}r
  ifr{\aa}n den generella tre-dimensionella volymintegralen av en
  laddningsdensitet $\rho({\bf x})$, bara f{\"o}r att illustrera hur vi
  kan till{\"a}mpa denna {\"a}ven f{\"o}r lokaliserade punktladdningar.}
blir d{\"a}rmed
$$
  \eqalign{
    \Phi({\bf x})&={{1}\over{4\pi\varepsilon_0}}
        \iiint_{V}{{\rho({\bf x}')}\over{|{\bf x}-{\bf x}'|}}\,dV'\cr
      &={{1}\over{4\pi\varepsilon_0}}\left(
         {{(+q)}\over{|{\bf x}-b{\bf e}_z|}}+{{(-q)}\over{|{\bf x}+b{\bf e}_z|}}
        \right)\cr
      &={{q}\over{4\pi\varepsilon_0}}\left(
          {{1}\over{r-b\cos\theta}}-{{1}\over{r+b\cos\theta}}
        \right)\cr
      &=\big\{\ \hbox{Definiera}\ \varepsilon \equiv (b/r)\cos\theta\ \big\}\cr
      &={{q}\over{4\pi\varepsilon_0 r}}\left(
          {{1}\over{1-\varepsilon}}-{{1}\over{1+\varepsilon}}
        \right)\cr
      &=\big\{\ \hbox{Taylor-utveckling f{\"o}r litet}\ \varepsilon\ \big\}\cr
      &={{q}\over{4\pi\varepsilon_0 r}}\left(
          (1+\varepsilon+\varepsilon^2+\varepsilon^3+\ldots)
            -(1-\varepsilon+\varepsilon^2-\varepsilon^3+\ldots)
        \right)\cr
      &={{q}\over{2\pi\varepsilon_0 r}}\left(
          \varepsilon+\varepsilon^3+\varepsilon^5+\ldots
        \right)\cr
%      &={{q}\over{2\pi\varepsilon_0 r}}\Bigg(
%         {{b}\over{r}}\cos\theta
%         +\Big({{b}\over{r}}\cos\theta\Big)^3
%         +\Big({{b}\over{r}}\cos\theta\Big)^5
%	 +\ldots
%        \Bigg)\cr
      &={{q}\over{2\pi\varepsilon_0}}\Bigg(
         \underbrace{{{b\cos\theta}\over{r^2}}}_{\rm dipol}
         +\underbrace{{{(b\cos\theta)^3}\over{r^4}}}_{\rm oktopol}
         +\underbrace{{{(b\cos\theta)^5}\over{r^6}}}_{\rm dotriacontapol}
	 +\ldots
        \Bigg)\cr
  }
$$
\vfill\eject
\centerline{\epsfxsize=144mm\epsfbox{multipoles/multipoles/lindipole.eps}}
\noindent
{\captionwide Skal{\"a}r potential $\Phi(x,y)$ f{\"o}r en elektrisk dipol.}
\medskip
\centerline{\epsfxsize=144mm\epsfbox{multipoles/multipoles/lindipole-str.eps}}
\noindent
{\captionwide Skal{\"a}r potential $\Phi(x,y)$ f{\"o}r en elektrisk dipol med
f{\"a}ltlinjer f{\"o}r ${\bf E}(x,y)=-\nabla\Phi(x,y)$.}
\vfill\eject

\section{Exempel p\aa\ multipoler}
Som en relativt enkel illustration av multipoler kan vi konstruera olika
multipolmoment utifr{\aa}n diskreta laddningar.
\epsfig{figs/quadrupole.1}
\epsfig{figs/octopole.1}
\noindent
En intressant {\"o}vning {\"a}r att g{\aa} igenom dessa specifika multipoler
och upprepa den geometriska analysen f{\"o}r dipolen, f{\"o}r att p{\aa} s{\aa}
s{\"a}tt extrahera styrkan p{\aa} de olika multipolmomenten.

\section{Multipolutveckling f\"or skal{\"a}r potential f\"or en
         linj{\"a}r elektrisk kvadrupol}
L{\aa}t oss g{\"o}ra en liten {\"a}ndring p{\aa} den elektriska dipolen i
f{\"o}reg{\aa}ende exempel, och ist{\"a}llet l{\"a}gga en punktladdning $-2q$
i centrum med tv{\aa} punktladdningar $+q$ p{\aa} diametralt motsatt sida om
denna. Denna konfiguration kan beskrivas med {\it distributionen}
$$
  \rho({\bf x})=(+q)\delta({\bf x}-b{\bf e}_z)
  +(-2q)\delta({\bf x})
  +(+q)\delta({\bf x}+b{\bf e}_z),
$$
precis analogt med det f{\"o}reg{\aa}ende fallet f{\"o}r den elektriska dipolen.
\epsfig{figs/linelquadpole.1}\noindent
P{\aa} exakt samma s{\"a}tt som tidigare, med den enda skillnaden att vi nu
har {\it tre} laddningar (k{\"a}ll\-termer) f{\"o}r den skal{\"a}ra potentialen
$\Phi({\bf x})$, s{\aa} har vi att
$$
  \eqalign{
    \Phi({\bf x})&={{1}\over{4\pi\varepsilon_0}}
        \iiint_{V}{{\rho({\bf x}')}\over{|{\bf x}-{\bf x}'|}}\,dV'\cr
      &={{1}\over{4\pi\varepsilon_0}}\left(
          {{(+q)}\over{|{\bf x}-b{\bf e}_z|}}
            +{{(-2q)}\over{|{\bf x}|}}
            +{{(+q)}\over{|{\bf x}+b{\bf e}_z|}}
        \right)\cr
      &={{q}\over{4\pi\varepsilon_0}}\left(
          {{1}\over{r-b\cos\theta}}
            -{{2}\over{r}}
            +{{1}\over{r+b\cos\theta}}
        \right)\cr
      &=\big\{\ \hbox{Definiera}\ \varepsilon \equiv (b/r)\cos\theta\ \big\}\cr
      &={{q}\over{4\pi\varepsilon_0 r}}\left(
          {{1}\over{1-\varepsilon}}
            -2
            +{{1}\over{1+\varepsilon}}
        \right)\cr
      &=\big\{\ \hbox{Taylor-utveckling f{\"o}r litet}\ \varepsilon\ \big\}\cr
      &={{q}\over{4\pi\varepsilon_0 r}}\left(
          (1+\varepsilon+\varepsilon^2+\varepsilon^3+\ldots)
            -2
            +(1-\varepsilon+\varepsilon^2-\varepsilon^3+\ldots)
        \right)\cr
      &={{q}\over{2\pi\varepsilon_0 r}}\left(
          \varepsilon^2+\varepsilon^4+\varepsilon^6+\ldots
        \right)\cr
%      &={{q}\over{2\pi\varepsilon_0 r}}\Bigg(
%         {{b}\over{r}}\cos\theta
%         +\Big({{b}\over{r}}\cos\theta\Big)^3
%         +\Big({{b}\over{r}}\cos\theta\Big)^5
%	 +\ldots
%        \Bigg)\cr
      &={{q}\over{2\pi\varepsilon_0}}\Bigg(
         \underbrace{{{(b\cos\theta)^2}\over{r^3}}}_{\rm kvadrupol}
         +\underbrace{{{(b\cos\theta)^4}\over{r^5}}}_{\rm hexadecapol}
         +\underbrace{{{(b\cos\theta)^6}\over{r^7}}}_{\rm hexacontatetrapol}
         +\ldots
        \Bigg)\cr
  }
$$
I denna serieutveckling {\aa}terfinns nu endast {\it udda} potenser av
avst{\aa}ndet $r$, varav den dominerande termen p{\aa} stora avst{\aa}nd
kommer att vara den som g{\aa}r som $O(1/r^3)$, svarande mot en elektrisk
{\it kvadrupol}. V{\"a}rt att notera {\"a}r att i denna serieutveckling
{\aa}terfinns {\it ingen dipolterm}.

\section{Klassificering av multipolmoment}
Multipolutvecklingen klassificeras term f{\"o}r term av hur beroendet av
avst{\aa}ndet $r=|{\bf x}-{\bf x}'|$ mellan k{\"a}llpunkt ${\bf x}'$ och
f{\"a}ltpunkt ${\bf x}$ ser ut:
\medskip
\litem[1.]{\hbox to140pt{Monopol (1-pol):\hfil} $\Phi({\bf x})\sim 1/r$}
\litem[2.]{\hbox to140pt{Dipol (2-pol):\hfil} $\Phi({\bf x})\sim 1/r^2$}
\litem[3.]{\hbox to140pt{Kvadrupol (4-pol):\hfil} $\Phi({\bf x})\sim 1/r^3$}
\litem[4.]{\hbox to140pt{Oktopol (8-pol):\hfil} $\Phi({\bf x})\sim 1/r^4$}
\litem[5.]{\hbox to140pt{Hexadecapol (16-pol):\hfil} $\Phi({\bf x})\sim 1/r^5$}
\litem[6.]{\hbox to140pt{Dotriacontapol (32-pol):\hfil}
  $\Phi({\bf x})\sim 1/r^6$}
\litem[7.]{\hbox to140pt{Hexacontatetrapol (64-pol):\hfil}
  $\Phi({\bf x})\sim 1/r^7$}
\litem[8.]{\hbox to110pt{$\ldots$\hfil} $\ldots$}
\medskip
\noindent
Sammanfattningsvis s\aa\ har en klassisk elektrisk dipol samtliga
multipolmoment som \"ar {\it j\"amna}, det vill s\"aga {\it dipol}
($\Phi\sim 1/r^2$), {\it oktopol} ($\Phi\sim 1/r^4$), {\it dotriacontapole}
($\Phi\sim 1/r^6$), etc.
\vfill\eject
\section{Skal{\"a}r potential och elektriskt f{\"a}lt f{\"o}r en linj{\"a}r
         elektrisk kvadrupol}
\centerline{\epsfxsize=142mm
  \epsfbox{multipoles/multipoles/linquadrupole.eps}}
\noindent
{\captionwide Skal{\"a}r potential $\Phi(x,y)$ f{\"o}r en linj{\"a}r elektrisk
kvadrupol.}
\medskip
\centerline{\epsfxsize=142mm
  \epsfbox{multipoles/multipoles/linquadrupole-str.eps}}
\noindent
{\captionwide Skal{\"a}r potential $\Phi(x,y)$ f{\"o}r en elektrisk linj{\"a}r
kvadrupol med f{\"a}ltlinjer f{\"o}r ${\bf E}(x,y)=-\nabla\Phi(x,y)$.}
\vfill\eject

\section{Skal{\"a}r potential och elektriskt f{\"a}lt f{\"o}r en kvadratisk
         elektrisk kvadrupol}
\centerline{\epsfxsize=142mm
  \epsfbox{multipoles/multipoles/quadquadrupole.eps}}
\noindent
{\captionwide Skal{\"a}r potential $\Phi(x,y)$ f{\"o}r en kvadratisk elektrisk
kvadrupol.}
\medskip
\centerline{\epsfxsize=142mm
  \epsfbox{multipoles/multipoles/quadquadrupole-str.eps}}
\noindent
{\captionwide Skal{\"a}r potential $\Phi(x,y)$ f{\"o}r en kvadratisk
elektrisk kvadrupol med f{\"a}ltlinjer ${\bf E}(x,y)=-\nabla\Phi(x,y)$.}

\section{Multipolutveckling f\"or generella laddningsf{\"o}rdelningar
         (i termer av skal{\"a}r potential)}
Betrakta en generell laddningsf{\"o}rdelning enligt figur.
Laddningsf{\"ordelningen} kan vara en generell distribution i 3D, men {\"a}ven
i kombinationer av 2D (ytladdningar, sk{\"a}rmar, jordplan),
1D (linjeladdningar, antenner) eller 0D (punktladdningar). Vi kommer i det
f{\"o}ljande att anv{\"a}nda en geometri i vilken laddningst{\"a}theten
$\rho({\bf x})$ {\"a}r lokaliserad i n{\"a}rheten av origo (vilket g{\"o}r
det enklare att r{\"a}tt av till{\"a}mpa en serieutveckling f{\"o}r
laddningsdensiteten i en Taylor-serie), enligt figur.
\epsfig{figs/chargedist.1}\noindent
Skal{\"a}r (elektrisk) potential $\Phi({\bf x})$ fr{\aa}n punktk{\"a}lla $q$ i
${\bf x}'$:
$$
  \Phi({\bf x})={{q}\over{4\pi\varepsilon_0|{\bf x}-{\bf x}'|}}
$$
\noindent
Skal{\"a}r (elektrisk) potential fr{\aa}n distributionen $\rho({\bf x})$ ges
analogt genom att helt enkelt summera upp alla bidrag fr{\aa}n samtliga
$\rho({\bf x}')dV'$ i volymen $V$:
$$
  \Phi({\bf x})={{1}\over{4\pi\varepsilon_0}}
     \iiint_{V}{{\rho({\bf x}')}\over{|{\bf x}-{\bf x}'|}}\,dV'
$$
I denna integral s{\aa} kan vi se det som att vi summerar upp alla element
$dV$'s infinitesimala laddningar $dq=\rho({\bf x}')dV'$ (eller
str{\"o}mt{\"a}theter ${\bf J}({\bf x}')dV$ i fallet med vektorpotentialen
${\bf A}$), viktade med en faktor
$$
  f({\bf x}')={{1}\over{|{\bf x}-{\bf x}'|}}
$$
som kort och gott {\"a}r det inversa geometriska avst{\aa}ndet mellan
k{\"a}llpunkten ${\bf x}'$ och f{\"a}ltpunkten (observationspunkten) ${\bf x}$.
Vi kan, om vi s{\aa} vill, se detta som en summation av potentialbidragen
fr{\aa}n alla infinitesimala elektriska {\it monopoler} som ryms i volymen
$V$.

Liksom i det enkla fallet med dipolen, siktar vi h{\"a}r mot en serieutveckling
av denna viktfunktion f{\"o}r att enklare kunna tolka de termer som blir
resultatet. F{\"o}r att rekapitulera, s{\aa} var serieutvecklingen i fallet
med dipolen i grund och botten en serieutveckling av position f{\"o}r
k{\"a}lltermerna, med ``litet separations-avst{\aa}nd $L$ i f{\"o}rh{\aa}llande
till avst{\aa}ndet till observationspunkten ${\bf x}$''. Vi {\"o}nskar
d{\"a}rmed att uttrycka serieutvecklingen i koordinaten ${\bf x}'$ f{\"o}r
{\it k{\"a}llan} $\rho({\bf x}')$. Generellt har vi att en Taylor-utveckling i
tre dimensioner ges av
$$
  f({\bf x}') = f({\bf 0})
  +\sum^{3}_{k=1} x'_k
     {{\partial f({\bf x}')}
       \over{\partial x'_k}}\bigg|_{{\bf x}'={\bf 0}}
  +{{1}\over{2}}\sum^{3}_{j=1}\sum^{3}_{k=1} x'_j x'_k
     {{\partial^2 f({\bf x}')}
       \over{\partial x'_j\partial x'_k}}\bigg|_{{\bf x}'={\bf 0}}
  +\ldots
$$
F{\"o}r ``viktfunktionen''
$f({\bf x}')$ har vi att
$$
  \eqalign{
    f({\bf x}')&={{1}\over{|{\bf x}-{\bf x}'|}}
      ={{1}\over{\sqrt{(x-x')^2+(y-y')^2+(z-z')^2}}},\cr
    {{\partial f({\bf x}')}\over{\partial x'_k}}
      &={{\partial}\over{\partial x'_k}}
      {{1}\over{\sqrt{(x-x')^2+(y-y')^2+(z-z')^2}}}\cr
      &=-{{1}\over{2}}
      {{-2(x_k-x'_k)}\over{({(x-x')^2+(y-y')^2+(z-z')^2})^{3/2}}}
      =\ldots
      ={{x_k-x'_k}\over{|{\bf x}-{\bf x}'|^3}},\cr
    {{\partial^2 f({\bf x}')}\over{\partial x'_j \partial x'_k}}
      &={{\partial^2}\over{\partial x'_j \partial x'_k}}
          {{1}\over{\sqrt{(x-x')^2+(y-y')^2+(z-z')^2}}}\cr
      &={{\partial}\over{\partial x'_j}}
          {{x_k-x'_k}\over{((x-x')^2+(y-y')^2+(z-z')^2)^{3/2}}}\cr
      &={{{{\partial(x_k-x'_k)}\over{\partial x'_j}}
          ((x-x')^2+(y-y')^2+(z-z')^2)^{3/2}-
            (x_k-x'_k){{\partial ((x-x')^2+(y-y')^2+(z-z')^2)^{3/2}}
	      \over{\partial x'_j}}}\over{((x-x')^2+(y-y')^2+(z-z')^2)^{3}}}\cr
      &=\ldots\cr
      &={{3(x_j-x'_j)(x_k-x'_k)
      -\delta_{jk}((x-x')^2+(y-y')^2+(z-z')^2)}
             \over{((x-x')^2+(y-y')^2+(z-z')^2)^{5/2}}}\cr
      &={{3(x_j-x'_j)(x_k-x'_k)-\delta_{jk}|{\bf x}-{\bf x}'|^2}
          \over{|{\bf x}-{\bf x}'|^5}}\cr
  }
$$
F{\"o}r koefficienterna i Taylor-utvecklingen inneb{\"a}r detta specifikt att
$$
    f({\bf 0})
      ={{1}\over{|{\bf x}|}},\qquad\qquad
    {{\partial f({\bf x}')}
      \over{\partial x'_k}}\bigg|_{{\bf x}'={\bf 0}}
        ={{x_k}\over{|{\bf x}|^3}},\qquad\qquad
    {{\partial^2 f({\bf x}')}
      \over{\partial x'_j \partial x'_k}}\bigg|_{{\bf x}'={\bf 0}}
        ={{3 x_j x_k - \delta_{jk}|{\bf x}|^2}\over{|{\bf x}|^5}},
$$
och v{\aa}r Taylor-utveckling av viktsfunktionen bist{\aa}r direkt med
respektive multipol-termer i uttrycket f{\"o}r skal{\"a}ra elektriska
potentialen som
$$
  \eqalign{
    \Phi({\bf x})
      &={{1}\over{4\pi\varepsilon_0}}
        \iiint_{V} 
        \bigg(
          {{1}\over{|{\bf x}|}}
            +\sum^{3}_{k=1} {{x_k}\over{|{\bf x}|^3}} x'_k
            +{{1}\over{2}}\sum^{3}_{j=1}\sum^{3}_{k=1}
              {{3 x_j x_k - \delta_{jk}|{\bf x}|^2}\over{|{\bf x}|^5}} x'_j x'_k
            +\ldots
        \bigg) \rho({\bf x}')\,dV'\cr
      &={{1}\over{4\pi\varepsilon_0}}
        \Bigg(
        {{1}\over{|{\bf x}|}}
	\underbrace{
	  \iiint_{V} \rho({\bf x}')\,dV'
	}_{{\rm monopol},\ q}
      +\sum^{3}_{k=1} {{x_k}\over{|{\bf x}|^3}}
	\underbrace{
	  \iiint_{V} x'_k \rho({\bf x}')\,dV'
	}_{{\rm dipolmoment},\ p_k}
	\cr&\hskip150pt
      +{{1}\over{2}}\sum^{3}_{j=1}\sum^{3}_{k=1}
          {{3 x_j x_k - \delta_{jk}|{\bf x}|^2}\over{|{\bf x}|^5}}
	\underbrace{
          \iiint_{V} x'_j x'_k \rho({\bf x}')\,dV'
	}_{{\rm kvadrupolmoment},\ Q_{jk}}
      +\ldots
        \Bigg)\cr
      &={{1}\over{4\pi\varepsilon_0}}
        \Bigg(
        {{q}\over{|{\bf x}|}}
      +\sum^{3}_{k=1} {{x_k p_k}\over{|{\bf x}|^3}}
      +{{1}\over{2}}\sum^{3}_{j=1}\sum^{3}_{k=1}
          {{(3 x_j x_k - \delta_{jk}|{\bf x}|^2)Q_{jk}}\over{|{\bf x}|^5}}
      +\ldots
        \Bigg)
  }
$$
Notera att f{\"o}r dipoltermen utg{\"o}rs skal{\"a}ra potentialen av en
{\it skal{\"a}rprodukt} ${\bf x}\cdot{\bf p}$, medan kvadrupoltermen svarar mot
en {\it matrisprodukt} fr{\aa}n vilken sp{\aa}ret (``trace'') subtraheras,
$3{\bf x}^{\rm T}{\Bbb Q}{\bf x}-|{\bf x}|^2\Tr[{\Bbb Q}]$.

Analysen f{\"o}r vektorpotentialen ${\bf A}({\bf x})$ f{\"o}ljer p{\aa} samma
s{\"a}tt, med en helt och h{\aa}llet analog serieutveckling av
``viktfunktionen'' $f({\bf x}')$ f{\"o}r den elektriska str{\"o}mdensiteten
${\bf J}({\bf x})$.

\section{Dipolapproximationen f{\"o}r station{\"a}ra laddningsf{\"o}rdelningar
   och str{\"o}mmar}
F\"or att sammanfatta ges den station\"ara skal\"ara potentialen
$\Phi({\bf x})$ och vektorpotentialen ${\bf A}({\bf x})$ fr{\aa}n en
laddnings- och str\"omt\"athet i n{\"a}rheten av origo, i dipolapproximation
och p\aa\ l{\aa}ngt avst{\aa}nd fr{\aa}n k{\"a}llan, som
$$
  \Phi({\bf x})\approx{{1}\over{4\pi\varepsilon_0}}\bigg(
      {{q}\over{|{\bf x}|}} + {{{\bf p}\cdot{\bf x}}\over{|{\bf x}|^3}}
    \bigg),\qquad\qquad
  {\bf A}({\bf x})\approx{{\mu_0}\over{4\pi}}
      {{{\bf m}\times{\bf x}}\over{|{\bf x}|^3}},
$$
d\"ar
$$
  q=\underbrace{
      \iiint_{V} \rho({\bf x}')\,dV'
    }_{{\rm nettoladdning}\ [{\rm C}]},\qquad
  {\bf p}=\underbrace{
      \iiint_{V} x'_k \rho({\bf x}')\,dV'
    }_{{\rm elektriskt\ dipolmoment}\ [{\rm C}{\rm m}]},\qquad
  {\bf m}=\underbrace{
      {{1}\over{2}}\iiint_{V} {\bf x}'\times{\bf J}({\bf x}')\,dV'
    }_{{\rm magnetiskt\ dipolmoment}\ [{\rm A}{\rm m}^2]}.
$$
Fr{\aa}n dessa ges de {\it statiska} elektriska och magnetiska f\"alten
som\numberedfootnote{Notera att Ekv.~(8.1.5) i Olov {\AA}grens
  {\it Elektromagnetism} (Studentlitteratur, 2014) definierar magnetiska
  f\"altet utifr{\aa}n en konstruerad {\it magnetisk skal\"ar potential}
  ist\"allet f\"or vektorpotentialen. Vektorpotentialen \"ar h\"ar dock mest
  naturlig att anv\"anda, utifr{\aa}n den grundl\"aggande egenskapen
  $\nabla\cdot{\bf B}=0\Leftrightarrow{\bf B}=\nabla\times{\bf A}$ hos
  magnetiska f\"altet, med ${\bf A}$ som vektorpotentialen.}
$$
  {\bf E}({\bf x})=-\nabla\Phi({\bf x}),\qquad\qquad
  {\bf B}({\bf x})=\nabla\times{\bf A}({\bf x}).
$$
Man kan h{\"a}r fr{\aa}ga sig varf{\"o}r multipolutvecklingen f{\"o}r
vektorpotentialen ${\bf A}$ b{\"o}rjar med dipolmomentet, som g{\aa}r som
$\sim 1/|{\bf x}|^3$, och inte som den skal{\"a}ra potentialen inneh{\aa}ller
n{\aa}gon term som g{\aa}r som $\sim 1/|{\bf x}|$? Svaret p{\aa} denna
fr{\aa}ga {\"a}r sj{\"a}lvfallet att vektorpotentialen, som {\"a}r direkt
l{\"a}nkad till magnetf{\"a}ltet genom ${\bf B}=\nabla\times{\bf A}$, till
skillnad fr{\aa}n det elektrostatiska fallet {\it aldrig kan involvera
magnetiska monopoler}, varf{\"o}r monopoltermer ocks{\aa} saknas f{\"o}r
just vektorpotentialen.
\vfill\eject

\section{Sammanfattning av F{\"o}rel{\"a}sning~8 -- Multipolutvecklingen}
\item{$\bullet$}{Multipolutvecklingen {\"a}r i grund och botten bara en
  serieutveckling av potentialen (skal{\"a}r eller vektor) i inversa
  potenser av distansen mellan k{\"a}lla och observationspunkt.}
\item{$\bullet$}{
  Vi tar som vanligt fram uttrycken f{\"o}r potentialerna som
  $$
    \Phi({\bf x})={{1}\over{4\pi\varepsilon_0}}
      \iiint_{{\Bbb R}^3}{{\rho({\bf x}')}\over{|{\bf x}-{\bf x}'|}}\,dV',
      \qquad\qquad
    {\bf A}({\bf x})={{\mu_0}\over{4\pi}}
      \iiint_{{\Bbb R}^3}{{{\bf J}({\bf x}')}\over{|{\bf x}-{\bf x}'|}}\,dV'.
  $$}
\item{$\bullet$}{Multipolutvecklingen klassificeras term f{\"o}r term
  utifr{\aa}n hur beroendet av avst{\aa}ndet $r=|{\bf x}-{\bf x}'|$ mellan
  k{\"a}llpunkt ${\bf x}'$ och f{\"a}ltpunkt ${\bf x}$ ser ut:
  \litem[1.]{\hbox to110pt{Monopol (1-pol):\hfil}
    $\Phi({\bf x})\sim 1/r$}
  \litem[2.]{\hbox to110pt{Dipol (2-pol):\hfil}
    $\Phi({\bf x})\sim 1/r^2$}
  \litem[3.]{\hbox to110pt{Kvadrupol (4-pol):\hfil}
    $\Phi({\bf x})\sim 1/r^3$}
  \litem[4.]{\hbox to110pt{Oktopol (8-pol):\hfil}
    $\Phi({\bf x})\sim 1/r^4$}
  \litem[5.]{\hbox to110pt{Hexadecapol (16-pol):\hfil}
    $\Phi({\bf x})\sim 1/r^5$}
  \litem[6.]{\hbox to110pt{Dotriacontapol (32-pol):\hfil}
    $\Phi({\bf x})\sim 1/r^6$}}
\item{$\bullet$}{Multipolutvecklingen f\"or den skal{\"a}ra potentialen f\"or
  en linj{\"a}r elektrisk kvadrupol ges som
  $$
    \Phi({\bf x})
       ={{q}\over{2\pi\varepsilon_0}}\Bigg(
         \underbrace{{{b\cos\theta}\over{r^2}}}_{\rm dipol}
         +\underbrace{{{(b\cos\theta)^3}\over{r^4}}}_{\rm oktopol}
         +\underbrace{{{(b\cos\theta)^5}\over{r^6}}}_{\rm dotriacontapol}
         +\ldots
        \Bigg).
  $$}
\item{$\bullet$}{Multipolutvecklingen f\"or den skal{\"a}ra potentialen f\"or
  en linj{\"a}r elektrisk kvadrupol ges som
  $$
    \Phi({\bf x})
      ={{q}\over{2\pi\varepsilon_0}}\Bigg(
         \underbrace{{{(b\cos\theta)^2}\over{r^3}}}_{\rm kvadrupol}
         +\underbrace{{{(b\cos\theta)^4}\over{r^5}}}_{\rm hexadecapol}
         +\underbrace{{{(b\cos\theta)^6}\over{r^7}}}_{\rm hexacontatetrapol}
         +\ldots
        \Bigg).
  $$}
\item{$\bullet$}{Dipolapproximationen f{\"o}r elektrostatiska
  (${\bf E}=-\nabla\Phi$) och magnetostatiska (${\bf B}=\nabla\times{\bf A}$)
  f{\"a}lt p\aa\ l{\aa}ngt avst{\aa}nd $|{\bf x}|$ fr{\aa}n
  laddningsf{\"o}rdelningen (k{\"a}llan) $\rho$ lyder
  $$
    \Phi({\bf x})\approx{{1}\over{4\pi\varepsilon_0}}\bigg(
        {{q}\over{|{\bf x}|}} + {{{\bf p}\cdot{\bf x}}\over{|{\bf x}|^3}}
      \bigg),\qquad\qquad
    {\bf A}({\bf x})\approx{{\mu_0}\over{4\pi}}
        {{{\bf m}\times{\bf x}}\over{|{\bf x}|^3}},
  $$
  d\"ar
  $$
    q=\underbrace{
        \iiint_{V} \rho({\bf x}')\,dV'
      }_{{\rm nettoladdning}\ [{\rm C}]},\qquad
    {\bf p}=\underbrace{
        \iiint_{V} x'_k \rho({\bf x}')\,dV'
      }_{{\rm elektriskt\ dipolmoment}\ [{\rm C}{\rm m}]},\qquad
    {\bf m}=\underbrace{
        {{1}\over{2}}\iiint_{V} {\bf x}'\times{\bf J}({\bf x}')\,dV'
      }_{{\rm magnetiskt\ dipolmoment}\ [{\rm A}{\rm m}^2]}.
  $$}

\bye
