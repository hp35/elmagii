%
% File: teach/elmagii/compiled/src/main.tex [plain TeX code]
% Github: https://github.com/hp35/elmagii/tree/main/compiled/src
% Last change: February 22, 2026
%
% Compilation of all Lecture Notes in the course "Elektromagnetism II, 1TE626",
% held 2022-2026 at Uppsala University, Sweden.
%
% Copyright (C) 2022-2026, Fredrik Jonsson, under Gnu General Public
% License (GPL) v3. See the enclosed LICENSE for details.
%
% This program is free software: you can redistribute it and/or modify
% it under the terms of the GNU General Public License as published by
% the Free Software Foundation, either version 3 of the License, or
% (at your option) any later version.
%
% This program is distributed in the hope that it will be useful,
% but WITHOUT ANY WARRANTY; without even the implied warranty of
% MERCHANTABILITY or FITNESS FOR A PARTICULAR PURPOSE.  See the
% GNU General Public License for more details.
%
% You should have received a copy of the GNU General Public License
% along with this program.  If not, see <https://www.gnu.org/licenses/>.
%
\input ../lect-01/macros/epsf.tex
\input ../lect-01/macros/eplain.tex
\input amssym % to get the {\Bbb E} font (strikethrough E)
\input color  % to get colored text output
%
% Various font definitions
%
\font\ninerm=cmr9
\font\tenssbx=cmssbx10
\font\twelvesc=cmcsc10 at 12 truept
\font\fivecmr=cmr7 at 5 truept
\font\sevencmr=cmr7
\font\eightcmr=cmr8
\font\eightcmssq=cmssq8
\font\eightcmssqi=cmssqi8
\font\ninecmti=cmti9
\font\tencmti=cmti10
\font\tencmss=cmss10
\font\tencmcsc=cmcsc10
\font\tencmssi=cmssi10
\font\twelvecmbx=cmbx12
\font\sixteencmbx=cmbx12 at 16 truept
\font\sixteencmsc=cmcsc10 at 16 truept
\font\fourteencmssbx=cmssbx10 at 14 truept
\font\twentyeightcmssbx=cmssbx10 at 28 truept
\def\thinspace{\kern .16667em }

%
% Make sure that the METAPOST logo can be loaded even in plain TeX.
%
\ifx\MP\undefined
    \ifx\manfnt\undefined
            \font\manfnt=logo10
    \fi
    \ifx\manfntsl\undefined
            \font\manfntsl=logosl10
    \fi
    \def\MP{{\ifdim\fontdimen1\font>0pt \let\manfnt = \manfntsl \fi
      {\manfnt META}\-{\manfnt POST}}\spacefactor1000 }%
\fi
\ifx\METAPOST\undefined \let\METAPOST=\MP \fi
%
% File: teach/elmagii/compiled/src/misc.tex [plain TeX code]
% Github: https://github.com/hp35/elmagii/tree/main/compiled/src
% Last change: February 22, 2026
%
% Miscellaneous formatting macros for the compilation of all Lecture
% Notes in the course "Elektromagnetism II, 1TE626", held 2022-2026 at
% Uppsala University, Sweden.
%
% Copyright (C) 2022-2026, Fredrik Jonsson, under Gnu General Public
% License (GPL) v3. See the enclosed LICENSE for details.
%
% This program is free software: you can redistribute it and/or modify
% it under the terms of the GNU General Public License as published by
% the Free Software Foundation, either version 3 of the License, or
% (at your option) any later version.
%
% This program is distributed in the hope that it will be useful,
% but WITHOUT ANY WARRANTY; without even the implied warranty of
% MERCHANTABILITY or FITNESS FOR A PARTICULAR PURPOSE.  See the
% GNU General Public License for more details.
%
% You should have received a copy of the GNU General Public License
% along with this program.  If not, see <https://www.gnu.org/licenses/>.
%
\def\section #1 {\advance\secno by 1
  \subsecno=0
  \writenumberedtocentry{section}{#1}{\thechapno.\the\secno}
  \medskip\goodbreak\noindent{\tenssbx{\thechapno.\the\secno.} #1}
  \par\nobreak\smallskip\noindent}
\def\subsection #1 {\advance\subsecno by 1
  \writenumberedtocentry{subsection}{#1}{\thechapno.\the\secno.\the\subsecno}
  \medskip\goodbreak\noindent{\it{\thechapno.\the\secno.\the\subsecno.} #1}
  \par\nobreak\smallskip\noindent}
\def\iint{\mathop{\int\kern-8pt\int}}
\def\iiint{\mathop{\int\kern-8pt\int\kern-8pt\int}}
\def\oiint{\mathop{\int\kern-8pt\int\kern-13.2pt{\bigcirc}}}
\def\sgn{\mathop{\rm sgn}\nolimits} % sign
\def\Re{\mathop{\rm Re}\nolimits}   % real part
\def\Im{\mathop{\rm Im}\nolimits}   % imaginary part
\def\Tr{\mathop{\rm Tr}\nolimits}   % quantum mechanical trace
\def\eqq{\mathop{\vbox{\hbox{\hskip2pt?}\vskip-6pt\hbox{=}}}}
\def\encircle#1{\kern3pt#1\kern-7.6pt$\bigcirc$\kern4pt}
\def\boxit#1{\vbox{\hrule\hbox{\vrule\kern3pt
  \vbox{\kern3pt#1\kern3pt}\kern3pt\vrule}\hrule}}
\def\quote#1{\par\leftskip=36pt\rightskip=36pt\smallskip\noindent#1\par
  \leftskip=0pt\rightskip=0pt\smallskip}
\long\def\plan#1{\par\leftskip=36pt\rightskip=36pt\bigskip%
  \noindent{\it Sammanfattning av f{\"o}rel{\"a}sningen}\smallskip
  \noindent{\it #1}\par\leftskip=0pt\rightskip=0pt} %\vfill\eject}
\def\threepointsummary#1#2#3{\par\leftskip=36pt\rightskip=36pt\bigskip
  \noindent{\it Sammanfattning i tre punkter}\smallskip
  \leftskip=48pt\rightskip=36pt\hangindent=20pt
  \noindent{\it\hbox to 20pt{1. }#1}\smallskip
  \leftskip=48pt\rightskip=36pt\hangindent=20pt
  \noindent{\it\hbox to 20pt{2. }#2}\smallskip
  \leftskip=48pt\rightskip=36pt\hangindent=20pt
  \noindent{\it\hbox to 20pt{3. }#3}\par%\medskip
  \leftskip=0pt\rightskip=0pt\vfill\eject}
\def\epsfig#1{\bigskip\centerline{\epsfbox{#1}}\medskip}
\def\captionwide{\advance\leftskip by 60pt
  \advance\rightskip by 60pt}
\newdimen\itemindent \itemindent=28pt
\newdimen\hangitemindent \hangitemindent=46pt
\def\litem[#1]{\smallbreak\noindent%
  \hbox to\itemindent{\hfil}\hbox to\itemindent{#1\hfill}%
  \hangindent\hangitemindent\ignorespaces}
\def\cleardoublepage{\ifodd\pageno\vfill\eject~\vskip260pt%
  \centerline{[Page intentionally left blank]} ~\vfill\eject%
  \else\vfill\eject\fi}
\newif\ifcolors % creates \ifcolors, \colorstrue, \colorsfalse
\def\red#1{\ifcolors{\color{red}#1}\else#1\fi\ignorespaces}
\colorsfalse     % \colorstrue turns colors on, \colorsfalse turns them off
\voffset=-10.2mm\topskip=0pt
%
% File: teach/elmagii/compiled/src/toc.tex [plain TeX code]
% Github: https://github.com/hp35/elmagii/tree/main/compiled/src
% Last change: February 22, 2026
%
% Table-of-contents formatting macros for the compilation of all Lecture
% Notes in the course "Elektromagnetism II, 1TE626", held 2022-2026 at
% Uppsala University, Sweden.
%
% Copyright (C) 2022-2026, Fredrik Jonsson, under Gnu General Public
% License (GPL) v3. See the enclosed LICENSE for details.
%
% This program is free software: you can redistribute it and/or modify
% it under the terms of the GNU General Public License as published by
% the Free Software Foundation, either version 3 of the License, or
% (at your option) any later version.
%
% This program is distributed in the hope that it will be useful,
% but WITHOUT ANY WARRANTY; without even the implied warranty of
% MERCHANTABILITY or FITNESS FOR A PARTICULAR PURPOSE.  See the
% GNU General Public License for more details.
%
% You should have received a copy of the GNU General Public License
% along with this program.  If not, see <https://www.gnu.org/licenses/>.
%
\countdef\chapno=28         % register to keep track of chapter numbers
\countdef\secno=29          % register to keep track of section numbers
\countdef\subsecno=30       % register to keep track of subsection numbers
\countdef\chapstartpage=31  % register to keep track of start page of chapter
\countdef\appno=32          % register to keep track of appendix numbers
\countdef\figno=32          % register to keep track of figure numbers
\chapno=0
\appno=0
\figno=0
\pageno=-1                  % start with roman page numbering

%
% Define a convenient macro for the expansion of the chapter numbers.
% This macro is introduced in order to provide for an easy change of
% the chapter labeling (from numbers to capital letters) whenever the
% appendices appear, via a redefinition of '\thechapno' in the '\appendix'
% macro definition.
%
\def\thechapno{\the\chapno}

%
% Define a macro for the expansion of page numbers, in similar to the
% '\thechapno' macro for expansion of chapter numbering. In this case
% we are in particular interested in typsetting the pages preceding the
% first chapter with roman numerals; this is done by associating them
% with negative page numbers.
%
\def\thepageno{\ifnum\pageno<0\romannumeral-\pageno \else\number\pageno\fi}
\def\abs#1{\ifnum#1<0 \number-#1\else\number#1\fi}

%
% The following redefinitions of the default eplain TeX table-of-contents
% line styles causes the generated table of contents to be of the form:
%
%        Preface . . . . . . . . . . . . . . . . . . . . . . .  v
%     1  Introduction                                           1
%     2  The classical picture                                  9
%        2.1  The mechanical spring oscillator . . . . . . . . 10
%        2.2  Damped motion  . . . . . . . . . . . . . . . . . 12
%             2.2.1 Longitudinal relaxation  . . . . . . . . . 14
%             2.2.2 Transverse relaxation  . . . . . . . . . . 15
%     3  Lagrangian formulation                                17
%     Bibliography                                             27
%     Index                                                    35
%
\newdimen\tocchapindent \tocchapindent=20pt
\newdimen\tocsecindent \tocsecindent=30pt
\newdimen\tocsubsecindent \tocsubsecindent=27pt
\def\tocpreentry#1#2#3{\line{
  \hbox to\tocchapindent{\hfil}
  \hbox{#1}\dotfill{#3}}}%
\def\tocchapterentry#1#2#3{\bigskip\goodbreak\line{
  \hbox to\tocchapindent{\bf #2\hfil}
  \hbox{\bf #1}\hfill{\bf #3}}}%
\def\tocsectionentry#1#2#3{\line{
  \hbox to\tocchapindent{\hfil}
  \hbox to\tocsecindent{#2\hfil}
  \hbox{#1}\dotfill\ {#3}}}%
\def\tocsubsectionentry#1#2#3{\line{
  \hbox to\tocchapindent{\hfil}
  \hbox to\tocsecindent{\hfil}
  \hbox to\tocsubsecindent{#2\hfil}
  \hbox{#1}\dotfill\ {#3}}}%
\def\tocappendixentry#1#2#3{\bigskip\line{
  \hbox to\tocchapindent{\bf #2\hfil}
  \hbox{\bf #1}\hfill{\bf #3}}}%
\def\tocbibentry#1#2#3{\bigskip\line{
  \hbox{\bf #1}\hfill{\bf #3}}}%
\def\tocindexentry#1#2#3{\tocbibentry{#1}{#2}{#3}}% identical to '\tocbibentry'

%
% The following redefinition of the '\writenumberedcontentsentry' macro of
% the eplain macro package has been done in order to avoid the '\sanitize'
% operation on the secondary number. If '\sanitized', any multiple appearance
% of chapter number, section number, etc. would otherwise be destroyed in the
% secondary number.
%
\def\writecontentsentry#1#2#3{\writenumberedcontentsentry{#1}{#2}{#3}{}}%
\def\writenumberedcontentsentry#1#2#3#4{%
  \csname ifrewrite#1file\endcsname
    \csname open#1file\endcsname
    \toks0 = {\expandafter\noexpand \csname #1#2entry\endcsname}%
    \def\temp{#3}%
    \toks2 = \expandafter{#4}%
    \edef\cs{\the\toks2}%
    \edef\@wr{%
      \write\csname #1file\endcsname{%
        \the\toks0 % the \toc...entry control sequence
        {\sanitize\temp}% the text
        \ifx\empty\cs\else {\cs}\fi % A secondary number, or nothing
        {\noexpand\folio}% the page number
      }%
    }%
    \@wr
  \fi
  \ignorespaces
}%

%
% The '\tableofcontents' macro typesets the table of contents, based on
% the entries written to the .toc file by the '\writenumberedtocentry'
% blocks in the '\chapter', '\section', and '\subsection' macros. The
% page layout and placement of the `Contents' label is identical to the
% preface.
%
\def\tableofcontents{
  \chapstartpage=\pageno
  \message{Contents}
  ~\vskip 5pc\goodbreak\noindent{\twelvecmbx Inneh{\aa}ll}\par
  \nobreak\vskip 48pt\noindent\ignorespaces
  \readtocfile\cleardoublepage}
%
% File: teach/elmagii/compiled/src/index.tex [plain TeX code]
% Github: https://github.com/hp35/elmagii/tree/main/compiled/src
% Last change: February 22, 2026
%
% Macros for the formatting of the index in the compilation of all Lecture
% Notes in the course "Elektromagnetism II, 1TE626", held 2022-2026 at
% Uppsala University, Sweden.
%
% Copyright (C) 2022-2026, Fredrik Jonsson, under Gnu General Public
% License (GPL) v3. See the enclosed LICENSE for details.
%
% This program is free software: you can redistribute it and/or modify
% it under the terms of the GNU General Public License as published by
% the Free Software Foundation, either version 3 of the License, or
% (at your option) any later version.
%
% This program is distributed in the hope that it will be useful,
% but WITHOUT ANY WARRANTY; without even the implied warranty of
% MERCHANTABILITY or FITNESS FOR A PARTICULAR PURPOSE.  See the
% GNU General Public License for more details.
%
% You should have received a copy of the GNU General Public License
% along with this program.  If not, see <https://www.gnu.org/licenses/>.
%
\colorsfalse     % \colorstrue turns colors on, \colorsfalse turns them off
\newif\ifshowindex
\showindextrue  % Use \showindextrue and \showindexfalse to enable/disable index
\def\index{\ifshowindex\vfill\eject%
  ~\vskip 5pc\goodbreak\noindent{\twelvecmbx{Index}}%
  \writenumberedtocentry{index}{\hskip24pt{Index}}%
  \nobreak\vskip 48pt\noindent\ignorespaces%
  \readindexfile{i}\cleardoublepage\fi}
\input src/copyrights.tex
\input src/frontpage.tex
\input src/preface.tex

%
% Define language constants. We are here only concerned with Swedish
% or English, and we for this document set the language to be used
% throughout to Swedish.
%
\newcount\swedish \swedish=0
\newcount\english \english=1
\newcount\lang \lang=\swedish  % Change to \english for English

\frontpage                             % Title page of compilation
\pageno=-2\footline{\hfil\folio\hfil}  % Roman pagenumbering for intro
\copyrights                            % Generic copyrights page
\colorsfalse                           % Turns colors off for the contents
\tableofcontents                       % Four pages, manual page advance
\colorstrue                            % Turns colors on again
\pageno=-9\footline{\hfil\folio\hfil}  % Roman pagenumbering for preface
\preface{{\it First things first}\medskip\noindent
  L{\aa}t oss b{\"o}rja med det viktigaste av allt: Vad av detta kommer p{\aa}
  tentan? I detta har f{\"o}rfattaren en {\it bra} och en {\it d{\aa}lig} nyhet:
  \medskip
  \item{$\bullet$}{Den {\it bra} nyheten {\"a}r att {\it praktiskt taget
                   ingenting} av det som h{\"a}r presenteras kommer p{\aa}
                   tentan.}
  \item{$\bullet$}{Den {\it d{\aa}liga} nyheten {\"a}r att {\it praktiskt
                   taget ingenting} av det som h{\"a}r presenteras kommer
                   p{\aa} tentan.}
  \medskip
  \noindent
  M{\aa}let med dessa {\it Lecture Notes} {\"a}r att bist{\aa} med
  {\it v{\"a}gledning} i den ganska sn{\aa}riga skog som utg{\"o}rs av
  alla teorem inom elektrostatik, magnetostatik och elektrodynamik.
  M{\aa}ls{\"a}ttningen {\"a}r att kondensera de viktigaste passagerna
  fr{\aa}n standardlitteraturen och f{\aa} ihop det sammanhang som utg{\"o}r
  den klassiska elektromagnetiska teorin, med f{\"o}rhoppningen att detta
  underl{\"a}ttar f{\"o}rst{\aa}elsen -- och kreativiteten! -- n{\"a}r det
  senare kommer till praktiskt probleml{\"o}sande. Vi kan p{\aa} s{\"a}tt
  och vis s{\"a}ga att vi har f{\"o}ljande trestegsraket framf{\"o}r oss:
  \medskip
  $$
    \underbrace{
      \hbox{Elektromagnetisk f{\"a}ltteori}
    }_{\vbox{
        \hbox{Kommer (n{\"a}stan) inte}
        \hbox{alls p{\aa} tentan; f{\"o}rst{\aa}else}
        \hbox{hj{\"a}lper dock praktisk}
        \hbox{probleml{\"o}sning!}
       }
      }
    \quad\to\quad
    \underbrace{
      \hbox{Elektromagnetisk probleml{\"o}sning}
    }_{\vbox{
        \hbox{Kommer {\it garanterat} p{\aa} tentan!}
        \hbox{L{\"o}s massor med uppgifter och}
        \hbox{f{\"o}rs{\"o}k att f{\"o}rst{\aa} vad den teore-}
        \hbox{tiska grunden handlar om, s{\aa}}
        \hbox{kommer det att g{\aa} bra!}
       }
      }
    \quad\to\quad
    \hbox{Godk{\"a}nd tenta!}
  $$
  \medskip
  \noindent
  Det praktiska probleml{\"o}sandet {\"a}r n{\aa}got som ni som g{\aa}r denna
  kurs f{\aa}r {\"o}va i lektionssalar p{\aa} tillf{\"a}llen utanf{\"o}r
  f{\"o}rel{\"a}sningstid, och det {\"a}r ocks{\aa} praktiskt probleml{\"o}sande
  som utg{\"o}r fokus i tentamensuppgifterna. Att avkr{\"a}va av teknologer att
  kunna dra fram en godtycklig och ofta algebraiskt komplex h{\"a}rledning ur
  byxfickan vore alldeles f{\"o}r mycket beg{\"a}rt, det f{\"o}rst{\aa}r vem
  som helst.
  \bigskip\noindent{\it Ett undantag}\medskip\noindent
  Med detta sagt s{\aa} finns det {\it ett specifikt undantag} till denna
  praktiska probleml{\"o}sning p{\aa} tentan, n{\"a}mligen h{\"a}rledningen
  av de generella elektromagnetiska v{\aa}gekvationerna som vi h{\"a}rleder
  fram i F{\"o}rel{\"a}sning~9. K{\"a}nn dock ingen oro f{\"o}r denna uppgift;
  tentauppgiften i sig samt l{\"o}sningsf{\"o}rslaget delas ut under f{\"o}rsta
  f{\"o}rel{\"a}sningen, samt att vi har en gemensam genomg{\aa}ng av
  l{\"o}sningen under F{\"o}rel{\"a}sning~9.
  Den som vill kan {\"a}gna denna uppgift {\aa}t att dissekera metodiken och
  hur vi v{\"a}ver in centrala begrepp fr{\aa}n kursen, via Maxwells ekvationer
  samt de konstitutiva relationerna -- eller s{\aa} kan man helt enkelt sl{\aa}
  in detta medelst mekanisk korvstoppning. {\it B{\aa}da varianterna {\"a}r helt
  okej}, och det {\"a}r f{\"o}rfattarens uppgift att {\"a}ven den som inte
  brinner f{\"o}r elektrodynamikens sk{\"o}nhet faktiskt kommer att f{\aa}
  utbyte av korvstoppningsmetoden.
  Ytinl{\"a}rning kommer f{\"o}re djupinl{\"a}rning.
  S{\aa}, med detta sp{\"o}rsm{\aa}l ur v{\"a}gen, l{\aa}t oss g{\aa} vidare!
  \bigskip\noindent{\it Let's move on}\medskip\noindent
  Detta {\"a}r den kompletta serien av f{\"o}rel{\"a}sningar i kursen
  {\it Elektromagnetism II} (1TE626, 5hp), givna 2022--2025 f{\"o}r teknologer
  i andra {\aa}rskursen f{\"o}r Teknisk Fysik p{\aa} Masterniv{\aa} (syftande
  mot civilingenj{\"o}rsexamen) p{\aa} {\AA}ngstr{\"o}mlaboratoriet, Uppsala
  Universitet, med varje kapitel motsvarande en f{\"o}rel{\"a}sning om tv{\aa}
  timmar ($2\times45$ minuter i svensk akademisk standard).
  Dessa f{\"o}rel{\"a}sningar publiceras h{\"a}rmed publikt med hopp om att
  de kan v{\"a}gleda i den stundtals sn{\aa}riga och matematiskt t{\"a}mligen
  intensiva disciplin som elektrostatik, magnetostatik och elektrodynamik
  utg{\"o}r.
  Vissa h{\"a}rledningar {\"a}r till sin natur t{\"a}mligen omst{\"a}ndliga
  och kan med f{\"o}rdel l{\"a}mnas till den intresserade l{\"a}saren att
  f{\"o}lja  p{\aa} egen kammare med papper och penna tillhands.

  F{\"o}rel{\"a}sningarna som s{\aa}dana str{\"a}var mot att f{\"o}rklara de
  grundl{\"a}ggande {\it principerna} p{\aa} vilka elektromagnetismen {\"a}r
  baserade, snarare {\"a}n att bist{\aa} med praktisk probleml{\"o}sning.
  F{\"o}rhoppningen {\"a}r dock att den teori som h{\"a}r presenteras p{\aa}
  ett djupare plan, genom h{\"a}rledning av de samband som d{\"a}refter kommer
  till anv{\"a}ndning just i probleml{\"o}sandet, kan v{\"a}gleda l{\"a}saren
  till en k{\"a}nsla f{\"o}r fysikaliska samband och tumregler.
  Som ett viktigt komplement till f{\"o}rel{\"a}sningarna ges l{\"a}ngs med
  kursens g{\aa}ng gott om {\"o}vningstillf{\"a}llen d{\"a}r kursdeltagarna
  f{\aa}r {\"o}va sig i oms{\"a}ttandet av teori i praktiskt probleml{\"o}sande.

  L{\"a}mpliga f{\"o}rkunskaper f{\"o}r att kunna tillg{\"o}ra sig
  inneh{\aa}llet som t{\"a}cks av f{\"o}rel{\"a}sningarna som h{\"a}rmed
  l{\"a}ggs fram {\"a}r en- och flervariabelanalys samt vektoranalys.
  Grundl{\"a}ggande kunskaper i elektrostatik och magnetostatik kan givetvis
  underl{\"a}tta, men {\"a}r inte n{\"o}dv{\"a}ndiga d{\aa} grunderna g{\aa}s
  igenom i F{\"o}rel{\"a}sning~1 och~4.
  {\"A}ven om f{\"o}rel{\"a}sningarna som h{\"a}r presenteras {\"a}r upplagda
  enligt en specifik kursplan, s{\aa} {\"a}r det f{\"o}rfattarens
  f{\"o}rhoppning att de kan bidra {\"a}ven utanf{\"o}r den akademiska
  sf{\"a}ren p{\aa} {\AA}ngstr{\"o}mlaboratoriet och Uppsala Universitet.

  \bigskip\noindent{\it Helikopterperspektivet {\"o}ver klassisk
                        elektromagnetism}\medskip\noindent
  F{\"o}r att l{\"a}saren skall f{\aa} en uppfattning om riktning och
  v{\"a}gen fram{\aa}t under varje tv{\aa}timmarspass introduceras varje
  f{\"o}rel{\"a}sning med en kortare sammanfattning samt en sammanst{\"a}llning
  av det som f{\"o}rfattaren anser vara de tre viktigaste punkterna som
  l{\"a}saren kan fokusera p{\aa} att ta med sig.
  Som en ytterligare assistans finns i denna sammanst{\"a}llning, p{\aa} sidan
  xix, en ``v{\"a}rldskarta {\"o}ver klassisk elektromagnetism''
  bifogad,\numberedfootnote{Finns {\"a}ven tillg{\"a}nglig som enskilt
    blad p{\aa} {\tt https://github.com/hp35/elmagii/tree/main/worldmap}}
  med hopp om att denna p{\aa} ett hyfsat {\"o}versk{\aa}dligt s{\"a}tt visar
  p{\aa} de centrala blocken, teoremen samt hur dessa l{\"a}nkar till varandra.
  Att sammanst{\"a}lla hela den klassiska elektromagnetismen p{\aa} en enda
  sida {\"a}r inte helt enkelt, men f{\"o}rfattaren har {\"a}nd{\aa}
  f{\"o}rhoppningen att denna kan hj{\"a}lpa teknologen till att skaffa
  sig en {\"o}verblick {\"o}ver {\"a}mnet.
  En kompakt formelsamling som sammanfattar f{\"o}rel{\"a}sningarna {\"a}r
  bifogad p{\aa} sidorna xi--xiv, vilka kan vara l{\"a}mpliga att skriva ut
  och ha till hands i fysisk form under det att l{\"a}saren l{\"o}ser problem
  under lektioner eller p{\aa} egen kammare.

  \bigskip\noindent{\it Kurslitteratur}\medskip\noindent
  I kursen {\it Elektromagnetism II} {\"a}r den rekommenderade kurslitteraturen
  David J.~Griffiths {\it Introduction to Electrodynamics}, och f{\"o}rekommande
  sidreferenser till denna i f{\"o}rel{\"a}sningarna till denna bok g{\"a}ller
  den fj{\"a}rde utg{\aa}van (2017). Den som {\"o}nskar inf{\"o}rskaffa detta
  standardverk inom elektromagnetism kan dock med f{\"o}rdel passa p{\aa} att
  ist{\"a}llet rikta in sig p{\aa} den uppdaterade femte utg{\aa}van (2023).

  \bigskip\noindent{\it Open source}\medskip\noindent
  Denna f{\"o}rel{\"a}sningsserie publiceras dels som ett elektroniskt dokument
  p{\aa} {\it Digitala Vetenskapliga Arkivet} (DiVA),
  {\tt https://www.diva-portal.org/smash/record.jsf?pid=diva2:25333}, men
  {\"a}ven som helt {\"o}ppen k{\"a}ll\-kod p{\aa}
  {\tt https://github.com/hp35/elmagii/}, med all text i ren \TeX\ samt kod
  i \MP\ och Python f{\"o}r generering av samtliga figurer.
  \vskip32pt
  \noindent Fredrik Jonsson,\par
  \noindent Februari 2026}

%
% Collection of formulas in electromagnetism.
%
\font\twelvecmbx=cmbx12
\input ../lect-01/macros/eplain.tex
\input amssym % to get the {\Bbb E} font (strikethrough E)
\def\iint{\mathop{\int\kern-8pt\int}}
\def\iiint{\mathop{\int\kern-8pt\int\kern-8pt\int}}
\def\oiint{\mathop{\int\kern-8pt\int\kern-13.2pt{\bigcirc}}}
\def\sgn{\mathop{\rm sgn}\nolimits} % sign
\def\Re{\mathop{\rm Re}\nolimits}   % real part
\def\Im{\mathop{\rm Im}\nolimits}   % imaginary part
\def\Tr{\mathop{\rm Tr}\nolimits}   % quantum mechanical trace
\def\cleardoublepage{\ifodd\pageno\vfill\eject~\vskip260pt%
  \centerline{[Page intentionally left blank]} ~\vfill\eject%
  \else\vfill\eject\fi}
\vskip 0pc\goodbreak\noindent{\twelvecmbx{Formelsamling}}%
  \writenumberedtocentry{index}{\hskip24pt{Formelsamling}}%
  \nobreak\vskip 0pt\noindent\ignorespaces
%%%%%%%%%%%%%%%%% Begin two-column mode %%%%%%%%%%%%%%%%%%
{\newdimen\fullhsize
\fullhsize=170mm \hsize=82mm
\topskip=0pt\baselineskip=12pt\parskip=0pt\leftskip=0pt\rightskip=0pt
\parindent=0pt
\def\fulline{\hbox to\fullhsize}
\footline={\hfill{\rm\folio}}
\let\lr=L \newbox\leftcolumn
\output={\if L\lr
  \global\setbox\leftcolumn=\columnbox \global\let\lr=R
  \else \doubleformat \global\let\lr=L\fi
  \ifnum\outputpenalty>-20000 \else\dosupereject\fi}
\def\doubleformat{\shipout\vbox{\makeheadline
    \fulline{\box\leftcolumn\hfil\columnbox}
  \makefootline}
  \advancepageno}
\def\columnbox{\leftline{\pagebody}}
\def\eqsection#1{%
  \vskip6pt\goodbreak%
  \hrule width\hsize\vskip1pt%
  \hrule width\hsize
  \vskip4pt%
  \centerline{\bf\vphantom{lq} #1}%
  \vskip4pt%
  \hrule width\hsize\vskip1pt%
  \hrule width\hsize\vskip10pt}
\def\eqtitle#1{\vskip2pt\goodbreak{\it #1}\par}
\def\versionnumber{1.0}             % Version of this collection of formulas
\def\versiondate{11 december, 2025} % Timestamp of this collection of formulas
\def\version{v.\versionnumber\ (\versiondate)}
\vskip4pt
\centerline{Elektromagnetism II, Uppsala Universitet}
\centerline{\version}
\centerline{{\tt github.com/hp35/elmagii}}
\vskip10pt
\eqsection{Variabler, konstanter och enheter}
\eqtitle{Beteckningar och SI-enheter}
\halign{#\hfil\ &#\hfil&\ \hfil(#)\cr
${\bf A}$
    & Vektorpotential
    & ${\rm V}\cdot{\rm s}/{\rm m}$\cr
${\bf B}$
    & Magnetisk fl{\"o}dest{\"a}thet ${\bf B}=\mu_0\mu_{\rm r}{\bf H}$
    & ${\rm T}$\cr
$c_0$
    & Ljushastighet i vakuum, $c^{-2}_0=\varepsilon_0\mu_0$
    & ${\rm m}/{\rm s}$\cr
$c$
    & Ljushastighet i medium, $c=c_0/n$
    & ${\rm m}/{\rm s}$\cr
${\bf D}$
    & Elektrisk fl{\"o}dest{\"a}thet
      ${\bf D}=\varepsilon_0\varepsilon_{\rm r}{\bf E}$
    & ${\rm C}/{\rm m}^2$\cr
$\varepsilon_0$
    & Elektrisk permittivitet i vakuum
    & ${\rm F}/{\rm m}$\cr
$\varepsilon_{\rm r}$
    & Relativ elektrisk permittivitet
    & $1$\cr
$e$
    & Elementarladdning
    & ${\rm C}$\cr
${\bf e}_r$
    & Riktning ${\bf e}_r=({\bf x}-{\bf x}')/|{\bf x}-{\bf x}'|$
    & $1$\cr
${\bf E}$
    & Elektrisk f{\"a}ltstyrka
    & ${\rm V}/{\rm m}$\cr
${\cal E}$
    & Elektromotorisk ``kraft'' (EMK)
    & ${\rm V}$\cr
$\phi$
    & Skal{\"a}r potential
    & ${\rm V}$\cr
$\Phi_{\rm E}$
    & Elektriskt fl{\"o}de $\Phi_{\rm E}=\int\kern-5pt\int{\bf E}\cdot d{\bf S}$
    & ${\rm V}\cdot{\rm m}$\cr
$\Phi_{\rm M}$
    & Magnetiskt fl{\"o}de $\Phi_{\rm M}=\int\kern-5pt\int{\bf B}\cdot d{\bf S}$
    & ${\rm T}\cdot{\rm m}^2$\cr
$f$
    & Frekvens
    & ${\rm Hz}$\cr
${\bf F}$
    & Kraft
    & ${\rm N}$\cr
${\bf F}_{\rm mag}$
    & Magnetisk kraft
    & ${\rm N}$\cr
${\bf H}$
    & Magnetiseringsstyrka
    & ${\rm A}/{\rm m}$\cr
${\bf I}$
    & Elektrisk str{\"o}m
    & ${\rm A}$\cr
$I_{\rm enc}$
    & Omsluten elektrisk str{\"o}m
    & ${\rm A}$\cr
${\bf J}$
    & Str{\"o}mt{\"a}thet ${\bf J}={\bf J}_{\rm b}+{\bf J}_{\rm f}$
    & ${\rm A}/{\rm m}^2$\cr
${\bf J}_{\rm b}$
    & Bunden elektrisk str{\"o}mt{\"a}thet
    & ${\rm A}/{\rm m}^2$\cr
${\bf J}_{\rm eff}$
    & Effektiv elektrisk str{\"o}mt{\"a}thet
    & ${\rm A}/{\rm m}^2$\cr
${\bf J}_{\rm f}$
    & Fri elektrisk str{\"o}mt{\"a}thet
    & ${\rm A}/{\rm m}^2$\cr
${\bf k}$
    & V{\aa}gvektor
    & $1/{\rm m}$\cr
$k$
    & V{\aa}gtal, $k=|{\bf k}|=2\pi/\lambda$
    & $1/{\rm m}$\cr
${\bf K}_{\rm b}$
    & Bunden ytstr{\"o}mt{\"a}thet
    & ${\rm A}/{\rm m}$\cr
$\lambda$
    & Linjeladdningst{\"a}thet
    & ${\rm C}/{\rm m}$\cr
$\lambda$
    & V{\aa}gl{\"a}ngd, $\lambda=c/f$
    & ${\rm m}$\cr
$\mu_0$
    & Magnetisk permeabilitet i vakuum
    & ${\rm H}/{\rm m}$\cr
$\mu_{\rm r}$
    & Relativ magnetisk permeabilitet
    & $1$\cr
$m_{\rm e}$
    & Elektronens vilomassa
    & ${\rm kg}$\cr
$m_{\rm p}$
    & Protonens vilomassa
    & ${\rm kg}$\cr
${\bf m}$
    & Magnetiskt dipolmoment
    & ${\rm A}\cdot{\rm m}^2$\cr
${\bf M}$
    & Magnetisering
    & ${\rm A}/{\rm m}$\cr
$n$
    & Brytningsindex $n=\varepsilon^{1/2}_{\rm r}$
    & $1$\cr
$\psi$
    & Gauge-funktion $\psi=\psi({\bf x},t)$
    & ${\rm V}\cdot{\rm s}$\cr
${\bf p}$
    & Elektriskt dipolmoment
    & ${\rm C}\cdot{\rm m}$\cr
${\bf P}$
    & Elektrisk polarisationsdensitet
    & ${\rm C}/{\rm m}^2$\cr
$P$
    & Effekt
    & ${\rm W}$\cr
$q$
    & Elektrisk laddning
    & ${\rm C}$\cr
$r$
    & Radie $r=|{\bf x}-{\bf x}'|$
    & ${\rm m}$\cr
$\rho$
    & Laddningst{\"a}thet $\rho=\rho_{\rm b}+\rho_{\rm f}$
    & ${\rm C}/{\rm m}^3$\cr
$\rho_{\rm b}$
    & Bunden elektrisk laddningst{\"a}thet
    & ${\rm C}/{\rm m}^3$\cr
$\rho_{\rm f}$
    & Fri elektrisk laddningst{\"a}thet
    & ${\rm C}/{\rm m}^3$\cr
$\sigma$
    & Ytladdningst{\"a}thet
    & ${\rm C}/{\rm m}^2$\cr
${\bf S}$
    & Poyntingvektor
    & ${\rm W}/{\rm m}^2$\cr
$\tau$
    & Vridmoment
    & ${\rm N}\cdot{\rm m}^2$\cr
$t$
    & Tid
    & ${\rm s}$\cr
${\bf v}$
    & Hastighet
    & ${\rm m}/{\rm s}$\cr
$v$
    & Fart, $v=|{\bf v}|$
    & ${\rm m}/{\rm s}$\cr
$V$
    & Volym
    & ${\rm m}^3$\cr
$\omega$
    & Vinkelfrekvens, $\omega=2\pi f$
    & $1/{\rm s}$\cr
$W_{\rm e}$
    & Elektrostatisk energi
    & ${\rm J}$\cr
$W_{\rm m}$
    & Magnetostatisk energi
    & ${\rm J}$\cr
${\bf x}$
    & Ortsvektor till observationspunkt
    & ${\rm m}$\cr
${\bf x}'$
    & Ortsvektor till k{\"a}llpunkt
    & ${\rm m}$\cr
$\chi_{\rm e}$
    & Elektrisk susceptibilitet, $\varepsilon_{\rm r}=1+\chi_{\rm e}$
    & $1$\cr
$\chi_{\rm m}$
    & Magnetisk susceptibilitet
    & $1$\cr
}

\eqsection{Naturkonstanter}
\halign{#\hfil\cr
  $\varepsilon_0=8.8541878188(14)\times10^{12}\ {\rm F}/{\rm m}$\cr
  $\mu_0=1.25663706127(20)\times10^{-6}\ {\rm N}/{\rm A}^2$\cr
  $c_0=299\,792\,458\ {\rm m}/{\rm s}$ (exakt per definition)\cr
  $e=1.602176634\times10^{-19}\ {\rm C}$ (exakt per definition)\cr
  $m_{\rm e}=9.1093837139(28)\times10^{-31}\ {\rm kg}$\cr
  $m_{\rm p}=1.67262192595(52)\times10^{-27}\ {\rm kg}$\cr
}

\eqsection{Konstitutiva relationer}
% \eqtitle{Elektrisk polarisationsdensitet ${\bf P}$}
\eqtitle{Elektrisk polarisationsdensitet}
$$
  {\bf P} \equiv \Big\langle{{d{\bf p}}\over{dV}}\Big\rangle
    = \varepsilon_0\chi_{\rm e}{\bf E},\qquad
  \nabla\cdot{\bf P}=-\rho_{\rm b}
$$
% \eqtitle{Elektrisk fl{\"o}dest{\"a}thet ${\bf D}$}
\eqtitle{Elektrisk fl{\"o}dest{\"a}thet}
$$
  {\bf D}\equiv\varepsilon_0{\bf E}+{\bf P}
    =\varepsilon_0\varepsilon_{\rm r}{\bf E}
$$
% \eqtitle{Magnetisering ${\bf M}$}
\eqtitle{Magnetisering}
$$
  {\bf M} \equiv \Big\langle{{d{\bf m}}\over{dV}}\Big\rangle
    = {{1}\over{\mu_0}}\Big(1-{{1}\over{\mu_{\rm r}}}\Big){\bf B}
$$
% \eqtitle{Magnetisk fl{\"o}dest{\"a}thet ${\bf B}$}
\eqtitle{Magnetisk fl{\"o}dest{\"a}thet}
$$
  {\bf B}\equiv\mu_0({\bf H}+{\bf M})
    =\mu_0\mu_{\rm r}{\bf H}
$$
\vfill\eject

\eqsection{Elektrostatik}
\eqtitle{Coulombs kraftlag}
$$
  {\bf F}({\bf x})={{qq'}\over{4\pi\varepsilon_0}}
     {{({\bf x}-{\bf x}')}\over{|{\bf x}-{\bf x}'|^3}}
     =q{\bf E}({\bf x})
$$
\eqtitle{Elektrisk f{\"a}ltstyrka fr{\aa}n punktladdning $q'$}
$$
  {\bf E}({\bf x})={{q'}\over{4\pi\varepsilon_0}}
    {{({\bf x}-{\bf x}')}\over{|{\bf x}-{\bf x}'|^3}}
      ={{q'}\over{4\pi\varepsilon_0 r^2}}{\bf e}_r
$$
\eqtitle{Coulombs generaliserade lag -- ``Coulombintegralen''}
$$
  {\bf E}({\bf x})={{1}\over{4\pi\varepsilon_0}}\iiint_V\rho({\bf x}')
    {{({\bf x}-{\bf x}')}\over{|{\bf x}-{\bf x}'|^3}}\,dV'
$$
\eqtitle{Existens av skal{\"a}r elektrostatisk potential}
Ur Coulombs generaliserade lag f{\"o}r elektrostatiska f{\"a}lt erh{\aa}lls
$\nabla\times{\bf E}={\bf 0}$, vilket visar att den elektriska f{\"a}ltstyrkan
alltid kan skrivas som en gradient av en skal{\"a}r funktion, den skal{\"a}ra
potentialen $\phi$, definierad som
$$
  {\bf E}=-\nabla\phi.
$$
\eqtitle{Elektriskt fl{\"o}de}
$$
  \Phi_{\rm E}=\oiint_S {\bf E}({\bf x})\cdot d{\bf S}
     ={{1}\over{\varepsilon_0}}\iiint_V\rho({\bf x})\,dV
     ={{q_{\rm tot}}\over{\varepsilon_0}}
$$
\eqtitle{Gauss lag f{\"o}r elektrisk fl{\"o}dest{\"a}thet}
$$
  \oiint{\bf D}\cdot d{\bf S}=\iiint\rho\,dV
$$
\eqtitle{Poissons ekvation (Coulombs lag) for skal{\"a}r potential}
$$
  \nabla^2\phi({\bf x})=-\rho({\bf x})/\varepsilon_0
$$
\eqtitle{Skal{\"a}r potential, explicit l{\"o}sning till Poissons ekvation}
$$
  \phi({\bf x})\equiv{{1}\over{4\pi\varepsilon_0}}\iiint_V
      {{\rho({\bf x}')}\over{|{\bf x}-{\bf x}'|}}\,dV'
  %  \ \Rightarrow\ {\bf E}({\bf x})=-\nabla\phi({\bf x})
$$
\eqtitle{Skal{\"a}r elektrostatisk potential fr{\aa}n punktladdning $q'$}
$$
  \phi({\bf x})
    ={{1}\over{4\pi\varepsilon_0}}{{q'}\over{|{\bf x}-{\bf x}'|}},
      \quad{\bf x}\ne{\bf x}'.
$$
\eqtitle{Arbete som tillf{\"o}rs punktladdning $q$ vid f{\"o}rflyttning fr{\aa}n
         $a$ till $b$}
$$
  W_{\rm e}=-\int^{{\bf x}_b}_{{\bf x}_a}{\bf F}({\bf x})\cdot d{\bf l}
          =q\big(\phi({{\bf x}_b})-\phi({{\bf x}_a})\big) % =W_b-W_a
$$

\eqsection{Magnetostatik}
\eqtitle{Lorentz-kraften p{\aa} fri laddning $q$}
$$
  {\bf F}=q\big({\bf E}+{\bf v}\times{\bf B}\big)
$$
\eqtitle{Amp\`eres kraftlag p{\aa} str{\"o}mslinga b{\"a}rande
  str{\"o}mmen $I$}
$$
  {\bf F}_{\rm mag}=\int^{{\bf x}_b}_{{\bf x}_a} ({\bf I}\times{\bf B})\,dl
$$
\eqtitle{Str{\"o}mmen $I$ genom en yta $S$ med str{\"o}mt{\"a}thet}
$$
  I=\iint_S {\bf J}\cdot d{\bf S}
$$
\eqtitle{Str{\"o}mt{\"a}thet i medium med konduktivitet $\sigma$}
$$
  {\bf J}=\sigma{\bf E}
$$
\eqtitle{Lagen om att elektrisk laddning inte kan f{\"o}rsvinna --
         kontinuitetsekvationen}
$$
  \nabla\cdot{\bf J}=-{{d\rho}\over{dt}}.
$$
\eqtitle{Statiska problem}
$$
  {{d\rho}\over{dt}}=0\quad\underline{\hbox{och}}\quad{{d{\bf J}}\over{dt}}=0.
$$
\eqtitle{Divergens av str{\"o}mt{\"a}theten i statiska problem}
$$
  {{d\rho}\over{dt}}=0\quad\Leftrightarrow\quad\nabla\cdot{\bf J}=0
$$
\eqtitle{Biot--Savarts lag f{\"o}r str{\"o}mslingor}
$$
  {\bf B}({\bf x})={{\mu_0}\over{4\pi}}\int^{{\bf x}_b}_{{\bf x}_a}
      {{{\bf I}({\bf x}')\times({\bf x}-{\bf x}')}
        \over{|{\bf x}-{\bf x}'|^3}}\,dl'
$$
\eqtitle{Biot--Savarts generella lag}
$$
  {\bf B}({\bf x})={{\mu_0}\over{4\pi}}\iiint_V
      {{{\bf J}({\bf x}')\times({\bf x}-{\bf x}')}
        \over{|{\bf x}-{\bf x}'|^3}}\,dV'
$$
\eqtitle{Existens av magnetostatisk vektorpotential}
Ur Biot--Savarts generalla lag f{\"o}r magnetostatiska f{\"a}lt erh{\aa}lls
$\nabla\cdot{\bf B}={\bf 0}$, vilket visar att den magnetiska
fl{\"o}dest{\"a}theten alltid kan skrivas som en rotation av en vektorfunktion,
vektorpotentialen ${\bf A}$, definierad som
$$
  {\bf B}=\nabla\times{\bf A}.
$$
\eqtitle{Gauss lag f{\"o}r magnetisk fl{\"o}dest{\"a}thet --
         icke-existens f{\"o}r magnetiska mono\-poler} % existerar inte}
$$
  \nabla\cdot{\bf B}=0\qquad\hbox{(alltid)}
$$
\eqtitle{Amp\`eres statiska lag}
$$
  \oint_{\Gamma}{\bf B}\cdot d{\bf l}
    =\mu_0\iint_S{\bf J}\cdot d{\bf S}
    =\mu_0I_{\rm enc}
  \quad\Leftrightarrow\quad
  \nabla\times{\bf B}=\mu_0{\bf J}
$$
\eqtitle{Bunden elektrisk str{\"o}mt{\"a}thet}
$$
  {\bf J}_{\rm b}=\nabla\times{\bf M}
$$
\eqtitle{Bunden ytstr{\"o}mt{\"a}thet}
$$
  {\bf K}_{\rm b}={\bf M}\times{\bf e}_n
$$
\eqtitle{Poissons ekvation (Amp\`eres lag) f{\"o}r vektorpotentialen}
$$
  \nabla^2{\bf A}=-\mu_0{\bf J},
$$
\eqtitle{Vektorpotential, explicit l{\"o}sning till Poissons ekvation}
$$
  {\bf A}({\bf x})={{\mu_0}\over{4\pi}}\iiint_V
    {{{\bf J}({\bf x}')}\over{|{\bf x}-{\bf x}'|}}\,dV'
$$

\eqsection{Elektrodynamik}
\eqtitle{Magnetiskt fl{\"o}de}
$$
  \Phi_{\rm M}=\iint_S{\bf B}\cdot d{\bf S}
$$
\eqtitle{Elektromotorisk ``kraft'' runt en sluten slinga $\Gamma$}
$$
  {\cal E}=\oint_{\Gamma}\bigg({{{\bf F}}\over{q}}\bigg)\cdot d{\bf l}
    =\oint_{\Gamma}[{\bf E}+({\bf v}\times{\bf B})\big]\cdot d{\bf l}
$$
\eqtitle{Faradays induktionslag}
$$
  {\cal E}=-{{d\Phi_{\rm M}}\over{dt}}
$$
\eqtitle{Lenz lag}
En inducerad str{\"o}m har en riktning som motverkar orsaken till att den
uppkom. Detta inneb{\"a}r att om magnetf{\"a}ltet genom en ledande slinga
{\"o}kar, s{\aa} kommer den i slingan inducerade str{\"o}mmen att ha en
riktning som skapar ett magnetf{\"a}lt som motverkar {\"o}kningen, och vice
versa.
\eqtitle{Faradays induktionslag f{\"o}r spole med $N$ varv}
$$
  {\cal E}=-N{{d\Phi_{\rm M}}\over{dt}}
$$
\eqtitle{Faradays lag p{\aa} differentialform}
$$
  \nabla\times{\bf E}({\bf r},t)
    =-{{\partial{\bf B}({\bf x},t)}\over{\partial t}}
$$
\eqtitle{Faradays lag p{\aa} integralform}
$$
  \oint_{\Gamma}{\bf E}({\bf r},t)\cdot d{\bf l}
    =-{{d}\over{dt}}\iint_S{\bf B}({\bf x},t)\cdot d{\bf S}
$$
\eqtitle{{\"O}msesidig induktans beskrivs av det magnetiska fl{\"o}de
som genereras i en sekund{\"a}rslinga $\Gamma$ fr{\aa}n en
str{\"o}m $I'(t)$ som drivs genom en prim{\"a}rslinga $\Gamma'$,}
$$
  \Phi_{\rm M}=M_{\Gamma\Gamma'} I'(t)
$$
\eqtitle{Neumanns formel f{\"o}r {\"o}msesidig induktans}
$$
  M_{\Gamma\Gamma'}={{\mu_0}\over{4\pi}}\oint_{\Gamma}\oint_{\Gamma'}
        {{d{\bf l}'\cdot d{\bf l}}\over{|{\bf x}-{\bf x}'|}},
  \qquad M_{\Gamma\Gamma'}=M_{\Gamma'\Gamma}
$$
\eqtitle{Maxwells ekvationer p{\aa} differentialform}
$$
  \eqalign{
    \nabla\times{\bf E}
       &=-{{\partial{\bf B}}\over{\partial t}},\hskip53pt
   \nabla\cdot{\bf D}=\rho_{\rm f},\cr
   \nabla\times{\bf H}&={\bf J}_{\rm f}
       +{{\partial{\bf D}}\over{\partial t}},\hskip40pt
    \nabla\cdot{\bf B}=0.\cr
  }
$$
\eqtitle{Maxwells ekvationer p{\aa} integralform}
$$
  \eqalign{
    \oint_{\Gamma}{\bf E}\cdot d{\bf l}
        &=-{{\partial}\over{\partial t}}\iint_S{\bf B}\cdot d{\bf S},\cr
    \oint{\bf H}\cdot d{\bf l} & =\iint{\bf J}_{\rm f}\cdot d{\bf S}
       +{{\partial}\over{\partial t}}\iint{\bf D}\cdot d{\bf S},\cr
    \oiint{\bf D}\cdot d{\bf S} & =\iiint\rho\,dV,\cr
    \oiint{\bf B}\cdot d{\bf S} & =0.\cr
  }
$$
\eqtitle{De elektromagnetiska v{\aa}gekvationerna}
$$
  \eqalign{
    \nabla\times\nabla\times{\bf E}
      +{{1}\over{c^2}}{{\partial^2{\bf E}}\over{\partial t^2}}&=
         -\mu_0{{\partial {\bf J}_{\rm eff}}\over{\partial t}},\cr
    \nabla\times\nabla\times{\bf B}
      +{{1}\over{c^2}}{{\partial^2{\bf B}}\over{\partial t^2}}&=
          \mu_0\nabla\times{\bf J}_{\rm eff},\cr
  }
$$
d{\"a}r den gemensamma k{\"a}lltermen
$$
  {\bf J}_{\rm eff}
    ={\bf J}_{\rm f}
      +{{\partial{\bf P}}\over{\partial t}}
      +\nabla\times{\bf M}
$$
{\"a}r den effektiva str{\"o}mt{\"a}theten, inkluderande den fria
str{\"o}mt{\"a}theten ${\bf J}_{\rm f}$, den elektriska
polarisationsdensi\-teten ${\bf P}$ samt magnetiseringen ${\bf M}$.
% Fr{\aa}n de elektromagnetiska v{\aa}gekvationerna kan samt\-liga
% elektrostatiska, magnetostatiska eller elektrodynamiska fall h{\"a}rledas,
% beroende p{\aa} vilka termer som kan s{\"a}ttas till noll.
\vfill\eject

\eqsection{Elektrodynamik och potentialer}
\eqtitle{Elektrodynamiska potentialer}
$$
  \eqalign{
    {\bf E}({\bf x},t)
      &=-\nabla\phi({\bf x},t)
          -{{\partial{\bf A}({\bf x},t)}\over{\partial t}},\cr
    {\bf B}({\bf x},t)
      &=\nabla\times{\bf A}({\bf x},t).\cr
  }
$$
\eqtitle{Gauge-transformen}
Med den godtyckliga och tv{\aa} g{\aa}nger i rum och tid kontinuerligt
differentierbara gauge-funktionen $\psi({\bf x},t)$, {\"a}r
gauge-transformationen definierad som
$$
  \eqalign{
    {\bf A}'({\bf x},t)
      &={\bf A}({\bf x},t)+\nabla\psi({\bf x},t),\cr
    \phi'({\bf x},t)
      &=\phi({\bf x},t)-{{\partial\psi({\bf x},t)}\over{\partial t}},\cr
  }
$$
vilken l{\"a}mnar de elektromagnetiska f{\"a}lten invarianta.
\eqtitle{Lorenz-villkoret}
Under Lorenz-villkoret v{\"a}ljs gauge-funktionen $\psi$ s{\aa} att
$$
  \nabla\cdot{\bf A}'({\bf x},t)+
    {{1}\over{c^2}}{{\partial\phi'({\bf x},t)}\over{\partial t}} = 0
$$
\eqtitle{Lorenz-villkoret -- V{\aa}gekvationer f{\"o}r potentialer}
Under Lorenz-villkoret frikopplas v{\aa}gekvationerna f{\"o}r den skal{\"a}ra
potentialen och vektorpotentialen, till
$$
  \eqalign{
    \Big(
      \nabla^2-{{1}\over{c^2}}{{\partial^2}\over{\partial t^2}}
    \Big)\phi({\bf x},t)
        &=-{{\rho({\bf x},t)}\over{\varepsilon_0\varepsilon_{\rm r}}},\cr
    \Big(
      \nabla^2-{{1}\over{c^2}}{{\partial^2}\over{\partial t^2}}
    \Big){\bf A}({\bf x},t)
        &=-\mu_0{\bf J}_{\rm f}({\bf x},t).\cr
  }
$$
\eqtitle{Lorenz-villkoret -- Retarderade potentialer}
Under Lorenz-villkoret erh{\aa}lls l{\"o}sningar till v{\aa}g\-ekva\-tion\-erna
f{\"o}r potentialerna som de retarderade potentialerna
$$
  \eqalign{
    \phi({\bf x},t)&={{1}\over{4\pi\varepsilon_0}}\iiint_{{\Bbb R}^3}
      {{\rho({\bf x}',t')}\over{|{\bf x}-{\bf x}'|}}\,dV',\cr
    {\bf A}({\bf x},t)&={{\mu_0}\over{4\pi}}\iiint_{{\Bbb R}^3}
      {{{\bf J}({\bf x}',t')}\over{|{\bf x}-{\bf x}'|}}\,dV',\cr
  }
$$
d{\"a}r $t'=t-|{\bf x}-{\bf x}'|/c$ {\"a}r den {\it retarderade tiden}
fr{\aa}n ${\bf x}'$ till ${\bf x}$.

\eqtitle{Coulomb-villkoret}



\vfill\eject

\eqsection{Vektoridentiteter}
\eqtitle{Notation}
\halign{#\hfil\ &#\hfil&\quad\hfil(#)\cr
  $dl$       & Skal{\"a}rt linjeelement l{\"a}ngs kurva & ${\rm m}$\cr
  $d{\bf l}$ & Riktat linjeelement $d{\bf l}={\bf e}_l dl$
               l{\"a}ngs kurva& ${\rm m}$\cr
  $dS$       & Skal{\"a}rt ytelement & ${\rm m}^2$\cr
  $d{\bf S}$ & Riktat ytelement $d{\bf S}={\bf e}_n dS$ & ${\rm m}^2$\cr
  $dV$       & Volymelement & ${\rm m}^3$\cr
}
\eqtitle{Trippelprodukter}
$$
  \eqalign{
    {\bf a}\cdot({\bf b}\times{\bf c})&=
      {\bf b}\cdot({\bf c}\times{\bf a})=
      {\bf c}\cdot({\bf a}\times{\bf b})\cr
    {\bf a}\times({\bf b}\times{\bf c})&=
      {\bf b}({\bf a}\cdot{\bf c})
        -{\bf c}({\bf a}\cdot{\bf b})
  }
$$
\eqtitle{Produktregler}
$$
  \eqalign{
    \nabla(fg) &= f\nabla g + g\nabla f\cr
    \nabla\cdot(f{\bf a})&=f\nabla\cdot{\bf a}+{\bf a}\cdot\nabla f\cr
    \nabla\times(f{\bf a}) &= f\nabla\times{\bf a} - {\bf a}\times\nabla f\cr
      \nabla({\bf a}\cdot{\bf b})&=
        {\bf a}\times\nabla\times{\bf b}
          +{\bf b}\times\nabla\times{\bf a}\cr
          &\hskip50pt
          +({\bf a}\cdot\nabla){\bf b}
          +({\bf b}\cdot\nabla){\bf a}\cr
    \nabla\cdot({\bf a}\times{\bf b})&=
      {\bf b}\cdot(\nabla\times{\bf a})-{\bf a}\cdot(\nabla\times{\bf b})\cr
      \nabla\times({\bf a}\times{\bf b})&=
        ({\bf b}\cdot\nabla){\bf a}
          -({\bf a}\cdot\nabla){\bf b}\cr
          &\hskip50pt
          +{\bf a}(\nabla\cdot{\bf b})
          -{\bf b}(\nabla\cdot{\bf a})
  }
$$
\eqtitle{``Tricket''}
$$
  \nabla{{1}\over{|{\bf x}-{\bf x}'|}}
    =-\nabla'{{1}\over{|{\bf x}-{\bf x}'|}}
    =-{{({\bf x}-{\bf x}')}\over{|{\bf x}-{\bf x}'|^3}}.
$$
\eqtitle{``Tricket II'' (f{\"o}r att visa att $\nabla\cdot B=0$)\sidx{Tricket
$\displaystyle\nabla{{1}\over{\char124}{\bf x}-{\bf x}'{\char124}}
=-{{({\bf x}-{\bf x}')}\over{{\char124}{\bf x}-{\bf x}'{\char124}^3}}$}}
$$
\nabla\times{{({\bf x}-{\bf x}')}\over{|{\bf x}-{\bf x}'|^3}}=0
$$
\eqtitle{Andraderivator}
$$
  \eqalign{
    \nabla\cdot(\nabla\times{\bf a})&=0\cr
    \nabla\times(\nabla f)&=0\cr
    \nabla\times(\nabla\times{\bf a})
      &=\nabla(\nabla\cdot{\bf a})-\nabla^2{\bf a}\cr
  }
$$
\eqtitle{Gauss teorem (``divergensteoremet'')\sidx{Gauss lag}}
$$
  \int_V(\nabla\cdot{\bf a})\,dV=\oiint_S{\bf a}\cdot d{\bf S}
$$
\eqtitle{Gauss teorem (rotations-versionen)}
$$
  \iiint_V(\nabla\times{\bf a})\,dV=\oiint_S d{\bf S}\times{\bf a}
$$
\eqtitle{Stokes teorem (Kelvin--Stokes teorem, ``rotationsteoremet'')
  \sidx{Stokes teorem}}
$$
  \iint_V(\nabla\times{\bf a})\cdot d{\bf S}=\oint_{\Gamma}{\bf a}\cdot d{\bf l}
$$
\eqtitle{Gradient-teoremet (analysens fundamentalsats)}
$$
  \int^{{\bf x}_b}_{{\bf x}_a}(\nabla f)\cdot d{\bf l}=f({\bf x}_b)-f({\bf x}_a)
$$
\eqtitle{Greens teorem}
$$
  \iiint_V(f\nabla^2g+(\nabla f)\cdot(\nabla g))\,dV
    =\oiint_S(f\nabla g)\cdot d{\bf S}
$$
\eqtitle{Greens andra teorem}
$$
  \iiint_V(f\nabla^2g-g\nabla^2f)\,dV
    =\oiint_S(f\nabla g-g\nabla f)\cdot d{\bf S}
$$
\eqtitle{Diverse anv{\"a}ndbara integraler i vektoranalys}
$$
  \iiint_V(\nabla f)\,dV=\oiint_S f d{\bf S}
$$
$$
  \iint_S(\nabla f)\times d{\bf S}=-\oint_{\Gamma}f\,d{\bf l}
$$
\vfill\eject
% \eqsection{Ofta anv{\"a}ndbara skal{\"a}ra integraler}
\eqsection{Skal{\"a}ra integraler}
$$
  \eqalign{
    \int&x^n\,dx={{1}\over{(n+1)}}x^{n+1},\quad n\ne-1\cr
    \int&{{dx}\over{x}}=\ln x\cr
    \int&\sqrt{x^2+a^2}\,dx={{1}\over{2}}
         [x\sqrt{x^2+a^2}\cr & \hskip80pt+a^2\ln(x+\sqrt{x^2+a^2})]\cr
    \int&{{dx}\over{\sqrt{x^2+a^2}}}=\ln(x+\sqrt{x^2+a^2})\cr
    \int&{{dx}\over{(x^2+a^2)^{3/2}}}={{x}\over{a^2\sqrt{x^2+a^2}}}\cr
    \int&{{dx}\over{\sqrt{a^2-x^2}}}=\arcsin(x/a)\cr
    \int&{{dx}\over{x^2+a^2}}={{1}\over{a}}\arctan(x/a)\cr
    \int&{{dx}\over{\cos^2x}}=\tan(x)\cr
    \int&{{dx}\over{\sin x}}=\ln|\tan(x/2)|\cr
    \int&\ln x\,dx=x\ln x - x\cr
  }
$$
\vfill\eject

\eqsection{Serieutvecklingar}
\eqtitle{Endimensionella Maclaurin-utvecklingar}
$$
  \eqalign{
    \sin(x)&=x-{{x^3}\over{3!}}+{{x^5}\over{5!}}-{{x^7}\over{7!}}+\ldots\cr
%            =\sum^{\infty}_{n=0}(-1)^n{{x^{2n+1}}\over{(2n+1)!}}\cr
    \cos(x)&=1-{{x^2}\over{2!}}+{{x^4}\over{4!}}-{{x^6}\over{6!}}+\ldots\cr
    {{1}\over{1+x}}&=1-x+x^2-x^3+\ldots\cr
    {{1}\over{1-x}}&=1+x+x^2+x^3+\ldots\cr
    \exp(x)&=1+x+{{x^2}\over{2}}+{{x^3}\over{3}}+\ldots\cr
    \ln(1+x)&=x-{{x^2}\over{2}}+{{x^3}\over{3}}-{{x^4}\over{4}}+\ldots\cr
  }
$$
\eqtitle{Tredimensionell Taylor-utveckling runt ${\bf x}={\bf x}_0$}
$$
  \eqalign{
  f({\bf x}) &= \sum^{\infty}_0 {{1}\over{n!}}
    [({\bf x}-{\bf x}_0)\cdot\nabla]^n f({\bf x}_0)\cr
  }
$$
\eqtitle{Tredimensionell Maclaurin-utveckling}
$$
  \eqalign{
  f({\bf x}) &= f({\bf 0})
  +\sum^{3}_{k=1} x'_k
     {{\partial f({\bf x}')}
       \over{\partial x'_k}}\bigg|_{{\bf x}'={\bf 0}}\cr
       &\qquad
  +{{1}\over{2}}\sum^{3}_{j=1}\sum^{3}_{k=1} x'_j x'_k
     {{\partial^2 f({\bf x}')}
       \over{\partial x'_j\partial x'_k}}\bigg|_{{\bf x}'={\bf 0}}
  +\ldots\cr
  }
$$
\vfill\eject

\eqsection{Koordinatsystem}
\eqtitle{Kartesiska koordinater ($x,y,z$)}
\halign{#\hfil\quad & #\hfil\cr
  Ortsvektor   & ${\bf x}={\bf e}_xx+{\bf e}_yy+{\bf e}_zz$ \cr
  Linjeelement & $d{\bf l}={\bf e}_xdx+{\bf e}_ydy+{\bf e}_zdz$ \cr
  Volymelement & $dV=dx\,dy\,dz$ \cr
}
Differentialoperatorer
$$
  \eqalign{
     &\nabla\phi={\bf e}_x{{\partial\phi}\over{\partial x}}
                +{\bf e}_y{{\partial\phi}\over{\partial y}}
                +{\bf e}_z{{\partial\phi}\over{\partial z}}\cr
     &\nabla^2\phi={{\partial^2\phi}\over{\partial x^2}}
                +{{\partial^2\phi}\over{\partial y^2}}
                +{{\partial^2\phi}\over{\partial z^2}}\cr
     &\nabla\cdot{\bf A}={{\partial A_x}\over{\partial x}}
                +{{\partial A_y}\over{\partial y}}
                +{{\partial A_z}\over{\partial z}}\cr
     &\nabla\times{\bf A}=
        {\bf e}_x\Big(
          {{\partial A_z}\over{\partial y}}-{{\partial A_y}\over{\partial z}}
        \Big)
       +{\bf e}_y\Big(
          {{\partial A_x}\over{\partial z}}-{{\partial A_z}\over{\partial x}}
        \Big)
        \cr & \hskip128pt
       +{\bf e}_z\Big(
          {{\partial A_y}\over{\partial x}}-{{\partial A_x}\over{\partial y}}
        \Big)\cr
  }
$$

\eqtitle{Cylindriska koordinater ($r,\varphi,z$)}
\halign{#\hfil\quad & #\hfil\cr
  Ortsvektor    & ${\bf x}={\bf e}_rr+{\bf e}_zz$ \cr
  Enhetsvektorer & ${\bf e}_r={\bf e}_x\cos\varphi+{\bf e}_y\sin\varphi$ \cr
              & ${\bf e}_{\varphi}=-{\bf e}_x\sin\varphi+{\bf e}_y\cos\varphi$ \cr
  Linjeelement  & $d{\bf l}={\bf e}_r\,dr
                     +{\bf e}_{\varphi}r\,d\varphi
                     +{\bf e}_z\,dz$ \cr
  Volymelement  & $dV=r\,dr\,d\varphi\,dz$ \cr
}
Differentialoperatorer
$$
  \eqalign{
     &\nabla\phi={\bf e}_r{{\partial\phi}\over{\partial r}}
                +{\bf e}_{\varphi}{{1}\over{r}}
                  {{\partial\phi}\over{\partial\varphi}}
                +{\bf e}_z{{\partial\phi}\over{\partial z}}\cr
     &\nabla^2\phi={{1}\over{r}}{{\partial}\over{\partial r}}
        \Big(r{{\partial\phi}\over{\partial r}}\Big)
                +{{1}\over{r^2}}{{\partial^2\phi}\over{\partial\varphi^2}}
                +{{\partial^2\phi}\over{\partial z^2}}\cr
     &\nabla\cdot{\bf A}={{1}\over{r}}{{\partial}\over{\partial r}}(rA_r)
                +{{1}\over{r}}{{\partial A_{\varphi}}\over{\partial\varphi}}
                +{{\partial A_z}\over{\partial z}}\cr
     &\nabla\times{\bf A}=
        {\bf e}_r\Big(
          {{1}\over{r}}{{\partial A_z}\over{\partial\varphi}}
            -{{\partial A_{\varphi}}\over{\partial z}}
        \Big)
       +{\bf e}_{\varphi}\Big(
          {{\partial A_r}\over{\partial z}}-{{\partial A_z}\over{\partial r}}
        \Big)
        \cr & \hskip110pt
       +{\bf e}_z{{1}\over{r}}\Big(
          {{\partial}\over{\partial r}}(rA_{\varphi})
            -{{\partial A_r}\over{\partial\varphi}}
        \Big)\cr
  }
$$
\vfill\eject

\eqtitle{Sf{\"a}riska koordinater ($r,\varphi,\theta$)}
\halign{#\hfil\quad & #\hfil\cr
  Ortsvektor    & ${\bf x}={\bf e}_rr$ \cr
}
\halign{#\hfil\quad & #\hfil\cr
  Enhetsvektorer & \cr
  \hskip36pt${\bf e}_r={\bf e}_x\sin\theta\cos\varphi
                        +{\bf e}_y\sin\theta\sin\varphi
                        +{\bf e}_z\cos\theta$ & \cr
  \hskip36pt${\bf e}_{\theta}={\bf e}_x\cos\theta\cos\varphi
                        +{\bf e}_y\cos\theta\sin\varphi
                        -{\bf e}_z\sin\theta$ & \cr
  \hskip36pt${\bf e}_{\varphi}=-{\bf e}_x\sin\varphi+{\bf e}_y\cos\varphi$ & \cr
}
\halign{#\hfil\quad & #\hfil\cr
  Linjeelement  & $d{\bf l}={\bf e}_r\,dr
                    +{\bf e}_{\varphi}r\sin\theta\,d\varphi
                    +{\bf e}_{\vartheta}r\,d\theta$ \cr
  Volymelement  & $dV=r^2\sin\theta\,dr\,d\varphi\,d\theta$ \cr
}
Differentialoperatorer
$$
  \eqalign{
     &\nabla\phi={\bf e}_r{{\partial\phi}\over{\partial r}}
                +{\bf e}_{\varphi}{{1}\over{r\sin\theta}}
                   {{\partial\phi}\over{\partial\varphi}}
                +{\bf e}_{\vartheta}{{1}\over{r}}
                   {{\partial\phi}\over{\partial\theta}}\cr
     &\nabla^2\phi={{1}\over{r^2}}{{\partial}\over{\partial r}}
         \Big(r^2{{\partial\phi}\over{\partial r}}\Big)
       +{{1}\over{r^2\sin^2\theta}}{{\partial^2\phi}\over{\partial\varphi^2}}
          \cr & \hskip90pt
       +{{1}\over{r^2\sin\theta}}{{\partial}\over{\partial\theta}}
           \Big(\sin\theta{{\partial\phi}\over{\partial\theta}}\Big)\cr
     &\nabla\cdot{\bf A}={{1}\over{r^2}}{{\partial}\over{\partial r}}(r^2A_r)
                +{{1}\over{r\sin\theta}}
                   {{\partial A_{\varphi}}\over{\partial\varphi}}
        \cr & \hskip110pt
                +{{1}\over{r\sin\theta}}
                   {{\partial}\over{\partial\theta}}(A_{\theta}\sin\theta)\cr
     &\nabla\times{\bf A}=
        {\bf e}_r{{1}\over{r\sin\theta}}\Big(
          {{\partial}\over{\partial\theta}}(A_{\varphi}\sin\theta)
            -{{\partial A_{\theta}}\over{\partial\varphi}}
        \Big)
        \cr & \hskip90pt
       +{\bf e}_{\varphi}{{1}\over{r}}\Big(
          {{\partial}\over{\partial r}}(rA_{\theta})
              -{{\partial A_r}\over{\partial\theta}}
        \Big)
        \cr & \hskip90pt
       +{\bf e}_{\theta}{{1}\over{r}}\Big(
          {{1}\over{\sin\theta}}{{\partial A_r}\over{\partial\varphi}}
            -{{\partial}\over{\partial r}}(rA_{\varphi})
        \Big)\cr
  }
$$
\vfill
\supereject
\if R\lr \null\vfill\eject\fi}
%%%%%%%%%%%%%% Back to normal single-column %%%%%%%%%%%%%%
~\cleardoublepage

%
% The World Map of Classical Electromagnetics.
%
%
% File: teach/elmagii/worldmap/src/worldmap-src.tex [plain TeX code]
% Github: https://github.com/elmagii/worldmap/
% Last change: February 20, 2026
%
% World map of the equations and relations of classical electromagnetics,
% forming the basis of the theory in the course ``Elektromagnetism II,
% 1TE626'', held 2022--2026 at Uppsala University, Sweden.
%
% Copyright (C) 2026, Fredrik Jonsson, under Gnu General Public
% License (GPL) v3. See the enclosed LICENSE for details.
%
% This program is free software: you can redistribute it and/or modify
% it under the terms of the GNU General Public License as published by
% the Free Software Foundation, either version 3 of the License, or
% (at your option) any later version.
%
% This program is distributed in the hope that it will be useful,
% but WITHOUT ANY WARRANTY; without even the implied warranty of
% MERCHANTABILITY or FITNESS FOR A PARTICULAR PURPOSE.  See the
% GNU General Public License for more details.
%
% You should have received a copy of the GNU General Public License
% along with this program.  If not, see <https://www.gnu.org/licenses/>.
%
~\vskip-4.0cm\presection{V{\"a}rldskarta i elektromagnetisk f{\"a}ltteori}
\vskip-1cm
\sidx{V{\"a}rldskarta i elektromagnetisk f{\"a}ltteori}
\sidx{Elektromagnetisk f{\"a}ltteori}[V{\"a}rldskarta]
\sidx{Elektromagnetism}[V{\"a}rldskarta]
%\noindent
%Elektromagnetism som {\"a}mne {\"a}r proppfyllt med celebriteter inom den
%klassiska fysiken, och det {\"a}r alltid trevligt att ha ett ansikte associerat
%med namnet. Detta portr{\"a}ttgalleri t{\"a}cker de prim{\"a}ra uppt{\"a}ckarna
%inom elektrostatik, magnetostatik och elektrodynamik.
%\medskip
\epsfig{../worldmap/figs/worldmap.1}
\cleardoublepage

%
% Portrait gallery in electromagnetism.
%
%
% File: teach/elmagii/gallery/src/portrait.tex [plain TeX code]
% Github: https://github.com/elmagii/gallery/
% Last change: January 22, 2026
%
% Macros for the typesetting of the gallery of primary physicists who
% formed the theory of electomagnetism through history. These physicists
% and mathematicians essentially formed the theory which we are describing
% in the course ``Elektromagnetism II, 1TE626'', held 2022--2026 at Uppsala
% University, Sweden.
%
% Copyright (C) 2026, Fredrik Jonsson, under Gnu General Public
% License (GPL) v3. See the enclosed LICENSE for details.
%
% This program is free software: you can redistribute it and/or modify
% it under the terms of the GNU General Public License as published by
% the Free Software Foundation, either version 3 of the License, or
% (at your option) any later version.
%
% This program is distributed in the hope that it will be useful,
% but WITHOUT ANY WARRANTY; without even the implied warranty of
% MERCHANTABILITY or FITNESS FOR A PARTICULAR PURPOSE.  See the
% GNU General Public License for more details.
%
% You should have received a copy of the GNU General Public License
% along with this program.  If not, see <https://www.gnu.org/licenses/>.
%

%
% Define the 'boxit' macro from D.E. Knuths "The TeXbook, Exercise 21.3.
% Set the parameter \boxedtrue or \boxedfalse in order to have boxed
% portrait generation.
%
\def\boxit#1{\vbox{\hrule\hbox{\vrule\kern1pt
  \vbox{\kern1pt#1\kern1pt}\kern1pt\vrule}\hrule}}
\newif\ifboxed
\boxedfalse   % \boxedtrue or \boxedfalse

%
% Define the macro which "typographically atomizes" the individual portraits.
%
\newdimen\portraitxsize  \portraitxsize=200pt
\newdimen\portraitysize  \portraitysize=300pt
\newdimen\portraitxmargin \portraitxmargin=36pt
\newdimen\portraitymargin \portraitymargin=14pt

%
% Define a macro for the display of an individual image, primarily using
% argument #1 for the image but if this is found to be missing instead
% using argument #2.
%
\newread\imgtest
\def\portraitimage#1#2{%
  \openin\imgtest=#1
  \ifeof\imgtest
    % File #1 does not exist -> Try using #2 instead
    \hbox to \portraitxsize{\epsfysize=164pt\hfil\epsfbox{#2}\hfil}%
  \else
    % File exists
    \closein\imgtest
    \hbox to \portraitxsize{\epsfysize=164pt\hfil\epsfbox{#1}\hfil}%
  \fi
}

%
% Define a macro for the typesetting of an individual portrait.
%   Argument #1: Primary image file (EPS) to be used.
%   Argument #2: Secondary image file (EPS) to be used if #1 is missing.
%   Argument #3: Name to be displayed in bold beneath the image.
%   Argument #4: Year span (yyyy-yyyy) to be displayed beneath the name.
%   Argument #5: Descriptive text to be displayed beneath the Year span.
%
\def\portrait#1#2#3#4#5{%
  \ifboxed
    \boxit{%
      \vbox{%
        \boxit{%
          \portraitimage{#1}{#2}%
        }%
        \hbox{\vbox{\hsize=\portraitxsize\parindent=0pt\hfil{\bf#3}\hfil}}%
        \hbox{\vbox{\hsize=\portraitxsize\parindent=0pt\hfil{(#4)}\hfil}}%
        \hbox{\vbox{\hsize=\portraitxsize\parindent=0pt{#5}}}%
      }%
    }%
  \else
    \vbox{%
      \portraitimage{#1}{#2}%
      \hbox{\vbox{\hsize=\portraitxsize\parindent=0pt\hfil{\bf#3}\hfil}}%
      \hbox{\vbox{\hsize=\portraitxsize\parindent=0pt\hfil{(#4)}\hfil}}%
      \hbox{\vbox{\hsize=\portraitxsize\parindent=0pt{#5}}}%
    }%
  \fi
}

%
% Define a macro for the typesetting of a single row containing a
% pair of portraits.
%
\def\portraitpair#1#2{%
  \ifboxed
  \hbox{%
    \hskip14pt%
    \boxit{\hbox{\vbox to \portraitysize{\hbox{#1}}}}%
    \hskip\portraitxmargin%
    \boxit{\hbox{\vbox to \portraitysize{\hbox{#2}}}}%
  }%
  \else
  \hbox{%
    \hskip14pt%
    \hbox{\vbox to \portraitysize{\hbox{#1}}}%
    \hskip\portraitxmargin%
    \hbox{\vbox to \portraitysize{\hbox{#2}}}%
  }%
  \fi
}

%
% Define a macro for the typesetting of a single page of four portraits,
% composed of two rows each containing a pair of portraits.
%
\def\portraitpage#1#2#3#4{%
  \portraitpair{#1}{#2}%
  \vskip\portraitymargin%
  \portraitpair{#3}{#4}%
  \vfill\eject%
}
%
% File: teach/elmagii/gallery/gallery-src.tex [plain TeX code]
% Github: https://github.com/elmagii/gallery/
% Last change: January 22, 2026
%
% Gallery of primary physicists who formed the theory of electomagnetism
% through history. These physicists and mathematicians essentially formed
% the theory which we are describing in the course ``Elektromagnetism II,
% 1TE626'', held 2022--2026 at Uppsala University, Sweden.
%
% Copyright (C) 2026, Fredrik Jonsson, under Gnu General Public
% License (GPL) v3. See the enclosed LICENSE for details.
%
% This program is free software: you can redistribute it and/or modify
% it under the terms of the GNU General Public License as published by
% the Free Software Foundation, either version 3 of the License, or
% (at your option) any later version.
%
% This program is distributed in the hope that it will be useful,
% but WITHOUT ANY WARRANTY; without even the implied warranty of
% MERCHANTABILITY or FITNESS FOR A PARTICULAR PURPOSE.  See the
% GNU General Public License for more details.
%
% You should have received a copy of the GNU General Public License
% along with this program.  If not, see <https://www.gnu.org/licenses/>.
%

%
% Jean le Rond d'Alembert (1717--1783)
%
\def\dalembert{%
  \portrait{images/jean_le_rond_dalembert_scaled_bw.eps}%
           {../gallery/images/jean_le_rond_dalembert_scaled_bw.eps}%
    {Jean le Rond d'Alembert}%
    {1717--1783}%
    {\ifnum\lang=\swedish
      {Fransk matematiker, fysiker och musikteore\-ti\-ker. I denna kurs
      anv{\"a}nder vi d'Alembert's metod f{\"o}r att visa att v{\aa}gor
      som f{\"o}ljer
      $$
        \abovedisplayskip=3pt
        \belowdisplayskip=3pt
        {{\partial^2 E(z,t)}\over{\partial z^2}}
          -{{1}\over{c^2}}{{\partial^2 E(z,t)}\over{\partial t^2}} = 0.
      $$
      ger l{\"o}sningar p{\aa} formen
      $$
        \abovedisplayskip=3pt
        \belowdisplayskip=0pt
        E(z,t)=f(z-ct)+g(z+ct).
      $$}
    \else
      {French mathematician, physicist, and music
      theorist. In this course we use d'Alembert's method to show that
      waves following
      $$
        \abovedisplayskip=3pt
        \belowdisplayskip=3pt
        {{\partial^2 E(z,t)}\over{\partial z^2}}
          -{{1}\over{c^2}}{{\partial^2 E(z,t)}\over{\partial t^2}} = 0.
      $$
      gives solutions of the form
      $$
        \abovedisplayskip=3pt
        \belowdisplayskip=0pt
        E(z,t)=f(z-ct)+g(z+ct).
      $$}
    \fi}%
    \sidx{d'Alembert, Jean le Rond (1717--1783)}
}

%
% Andre-Marie Ampere (1775--1836)
%
\def\ampere{%
  \portrait{images/andre_ampere_scaled_bw.eps}%
           {../gallery/images/andre_ampere_scaled_bw.eps}%
    {Andr\'e-Marie Amp\`ere}%
    {1775--1836}%
    {\ifnum\lang=\swedish
      {Fransk fysiker och matematiker. En av de prim{\"a}ra grundarna av
      klassisk elektromagne\-tism, vilken Amp\`ere betecknade elektrodyna\-mik.
      Amp\`eres lag f{\"o}r statiska magnetf{\"a}lt,
      $$
        \abovedisplayskip=3pt
        \belowdisplayskip=3pt
        \nabla\times{\bf B}=\mu_0{\bf J},
      $$
      modifierades senare av Maxwell att inkludera
      f{\"o}rskjutningsstr{\"o}mmen.}
    \else
      {French physicist and mathematician. One of the principal
      founders of classical electromagnetism, which he referred to as
      electrodynamics. Amp\`ere's law for static magnetic fields,
      $$
        \abovedisplayskip=3pt
        \belowdisplayskip=3pt
        \nabla\times{\bf B}=\mu_0{\bf J},
      $$
      was later modified by Maxwell to include the displacement current.}
    \fi}%
    \sidx{Amp\`ere, Andr\'e-Marie (1775--1836)}
    \sidx{Amp\`eres lag}[F{\"or}r statiska magnetiska f{\"a}lt]
}

%
% Jean-Baptiste Biot (1774--1862)
%
\def\biot{%
  \portrait{images/jean-baptiste_biot_scaled_bw.eps}%
           {../gallery/images/jean-baptiste_biot_scaled_bw.eps}%
    {Jean-Baptiste Biot}%
    {1774--1862}%
    {\ifnum\lang=\swedish
      {Fransk fysiker, astronom och matematiker som med F\'elix Savart
      formulerade Biot--Savarts lag f{\"o}r magnetostatik,
      $$
        \abovedisplayskip=3pt
        \belowdisplayskip=3pt
        {\bf B}({\bf x})={{\mu_0 I}\over{4\pi}}
          \int{{d{\bf l}'\times({\bf x}-{\bf x}')}\over{|{\bf x}-{\bf x}'|^3}},
      $$
      samt studerade polarisation hos elektromagnetiska v{\aa}gor.}
    \else
      {French physicist, astronomer, and mathematician who co-discovered
      the Biot--Savart law of magnetostatics with F\'elix Savart,
      $$
        \abovedisplayskip=3pt
        \belowdisplayskip=3pt
        {\bf B}({\bf x})={{\mu_0 I}\over{4\pi}}
          \int{{d{\bf l}'\times({\bf x}-{\bf x}')}\over{|{\bf x}-{\bf x}'|^3}},
      $$
      and studied the polarization of light.}
    \fi}%
    \sidx{Biot, Jean-Baptiste (1774--1862)}
    \sidx{Biot--Savarts lag}
}

%
% Louis de Broglie (1892--1987)
%
\def\debroglie{%
  \portrait{images/louis_de_broglie_scaled_bw.eps}%
           {../gallery/images/louis_de_broglie_scaled_bw.eps}%
    {Louis de Broglie}%
    {1892--1987}%
    {\ifnum\lang=\swedish
      {7th Duc de Broglie. Fransk teoretisk fysiker och aristokrat. I sin
      doktorsavhandling 1924 postulerade han att elektroner s{\aa}v{\"a}l
      som all annan materia har v{\aa}gegenskaper med v{\aa}g\-l{\"a}ngd
      $\lambda=h/p$, den s{\aa} kallade v{\aa}g--partikel\-dualiteten.
      De Broglie erh{\"o}ll 1929 Nobelpriset i Fysik efter att hans hypotes
      1927 bekr{\"a}ftats experimentellt av 
      C.~Davisson, L.~Germer och G.~P.~Thomson.}
    \else
      {7th Duke de Broglie. French theoretical physicist and aristocrat.
      In his doctoral dissertation in 1924, he postulated that electrons,
      as well as all other matter, possess wave properties with wavelength
      $\lambda=h/p$, the so-called wave–particle duality. De Broglie received
      the 1929 Nobel Prize in Physics after his hypothesis was experimentally
      confirmed in 1927 by George Paget Thomson, Clinton Davisson och Lester
      Germer.}
    \fi}%
    \sidx{de Broglie, Louis (1892--1987)}
}

%
% Charles-Augustin de Coulomb (1736--1806)
%
\def\coulomb{%
  \portrait{images/charles_de_coulomb_scaled_bw.eps}%
           {../gallery/images/charles_de_coulomb_scaled_bw.eps}%
    {Charles-Augustin de Coulomb}%
    {1736--1806}%
    {\ifnum\lang=\swedish
      {Fransk officer, ingenj{\"o}r och fysiker. Formule\-ra\-de 1785 den
      fysikaliska lag som beskriver elektrostatisk kraft som attraktion
      eller repulsion,
      $$
        \abovedisplayskip=3pt
        \belowdisplayskip=3pt
        {\bf F}={{qq'}\over{4\pi\varepsilon_0}}
          {{({\bf x}-{\bf x}')}\over{|{\bf x}-{\bf x}'|^3}},
      $$
      som l{\"a}gger grunden f{\"o}r konceptet och definitionen av det
      elektriska f{\"a}ltet.}
    \else
      {French officer, engineer, and physicist. In 1785 discovered of the law
      describing the electrostatic force of attraction and repulsion,
      $$
        \abovedisplayskip=3pt
        \belowdisplayskip=3pt
        {\bf F}={{qq'}\over{4\pi\varepsilon_0}}
          {{({\bf x}-{\bf x}')}\over{|{\bf x}-{\bf x}'|^3}},
      $$
      forming the basis for the concept and definition of the electric field.}
    \fi}%
    \sidx{de Coulomb, Charles-Augustin (1736--1806)}
    \sidx{Coulombs kraftlag}
}

%
% Michael Faraday (1791--1867)
%
\def\faraday{%
  \portrait{images/michael_faraday_scaled_bw.eps}%
           {../gallery/images/michael_faraday_scaled_bw.eps}%
    {Michael Faraday}%
    {1791--1867}%
    {\ifnum\lang=\swedish
      {Engelsk kemist och fysiker. Formulerade prin\-cip\-en f{\"o}r
      elektromagnetisk induktion, diamagnetism och elektrolys. I denna
      kurs anv{\"a}nder vi frekvent Faradays induktionslag
      $$
        \abovedisplayskip=3pt
        \belowdisplayskip=3pt
        {\cal E}= -{{d\Phi_{\rm M}}\over{dt}}
          \quad\Leftrightarrow\quad
          \nabla\times{\bf E}=-{{\partial{\bf B}}\over{\partial t}},
      $$
      vilken kan h{\"a}rledas fr{\aa}n Lorentz kraftlag, obe\-ro\-ende av
      Coulombs eller Biot--Savarts lag.}
    \else
      {English chemist and physicist. Discovered the principles underlying
      electromagnetic induction, diamagnetism, and electrolysis. In this
      course, we make frequent use of Faraday's law of induction
      $$
        \abovedisplayskip=3pt
        \belowdisplayskip=3pt
        {\cal E}= -{{d\Phi_{\rm M}}\over{dt}}
          \quad\Leftrightarrow\quad
          \nabla\times{\bf E}=-{{\partial{\bf B}}\over{\partial t}},
      $$
      which can be derived from the Lorentz force, independently of
      Coulombs' eller Biot--Savart's law.}
    \fi}%
    \sidx{Faraday, Michael (1791--1867)}
    \sidx{Faradays induktionslag}
    \sidx{Faradays lag}[Differentialform]
}

%
% Charles Francois de Cisternay du Fay (1698--1739)
%
\def\dufay{%
  \portrait{images/charles_francois_de_cisternay_du_fay_scaled_bw.eps}%
           {../gallery/images/charles_francois_de_cisternay_du_fay_scaled_bw.eps}%
    {Charles Fran\c{c}ois de Cisternay du Fay}%
    {1698--1739}%
    {\ifnum\lang=\swedish
      {Fransk kemist och f{\"o}rest{\aa}ndare f{\"o}r {\it Jardin du Roi}
      i Paris. Uppt{\"a}ckte att vissa elektriskt laddade objekt attraherade
      varandra medan andra repellerade, och formulerade hypotesen att det
      fanns tv{\aa} typer av ``elektrisk v{\"a}tska'' som han betecknade
      ``vitreous'' and ``resinous'' (senare betecknat som respektive positiv
      och negativ elektrisk ladd\-ning).}
    \else
      {French chemist and superintendent of {\it Jardin du Roi} in Paris.
      Discovered that some charged items would repel while some would
      attract and posited the existence of two types of electricity,
      named ``vitreous'' and ``resinous'' (later to be known as
      positive and negative charge, respectively).}
    \fi}%
    \sidx{de Cisternay du Fay, Fran\c{c}ois (1698--1739)}
    \sidx{de Cisternay du Fay, Fran\c{c}ois (1698--1739)}[{{\it Sixi\`eme
      m\'emoire sur l'\'electricit\'e (1734)}}]
}

%
% Richard Feynman (1918--1988)
%
\def\feynman{%
  \portrait{images/richard_feynman_drums_scaled_bw.eps}%
           {../gallery/images/richard_feynman_drums_scaled_bw.eps}%
    {Richard Feynman}%
    {1918--1988}%
    {\ifnum\lang=\swedish
      {Amerikansk teoretisk fysiker vid California Institute of Technology
      (Caltech). Feynman formulerade meto\-di\-ken f{\"o}r v{\"a}gintegraler
      ({\it path integrals}) inom kvantmekanik samt grundl{\"a}gg\-an\-de
      teori f{\"o}r kvantelektrodynamik, f{\"o}r vilka han 1965 erh{\"o}ll
      Nobelpriset i Fysik tillsammans med Julian Schwing\-er och
      Shin'ichir{\=o} Tomonaga.}
    \else
      {American theoretical physicist. Developed the path integral formulation
      of quantum mechanics and the development of the theory of quantum
      electrodynamics. For his contributions to the development of quantum
      electrodynamics, Feynman in 1965 received the Nobel Prize in Physics
      along with Julian Schwinger and Shin'ichir{\=o} Tomonaga.}
    \fi}%
    \sidx{Feynman, Richard (1918--1988)}
    \sidx{Elektrodynamik}[Kvantelektrodynamik]
}

%
% Benjamin Franklin (1706--1790)
%
\def\franklin{%
  \portrait{images/benjamin_franklin_scaled_bw.eps}%
           {../gallery/images/benjamin_franklin_scaled_bw.eps}%
    {Benjamin Franklin}%
    {1706--1790}%
    {\ifnum\lang=\swedish
      {Amerikansk politiker, vetenskapsman och universalgeni ({\it polymath}).
      En av F{\"o}renta staternas grundlagsf{\"a}der och undertecknare av
      sj{\"a}lv\-st{\"a}ndighetsf{\"o}rklaringen. Utf{\"o}rde experiment
      som f{\"o}reslog att du Fays ``vitre{\"o}sa'' och ``resin\-{\"o}sa''
      elektricitet inte var olika typer av ``elektrisk v{\"a}tska'', utan
      av samma slag, betecknade som positiv och negativ. F{\"o}rst med att
      beskriva principen om laddningens bevarande.}
    \else
      {American polymath. One of the Founding Fathers of the United States
      and signer of the Declaration of Independence. Conducted experiments
      proposing that du Fay's ``vitreous'' and ``resinous'' electricity were
      not different types of ``electrical fluid'', but of same type, labeled
      as positive and negative. First to discover the principle of conservation
      of charge.}
    \fi}%
    \sidx{Franklin, Benjamin (1706--1790)}
}

%
% Carl Friedrich Gauss (1777--1855)
%
\def\gauss{%
  \portrait{images/carl_friedrich_gauss_scaled_bw.eps}%
           {../gallery/images/carl_friedrich_gauss_scaled_bw.eps}%
    {Carl Friedrich Gauss}%
    {1777--1855}%
    {\ifnum\lang=\swedish
      {Tysk matematiker och astronom. F{\"o}rest{\aa}nd\-are f{\"o}r
      G{\"o}ttingens Observatorium och professor i astronomi. I denna
      kurs anv{\"a}nder vi frek\-vent Gauss teorem
      $$
        \abovedisplayskip=3pt
        \belowdisplayskip=3pt
        \def\oiint{\mathop{\int\kern-8pt\int\kern-13.2pt{\bigcirc}}}
        \def\iiint{\mathop{\int\kern-8pt\int\kern-8pt\int}}
        \iiint(\nabla\cdot{\bf a})\,dV=\oiint{\bf a}\cdot d{\bf S}.
      $$}
    \else
      {German mathematician and astronomer. Director of the G{\"o}ttingen
      Observatory in Germany and professor of astronomy. We are in this
      course primarily concerned with Gauss' theorem
      $$
        \abovedisplayskip=3pt
        \belowdisplayskip=3pt
        \def\oiint{\mathop{\int\kern-8pt\int\kern-13.2pt{\bigcirc}}}
        \def\iiint{\mathop{\int\kern-8pt\int\kern-8pt\int}}
        \iiint(\nabla\cdot{\bf a})\,dV=\oiint{\bf a}\cdot d{\bf S}.
      $$}
    \fi}%
    \sidx{Gauss, Carl Friedrich (1777--1855)}
    \sidx{Gauss teorem}
}

%
% William Gilbert (1544--1603)
%
\def\gilbert{%
  \portrait{images/william_gilbert_scaled_bw.eps}%
           {../gallery/images/william_gilbert_scaled_bw.eps}%
    {William Gilbert}%
    {1544--1603}%
    {\ifnum\lang=\swedish
      {Engelsk l{\"a}kare, fysiker och naturfilosof. {\"A}r idag fr{\"a}mst
      k{\"a}nd f{\"o}r sin bok  {\it De Magnete} ({\it On Magnetism}, 1600),
      d{\"a}r elektrostatisk attraktion f{\"o}r f{\"o}rsta g{\aa}ngen
      separerades fr{\aa}n magnetism. Gilbert-modellen f{\"o}r det magnetiska
      dipolmomentet, med en ``nordpol'' och ``sydpol'', {\"a}r uppkallad efter
      honom.}
    \else
      {English physician, physicist and natural philo\-sopher. He is
      remembered today largely for his book {\it De Magnete} ({\it On
      Magnetism}, 1600), where electrostatic attraction was separated
      from magnetism. The Gilbert model of magnetic dipole moment and
      magnetization, with a ``north'' and ``south'' pole is named after
      him.}
    \fi}%
    \sidx{Gilbert, William (1544--1603)}
    \sidx{Gilbert, William (1544--1603)}[{\it On Magnetism} (1600)]
}

%
% Oliver Heaviside (1850--1925)
%
\def\heaviside{%
  \portrait{images/oliver_heaviside_scaled_bw.eps}%
           {../gallery/images/oliver_heaviside_scaled_bw.eps}%
    {Oliver Heaviside}%
    {1850--1925}%
    {\ifnum\lang=\swedish
      {Brittisk matematiker och elektroingenj{\"o}r som utvecklade
      tekniker f{\"o}r att l{\"o}sa differentialekvationer, ekvivalent
      till Laplace-transformen, utvecklade den moderna vektoranalysen
      och formulerade Maxwells ekvationer p{\aa} den kocisa och kompakta
      form som idag anv{\"a}nds. Uppfann koaxialkabeln och {\"a}r namnet
      bakom Heaviside-funktionen $H(x)$.}
    \else
      {British mathematician and electrical engineer who invented a new
      technique for solving differential equations (equivalent to the
      Laplace transform), independently developed modern vector calculus,
      and rewrote Maxwell's equations in the concise and compact form as
      used today. Invented the coaxial cable and is the name behind the
      Heaviside function $H(x)$.}
    \fi}%
    \sidx{Heaviside, Oliver (1850--1925)}
    \sidx{Maxwells ekvationer}
}

%
% William Thomson, 1st Baron Kelvin (1824--1907)
%
\def\kelvin{%
  \portrait{images/lord_kelvin_scaled_bw.eps}%
           {../gallery/images/lord_kelvin_scaled_bw.eps}%
    {Lord Kelvin}%
    {1824--1907}%
    {\ifnum\lang=\swedish
      {William Thomson, 1:e Baron Kelvin. Brittisk ingenj{\"o}r, matematiker
      och fysiker. Kom fram till att absoluta nollpunkten motsvarar cirka
      $-273.15^{\circ}\ {\rm C}$. I denna kurs anv{\"a}nder vi frekvent
      Kelvin--Stokes teorem (``Stokes teorem''),
      $$
        \abovedisplayskip=3pt
        \belowdisplayskip=3pt
        \def\iint{\mathop{\int\kern-8pt\int}}
        \iint(\nabla\times{\bf a})\cdot d{\bf S}
          =\oint{\bf a}\cdot d{\bf l},
      $$
      efter Lord Kelvin och George Stokes.}
    \else
      {William Thomson, 1st Baron Kelvin. British engineer, mathematician
      and physicist. Determined the value of absolute zero temperature as
      approximately $-273.15^{\circ}\ {\rm C}$. In this course we use the
      Kelvin--Stokes theorem,
      $$
        \abovedisplayskip=3pt
        \belowdisplayskip=3pt
        \def\iint{\mathop{\int\kern-8pt\int}}
        \iint(\nabla\times{\bf a})\cdot d{\bf S}
          =\oint{\bf a}\cdot d{\bf l},
      $$
      after Lord Kelvin and George Stokes, being the fundamental theorem
      for curls.}
    \fi}%
    \sidx{Kelvin, Lord (1824--1907)}[William Thomson, 1st Baron Kelvin]
    \sidx{Kelvin--Stokes teorem}[Se Stokes teorem]
}

%
% Pierre-Simon Laplace (1749--1827)
%
\def\laplace{%
  \portrait{images/pierre-simon_de_laplace_scaled_bw.eps}%
           {../gallery/images/pierre-simon_de_laplace_scaled_bw.eps}%
    {Pierre-Simon Laplace}%
    {1749--1827}%
    {\ifnum\lang=\swedish
      {Franskt universalgeni ({\it polymath}) vars arbeten haft en fundamental
      betydelse inom fysik, astronomi, matematik, ingenj{\"o}rskonst, statistik
      och filosofi. I denna kurs behandlar vi prim{\"a}rt Laplaces ekvation
      $$
        \abovedisplayskip=3pt
        \belowdisplayskip=3pt
        \nabla^2\phi=0
      $$
      f{\"o}r den skal{\"a}ra elektrostatiska potentialen $\phi$.}
    \else
      {French polymath, a scholar whose work has been instrumental in the
      fields of physics, astronomy, mathematics, engineering, statistics,
      and philosophy. In this course we are primarily concerned with the
      Laplace equation
      $$
        \abovedisplayskip=3pt
        \belowdisplayskip=3pt
        \nabla^2\phi=0
      $$
      for the scalar electrostatic potential $\phi$.}
    \fi}%
    \sidx{Laplace, Pierre-Simon (1749--1827)}
    \sidx{Laplaces ekvation}
    \sidx{Laplace-operatorn $\nabla^2$}
}

%
% Emil Lenz (1804--1865)
%
\def\lenz{%
  \portrait{images/emil_lenz_scaled_bw.eps}%
           {../gallery/images/emil_lenz_scaled_bw.eps}%
    {Emil Lenz}%
    {1804--1865}%
    {\ifnum\lang=\swedish
      {Rysk fysiker av baltisk-tysk h{\"a}rkomst som {\"a}r mest
      k{\"a}nd f{\"o}r att ha formulerat Lenz lag inom elektrodynamiken
      {\aa}r 1834. I viss mening kan man associera Lenz med minustecknet
      i Faradays induktionslag,
      $$
        \abovedisplayskip=3pt
        \belowdisplayskip=3pt
        {\cal E}= -{{d\Phi_{\rm M}}\over{dt}}
          \quad\Leftrightarrow\quad
          \nabla\times{\bf E}=-{{\partial{\bf B}}\over{\partial t}}.
      $$}
    \else
      {Russian physicist of Baltic German descent who is most noted for
      formulating Lenz's law in electrodynamics in 1834. In some sense, we
      can associate Lenz with the minus sign in Faraday's law of induction,
      $$
        \abovedisplayskip=3pt
        \belowdisplayskip=3pt
        {\cal E}= -{{d\Phi_{\rm M}}\over{dt}}
          \quad\Leftrightarrow\quad
          \nabla\times{\bf E}=-{{\partial{\bf B}}\over{\partial t}}.
      $$}
    \fi}%
    \sidx{Lenz, Emil (1804--1865)}
    \sidx{Lenz lag}
}

%
% Hendrik Antoon Lorentz (1853--1928)
%
\def\lorentz{%
  \portrait{images/hendrik_lorentz_scaled_bw.eps}%
           {../gallery/images/hendrik_lorentz_scaled_bw.eps}%
    {Hendrik Antoon Lorentz}%
    {1853--1928}%
    {\ifnum\lang=\swedish
      {Nederl{\"a}ndsk teoretisk fysiker som {\aa}r 1902 delade Nobelpriset
      i fysik med Pieter Zeeman f{\"o}r uppt{\"a}ckten och den teoretiska
      f{\"o}rklaringen av Zeemaneffekten.
      H{\"a}rledde Lorentz-trans\-forma\-tionen inom den speciella
      relativitetsteorin samt Lorentzkraften
      $$
        \abovedisplayskip=3pt
        \belowdisplayskip=3pt
        {\bf F}=q\big({\bf E}+({\bf v}\times{\bf B})\big),
      $$
      som verkar p{\aa} en r{\"o}rlig laddad partikel.}
    \else
      {Dutch theoretical physicist who in 1902 shared the Nobel
      Prize in Physics with Pieter Zeeman for the discovery and theoretical
      explanation of the Zeeman effect. Derived the Lorentz transformation of
      the special theory of relativity, as well as the Lorentz force
      $$
        \abovedisplayskip=3pt
        \belowdisplayskip=3pt
        {\bf F}=q\big({\bf E}+({\bf v}\times{\bf B})\big),
      $$
      acting upon a moving charged particle.}
    \fi}%
    \sidx{Lorentz, Hendrik Antoon (1853--1928)}
}

%
% Ludvig Lorenz (1829--1891)
%
\def\lorenz{%
  \portrait{images/ludvig_valentin_lorenz_scaled_bw.eps}%
           {../gallery/images/ludvig_valentin_lorenz_scaled_bw.eps}%
    {Ludvig Lorenz}%
    {1829--1891}%
    {\ifnum\lang=\swedish
      {Dansk fysiker och matematiker. {\AA}r 1867 presenterade Lorenz de
      allm{\"a}nna integrall{\"o}s\-ning\-arna till differentialekva\-tion\-erna
      f{\"o}r elektrodynamiken, inklusive retardation (retarderade potentialer)
      till f{\"o}ljd av den {\"a}ndliga ljushastig\-heten, samt Lorenz-villkoret
      ({\it Lorenz gauge})
      $$
        \abovedisplayskip=3pt
        \belowdisplayskip=3pt
        \nabla\cdot{\bf A}+
          {{1}\over{c^2}}{{\partial\phi}\over{\partial t}} = 0.
      $$}
    \else
      {Danish physicist and mathematician. In 1867, Lorenz gave
      completely general integral solutions to the differential equations of
      electromagnetism, including retardation due to the finite speed of
      light, and the introduction of the Lorenz gauge
      $$
        \abovedisplayskip=3pt
        \belowdisplayskip=3pt
        \nabla\cdot{\bf A}+
          {{1}\over{c^2}}{{\partial\phi}\over{\partial t}} = 0.
      $$}
    \fi}%
    \sidx{Lorenz, Ludvig (1829--1891)}
}

%
% James Clerk Maxwell (1831--1879)
%
\def\maxwell{%
  \portrait{images/james_clerk_maxwell_scaled_bw.eps}%
           {../gallery/images/james_clerk_maxwell_scaled_bw.eps}%
    {James Clerk Maxwell}%
    {1831--1879}%
    {\ifnum\lang=\swedish
      {Skotsk fysiker och matematiker som konsliderade den klassiska teorin
      f{\"o}r elektromagnetisk str{\aa}lning, vilken var den f{\"o}rsta
      teorin som be\-skrev elektricitet, magnetism och ljus som oli\-ka
      manifestationer av samma fenomen. Max\-wells ekvationer f{\"o}r
      elektromagnetismen kom att utg{\"o}ra den andra stora f{\"o}reningen
      ({\it second great unification}) inom fysiken, d{\"a}r den f{\"o}rsta
      hade f{\"o}rverkligats av Isaac Newton.}
    \else
      {Scottish physicist and mathematician who was responsible for the
      classical theory of electromagnetic radiation, which was the first
      theory to describe electricity, magnetism and light as different
      manifestations of the same phenomenon. Maxwell's equations for
      electromagnetism achieved the second great unification in physics,
      where the first one had been realised by Isaac Newton.}
    \fi}%
    \sidx{Maxwell, James Clerk (1831--1879)}
    \sidx{Maxwells ekvationer}
}

%
% Franz Ernst Neumann (1798--1895)
%
\def\neumann{%
  \portrait{images/franz_ernst_neumann_scaled_bw.eps}%
           {../gallery/images/franz_ernst_neumann_scaled_bw.eps}%
    {Franz Ernst Neumann}%
    {1798--1895}%
    {\ifnum\lang=\swedish
      {Tysk mineralog och fysiker. Utvecklade de f{\"o}rsta formlerna
      f{\"o}r att ber{\"a}kna induktiv koppling och induktans, samt den
      rent geometr\-iska formeln f{\"o}r {\"o}msesidig induktans som
      b{\"a}r hans namn. Inom elektromagnetism tillskrivs han
      introduktionen av den magnetiska vektorpotentialen ${\bf A}$,
      som anv{\"a}nds omfattande i denna kurs.}
    \else
      {German mineralogist and physicist. Devised the first formulas to
      calculate inductance, and the purely geometrical formula for mutual
      inductance is named after him. In electromagnetism, he is credited
      for introducing the magnetic vector potential ${\bf A}$, as extensively
      used throughout this course.}
    \fi}%
    \sidx{Neumann, Franz Ernst (1798--1895)}
    \sidx{Neumanns formel}[{\"O}msesidig induktans]
}

%
% Hans Christian Oersted (1777--1851)
%
\def\oersted{%
  \portrait{images/hans_christian_oersted_scaled_bw.eps}%
           {../gallery/images/hans_christian_oersted_scaled_bw.eps}%
    {Hans Christian {\OE}rsted}%
    {1777--1851}%
    {\ifnum\lang=\swedish
      {Dansk kemist och fysiker som uppt{\"a}ckte att elektriska str{\"o}mmar
      skapar magnetf{\"a}lt. Enheten Oe (oersted) f{\"o}r magnetf{\"a}ltets
      styrka ${\bf H}$ {\"a}r uppkallad efter honom och definieras som
      $1\ {\rm Oe}=(4\pi)^{-1}\times10^3\ {\rm A}/{\rm m}\approx 79.58
      \ {\rm A}/{\rm m}$.}
    \else
      {Danish chemist and physicist who discovered that electric currents
      create magnetic fields. The unit Oe (oersted) of the magnetic field
      strength ${\bf H}$ is named after him, and defined as $1\ {\rm Oe}=
      (4\pi)^{-1}\times10^3\ {\rm A}/{\rm m}\approx 79.58\ {\rm A}/{\rm m}$.}
    \fi}%
    \sidx{{\OE}rsted, Hans Christian (1777--1851)}
    \sidx{Magnetisk f{\"a}ltstyrka ${\bf H}$}[{\OE}rsted]
}

%
% Siméon Denis Poisson (1781--1840)
%
\def\poisson{%
  \portrait{images/simeon_denis_poisson_scaled_bw.eps}%
           {../gallery/images/simeon_denis_poisson_scaled_bw.eps}%
    {Sim\'eon Denis Poisson}%
    {1781--1840}%
    {\ifnum\lang=\swedish
      {Fransk matematiker och fysiker. Poisson f{\"o}r\-ut\-sade
      {\it Aragos fl{\"a}ck} i sitt f{\"o}rs{\"o}k att motbevisa
      Augustin-Jean Fresnels v{\aa}gteori. I denna kurs {\"a}r
      anv{\"a}nder vi frekvent Poissons ekvation
      $$
        \abovedisplayskip=3pt
        \belowdisplayskip=3pt
        \nabla^2\phi=\rho/\varepsilon_0
      $$
      f{\"o}r den skal{\"a}ra elektrostatiska potentialen $\phi$.}
    \else
      {French mathematician and physicist. Poisson predicted the
      {\it Arago spot} in his attempt to disprove the wave theory
      of Augustin-Jean Fresnel. In this course we are primarily
      concerned with the Poisson equation
      $$
        \abovedisplayskip=3pt
        \belowdisplayskip=3pt
        \nabla^2\phi=\rho/\varepsilon_0
      $$
      for the scalar electrostatic potential $\phi$.}
    \fi}%
    \sidx{Poisson, Sim\'eon Denis (1781--1840)}
    \sidx{Poissons ekvation}
}

%
% Felix Savart (1791--1841)
%
\def\savart{%
  \portrait{images/felix_savart_scaled_bw.eps}%
           {../gallery/images/felix_savart_scaled_bw.eps}%
    {F\'elix Savart}%
    {1791--1841}%
    {\ifnum\lang=\swedish
      {Fransk fysiker och matematiker som fr{\"a}mst {\"a}r k{\"a}nd
      f{\"o}r Biot--Savarts lag inom elektromagnetism,
      $$
        \abovedisplayskip=3pt
        \belowdisplayskip=3pt
        {\bf B}({\bf x})={{\mu_0 I}\over{4\pi}}
          \int{{d{\bf l}'\times({\bf x}-{\bf x}')}\over{|{\bf x}-{\bf x}'|^3}},
      $$
      som han utvecklade tillsammans med Jean-Baptiste Biot.}
    \else
      {French physicist and mathematician who is primarily known for the
      Biot--Savart law of electromagnetism,
      $$
        \abovedisplayskip=3pt
        \belowdisplayskip=3pt
        {\bf B}({\bf x})={{\mu_0 I}\over{4\pi}}
          \int{{d{\bf l}'\times({\bf x}-{\bf x}')}\over{|{\bf x}-{\bf x}'|^3}},
      $$
      co-discovered with Jean-Baptiste Biot.}
    \fi}%
    \sidx{Savart, F\'elix (1791--1841)}
    \sidx{Biot--Savarts lag}
}

%
% George Stokes (1819--1903)
%
\def\stokes{%
  \portrait{images/george_stokes_scaled_bw.eps}%
           {../gallery/images/george_stokes_scaled_bw.eps}%
    {George Stokes, 1st Baronet}%
    {1819--1903}%
    {\ifnum\lang=\swedish
      {Irl{\"a}ndsk matematiker och fysiker. Formule\-rade Navier--Stokes
      ekvationer och bidrog till teorin om optisk polarisation (Stokes-vektor).
      I denna kurs {\"a}r vi fr{\"a}mst intresserade av til\-l{\"a}mpningen
      av Stokes teorem,
      $$
        \abovedisplayskip=3pt
        \belowdisplayskip=3pt
        \def\iint{\mathop{\int\kern-8pt\int}}
        \iint(\nabla\times{\bf a})\cdot d{\bf S}
          =\oint{\bf a}\cdot d{\bf l},
      $$
      {\"a}ven betecknat {\it Kelvin--Stokes teorem}.}
    \else
      {Irish mathematician and physicist. Formulated the Navier--Stokes
      equations and contributed to the theory of optical polarization
      (Stokes vector). We are in this course primarily concerned with
      Stokes' theorem
      $$
        \abovedisplayskip=3pt
        \belowdisplayskip=3pt
        \def\iint{\mathop{\int\kern-8pt\int}}
        \iint(\nabla\times{\bf a})\cdot d{\bf S}
          =\oint{\bf a}\cdot d{\bf l},
      $$
      also denoted {\it Kelvin--Stokes theorem}.}
    \fi}%
    \sidx{Stokes, George (1819--1902)}
    \sidx{Stokes teorem}
    \sidx{Kelvin--Stokes teorem}[Se Stokes teorem]
}

%
% Nikola Tesla (1856--1943)
%
\def\tesla{%
  \portrait{images/nikola_tesla_scaled_bw.eps}%
           {../gallery/images/nikola_tesla_scaled_bw.eps}%
    {Nikola Tesla}%
    {1856--1943}%
    {\ifnum\lang=\swedish
      {Serbisk-amerikansk uppfinnare samt maskin- och elektroingenj{\"o}r.
      Hans patent och teoretis\-ka arbeten lade grunden f{\"o}r den
      till{\"a}mpade v{\"a}xelstr{\"o}mstekniken, som distributionssystem
      och v{\"a}xelstr{\"o}msmotorer med tre faser, vilket banade v{\"a}gen
      f{\"o}r den andra industriella revolutionen. SI-enheten tesla (T)
      f{\"o}r den magnetiska fl{\"o}dest{\"a}theten ${\bf B}$ namngavs 1960
      efter honom p{\aa} Conf\'erence G\'en\'erale des Poids et Mesures i
      Paris, Frankrike.}
    \else
      {Serbian-American engineer, futurist, and inventor. Primarily known
      for his contributions to applied electrodynamics, such as the design
      of the modern alternating current (AC) electricity supply system.
      The unit tesla (T) of the magnetic field ${\bf B}$ is named after him.}
    \fi}%
    \sidx{Tesla, Nikola (1856--1943)}
}

%
% Joseph John ("J. J.") Thomson (1856--1940)
%
\def\thomson{%
  \portrait{images/joseph_john_thomson_scaled_bw.eps}%
           {../gallery/images/joseph_john_thomson_scaled_bw.eps}%
    {Joseph John ("J. J.") Thomson}%
    {1856--1940}%
    {\ifnum\lang=\swedish
      {Brittisk fysiker, Nobelpris i Fysik 1906. I experimentella arbeten med
      katodstr{\aa}ler{\"o}r p{\aa}\-visa\-de han 1897 existensen av elektroner.
      I sina m{\"a}t\-ningar av hur katodstr{\aa}lar kunde f{\aa}s att
      avvika fr{\aa}n en r{\"a}t linje med elektriska och magnetiska
      f{\"a}lt drog han slutsatsen att str{\aa}len bestod av vad han
      fr{\aa}n b{\"o}rjan kallade ``korpuskler'' med negativ laddning
      och en storlek som var v{\"a}\-sent\-ligt mindre {\"a}n atomer.}
    \else
      {British physicist whose study of cathode rays led to his discovery
      of the electron. In 1897, he showed that cathode rays were composed
      of previously unknown negatively charged particles (now called electrons),
      which he calculated must have bodies much smaller than atoms and a very
      large charge-to-mass ratio.}
    \fi}%
    \sidx{Thomson, Joseph John (1856--1940)}
}

%
% George Paget Thomson (1892--1975)
%
\def\gpthomson{%
  \portrait{images/george_paget_thomson_scaled_bw.eps}%
           {../gallery/images/george_paget_thomson_scaled_bw.eps}%
    {George Paget Thomson}%
    {1892--1975}%
    {\ifnum\lang=\swedish
      {Brittisk fysiker, k{\"a}nd f{\"o}r sina experi\-mentella arbeten
      p{\aa} v{\aa}gegenskaper f{\"o}r elektronen. Han demonstrerade 1927
      elektrondiffraktion, vilket bekr{\"a}ftade Louis de Broglies hypotes
      och f{\"o}r vilket han 1937 erh{\"o}ll Nobelpriset i Fysik.
      Ett m{\"a}rkligt sammantr{\"a}ffande {\"a}r att han var son till
      J.~J.~Thomson, {\it med andra ord visade fad\-ern att elektroner
      {\"a}r partiklar medan sonen visade att de beter sig som v{\aa}gor!}}
    \else
      {British physicist best known for experimental work on the wave nature of
      electrons. In 1927 he experimentally demonstrated electron diffraction,
      confirming Louis de Broglie’s hypothesis. For this, he shared the 1937
      Nobel Prize in Physics. A remarkable twist is that he was the son of
      J. J. Thomson, {\it in other words, the father showed that electrons
      are particles, and the son showed they behave like waves!}}
    \fi}%
    \sidx{Thomson, Joseph John (1856--1940)}
}

%
% Wilhelm Eduard Weber (1804--1891)
%
\def\weber{%
  \portrait{images/wilhelm_eduard_weber_scaled_bw.eps}%
           {../gallery/images/wilhelm_eduard_weber_scaled_bw.eps}%
    {Wilhelm Eduard Weber}%
    {1804--1891}%
    {\ifnum\lang=\swedish
      {Tysk fysiker som tillsammans med C.~F.~Gauss uppfann den f{\"o}rsta
      elektromagnetiska tele\-graf\-en. {\it Weber-elektrodynamik} var en
      elektromagnetisk teori som f{\"o}regick Maxwells elektrodynamik och
      ersattes av den senare under slutet av 1800-talet.
      I Weber-elektrodynamiken blir Coulombs lag hastighets- och
      accelerations\-be\-roende, och teorin {\"a}r d{\"a}rmed endast
      giltig f{\"o}r kvasistation{\"a}ra f{\"a}lt.}
    \else
      {German physicist, together with C.~F.~Gauss inventor of the first
      electromagnetic telegraph. {\it Weber electrodynamics} was a theory
      of electromagnetism that preceded Maxwell electrodynamics,
      and was replaced by it by the end of the 19th century. In the
      formulation of Weber electrodynamics, Coulomb's law becomes velocity
      and acceleration dependent; hence the Weber electrodynamics was
      only applicable to quasi-static electromagnetic fields.}
    \fi}%
    \sidx{Weber, Wilhelm Eduard (1804--1891)}
}

%
% Rickard Wilson (1930--2000)
%
\def\wilson{%
  \portrait{images/rickard_wilson_scaled_bw.eps}%
           {../gallery/images/rickard_wilson_scaled_bw.eps}%
    {Rickard Wilson}%
    {1930--2000}%
    {\ifnum\lang=\swedish
      {Svensk civilingenj{\"o}r, vetenskapsman och bro\-konstrukt{\"o}r.
      Upphovsman till fatilarkalkylen, presenterad 1955. Kortfattat beskriver
      fatilarkalkylen den elastiska fekansens beroende av den fatilara
      koefficienten, dulationen, contaviseringen och till uttrycket med
      dessa fogas coaviliseringstermen.}
    \else
      {Swedish scientist and architect. Discoverer of the fatilar calculus,
      presented in 1955.}
    \fi}%
    \sidx{Wilson, Rickard (1930--2000)}
}

\ifnum\lang=\swedish
  \presection{Portr{\"a}ttgalleri inom klassisk elektromagnetism}
  \sidx{Klassisk elektromagnetism}[Portr{\"a}ttgalleri]
\else
  \presection{Portrait gallery in classical electromagnetism}
  \sidx{Classical electromagnetism}[Portrait gallery]
\fi

\noindent
\ifnum\lang=\swedish
Elektromagnetism som {\"a}mne {\"a}r proppfyllt med celebriteter inom den
klassiska fysiken, och det {\"a}r alltid trevligt att ha ett ansikte associerat
med namnet. Detta portr{\"a}ttgalleri t{\"a}cker de prim{\"a}ra uppt{\"a}ckarna
inom elektrostatik, magnetostatik och elektrodynamik.
\else
Electromagnetism as a subject is packed with celebrities from classical physics,
and it is always nice to have a face associated with the name. This portrait
gallery covers the primary discoverers in electrostatics, magnetostatics, and
electrodynamics.
\fi
\medskip
\newread\imgtest
\openin\imgtest=timeline/timeline.eps
\ifeof\imgtest
  % File does NOT exist
  \centerline{{\epsfxsize=480pt\epsfbox{../gallery/timeline/timeline.eps}}}
\else
  % File exists
  \closein\imgtest
  \centerline{{\epsfxsize=480pt\epsfbox{timeline/timeline.eps}}}
\fi
\vfill\eject
\portraitpage{\dalembert}{\ampere}{\biot}{\debroglie}   % Page 2 of gallery
\portraitpage{\coulomb}{\faraday}{\dufay}{\feynman}     % Page 3 of gallery
\portraitpage{\franklin}{\gauss}{\gilbert}{\heaviside}  % Page 4 of gallery
\portraitpage{\kelvin}{\laplace}{\lenz}{\lorentz}       % Page 5 of gallery
\portraitpage{\lorenz}{\maxwell}{\neumann}{\oersted}    % Page 6 of gallery
\portraitpage{\poisson}{\savart}{\stokes}{\tesla}       % Page 7 of gallery
\portraitpage{\thomson}{\gpthomson}{\weber}{\wilson}    % Page 8 of gallery

\pageno=1

%%%%%%%%%%%%%%%%%%%%%%%%%%%%%%%%%%%%%%%%%%%%%%%%%%%%%%%%%%%%%%%%%%%%%%%%
%%% The rest of this document is autoloaded and formatted externally %%%
%%%%%%%%%%%%%%%%%%%%%%%%%%%%%%%%%%%%%%%%%%%%%%%%%%%%%%%%%%%%%%%%%%%%%%%%
%%% Begin of auto-extracted text from ../lect-01/lecture-01.tex %%%
%
% File: teach/elmagii/lect-01/lecture-01.tex [plain TeX code]
% Github: https://github.com/elmagii/lect-01/
% Last change: November 2, 2025
%
% Lecture No 1 in the course ``Elektromagnetism II, 1TE626 (2025)'',
% held November 3, 2025, at Uppsala University, Sweden.
%
% Copyright (C) 2022-2025, Fredrik Jonsson, under Gnu General Public
% License (GPL) v3. See the enclosed LICENSE for details.
%
% This program is free software: you can redistribute it and/or modify
% it under the terms of the GNU General Public License as published by
% the Free Software Foundation, either version 3 of the License, or
% (at your option) any later version.
%
% This program is distributed in the hope that it will be useful,
% but WITHOUT ANY WARRANTY; without even the implied warranty of
% MERCHANTABILITY or FITNESS FOR A PARTICULAR PURPOSE.  See the
% GNU General Public License for more details.
%
% You should have received a copy of the GNU General Public License
% along with this program.  If not, see <https://www.gnu.org/licenses/>.
%
\def\coursename{Elektromagnetism II}
\def\coursecode{1TE626}
\def\courseyear{2025}
\def\courserepo{https://github.com/hp35/elmagii/}
\def\lecturenumber{1}
\def\lecturetitle{Elektrostatik, superpositionsprincipen och Gauss lag}
\def\lecturesubtitle{}
\def\lectureauthor{Fredrik Jonsson}
\def\lectureplace{Uppsala Universitet}
\def\lecturedate{3 november 2025}
%-------------------- BEGIN OF LOCAL MACROS --------------------
\edef\expandedlecturenumber{1}
\def\ifempty#1{\ifx\relax#1\relax}
\advance\chapno by 1
\secno=0
\footnotenumber=0
\message{==================== Lecture 1 ====================}
\writenumberedtocentry{chapter}{F{\"o}rel{\"a}sning 1 -- {Elektrostatik, superpositionsprincipen och Gauss lag}}{\thechapno}
\hsize=150mm\hoffset=4.6mm\vsize=230mm\voffset=7mm
\topskip=0pt\baselineskip=12pt\parskip=0pt\leftskip=0pt\parindent=15pt
\ifcolors
  \voffset=-10.2mm\topskip=0pt
\fi
\headline={\ifnum\secno>0\ifodd\pageno\rightheadline\else\leftheadline\fi
  \else\hfill\fi}
\def\rightheadline{\tenrm{\it F\"orel\"asning 1}
  \hfil{\it \coursename, \coursecode\ (\courseyear)}}
\def\leftheadline{\tenrm{\it \coursename, \coursecode\ (\courseyear)}
  \hfil{\it F\"orel\"asning 1}}
\noindent~\vskip-60pt\hskip-40pt{\epsfbox{../lect-01/macros/UU_logo_color.eps}}
\vskip-42pt\hfill\vbox{
    \hbox{{\it \coursename, \coursecode\ (\courseyear)}}
    \hbox{{\it Lecture Notes, \lectureauthor}}
    \hbox{{\it Document Revision \today}}
    \hbox{{\it \courserepo}}}\vskip 36pt
\centerline{\twelvesc F\"orel\"asning 1}
\vskip 24pt\noindent
\centerline{\twelvesc{Elektrostatik, superpositionsprincipen och Gauss lag}}
\expandafter\ifempty\expandafter{\lecturesubtitle}%
  \else\centerline{\twelvesc\lecturesubtitle}\fi
\bigskip
\centerline{\lectureauthor, \lectureplace, \lecturedate}
\vskip24pt
%--------------------- END OF LOCAL MACROS ---------------------



\plan{Med en kort sammanfattning av historiken bakom elektrostatik och
  uppt{\"a}ckten av elektronen som elementarladdning g{\aa}r vi direkt in
  p{\aa} Coulombs lag f{\"o}r v{\"a}xelverkan mellan laddade partiklar.
  Coulombs lag, som i grund och botten kan h{\"a}rledas fr{\aa}n utbyte av
  virtuella fotoner mellan laddade elementarpartiklar, tas h{\"a}r som ett
  axiom, fr{\aa}n vilket vi h{\"a}rleder fram motsvarande kraft p{\aa} en
  testladdning fr{\aa}n ett system av laddningar, ur vilken det elektriska
  f{\"a}ltet definieras.
  
  En genomg{\aa}ng av superpositionsprincipen f{\"o}r elektriska f{\"a}lt
  f{\"o}ljs av en h{\"a}r\-led\-ning av Coulomb-v{\"a}xelverkan f{\"o}r en
  kontinuerlig distribution av laddningst{\"a}thet, den s{\aa} kallade
  Coulombs generaliserade lag, eller kort och gott ``Coulombintegralen''.
  Vi intro\-ducerar det elektriska fl{\"o}det som integralen av det elektriska
  f{\"a}ltet {\"o}ver en godtycklig yta, utifr{\aa}n vilket vi h{\"a}rleder
  Gauss lag f{\"o}r elektriska f{\"a}lt, till{\"a}mpbar p{\aa} godtyckliga
  laddningsf{\"o}rdelningar i form av punkt- linje- yt- eller volymladdningar.
  Slutligen avslutar vi med en h{\"a}rledning av Gauss lag p{\aa}
  differentialform.}

\threepointsummary{%
  Superpositionsprincipen, som {\"a}r central i hela f{\"o}rel{\"a}sningsserien,
  inneb{\"a}r att vi kan addera separata l{\"o}sningar f{\"o}r elektriska
  f{\"a}lt fr{\aa}n separata ladd\-nin\-gar och laddningsf{\"o}rdelningar
  till en l{\"o}sning f{\"o}r det totala f{\"a}ltet.
}{%
  Det elektriska f{\"a}ltet fr{\aa}n ett system av punktladdningar $q'_k$,
  placerade i k{\"a}llpunkter ${\bf x}'_k$, kan f{\"o}r en kontinuerlig
  laddningsf{\"o}rdelning (laddnings\-t{\"a}thet) $\rho({\bf x})$ formuleras
  som Coulombs generaliserade lag (``Coulombintegralen''),
  $$
    {\bf E}({\bf x})={{1}\over{4\pi\varepsilon_0}}\sum^N_{k=1}
       q'_k {{({\bf x}-{\bf x}'_k)}\over{|{\bf x}-{\bf x}'_k|^3}},
       \quad\Leftrightarrow\quad
    {\bf E}({\bf x})={{1}\over{4\pi\varepsilon_0}}\iiint_V\rho({\bf x}')
      {{({\bf x}-{\bf x}')}\over{|{\bf x}-{\bf x}'|^3}}\,dV'.
  $$
  d{\"a}r laddningst{\"a}theten i sin tur kan beskriva punktladdningar som
  $$
    \rho({\bf x})=\sum_k q_k \delta({\bf x}-{\bf x}_k)
  $$
}{%
  Gauss lag p{\aa} integral- respektive differentialform:
  $$
    \oiint_S {\bf E}\cdot d{\bf S}
       ={{1}\over{\varepsilon_0}}\iiint_V\rho({\bf x})\,dV
    \qquad\Leftrightarrow\qquad
    \nabla\cdot{\bf E}={{\rho({\bf x})}\over{\varepsilon_0}}.
  $$
}
\vfill\eject\copyrights

\section{Introduktion till elektrostatik}
Vi kommer i denna f{\"o}rsta f{\"o}rel{\"a}sning\numberedfootnote{Detta
  avsnitt har sannolikt ett visst {\"o}verlapp med tidigare kurser;
  anledningen till att vi trots allt v{\"a}ljer att inkludera fundamentan
  av v{\"a}xelverkan mellan laddningar {\"a}r att notation och beteckningar
  kommer att {\aa}terkomma frekvent genom kursen, samt att vissa detaljer
  som superpositionsprincipen {\"a}r basen f{\"o}r senare, mer avancerade
  till{\"a}mpningar.}
att behandla  {\it elektrostatik}, som {\"a}r l{\"a}ran om hur
{\it station{\"a}ra} elektriska laddningar v{\"a}xelverkar.
Vi passar redan nu p{\aa} att sammanfatta begreppen elektrostatik och
magnetostatik enligt f{\"o}ljande:
\medskip
\item{$\bullet$}{Station{\"a}ra, tidsoberoende laddningar
  $\Rightarrow$ Konstanta elektriska f{\"a}lt (elektro{\it statik})}
\item{$\bullet$}{Station{\"a}ra, tidsoberoende str{\"o}mmar
  $\Rightarrow$ Konstanta magnetiska f{\"a}lt (magneto{\it statik})}

\section{Historik -- Elektrostatik fr{\aa}n de gamla grekerna till Coulomb}
\sidx{Elektrostatik}
\sidx{Elektrostatik}[Tidig historik]
Elektrostatikens historik\numberedfootnote{Se g{\"a}rna persongalleri f{\"o}r
  ofta f{\"o}rekommande fysiker och matematiker p{\aa} kursens GitHub-repo,
  p{\aa} {\tt https://github.com/hp35/elmagii/tree/main/gallery}}
kan s{\"a}gas ha b{\"o}rjat med ``de gamla grekerna'', som observerade att
b{\"a}rnsten som gnuggades med skinn efter{\aa}t attraherade l{\"a}tta objekt
som damm eller h{\aa}rstr{\aa}n. Detta fenomen d{\"o}ptes till ``elektricitet''
fr{\aa}n det gammalgrekiska (klassiskt grekiska) ordet f{\"o}r b{\"a}rnsten,
``elektron''.\numberedfootnote{%
  $\eta\lambda\varepsilon\kern-.4pt\kappa\tau\kern-1.4pt\rho o$;
  fr{\aa}n klassisk grekiska
  $\eta\lambda\varepsilon\kern-.4pt\kappa\tau\kern-1.4pt\rho o\nu$
  (elektron, ``b{\"a}rnsten'').}
Under 1600- och 1700-talen gjordes framsteg prim{\"a}rt i och med William
Gilberts \sidx{Gilbert, William (1544--1603)} (1544--1603)
{\it On Magnetism} \sidx{Gilbert, William (1544--1603)}[{{\it On Magnetism}}]
(1600), d{\"a}r elektrisk attraktionskraft \sidx{Attraktionskraft}[Elektrisk]
separerades fr{\aa}n magnetism, samt
Fran\c{c}ois de Cisternay du Fays (1698--1739) \sidx{de Cisternay du Fay,
Fran\c{c}ois (1698--1739)}[{{\it Sixi\`eme m\'emoire sur l'\'electricit\'e
(1734)}}]\numberedfootnote{Vilket namn!} uppt{\"a}ckt (1730) av {\it tv{\aa}
typer} av laddning, ``vitreous'' (``glasaktig'', associerat med att den uppstod
d{\aa} glas gnidits med silke) och ``resinous'' (``k{\aa}daktig'',
``hartsartad'', associerat med att den uppstod d{\aa} harts eller b{\"a}rnsten
gnidits med p{\"a}ls). Detta kom senare att associeras med ``positiv''
respektive ``negativ'' laddning.

Benjamin Franklin \sidx{Franklin, Benjamin (1706--1790)} (1706--1790)
utf{\"o}rde under {\aa}ren 1747--1749 experiment f{\"o}r att bevisa sin teori
kring ``en-v{\"a}tske\-hypo\-tesen'', d{\"a}r han argumenterade f{\"o}r att du
Fays ``glasaktiga'' och ``k{\aa}daktiga'' (``vitreous and resinuous'')
dubbel-v{\"a}tsketori var felaktig och att det bara fanns en typ av ``elektrisk
v{\"a}tska'', samt att alla elektrostatiska effekter kunde sp{\aa}ras till ett
{\"o}verfl{\"o}d respektive underskott av denna ``v{\"a}tska''.
Franklin betecknade dessa som {\it positivt} ({\"o}verfl{\"o}d) respektive
{\it negativt} (underskott) av elektrisk v{\"a}tska, termer som fortfarande
idag lever kvar f{\"o}r att beteckna laddningens polaritet.%
\sidx{Elektrisk laddning}[Polaritet]

Charles-Augustin de Coulomb \sidx{de Coulomb, Charles-Augustin (1736--1806)}
(1736--1806) f{\"o}rtj{\"a}nar ett eget kapitel i elektrostatikens historia, i
och med publiceringen av hans {\it Second M\'emoire sur l'\'Electricit\'e et le
Magn\'etisme (1785)} \sidx{de Coulomb, Charles-Augustin (1736--1806)}[
{{\it Second M\'emoire sur l'\'Electricit\'e\break et le Magn\'etisme}}]
[{\it Histoire de l'Acad\'emie Royale des Sciences}, sid.~578--611
(1785)].\numberedfootnote{%
  V{\"a}l v{\"a}rd att ta en titt p{\aa}:
  {\tt https://books.google.se/books?id=by5EAAAAcAAJ\&pg=PA578}}
I denna publikation, p{\aa} sidan 579, fastst{\"a}lls det f{\"o}r f{\"o}rsta
g{\aa}ngen att den attraktiva kraften \sidx{Attraktionskraft}[Elektrisk]
mellan tv{\aa} motsatt laddade sf{\"a}rer {\"a}r proportionell mot produkten av
laddningarna p{\aa} sf{\"a}rerna, samt inverst proportionell mot kvadraten p{\aa}
avst{\aa}ndet mellan sf{\"a}rerna;\numberedfootnote{Coulombs lag var
  sj{\"a}lvfallet ej uttryckt i moderna enheter i denna tidiga artikel;
  SI-systemet skulle inte komma att tr{\"a}da i kraft f{\"o}rr{\"a}n 1954.}
med andra ord Coulombs lag \sidx{Coulombs kraftlag} s{\aa} som vi idag k{\"a}nner
den, och som dessutom {\"a}r den prim{\"a}ra basen f{\"o}r {\"a}mnet f{\"o}r
denna f{\"o}rel{\"a}sning.

Rent historiskt {\"a}r det anm{\"a}rkningsv{\"a}rt att det tog mer {\"a}n hundra
{\aa}r innan den Brittiska fysikern Joseph John ``J.~J.'' Thomson \sidx{Thomson,
Joseph John (1856--1940)} (1856--1940, Nobelpris i fysik 1906) visade p{\aa}
existensen av elektroner \sidx{Elektron} 1897, genom experiment med elektroner i
katodstr{\aa}ler{\"o}r. Genom att m{\"a}ta hur katodstr{\aa}len kunde f{\aa}s
att avvika fr{\aa}n en r{\"a}t linje med elektriska och magnetiska f{\"a}lt drog
han slutsatsen att katodstr{\aa}len bestod av vad han fr{\aa}n b{\"o}rjan
kallade ``korpuskler'' \sidx{Korpuskler} med negativ laddning och en storlek
som var v{\"a}sentligt mindre {\"a}n atomer.
Den vetenskapliga v{\"a}rlden hade dock redan b{\"o}rjat anamma ordet
``elektron'' f{\"o}r dessa elementarladdningar, \sidx{Elementarladdning $e$}
och Thomson anpassade sin terminologi d{\"a}refter.

J.~J.~Thomsons uppt{\"a}ckt av elektronen hade effekt inte bara p{\aa}
elektrostatiken i sig, utan p{\aa} vetenskapen i stort d{\aa} han p{\aa}visade
att atomer inte var de fundamentala partiklar som man tidigare hade trott, i
form av homogena sf{\"a}rer, utan kan s{\"a}gas vara startskottet f{\"o}r nya
atommodeller, som till exempel Ernest Rutherfords \sidx{Rutherford, Ernest
(1871--1937)} (1871--1937, Nobelpris i kemi 1908) uppt{\"a}ckt av atomk{\"a}rnan
1912, och Niels Bohrs \sidx{Bohr, Niels (1885--1962)} (1885--1962, Nobelpris i
fysik 1922) skalmodell av atomen 1913.
\vfill\eject

\section{\red{Coulombs lag f{\"o}r punktladdningar}}
\sidx{Coulombs kraftlag}
\epsfig{../lect-01/figs/coulomb.1}\noindent
Som den mest fundamentala byggstenen i elektrostatiken har vi att tv{\aa}
punktladdningar \sidx{Punktladdning}\sidx{Elektrisk laddning}
\sidx{Elektrisk laddning}[Polaritet] $q$ och $q'$, r{\"a}knade med sina
respektive tecken f{\"o}r positiv eller negativ laddning i enheter av Coulomb
(C) \sidx{Coulomb, enhet f{\"o}r laddning} och placerade i
respektive {\it observationspunkten} \sidx{Observationspunkt} ${\bf x}$ och
{\it k{\"a}llpunkten} \sidx{K{\"a}llpunkt} ${\bf x}'$,
attraherar \sidx{Attraktionskraft}[Elektrisk] eller repellerar varandra med en
kraft ${\bf F}$ genom Coulombs lag\numberedfootnote{Griffiths Ekv.~(2.1),
  sid.~60; laddningen i observationspunkten betecknas som ``test charge''.
  Observera ocks{\aa} Griffiths lite udda stil i notationen av
  ``$({\bf x}-{\bf x}')$'' som ``scriptat ${\bf x}$''
  (se Griffiths Ekv.~(2.2) p{\aa} sid.~60). Den notation som
  Griffiths anv{\"a}nder {\"a}r lite olycklig i det att tolkningen
  av en ortsvektor ${\bf x}$ d{\"a}rmed blir beroende av vilken
  stil p{\aa} typsnittet som anv{\"a}nts; i denna f{\"o}rel{\"a}sningsserie
  kommer vi att helt undvika denna f{\"o}rbryllande notation och
  ist{\"a}llet genomg{\aa}ende att i klartext skriva ut
  ``$({\bf x}-{\bf x}')$''.}
\red{$$
  {\bf F}={{qq'}\over{4\pi\varepsilon_0}}
  \underbrace{
     {{({\bf x}-{\bf x}')}\over{|{\bf x}-{\bf x}'|^3}}
  }_{\sim 1/r^2},
$$}
d{\"a}r
\red{$$
  \varepsilon_0\approx8.854\times10^{12}\ {\rm F}/{\rm m}
$$}
{\"a}r konstanten f{\"o}r den {\it elektriska permittiviteten i vakuum}, eller
kort och gott {\it vakuumpermittiviteten}.
\sidx{Elektrisk permittivitet}[Vakuumpermittivitet $\varepsilon_0$]
Sj{\"a}lvfallet agerar denna kraft reciprokt p{\aa} k{\"a}lladdningen
\sidx{K{\"a}lladdning} $q'$, och vi kan i uttrycket f{\"o}r \idx{Coulombs
kraftlag} ovan helt sonika v{\"a}xla primmet {\"o}ver till den andra positionen
och direkt erh{\aa}lla
\red{$$
  \hbox{Recipricitet}\quad\Rightarrow\quad
  {\bf F}'={{q'q}\over{4\pi\varepsilon_0}}
     {{({\bf x}'-{\bf x})}\over{|{\bf x}'-{\bf x}|^3}}.
  \hphantom{\hbox{Recipricitet}\quad\Rightarrow\quad}
$$}
\vfill\eject

\epsfig{../lect-01/figs/coulombsys.1}\noindent
Om vi ist{\"a}llet f{\"o}r en enskild punktladdning vid k{\"a}llpunkten
\sidx{K{\"a}llpunkt} betraktar ett {\red{{\it system}
\sidx{Punktladdning}[System av punktladdningar] av $N$ ladd\-ningar
$q'_k$ vid respektive k{\"a}llpositioner ${\bf x}'_k$}} (med prim f{\"o}r
konsekvent notation f{\"o}r k{\"a}llpunkter), {\"a}r den totala kraften som
verkar p{\aa} laddningen $q$ vid observationspunkten uppenbarligen
\red{$$
  \eqalign{
  {\bf F}&=\sum^N_{k=1} {\bf F}_k\cr
    &={{1}\over{4\pi\varepsilon_0}}\sum^N_{k=1}
     q q'_k {{({\bf x}-{\bf x}'_k)}\over{|{\bf x}-{\bf x}'_k|^3}}\cr
    &=q\bigg({{1}\over{4\pi\varepsilon_0}}\sum^N_{k=1}
     q'_k {{({\bf x}-{\bf x}'_k)}\over{|{\bf x}-{\bf x}'_k|^3}}\bigg)\cr
     &=q{\bf E}({\bf x}),
  }
$$}
d{\"a}r vi {\it definierade} det elektriska f{\"a}ltet \sidx{Elektrisk
f{\"a}ltstyrka} ${\bf E}({\bf x})$ fr{\aa}n de $N$ punktladdningarna
som\numberedfootnote{Griffiths Ekv.~(2.4), sid.~61.}
\red{$$
  \hbox{Definition:}\quad
  {\bf E}({\bf x})\equiv{{1}\over{4\pi\varepsilon_0}}\sum^N_{k=1}
     q'_k {{({\bf x}-{\bf x}'_k)}\over{|{\bf x}-{\bf x}'_k|^3}}.
  \hphantom{\hbox{Definition:}\quad}
$$}
Ett par observationer kring det elektriska f{\"a}ltet:
\medskip
\item{$\bullet$}{Formuleringen av uttrycket f{\"o}r det elektriska f{\"a}ltet
  {\"a}r {\it helt oberoende av testladdningen} \sidx{Testladdning} $q$. Detta
  kan tyckas
  sj{\"a}lvklart, men har en fundamental betydelse n{\"a}r vi alldeles strax
  kommer att generalisera f{\"a}ltbeskrivningen som Gauss lag. Specifikt, s{\aa}
  kan vi till ett elektriskt f{\"a}lt associerat med en viss grupp av laddningar
  det elektriska f{\"a}ltet associerat med en komplement{\"a}r grupp av
  laddningar; exempelvis kan vi betrakta ett totalt f{\"a}lt som uppbyggt
  dels av k{\"a}ll-laddningarna $q'_k$ dels av f{\"a}ltet som {\"a}r
  associerat till testladdningen $q$.}
\item{$\bullet$}{Summeringen av alla delbidrag vilar p{\aa} att vi kan
  betrakta varje laddning som oberoende av alla andra laddningar.
  I grund och botten antar vi att detta {\"a}r ett {\it linj{\"a}rt}
  problem. Mer om detta strax.}
\item{$\bullet$}{Denna m{\"o}jlighet att addera individuella del-l{\"o}sningar
  till en l{\"o}sning f{\"o}r det totala problemet brukar vi beteckna med
  {\it superpositionsprincipen}, \sidx{Superpositionsprincipen} vilken {\"a}r
  generellt giltig enbart f{\"o}r {\it linj{\"a}ra problem}.}
\item{$\bullet$}{I detta antagande ligger implicit antagandet om att
  samtliga laddningar i problemet har fixa positioner som inte {\"a}ndras
  genom n{\"a}rvaro av andra laddningar. Vi kommer senare i kursen att se
  hur exempelvis ytladdningar p{\aa} ledande material justeras utifr{\aa}n
  elektrostatiken till att bilda f{\"o}rdelningar beroende p{\aa} externa
  faktorer (externa laddningar); i dessa fall {\"a}r dock den station{\"a}ra
  l{\"o}sningen i {\it steady-state} fortfarande giltig under
  superpositionsprincipen.}

\section{Vad {\"a}r ursprunget f{\"o}r denna v{\"a}xelverkan?
  Kan vi h{\"a}rleda Coulombs lag?}
Vi kan alla alltsedan sedan skoltiden i gymnasiet formen p{\aa} Coulombs
lag, \sidx{Coulombs kraftlag} och till slut s{\"a}tter sig denna k{\"a}nsla
f{\"o}r ``proportionalitet mot produkten av laddningarnas v{\"a}rde och
inversen av deras avst{\aa}nd i kvadrat'' i ryggm{\"a}rgen som en given,
praktiskt taget axiomatisk\numberedfootnote{{\it Axiomatisk}:
  ``sj{\"a}lvklart sann'' eller baserad p{\aa} ett eller flera axiom.
  Ett axiom {\"a}r en grund\-l{\"a}ggande sats eller ett p{\aa}st{\aa}ende
  som antas vara sant {\it utan bevis} och som anv{\"a}nds f{\"o}r att
  h{\"a}rleda andra sanningar i ett system.}
naturlag. Om man b{\"o}rjar fundera lite p{\aa} det och l{\"a}mnar den
intuition som vi erh{\aa}llit kring Coulombs lag, s{\aa} blir det dock
aningen konstigt.
Varf{\"o}r skulle laddningar {\"o}verhuvudtaget ha egenskaper som g{\"o}r att
de attraheras eller repelleras av varandra? Vi vet ju dessutom att dessa
laddningar (typiskt elek\-tron\-er och protoner) {\"a}r mycket sm{\aa}, och
{\"a}ven om vi har m{\"a}ngder av dessa elementarpartiklar i ett material,
varf{\"o}r skulle dessa sub-mikroskopiska partiklar ha en r{\"a}ckvidd som
str{\"a}cker sig {\"o}ver makroskopiska avst{\aa}nd?
\medskip
\quote{{\it Fr{\aa}gan blir d{\"a}rmed: Kan vi h{\"a}rleda Coulombs lag?}}
\medskip
\noindent
Svaret p{\aa} denna fr{\aa}ga {\"a}r utan tvekan ``ja'', men tyv{\"a}rr inte
med de klassiska verktyg som vi f{\"o}rfogar {\"o}ver i denna kurs.
Om vi skall f{\"o}rs{\"o}ka sammanfatta grunden i v{\"a}xelverkan mellan
laddade elementarpartiklar i ett par punkter, s{\aa} bygger denna p{\aa}
f{\"o}ljande.
\medskip
\item{$\bullet$}{V{\"a}xelverkan mellan elementarpartiklar som b{\"a}r laddning
  sker via s{\aa} kallade {\it virtuella fotoner} \sidx{Virtuella fotoner}, vilka
  skiljer sig fr{\aa}n v{\aa}ra ``vardagliga'' fotoner som {\"a}r synliga kvanta
  av ljus.
  Virtuella fotoner {\"a}r tempor{\"a}ra och existerar endast under det att
  v{\"a}xelverkan sker, och {\"a}r ``b{\"a}rarna av kraft'' f{\"o}r
  elektromagnetiska krafter.}
\item{$\bullet$}{Laddade partiklar som elektroner \sidx{Elektron} och
  protoner \sidx{Proton} utbyter virtuella fotoner mellan sig hela tiden,
  men med starkare intensitet n{\"a}r de kommer n{\"a}ra varandra.
  Detta konstanta utbyte till{\aa}ter partiklarna att ut{\"o}va krafter p{\aa}
  varandra, s{\aa} som attraktion \sidx{Attraktionskraft}[Elektrisk] eller
  repulsion.}
\item{$\bullet$}{Viktig po{\"a}ng: {\it Virtuella fotoner kan inte observeras
  direkt, till skillnad fr{\aa}n vanliga fotoner.}}
\item{$\bullet$}{Virtuella fotoner existerar och kan p{\aa}visas inom ramverket
  f{\"o}r kvantf{\"a}ltteori ({\it quantum field theory}). De {\"a}r dock inte
  v{\"a}ldefinierade utanf{\"o}r f{\"a}ltet f{\"o}r v{\"a}xelverkan mellan
  partiklar, och de kan (i likhet med vanliga fotoner) ``l{\aa}na'' eneri och
  moment under korta tidsrymder, helt enligt Heisenbergs os{\"a}kerhetsteori
  i kvantmekaniken.}
\medskip
\noindent
Med detta sagt, s{\aa} kommer vi fram{\"o}ver i kursen helt och h{\aa}llet att
betrakta Coulombs lag \sidx{Coulombs kraftlag} som en axiomatiskt given naturlag.
\vfill\eject

\section{\red{Superpositionsprincipen}}
\sidx{Superpositionsprincipen}
Att separat framtagna elektriska f{\"a}lt kan ses som och adderas som
komponenter av ett totalt elektriskt f{\"a}lt {\"a}r basen i vad vi kallar
{\it superpositionsprincipen}.\numberedfootnote{Generellt g{\"a}ller det
  under {\it superposition} att vi rent matematiskt har att g{\"o}ra med
  en funktion $F$ som har egenskaperna att $F(x+y)=F(x)+F(y)$ (additivitet)
  samt att $F(ax)=aF(x)$ (homogenitet). Ordet {\it superposition}
  h{\"a}rr{\"o}r i sig fr{\aa}n det senlatinska {\it superpositionem},
  med betydelse {\it att placera {\"o}ver}. Ordet kommer sig av {\it super}
  (``ovanf{\"o}r'') och {\it ponere} (``att placera'').}
Vi kan utifr{\aa}n f{\"a}ltet vi nyss tog fram illustrera detta med en
{\red{godtycklig uppdelning av det elektriska f{\"a}ltet i tv{\aa} delar}},
med tv{\aa} separata grupper av de $N$ k{\"a}lladdningarna,
\sidx{K{\"a}lladdning} enligt
\red{$$
  {\bf E}({\bf x})\equiv
  \underbrace{
    {{1}\over{4\pi\varepsilon_0}}\sum^M_{k=1}
    q'_k {{({\bf x}-{\bf x}'_k)}\over{|{\bf x}-{\bf x}'_k|^3}}
  }_{\vbox{\hbox{Grupp 1 $\rightarrow\ {\bf E}_1({\bf x})$}\hbox{($M$ laddningar)}}}
  +\underbrace{
     {{1}\over{4\pi\varepsilon_0}}\sum^{N}_{k=M+1}
     q'_k {{({\bf x}-{\bf x}'_k)}\over{|{\bf x}-{\bf x}'_k|^3}}
   }_{\vbox{\hbox{Grupp 2 $\rightarrow\ {\bf E}_2({\bf x})$}\hbox{($N-M$ laddningar)}}}
   ={\bf E}_1({\bf x})+{\bf E}_2({\bf x}).
$$}
\epsfig{../lect-01/figs/coulombsyspart.1}\noindent
Med andra ord, s{\aa} l{\"a}nge som inte de inb{\"o}rdes positionerna eller
laddningarna p{\aa}verkas av varandra (inga fria r{\"o}relser eller
tillf{\"o}rsel av str{\"o}mmar) och om vi r{\aa}kar ha en geometri som
p{\aa} n{\aa}got s{\"a}tt gynnar framtagandet av komponenter f{\"o}r det
totala elektriska f{\"a}ltet, s{\aa} st{\aa}r det oss fritt att {\it ber{\"a}kna
komponenterna separat och d{\"a}refter sammanfoga dessa till en total
l{\"o}sning f{\"o}r det elektriska f{\"a}ltet}.
Med andra ord kan vi {\it superponera}, eller {l{\"a}gga samman}, tv{\aa}
separat framtagna l{\"o}sningar f{\"o}r det elektriska f{\"a}ltet p{\aa}
varandra.
Denna superpositionsprincip g{\"a}ller generellt f{\"o}r s{\aa} kallade
{\it linj{\"a}ra problem}.

\section{\red{Coulombintegralen - Coulombs generaliserade lag}}
\sidx{Coulombs generaliserade lag}[Coulombintegralen]
S{\aa} l{\aa}ngt har vi endast betraktat system av {\it punktladdningar}, men
med superpositionsprincipen i beaktande {\"a}r naturligtvis steget minimalt att
att {\"o}verf{\"o}ra diskussionen till ett {\it kontinuum av laddning i rummet},
f{\"o}rdelad enligt en laddningst{\"a}thet \sidx{Elektrisk laddningst{\"a}thet}
$\rho({\bf x})$ (${\rm C}/{\rm m}^3$). I detta fall kan vi se det som att varje
volymelement $\Delta V_k$ i k{\"a}llpunkterna ${\bf x}'_k$ i rummet uppb{\"a}r
en laddning som vi kan se som en punktladdning, vilket i ett {\red{kontinuum}}
{\"o}verg{\aa}r till
\red{$$
  q'_k=\rho({\bf x}')\Delta V_k\to\rho({\bf x}')dV'.
$$}
D{\aa} alla dessa infinitesimala volymelement summeras upp erh{\aa}ller
vi {\red{{\it Coulombintegralen}}} \sidx{Integral}[Volymintegral]\sidx{Coulombs
generaliserade lag}[Coulombintegralen] f{\"o}r det elektriska f{\"a}ltet i en
observationspunkt \sidx{Observationspunkt} ${\bf x}$ som\numberedfootnote{Notera
  att Griffiths Ekv.~(2.8), sid.~63, betecknar {\it inte} denna som
  ``{\it Coulomb integral}'', vilket {\"a}r lite synd d{\aa} denna term
  p{\aa} pricken beskriver vad det handlar om.}
\red{$$
  {\bf E}({\bf x})={{1}\over{4\pi\varepsilon_0}}\iiint_V\rho({\bf x}')
    {{({\bf x}-{\bf x}')}\over{|{\bf x}-{\bf x}'|^3}}\,dV'.
$$}

\section{\red{Vad {\"a}r egentligen po{\"a}ngen med att anv{\"a}nda elektriska
  f{\"a}lt?}}
N{\"a}r vi nu framst{\"a}llt tv{\aa} olika representationer f{\"o}r att behandla
elektrostatiska f{\"a}ltproblem (och vi kommer alldeles strax att dessutom
introducera en {\it tredje} variant i och med potentialer), s{\aa} infinner
sig naturligtvis fr{\aa}gan varf{\"o}r vi inte bara kan n{\"o}ja oss med
Coulombs lag? \sidx{Coulombs kraftlag} Denna ger ju direkt kraften, s{\aa} vad
{\"a}r {\"o}verhuvud vitsen med att (till synes) bara komplicera saker och ting
genom att inf{\"o}ra elektriska f{\"a}lt och potentialer?

L{\aa}t oss med anledning av denna retoriska fr{\aa}ga sammanfatta ett par
anledningar till varf{\"o}r konceptet med f{\"a}lt underl{\"a}ttar f{\"o}r
oss.\numberedfootnote{Fundera g{\"a}rna igenom argumenten och kom p{\aa}
  fler exempel!}

\subsection{{\red{F{\"a}ltkonceptet eliminerar ``verkan p{\aa} distans''}}
  (``{\it action at a distance}'')}
\item{$\bullet$}{Enbart utifr{\aa}n Coulombs lag s{\aa} som den h{\"a}r
  fastst{\"a}llts, s{\aa} verkar varje enskild laddning p{\aa} alla andra
  laddningar momentant (direkt utan n{\aa}gon f{\"o}rdr{\"o}jning) oavsett
  avst{\aa}nd.}
\item{$\bullet$}{F{\"a}ltmodellen till{\aa}ter oss att l{\aa}ta en laddnings
  p{\aa}verkan att propagera genom rummet, med en fix hastighet som normalt
  {\"a}r ljushastigheten \sidx{Ljushastighet} eller denna nedskalad med
  motsvarande brytningsindex \sidx{Brytningsindex} f{\"o}r mediet i rummet.}

\subsection{\red{Lokalitet och kausalitet}}
\item{$\bullet$}{Vi kan med f{\"a}ltkonceptet analysera kraften p{\aa} en
  laddning {\it lokalt} vid punkten d{\"a}r den {\"a}r placerad, utan att
  beh{\"o}va bry oss om sj{\"a}lva k{\"a}llan. (Detta n{\"a}r vi v{\"a}l
  har ber{\"a}knat sj{\"a}lva f{\"a}ltet, sj{\"a}lvfallet!)}
\item{$\bullet$}{Detta g{\"o}r f{\"a}ltkonceptet kompatibelt med
  relativitetsteori \sidx{Relativitetsteori} och modern
  f{\"a}ltteori\sidx{F{\"a}ltteori}.}

\subsection{\red{F{\"o}renkling av m{\aa}ngkropps-problem}}
\item{$\bullet$}{Med $N$ laddningar blir ber{\"a}kningen av alla parvisa
  krafter som verkar i systemet snabbt mycket omst{\"a}ndligt, med $N(N-1)$
  v{\"a}xelverkningar.}
\item{$\bullet$}{Med f{\"a}ltkonceptet kan vi ber{\"a}kna f{\"a}ltet fr{\aa}n
  de $N$ laddningarna en g{\aa}ng f{\"o}r alla (med superpositionsprincipen!)
  och d{\"a}refter ta fram kraften p{\aa} varje laddning $q_k$ som
  ${\bf F}_k=q_k{\bf E}$.}

\subsection{\red{F{\"a}lt kan existera oberoende av laddning}}
\item{$\bullet$}{F{\"a}lt kan existera och breda ut sig utan (direkt)
  n{\"a}rvaro av laddning, till exempel elektromagnetiska v{\aa}gor.}
\item{$\bullet$}{Detta visar p{\aa} att f{\"a}lt {\"a}r inte bara ett
  bekv{\"a}mt matematiskt verktyg, de har en fysikalisk realitet bortom
  enbart varandes ett s{\"a}tt att dela upp \sidx{Coulombs kraftlag} i faktorer.}
\vfill\eject

\section{\red{Gauss lag h{\"a}rledd i fyra steg}}
Vi skulle h{\"a}r i princip bara kunna h{\"a}nvisa till den generella formen
av Gauss\numberedfootnote{Efter Carl Friedrich Gauss (1777--1855) \sidx{Gauss,
  Carl Friedrich (1777--1855)}, tysk matematiker och naturvetare, ofta kallad
  {\it Princeps Mathematicorum} (``matematikernas furste'').}
lag, \sidx{Gauss lag}[F{\"o}r elektrisk f{\"a}ltstyrka] men skulle d{\aa} riskera
att missa n{\aa}gra intressanta po{\"a}nger.
L{\aa}t oss d{\"a}rf{\"o}r ta detta fr{\aa}n grunden, med ut\-g{\aa}ngs\-punkt
i Coulombs lag, i fyra enkla steg som f{\"o}rhoppningsvis ger oss en djupare
fysikalisk f{\"o}rst{\aa}else f{\"o}r vad Gauss lag inneb{\"a}r.

\subsection{Definition: \red{Elektriskt fl{\"o}de}}
\sidx{Elektriskt fl{\"o}de}
F{\"o}r att diskutera resultaten i dessa fyra steg kommer vi att anv{\"a}nda
konceptet {\it elektriskt fl{\"o}de} $\Phi_{\rm E}$
(enhet: ${\rm V}\cdot{\rm m}$), som definieras som den integrerade
normalkomponenten av den elektriska f{\"a}ltstyrkan {\"o}ver en yta~$S$,
\epsfig{../lect-01/figs/elecflow.1}\noindent
Notera att det {\it inte finns n{\aa}got fysikaliskt fl{\"o}de associerat med
ett elektriskt f{\"a}lt}, men att vi i analogi med andra vektorf{\"a}lt inom
``riktiga fl{\"o}den'' hos gaser eller v{\"a}tskor t{\"a}nker oss ett fl{\"o}de
{\"a}ven f{\"o}r det elektriska f{\"a}ltet.\numberedfootnote{Notera {\"a}ven
  den lite olyckliga associationen man l{\"a}tt g{\"o}r till ``elektriskt
  fl{\"o}de'' som en slags str{\"o}m; det elektriska f{\"a}ltet i sig
  inneb{\"a}r ju dock ej n{\aa}gon explicit transport av laddning, vilket
  i s{\aa} fall skulle betecknas som en elektrisk str{\"o}m. F{\"o}rst i
  n{\"a}rvaro av fria laddningar har vi ett fysikaliskt fl{\"o}de i form
  av en str{\"o}m associerad med det elektriska fl{\"o}det.}
Vi kan lite handviftande s{\"a}ga att det elektriska fl{\"o}det {\"a}r ett
m{\aa}tt p{\aa} ``hur m{\aa}nga elektriska f{\"a}ltlinjer som passerar ut
genom ytan'', r{\"a}knat med tecken utifr{\aa}n ytans
normalvektor~${\bf n}$.\sidx{Normalvektor}
\vfill\eject

\subsection{\red{Steg~1: Sf{\"a}risk symmetrisk omslutande yta och
  punktladdning}}
\sidx{Gauss lag}[H{\"a}rledning fr{\aa}n Coulombs lag]
Antag att vi har en punktladdning \sidx{Punktladdning} $q'$ placerad i position
vid ortsvektorn ${\bf x}'$. D{\aa} vi ber{\"a}knar det elektriska
fl{\"o}det \sidx{Elektriskt fl{\"o}de} ut fr{\aa}n denna laddning, s{\aa} kan vi
se det som att fl{\"o}det h{\"a}rr{\"o}r fr{\aa}n en {\it k{\"a}lla}, och vi
kommer fram{\"o}ver i kursen ofta att relatera till ``k{\"a}lladdningar'' som
ger upphov till elektriska f{\"a}lt och
fl{\"o}den.\numberedfootnote{Genomg{\aa}ende kommer vi i kursen att s{\"a}tta
  ett prim ($'$) p{\aa} de objekt som vi betraktar som k{\"a}llor,
  \sidx{K{\"a}lladdning} som ${\bf x}'$, $dS'$ (ytelement) eller $dV'$
  (volymelement), och l{\aa}ta observationspunkt och m{\"a}tetal vid denna
  vara oprimmade.
  Vi kommer {\"a}ven att genomg{\aa}ende explicit anv{\"a}nda
  Leibniz \sidx{Leibniz, Gottfried Wilhelm (1646--1716)} notation
  (${{\partial}\over{\partial x}}$, ${{d}\over{dx}}$) f{\"o}r derivator,
  f{\"o}r att inte r{\aa}ka in i situationer d{\"a}r ett prim kan misstas
  f{\"o}r en derivata.}

Vi l{\"a}gger en hypotetisk sf{\"a}r, eller ``virtuell sf{\"a}r'',
\sidx{Virtuell sf{\"a}r} $S$ med radien $|{\bf x}-{\bf x}'|=r=\hbox{konstant}$
runt punktladdningen, med syfte att f{\"o}rs{\"o}ka ber{\"a}kna det totala
elektriska fl{\"o}det \sidx{Elektriskt fl{\"o}de} ut genom ytan.
\epsfig{../lect-01/figs/gauss-step-1.1}\noindent
Det {\red{elektriska fl{\"o}det}} ut genom den slutna sf{\"a}ren $S$, r{\"a}knat
gentemot ytans normalvektor ${\bf n}$, ges med utnyttjandet av den sf{\"a}riska
symmetrin som\sidx{Integral}[Ytintegral]
\red{$$
  \eqalign{
    \Phi_{\rm E}&=\oiint_S {\bf E}({\bf x})\cdot d{\bf S}\cr
      &=\oiint_S
      \underbrace{
        {{q'}\over{4\pi\varepsilon_0}}
        {{({\bf x}-{\bf x}')}\over{|{\bf x}-{\bf x}'|^3}}
      }_{=E_r(r){\bf e}_r}\cdot
      \underbrace{
        {{({\bf x}-{\bf x}')}\over{|{\bf x}-{\bf x}'|}}dS
      }_{={\bf e}_r dS}
      =\big\{\hbox{Tag $r\equiv|{\bf x}-{\bf x}'|$}\big\}\cr
      &=\oiint_S{{q'}\over{4\pi\varepsilon_0 r^2}}\,dS
      ={{q'}\over{4\pi\varepsilon_0 r^2}}\oiint_S\,dS
      ={{q'}\over{4\pi\varepsilon_0 r^2}}4\pi r^2
      ={{q'}\over{\varepsilon_0}}.\cr
  }
$$}
Notera att det elektriska fl{\"o}det $\Phi_{\rm E}$ tolkat som ``hur m{\aa}nga
f{\"a}ltlinjer \sidx{Elektrisk f{\"a}ltlinje} som passerar ytan'' {\"a}r
obe\-ro\-ende av radien p{\aa} den omslutande sf{\"a}ren, och {\it enbart
beror p{\aa} styrkan av den inneslutna laddningen} (som kan vara positivt
eller negativt) samt vakuumpermittiviteten $\varepsilon_0$. \sidx{Elektrisk
permittivitet}[Vakuumpermittivitet $\varepsilon_0$]
Vi kommer att utnyttja detta faktum i n{\"a}sta steg.
\vfill\eject

\subsection{\red{Steg~2: Godtycklig omslutande yta och punktladdning}}
L{\aa}t oss nu generalisera Steg~1 genom att ers{\"a}tta den sf{\"a}riska
referensytan mot en yta av godtycklig form, med det enda kravet att vi
fortfarande omsluter punktladdningen $q'$. Denna nya yta beh{\"o}ver inte
vara, s{\"a}g, {\"o}verallt konvex, och vi till{\aa}ter {\"a}ven att ytan
exempelvis f{\aa}r vika sig runt sig sj{\"a}lv.
\epsfig{../lect-01/figs/gauss-step-2.1}\noindent
Om vi nu betraktar det elektriska fl{\"o}det som m{\aa}ttet p{\aa} ``hur
m{\aa}nga fl{\"o}deslinjer som passerar ytan, r{\"a}knat gentemot ytans
normalvektor'', s{\aa} ser vi att oavsett hur den omslutande ytan {\"a}r
formad s{\aa} kommer exakt lika m{\aa}nga sk{\"a}rningspunkter att erh{\aa}llas
som i Steg~1 under sf{\"a}risk symmetri. I de fall d{\"a}r en f{\"a}ltlinje
sk{\"a}r en del av den omslutande ytan som {\"a}r vikt omlott, s{\aa} kommer
varje sk{\"a}rning in i volymen att exakt motsvaras av en sk{\"a}rning ut
ifr{\aa}n ytan, med f{\"o}ljd att samtliga f{\"a}ltlinjer kommer att ha ett
resultat av exakt en sk{\"a}rning ut{\aa}t.

Ett s{\"a}tt att bokf{\"o}ringsm{\"a}ssigt s{\"a}tt hantera antalet
sk{\"a}rningar mellan varje f{\"a}ltlinje och ytan omslutande punktladdningen
{\"a}r att associera varje sk{\"a}rning {\it ut} fr{\aa}n ytan med v{\"a}rdet
$+1$ (${\bf n}\cdot{\bf E}>0$) och varje sk{\"a}rning {\it in} mot ytan med
v{\"a}rdet $-1$ (${\bf n}\cdot{\bf E}<0$). Om man f{\"o}ljer varje f{\"a}ltlinje
fr{\aa}n k{\"a}llan ut mot o{\"a}ndligheten s{\aa} inser man direkt att summan
av alla sk{\"a}rningar, och d{\"a}rmed varje f{\"a}ltlinjes bidrag till det
totala elektriska fl{\"o}det, {\"a}r {\it exakt en ekvivalent sk{\"a}rning
ut{\aa}t genom ytan}.

Notera att laddningar som ligger {\it utanf{\"o}r} volymen alltid kommer att ha
f{\"a}ltlinjer som har ett {\it j{\"a}mnt antal sk{\"a}rningar} med den
godtyckliga ytan $S$ (inklusive m{\"o}jligheten att en f{\"a}ltlinje inte
sk{\"a}r ytan alls). Slutsatsen av detta {\"a}r att {\it laddningar utanf{\"o}r
volymen alltid kommer att ha exakt noll i sina bidrag till det totala elektriska
fl{\"o}det genom den slutna ytan~$S$}.

Resultatet av detta resonemang {\"a}r att vi {\it fortfarande har exakt samma
totala elektriska fl{\"o}de} $\Phi_{\rm E}$ ut genom ytan som omsluter
punktladdningen $q'$. Med andra ord g{\"a}ller det {\"a}ven f{\"o}r en
{\red{{\it godtycklig} omslutande yta $S$}} att\sidx{Integral}[Ytintegral]
\red{$$
  \Phi_{\rm E}=\oiint_S {\bf E}({\bf x})\cdot d{\bf S}={{q'}\over{\varepsilon_0}},
$$}
det vill s{\"a}ga att {\it det totala elektriska fl{\"o}det ut genom den slutna
ytan enbart best{\"a}ms av v{\"a}rdet p{\aa} den inneslutna punktladdningen}.
Vi kommer nu att anv{\"a}nda detta resultat i Steg~3.
\vfill\eject

\subsection{\red{Steg~3: Godtycklig omslutande yta och system av
  punktladdningar}}
Vi kommer nu att ytterligare generalisera f{\"o}reg{\aa}ende resultat genom att
betrakta ett system av $N$ statiska punktladdningar, fixerade i rummet och
liksom tidigare inneslutna av en hypotetisk godtyckig yta $S$ med samma
egenskaper som tidigare.
\epsfig{../lect-01/figs/gauss-step-3.1}\noindent
Vi ser att situationen f{\"o}r varje enskild laddning i sig {\"a}r identisk med
situationen som vi analyserade i Steg~2. Varje enskild innesluten punktladdning
(``k{\"a}lla'') $q'_k$ skulle d{\"a}rmed ge ett bidrag till det totala
elektriska fl{\"o}det som $q'_k/\varepsilon_0$.

Rent formellt {\"a}r det totala elektriska fl{\"o}det ut genom den slutna
generella ytan $S$ enligt superpositionsprincipen \sidx{Superpositionsprincipen}
given
som\sidx{Integral}[Ytintegral]
\red{$$
  \eqalign{
    \Phi_{\rm E}&=\oiint_S {\bf E}({\bf x})\cdot d{\bf S}
      =\big\{\hbox{Superpositionsprincipen}\big\}\cr
      &=\oiint_S \Bigg(\sum^N_{k=1}{\bf E}_k({\bf x})\Bigg)\cdot d{\bf S}
       =\big\{\hbox{Bryt ut summationen}\big\}\cr
      &=\sum^N_{k=1}\underbrace{
             \oiint_S {\bf E}_k({\bf x})\cdot d{\bf S}
           }_{=q'_k/\varepsilon_0}
       ={{1}\over{\varepsilon_0}}\sum^N_{k=1}q'_k
       ={{q'_{\rm tot}}/{\varepsilon_0}}.\cr
  }
$$}
Slutsatsen av detta resultat {\"a}r att {\it det totala elektriska fl{\"o}det
$\Phi_{\rm E}$ ut genom den slutna ytan fortfarande enbart beror av den inneslutna
laddningen $q'_{\rm tot}$.} \sidx{Elektriskt fl{\"o}de} I det sista steget kommer
vi nu att gene\-ra\-li\-sera detta till godtyckliga kontinuerliga
laddningsf{\"o}rdelningar.
\vfill\eject

\subsection{\red{Steg~4: Godtycklig omslutande yta och kontinuerlig
  laddningsf{\"o}rdelning}}
Antag att vi nu ist{\"a}llet f{\"o}r diskreta punktladdningar $q'_k$ har en
laddningsf{\"o}rdelning $\rho({\bf x})$ \sidx{Elektrisk laddningst{\"a}thet}
(enhet ${\rm C}/{\rm m}^3$) i en sluten (i rummet begr{\"a}nsad) volym $V$.
Denna laddningsf{\"o}rdelning kan variera kontinuerligt (j{\"a}mnt) i rummet
s{\aa}v{\"a}l som diskontinuerligt (stegvis), och vi l{\"a}mnar {\"a}ven
{\"o}ppet f{\"o}r att $\rho({\bf x})$ skall kunna tolkas som en f{\"o}rdelning
inneh{\aa}llande (Dirac-)delta-funktioner \sidx{Diracs delta-distribution
$\delta$} $\delta({\bf x}-{\bf x}'_k)$, med betydelsen av diskreta
punktladdningar placerade vid k{\"a}llpositioner ${\bf x}'_k$.

I termer av laddningsf{\"o}rdelningen $\rho({\bf x})$ uppb{\"a}r d{\aa} varje
infinitesimalt {\it k{\"a}llelement} med volymen $dV'$ vid k{\"a}llpositionen
${\bf x}'$ laddningen $dq'=\rho({\bf x}')dV'$, vilken vi kan betrakta som en
infinitesimal punktladdning.
\epsfig{../lect-01/figs/gauss-step-4.1}\noindent
Med det tidigare resultatet f{\"o}r framtagandet av det elektriska f{\"a}ltet
fr{\aa}n diskreta laddningar s{\aa} f{\"o}ljer det kontinuerliga fallet helt
analogt, och med anv{\"a}ndande av superpositionsprincipen f{\aa}r vi direkt
att den tidigare summan {\"o}ver diskreta laddningar i rummet ers{\"a}tts av
\sidx{Integral}[Volymintegral]\sidx{Integral}[Ytintegral]
volymintegralen\numberedfootnote{Notera att Griffiths (sid.~70)
  olyckligtvis anv{\"a}nder den udda och vilseledande notationen
  $Q_{\rm enc}=\int_V\rho\,d\tau$ f{\"o}r volymintegralen. Normalt
  anv{\"a}nder vi $\tau$ som integrationsvariabel i {\it tid}.}
\red{$$
  \eqalign{
    \Phi_{\rm E}=\oiint_S {\bf E}({\bf x})\cdot d{\bf S}
       =\big\{
          \hbox{ Steg 3: ``${{1}\over{\varepsilon_0}}\sum^N_{k=1} dq'_k$'' }
        \big\}
       ={{1}\over{\varepsilon_0}}\iiint_V\rho({\bf x}')\,dV'
       =q_{\rm tot}/\varepsilon_0.
  }
$$}
Vi har d{\"a}rmed kommit fram till den generella formen av {\red{{\it Gauss lag
p{\aa} integralform}}}, \sidx{Gauss lag}[Integralform] vilken vi
sam\-man\-fattar med
\red{$$
  \oiint_S {\bf E}({\bf x})\cdot d{\bf S}
     ={{1}\over{\varepsilon_0}}\iiint_V\rho({\bf x}')\,dV'.
$$}
Vi rekapitulerar att {\it Gauss lag har h{\"a}rletts enbart utifr{\aa}n
Coulombs klassiska lag f{\"o}r punktladd\-ningar samt superpositionsprincipen}.

D{\aa} Coulombs lag bygger p{\aa} ett (f{\"o}r v{\aa}r del) mer eller mindre
heuristiskt $1/r^2$-beroende f{\"o}r det elektriska f{\"a}ltets avtagande
fr{\aa}n punktladdningen, s{\aa} {\"a}r det l{\"a}tt att tro att detta beroende
p{\aa} n{\aa}got s{\"a}tt ocks{\aa} kommer att sl{\aa} in p{\aa} det elektriska
f{\"a}ltet fr{\aa}n en godtycklig laddningsf{\"o}rdelning.
Detta {\"a}r dock mer komplicerat till sin natur, och som vi senare kommer att
se i F{\"o}rel{\"a}sning~8 kring multipolutveckling av laddningsf{\"o}rdelningar
s{\aa} finns det en uppsj{\"o} av olika s{\aa} kallade {\it multipoler} med
olika grad av avklingande.
\vfill\eject

\section{\red{Kontinuerliga laddningsf{\"o}rdelningar}}
\sidx{Elektrisk laddningst{\"a}thet}[Volym-, yt-, linje-, punkt-]
Konceptet kontinuerlig laddningsf{\"o}rdelning kan sj{\"a}lvfallet appliceras
p{\aa} {\"a}ven trajektorior i rummet (linjeladdningar), ytor (ytladdningar).
I de fall d{\"a}r man har att g{\"o}ra med punktladdningar p{\aa} linjer, ytor
eller i volymer, s{\aa} kan dessa modelleras som spatiala delta-pulser
$\delta({\bf x}-{\bf x}')$ d{\"a}r ${\bf x}'$ {\"a}r positionen f{\"o}r
punktladdningen.
\epsfig{../lect-01/figs/chargetypes.1}

\section{\red{Fr{\aa}n Gauss lag till Coulombs lag}}
\sidx{Coulombs kraftlag}[H{\"a}rledd fr{\aa}n Gauss lag]
En enkel exercis f{\"o}r att f{\aa} en k{\"a}nsla f{\"o}r Gauss lag och vad det
inneb{\"a}r {\"a}r att g{\aa} andra v{\"a}gen, och fr{\aa}n v{\aa}rt sista
resultat h{\"a}rleda Coulombs lag f{\"o}r v{\"a}xelverkan mellan
punktladdningar.
Antag att vi placerat en ``k{\"a}lla'' i form av en punktladdning $q'$ i
k{\"a}llpunkten ${\bf x}'$. Sett som en {\red{distribution}} motsvarar detta
laddningst{\"a}theten\numberedfootnote{Griffiths anv{\"a}nder notationen
  $\delta^3({\bf x})$ f{\"o}r den skal{\"a}ra (Dirac-)delta-funktionen
  i tre dimensioner; exponentl{\"a}gets ``3'' {\"a}r dock on{\"o}digt
  d{\aa} det utifr{\aa}n argumentet ${\bf x}$ {\"a}r uppenbart att det
  handlar om just tre dimensioner.}
\red{$$
  \rho({\bf x})=q'\delta({\bf x}-{\bf x}').
$$}
Antag vidare att vi l{\"a}gger en hypotetisk sf{\"a}r, vilket vi {\"a}ven kan
beteckna som en {\red{``virtuell sf{\"a}r''}} \sidx{Virtuell sf{\"a}r} i detta
tankeexperiment, {\red{med radien $r$}} centrerad runt denna punktladdning.
Gauss lag \sidx{Gauss lag}[Integralform] f{\"o}r laddnings\-f{\"o}rdel\-ningar
ger oss d{\aa} med anv{\"a}ndande av sf{\"a}risk symmetri att
\red{$$
  \underbrace{
    \oiint_S {\bf E}({\bf x})\cdot d{\bf S}
  }_{=E_r(r)4\pi r^2}
  ={{1}\over{\varepsilon_0}}
  \underbrace{
       \iiint_V\underbrace{
           q'\delta({\bf x}-{\bf x}')
       }_{\rho({\bf x})}\,dV'
  }_{=q'}
  \qquad\Rightarrow\qquad
  E_r(r) = {{q'}\over{4\pi\varepsilon_0 r^2}}.
$$}
Ut{\"o}ver att visa p{\aa} hur man kan ``g{\aa} bakl{\"a}nges'' fr{\aa}n Gauss
lag till Coulombs lag, med det v{\"a}lk{\"a}nda $1/r^2$-beroendet p{\aa}
avst{\aa}nd fr{\aa}n punktk{\"a}llan, s{\aa} ger denna h{\"a}rledning ocks{\aa}
vid hand hur koefficienten ``$4\pi\varepsilon_0$'' dyker upp.
Vi kommer fram{\"o}ver i kursen att se hur denna koefficient dyker upp praktiskt
taget {\"o}verallt i elektrostatiken.
\vfill\eject

\section{\red{Gauss lag p{\aa} differentialform}}
S{\aa} som vi formulerat Gauss lag hittills {\"a}r den p{\aa} {\it integralform}.
Denna form f{\"o}ljer mer eller mindre intuitivt utifr{\aa}n s{\"a}ttet vi
h{\"a}rlett den, genom successiva generaliseringar d{\"a}r vi adderar
(integrerar) infinitesimala laddningar och via superpositionsprincipen
l{\"a}gger ihop delresultaten f{\"o}r f{\"a}lt och fl{\"o}den till en total
l{\"o}sning.
I m{\aa}nga fall {\"a}r det dock anv{\"a}ndbart att ist{\"a}llet ha Gauss lag
p{\aa} differentialform till hands, och en f{\"o}rdel med denna form {\"a}r att
vi samtidigt enklare ser hur vi kan se laddningst{\"a}theten $\rho({\bf x})$
som en {\it k{\"a}llf{\"o}rdelning} i elektrostatiska (och elektrodynamiska)
problem.

Vi applicerar f{\"o}rst divergensteoremet\numberedfootnote{Se exempelvis
  innerp{\"a}rmen p{\aa} Griffiths, ``{\it Divergence Theorem}''.
  {\AA}terigen, notera att Griffiths anv{\"a}nder den aningen olyckliga
  notationen ``$d\tau$'' f{\"o}r volymelement.}
p{\aa} det elektriska f{\"a}ltet, som f{\"o}r en godtycklig volym $V$ omsluten
av en yta $S$ lyder
\red{$$
  \iiint_V\nabla\cdot{\bf E}\,dV=\oiint_S{\bf E}\cdot d{\bf S}.
$$}
Vi har samtidigt visat att detta uttryck enligt Gauss lag uppenbarligen ges
som\sidx{Integral}[Sluten ytintegral]\numberedfootnote{Vi tar oss h{\"a}r
  friheten att skippa primmet p{\aa} k{\"a}llorna; det {\"a}r i sammanhanget
  uppenbart {\"o}ver vilka dom{\"a}ner integralerna skall tolkas.}
\red{$$
  \oiint_S {\bf E}\cdot d{\bf S}
     ={{1}\over{\varepsilon_0}}\iiint_V\rho({\bf x})\,dV,
$$}
vilket i sin tur betyder att
\red{$$
  \iiint_V\nabla\cdot{\bf E}\,dV
     ={{1}\over{\varepsilon_0}}\iiint_V\rho({\bf x})\,dV.
$$}
Eftersom denna relation g{\"a}ller f{\"o}r en {\red{{\it godtyckligt} vald
volym $V$}} och f{\"o}r en {\it godtycklig} laddnings\-t{\"a}t\-het
$\rho({\bf x})$, s{\aa} betyder detta att integranderna i v{\"a}nster- och
h{\"o}gerledet m{\aa}ste vara identiska, det vill s{\"a}ga att
\red{$$
  \nabla\cdot{\bf E}={{\rho({\bf x})}\over{\varepsilon_0}},
$$}
vilket sammanfattar {\it Gauss lag p{\aa} differentialform}.
\sidx{Gauss lag}[Differentialform]
\vfill\eject

\section{Sammanfattning av F{\"o}rel{\"a}sning~1 -- Elektrostatik,
  superpositionsprincipen och Gauss lag}
\item{$\bullet$}{I elektrostatiken, och {\"a}ven senare i elektrodynamiken,
  har vi i huvudsak tre s{\"a}tt att betrakta v{\"a}xelverkan mellan
  station{\"a}ra laddningar: Som krafter mellan v{\"a}xelverkande laddningar,
  som f{\"a}lt och som potentialer. Mer om potentialer i kommande
  f{\"o}rel{\"a}sningar.}
\item{$\bullet$}{Coulombs kraftlag f{\"o}r punktladdningar lyder
  $$
    {\bf F}
      ={{qq'}\over{4\pi\varepsilon_0}}
        {{({\bf x}-{\bf x}')}\over{|{\bf x}-{\bf x}'|^3}}
      ={{qq'}\over{4\pi\varepsilon_0}}{{1}\over{r^2}}{\bf e}_r,
  $$
  d{\"a}r
  $$
    \varepsilon_0\approx8.854\times10^{12}\ {\rm F}/{\rm m}
  $$
  {\"a}r konstanten f{\"o}r den {\it elektriska permittiviteten i
  vakuum}, \sidx{Elektrisk permittivitet}[Vakuumpermittivitet $\varepsilon_0$]
  eller kort och gott {\it vakuumpermittiviteten}. Denna permittivitet
  kommer genom kursen att h{\"a}nga med som en signatur p{\aa} allt som
  h{\"a}danefter kommer att h{\"a}rledas fr{\aa}n Coulombs lag.}
\item{$\bullet$}{Superpositionsprincipen inneb{\"a}r att vi kan
  {\it addera separata l{\"o}sningar} f{\"o}r elektriska f{\"a}lt
  och fl{\"o}den fr{\aa}n separata laddningar och laddningsf{\"o}rdelningar
  till en l{\"o}sning f{\"o}r det {\it totala} f{\"a}ltet och fl{\"o}det.
  Superpositionsprincipen g{\"a}ller {\it enbart f{\"o}r linj{\"a}ra
  problem}, i vilka inga potenser av det elektriska f{\"a}ltet finns
  i de grund\-l{\"a}ggande ekvationerna.}
\item{$\bullet$}{Det elektriska f{\"a}ltet fr{\aa}n ett system av
  punktladdningar $q'_k$ placerade i k{\"a}llpunkter ${\bf x}'_k$ {\"a}r
  $$
    {\bf E}({\bf x})={{1}\over{4\pi\varepsilon_0}}\sum^N_{k=1}
       q'_k {{({\bf x}-{\bf x}'_k)}\over{|{\bf x}-{\bf x}'_k|^3}},
  $$
  med kraften p{\aa} en testladdning (punktladdning) $q$ placerad i
  observationspunkten ${\bf x}$ given som
  $$
    {\bf F}=q{\bf E}({\bf x}).
  $$}
\item{$\bullet$}{Coulomb-integralen f{\"o}r ett kontinuum av laddning
  f{\"o}rdelad enligt en laddningst{\"a}thet $\rho({\bf x})$ lyder
  $$
    {\bf E}({\bf x})={{1}\over{4\pi\varepsilon_0}}\iiint_V\rho({\bf x}')
      {{({\bf x}-{\bf x}')}\over{|{\bf x}-{\bf x}'|^3}}\,dV'.
  $$}
\item{$\bullet$}{F{\"a}lt kan existera och breda ut sig {\it utan
  (direkt) n{\"a}rvaro av laddning}, till exempel elektromagnetiska
  v{\aa}gor. Detta visar p{\aa} att f{\"a}lt {\"a}r inte bara ett
  bekv{\"a}mt matematiskt verktyg, de har en fysikalisk realitet
  bortom enbart varandes ett s{\"a}tt att dela upp Coulombs kraftlag
  i faktorer.}
\item{$\bullet$}{Punktladdning $q$ (${\rm C}$),
  linjeladdning $\lambda$ (${\rm C}/{\rm m}$),
  ytladdning $\sigma$ (${\rm C}/{\rm m}^2$),
  volymladdning $\rho$ (${\rm C}/{\rm m}^3$).}
\item{$\bullet$}{Gauss lag p{\aa} integral- respektive differentialform:
  $$
    \oiint_S {\bf E}\cdot d{\bf S}
       ={{1}\over{\varepsilon_0}}\iiint_V\rho({\bf x})\,dV
    \qquad\Leftrightarrow\qquad
    \nabla\cdot{\bf E}={{\rho({\bf x})}\over{\varepsilon_0}}.
  $$\sidx{Gauss lag}[Integralform]\sidx{Gauss lag}[Differentialform]}
\item{$\bullet$}{Tolkningen av Gauss lag {\"a}r att {\it det elektriska
  fl{\"o}det $\Phi_{\rm E}$ ut genom en sluten yta $S$ ges som den av ytan
  inneslutna laddningen $q_{\rm tot}$ dividerad med vakuumpermittiviteten}
  \sidx{Elektrisk permittivitet}[Vakuumpermittivitet $\varepsilon_0$]
  $\varepsilon_0$, som
  $$
    \Phi_{\rm E}
       \equiv\oiint_S {\bf E}({\bf x})\cdot d{\bf S}
       ={{1}\over{\varepsilon_0}}\iiint_V\rho({\bf x})\,dV
       \equiv q_{\rm tot}/\varepsilon_0.
  $$}

\cleardoublepage
%%% End of auto-extracted text from ../lect-01/lecture-01.tex %%%
%%% Begin of auto-extracted text from ../lect-02/lecture-02.tex %%%
%
% File: teach/elmagii/lect-01/lecture-02.tex [plain TeX code]
% Github: https://github.com/elmagii/lect-02/
% Last change: November 4, 2025
%
% Lecture No 2 in the course ``Elektromagnetism II, 1TE626 (2025)'',
% held November 4, 2025, at Uppsala University, Sweden.
%
% Copyright (C) 2022-2025, Fredrik Jonsson, under Gnu General Public
% License (GPL) v3. See the enclosed LICENSE for details.
%
% This program is free software: you can redistribute it and/or modify
% it under the terms of the GNU General Public License as published by
% the Free Software Foundation, either version 3 of the License, or
% (at your option) any later version.
%
% This program is distributed in the hope that it will be useful,
% but WITHOUT ANY WARRANTY; without even the implied warranty of
% MERCHANTABILITY or FITNESS FOR A PARTICULAR PURPOSE.  See the
% GNU General Public License for more details.
%
% You should have received a copy of the GNU General Public License
% along with this program.  If not, see <https://www.gnu.org/licenses/>.
%
\def\coursename{Elektromagnetism II}
\def\coursecode{1TE626}
\def\courseyear{2025}
\def\courserepo{https://github.com/hp35/elmagii/}
\def\lecturenumber{2}
\def\lecturetitle{Elektrisk potential och till{\"a}mpningar av Gauss lag}
\def\lecturesubtitle{}
\def\lectureauthor{Fredrik Jonsson}
\def\lectureplace{Uppsala Universitet}
\def\lecturedate{4 november 2025}
%-------------------- BEGIN OF LOCAL MACROS --------------------
\edef\expandedlecturenumber{2}
\def\ifempty#1{\ifx\relax#1\relax}
\advance\chapno by 1
\secno=0
\footnotenumber=0
\message{==================== Lecture 2 ====================}
\writenumberedtocentry{chapter}{F{\"o}rel{\"a}sning 2 -- {Elektrisk potential och till{\"a}mpningar av Gauss lag}}{\thechapno}
\hsize=150mm\hoffset=4.6mm\vsize=230mm\voffset=7mm
\topskip=0pt\baselineskip=12pt\parskip=0pt\leftskip=0pt\parindent=15pt
\ifcolors
  \voffset=-10.2mm\topskip=0pt
\fi
\headline={\ifnum\secno>0\ifodd\pageno\rightheadline\else\leftheadline\fi
  \else\hfill\fi}
\def\rightheadline{\tenrm{\it F\"orel\"asning 2}
  \hfil{\it \coursename, \coursecode\ (\courseyear)}}
\def\leftheadline{\tenrm{\it \coursename, \coursecode\ (\courseyear)}
  \hfil{\it F\"orel\"asning 2}}
\noindent~\vskip-60pt\hskip-40pt{\epsfbox{../lect-01/macros/UU_logo_color.eps}}
\vskip-42pt\hfill\vbox{
    \hbox{{\it \coursename, \coursecode\ (\courseyear)}}
    \hbox{{\it Lecture Notes, \lectureauthor}}
    \hbox{{\it Document Revision \today}}
    \hbox{{\it \courserepo}}}\vskip 36pt
\centerline{\twelvesc F\"orel\"asning 2}
\vskip 24pt\noindent
\centerline{\twelvesc{Elektrisk potential och till{\"a}mpningar av Gauss lag}}
\expandafter\ifempty\expandafter{\lecturesubtitle}%
  \else\centerline{\twelvesc\lecturesubtitle}\fi
\bigskip
\centerline{\lectureauthor, \lectureplace, \lecturedate}
\vskip24pt
%--------------------- END OF LOCAL MACROS ---------------------



\plan{Ett par enkla exempel p{\aa} utnyttjande av symmetrier inom elektrostatik
  med Gauss lag g{\aa}s igenom. Vi bevisar att i elektrostatiska problem {\"a}r
  alltid $\nabla\times{\bf E}={\bf 0}$, vilket f{\"o}ljer direkt av Stokes
  teorem applicerat p{\aa} en sluten slinga i ett statiskt elektriskt f{\"a}lt
  fr{\aa}n en punktladdning. Detta resultat generaliseras d{\"a}refter med
  superpositionsprincipen f{\"o}r en godtycklig laddningsf{\"o}rdelning.

  Att $\nabla\times{\bf E}={\bf 0}$ g{\"o}r att vi direkt kan formulera det
  statiska elektriska f{\"a}ltet i termer av en skal{\"a}r potential $\phi$
  enligt ${\bf E}=-\nabla\phi$, en potential som vi d{\"a}refter h{\"a}rleder
  den explicita integralformen f{\"o}r, uttryckt i laddningst{\"a}thet.
  Vi h{\"a}rleder uttrycken f{\"o}r upplagrad potentiell energi i termer
  av den elektriska potentialen, och vi g{\aa}r igenom paradoxen i att det
  vektorv{\"a}rda elektriska f{\"a}ltet ${\bf E}$ kan extraheras ur en enda
  skal{\"a}r potential $\phi$.

  Vi avslutar f{\"o}rel{\"a}sningen med att utifr{\aa}n Gauss lag f{\"o}r det
  elektriska f{\"a}ltet p{\aa} differentialform omformulera denna i termer av
  den skal{\"a}ra potentialen som Poissons ekvation $\nabla^2\phi=-\rho/
  \varepsilon_0$.}

\threepointsummary{%
  Den skal{\"a}ra (elektrostatiska) potentialen $\phi$ f{\"o}ljer direkt av
  att Coulombs generaliserade lag kan tolkas som en gradient av en skal{\"a}r
  funktion, som
  $$
    {\bf E}({\bf x})
      ={{1}\over{4\pi\varepsilon_0}}\iiint_V\rho({\bf x}')
        {{({\bf x}-{\bf x}')}\over{|{\bf x}-{\bf x}'|^3}}\,dV'
      =-{{1}\over{4\pi\varepsilon_0}}\nabla\iiint_V
        {{\rho({\bf x}')}\over{|{\bf x}-{\bf x}'|}}\,dV'
      =-\nabla\phi({\bf x}),
  $$
  d{\"a}r den skal{\"a}ra potentialen kort och gott {\it definieras} som
  $$
    \phi({\bf x})\equiv{{1}\over{4\pi\varepsilon_0}}\iiint_V
        {{\rho({\bf x}')}\over{|{\bf x}-{\bf x}'|}}\,dV'.
  $$
}{%
  Arbete $W$ f{\"o}r att f{\"o}rflytta en punktladdning $q$ fr{\aa}n
  ${\bf x}_a$ till ${\bf x}_b$ ges i termer av den elektrostatiska potentialen
  $\phi({\bf x})$ som
  $$
    W=q\big(\phi({{\bf x}_b})-\phi({{\bf x}_a})\big)=W_b-W_a,
  $$
  d{\"a}r skillnaden $W_b-W_a$ {\"a}r skillnaden i potentiell energi.
}{%
  Den skal{\"a}ra potentialen $\phi$ lyder {\it Poissons ekvation},
  $$
    \nabla^2\phi({\bf x})=-\rho({\bf x})/\varepsilon_0.
  $$
}
\vfill\eject\copyrights

\section{Till{\"a}mpning av Gauss lag - Rak linjeladdning}
Antag att vi vill ber{\"a}kna den elektriska f{\"a}ltstyrkan
\sidx{Elektrisk f{\"a}ltstyrka} (``det elektriska f{\"a}ltet'') p{\aa}
avst{\aa}ndet $r$ vinkelr{\"a}tt fr{\aa}n fr{\aa}n en laddning f{\"o}rdelad
p{\aa} en o{\"a}ndlig och rak linje, med
linjeladdningst{\"a}theten \sidx{Elektrisk laddningst{\"a}thet}[Linjeladdning]
$\lambda$ (${\rm C}/{\rm m}$).

En taktik att angripa detta problem vore att betrakta varje liten del $dl$
av linjeladdningen som en punktladdning $dq=\lambda dl$ och d{\"a}refter
integrera alla delbidrag genom Coulombs lag \sidx{Coulombs kraftlag},
f{\"o}rhoppningsvis med konvergens trots att vi integrerar {\"o}ver
o{\"a}ndligheten.\numberedfootnote{Det {\"a}r
  l{\"a}tt att h{\"a}r falla i f{\"a}llan och tycka att konvergensen av
  integralen ju redan p{\aa} f{\"o}rhand {\"a}r given, med h{\"a}nvisning
  till att ``f{\"a}ltet ju avklingar med kvadraten p{\aa} avst{\aa}ndet''.
  Observera dock att detta g{\"a}ller f{\"o}r {\it punktladdningar}, medan
  vi h{\"a}r har att g{\"o}ra med en {\it linjeladdning} som vi ju faktiskt
  {\"a}nnu inte k{\"a}nner till den elektriska f{\"a}ltf{\"o}rdelningen
  f{\"o}r!}
Att utf{\"o}ra denna integral {\"a}r f{\"o}rvisso m{\"o}jligt, men genom att
anv{\"a}nda Gauss lag \sidx{Gauss lag} applicerad p{\aa} symmetrin i detta
specifika problem kan vi komma fram till l{\"o}sningen v{\"a}sentligt mycket
enklare.
\epsfig{../lect-02/figs/linecharge.1}\noindent
Vi placerar en ``Gaussisk cylinder'' \sidx{Gauss lag}[Cylindrisk geometri] av
radie $r$ och l{\"a}ngd $l$ centrerad runt linjeladdningen och antar vidare att
linjeladdningen inte kommer att ge n{\aa}got nettobidrag av f{\"a}ltlinjer
genom {\"a}ndytorna av cylindern, med andra ord att vi ignorerar randeffekter
i problemet. Gauss lag ger d{\aa} direkt, utan att beh{\"o}va l{\"o}sa
n{\aa}gon integral, att
$$
  \oiint_S {\bf E}\cdot d{\bf S}
     ={{1}\over{\varepsilon_0}}
       \underbrace{
         \iiint_V\rho({\bf x})\,dV
       }_{\vbox{\hbox{Innesluten}\hbox{laddning}}}
  \qquad\Leftrightarrow\qquad
  E_r(r)2\pi r l = {{1}\over{\varepsilon_0}}\lambda l
  \qquad\Leftrightarrow\qquad
  E_r(r) = {{\lambda}\over{2\pi \varepsilon_0 r}}.
$$
\vfill\eject

\subsection{Alternativ analys f{\"o}r rak linjeladdning}
Det finns sj{\"a}lvfallet alltid ett alternativ till anv{\"a}ndandet av Gauss
lag, som i fall d{\"a}r symmetrier saknas kan vara en omst{\"a}ndligare v{\"a}g
fram{\aa}t.
L{\aa}t oss d{\"a}rf{\"o}r illustrera en alternativ l{\"o}sningsmetod f{\"o}r
samma problem. Om vi ist{\"a}llet v{\"a}ljer att summera upp samtliga delbidrag
till det elektriska f{\"a}ltet i observationspunkten ${\bf x}={\bf e}_r r$
(vi v{\"a}ljer $z=0$ f{\"o}r observationspunkten ${\bf x}$) fr{\aa}n
linjeladdningen via Coulomb-integralen,
\sidx{Coulombs generaliserade lag}[Coulombintegralen] s{\aa} har vi med
k{\"a}llpunkter\sidx{K{\"a}lladdning} ${\bf x}'={\bf e}_z z'$ att
$$
  \eqalign{
    {\bf E}({\bf x})
      &={{1}\over{4\pi\varepsilon_0}}
        \int^{\infty}_{-\infty}{{({\bf x}-{\bf x}')}
          \over{|{\bf x}-{\bf x}'|^3}}dq'
       ={{1}\over{4\pi\varepsilon_0}}
        \int^{\infty}_{-\infty}{{({\bf e}_r r-{\bf e}_z z')}
          \over{|{\bf e}_r r-{\bf e}_z z'|^3}}\lambda dz'
          =\big\{\hbox{ Antisymmetri l{\"a}ngs $z$ }\big\}\cr
      &={\bf e}_r {{\lambda r}\over{4\pi\varepsilon_0}}
        \int^{\infty}_{-\infty}{{dz'}\over{(r^2+z'^2)^{3/2}}}
       ={\bf e}_r {{\lambda r}\over{4\pi\varepsilon_0}}
         \underbrace{
           \bigg[{{z'}\over{r^2(r^2+z'^2)^{1/2}}}\bigg]^{\infty}_{z'=-\infty}
         }_{=2/r^2}
       ={\bf e}_r \underbrace{
         {{\lambda}\over{2\pi\varepsilon_0 r}}
       }_{=E_r(r)}.\cr
  }
$$

\section{Till{\"a}mpning av Gauss lag - Plan ytladdning}
N{\"a}sta exempel p{\aa} till{\"a}mpning av Gauss lag
\sidx{Gauss lag}[Plan geometri] g{\"a}ller att best{\"a}mma det elektriska
f{\"a}ltet p{\aa} avst{\aa}ndet $a$ fr{\aa}n en o{\"a}ndlig plan yta,
b{\"a}rande ytladdningst{\"a}theten
\sidx{Elektrisk laddningst{\"a}thet}[Ytladdning] $\sigma$
(${\rm C}/{\rm m}^2$). Vi utnyttjar planariteten genom att l{\"a}gga en
plan-parallell ``Gaussisk burk'' inneslutande en del av ytan.
Om vi konstruerar burken s{\aa} att de planparallella ytorna omsluter ytan med
samma avst{\aa}nd till ytan, s{\aa} kan vi dessutom utnyttja {\"o}msesidig
symmetri i $z$-led mellan dessa.
I figuren {\"a}r denna ``Gaussiska burk'' utritad som en cylinder, men formen
av burken {\"a}r betydelsel{\"o}s s{\aa} l{\"a}ngs som locket och botten {\"a}r
planparallella mot ytan.\numberedfootnote{Griffiths anv{\"a}nder i Exempel~2.5,
  sid. 74, en rektangul{\"a}r ``Gaussian pillbox'' f{\"o}r samma uppgift.}
\epsfig{../lect-02/figs/surfcharge.1}\noindent
P{\aa} samma s{\"a}tt som f{\"o}r linjeladdningen i f{\"o}reg{\aa}ende exempel
ger Gauss lag direkt, utan att beh{\"o}va l{\"o}sa n{\aa}gon integral, att
$$
  \oiint_S {\bf E}\cdot d{\bf S}
     ={{1}\over{\varepsilon_0}}
       \underbrace{
         \iiint_V\rho({\bf x})\,dV
       }_{\vbox{\hbox{Innesluten}\hbox{laddning}}}
  \qquad\Leftrightarrow\qquad
  \underbrace{
    ({\bf e}_zE_z(a))\cdot(A{\bf e}_z)
      +({\bf e}_z\underbrace{E_z(-a)}_{=-E_z(a)})\cdot(-A{\bf e}_z)
  }_{=2E_z(a)A}
  = {{1}\over{\varepsilon_0}}\sigma A,
$$
det vill s{\"a}ga, med h{\"a}nsyn tagen till symmetrin $E_z(-z)=-E_z(z)_z$,
$$
  E_z(z) = {{\sigma}\over{2\varepsilon_0}}\sgn(z).
$$
Vi noterar att det elektriska f{\"a}ltet\sidx{Elektrisk f{\"a}ltstyrka} ut
fr{\aa}n den (i detta exempel) o{\"a}ndliga ytladdningen {\"a}r {\it oberoende
av avst{\aa}ndet fr{\aa}n ytan}, n{\aa}got som rent fysikaliskt {\"a}r l{\"a}tt
att inse d{\aa} f{\"a}ltlinjerna \sidx{Elektrisk f{\"a}ltlinje} rent
geometriskt alla m{\aa}ste vara parallella med varandra, med f{\"o}ljd att
det elektriska fl{\"o}det $\Phi_{\rm E}$ genom en godtycklig testyta ett
avst{\aa}nd fr{\aa}n ytladdningen m{\aa}ste vara konstant, med lika m{\aa}nga
f{\"a}ltlinjer sk{\"a}rande testytan oavsett avst{\aa}nd $a$ fr{\aa}n planet
som den placerats p{\aa}.

Sj{\"a}lvfallet {\"a}r det i praktiken ofysikaliskt med ett konstant elektriskt
f{\"a}lt som str{\"a}cker sig ut mot o{\"a}ndligheten, d{\aa} detta i s{\aa}
fall skulle svara mot en o{\"a}ndlig upplagrad energi i f{\"a}ltet.
Vi skall h{\aa}lla i minnet att en ``o{\"a}ndlig yta'' h{\"a}r betyder att vi
har en yta f{\"o}r vilken vi f{\"o}r den aktuella h{\"o}jden $z$ kan bortse
fr{\aa}n randeffekter.
\vfill\eject

\section{Rotationen f{\"o}r det statiska elektriska f{\"a}ltet}
\sidx{Elektrostatiskt f{\"a}lt}[Rotation f{\"o}r]
Som vi har sett kan det statiska elektriska f{\"a}ltet \sidx{Elektrisk
f{\"a}ltstyrka} r{\"a}knas fram genom att exempelvis summera upp (eller
integrera) bidrag fr{\aa}n punktladdningar~$q$ (${\rm C}$),
linjeladdningar~$\lambda$ (${\rm C}/{\rm m}$),
ytladdningar~$\sigma$ (${\rm C}/{\rm m}^2$) eller
volymladdningar~$\rho$ (${\rm C}/{\rm m}^3$)
\sidx{Elektrisk laddningst{\"a}thet}[Volym-, yt-, linje-, punkt-]
via Coulomb-integralen, varefter vi genom att applicera superpositionsprincipen
\sidx{Superpositionsprincipen} kan ta fram det totala f{\"a}ltet.
Vi har {\"a}ven konstaterat att divergensen f{\"o}r det elektriska f{\"a}ltet
ges av Gauss lag p{\aa} differentialform, \sidx{Gauss lag}[Differentialform]
som $\nabla\cdot{\bf E}=\rho/\varepsilon_0$.
Av ren nyfikenhet, l{\aa}t oss d{\"a}rf{\"o}r se vad {\it rotationen} hos det
statiska elektriska f{\"a}ltet kan uttryckas som.\numberedfootnote{Se
  Griffiths sid.~77--78.}

Betrakta en punktladdning $q'$, som vi f{\"o}r enkelhets skull nu placerar i
origo ${\bf x}'={\bf 0}$ f{\"o}r observationssystemet.\sidx{Observationspunkt}
Vi kommer i denna analys att utnytta Stokes teorem \sidx{Stokes teorem}
applicerat p{\aa} en linjeintegral\sidx{Integral}[Linjeintegral] f{\"o}r en
godtycklig trajektoria runt om i det statiska elektriska f{\"a}lt som omger
punktladdningen. Redan nu kan vi passa p{\aa} att mentalt associera denna
situation med en analogi med massa och gravitation.\sidx{Gravitation}
\epsfig{../lect-02/figs/lineintegral.1}\noindent
Med punktladdningen $q'$ placerad i origo har vi det statiska elektriska
f{\"a}ltet uttryckt i sf{\"a}riska koordinater \sidx{Sf{\"a}riska koordinater}
som
$$
  {\bf E}(r) % ={{q}\over{4\pi\varepsilon_0}}{{{\bf x}}\over{|{\bf x}|^3}}
    =\underbrace{
    {{q'}\over{4\pi\varepsilon_0 r^2}}
    }_{=E_r(r)} {\bf e}_r.
$$
Om vi analyserar linjeintegralen f{\"o}r en godtycklig trajektoria $\Gamma$
mellan tv{\aa} godtyckliga punkter ${\bf x}_a$ och ${\bf x}_b$ i rummet, s{\aa}
har vi uttryckt i sf{\"a}riska koordinater att
$$
  \eqalign{
    \int_{\Gamma}{\bf E}({\bf x})\cdot d{\bf l}
    &=\int_{\Gamma}(
        {\bf e}_rE_r+{\bf e}_{\varphi}\underbrace{E_{\varphi}}_{0}
            +{\bf e}_{\vartheta}\underbrace{E_{\vartheta}}_{0}
      )\cdot\underbrace{(
        {\bf e}_r dr+{\bf e}_{\varphi}r\sin(\vartheta)d\varphi
          +{\bf e}_{\vartheta}r d\vartheta
      )}_{\hbox{$d{\bf l}$ i sf{\"a}riska koordinater}}\cr
    &=\int^{r_b}_{r_a} {{q'}\over{4\pi\varepsilon_0 r^2}}\,dr\cr
    &={{q'}\over{4\pi\varepsilon_0}}
      \bigg({{1}\over{r_a}}-{{1}\over{r_b}}\bigg),\cr
  }
$$
d{\"a}r $r_a=|{\bf x}_a|$ och $r_b=|{\bf x}_b|$ {\"a}r avst{\aa}nden fr{\aa}n
origo (k{\"a}llpunkten) till punkterna ${\bf x}_a$ respektive ${\bf x}_b$.
Vi kan redan fr{\aa}n detta uttryck ana oss till att vi strax kommer att tolka
detta som en {\it potentialskillnad} mellan punkterna, men vi kan f{\"o}rst
konstatera att om vi analyserar $\Gamma$ i form av en {\it sluten} trajektoria,
s{\aa} kommer start- och slutpunkten att sj{\"a}lvfallet ha samma avst{\aa}nd
$r_a=r_b$ till origo, med f{\"o}ljd att
$$
  \oint_{\Gamma}{\bf E}\cdot d{\bf l}=0.
$$
D{\aa} vi applicerar {\it Stokes teorem} \sidx{Stokes teorem} p{\aa} detta
resultat,\numberedfootnote{Se exempelvis innerp{\"a}rmen p{\aa} Griffiths,
  ``{\it Curl Theorem}''.}
$$
  \iint_S\nabla\times{\bf E}\cdot d{\bf S}=\oint_{\Gamma}{\bf E}\cdot d{\bf l}=0,
$$
d{\"a}r $S$ {\"a}r den yta som innesluts av den slutna trajektorian $\Gamma$,
s{\aa} ser vi att -- eftersom $\Gamma$ kan v{\"a}ljas som en {\it godtycklig}
sluten trajektoria -- rotationen av det {\it statiska} elektriska f{\"a}ltet,
det vill s{\"a}ga integranden i integralen, m{\aa}ste vara identiskt noll,
$$
  \nabla\times{\bf E}={\bf 0}.
$$
Detta resultat h{\"a}rleddes h{\"a}r f{\"o}r en enskild punktladdning; dock
{\"a}r detta resultat direkt m{\"o}jligt att generalisera f{\"o}r en godtycklig
{\it distribution} av elektrisk laddning, eftersom superpositionsprincipen
direkt ger att ett totalt f{\"a}lt som {\"a}r sammansatt av ett antal delbidrag
${\bf E}_k$ -- oavsett k{\"a}llorna f{\"o}r dessa delbidrag {\"a}r -- uppfyller
att
$$
  \nabla\times{\bf E}
    =\nabla\times\sum_k{\bf E}_k
    =\sum_k\underbrace{\nabla\times{\bf E}_k}_{=0}
    ={\bf 0}.
$$
oavsett vad k{\"a}llorna f{\"o}r dessa delbidrag {\"a}r, s{\aa} l{\"a}nge som
de {\"a}r statiska (oberoende av tid). Vi kan h{\"a}r notera hur kraftfullt
{\it superpositionsprincipen} \sidx{Superpositionsprincipen} {\aa}terigen kommer
till v{\aa}r assistans, genom att l{\aa}ta oss f{\"o}rst l{\"o}sa ett
f{\"o}rh{\aa}llandevis enkelt problem f{\"o}r punktladdningar och d{\"a}refter
n{\"a}stintill trivialt l{\aa}ta oss {\it generalisera speciall{\"o}sningen till
en godtycklig distribution av elektriska laddningar}.

\section{Elektrostatisk skal{\"a}r potential}
\sidx{Skal{\"a}r potential}\sidx{Skal{\"a}r potential}[Elektrostatisk]
Utifr{\aa}n resonemanget ovan, kring den slutna linjeintegralen som vi kunde
anv{\"a}nda f{\"o}r att via Stokes teorem p{\aa}visa att rotationen av det
statiska elektriska f{\"a}ltet m{\aa}ste vara identiskt noll, s{\aa} {\"a}r
inte steget l{\aa}ngt till att associera den elektriska laddningen $q'$ med
analogin av massa och gravitation.\sidx{Gravitation}\numberedfootnote{I
  framtagandet av potentialen g{\"o}r vi h{\"a}r ett avsteg fr{\aa}n Griffiths,
  som ist{\"a}llet valt att visa p{\aa} existensen av en skal{\"a}r potential
  via linjeintegraler i det rotationsfria statiska elektriska f{\"a}ltet.
  Vi kommer h{\"a}r ist{\"a}llet att visa hur potentialen direkt f{\"o}ljer
  av hur den generaliserade formen av Coulombs lag kan tolkas som en gradient.
  Den variant av h{\"a}rledning som h{\"a}r presenteras f{\"o}ljer exempelvis
  J.~D.~Jackson, {\it Classical Electrodynamics}.\sidx{Jackson, John David
  (1925--2016)}[{{\it Classical Electrodynamics}}]}

F{\"o}r att rekapitulera s{\aa} har vi funnit att $\nabla\times{\bf E}({\bf x})
={\bf 0}$ {\"o}verallt i elektrostatiska problem, och vi kan samtidigt erinra
oss att vi har en vektoridentitet r{\"o}rande just rotationen som
lyder\numberedfootnote{Se exempelvis innerp{\"a}rmen p{\aa} Griffiths,
  ``{\it Second Derivatives}'', Ekv.~(10).}
$$
  \nabla\times(\nabla f)={\bf 0},
$$
f{\"o}r godtycklig ``well behaved'' skal{\"a}r funktion $f({\bf x})$.
Redan h{\"a}r kan vi dra slutsatsen att det elektriska f{\"a}ltet d{\"a}rmed
h{\"o}gst sannolikt m{\aa}ste g{\aa} att {\it uttrycka som en gradient av
n{\aa}gon skal{\"a}r funktion}, och det {\"a}r i stort sett detta som {\"a}r
det grundl{\"a}ggande argumentet f{\"o}r existensen av den skal{\"a}ra
elektrostatiska potentialen.
Vi kan {\"a}ven rekapitulera att vi tog fram $\nabla\times{\bf E}({\bf x})
={\bf 0}$ som en direkt f{\"o}ljd av formen av Coulombs generaliserade lag,
eller {\it Coulombintegralen},
\sidx{Coulombs generaliserade lag}[Coulombintegralen]
$$
  {\bf E}({\bf x})={{1}\over{4\pi\varepsilon_0}}\iiint_V\rho({\bf x}')
    \underbrace{
    {{({\bf x}-{\bf x}')}\over{|{\bf x}-{\bf x}'|^3}}
    }_{{\rm beror\ av\ }{\bf x}}\,dV'
    \eqq\hbox{``$\nabla f$''},
$$
s{\aa} fr{\aa}gan {\"a}r hur vi kan omformulera detta uttryck som en gradient
av n{\aa}gon skal{\"a}r funktion?
Till att b{\"o}rja med skall vi notera att en s{\aa}dan gradient opererar p{\aa}
koordinaterna i den ``oprimmade'' observationspunkten ${\bf x}$, och {\it inte}
p{\aa} k{\"a}llpunkterna ${\bf x}$'.
\vfill\eject

``Tricket'' i hur denna tolkning av Coulombintegralen skall ske ligger i
observationen att faktorn  ${{({\bf x}-{\bf x}')}/{|{\bf x}-{\bf x}'|^3}}$ i
integranden, som ju {\"a}r det enda i integralen som beror p{\aa}
observationspositionen ${\bf x}$, kan omformuleras som gradienten \sidx{Tricket
$\displaystyle\nabla{{1}\over{\char124}{\bf x}-{\bf x}'{\char124}}
=-{{({\bf x}-{\bf x}')}\over{{\char124}{\bf x}-{\bf x}'{\char124}^3}}$}%
\numberedfootnote{Vi kommer fram{\"o}ver att anv{\"a}nda detta mycket
  anv{\"a}ndbara ``trick'' ett flertal g{\aa}nger i denna
  f{\"o}rel{\"a}sningsserie. En analogi i vektoranalys till
  ``$(d/dr)(1/r)=-1/r^2$'', om man s{\aa} vill!}
$$
  \eqalign{
    \nabla{{1}\over{|{\bf x}-{\bf x}'|}}
      &=\bigg(
          {{\partial}\over{\partial x}},
          {{\partial}\over{\partial y}},
          {{\partial}\over{\partial z}}
        \bigg)
        {{1}\over{\big((x-x')^2+(y-y')^2+(z-z')^2\big)^{1/2}}}\cr
      &=-{{1}\over{2}}
        {{\big(2(x-x'),2(y-y'),2(z-z')\big)}
          \over{\big((x-x')^2+(y-y')^2+(z-z')^2\big)^{3/2}}}\cr
      &=-{{({\bf x}-{\bf x}')}\over{|{\bf x}-{\bf x}'|^3}}.\cr
  }
$$
Eftersom $\nabla$ opererar p{\aa} koordinater ${\bf x}$ i observationssystemet
d{\"a}r vi ju observerar det elektriska f{\"a}ltet, eller {\it labbsystemet}
\sidx{Labbsystem} om vi s{\aa} vill, och eftersom integralen utf{\"o}rs i det
{\it primmade} systemet ${\bf x}'$ d{\"a}r vi summerar upp alla bidrag
fr{\aa}n den elektriska laddningen, eller {\it k{\"a}llor},
\sidx{K{\"a}lladdning} s{\aa} kan vi bryta ut gradienten utanf{\"o}r
integralen, med resultatet
$$
  \eqalign{
    {\bf E}({\bf x})
      &={{1}\over{4\pi\varepsilon_0}}\iiint_V\rho({\bf x}')
        {{({\bf x}-{\bf x}')}\over{|{\bf x}-{\bf x}'|^3}}\,dV'\cr
      &=-{{1}\over{4\pi\varepsilon_0}}\iiint_V\rho({\bf x}')
        \underbrace{
          \nabla{{1}\over{|{\bf x}-{\bf x}'|}}
        }_{\hbox{$\nabla$ op.~p{\aa} ${\bf x}$}}\,dV'\cr
      &=-{{1}\over{4\pi\varepsilon_0}}\nabla\iiint_V
        {{\rho({\bf x}')}\over{|{\bf x}-{\bf x}'|}}\,dV'\cr
      &=-\nabla\phi({\bf x}),\cr
  }
$$
d{\"a}r vi {\it definierade} den skal{\"a}ra elektrostatiska potentialen
\sidx{Skal{\"a}r potential}[Elektrostatisk]
$\phi({\bf x})$ som\numberedfootnote{Vi kommer i denna serie av
  f{\"o}rel{\"a}sningar att genomg{\aa}ende anv{\"a}nda $\phi$ f{\"o}r
  att beteckna skal{\"a}r potential, f{\"o}r det statiska s{\aa}v{\"a}l
  som dynamiska fallet. Denna notation avviker fr{\aa}n Griffiths, som
  olyckligtvis anv{\"a}nder $V$ som notation f{\"o}r variabeln f{\"o}r
  potential, som d{\"a}rmed l{\"a}tt kan r{\aa}ka f{\"o}rv{\"a}xlas med
  den med potentialen intimt f{\"o}rknippade {\it enheten} Volt.}
$$
  \phi({\bf x})={{1}\over{4\pi\varepsilon_0}}\iiint_V
    {{\rho({\bf x}')}\over{|{\bf x}-{\bf x}'|}}\,dV'.
$$
Notera att formen av det elektriska f{\"a}ltet \sidx{Elektrisk f{\"a}ltstyrka}
som en gradient \sidx{Gradient} av en skal{\"a}r funktion g{\"o}r att vi
trivialt och helt enligt f{\"o}rv{\"a}ntan erh{\aa}ller
$$
  \nabla\times{\bf E}({\bf x})=-\nabla\times(\nabla\phi({\bf x}))={\bf 0}.
$$

\subsection{Tolkning av $-\nabla\phi$}
Utifr{\aa}n formen p{\aa} kopplingen mellan skal{\"a}r potential $\phi$ och det
elektriska f{\"a}ltet,
$$
  {\bf E}({\bf x})=-\nabla\phi({\bf x}),
$$
s{\aa} kan vi direkt g{\"o}ra oss en bild av den skal{\"a}ra potentialen som
en slags ``potentiell energi'' (l{\aa}t oss dock beh{\aa}lla citattecknen
h{\"a}r, eftersom $\phi$ inte har den fysikaliska dimensionen av energi),
s{\"a}g f{\"o}r en vandring uppf{\"o}r en kulle, f{\"o}r vilken den negativa
gradienten $-\nabla\phi$ d{\aa} kan tolkas som ``lutningen ned{\aa}t'' i
riktningarna runt den punkt d{\"a}r vi befinner oss.
\vfill\eject

\section{Arbete och upplagrad energi vid f{\"o}rflyttning av elektriska
  laddningar}
\sidx{Arbete och upplagrad energi}[Elektrostatisk laddning]
Med definitionen av den elektrostatiska skal{\"a}ra potentialen i bagaget
betraktar vi nu en testladdning $q$ som f{\"o}rflyttas\numberedfootnote{Notera
  sj{\"a}lvmots{\"a}gelsen i detta, i och med att vi s{\aa} fort vi
  {\it f{\"o}rflyttar en laddning} ju fakt\-iskt inte l{\"a}ngre har
  att g{\"o}ra med n{\aa}gon ``elektro{\it statik}'' hos stillast{\aa}ende
  laddningar; vi antar h{\"a}r dock att f{\"o}rflyttningen sker
  s{\aa} pass l{\aa}ngsamt (adiabatiskt) att Lorentz-kraften p{\aa}
  laddningen kan f{\"o}rsummas, och att vi d{\"a}rmed {\"a}ven
  f{\"o}rsummar eventuella genererade magnetf{\"a}lt genom
  f{\"o}rflyttningen, som de facto definierar en {\it str{\"o}m}
  i rummet, om {\"a}n en ``str{\"o}m f{\"o}r en punktladdning''.}
i ett godtyckligt elektrostatiskt f{\"a}lt ${\bf E}({\bf x})$, fr{\aa}n en
punkt ${\bf x}_a$ till en punkt ${\bf x}_b$ l{\"a}ngs en trajektoria $\Gamma$.
\epsfig{../lect-02/figs/potential.1}\noindent
Kraften som verkar p{\aa} laddningen $q$ vid en given punkt ${\bf x}$ l{\"a}ngs
trajektorian {\"a}r per definitionen av det elektriska f{\"a}ltet
$$
  {\bf F}({\bf x})=q{\bf E}({\bf x}),
$$
och det arbete $W$ som utf{\"o}rs d{\aa} vi f{\"o}rflyttar testladdningen ges
d{\"a}rmed som\numberedfootnote{Se exempelvis innerp{\"a}rmen p{\aa} Griffiths,
``{\it Gradient Theorem}''.}\sidx{Gradient-teoremet}
\sidx{Gradient}[Gradient-teoremet]
$$
  \eqalign{
    W&=-\int^{{\bf x}_b}_{{\bf x}_a}{\bf F}({\bf x})\cdot d{\bf l}\cr
     &=-q\int^{{\bf x}_b}_{{\bf x}_a}{\bf E}({\bf x})\cdot d{\bf l}
      =\big\{\hbox{ Anv{\"a}nd definitionen
                    ${\bf E}({\bf x})=-\nabla\phi({\bf x})$ }\big\}\cr
     &=q\int^{{\bf x}_b}_{{\bf x}_a}\nabla\phi({\bf x})\cdot d{\bf l}
      =\big\{\hbox{ Gradient-teoremet:
          $\int^{\bf b}_{\bf a}\nabla f\cdot d{\bf l} = f({\bf b})-f({\bf a})$}
       \big\}\cr
     &=q\big(\phi({{\bf x}_b})-\phi({{\bf x}_a})\big)
      =W_b-W_a,\cr
  }
$$
d{\"a}r skillnaden $W_b-W_a$ {\"a}r {\it skillnaden i potentiell energi} f{\"o}r
\sidx{Potentiell energi}[Elektrostatisk] testladdningen under det att den
f{\"o}rflyttats fr{\aa}n ${\bf x}_a$ till ${\bf x}_b$. Vi har nu allts{\aa} till
slut anl{\"a}nt till punkten d{\"a}r vi faktiskt kan tala om just ``potentiell
energi''.

Vi b{\"o}r h{\"a}r passa p{\aa} att erinra oss att sj{\"a}lva ordet ``potential''
\sidx{Skal{\"a}r potential}[Elektrostatisk] medf{\"o}r en stor risk att man per
automatik leds in till tankebanan att $\phi$ i sig skulle vara en ``potentiell
energi'', vilket ej {\"a}r fallet. V{\aa}r potential $\phi$ har dock den
fysikaliska dimensionen av volt (V), och en potentialskillnad l{\aa}ter sig
sj{\"a}lvfallet uttryckas i denna enhet.

\section{En paradox f{\"o}r den skal{\"a}ra potentialen}
En annan m{\"a}rklig egenskap hos den skal{\"a}ra elektrostatiska potentialen
{\"a}r att denna variabel via gradienten i definitionen av det elektrostatiska
f{\"a}ltet ${\bf E}=-\nabla\phi$ ger upphov till {\it tre} komponenter
$(E_x,E_y,E_z)$. Hur {\"a}r detta magiska m{\"o}jligt? Hur kan {\it en}
variabel pl{\"o}tsligt ge upphov till {\it tre} oberoende variabler?

Svaret till denna paradox\numberedfootnote{{\it Paradox} (av latin
  {\it para'doxus} ``paradoxal'', av likabetydande grekiska
  {\it paraʹdoxos}, av para- och do'xa ``mening'', ``{\aa}sikt''),
  p{\aa}st{\aa}ende, ofta i komprimerad form, som inneb{\"a}r en
  {\it skenbar} mots{\"a}gelse mot vanlig uppfattning men kan
  inneh{\aa}lla en djupare sanning.}
{\"a}r att de tre komponenterna hos det elektrostatiska f{\"a}ltet inte {\"a}r
oberoende, utan {\"a}r sammanl{\"a}nkade via $\nabla\times{\bf E}={\bf 0}$ som
$$
  {{\partial E_x}\over{\partial y}}={{\partial E_y}\over{\partial x}},\qquad
  {{\partial E_z}\over{\partial y}}={{\partial E_y}\over{\partial z}},\qquad
  {{\partial E_x}\over{\partial z}}={{\partial E_z}\over{\partial x}}.
$$

\section{Poissons ekvation f{\"o}r den skal{\"a}ra potentialen}
\sidx{Poissons ekvation}
Som en avslutning p{\aa} denna f{\"o}rel{\"a}sning vi passar vi p{\aa} att
konstatera att Gauss lag p{\aa} differentialform, $\nabla\cdot{\bf E}
=\rho/\varepsilon_0$, sammantaget med sj{\"a}lva definitionen f{\"o}r den
skal{\"a}ra potentialen, ${\bf E}=-\nabla\phi$, ger
$$
\nabla\cdot{\bf E}=
\nabla\cdot(-\nabla\phi)=
-\nabla^2\phi=\rho/\varepsilon_0,
$$
det vill s{\"a}ga att den partiella differentialekvation som beskriver den
elektrostatiska potentialen ges som {\it Poissons ekvation},
$$
  \nabla^2\phi({\bf x})=-\rho({\bf x})/\varepsilon_0,
$$
vilken {\aa}terigen kan tolkas med volymladdningst{\"a}theten $\rho$ som en
{\it k{\"a}llterm} i h{\"o}gerledet. Denna skal{\"a}ra ekvation {\"a}r
fundamental vid ber{\"a}kningar av elektrostatiska f{\"a}ltproblem och
{\"a}r synnerligen v{\"a}l l{\"a}mpad f{\"o}r numerisk analys. Griffiths
anser den s{\aa} fundamental att den till och med {\"a}r en av de tv{\aa}
ekvationer som listas p{\aa} omslaget till hans {\it Introduction to
Electrodynamics}.
\vfill\eject

\section{Sammanfattning av F{\"o}rel{\"a}sning~2 -- Elektrisk potential
  och till{\"a}mpningar av Gauss lag}
\item{$\bullet$}{Om m{\"o}jligt, se till att utnyttja eventuella symmetrier
  f{\"o}r att f{\"o}renkla l{\"o}sande av problem genom att applicera Gauss lag,
  $$
    \oiint_S {\bf E}\cdot d{\bf S}={{1}\over{\varepsilon_0}}
       \underbrace{
         \iiint_V\rho({\bf x})\,dV
       }_{\vbox{\hbox{Innesluten}\hbox{laddning}}}
  $$
  Gauss lag {\"a}r alltid giltig, men det {\"a}r inte alltid som den har
  n{\aa}got att bidra i praktiskt probleml{\"o}sande; dock {\"a}r n{\"a}rvaron
  av symmetrier ofta en v{\"a}gledning f{\"o}r v{\"a}gen fram{\aa}t.}
\item{$\bullet$}{Som exempel p{\aa} till{\"a}mpning av Gauss lag, s{\aa}
  f{\aa}r vi specifikt f{\"o}r linjeladdningar med laddnings\-t{\"a}t\-heten
  $\lambda$ (${\rm C}/{\rm m}$) att
  $$
    E_r(r)={{\lambda}\over{2\pi\varepsilon_0 r}},
  $$
  samt f{\"o}r ytladdningar p{\aa} ett plan med laddningst{\"a}theten $\sigma$
  (${\rm C}/{\rm m}^2$) att
  $$
    E_z(z)={{\sigma}\over{2\varepsilon_0}}\sgn(z),
  $$
  under antagandet att randeffekter fr{\aa}n laddningsf{\"o}rdelningarna kan
  f{\"o}rsummas.}
\item{$\bullet$}{Rotationen f{\"o}r ett {\it statiskt} elektriskt f{\"a}lt
  {\"a}r alltid noll,
  $$
    \nabla\times{\bf E}={\bf 0},
  $$
  vilket {\"a}r en {\it direkt f{\"o}ljd av formen p{\aa} Coulombs
  generaliserade lag},
  $$
    {\bf E}({\bf x})
      ={{1}\over{4\pi\varepsilon_0}}\iiint_V\rho({\bf x}')
        {{({\bf x}-{\bf x}')}\over{|{\bf x}-{\bf x}'|^3}}\,dV'.
  $$}
\item{$\bullet$}{Existensen av en skal{\"a}r elektrostatisk potential
  $\phi$ f{\"o}ljer direkt av att Coulombs generaliserade lag (eller
  {\it Coulombintegralen}) kan tolkas som en gradient av en skal{\"a}r
  funktion,
  $$
    {\bf E}({\bf x})
      ={{1}\over{4\pi\varepsilon_0}}\iiint_V\rho({\bf x}')
        {{({\bf x}-{\bf x}')}\over{|{\bf x}-{\bf x}'|^3}}\,dV'
      =-{{1}\over{4\pi\varepsilon_0}}\nabla\iiint_V
        {{\rho({\bf x}')}\over{|{\bf x}-{\bf x}'|}}\,dV'
      =-\nabla\phi({\bf x}),
  $$
  d{\"a}r den skal{\"a}ra potentialen utifr{\aa}n detta resultat kort och
  gott {\it definieras} som
  $$
    \phi({\bf x})\equiv{{1}\over{4\pi\varepsilon_0}}\iiint_V
        {{\rho({\bf x}')}\over{|{\bf x}-{\bf x}'|}}\,dV'.
  $$}
\item{$\bullet$}{F{\"o}r specialfallet med en punktladdning placerad i
  k{\"a}llpunkten ${\bf x}'$ motsvaras laddningst{\"a}theten av en
  delta-puls $\rho({\bf x})=q\delta({\bf x}-{\bf x}')$, med resulterande
  potential
  $$
    \phi({\bf x})
      ={{1}\over{4\pi\varepsilon_0}}\iiint_V
        {{q\delta({\bf x}-{\bf x}')}\over{|{\bf x}-{\bf x}'|}}\,dV'
      ={{1}\over{4\pi\varepsilon_0}}{{q}\over{|{\bf x}-{\bf x}'|}},
        \quad\hbox{f{\"o}r}\quad{\bf x}\ne{\bf x}'.
  $$}
\item{$\bullet$}{Arbetet $W$ som tillf{\"o}rs en punktladdning $q$ d{\aa}
  den f{\"o}rflyttas fr{\aa}n positionen ${\bf x}_a$ till ${\bf x}_b$
  i ett elektrostatiskt f{\"a}lt ges via den elektrostatiska potentialen
  $\phi({\bf x})$ som
  $$
    W=q\big(\phi({{\bf x}_b})-\phi({{\bf x}_a})\big)=W_b-W_a,
  $$
  d{\"a}r skillnaden $W_b-W_a$ {\"a}r skillnaden i potentiell energi f{\"o}r
  testladdningen mellan punkterna.}
\item{$\bullet$}{Den skal{\"a}ra potentialen $\phi$ lyder {\it Poissons
  ekvation} (f{\"o}rsta ekvationen p{\aa} omslaget p{\aa} Griffiths!),
  $$
    \nabla^2\phi({\bf x})=-\rho({\bf x})/\varepsilon_0.
  $$}

\cleardoublepage
%%% End of auto-extracted text from ../lect-02/lecture-02.tex %%%
%%% Begin of auto-extracted text from ../lect-03/lecture-03.tex %%%
%
% File: teach/elmagii/lect-03/lecture-03.tex [plain TeX code]
% Github: https://github.com/elmagii/lect-03/
% Last change: November 8, 2025
%
% Lecture No 3 in the course ``Elektromagnetism II, 1TE626 (2025)'',
% held November 10, 2025, at Uppsala University, Sweden.
%
% Copyright (C) 2022-2025, Fredrik Jonsson, under Gnu General Public
% License (GPL) v3. See the enclosed LICENSE for details.
%
% This program is free software: you can redistribute it and/or modify
% it under the terms of the GNU General Public License as published by
% the Free Software Foundation, either version 3 of the License, or
% (at your option) any later version.
%
% This program is distributed in the hope that it will be useful,
% but WITHOUT ANY WARRANTY; without even the implied warranty of
% MERCHANTABILITY or FITNESS FOR A PARTICULAR PURPOSE.  See the
% GNU General Public License for more details.
%
% You should have received a copy of the GNU General Public License
% along with this program.  If not, see <https://www.gnu.org/licenses/>.
%
\def\coursename{Elektromagnetism II}
\def\coursecode{1TE626}
\def\courseyear{2025}
\def\courserepo{https://github.com/hp35/elmagii/}
\def\lecturenumber{3}
\def\lecturetitle{Spegelladdningar, randvillkor och entydighet}
\def\lecturesubtitle{f{\"o}r l{\"o}sningar till potentialproblem}
\def\lectureauthor{Fredrik Jonsson}
\def\lectureplace{Uppsala Universitet}
\def\lecturedate{10 november 2025}
%-------------------- BEGIN OF LOCAL MACROS --------------------
\edef\expandedlecturenumber{3}
\def\ifempty#1{\ifx\relax#1\relax}
\advance\chapno by 1
\secno=0
\footnotenumber=0
\message{==================== Lecture 3 ====================}
\writenumberedtocentry{chapter}{F{\"o}rel{\"a}sning 3 -- {Spegelladdningar, randvillkor och entydighet}}{\thechapno}
\hsize=150mm\hoffset=4.6mm\vsize=230mm\voffset=7mm
\topskip=0pt\baselineskip=12pt\parskip=0pt\leftskip=0pt\parindent=15pt
\ifcolors
  \voffset=-10.2mm\topskip=0pt
\fi
\headline={\ifnum\secno>0\ifodd\pageno\rightheadline\else\leftheadline\fi
  \else\hfill\fi}
\def\rightheadline{\tenrm{\it F\"orel\"asning 3}
  \hfil{\it \coursename, \coursecode\ (\courseyear)}}
\def\leftheadline{\tenrm{\it \coursename, \coursecode\ (\courseyear)}
  \hfil{\it F\"orel\"asning 3}}
\noindent~\vskip-60pt\hskip-40pt{\epsfbox{../lect-01/macros/UU_logo_color.eps}}
\vskip-42pt\hfill\vbox{
    \hbox{{\it \coursename, \coursecode\ (\courseyear)}}
    \hbox{{\it Lecture Notes, \lectureauthor}}
    \hbox{{\it Document Revision \today}}
    \hbox{{\it \courserepo}}}\vskip 36pt
\centerline{\twelvesc F\"orel\"asning 3}
\vskip 24pt\noindent
\centerline{\twelvesc{Spegelladdningar, randvillkor och entydighet}}
\expandafter\ifempty\expandafter{\lecturesubtitle}%
  \else\centerline{\twelvesc\lecturesubtitle}\fi
\bigskip
\centerline{\lectureauthor, \lectureplace, \lecturedate}
\vskip24pt
%--------------------- END OF LOCAL MACROS ---------------------



\plan{Genom att visa p{\aa} att Laplaces ekvation $\nabla^2\phi=0$ saknar
  lokala extrempunkter, och har d{\"a}rmed samtliga extrempunkter i form
  av randv{\"a}rden till dom{\"a}nen d{\"a}r vi l{\"o}ser ekvationen, s{\aa}
  kan vi dra slutsatsen att en l{\"o}sning $\phi$ till Laplaces ekvation
  ocks{\aa} {\"a}r entydig, det vill s{\"a}ga att om vi finner en l{\"o}sning
  $\phi$ s{\aa} {\"a}r det ocks{\aa} den enda existerande l{\"o}sningen.
  Med utg{\aa}ngspunkt i detta finner vi d{\"a}refter att {\"a}ven Poissons
  ekvation $\nabla^2\phi=-\rho/\varepsilon_0$ i direkt n{\"a}rvaro av en
  laddningst{\"a}thet $\rho$ {\"a}ven den ger entydighet f{\"o}r l{\"o}sningar
  $\phi$.
  
  Vi kan med detta visa att en dom{\"a}n som {\"a}r omgiven av ett skal
  som h{\aa}lls vid konstant potential $\phi_0$ direkt ger att det elektriska
  f{\"a}ltet innanf{\"o}r skalet {\"a}r identiskt noll, vilket {\"a}r principen
  f{\"o}r ``Faradays bur''.
  
  Utifr{\aa}n entydighetsteoremet f{\"o}r Laplaces och Poissons ekvationer kan
  vi visa hur s{\aa} kallade ``virtuella spegelladdningar'' kan konstrueras
  f{\"o}r att l{\"o}sa elektrostatiska problem i n{\"a}rvaro av fria laddningar
  i elektriskt ledande volymer eller ytor, specifikt f{\"o}r plana ytor
  mellan ledare eller dielektrika samt ledande cylindrar och sf{\"a}rer.
  
  Slutligen s{\aa} tar vi fram en metod f{\"o}r hur den resulterande
  laddningst{\"a}theten $\sigma$ p{\aa} en yta av ledande material kan tas
  fram med de virtuella spegelladdningarna.}

\threepointsummary{%
  L{\"o}sningen till Laplaces ekvation $\nabla^2\phi=0$ i en godtycklig volym
  $V$ {\"a}r unikt (entydigt) best{\"a}md om potentialen $\phi$ {\"a}r
  specificerad p{\aa} randen $S$ till volymen.
  Om vi finner en l{\"o}sning till Laplace ekvation, s{\aa} {\"a}r detta den
  enda l{\"o}sningen, oavsett hur vi funnit eller konstruerat l{\"o}sningen
  s{\aa} l{\"a}nge som l{\"o}sningen uppfyller de f{\"o}reskrivna randvillkoren.
}{%
  L{\"o}sningen $\phi({\bf x})$ till Poissons ekvation f{\"o}r en punktladdning
  $q$ placerad ett avst{\aa}nd $z=h$ ovanf{\"o}r ett perfekt ledande plan $z=0$
  ges som frirymd\-l{\"o}s\-ningen med en virtuell spegelladdning $-q$ placerad
  p{\aa} samma avst{\aa}nd bakom planet, vid $z=-h$.
}{%
  L{\"o}sningen till Poissons ekvation f{\"o}r en linjeladdning $\lambda$
  placerad p{\aa} ett avst{\aa}nd $R+h$ fr{\aa}n centrum av en perfekt ledande
  cylinder med radie $R$ ges som fri\-rymds\-l{\"o}s\-ningen med en virtuell
  linjeladdning $\lambda'=-\lambda$ placerad p{\aa} avst{\aa}ndet
  $$
    d={{R^2}\over{R+h}},
  $$
  fr{\aa}n cylinderns centrum i riktning mot den externa linjeladdningen.
}
\vfill\eject\copyrights

\section{Laplaces ekvation f{\"o}r den elektrostatiska (skal{\"a}ra)
  potentialen}
\sidx{Laplaces ekvation}\sidx{Skal{\"a}r potential}[Elektrostatisk]
Det kan tyckas aningen {\"o}verdrivet att ge sig in p{\aa} mer matematiskt
betingade sp{\"o}rsm{\aa}l kring huruvida l{\"o}sningar till problem inom
elektrostatiken {\"a}r unika eller ej, det vill s{\"a}ga om det finns fler
{\"a}n en entydig l{\"o}sning till ett givet problem. Trots allt, s{\aa}
{\"a}r vi ju inte speciellt oroliga f{\"o}r att vi i det ``verkliga livet''
som ingenj{\"o}rer skall kunna r{\aa}ka ut f{\"o}r tv{\aa} olika f{\"a}lt
som uppfyller samma ekvation och randvillkor, right?

Fr{\aa}gan {\"a}r dock befogad rent generellt, och s{\aa} snart som vi
inkluderar ickelinj{\"a}ra fenomen i v{\aa}ra statiska eller dynamiska modeller,
exempelvis ett brytningsindex vars v{\"a}rde i sig beror p{\aa} elektrisk
f{\"a}ltstyrkan ${\bf E}$ eller en magnetisering som beror ickelinj{\"a}rt
p{\aa} den magnetiska fl{\"o}dest{\"a}theten ${\bf B}$, s{\aa} finns det gott
om bistabila l{\"o}sningar med olika f{\"a}ltf{\"o}rdelningar som kan uppfylla
samma ekvationer och randvillkor. Inom ramen f{\"o}r elektrostatiken i denna
f{\"o}rel{\"a}sningsserie behandlar vi dock uteslutande linj{\"a}ra problem i
det elektriska f{\"a}ltet, men som vi skall se finns det icke desto mindre
mycket att h{\"a}mta i praktiskt probleml{\"o}sande med bas i entydighet
f{\"o}r l{\"o}sningar till den elektrostatiska potentialen.

\subsection{Angreppss{\"a}tt~1 f{\"o}r elektrostatiska problem
  -- Coulombintegralen}
\sidx{Coulombs generaliserade lag}[Coulombintegralen]
\sidx{Skal{\"a}r potential}[Elektrostatisk]
Vi kan b{\"o}rja med att konstatera att vi i elektrostatiska problem alltsom
oftast s{\"o}ker en l{\"o}sning till det {\it elektriska f{\"a}ltet}, vilket vi
fr{\aa}n den f{\"o}rsta f{\"o}rel{\"a}sningen vet kan ber{\"a}knas genom
Coulombs generaliserade lag, eller {\it Coulombintegralen}\numberedfootnote{Vi
  f{\"o}ljer h{\"a}r i huvudsak Griffiths sid.~113--124.}\sidx{Elektrisk
f{\"a}ltstyrka}
$$
  {\bf E}({\bf x})={{1}\over{4\pi\varepsilon_0}}\iiint_V\rho({\bf x}')
    {{({\bf x}-{\bf x}')}\over{|{\bf x}-{\bf x}'|^3}}\,dV'.
$$
Oturligt nog f{\"o}r oss {\"a}r denna explicita form inte s{\"a}rskilt v{\"a}l
l{\"a}mpad vare sig f{\"o}r ber{\"a}kning med papper och penna eller med
numeriska metoder. En del av detta problem ligger i redundansen hos det
elektriska f{\"a}ltet\numberedfootnote{Vi visade f{\"o}rra
  f{\"o}rel{\"a}sningen p{\aa} kopplingen mellan komponenterna hos det
  elektriska f{\"a}ltet via $\nabla\times{\bf E}={\bf 0}$, resulterande
  i kopplingen ${{\partial E_x}/{\partial y}}={{\partial E_y}/{\partial x}}$,
  etc.},
d{\"a}r vi alltid har en implicit koppling mellan komponenterna hos det
elektriska f{\"a}ltet, och vi kan d{\"a}rf{\"o}r redan i detta f{\"o}rsta
stadie konstatera att det nog {\"a}r l{\"a}mpligare att ist{\"a}llet ``g{\aa}
till pudelns k{\"a}rna'' och ist{\"a}llet s{\"o}ka ber{\"a}kna den skal{\"a}ra
potentialen $\phi$.

\subsection{Angreppss{\"a}tt~2 f{\"o}r elektrostatiska problem
  -- Skal{\"a}r potential p{\aa} integralform}
Under F{\"o}rel{\"a}sning~2 fann vi att det elektriska f{\"a}ltet kunde tolkas
som en (negativ) gradient\numberedfootnote{Vi erinrar oss fr{\aa}n
  F{\"o}rel{\"a}sning~2 att tolkningen av det elektriska f{\"a}ltet som
  en negativ gradient av en skal{\"a}r potential kommer sig fr{\aa}n
  observationen att Coulomb-integralen ju kunde skrivas just som en gradient
  $$
    {\bf E}({\bf x})
      ={{1}\over{4\pi\varepsilon_0}}\iiint_V\rho({\bf x}')
        {{({\bf x}-{\bf x}')}\over{|{\bf x}-{\bf x}'|^3}}\,dV'
      =-{{1}\over{4\pi\varepsilon_0}}\nabla\iiint_V
        {{\rho({\bf x}')}\over{|{\bf x}-{\bf x}'|}}\,dV'
      =-\nabla\phi({\bf x}),
  $$}
$$
  {\bf E}({\bf x})=-\nabla\phi({\bf x}),
$$
d{\"a}r vi definierade den {\it skal{\"a}ra potentialen} $\phi({\bf x})$
explicit som integralen
$$
  \phi({\bf x})={{1}\over{4\pi\varepsilon_0}}\iiint_V
    {{\rho({\bf x}')}\over{|{\bf x}-{\bf x}'|}}\,dV'.
$$
{\AA}terigen, tyv{\"a}rr, {\"a}r denna explicita form inte s{\aa} v{\"a}ldigt
mycket b{\"a}ttre i analytiskt h{\"a}nseende annat {\"a}n f{\"o}r att
ber{\"a}kna potentialen i mycket enkla geometrier. Vidare, s{\aa} {\"a}r
integralformerna ovan prim{\"a}rt l{\"a}mpade f{\"o}r att ber{\"a}kna
resulterande f{\"a}lt och potentialer fr{\aa}n fixa, givna
laddnings\-f{\"o}r\-del\-ningar.
I en verklig situation kommer laddningsf{\"o}rdelningen att geometriskt flytta
sig i rummet s{\aa} snart som vi har n{\"a}rvaro av ledare med fria elektroner.
Med andra ord s{\aa} beh{\"o}ver vi g{\aa} ytterligare ett sn{\"a}pp innan vi
har n{\aa}got som handfast kan utnyttjas i fall d{\aa} laddningarna har frihet
att r{\"o}ra p{\aa} sig.

\subsection{Angreppss{\"a}tt~3 f{\"o}r elektrostatiska problem
  -- Laplaces ekvation f{\"o}r skal{\"a}r potential}
\sidx{Poissons ekvation}\sidx{Laplaces ekvation}
Som en intressant slutkl{\"a}m p{\aa} f{\"o}rra f{\"o}rel{\"a}sningen
konstaterade vi att Gauss lag p{\aa} differentialform,
$\nabla\cdot{\bf E}=\rho/\varepsilon_0$, trivialt kunde omformuleras till
Poissons ekvation f{\"o}r den skal{\"a}ra potentialen genom identiteten
${\bf E}=-\nabla\phi$, som
$$
  \nabla^2\phi({\bf x})=-\rho({\bf x})/\varepsilon_0.
$$
Evaluerad med l{\"a}mpliga randvillkor {\"a}r denna partiella
differentialekvation f{\"o}r $\phi$ ekvivalent med integralekvationen ovan.
Dessutom {\"a}r Poissons ekvation f{\"o}r $\phi$ synnerligen v{\"a}l l{\"a}mpad
f{\"o}r ber{\"a}kning i dom{\"a}ner d{\"a}r vi inte befinner oss mitt i
n{\aa}gon laddningst{\"a}thet, eftersom vi d{\aa} har att $\nabla^2\phi({\bf x})
=0$ med l{\"o}sningar best{\"a}mda av randvillkor, dessutom -- som vi strax
skall visa -- l{\"o}sningar som {\"a}r{\it entydigt} best{\"a}mda.
Att laddningst{\"a}theten i omr{\aa}det vi tittar p{\aa} {\"a}r fritt fr{\aa}n
laddningar betyder inte p{\aa} n{\aa}got vis att inga laddningar finns med i
problemet {\"o}verhuvud taget, bara att de {\it inte r{\aa}kar finnas
n{\"a}rvarande precis i observationspunkten} ${\bf x}$.
I dessa fall reduceras d{\"a}rmed problemet med f{\"a}ltbeskrivningen, eller
om man s{\aa} vill {\it potentialbeskrivningen}, till
{\it Laplaces ekvation}\numberedfootnote{Efter Pierre-Simon Laplace (1749--1827).
  \sidx{Laplace, Pierre-Simon (1749--1827)} L{\"o}sningen till Laplaces ekvation
  $\nabla^2\phi=0$ kallas ofta ``harmonisk funktion'' ({\it harmonic function}).}
f{\"o}r $\phi$,
$$
  \nabla^2\phi({\bf x})=0.
$$
Griffiths g{\aa}r s{\aa} l{\aa}ngt att beskriva Laplaces ekvation som s{\aa}
fundamental att l{\"a}ran om elektrostatik\sidx{Elektrostatik} praktiskt taget
{\it {\"a}r} studiet av just Laplaces ekvation, som ut{\"o}ver till{\"a}mpningen
i elektrostatik {\"a}r av samma form inom magnetism, gravitation och elastisk
mekanik.\numberedfootnote{Se
  {\tt https://www.robots.ox.ac.uk/{\char'176}jmb/lectures/pdelecture5.pdf}
  f{\"o}r detta sido\-sp{\aa}r.}

\section{Entydighet hos l{\"o}sningar till Laplace ekvation}
\sidx{Entydighetsteoremet}
Vi har s{\aa} l{\"a}ngt reducerat det elektrostatiska problemet till Laplaces
ekvation, m{\"o}jligen med brasklappen att vi i dom{\"a}ner d{\"a}r laddning
f{\"o}rekommer direkt i observationspunkten ist{\"a}llet m{\aa}ste anv{\"a}nda
Poissons ekvation, men enbart denna kompakta ekvation kommer inte att bist{\aa}
med l{\"o}sningen till potentialen. Det som ytterligare beh{\"o}vs {\"a}r
l{\"a}mpliga (fysikaliska) randvillkor \sidx{Randvillkor}[F{\"o}r Laplaces
ekvation] till den dom{\"a}n d{\"a}r vi {\"o}nskar l{\"o}sa
ekvationen.\numberedfootnote{Vi f{\"o}ljer h{\"a}r i huvudsak Griffiths
  sid.~119--124.}

\subsection{Laplaces ekvation till{\aa}ter bara extrempunkter p{\aa} randen
  till en dom{\"a}n}
\sidx{Randvillkor}[F{\"o}r Laplaces ekvation]
V{\aa}r plan fram{\"o}ver {\"a}r att skapa en differens $U=\phi_2-\phi_1$ mellan
tv{\aa} f{\"o}rmodat oberoende separata l{\"o}sningar $\phi_1$ och $\phi_2$ till
Laplaces ekvation, {\it uppfyllande exakt samma randv{\"a}rde}, och d{\"a}refter
s{\"o}ka motbevisa att $\phi_1$ och $\phi_2$ skulle vara olika genom att visa
p{\aa} att ekvationen $\nabla^2U=0$ f{\"o}r differensen dessutom betyder att
$U=0$ {\"o}verallt.
Innan vi g{\aa}r {\"o}ver till detta bevis skall vi f{\"o}rst visa p{\aa} en
annan egenskap hos l{\"o}sningar till Laplaces ekvation, n{\"a}mligen att
l{\"o}sningarna ej kan ha n{\aa}gra andra extrempunkter (minima eller maxima)
annat {\"a}n de som finns p{\aa} randytan $S$ till volymen $V$ d{\"a}r vi
betraktar ekvationen.
Med andra ord:\numberedfootnote{Vi hoppar h{\"a}r direkt in i fallet
  f{\"o}r tre dimensioner. Griffiths beskriver {\"a}ven en- och
  tv{\aa}\-dimen\-sion\-ella fallen (sid.~114--116), vilka kan vara
  l{\"a}mpliga att studera f{\"o}r att f{\aa} en mer intuitiv
  k{\"a}nsla f{\"o}r det tredimensionella fallet. Specifikt {\"a}r
  den tv{\aa}dimensionella analogin med ett uppsp{\"a}nt gummimembran
  en mycket pedagogisk illustration av Laplace ekvation och
  fr{\aa}nvaron av lokala extrempunkter. Teorem~I betecknas
  i internationell litteratur ofta som {\it Maximum (or minimum)
  principle for harmonic functions}.}
\sidx{Extrempunkter}[F{\"o}r l{\"o}sningar till Laplaces ekvation]
\quote{{\bf Teorem~I.}
  L{\"o}sningar till Laplaces ekvation saknar lokala extrempunkter.}
\noindent
Vi kommer nu att bevisa detta p{\aa}st{\aa}ende p{\aa} ett lite annorlunda
s{\"a}tt {\"a}n i Griffiths, d{\"a}r man ist{\"a}llet fokuserar p{\aa}
medelv{\"a}rde hos potentialen i en omgivning till en laddning (``k{\"a}lla'')
via Coulombs lag. H{\"a}r kommer vi ist{\"a}llet att anta en mer matematisk
approach.
\vfill\eject

\quote{{\bf Matematiskt bevis.} [Motbevis] Antag att den skal{\"a}ra
  potentialen $\phi$ {\it uppfyller Laplaces ekvation} $\nabla\phi^2=0$ och
  att den har ett lokalt {\it maximum} i punkten ${\bf x}'$.
  I denna punkt har vi d{\"a}rmed att:
  (1)~Alla f{\"o}rstaderivator f{\"o}rsvinner, det vill s{\"a}ga att
  $$
    \hbox{Extrempunkt }\quad\Rightarrow\quad
    \nabla\phi({\bf x}')={\bf 0},
  $$
  samt (2) att alla andraderivator {\"a}r negativa i en omgivning
  till ${\bf x}'$,
  $$
    \hbox{Maximum }\quad\Rightarrow\quad
    {{\partial^2\phi}\over{\partial x^2_k}}<0,\quad k=1,2,3,
  $$
  detta eftersom vi har att g{\"o}ra med ett {\it maximum} och att
  funktionen $\phi$ d{\"a}rmed m{\aa}ste ``halka ned{\aa}t'' i alla
  riktningar runt ${\bf x}'$.
  Genom att summera upp alla andraderivator erh{\aa}ller vi
  \sidx{Extrempunkter}[F{\"o}r l{\"o}sningar till Laplaces ekvation]
  $$
    \hbox{Laplace }\quad\Rightarrow\quad
    \sum^3_{k=1}{{\partial^2\phi}\over{\partial x^2_k}}=\nabla^2\phi<0.
  $$
  Laplace ekvation s{\"a}ger dock att $\nabla^2\phi=0$, vilket endast
  kan vara uppfyllt om {\it samtliga} andraderivator {\"a}r noll,
  vilket ger en mots{\"a}gelse.\numberedfootnote{Notera att om vi skulle
    r{\aa}ka ha situationen att vissa andraderivator {\"a}r negativa och andra
    positiva, s{\aa} kan vi ju fortfarande ha att $\nabla^2\phi({\bf x}')=0$
    vid en punkt d{\"a}r $\nabla\phi({\bf x}')={\bf 0}$.
    Detta {\"a}r dock i s{\aa} fall ej l{\"a}ngre n{\aa}gon {\it extrempunkt},
    utan snarare en {\it sadelpunkt}. Att $\nabla\phi({\bf x}')={\bf 0}$ {\"a}r
    ett {\it n{\"o}dv{\"a}ndigt men ej till r{\"a}ckligt villkor} f{\"o}r
    att vi skall ha att g{\"o}ra med en extrempunkt.}
  {\it Den enda m{\"o}jligheten {\"a}r d{\"a}rmed att $\phi$ inte kan ha ett
  lokalt maximum i en dom{\"a}n d{\"a}r $\nabla^2\phi=0$.} Det motsvarande
  motbeviset f{\"o}r lokala minima f{\"o}ljer analogt ur detta.}
\noindent
L{\aa}t oss {\"a}ven g{\aa} igenom ett alternativt, aningen mer fysikaliskt
fokuserat bevis.
\quote{{\bf ``Fysikaliskt'' bevis.} Vi har att $\phi$ {\"a}r en l{\"o}sning
  till Laplaces ekvation f{\"o}r den elektrostatiska potentialen, en ekvation
  vilken vi rekapitulerar per definition beskriver en {\it laddningsfri region},
  med $\rho({\bf x})=0$ f{\"o}r alla punkter ${\bf x}$ i volymen $V$.
  Detta betyder att f{\"o}r {\it godtyckligt vald punkt} ${\bf x}$ i
  volymen $V$, s{\aa} {\"a}r potentialen given som ett {\it medelv{\"a}rde}
  av potentialen i en omgivning av observationspunkten, s{\"a}g i form av
  en ``tillr{\"a}ckligt liten'' sf{\"a}r $S=\{{\bf x}:|{\bf x}-{\bf x}'|=r\}$
  med radien $r$ centrerad i punkten ${\bf x}$ som
  \sidx{Extrempunkter}[F{\"o}r l{\"o}sningar till Laplaces ekvation]
  $$
    \phi({\bf x})={{1}\over{4\pi r^2}}\oiint_{S}\phi({\bf x}')\,dS'.
  $$
  Med andra ord {\"a}r v{\"a}rdet f{\"o}r potentialen vid ${\bf x}$ ett
  medelv{\"a}rde av potentialen i alla n{\"a}rliggande punkter p{\aa} den
  omgivande sf{\"a}ren. Ett medelv{\"a}rde kan dock aldrig vara st{\"o}rre
  {\"a}n det maximala v{\"a}rdet eller mindre {\"a}n det minimala v{\"a}rdet
  f{\"o}r potentialen p{\aa} denna sf{\"a}r, med andra ord kan inga lokala
  extremv{\"a}rden finnas f{\"o}r $\phi$ i dom{\"a}nen d{\"a}r den uppfyller
  Laplaces ekvation $\nabla^2\phi=0$.}
\bigskip
\noindent
Intuitiv fysikalisk tolkning: Att $\nabla^2\phi=0$ i volymen $V$ betyder att det
inte finns n{\aa}gra k{\"a}llor ({\it sources}) eller s{\"a}nkor ({\it sinks})
f{\"o}r potentialen i $V$, vilket vi kan {\"o}vers{\"a}tta till att det inte
finns n{\aa}gon m{\"o}jlighet att ``bygga upp toppar'' eller ``dr{\"a}nera
dalar'' inom dom{\"a}nen. D{\"a}rf{\"o}r kan potentialen $\phi$ inte heller ha
n{\aa}gra toppar eller dalar inom dom{\"a}nen annat {\"a}n p{\aa} randen $S$.
\vfill\eject

\subsection{F{\"o}rsta entydighetsteoremet f{\"o}r den elektrostatiska
  potentialen}
\sidx{Entydighetsteoremet}[F{\"o}r Laplaces ekvation]
\quote{{\bf Teorem~II.} [{\it First uniqueness theorem} enligt Griffiths]
  L{\"o}sningen till {\it Laplaces ekvation} $\nabla^2\phi=0$ i en godtycklig
  volym $V$ {\"a}r entydigt (unikt) best{\"a}md om potentialen $\phi$ {\"a}r
  specificerad p{\aa} randen $S$ till volymen. [Griffiths, sid.~119]}
\epsfig{../lect-03/figs/unique.1}\noindent
\quote{{\bf Bevis.}\numberedfootnote{Ett alternativt bevis g{\aa}s
    igenom av exempelvis J.~D.~Jackson, {\it Classical Electrodynamics},
    \sidx{Jackson, John David (1925--2016)}[{{\it Classical Electrodynamics}}]
    med anv{\"a}ndandet av Greens f{\"o}rsta teorem,
    \sidx{Greens f{\"o}rsta teorem} som f{\"o}r godtyckliga
    funktioner $\varphi$ och $\psi$ (d{\"a}r vi allts{\aa} skriver
    ``$\varphi$'' ist{\"a}llet f{\"o}r ``$\phi$'' f{\"o}r att undvika
    f{\"o}rv{\"a}xling) lyder
    $$
      \iiint_V\big(\varphi\nabla^2\psi+(\nabla\varphi)\cdot(\nabla\psi)\big)\,dV
        =\oiint \varphi{{\partial\psi}\over{\partial n}}\,dS.
    $$
    Vi definierar i samma notation som Griffiths differensen $U\equiv\phi_2
    -\phi_1$ i en sluten volym $V$ med randen $S$, med egenskaperna
    $\nabla^2U=0$ inuti $V$ samt att $U=0$ och $\partial U/\partial n=0$
    p{\aa} randen $S$, d{\"a}r ${{\partial U}/{\partial n}}$ betecknar
    normalderivatan av $U$ p{\aa} densamma. Vi erh{\aa}ller d{\aa} med
    Greens f{\"o}rsta teorem, med $\varphi=\psi=U$, att
    $$
      \iiint_V\big(U\nabla^2U+(\nabla U)\cdot(\nabla U)\big)\,dV
        =\oiint U{{\partial U}\over{\partial n}}\,dS.
    $$
    Med egenskaperna hos $U$ reduceras detta till
    $$
      \iiint_V|\nabla U|^2\,dV=0,
    $$
    det vill s{\"a}ga att $U$ {\"o}verallt {\"a}r konstant. F{\"o}r
    Dirichlet-randvillkor $U=0$ p{\aa} randen $S$ till $V$ (eftersom vi
    kr{\"a}ver samma v{\"a}rden f{\"o}r $\phi_1$ och $\phi_2$ p{\aa} $S$),
    betyder detta att {\"o}verallt i $V$ {\"a}r $\phi_1=\phi_2$, vilket i
    sin tur visar att en l{\"o}sning till Laplaces ekvation i $V$ alltid
    {\"a}r unik. Greens teorem och Greensfunktioner\sidx{Greensfunktion}
    {\"a}r dock utanf{\"o}r vad denna kurs omfattar.}
  Antag att vi har {\it tv{\aa}} av varandra oberoende l{\"o}sningar $\phi_1$
  och $\phi_2$ till Laplaces ekvation i en volym $V$, uppfyllande
  $\nabla^2\phi_1=0$ och $\nabla^2\phi_2=0$, och med {\it b{\"a}gge
  l{\"o}sningarna antagande samma v{\"a}rden p{\aa} randen} $S$ till $V$.
  Vi kan i detta {\"a}ven inkludera ett godtyckligt antal av eventuella
  interna randytor $S'$. Vi definierar differensen mellan de tv{\aa}
  hypotetiska oberoende l{\"o}sningarna som
  $$
    U\equiv\phi_2-\phi_1,
  $$
  som trivialt {\"a}ven den satisfierar Laplaces ekvation
  $$
    \nabla^2 U=\underbrace{\nabla^2\phi_2}_{=0}-\underbrace{\nabla^2\phi_1}_{=0}=0
  $$
  {\"o}verallt i volymen $V$. Specifikt har differensen v{\"a}rdet $U=0$ p{\aa}
  alla punkter som tillh{\"o}r randen $S$ (eller f{\"o}r den delen p{\aa}
  interna randytor $S'$) till $V$,
  $$
    U({\bf x})=0,\qquad{\bf x}\in S.
  $$
  Som vi just sett i Teorem~I ovan till{\aa}ter dock Laplaces ekvation inga
  lokala extrempunkter inuti $V$ (specifikt f{\"o}r differensen $U$ som ju
  lyder Laplace ekvation) och med andra ord {\"a}r
  $$
    \min(U({\bf x}))=\max(U({\bf x}))=0
    \quad\Leftrightarrow\quad
    U({\bf x})\equiv0
    \quad\Leftrightarrow\quad
    \phi_1({\bf x})=\phi_2({\bf x}),
  $$
  vilket d{\"a}rmed leder till slutsatsen att enda m{\"o}jligheten {\"a}r att
  det endast finns en enda unik (entydig) l{\"o}sning f{\"o}r den skal{\"a}ra
  potentialen $\phi$.}

\quote{{\bf F{\"o}ljdsats} [Inkludering av laddning i volymen $V$]
  L{\"o}sningen till {\it Poissons ekvation}\numberedfootnote{``Laplaces
    ekvation fast med en k{\"a}lla i h{\"o}gerledet.''}
  $\nabla^2\phi=-\rho/\varepsilon_0$ i en godtycklig volym $V$ {\"a}r unikt
  best{\"a}md om (I) potentialen $\phi$ {\"a}r specificerad p{\aa} randen $S$
  till volymen samt (II) laddningst{\"a}theten $\rho$ i dom{\"a}nen {\"a}r
  specificerad. [Griffiths, sid.~121]
  \sidx{Entydighetsteoremet}[F{\"o}r Poissons ekvation]}
\quote{{\bf Bevis} Vi har just visat att i en dom{\"a}n d{\"a}r inga laddningar
  finns, s{\aa} {\"a}r en l{\"o}sning till {\it Laplaces} ekvation entydig.
  Vad h{\"a}nder d{\aa} om vi l{\"a}gger till en godtycklig laddningst{\"a}thet
  $\rho({\bf x})$ i volymen, s{\aa} att vi i sj{\"a}lva verket har att g{\"o}ra
  med {\it Poissons} ekvation?
  I detta fall f{\"o}ljer vi samma argument som tidigare, med en differens
  $U=\phi_2-\phi_1$, men d{\"a}r de tv{\aa} potentialerna ist{\"a}llet
  f{\"o}ljer\sidx{Entydighetsteoremet}[F{\"o}r Poissons ekvation]
  $$
    \nabla^2\phi_1=-\rho/\varepsilon_0,\qquad
    \nabla^2\phi_2=-\rho/\varepsilon_0,
  $$
  s{\aa} att differensen $U$ i dom{\"a}nen $V$ liksom tidigare blir
  $$
    \nabla^2U
      =\nabla^2\phi_2-\nabla^2\phi_1
      =-\rho/\varepsilon_0-(-\rho/\varepsilon_0)=0.
  $$
  {\AA}terigen satisfierar $U({\bf x})$ Laplaces ekvation och har liksom
  tidigare v{\"a}rdet noll f{\"o}r alla punkter p{\aa} randen,
  $$
    U({\bf x})=0,\qquad{\bf x}\in S.
  $$
  Med exakt samma argument kring $\min(U)=\max(U)=0$ som tidigare, s{\aa} drar
  vi d{\"a}rmed slutsatsen att {\"a}ven n{\"a}r laddningar {\"a}r n{\"a}rvarande
  i dom{\"a}nen, s{\aa} g{\"a}ller entydighetsteoremet. Notera att ``n{\"a}rvaro
  av laddningar'' h{\"a}r m{\aa}ste tolkas som ``n{\"a}rvaron av laddningar som
  {\"a}r station{\"a}ra i rummet'', det vill s{\"a}ga att eventuella laddningar
  som {\"a}r r{\"o}rliga i metaller eller liknande m{\aa}ste ha uppn{\aa}tt
  j{\"a}mvikt med det omgivande elektrostatiska f{\"a}ltet.}

\subsection{Elektriska ledare och andra entydighetsteoremet f{\"o}r
  den elektrostatiska potentialen}
\quote{{\bf Teorem~III.} [{\it Second uniqueness theorem} enligt Griffiths]
  I en volym $V$ omgiven av ledare och inneslutande en laddningst{\"a}thet
  $\rho$, {\"a}r det elektriska f{\"a}ltet unikt best{\"a}mt om den totala
  laddningen p{\aa} varje ledare {\"a}r given. Hela dom{\"a}nen kan vara
  begr{\"a}nsad av en annan ledare, alternativt obegr{\"a}nsad.
  [Griffiths, sid.~121]}
\quote{{\bf Bevis.} [{\it Att eventuellt inkluderas i dessa Lecture Notes.
  Se Griffiths sid.~121--123.}]}
\par\vfill\eject

\section{Faradays bur}
\sidx{Faradays bur}
Som en direkt f{\"o}ljd av Teorem~II -- g{\"a}llande att l{\"o}sningen till
Laplaces ekvation $\nabla^2\phi=0$ i en godtycklig volym $V$ {\"a}r unikt
best{\"a}md om potentialen $\phi$ {\"a}r specificerad p{\aa} randen $S$ till
volymen -- s{\aa} f{\"o}ljer det att om en perfekt ledare omsluter en dom{\"a}n
$V$, med andra ord att den omslutande ytan {\"o}verallt {\"a}r knuten till samma
konstanta potential $\phi=\phi_0$, {\it s{\aa} {\"a}r den elektriska potentialen
konstant {\"o}verallt inuti volymen}, givet att ingen laddning omsluts.
\epsfig{../lect-03/figs/faradaycage.1}\noindent
Av detta f{\"o}ljer trivialt att det elektriska f{\"a}ltet i hela volymen {\"a}r
identiskt noll, eftersom vi f{\"o}r $\phi=\phi_0=\hbox{konstant}$ p{\aa} $S$ har
att
$$
  {\bf E}=-\nabla\phi_0=0
$$

{\it Exempel~I:} Luckan till en mikrov{\aa}gsugn fungerar som d{\"o}rr till en
Faraday-bur, d{\"a}r det elektriskt ledande n{\"a}tet i luckan har h{\aa}l
som {\"a}r v{\"a}sentligt mindre {\"a}n v{\aa}gl{\"a}ngden f{\"o}r det
elektromagnetiska f{\"a}lt som utg{\aa}r mikrov{\aa}gorna. (Mikrov{\aa}gor
som anv{\"a}nds i en mikrov{\aa}gsugn\numberedfootnote{Det {\"a}r en modern
  myt att denna frekvens skulle ha valts f{\"o}r att den skulle matcha
  resonansfrekvensen hos vattenmolekylen; i sj{\"a}lva verket handlar
  det snarare om att vatten {\"a}r mycket bredbandigt i sitt
  absorptionsspektrum och att 2.4~GHz, vilket f{\"o}r {\"o}vrigt
  {\"a}ven anv{\"a}nds av Bluetooth$^{\rm TM}$, {\"a}r en praktisk och
  enkel frekvens att generera med en kompakt {\it magnetron}.
  Till exempel, om vatten hade haft en resonansfrekvens p{\aa} just
  2.4~GHz (den f{\"o}rsta resonansfrekvensen ligger i sj{\"a}lva verket
  p{\aa} 22.235~GHz), s{\aa} skulle all effekt effektivt absorberas bara
  p{\aa} ytan av det som v{\"a}rms, med v{\"a}rme p{\aa} djupet endast
  genom en termisk diffusionsprocess. (En process som i steady-state
  {\aa}terigen beskrivs av just -- tadaa! -- Laplaces ekvation!)}
har typiskt en frekvens p{\aa} cirka $2.4\ {\rm GHz}$, vilket ger en
v{\aa}gl{\"a}ngd p{\aa} cirka $(3\times10^8\ {\rm m}/{\rm s})/(2.4\times10^9
\ 1/{\rm s})\approx 12.5\ {\rm cm}$). Genom att det elektromagnetiska
f{\"a}ltet inte kan transmitteras genom n{\"a}tet (eller h{\aa}lmatrisen)
s{\aa} {\"a}r mikrov{\aa}gsugnen fortfarande s{\"a}ker trots att man kan
se igenom luckan.\sidx{Elektromagnetisk v{\aa}gl{\"a}ngd}

{\it Exempel~II:} En bilkaross s{\"a}gs effektivt skydda mot blixtnedslag, men
vad om de ganska stora {\"o}ppna glasytorna? Riskerar inte ett blixtnedslag att
leta sig in i kup\'en genom glasrutorna? L{\"o}sningen ligger {\"a}ven h{\"a}r
i att betrakta vilken v{\aa}gl{\"a}ngd den elektromagnetiska pulsen har:
\item{$\bullet$}{Den dominerande v{\aa}gl{\"a}ngden\sidx{Elektromagnetisk
  v{\aa}gl{\"a}ngd} hos det emitterade elektromagnetiska f{\"a}ltet i ett
  blixtnedslag ligger i VLF-spannet\numberedfootnote{VLF = Very Low Frequency,
    {\tt https://en.wikipedia.org/wiki/Very\_low\_frequency}.}
  3--30~kHz, motsvarande v{\aa}gl{\"a}ngder p{\aa} 10--100~km.
  Dessa v{\aa}gl{\"a}ngder {\"a}r mycket st{\"o}rre {\"a}n
  {\"o}ppningen i det ledande skalet hos bilen, vilka typiskt {\"a}r
  i storleksordningen 1~m.\numberedfootnote{V{\"a}rldsarvet radiostationen
    Grimeton, Varberg, har regelbundet s{\"a}ndningar p{\aa} 17.2~kHz
    (ja, {\it kHz}, inte {MHz}), eller en elektromagnetisk v{\aa}gl{\"a}ngd
    p{\aa} cirka 17.4~km, och vi kan med detta dra slutsatsen att vi ej
    kan uppf{\aa}nga Grimetons uts{\"a}ndningar inne i en bilkup\'e!
    {\tt https://grimeton.org}}}
\item{$\bullet$}{Blixtnedslaget har naturligtvis ocks{\aa} synligt ljus med
  en v{\aa}gl{\"a}ngd av cirka 500~nm i sitt spektrum; det {\"a}r ju trots
  allt s{\aa} att vi tydligt ser blixtnedslaget, vilket {\"a}r en f{\"o}ljd
  av att denna del av spektrum har en v{\aa}gl{\"a}ngd som {\"a}r markant
  mindre {\"a}n f{\"o}nster{\"o}ppningarna i bilens kaross.}
\item{$\bullet$}{Ut{\"o}ver detta inneh{\aa}ller urladdningen {\"a}ven
  extremt kortv{\aa}gig gammastr{\aa}lning ({\it gamma-ray bursts}, GRB).}
\vfill\eject

\section{Det elektriska f{\"a}ltet {\"a}r identiskt noll i en perfekt ledare,
  s{\aa} hur kan vi d{\aa} skapa str{\"o}mmar?}
F{\"o}r en perfekt ledare\sidx{Perfekt ledare} i elektrostatisk j{\"a}mvikt
{\"a}r potentialen $\phi({\bf x})$ rent definitionsm{\"a}ssigt {\"o}verallt
densamma f{\"o}r alla punkter ${\bf x}$ inuti ledarens volym. Detta f{\"o}ljer
av samma argument som f{\"o}r Faradays bur, n{\"a}mligen att om
$$
  \phi({\bf x})=\phi_0=\hbox{konst.}
$$
{\"o}verallt inuti ledaren, s{\aa} har vi trivialt att det elektriska f{\"a}ltet
i den pefekta ledaren {\"a}r identiskt noll,
$$
  {\bf E}({\bf x})=-\nabla\phi_0={\bf 0}.
$$
Fr{\aa}gan som d{\aa} direkt infinner sig {\"a}r paradoxen kring hur elektroner
{\"o}verhuvudtaget kan f{\aa}s att r{\"o}ra sig i just en perfekt ledare,
eftersom kraftlagen ju d{\aa} direkt ger vid hand att
${\bf F}=q{\bf E}={\bf 0}$. Hur h{\"a}nger detta ihop?
Vi vet ju trots allt att just en {\it perfekt ledare} {\"a}r ett idelaiskt
medium f{\"o}r laddningstransport, eller?

Ett annat s{\"a}tt att se p{\aa} detta problem {\"a}r att en perfekt ledare
{\"a}r ett medium d{\"a}r konduktiviteten $\sigma\to\infty$, men Ohms lag
f{\"o}r str{\"o}mt{\"a}theten s{\"a}ger oss samtidigt att
${\bf J}=\sigma{\bf E}$. Om $\sigma\to\infty$ s{\aa} m{\aa}ste vi f{\"o}r en
finit str{\"o}mt{\"a}thet ${\bf J}$ sj{\"a}lvfallet ha att ${\bf E}\to{\bf 0}$.
\quote{{\it Fr{\aa}gan {\"a}r hur kan vi skapa en str{\"o}m om
  ${\bf E}={\bf 0}$?}}
\noindent
L{\"o}sningen till denna paradox {\"a}r som f{\"o}ljer. I en perfekt ledare
kan inget elektriskt f{\"a}lt existera i j{\"a}mviktstillst{\aa}nd
({\it steady-state}) , med h{\"a}nsyn till ovan. Dock kan transienta (snabba
tidsberoende) elektriska f{\"a}lt existera f{\"o}r den infinitesimala tidsrymd
som det tar f{\"o}r laddningarna l ledaren att flytta sig till dess att
j{\"a}mviktstillst{\aa}ndet uppn{\aa}s. D{\aa} man applicerar ett externt
p{\aa}lagt elektriskt f{\"a}lt, s{\"a}g genom att l{\"a}gga p{\aa} en
potentialskillnad {\"o}ver {\"a}ndytorna p{\aa} ledaren, kommer elektronerna
i ledaren att omedelbart ({\aa}tminstone ``omedelbart sett som en idelaiserad
modell'') s{\"a}ttas i r{\"o}relse, prim{\"a}rt p{\aa} ytan av ledaren, med
effekten att det elektriska f{\"a}ltet {\"o}ver ledaren kancelleras s{\aa}
snart som j{\"a}mvikt uppn{\aa}tts.

Denna r{\"o}relse av laddning under det korta f{\"o}rloppet innan j{\"a}mvikt
uppn{\aa}s, som i sig {\"a}r ett brott mot antagandet om just statisk
j{\"a}mvikt, {\"a}r just att vi skapar en str{\"o}m fram till dess att vi
n{\aa}r ${\bf E}={\bf O}$ inuti ledaren. Efter att j{\"a}mvikt uppn{\aa}tts har
vi f{\"o}rvisso en laddningsf{\"o}rdelning som skapar just ${\bf E}={\bf 0}$,
men vi har under detta f{\"o}rlopp ocks{\aa} ocks{\aa} en str{\"o}m som i sig
g{\"o}r att vi {\it l{\"a}mnat antagandet om elektrostatik}.

\section{Sammanfattning av vitsen med entydighetsteoremet}
Griffiths sammanfattar p{\aa} sid.~120 (sektion 3.1.5) vitsen med hela denna
exercis kring entydighet mycket precist:\numberedfootnote{``The uniqueness
  theorem is a license to your imagination. It doesn't matter {\it how} you
  come by your solution; if (a) it satisfies Laplace's equation and (b) it
  has the correct value on the boundaries, then it's {\it right}.''}
\quote{``Entydighetsteoremet {\"a}r en licens till din fantasi. Det spelar
  ingen roll {\it hur} du finner din l{\"o}sning; om den (a) uppfyller
  Laplaces ekvation och (b) har korrekt v{\"a}rde p{\aa} randen, s{\aa}
  {\"a}r den {\it korrekt}.''}
\noindent
Vi kommer nu att utnyttja v{\aa}r erh{\aa}llna licens p{\aa} l{\"o}sandet av
problem med hj{\"a}lp av s{\aa} kallade {\it spegelladdningar}.
\vfill\eject

\section{Spegelladdningar i perfekt ledande plan}
\sidx{Spegelladdning}\sidx{Spegelladdning}[Perfekt ledande plan]
S{\aa} l{\aa}ngt har vi haft att g{\"o}ra med givna laddningar och
laddningsf{\"o}rdelningar, fixt placerade i rummet. Vad h{\"a}nder om
vi har ledare som till{\aa}ter laddningsf{\"o}rdelningar att relaxera till
steady-state, men d{\"a}r vi inte p{\aa} rak arm vet exakt {\it hur} dessa
laddningar kommer att placera sig? Vi kommer nu att visa p{\aa} ett s{\"a}tt
att angripa s{\aa}dana problem, med metoden f{\"o}r {\it spegelladdningar}.

Antag att vi har en punktladdning $q$, av godtyckligt tecken p{\aa} laddningen
och placerad p{\aa} ett avst{\aa}nd $h$ ovan ett o{\"a}ndligt och perfekt
ledande plan $z=0$.
\epsfig{../lect-03/figs/mirrorplane.1}\noindent
Att planet $z=0$ {\"a}r perfekt ledande betyder att laddningar fritt kan
transporteras i planet utan f{\"o}rlust, och om vi t{\"a}nker oss att laddningen
$q$ {\"a}r positiv, s{\aa} inneb{\"a}r detta att negativa laddningar (via
Coulombs kraftlag) i planet $z=0$ kommer att attraheras mot origo $(x,y)=(0,0)$.
Det som hindrar den negativa laddningen att ackumuleras just vid punkten
$(x,y)=(0,0)$ i planet under laddningen $q$ {\"a}r att denna ansamling av
negativ laddning ocks{\aa} inneb{\"a}r att laddningarna i planet kommer att
repellera varandra till dess att j{\"a}mvikt uppst{\aa}r.
I praktiken inneb{\"a}r sj{\"a}lvfallet ``transport av negativa laddningar''
(elektroner) att positiv laddning (vakanser av elektroner) skyfflas undan,
eller repelleras fr{\aa}n den netto-negativa laddningen; vi kan se detta som
att vi har en tv{\aa}dimensionell ``gas'' av fria elektroner i ytan.
\vfill\eject

\subsection{Problemformulering f{\"o}r elektrostatik ovanf{\"o}r perfekt
  ledande plan}
\sidx{Spegelladdning}[Perfekt ledande plan]
Rent matematiskt formuleras detta problem enligt f{\"o}ljande:
\smallskip
\item{1.}{Vi har laddningst{\"a}theten $\rho=q\delta({\bf x}-h{\bf e}_z)$
  i rummet $z>0$. (K{\"a}llpunkten {\"a}r vid ${\bf x}'=h{\bf e}_z$.)}
\item{2.}{Den skal{\"a}ra potentialen {\it uppfyller Laplaces ekvation}
  $\nabla^2\phi=0$ {\"o}verallt i $z>0$ utom just i
  ${\bf x}'=h{\bf e}_z$ f{\"o}r punktladdningen.}
\item{3.}{P{\aa} randen $z=0$ m{\aa}ste potentialen {\it uppfylla
  randvillkoret $\phi=0$}.}
\item{4.}{P{\aa} ett avst{\aa}nd l{\aa}ngt fr{\aa}n laddningen
  f{\"o}rv{\"a}ntar vi oss $\phi({\bf x})\to0$, f{\"o}r $|{\bf x}|\to\infty$.}
\smallskip
\noindent
Utifr{\aa}n f{\"o}ljdsatsen till entydighetsteoremet f{\"o}r Laplaces och
Poissons ekvation kan vi lita p{\aa} att {\it om vi finner en l{\"o}sning
till problemet ovan, s{\aa} {\"a}r det den korrekta l{\"o}sningen, oavsett
hur vi kommit fram till den.} S{\aa}, hur skall vi resonera h{\"a}r f{\"o}r
att finna denna l{\"o}sning?

\subsection{Att l{\"a}gga till virtuell laddning s{\aa} att vi n{\aa}r
  randvillkoret $\phi=0$ p{\aa} $z=0$}
Om vi utf{\"o}r ett litet {\it gedankenexperiment} kring hur vi till att
b{\"o}rja med skulle {\it skapa} en l{\"o}sning som satisfierar $\phi=0$ p{\aa}
ytan $z=0$, s{\aa} kan vi se laddningen i $h{\bf e}_z$ som en k{\"a}lla till
potentialen, som s{\aa} att s{\"a}ga ``lyfts'' till en viss f{\"o}rdelning i
rummet $z>0$. Detta ``lyft'' kan vi f{\"o}rest{\"a}lla oss som till{\"a}mpbart
p{\aa} en positiv laddning $q$, men tanke-experimentet {\"a}r sj{\"a}lvfallet
lika giltigt f{\"o}r negativa laddingar. Fr{\aa}gan {\"a}r hur vi trycker
ner denna potential s{\aa} att randvillkoret $\phi=0$ erh{\aa}lls p{\aa} $z=0$?
\epsfig{../lect-03/figs/mirrorplane-gedanken.1}\noindent
L{\aa}t oss t{\"a}nka ``utanf{\"o}r l{\aa}dan'' och f{\"o}rest{\"a}lla oss att
vi hade friheten att l{\"a}gga en laddning med motsatt tecken n{\aa}gonstans i
regionen $z<0$, under det att vi samtidigt tar bort det elektriskt ledande
planet\numberedfootnote{Vi erinrar oss att det ju {\"a}r de fria elektronerna
  i den perfekta ledaren som h{\"a}r prim{\"a}rt st{\"a}ller till det f{\"o}r
  oss, och som vi kan rekapitulera var sk{\"a}let till att vi i b{\"o}rjan av
  denna f{\"o}rel{\"a}sning i praktiken spolade anv{\"a}ndningen av
  potentialformen av Coulombs lag,
  $$
    \phi({\bf x})={{1}\over{4\pi\varepsilon_0}}\iiint_V
      {{\rho({\bf x}')}\over{|{\bf x}-{\bf x}'|}}\,dV',
  $$
  just p{\aa} grund av att vi {\it a priori} inte har full information kring
  laddningsf{\"o}rdelningen $\rho({\bf x})$.}
vid $z\le0$.

Genom ren symmetri b{\"o}r i s{\aa} fall en laddning av samma belopp men motsatt
tecken rimligen placeras p{\aa} exakt samma avst{\aa}nd fr{\aa}n ytan, men
ist{\"a}llet i negativ $z$-led, vid punkten $-{\bf e}_z$.
Vi inser att detta skulle inneb{\"a}ra att $\phi=0$ vid $z=0$ p{\aa} grund av
antisymmetrin i det nya problemet, men vi b{\"o}r undvika att f{\"o}rlita oss
p{\aa} intuition s{\aa} l{\aa}t oss f{\"o}r s{\"a}kerhets skull kontrollera den
resulterande potentialen.
Om nu vi genom detta uppfyller potentialen p{\aa} $z=0$, s{\aa} kan vi helt
sonika ta bort det ledande planet och ist{\"a}llet betrakta $z=0$ enbart som
en yta som genom ``magisk konstruktion'' uppfyller just randvillkoret $\phi=0$.
\vfill\eject

P{\aa} planet $z=0$, om nu nu t{\"a}nker oss att vi tagit bort den perfekt
ledande ytan i denna beskrivning med virtuella spegelladdningar, s{\aa} {\"a}r
potentialen given fr{\aa}n de tv{\aa} laddningarna $q$ och $-q$
som\numberedfootnote{Recap: Med en punktladdning placerad i
  k{\"a}llpunkten ${\bf x}'$ blir den resulterande skal{\"a}ra potentialen
  $$
    \phi({\bf x})={{1}\over{4\pi\varepsilon_0}}{{q}\over{|{\bf x}-{\bf x}'|}},
        \quad\hbox{f{\"o}r}\quad{\bf x}\ne{\bf x}'.
  $$}
$$
  \phi(x,y,z=0)={{1}\over{4\pi\varepsilon_0}}\bigg(
    {{q}\over{\sqrt{x^2+y^2+h^2}}}+{{(-q)}\over{\sqrt{x^2+y^2+h^2}}}
  \bigg)=0,
$$
med andra ord har vi nu {\it visat att randvillkoret vid $z=0$ {\"a}r
uppfyllt.}

\subsection{Skal{\"a}r potential med det ledande planet ersatt med virtuell
  spegelladdning}
\sidx{Spegelladdning}[Perfekt ledande plan]
\epsfig{../lect-03/figs/mirrorplane-gedanken-general.1}\noindent
V{\aa}rt antagande att vi kan {\it konstruera} en l{\"o}sning genom att
l{\"a}gga en {\it spegelladdning} p{\aa} motsatt sida om den perfekta
ledaren\numberedfootnote{Rekapitulera att en spegel faktiskt {\"a}r just
  ett mer eller mindre perfekt ledande plan!}
har d{\"a}rmed markant f{\"o}rst{\"a}rkts, och vi kan redan nu gissa att
l{\"o}sningen skall formuleras som den potential i rummet som motsvaras av
de tv{\aa} laddningarna, som
$$
  \phi({\bf x})={{1}\over{4\pi\varepsilon_0}}\bigg(
  {{q}\over{\sqrt{x^2+y^2+(z-h)^2}}}+{{(-q)}\over{\sqrt{x^2+y^2+(z+h)^2}}}
  \bigg),\qquad z\ge0.
$$
Med denna konstruktion {\it uppfylls {\"a}ven randvillkoret d{\aa}
${\bf x}\to\infty$ trivialt}, s{\aa} slutsatsen blir att potentialen
enligt ovan faktiskt {\"a}r den korrekta l{\"o}sningen.

\subsection{Sammanfattning av virtuell spegelladdning i perfekt ledande plan}
\sidx{Spegelladdning}[Perfekt ledande plan]
L{\aa}t oss sammanfatta denna f{\"o}rsta {\"o}vning i spegelladdningar med att:
\quote{{\it L{\"o}sningen $\phi({\bf x})$ till Poissons ekvation f{\"o}r en
  punktladdning $q$ placerad ett avst{\aa}nd $z=h$ ovanf{\"o}r ett perfekt
  ledande plan $z=0$ ges som frirymdl{\"o}sningen med en virtuell
  spegelladdning $-q$ placerad p{\aa} samma avst{\aa}nd bakom planet,
  vid $z=-h$.}}
\noindent
Det {\"a}r v{\"a}rt att notera hur fundamentalt entydighetsteoremet (f{\"o}r
Laplaces ekvation med $\rho=0$) och dess f{\"o}ljdsats (f{\"o}r Poissons
ekvation med $\rho\ne0$) {\"a}r f{\"o}r konstruktionen med spegelladdningar.

\subsection{Det resulterande elektriska f{\"a}ltet med den speglade laddningen}
\sidx{Spegelladdning}[Perfekt ledande plan]
\sidx{Ekvipotential}\sidx{Dipol}[Elektrisk]
Vi ser direkt att f{\"o}r det ekvivalenta problemet med en spegelladdning $-q$
placerad i $-h{\bf e}_z$ s{\aa} {\"a}r den resulterande potentialen $\phi$ och
elektriska f{\"a}ltet ${\bf E}$ givet som det fr{\aa}n en {\it elektrisk dipol}
med dipolmoment
$$
  {\bf p}=2qh{\bf e}_z
$$
placerad i origo. Det elektriska f{\"a}ltet f{\aa}s direkt fr{\aa}n den
skal{\"a}ra potentialen som
$$
  \eqalign{
    {\bf E}({\bf x})
      &=-\nabla\phi({\bf x})\cr
      &=-{{1}\over{4\pi\varepsilon_0}}
        \nabla\bigg(
          {{q}\over{|{\bf x}-h{\bf e}_z|}}
          +{{(-q)}\over{|{\bf x}-(-h{\bf e}_z)|}}
        \bigg)=\ldots\cr
      &={{q}\over{4\pi\varepsilon_0}}
        \bigg(
          {{{\bf x}-h{\bf e}_z}\over{|{\bf x}-h{\bf e}_z|^3}}
          -{{{\bf x}+h{\bf e}_z}\over{|{\bf x}+h{\bf e}_z|^3}}
        \bigg),\cr
  }
$$
vilket vi fr{\aa}n F{\"o}rel{\"a}sning~1 kan erinra oss beskriver det elektriska
f{\"a}ltet i fri rymd fr{\aa}n de tv{\aa} laddningarna. Som av en h{\"a}ndelse
har vi d{\"a}rmed {\"a}ven tagit fram den vektoriella beskrivningen av det
elektriska f{\"a}ltet fr{\aa}n en elektrisk dipol ${\bf p}$, n{\aa}got som vi
senare i denna kurs kommer att {\aa}terv{\"a}nda till och vidareutveckla i
F{\"o}rel{\"a}sning~8.
\par\centerline{\epsfxsize=144mm\epsfbox{../lect-03/potentials/planarmirror-efield.eps}}
\noindent
{\captionwide Skal{\"a}r potential $\phi(x,y)$ f{\"o}r en punktladdning
  $+q$ placerad i $(x,y)=(0,h)$ och en virtuell, speglad punktladdning
  $-q$ placerad i $(x,y)=(0,-h)$, med resulterande elektriska f{\"a}ltlinjer
  erh{\aa}llna fr{\aa}n potentialen som ${\bf E}(x,y)=-\nabla\phi(x,y)$.
  Det perfekt ledande, speglande planet vid $z=0$ {\"a}r i f{\"a}ltet
  inritad som en streckad r{\"o}d linje. Samtliga f{\"a}ltlinjer sk{\"a}r
  det perfekt ledande planet ortogonalt mot denna.}
\vfill\eject

\subsection{Laddningsf{\"o}rdelningen i det perfekt ledande planet $z=0$}
\sidx{Laddningst{\"a}thet}[Ytladdning i perfekt ledande plan]
Vi kommer nu att visa p{\aa} hur vi utifr{\aa}n den skal{\"a}ra elektrostatiska
potentialen kan ber{\"a}kna laddningst{\"a}theten $\sigma$ p{\aa} ytan.
\epsfig{../lect-03/figs/condsurf.1}\noindent
Potentialen $\phi$ {\"a}r kontinuerlig {\"o}ver en godtycklig gr{\"a}nsyta,
specifikt har vi alltid att potentialskillnaden mellan tv{\aa} punkter
${\bf x}_a$ och ${\bf x}_b$ kan tas fram genom linjeintegralen
$$
  \phi({\bf x}_b)-\phi({\bf x}_a)=-\int^{{\bf x}_b}_{{\bf x}_a}{\bf E}({\bf x})\,dl,
$$
och om vi l{\aa}ter punkterna g{\aa} mot varandra fr{\aa}n motsatta sidor av
gr{\"a}nsytan $z=0$ s{\aa} f{\aa}r vi att
$$
  \phi(z=0^+)=\phi(z=0^-).
$$
Dock {\"a}r {\it gradienten} av potentialen diskontinuerlig, vilket vi intuitivt
kan se direkt fr{\aa}n att f{\"a}lt\-linjerna s{\aa} att s{\"a}ga ``str{\aa}lar
ut'' {\aa}t motsatta h{\aa}ll fr{\aa}n en yta som uppb{\"a}r laddningar, helt i
analogi med Gauss lag f{\"o}r punktladdningar.
Eftersom ${\bf E}=-\nabla\phi$, s{\aa} har vi fr{\aa}n F{\"o}rel{\"a}sning~2
att\numberedfootnote{Recap: Det elektriska f{\"a}ltet fr{\aa}n en plan och
  perfekt ledande o{\"a}ndlig yta, uppb{\"a}rande laddningen $\sigma$
  (${\rm C}/{\rm m}^2$), ges som
  $$
    E_z(z)=-[\nabla\phi]_z
      =-{{\partial\phi}\over{\partial z}}
      ={{\sigma}\over{2\varepsilon_0}}\sgn(z).
  $$}
$$
  {\bf E}({\bf x})={{\sigma({\bf x})}\over{\varepsilon_0}}{\bf e}_z,
$$
d{\"a}r ${\bf x}$ {\"a}r i en omgivning n{\"a}ra ytan $z=0$. Att det elektriska
f{\"a}ltet {\"a}r ortogonalt mot ytan f{\"o}ljer av att ytan antas vara perfekt
ledande. Vi kan v{\"a}nda p{\aa} detta resonemang och ist{\"a}llet betrakta
detta som ett samband som ger laddningst{\"a}theten i planet som funktion av
det elektriska f{\"a}ltet, som
$$
  \sigma({\bf x})
    =\varepsilon_0 E_z({\bf x})
    =-\varepsilon_0 [\nabla\phi]_z
    =-\varepsilon_0 {{\partial\phi({\bf x})}\over{\partial z}}.
$$
Vi har dock redan r{\"a}knat fram det elektriska f{\"a}ltet ovan, och vi kan
sammanfatta detta med att ytladdnings\-t{\"a}t\-heten $\sigma({\bf x})$ i
planet $z=0$ erh{\aa}lls som
$$
  \eqalign{
    \sigma(z=0)
      &=\varepsilon_0 E_z(x,y,z=0^+)\cr
      &=\varepsilon_0 {{q}\over{4\pi\varepsilon_0}}
        {\bf e}_z\cdot
        \bigg(
          {{{\bf x}-h{\bf e}_z}\over{|{\bf x}-h{\bf e}_z|^3}}
          -{{{\bf x}+h{\bf e}_z}\over{|{\bf x}+h{\bf e}_z|^3}}
        \bigg)\bigg|_{z=0^+}\cr
      &={{q}\over{4\pi}}
        \Bigg(
          {{(-h)}\over{\big(x^2+y^2+(-h)^2\big)^{3/2}}}
          -{{(+h)}\over{\big(x^2+y^2+(+h)^2\big)^{3/2}}}
        \Bigg)\cr
      &=-{{qh}\over{2\pi(x^2+y^2+h^2)^{3/2}}}.\cr
  }
$$
Vi kan kontrollera att den fysikaliska dimensionen p{\aa} detta uttryck som
f{\"o}rv{\"a}ntat {\"a}r ${\rm C}/{\rm m}^2$, samt att $\sigma\to0$ d{\aa}
$|(x,y)|\to\infty$. Vi kan {\"a}ven notera att laddningen hos den totala
ytladdningst{\"a}theten $\sigma(x,y)$ i ytan $z=0$ ges som
$$
  \eqalign{
    Q&=\iint_S\sigma(x,y)\,dS\cr
     &=-{{qh}\over{2\pi}}\int^{\infty}_{-\infty}\int^{\infty}_{-\infty}
          {{1}\over{(x^2+y^2+h^2)^{3/2}}}\,dx\,dy\cr
     &=\big\{ \ldots \big\}\cr
     &=-q,\cr
  }
$$
ett v{\"a}rde som vi nog egentligen faktiskt kunde ha gissat oss till p{\aa}
grund av symmetrin i konstruktionen av spegelladdningen f{\"o}r att uppfylla
potentialen $\phi=0$ p{\aa} ytan $z=0$.

\section{Spegelladdningar i plana gr{\"a}nsytor mellan dielektrika}
\sidx{Spegelladdning}[Gr{\"a}nsytor mellan dielektrika]
Spegling av laddning i gr{\"a}nsytor mellan tv{\aa}
dielektrika\numberedfootnote{Vi g{\aa}r h{\"a}r h{\"a}ndelserna i
  f{\"o}rv{\"a}g en aning; dielektrika behandlas egentligen f{\"o}rst
  l{\"a}ngre fram i kursen i F{\"o}rel{\"a}sning~6.}
f{\"o}ljer p{\aa} liknande s{\"a}tt som f{\"o}r spegling i perfekt ledande plan.
\epsfig{../lect-03/figs/mirrorplane-dielectrica.1}\noindent
Om laddningen $q$ placeras i ett dielektrikum med den relativa elektriska
permittiviteten $\varepsilon_{\rm r}$ och med ett avst{\aa}nd $z=h$ fr{\aa}n
ytan $z=0$ som avgr{\"a}nsar fr{\aa}n den relativa permittiviteten
$\varepsilon'_{\rm r}$, s{\aa} kommer {\it bundna} laddningar (till skillnad
fr{\aa}n de fria laddningarna i fallet med ett perfekt ledande plan) att
fortfarande motsvara en spegelladdning $q'$ symmetriskt placerad vid $z=-h$
men ist{\"a}llet ha v{\"a}rdet
$$
  q'=\bigg({{\varepsilon_{\rm r}-\varepsilon'_{\rm r}}
       \over{\varepsilon_{\rm r}+\varepsilon'_{\rm r}}}\bigg) q.
$$
\vfill\eject

\section{Spegelladdningar i cylindriska perfekt ledande gr{\"a}nsytor}
\sidx{Spegelladdning}[Cylindriska perfekt ledande gr{\"a}nsytor]
L{\aa}t oss nu betrakta fallet\numberedfootnote{Oturligt nog tar inte Griffiths
  upp fallet med just spegling av en linjeladdning i en ledande cylinder,
  vilket {\"a}r lite synd d{\aa} detta tv{\aa}dimensionella problem illustrerar
  ett antal intressanta aspekter av arbetsg{\aa}ngen med att konstruera
  l{\"o}sningar till Laplaces ekvation f{\"o}r potentialen $\phi$.
  M{\aa}h{\"a}nda {\"a}r detta exempel exkluderat d{\aa} en del algebra
  onekligen m{\aa}ste g{\aa}s igenom i och med argumenterandet f{\"o}r
  hur vi skapar en l{\"o}sning som uppfyller randvillkoret runt om p{\aa}
  cylinderns mantelyta. Det b{\"o}r noteras att {\"a}ven J~D.~Jacksons
  standardverk {\it Classical Electrodynamics}\sidx{Jackson, John David
  (1925--2016)}[{{\it Classical Electrodynamics}}] saknar denna spegling av
  linjeladdningar i cylinderytor.}
med en rak och o{\"a}ndlig linjeladdning $\lambda$ (${\rm C}/{\rm m}$) som
l{\"o}per utanf{\"o}r en likaledes o{\"a}ndlig och perfekt ledande cylinder
med radien $R$ och med utstr{\"a}ckning l{\"a}ngs $z$-axeln samt
jordad\sidx{Jord} till potentialen $\phi_0=0$. Linjeladdningen {\"a}r placerad
ett avst{\aa}nd $h$ utanf{\"o}r cylinderytan, vid en position $(R+h){\bf e}_x$
l{\"a}ngs $x$-axeln enligt figur.
\epsfig{../lect-03/figs/mirrorcharge-cylinder.1}\noindent
Vi s{\"o}ker {\"a}ven i detta problem potentialen $\phi({\bf x})$ i rummet
utanf{\"o}r ledaren, och utifr{\aa}n det tidigare resonemanget kring virtuell
spegelladdning mot ett o{\"a}ndligt och perfekt ledande plan kan vi redan nu
gissa oss till att uppgiften kommer att g{\aa} ut p{\aa} att formulera var en
{virtuell speglad linjeladdning} b{\"o}r placeras s{\aa} att vi i detta problem
uppfyller randvillkoret $\phi=\phi_0=0$ p{\aa} cylinderytan vid radien $r=R$.

Vi kan f{\"o}r {\"o}vrigt konstatera att vi effektivt har att g{\"o}ra med ett
tv{\aa}dimensionellt problem, s{\aa} utan att tulla p{\aa} generalitet kan vi
s{\"a}tta att ${\bf x}=x{\bf e}_x+y{\bf e}_y$, med cylinderytan definierad av
$|{\bf x}|=R$ f{\"o}r alla v{\"a}rden p{\aa} den longitudinella koordinaten $z$.

\subsection{Potential fr{\aa}n en linjeladdning}
\sidx{Spegelladdning}[Cylindriska perfekt ledande gr{\"a}nsytor]
\sidx{Laddningst{\"a}thet}[Linjeladdning]
I detta fall {\"a}r uppgiften att best{\"a}mma den skal{\"a}ra potentialen
$\phi({\bf x})$ i rummet givet en linjeladdning $\lambda$ och ({\aa}tminstone
gissar vi s{\aa}) en virtuell linjeladdning $\lambda'$. D{\"a}rmed infinner sig
fr{\aa}gan, hur f{\aa}r vi fram denna potential? Vi har ju s{\aa} l{\aa}ngt
trots allt bara diskuterat potentialen fr{\aa}n system rent generellt.

Som tur {\"a}r gick vi i F{\"o}rel{\"a}sning~2 igenom hur vi i fall med
symmetrier kan anv{\"a}nda ``Gaussiska ytor'' omslutande laddning, och att vi
p{\aa} detta s{\"a}tt fick ut det elektriska f{\"a}ltet. Specifikt fick vi fram
att det radiella elektriska f{\"a}ltet fr{\aa}n en linjeladdning $\lambda$,
utan att ens beh{\"o}va l{\"o}sa n{\aa}gon integral, var
$$
  E_r(r) = {{\lambda}\over{2\pi \varepsilon_0 r}}.
$$
Detta f{\"a}lt kommer ju i sin tur fr{\aa}n en skal{\"a}r potential genom
${\bf E}=-\nabla\phi$, eller p{\aa} skal{\"a}r och radiell form
$$
  E_r(r)=-{{\partial\phi(r)}\over{\partial r}}
    \quad\Leftrightarrow\quad
    \phi(r)=-\int^r_{r_0}E_r(r)\,dr
    =-{{\lambda}\over{2\pi \varepsilon_0}}\int^r_{r_0}{{1}\over{r}}\,dr
    ={{\lambda}\over{2\pi \varepsilon_0}}\ln\Big({{r_0}\over{r}}\Big)
$$
f{\"o}r n{\aa}gon l{\"a}mpligt vald referens $r_0$ som vi utg{\aa}r ifr{\aa}n
n{\"a}r vi integrerar upp potentialen till godtycklig radie $r$ ortogonalt ut
fr{\aa}n linjeladdningen $\lambda$. Notera att denna referensradie l{\"a}mpligen
ej kan v{\"a}ljas som varken $r_0=0$ eller $r_0=\infty$, eftersom vi d{\aa}
l{\"a}mnar potentialen vid avst{\aa}ndet $r$ fr{\aa}n linjeladdningen
odefinierad. Vi kan bara konstatera att referensradien $r_0$ {\"a}r
{\it n{\aa}got} finit avst{\aa}nd, som i det slutliga resultatet inte
b{\"o}r ha n{\aa}gon inverkan p{\aa} det slutliga svaret.

\subsection{Att l{\"a}gga till virtuell laddning s{\aa} att vi n{\aa}r
  randvillkoret $\phi=0$ p{\aa} cylinderytan}
\sidx{Spegelladdning}[Cylindriska perfekt ledande gr{\"a}nsytor]
Helt i analogi med vad vi gjorde f{\"o}r speglingen i ett perfekt ledande plan
med en punktladdning, kommer vi nu att s{\"o}ka belopp och position f{\"o}r en
{\it virtuell} linjeladdning lagd n{\aa}gonstans i $(x,y)$-planet s{\aa} att vi
(om m{\"o}jligt) n{\aa}r randvillkoret att $\phi=0$ p{\aa} cylinderns yta
$|{\bf x}|=R$.

Det f{\"o}rsta vi kan notera {\"a}r att vi har en symmetri speglad i $x$-axeln,
eftersom vi lagt denna s{\aa} att linjeladdningen $\lambda$ ligger l{\"a}ngs
denna.\numberedfootnote{Notera hur vi genom att l{\"a}gga ut ett l{\"a}mpligt
  valt koordinatsystem markant kan underl{\"a}tta f{\"o}r oss sj{\"a}lva
  redan innan vi ens b{\"o}rjar skissa p{\aa} algebran i problemet!}
Linjeladdningen m{\aa}ste av symmetrisk{\"a}l trivialt sj{\"a}lvfallet ocks{\aa}
f{\"o}lja cylinderns ytstr{\"a}ckning, och d{\"a}rmed {\"a}ven den ligga
l{\"a}ngs $z$-axeln, liksom den ``reella'' linjeladdningen $\lambda$.
Vi antar att vi har en virtuell linjeladdning om $\lambda'$ (${\rm C}/{\rm m}$,
och vi antar att denna {\"a}r lokaliserad vid n{\aa}gon position $d{\bf e}_x$
l{\"a}ngs $x$-axeln.

Givet denna initiala {\it gissning}, som vi erinrar oss alltid {\"a}r
till{\aa}ten i och med att entydikhetsteoremet ger oss en ``licens'' att jaga
efter l{\"o}sningar till Poissons och Laplaces ekvation (av vilka vi f{\"o}r
tillf{\"a}llet analyserar en dom{\"a}n f{\"o}r den senare) utan att beh{\"o}va
bekymra oss om exakt {\it hur} vi kommer fram till en l{\"o}sning, s{\aa}
l{\"a}nge som den uppfyller randvillkoren, s{\aa} {\"a}r vi redo att s{\"a}tta
ig{\aa}ng med att s{\"a}tta samman ett l{\"o}sningsf{\"o}rslag med
spegelladdningar. I en ekvivalent frirymdsmodell av potentialen, d{\"a}r vi
givetvis beh{\aa}ller den reella linjeladdningen $\lambda$ vid positionen
$(R+h){\bf e}_x$ man nu tar bort den ledande cylindern och ers{\"a}tter den med
en {\it virtuell} linjeladdning $\lambda'$ vid $d{\bf e}_x$, s{\aa} f{\aa}r vi
i rummet den sammanlagda potentialen\numberedfootnote{Notera {\aa}terigen hur
kraftfullt superpositionsprincipen kommer till v{\aa}r assistans!}
$$
  \phi({\bf x})
    =\underbrace{
        {{\lambda}\over{2\pi\varepsilon_0}}
          \ln\bigg({{r_0}\over{|{\bf x}-(R+h){\bf e}_x|}}\bigg)
     }_{\hbox{potential fr{\aa}n $\lambda$}}
    +\underbrace{
        {{\lambda'}\over{2\pi\varepsilon_0}}
          \ln\bigg({{r_0}\over{|{\bf x}-d{\bf e}_x|}}\bigg)
     }_{\hbox{potential fr{\aa}n $\lambda'$}}
$$
d{\"a}r vi {\aa}terigen erinrar oss att $r_0$ {\"a}r en godtyckligt vald
referensradie som i det slutliga svaret inte f{\aa}r p{\aa}verka
l{\"o}sningen.\numberedfootnote{Den observante ser h{\"a}r att vi egentligen
  har full frihet att v{\"a}lja $r_0$ {\it individuellt} f{\"o}r
  linjeladdningarna $\lambda$ och $\lambda'$. Naturligtvis kan vi g{\"o}ra
  detta, men i den f{\"o}ljande elimineringen har detta ingen annan betydelse
  {\"a}n att st{\"o}ka till det f{\"o}r oss rent algebraiskt. Vi v{\"a}ljer
  h{\"a}r samma referens-radie $r_0$ f{\"o}r b{\aa}de den reella och virtuella
  linjeladdningen.}
Genom att anv{\"a}nda logaritmlagen $\ln(a/b)=\ln(a)-\ln(b)$ kan vi
vidareutveckla potentialen till en gemensam term involverande referens-radien
$r_0$ som
$$
  \phi({\bf x})
    ={{(\lambda+\lambda')}\over{2\pi\varepsilon_0}}\ln(r_0)
    -{{\lambda}\over{2\pi\varepsilon_0}}
          \ln\big(|{\bf x}-(R+h){\bf e}_x|\big)
    -{{\lambda'}\over{2\pi\varepsilon_0}}
          \ln\big(|{\bf x}-d{\bf e}_x|\big).
$$
\sidx{Spegelladdning}[Cylindriska perfekt ledande gr{\"a}nsytor]
F{\"o}r att uppfylla kravet om att $r_0$ inte skall p{\aa}verka slutresultatet,
s{\aa} ser vi direkt att en m{\"o}jlighet {\"a}r att den speglade, virtuella
linjeladdningen, helt i analogi med fallet med punktladdning $q$, skall
l{\aa}sas till den reella linjeladdningen med omv{\"a}nt tecken som
$$
  \lambda'=-\lambda.
$$
Med detta konstaterande reduceras d{\"a}rmed potentialen $\phi$ i cylindriska
koordinater ${\bf x}=(x,y,z)=(r\cos\varphi,r\sin\varphi,z)$ till att p{\aa}
cylinderytan\numberedfootnote{Vi s{\"o}ker ju konstruera en potential
$\phi({\bf x})$ s{\aa} att $\phi=0$ p{\aa} denna yta!} $r=R$ lyda
$$
  \eqalign{
    \phi({\bf x})\Big|_{|{\bf x}|=R}
      &=-{{\lambda}\over{2\pi\varepsilon_0}}
            \ln\big(|{\bf x}-(R+h){\bf e}_x|\big)
        -{{(-\lambda)}\over{2\pi\varepsilon_0}}
            \ln\big(|{\bf x}-d{\bf e}_x|\big)\cr
      &={{\lambda}\over{2\pi\varepsilon_0}}
           \ln\bigg({{|R\cos\varphi{\bf e}_x+R\sin\varphi{\bf e}_y-d{\bf e}_x|}
            \over{|R\cos\varphi{\bf e}_x+R\sin\varphi{\bf e}_y-(R+h){\bf e}_x|}}
            \bigg)\cr
      &={{\lambda}\over{2\pi\varepsilon_0}}
            \ln\bigg({{(R\cos\varphi-d)^2+R^2\sin^2\varphi}
             \over{{(R\cos\varphi-(R+h))^2+R^2\sin^2\varphi}}}
            \bigg)\cr
      &=\big\{\hbox{ Utveckla }\big\}\cr
      &={{\lambda}\over{2\pi\varepsilon_0}}
           \ln\bigg({{R^2\cos^2\varphi-2Rd\cos\varphi+d^2+R^2\sin^2\varphi}
           \over{{R^2\cos^2\varphi-2R(R+h)\cos\varphi+(R+h)^2+R^2\sin^2\varphi}}}
            \bigg)\cr
      &=\big\{\hbox{ Anv{\"a}nd $\cos^2\varphi+\sin^2\varphi=1$ }\big\}\cr
      &={{\lambda}\over{2\pi\varepsilon_0}}
            \ln\bigg({{R^2-2Rd\cos\varphi+d^2}
            \over{{R^2-2R(R+h)\cos\varphi+(R+h)^2}}}
            \bigg).\cr
  }
$$
Det {\"a}r h{\"a}r mycket frestande att direkt konstatera att f{\"o}r att detta
uttryck skall vara noll p{\aa} ytan $|(x,y)|=R$ s{\aa} kr{\"a}ver vi att
argumentet till logaritmen m{\aa}ste vara ett, eftersom $\ln(1)=0$.
Vi skall dock v{\"a}nta lite med detta och f{\"o}rst konstatera att om detta
uttryck skall vara {\it oberoende av vinkeln $\varphi$}, s{\aa} kr{\"a}ver vi
{\aa}tminstone att argumentet {\"a}r konstant, s{\"a}g
$$
  {{R^2-2Rd\cos\varphi+d^2}\over{{R^2-2R(R+h)\cos\varphi+(R+h)^2}}}
    =C=\hbox{konst.}
$$
I och med detta f{\"o}r vi in ytterligare en frihetsgrad i v{\aa}rt problem, i
och med att $C$ d{\"a}rmed f{\"o}rr eller senare kommer att beh{\"o}va
best{\"a}mmas. Samtidigt har vi en viss paradox n{\"a}rvarande, i och med att
vi har att g{\"o}ra med ett {\it underbest{\"a}mt} ``system'' med en ekvation
och tv{\aa} obest{\"a}mda parametrar $d$ och nu {\"a}ven $C$. Minns dock att
vi {\"a}ven har kravet $\phi=0$, vilket tillf{\"o}r den andra ekvation som
beh{\"o}vs f{\"o}r att ha fullst{\"a}ndig information f{\"o}r l{\"o}sande av
uppgiften.

S{\aa} vad kan vi d{\aa} kr{\"a}va av $d$ och $C$ f{\"o}r att g{\"o}ra uttrycket
ovan konstant och oberoende av vinkeln $\varphi$? Genom att samla ihop termerna
f{\"o}r $\cos\varphi$ i uttrycket ovan, s{\aa} erh{\aa}ller vi att
$$
  2R(C(R+h)-d)\cos\varphi=C(R^2+(R+h)^2)-(R^2+d^2),
$$
vilket vi direkt kan konstatera kan g{\"o}ras oberoende av vinkeln $\varphi$
genom att i v{\"a}nsterledet se till att v{\"a}lja\numberedfootnote{Ja,
  {\it v{\"a}lja}! Kom ih{\aa}g att entydighetsteoremet ger oss ``licens
  att skjuta fr{\aa}n h{\"o}ften'', s{\aa} l{\"a}nge som vi i slutet har
  en l{\"o}sning som uppfyller Laplaces ekvation samt randvillkoret
  $\phi({\bf x})=0$ p{\aa} ytan $|(x,y)|=R$.}
$C$ s{\aa} att koefficienten f{\"o}r $\cos\varphi$ blir noll, som
$$
  2R(C(R+h)-d)=0 \qquad\Leftrightarrow\qquad C={{d}\over{R+h}}.
$$
\sidx{Spegelladdning}[Cylindriska perfekt ledande gr{\"a}nsytor]
L{\aa}t oss se vad detta kan ge f{\"o}r l{\"o}sning f{\"o}r placeringen $d$
f{\"o}r den speglade virtuella linjeladdningen! Med $C$ enligt detta
konstaterande har vi enligt ovan att v{\"a}nsterledet blir noll (dessutom
oberoende av $\cos\varphi$, som f{\"o}rv{\"a}ntat), och att h{\"o}gerledet
samtidigt tar formen
$$
  0={{d}\over{(R+h)}}(R^2+(R+h)^2)-(R^2+d^2)
  \qquad\Leftrightarrow\qquad
  (R^2+d^2)(R+h)=d(R^2+(R+h)^2).
$$
Detta definierar en andragradsekvation f{\"o}r den s{\"o}kta variabeln $d$,
vilken enkelt kan l{\"o}sas till att ge tv{\aa} separata l{\"o}sningar
$$
  d=\cases{
    R^2/(R+h) & (L{\"o}sning 1) \cr
    R+h       & (L{\"o}sning 2) \cr
  }
$$
{\it Stopp och bel{\"a}gg nu!} Betyder detta inte att vi just visat p{\aa}
{\it tv{\aa} olika l{\"o}sningar} f{\"o}r potentialen $\phi({\bf x})$ som
satisfierar exakt samma problem?
Vi har ju just visat att det finns {\it tv{\aa}} olika positioner f{\"o}r
spegelladdningen som ju b{\aa}da ger en l{\"o}sning till problemet.
Sade inte entydighetsteoremet oss alldeles nyss att {\it endast en unik
l{\"o}sning existerar till Laplaces ekvation?}
\sidx{Spegelladdning}[Cylindriska perfekt ledande gr{\"a}nsytor]

Denna observation, att vi har tv{\aa} l{\"o}sningar f{\"o}r positionen $d$
som ploppar fram ur andragradsekvationen f{\"o}r denna, {\"a}r helt korrekt;
l{\aa}t oss d{\"a}rf{\"o}r studera dessa tv{\aa} l{\"o}sningar i detalj.

Vi b{\"o}rjar med L{\"o}sning~2, som ju talar om f{\"o}r oss att en l{\"o}sning
till problemet med att uppn{\aa} en fix potential ($\phi=0$) som randvillkor
p{\aa} cylinderytan {\"a}r att helt enkelt l{\"a}gga den virtuella, speglade
linjeladdningen $\lambda'=-\lambda$ direkt ovanp{\aa} den reella linjeladdningen
$\lambda$, vid punkten $(R+h){\bf e}_x$.
Detta ger direkt att den resulterande linjeladdningen ({\aa}terigen, via
superpositionsprincipen!) {\"a}r noll, och all potential runt om i problemet
blir trivialt noll. Specifikt s{\aa} uppfyller denna triviala noll-l{\"o}sning
sj{\"a}lvfallet randvillkoret $\phi=0$ p{\aa} cylinderytan, men den f{\"o}r oss
inte fram{\aa}t i problemets l{\"o}sning och {\"a}r uppenbart inte den
l{\"o}sning vi s{\"o}ker. Med andra ord f{\"o}rkastar vi p{\aa} st{\aa}ende
for L{\"o}sning~2.

Om vi ist{\"a}llet betraktar L{\"o}sning~1, s{\aa} ger den {\aa} andra sidan vid
hand att den virtuella, speglade linjeladdningen skall placeras vid en punkt
$d=R^2/(R+h)$ som ligger n{\aa}gonstans mellan $x=0$ och $x=R/2$, vilket
l{\aa}ter rimligt.\numberedfootnote{Att den virtuella, speglade
  linjeladdningens position $d$ ligger n{\aa}gonstans mellan $x=0$
  och $x=R/2$ f{\"o}r L{\"o}sning~1 f{\"o}ljer av att $h>R$, vilket
  i sin tur betyder att
  $$
    d={{R^2}\over{R+h}}
     =\underbrace{\bigg({{R}\over{R+h}}\bigg)}_{\in[0,1/2[}R
     \in[0,R/2[.
  $$}
Trots allt s{\aa} {\"a}r ett implicit villkor i allt vi g{\"o}r i denna metod
med speglade laddningar att ingen virtuell laddning f{\aa}r placeras i
dom{\"a}nen d{\"a}r vi analyserar potentialen, d{\aa} detta vore ett brott mot
sj{\"a}lva grundantagandet om att vi har Laplace ekvation {\"o}verallt med
undantag f{\"o}r sj{\"a}lva k{\"a}lladdningen som vi speglar, oavsett om det
{\"a}r en punktladdning eller linjeladdning eller n{\aa}gon annan konfiguration
som {\"a}r sj{\"a}lva k{\"a}llan i problemet.
\vfill\eject

\subsection{Sammanfattning av virtuell speglad linjeladdning i cylindrisk yta}
\sidx{Spegelladdning}[Cylindriska perfekt ledande gr{\"a}nsytor]
Vi kan sammanfatta denna exercis i konstruktion av en l{\"o}sning till problemet
med spegling av en linjeladdning i en cylindrisk yta till ett perfekt ledande
medium att:
\quote{L{\"o}sningen till Poissons ekvation f{\"o}r en linjeladdning $\lambda$
  placerad p{\aa} ett avst{\aa}nd $R+h$ fr{\aa}n centrum av en perfekt ledande
  cylinder med radie $R$ ges som fri\-rymds\-l{\"o}s\-ningen med en virtuell
  linjeladdning $\lambda'=-\lambda$ placerad p{\aa}
  avst{\aa}ndet\numberedfootnote{Vi kallar emellan{\aa}t punkten f{\"o}r
    den speglade linjeladdningen f{\"o}r {\it inverspunkten till
    k{\"a}llpunkten med avseende till en cirkel med radien $R$.}
    Se {\tt https://sv.wikipedia.org/wiki/Inversion}}
  $$
    d={{R^2}\over{R+h}},
  $$
  fr{\aa}n cylinderns centrum i riktning mot den externa linjeladdningen.}
\noindent
Vi b{\"o}r f{\"o}r sakens skull {\"a}ven passa p{\aa} att sammanfatta detta
som en explicit form involverande den s{\"o}kta potentialen, som med
ovanst{\aa}ende form p{\aa} $d$ lyder
$$
  \phi({\bf x})
    ={{\lambda}\over{2\pi\varepsilon_0}}
       \ln\bigg({{|{\bf x}-d{\bf e}_x|}\over{|{\bf x}-(R+h){\bf e}_x|}}\bigg),
$$
med principellt utseende enligt nedan.
\par\centerline{\epsfxsize=144mm\epsfbox{../lect-03/potentials/cylindricmirror-efield.eps}}
\noindent
{\captionwide Skal{\"a}r potential $\phi(x,y)$ f{\"o}r en linjeladdning
  $\lambda$ placerad vid $x=R+h$ (med $R=0.7$, $h=1.5$) utanf{\"o}r en
  perfekt ledande cylinder med radien $R$, med resulterande elektriska
  f{\"a}ltlinjer fr{\aa}n potentialen som ${\bf E}(x,y)=-\nabla\phi(x,y)$.
  Cylindern {\"a}r i f{\"a}ltet inritad som en streckad r{\"o}d cirkel.
  Samtliga f{\"a}ltlinjer sk{\"a}r den perfekt ledande cylindern ortogonalt
  mot denna.}
\vfill\eject

\section{Spegelladdningar i sf{\"a}riska perfekt ledande gr{\"a}nsytor}
\sidx{Spegelladdning}[Sf{\"a}riska perfekt ledande gr{\"a}nsytor]
F{\"o}r spegling av punktladdningar i sf{\"a}riska ytor f{\"o}ljer resonemanget
likt det f{\"o}r punktladdningar ovan, med observationen att ekvipotentialytorna
f{\"o}r en elektrisk dipol ges av sf{\"a}rer som {\"a}r f{\"o}r\-skjutna mellan
laddningarna.\numberedfootnote{Se till exempel grafen f{\"o}r
  f{\"a}ltf{\"o}rdelningen f{\"o}r den elektriska dipolen som
  diskuterats tidigare i denna f{\"o}rel{\"a}sning.}
Vi sammanfattar h{\"a}r helt enkelt bara l{\"o}sningen\numberedfootnote{Vilken
  f{\"o}r {\"o}vrigt {\"a}r en lysande {\"o}vningsuppgift! Se Griffiths
  sid.~128 f{\"o}r detaljer.} med f{\"o}ljande.
\quote{L{\"o}sningen till Poissons ekvation f{\"o}r en punktladdning $q$
  placerad p{\aa} ett avst{\aa}nd $R+h$ fr{\aa}n centrum av en perfekt ledande
  sf{\"a}r med radie $R$ ges som fri\-rymds\-l{\"o}s\-ningen med en virtuell
  punktladdning
  $$
    q'=-{{R}\over{R+h}}q,
  $$
  placerad p{\aa} avst{\aa}ndet
  $$
    d={{R^2}\over{R+h}},
  $$
  fr{\aa}n sf{\"a}rens centrum i riktning mot den externa punktladdningen.}
\noindent
Notera att placeringen f{\"o}r den virtuella punktladdningen st{\"a}mmer
{\"o}verens till punkt och pricka med avst{\aa}ndet som vi nyss r{\"a}knade
ut f{\"o}r en cylinder av samma radie $R$; f{\"o}r spegling av punktladdningen
$q$ i en sf{\"a}r skalas dock den virtuella, speglade laddningen med faktorn
$R/(R+h)$.

\section{Kan jord ers{\"a}ttas med godtycklig konstant potential?}
\sidx{Jord}
I samtliga uppgifter ovan har vi antagit att de perfekta ledarna kopplats till
en konstant potential, och vi har f{\"o}r enkelhets skull antagit koppling till
jord, vilket vi h{\"a}r kort och gott s{\"a}tter som $\phi_0=0$.
Fr{\aa}gan {\"a}r dock om detta antagande om $\phi_0=0$ verkligen {\"a}r
n{\"o}dv{\"a}ndigt? Vad h{\"a}nder om vi i resonemangen ovan bara antar att
$\phi_0=\hbox{konstant}$?
\vfill\eject

\section{Sammanfattning av F{\"o}rel{\"a}sning~3 -- Spegelladdningar,
  randvillkor och entydighet}
\item{$\bullet$}{L{\"o}sningar till Laplaces ekvation $\nabla^2\phi=0$
  saknar lokala extrempunkter. Extrempunkter till $\phi$ finns {\it endast}
  p{\aa} randen $S$ till den volym $V$ i vilket Laplace ekvation g{\"a}ller.
  [Teorem~I]}
\item{$\bullet$}{Genom att anv{\"a}nda denna information kan vi visa
  att {\it l{\"o}sningen till Laplaces ekvation $\nabla^2\phi=0$ i en
  godtycklig volym $V$ {\"a}r unikt (entydigt) best{\"a}md om potentialen
  $\phi$ {\"a}r specificerad p{\aa} randen $S$ till volymen.}
  [Teorem~II, {\it First uniqueness theorem} enligt Griffiths]}
\item{$\bullet$}{Detta betyder i sin tur att om vi finner en l{\"o}sning
  till Laplace ekvation, {\it s{\aa} {\"a}r detta den enda l{\"o}sningen,
  oavsett hur vi funnit eller konstruerat l{\"o}sningen s{\aa} l{\"a}nge
  som l{\"o}sningen uppfyller de f{\"o}reskrivna randvillkoren}.
  Detta argument {\"a}r sj{\"a}lva k{\"a}rnan i hur vi motiverar
  anv{\"a}ndandet av spegelladdningar i l{\"o}sningen av potentialproblem,
  med n{\"a}rvaro av ledare med fria laddningar.}
\item{$\bullet$}{``Entydighetsteoremet {\"a}r en licens till din fantasi.
  Det spelar ingen roll {\it hur} du finner din l{\"o}sning; om den
  (a) uppfyller Laplaces (eller Poissons!) ekvation och (b) har korrekt
  v{\"a}rde p{\aa} randen, s{\aa} {\"a}r den {\it korrekt}.''}
\item{$\bullet$}{[Spegling av punktladdning i perfekt ledande plan]
  L{\"o}sningen $\phi({\bf x})$ till Poissons ekvation f{\"o}r en punktladdning
  $q$ placerad ett avst{\aa}nd $z=h$ ovanf{\"o}r ett perfekt ledande plan $z=0$
  ges som frirymdl{\"o}sningen med en virtuell spegelladdning $-q$ placerad
  p{\aa} samma avst{\aa}nd bakom planet, vid $z=-h$.}
\item{$\bullet$}{[Spegling av punktladdning i plan mellan dielektrika]
  Spegelladdningen placeras p{\aa} samma s{\"a}tt som i fallet med perfekt
  ledande plan, men ist{\"a}llet med v{\"a}rdet
  $$
    q'={{\varepsilon_{\rm r}-\varepsilon'_{\rm r}}
         \over{\varepsilon_{\rm r}+\varepsilon'_{\rm r}}}q,
  $$
  d{\"a}r $\varepsilon_{\rm r}$ och $\varepsilon'_{\rm r}$ {\"a}r de relativa
  elektriska permittiviteterna f{\"o}r respektive $z>0$ och $z<0$.}
\item{$\bullet$}{[Spegling av linjeladdning i perfekt ledande cylinder]
  L{\"o}sningen till Poissons ekvation f{\"o}r en linjeladdning $\lambda$
  placerad p{\aa} ett avst{\aa}nd $R+h$ fr{\aa}n centrum av en perfekt ledande
  cylinder med radie $R$ ges som fri\-rymds\-l{\"o}s\-ningen med en virtuell
  linjeladdning $\lambda'=-\lambda$ placerad p{\aa} avst{\aa}ndet
  $$
    d={{R^2}\over{R+h}},
  $$
  fr{\aa}n cylinderns centrum i riktning mot den externa linjeladdningen.}
\item{$\bullet$}{[Spegling av punktladdning i perfekt ledande sf{\"a}r]
  L{\"o}sningen till Poissons ekvation f{\"o}r en punktladdning $q$ placerad
  p{\aa} ett avst{\aa}nd $R+h$ fr{\aa}n centrum av en perfekt ledande sf{\"a}r
  med radie $R$ ges som fri\-rymds\-l{\"o}s\-ningen med en virtuell
  punktladdning
  $$
    q'=-{{R}\over{R+h}}q,
  $$
  placerad p{\aa} avst{\aa}ndet
  $$
    d={{R^2}\over{R+h}},
  $$
  fr{\aa}n sf{\"a}rens centrum i riktning mot den externa punktladdningen.}

\cleardoublepage
%%% End of auto-extracted text from ../lect-03/lecture-03.tex %%%
%%% Begin of auto-extracted text from ../lect-04/lecture-04.tex %%%
%
% File: teach/elmagii/lect-04/lecture-04.tex [plain TeX code]
% Github: https://github.com/elmagii/lect-04/
% Last change: November 10, 2025
%
% Lecture No 4 in the course ``Elektromagnetism II, 1TE626 (2025)'',
% held November 11, 2025, at Uppsala University, Sweden.
%
% Copyright (C) 2022-2025, Fredrik Jonsson, under Gnu General Public
% License (GPL) v3. See the enclosed LICENSE for details.
%
% This program is free software: you can redistribute it and/or modify
% it under the terms of the GNU General Public License as published by
% the Free Software Foundation, either version 3 of the License, or
% (at your option) any later version.
%
% This program is distributed in the hope that it will be useful,
% but WITHOUT ANY WARRANTY; without even the implied warranty of
% MERCHANTABILITY or FITNESS FOR A PARTICULAR PURPOSE.  See the
% GNU General Public License for more details.
%
% You should have received a copy of the GNU General Public License
% along with this program.  If not, see <https://www.gnu.org/licenses/>.
%
\def\coursename{Elektromagnetism II}
\def\coursecode{1TE626}
\def\courseyear{2025}
\def\courserepo{https://github.com/hp35/elmagii/}
\def\lecturenumber{4}
\def\lecturetitle{Magnetostatik}
\def\lecturesubtitle{}
\def\lectureauthor{Fredrik Jonsson}
\def\lectureplace{Uppsala Universitet}
\def\lecturedate{11 november 2025}
%-------------------- BEGIN OF LOCAL MACROS --------------------
\edef\expandedlecturenumber{4}
\def\ifempty#1{\ifx\relax#1\relax}
\advance\chapno by 1
\secno=0
\footnotenumber=0
\message{==================== Lecture 4 ====================}
\writenumberedtocentry{chapter}{F{\"o}rel{\"a}sning 4 -- {Magnetostatik}}{\thechapno}
\hsize=150mm\hoffset=4.6mm\vsize=230mm\voffset=7mm
\topskip=0pt\baselineskip=12pt\parskip=0pt\leftskip=0pt\parindent=15pt
\ifcolors
  \voffset=-10.2mm\topskip=0pt
\fi
\headline={\ifnum\secno>0\ifodd\pageno\rightheadline\else\leftheadline\fi
  \else\hfill\fi}
\def\rightheadline{\tenrm{\it F\"orel\"asning 4}
  \hfil{\it \coursename, \coursecode\ (\courseyear)}}
\def\leftheadline{\tenrm{\it \coursename, \coursecode\ (\courseyear)}
  \hfil{\it F\"orel\"asning 4}}
\noindent~\vskip-60pt\hskip-40pt{\epsfbox{../lect-01/macros/UU_logo_color.eps}}
\vskip-42pt\hfill\vbox{
    \hbox{{\it \coursename, \coursecode\ (\courseyear)}}
    \hbox{{\it Lecture Notes, \lectureauthor}}
    \hbox{{\it Document Revision \today}}
    \hbox{{\it \courserepo}}}\vskip 36pt
\centerline{\twelvesc F\"orel\"asning 4}
\vskip 24pt\noindent
\centerline{\twelvesc{Magnetostatik}}
\expandafter\ifempty\expandafter{\lecturesubtitle}%
  \else\centerline{\twelvesc\lecturesubtitle}\fi
\bigskip
\centerline{\lectureauthor, \lectureplace, \lecturedate}
\vskip24pt
%--------------------- END OF LOCAL MACROS ---------------------



\plan{Med en kort sammanfattning av historiken bakom uppt{\"a}ckandet av
  magnetiska f{\"a}lt g{\aa}r vi in p{\aa} sj{\"a}lva definitionen av ett
  magnetiskt f{\"a}lt som den kraft som  via Lorentz kraftlag ut{\"o}vas
  p{\aa} en laddad partikel i r{\"o}relse. Utifr{\aa}n denna formuleras
  Amp\`eres kraftlag f{\"o}r str{\"o}mslingor, samt att vi kan dra slutsatsen
  att kraften p{\aa} fria laddningar aldrig utf{\"o}r n{\aa}got arbete.

  D{\aa} vi generaliserar str{\"o}mbegreppet till en str{\"o}mt{\"a}thet
  ${\bf J}$ kan vi under anv{\"a}n\-dande av Gauss lag ta fram
  kontinuitets\-ekvationen f{\"o}r laddning, $\nabla\cdot{\bf J}=-d\rho/dt$,
  som l{\"a}nkar ihop divergensen hos str{\"o}m\-t{\"a}theten med
  tidsderivatan av laddnings\-t{\"a}t\-heten. Biot--Savarts lag introduceras
  som ett axiom f{\"o}r det magnetf{\"a}lt som genereras av en str{\"o}m
  traverserande en str{\"o}mslinga, vilket f{\"o}r {\"o}vrigt {\"a}r
  f{\"o}rsta momentet d{\"a}r den magnetiska permea\-bili\-teten $\mu_0$
  introduceras.
  Utifr{\aa}n formen p{\aa} Biot--Savarts lag f{\"o}r generering av
  magnetf{\"a}lt visar vi att $\nabla\cdot{\bf B}=0$ alltid {\"a}r uppfyllt,
  vilket p{\aa}visar att magnetiska monopoler (magnetisk laddning) ej existerar,
  och att magnetism alltid endast yttrar sig i form av magnetiska dipoler eller
  h{\"o}gre ordningar i multipolutvecklingen.
  Vi visar att i magnetostatik kan rotationen av magnetf{\"a}ltet erh{\aa}llas
  som $\nabla\times{\bf B}=\mu_0{\bf J}$, kallad Amp\`eres lag.
  \sidx{Amp\`eres lag[F{\"or}r statiska magnetiska f{\"a}lt]}

  Slutligen visar vi p{\aa} att icke-existensen av magnetiska monopoler
  direkt har som f{\"o}ljd att vi kan tolka det magnetiska f{\"a}ltet
  som h{\"a}rr{\"o}rande fr{\aa}n en vektorpotential ${\bf A}$, i analogi
  med den skal{\"a}ra potentialen $\phi$ inom elektrostatik, som
  ${\bf B}=\nabla\times{\bf A}$.
  Amp\`eres lag kan tolkas i termer av denna vektorpotential som Poissons
  ekvation $\nabla^2{\bf A}=-\mu_0{\bf J}$ med str{\"o}mt{\"a}theten som
  k{\"a}llterm.}

\threepointsummary{%
  Lagen om att elektrisk laddning inte kan f{\"o}rsvinna,
  $$
    \nabla\cdot{\bf J}=-{{d\rho}\over{dt}}.
  $$
}{%
  Ur ``lagen om att inga magnetiska monopoler existerar''
  kan vi direkt formulera vektorpotentialen ${\bf A}$ som
  $\nabla\cdot{\bf B}=0\ \Leftrightarrow\ {\bf B}=\nabla\times{\bf A}$.
}{%
  Amp\`eres lag f{\"o}r den magnetostatiska vektorpotentialen ${\bf A}$ ges
  som Poissons ekvation med den fria str{\"o}mt{\"a}theten ${\bf J}$ som
  k{\"a}llterm,
  $$
    \nabla^2{\bf A}=-\mu_0{\bf J},
  $$
  med den explicita l{\"o}sningen
  $$
    {\bf A}({\bf x})={{\mu_0}\over{4\pi}}\iiint_V
      {{{\bf J}({\bf x}')}\over{|{\bf x}-{\bf x}'|}}\,dV'.
  $$
}
\vfill\eject\copyrights

\section{Introduktion}
\sidx{Elektrostatik}\sidx{Magnetostatik}\sidx{Elektron}\sidx{Str{\"o}m}
Vi kommer i denna f{\"o}rel{\"a}sning att behandla konstanta str{\"o}mmar, i
vilka laddningar (l{\"a}s: elektroner) r{\"o}r sig l{\"a}ngs givna f{\"a}lt
eller trajektorior som i sig {\"a}r konstanta i rum och tid.\numberedfootnote{Vi
  kommer i denna f{\"o}rel{\"a}sning att i huvudsak f{\"o}lja Griffiths
  kapitel~5, sid.~210--.}
Vi kan som en rekapitulation fr{\aa}n F{\"o}rel{\"a}sning~1 sammanfatta
begreppen elektrostatik och magnetostatik enligt f{\"o}ljande:
\medskip
\item{$\bullet$}{Station{\"a}ra laddningar $\Rightarrow$
  Konstanta elektriska f{\"a}lt (elektrostatik)}
\item{$\bullet$}{Konstanta str{\"o}mmar $\Rightarrow$
  Konstanta magnetiska f{\"a}lt (magnetostatik)}
\medskip
\noindent
Magnetostatik {\"a}r studiet av magnetf{\"a}lt i system d{\"a}r n{\"a}rvarande
elektriska str{\"o}mmar {\"a}r konstanta. Detta {\"a}r den magnetiska analogin
med elektrostatik, d{\"a}r ist{\"a}llet de elektriska laddningarna {\"a}r
station{\"a}ra och fixa i tid och rum.
Att vi h{\"a}r har att g{\"o}ra med statiska magnetf{\"a}lt betyder inte att
teorin inte g{\aa}r att applicera p{\aa} tidsberoende problem, bara att det
m{\aa}ste g{\"o}ras under f{\"o}ruts{\"a}ttningen att f{\"o}rloppen {\"a}r
l{\aa}ngsamma nog att f{\"o}r att den elektromagnetiska v{\aa}gl{\"a}ngden
\sidx{Elektromagnetisk v{\aa}gl{\"a}ngd} i problemet rej{\"a}lt {\"o}verstiger
storleken p{\aa} dom{\"a}nen som analyseras.
Till exempel kan vanliga elektriska generatorer och motorer, exempelvis
startmotorn i en bil eller generatorn i ett vindkraftverk, med mycket god
approximation behandlas som just magnetostatiska problem, trots den uppenbara
r{\"o}relsen och tidsberoendet. Nyckeln {\"a}r att det {\"a}r de {\it olika
tidsskalorna} mellan den mekaniska r{\"o}relsen och f{\"o}r utbredningen av
elektromagnetiska v{\aa}gor som avg{\"o}r om vi kan betrakta problemet som
magnetostatiskt eller ej.

\section{Historik}
Magnetostatiken kan s{\"a}gas ha uppt{\"a}ckts 1269 av fransmannen Petrus
Peregrinus de Maricourt,\sidx{de Maricourt, Petrus Peregrinus} som
unders{\"o}kte det magnetiska f{\"a}ltet p{\aa} ytan av en sf{\"a}risk
magnet\sidx{Permanentmagnet} med n{\aa}lar av j{\"a}rn. Han noterade att de
resulterande f{\"a}ltlinjerna som n{\aa}larna beskrev korsades p{\aa} tv{\aa}
p{\aa} sf{\"a}ren motsatta punkter, som han betecknade som ``poler''
\sidx{Magnetiska poler} i analogi
med jordens poler.\numberedfootnote{Intressant nog, som en litet sidosp{\aa}r
  till Maricourts observationer kring detta med poler, s{\aa} formulerade
  grekiska filosofer som \idx{Empedocles (494--434~BC)} och \idx{Anaxagoras
  (500--428~BC)} redan kring 500~BC hypotesen att jorden sannolikt var rund,
  utifr{\aa}n den runda skugga som jorden gav p{\aa} m{\aa}nen vid en
  m{\aa}nf{\"o}rm{\"o}rkelse.
  Kring 350~BC bistod \idx{Aristoteles (384--322~BC)} med observationen att
  d{\aa} skepp som seglar iv{\"a}g f{\"o}rsvinner skrovet f{\"o}rst ur sikte,
  f{\"o}re masten, och att detta pekar p{\aa} att jorden har en rund form.
  F{\"o}rst 1522~AD fick vi dock ``h{\aa}rt bevis'' p{\aa} att jorden
  {\"a}r rund genom Magellan--Elcanos expedition som genomf{\"o}rde den
  f{\"o}rsta v{\"a}rldsomseglingen och d{\"a}rmed en g{\aa}ng f{\"o}r
  alla bevisade att vi kan resa hela v{\"a}gen runt jordklotet.}

Maricourt formulerade ocks{\aa} den synnerligen intressanta observationen att
{\it oavsett hur fint vi skivar en magnet, s{\aa} har den alltid en nord- och
sydpol}. Som vi skall se fram{\"o}ver i denna f{\"o}rel{\"a}sningsserie
h{\"a}nger detta intimt samman med att magnetism alltid manifesterar sig
som dipoler (eller h{\"o}gre ordningars multipoler), och {\it aldrig som
magnetisk laddning (monopoler)},\sidx{Magnetisk monopol} detta som en markant
skillnad gentemot elektrisk laddning.\sidx{Elektrisk monopol}

Den som r{\"a}knas som uppt{\"a}ckaren av att elektriska str{\"o}mmar genererar
magnetiska f{\"a}lt {\"a}r Hans Christian {\OE}rsted,\sidx{{\OE}rsted, Hans
Christian (1777--1851)} som 1820 publicerade sin uppt{\"a}ckt att orienteringen
hos en kompassn{\aa}l p{\aa}verkas av en elektrisk str{\"o}m i n{\"a}rheten av
n{\aa}len.
F{\"o}r sin uppt{\"a}ckt bel{\"o}nades {\OE}rsted av {\it The Royal Society} i
England med Copley-medaljen, samt att den \idx{Franska Akademien} bel{\"o}nade
honom med $3\,000$ franc.
{\OE}rsteds uppt{\"a}ckt kan s{\"a}gas vara startskottet f{\"o}r arbetet med
att formulera den moderna elektromagnetismen, speciellt inspirerade detta den
franske fysikern Andr\'e-Marie Amp\`ere\sidx{Amp\`ere, Andr\'e-Marie
(1775--1836)} till att formulera en matematisk formel f{\"o}r att beskriva den
magnetiska kraften mellan str{\"o}mslingor.
\vfill\eject

\section{Vad {\"a}r ett magnetf{\"a}lt?}
\sidx{Magnetf{\"a}lt}\sidx{Magnetisk fl{\"o}dest{\"a}thet}
\sidx{Elektrisk str{\"o}m}
\quote{{\bf Definition.} Vi definierar ett magnetf{\"a}lt\numberedfootnote{I
    denna kurs betecknar vi magnetf{\"a}ltet ocks{\aa} som $B$-f{\"a}lt,
    men i andra sammanhang betecknas f{\"a}ltet {\"a}ven som $H$-f{\"a}lt,
    beroende vilken ing{\aa}ng och historisk konvention man r{\aa}kar ha.
    Den korrekta svenska termen f{\"o}r $B$-f{\"a}ltet {\"a}r egentligen
    {\it magnetisk fl{\"o}dest{\"a}thet}.}
  som ett {\it fysiskt f{\"a}lt som beskriver magnetisk p{\aa}verkan p{\aa}
  r{\"o}rliga elektriska laddningar, elektriska str{\"o}mmar och magnetiska
  material.}}
\noindent
I n{\aa}gon m{\aa}n kan vi s{\"a}ga att dessa tre m{\"o}jligheter egentligen
kokar ner till en enda sak, n{\"a}mligen p{\aa}verkan av r{\"o}rliga elektriska
laddningar,\sidx{Elektrisk laddning}  eftersom elektrisk str{\"o}m utg{\"o}rs
av r{\"o}rliga laddningar samt att magnetiska material handlar om hur
materialet p{\aa} en mikroskopisk niv{\aa}, eller snarare kvantmekanisk
niv{\aa}, beter sig i linjering av spinn och magnetiska moment som kan ses
som mikroskopiska slutna str{\"o}mslingor.\sidx{Str{\"o}mslinga}[Sluten]
Den kvantmekaniska\sidx{Kvantmekanik} behandlingen av spinn och magnetiska
moment\sidx{Magnetiskt dipolmoment} ligger dock utanf{\"o}r omfattningen av
denna kurs.

\section{Lorentz-kraften - R{\"o}rliga laddningar i statiska elektriska och
  magnetiska f{\"a}lt}
\sidx{Lorentz-kraften}
\epsfig{../lect-04/figs/lorentz.1}\noindent
Problemet med magnetiska f{\"a}lt {\"a}r att det fr{\aa}n ett klassiskt
angreppss{\"a}tt {\"a}r om{\"o}jligt att strikt h{\"a}rleda dem {\it a priori}
fr{\aa}n klassiska elektromagnetiska modeller. Vi kommer h{\"a}r att rent
axiomatiskt konstatera att ett magnetf{\"a}lt ${\bf B}$ {\"a}r det f{\"a}lt
som ger kraften p{\aa} en r{\"o}rlig och elektriskt laddad partikel med
laddningen $q$ och hastigheten ${\bf v}$ som
$$
  {\bf F}=q({\bf v}\times{\bf B}).
$$
Notera h{\"a}r hur vi i likhet med den elektrostatik som vi behandlade
i F{\"o}rel{\"a}sning~1 kan se det magnetiska f{\"a}ltet ${\bf B}$ som
{\it definierat} av den kraft ${\bf F}$ som ut{\"o}vas p{\aa} en laddning,
bara det att denna kraft nu relaterar till laddningens {\it r{\"o}relse}
och inte bara till dess belopp.
Om vi l{\"a}gger till kraften p{\aa} laddningen fr{\aa}n ett statiskt elektriskt
f{\"a}lt, s{\aa} erh{\aa}ller vi {\it Lorentz-kraften}\numberedfootnote{Som,
  liksom Griffiths korrekt p{\aa}pekar, ursprungligen formulerades av
  Oliver Heaviside,\sidx{Heaviside, Oliver (1850--1925)} som senare kom att
  ha en instrumentell del i formulerandet av den moderna formen av
  {\it Maxwell's ekvationer}\sidx{Maxwells ekvationer} s{\aa} som vi idag
  k{\"a}nner dem. F{\"o}r en intressant sammanfattning
  av Heavisides arbete med att sl{\aa} samman de fr{\aa}n b{\"o}rjan tjugo
  ekvationerna beskrivande elektrodynamik till de fyra som utg{\"o}r den
  moderna formen av Maxwells ekvationer, se exempelvis Damian P. Hampshire,
  {\it A derivation of Maxwell's equations using the Heaviside notation},
  Phil. Trans. Royal Society A {\bf 376}, 20170447 (2017).
  {\tt https://royalsocietypublishing.org/doi/10.1098/rsta.2017.0447}}
$$
  {\bf F}=q\big({\bf E}+{\bf v}\times{\bf B}\big).
$$
{\AA}terigen, vi h{\"a}vdar h{\"a}r inte att vi i denna kurs p{\aa} n{\aa}got
vis kommer att h{\"a}rleda denna relation; vi kommer h{\"a}r ist{\"a}llet att
helt luta oss mot att denna form {\"a}r experimentellt verifierad efter alla
konstens regler och d{\"a}rmed n{\"o}ja oss med det.

\subsection{Magnetisk kraft utf{\"o}r inget arbete}
\sidx{Magnetisk kraft utf{\"o}r inget arbete}
Utifr{\aa}n formen p{\aa} Lorentz-kraften, som {\"a}r ortogonal mot
s{\aa}v{\"a}l det magnetiska f{\"a}ltet ${\bf B}$ som hastigheten ${\bf v}$,
kan vi dra en viktig och generell slutsats:
\quote{{\it Magnetiska krafter utf{\"o}r inget arbete.}}
\noindent
Detta kan spontant tyckas vara mots{\"a}gelsefullt; trots allt vet vi ju att
generatorer och elektriska motorer bygger just p{\aa} magnetf{\"a}lt (och som
vi konstaterat s{\aa} kan vi i dessa fall dessutom behandla de magnetiska
f{\"a}ltproblemen som just {\it magnetostatiska}), s{\aa} hur skulle dessa
inte utf{\"o}ra n{\aa}got arbete?

Argumentet h{\"a}r g{\"a}ller dock att {\it magnetiska krafter} faktiskt inte
utf{\"o}r arbete p{\aa} {\it fri} elektrisk laddning, eftersom om laddningen
$q$ f{\"o}rflyttas en str{\"a}cka
$$
  d{\bf l}={\bf v}dt
  \qquad\Rightarrow\qquad\hbox{Utf{\"o}rt arbete:}\quad
  dW={\bf F}\cdot d{\bf l}
    =q\underbrace{({\bf v}\times{\bf B})}_{\perp{\bf v}}\cdot{\bf v}dt
    =0,
$$
det vill s{\"a}ga att hur vi {\"a}n f{\"o}rflyttar laddningen i ett magnetiskt
f{\"a}lt, s{\aa} kommer den resulterande Lorentz-kraften att vara ortogonal mot
f{\"o}rflyttningen och det resulterande {\it magnetiska} arbetet kommer att
vara noll.

S{\aa} hur kan generatorer och elektriska motorer fungera om den magnetiska
kraften inte utf{\"o}r n{\aa}got arbete? L{\"o}sningen till denna paradox
{\"a}r att ovanst{\aa}ende argument h{\aa}ller f{\"o}r en {\it fri} laddning
som {\it inte {\"a}r l{\aa}st till en viss trajektoria}.
F{\"o}r en fri laddning kommer den ortogonala kraften att kontinuerligt
{\"a}ndra riktningen p{\aa} laddningen, typiskt resulterande i cirkul{\"a}ra
eller helixformade banor\numberedfootnote{Vilket exempelvis {\"a}r vad som
  h{\"a}nder med de laddade partiklar som n{\"a}r de infaller i jordens
  magnetf{\"a}lt resulterar i h{\"o}gfrekventa helix-formade trajektorior
  som genererar synligt ljus, s{\aa} kallat {\it norrsken}.}
Om vi exempelvis har ett fl{\"o}de av elektroner i en str{\"o}mslinga, s{\aa}
{\"a}r dessa l{\aa}sta i sin r{\"o}relse, och genom att de inte kan l{\"a}mna
ledaren bidrar de kollektivt till att ut{\"o}va en kraft p{\aa} ledaren.
Denna kraft p{\aa} r{\"o}relse av laddningar som genom str{\"o}mslingor {\"a}r
{\it begr{\"a}nsade} i sin r{\"o}relse utf{\"o}r sj{\"a}lvfallet arbete.

\section{Amp\`eres kraftlag - Kraften p{\aa} str{\"o}mslingor i magnetf{\"a}lt}
\sidx{Amp\`eres kraftlag}
Vi skall nu g{\aa} in p{\aa} hur krafter verkar p{\aa} laddningar som
transporteras i f{\"o}rutbest{\"a}mda banor, det vill s{\"a}ga elektrisk
str{\"o}m i {\it str{\"o}mslingor}.\sidx{Str{\"o}mslinga}
\quote{{\bf Definition:} Vi definierar str{\"o}m \sidx{Str{\"o}m} som den
  laddning, vanligtvis elektroner, som transporteras genom en givet
  tv{\"a}rsnitt per tidsenhet.}
\noindent
Kort och gott kommer vi i praktiken att definiera str{\"o}m som det antal
Coulomb som passerar en ledares tv{\"a}rsnitt per sekund, som
\sidx{Amp\`ere, enhet f{\"o}r str{\"o}m}\sidx{Coulomb, enhet f{\"o}r laddning}
$$
  1~\hbox{A} = 1~\hbox{C}/\hbox{s}.
$$
Antag att vi har en linjeladdning $\lambda$ (${\rm C}/{\rm m}$), det vill
s{\"a}ga en viss laddning $q$ utsmetad l{\"a}ngs en viss str{\"a}cka, och att
denna linjeladdning r{\"o}r sig l{\"a}ngs en fix trajektoria (ledare) $\Gamma$
i rummet med farten $v$. Under ett {\"o}gonblick $\Delta t$ r{\"o}r sig med
andra ord denna linjeladdning en str{\"a}cka $v\Delta t$ l{\"a}ngs trajektorian.
I ett tv{\"a}rsnitt av ledaren har vi d{\"a}rmed str{\"o}mmen
$$
  I={{\hbox{(Laddning)}}\over{\hbox{(tid)}}}
   ={{\lambda v \Delta t}\over{\Delta t}}
   =\lambda v.
$$
\epsfig{../lect-04/figs/ampereforce.1}\noindent
Den magnetiska kraften p{\aa} en str{\"o}mslinga l{\"a}ngs en trajektoria
fr{\aa}n ${\bf x}_a$ till ${\bf x}_b$ b{\"a}randes denna str{\"o}m, erh{\aa}lls
d{\"a}rmed genom att summera upp alla delbidrag fr{\aa}n de infinitesimala
laddningarna i r{\"o}relse enligt\numberedfootnote{Griffiths g{\aa}r
  i Ekv.~(5.16), sid.~217, vidare med denna form och konstaterar att
  str{\"o}mmen $I$ {\"o}verallt l{\"a}ngs str{\"o}mslingan {\"a}r
  riktad l{\"a}ngs linjelementen $d{\bf l}$, och att vi d{\"a}rmed
  f{\"o}r en konstant str{\"o}m $I$ l{\"a}ngs str{\"o}mslingan kan
  skriva om detta som
  $$
    {\bf F}_{\rm mag}=I\int^{{\bf x}_b}_{{\bf x}_a} (d{\bf l}\times{\bf B}).
  $$
  Denna form {\"a}r f{\"o}r v{\aa}ra {\"a}ndam{\aa}l dock lite
  f{\"o}rvirrande i notationen, d{\aa} det {\it linjeelement} $d{\bf l}$
  som ing{\aa}r i kryssprodukten {\"a}r f{\"o}rvillande likt ett
  {\it str{\"o}melement} $d{\bf I}$. Vi f{\"o}rs{\"o}ker h{\"a}r
  d{\"a}rf{\"o}r att i m{\"o}jligaste m{\aa}n undvika denna form.}
$$
  \eqalign{
    {\bf F}_{\rm mag}
      &=\lim_{\Delta l\to0}\sum_k
          \underbrace{
          ({\bf v}_k\times{\bf B}_k)
          \underbrace{\lambda\Delta l}_{=\Delta q}
          }_{=\Delta{\bf F}_{\rm mag}}\cr
      &=\int^{{\bf x}_b}_{{\bf x}_a} ({\bf v}\times{\bf B})\lambda\,dl\cr
      &=\big\{\hbox{ Str{\"o}mmen {\"a}r en vektor,
                     ${\bf I}=\lambda{\bf v}$ }\big\}\cr
      &=\int^{{\bf x}_b}_{{\bf x}_a} ({\bf I}\times{\bf B})\,dl.\cr
  }
$$
Vi brukar beteckna detta som {\it Amp\`eres kraftlag} f{\"o}r str{\"o}mslingor,
vilket inte skall f{\"o}rv{\"a}xlas med {\it Amp\`eres lag} som vi strax skall
h{\"a}rleda, och som beskriver hur magnetiska f{\"a}ltet i sig kopplar till en
str{\"o}mt{\"a}thet.\sidx{Amp\`eres kraftlag}\sidx{Str{\"o}mt{\"a}thet}
\sidx{Str{\"o}mslinga}
\vfill\eject

\section{Volymstr{\"o}mmar och ``lagen om att laddning inte kan f{\"o}rsvinna''}
\sidx{Str{\"o}mt{\"a}thet}\sidx{Lagen om att laddning inte kan f{\"o}rsvinna}
Enligt definitionen som vi h{\"a}r anv{\"a}nder f{\"o}r str{\"o}mmen $I$, s{\aa}
{\"a}r denna definierad som den laddning som per tidsenhet passerar {\it genom
ett givet tv{\"a}rsnitt}. Vi kan formulera detta som att vi genom en yta $S$,
mad randen i form av en (sluten) trajektoria $\Gamma$, har str{\"o}mmen given
i termer av en {\it str{\"o}mt{\"a}thet} ${\bf J}$, med den senare m{\"a}tt i
den laddning som transporteras per ytenhet och per tidsenhet, eller
${\rm C}/({\rm m}^2{\rm s})$,
$$
  I=\iint_S {\bf J}\cdot d{\bf S}.
$$
Vi kan sj{\"a}lvfallet utveckla detta till att g{\"a}lla den totala str{\"o}m
som passerar genom en {\it sluten} yta $S$ som omsluter en volym $V$, genom att
anv{\"a}nda Gauss lag\numberedfootnote{Se exempelvis innerp{\"a}rmen p{\aa}
  Griffiths, {\it Divergence theorem}:\sidx{Gauss lag}
  $$
    \int(\nabla\cdot{\bf A})\,dV=\oiint{\bf A}\,d{\bf S}.
  $$
  {\AA}terigen, notera att Griffiths olyckligtvis anv{\"a}nder den udda
  och vilseledande notationen $d\tau$ f{\"o}r volymelement. Normalt
  anv{\"a}nder vi $\tau$ som integrationsvariabel i {\it tid}. F{\"o}r
  att inte f{\"o}rvirra oss ytterligare v{\"a}ljer vi dessutom att
  anv{\"a}nda notationen $d{\bf S}$ f{\"o}r ytelement (``S'' f{\"o}r
  {\it surface}) inkluderande normalriktning.}
$$
  I=\Big[\hbox{str{\"o}mmen ut genom ytan $S$}\Big]
   =\oiint_S {\bf J}\cdot d{\bf S}
   =\iiint_V (\nabla\cdot{\bf J})\,dV.
$$
\sidx{Elektrisk laddning}
Eftersom ingen laddning kan skapas eller f{\"o}rintas internt i volymen (vi
erinrar oss att all laddnings\-transport in eller ut fr{\aa}n volymen sker
genom den slutna ytan $S$, s{\aa} m{\aa}ste den laddning som fl{\"o}dar
{\it ut genom ytan} g{\"o}ra att den i volymen $V$ {\it inneslutna laddningen
minskar} i motsvarande grad, det vill s{\"a}ga
$$
  \iiint_V (\nabla\cdot{\bf J})\,dV
    =-{{d}\over{dt}}\Big[\hbox{Innesluten laddning}\Big]
    =-{{d}\over{dt}}\iiint_V \rho\,dV
    =-\iiint_V {{d\rho}\over{dt}}\,dV
$$
Fl{\"o}det av laddning i den slutna ytintegralen definieras som fl{\"o}det
{\it ut genom ytan}, och vi kan som en liten {\it sanity check} konstatera att
minustecknet i h{\"o}gerledet d{\"a}rmed betyder att positiv laddning som
fl{\"o}dar {\it ut genom ytan} motsvaras av en motsvarande {\it minskning av
positiv laddning i volymen} $V$, helt enligt f{\"o}rv{\"a}ntan.

Eftersom detta argument kring fl{\"o}de av laddning ut genom en sluten yta
{\"a}r giltigt f{\"o}r en {\it godtycklig} volym $V$, s{\aa} betyder detta att
$$
  \nabla\cdot{\bf J}=-{{d\rho}\over{dt}},
$$
vilket vi betecknar\numberedfootnote{Det m{\aa}ste medges att detta {\"a}r
  en av de trixigare termerna att uttrycka p{\aa} svenska, d{\aa} man
  g{\"a}rna vill uttrycka detta som ``konservering av laddning'', vilket
  leder tanken till inl{\"a}ggningar av sill och frukt, eller ``bevarande
  av laddning'', vilket ist{\"a}llet har en air av bevarande av n{\aa}gon
  kulturhistorisk artefakt. Vi h{\aa}ller oss h{\"a}r till det tydliga om
  {\"a}n lite klumpigare ``lagen om att laddning inte kan f{\"o}rsvinna''.}
som {\it lagen om att laddning inte kan f{\"o}rsvinna}.\numberedfootnote{{\it
  Continuity Equation}; se Griffiths Ekv.~(5.29), sid.~222 samt
  Griffiths Ekv.~(8.4), sid.~356.}

\section{Aprop{\aa} detta med magnetostatik vs elektrostatik}
\sidx{Magnetostatik}\sidx{Elektrostatik}
L{\aa}t oss g{\"o}ra en liten utvikning kring detta med elektrostatik och
magnetostatik, och konstatera att vi i dessa {\it statiska} problem formellt
har att laddningst{\"a}theten $\rho$ och str{\"o}mt{\"a}theten ${\bf J}$
{\"a}r tidsoberoende {\"o}verallt i problemet, eller
$$
  {{d\rho}\over{dt}}=0
  \qquad\underline{\hbox{och}}\qquad
  {{d{\bf J}}\over{dt}}=0.
$$
{\AA}terigen, s{\aa} kan vi dock med god precision betrakta alla problem som
har en s{\aa} pass l{\aa}g associerad frekvens att den elektromagnetiska
v{\aa}gl{\"a}ngden $\lambda=c/f$ vida {\"o}verstiger problemets spatiala
utstr{\"a}ckning\numberedfootnote{Fina ord: ``spatial'' = ``i rummet'',
  ``temporal'' = ``i tiden'', ``spatiotemporal'' = ``i rumtid''.}
som just statiska problem. Exempel p{\aa} saker som vi inte kan behandla som
statiska {\"a}r typiskt radioantenner (som per definition har en
utstr{\"a}ckning i storleksordningen av en halv v{\aa}gl{\"a}ngd av den
elektromagnetiska str{\aa}lning som skall f{\aa}ngas upp eller skickas ut)
eller elektronisk apparatur i GHz-omr{\aa}det ($\lambda\sim0.3\ {\rm m}$)
och upp{\aa}t.

Ett annat s{\"a}tt att se p{\aa} statiska problem {\"a}r till exempel att vi
inte till{\aa}ter str{\"o}mmen $I$ att variera l{\"a}ngs en str{\"o}mslinga,
eftersom det ju direkt skulle betyda att vi l{\"a}ngs slingan ackumulerar
elektrisk laddning n{\aa}gonstans. Eftersom vi i magnetostatiken (enligt
observationen ovan) dessutom kr{\"a}ver att laddningst{\"a}theten $\rho$
{\"a}r konstant i tiden, s{\aa} {\"a}r divergensen av str{\"o}mt{\"a}theten
noll i statiska problem,
$$
  {{d\rho}\over{dt}}=0\qquad\Rightarrow\qquad\nabla\cdot{\bf J}=0.
$$
En tolkning av divergensen $\nabla\cdot{\bf J}=0$ {\"a}r att vi i statiska
problem (inom gr{\"a}nsen av en giltighet f{\"o}r statik som vi nyss
konstaterat {\"a}r t{\"a}mligen vid) i strikt mening inte till{\aa}ter
laddning att ackumuleras n{\aa}gonstans i problemet.

Vi n{\"a}rmar oss nu pudelns k{\"a}rna i problemet med att f{\aa} fram hur
magnetiska f{\"a}lt genereras av str{\"o}mmar.
Vi kunde i elektrostatiken se att Coulombs lag f{\"o}r v{\"a}xelverkan mellan
statiska punktladdningar\numberedfootnote{En lag f{\"o}r v{\"a}xelverkan som,
  {\it nota bene}, vi helt sonika har stadsf{\"a}st som varandes en
  fundamentalt giltig beskrivning mellan statiska laddningar utan att
  vi f{\"o}r den skull ens skissat p{\aa} ett bevis f{\"o}r den!}
via superpositionsprincipen kunde generaliseras till kontinuerliga elektriska
laddningsf{\"o}rdelningar, och att vi utifr{\aa}n dessa kunde visa p{\aa}
existensen av en skal{\"a}r, elektrostatisk potential $\phi$ definierad av
${\bf E}=-\nabla\phi$. S{\"a}kerligen kan vi nu dra fram hur {\it r{\"o}relsen}
hos en punktladdning genererar n{\aa}gon sorts ``svallv{\aa}gor'', som vi i
analogi med elektrostatiken kan generalisera och analysera f{\"o}r
str{\"o}mslingor med ett kontinuum av laddning. Eller?
\vfill\eject

\section{Biot--Savarts lag - Magnetf{\"a}lt fr{\aa}n str{\"o}mslingor}
\subsection{Historiken f{\"o}r Biot--Savarts lag}
\sidx{Biot--Savarts lag}
Efter att Hans Christian {\OE}rsted {\aa}r 1820 hade gjort sin banbrytande
uppt{\"a}ckt att en magnetn{\aa}l p{\aa}verkas av elektriska str{\"o}mmar, tog
de franska fysikerna Jean-Baptiste Biot (1774--1862) \sidx{Biot, Jean-Baptiste
(1774--1862)} och F\'elix Savart (1791--1841) \sidx{Savart, F\'elix
(1791--1841)} (b{\aa}da i Paris) samma {\aa}r upp f{\"o}rs{\"o}k med att
fysikaliskt m{\"a}ta hur stort det genererade magnetf{\"a}ltet var och vilka
lagar som kunde t{\"a}nkas styra det.\numberedfootnote{Herman Erlichson,
  {\it The experiments of Biot and Savart concerning the force exerted by
  a current on a magnetic needle}, Am.~J.~Phys. {\bf 66} (1998).}
I sina experiment sp{\"a}nde Biot och Savart upp en l{\aa}ng ledande tr{\aa}d
genom vilken de kunde leda en elektrisk str{\"o}m vertikalt och upph{\"a}ngde
vid sidan av tr{\aa}dens mitt en liten horisontell magnetn{\aa}l, som skyddades
mot luftstr{\"o}mmar av ett glasomh{\"o}lje.
F{\"o}r att s{\aa} mycket som m{\"o}jligt undg{\aa} p{\aa}verkan fr{\aa}n
jordens magnetf{\"a}lt p{\aa} experimentet anv{\"a}nde de en v{\"a}xlande
str{\"o}m genom tr{\aa}den, och kunde p{\aa} s{\aa} s{\"a}tt anv{\"a}nda
amplituden p{\aa} n{\aa}lens r{\"o}relse som m{\aa}tt p{\aa} styrkan hos
magnetf{\"a}ltet fr{\aa}n str{\"o}mb{\"a}rande tr{\aa}den.

\subsection{Sv{\aa}righeten med att formellt h{\"a}rleda Biot--Savarts lag}
Griffiths pekar i sin {\it Introduction to Electrodynamics} p{\aa} ett mycket
m{\aa}lande s{\"a}tt hur han sj{\"a}lv som f{\"o}rfattare {\"a}r mycket
frustrerad {\"o}var att ingen enkel modell kan g{\"o}ras f{\"o}r
magnetf{\"a}ltet fr{\aa}n en punktladdning i r{\"o}relse utan att ta till ett
maskineri som ligger l{\aa}ngt utanf{\"o}r omfattningen av hans bok och f{\"o}r
den delen denna kurs.
Specifikt s{\aa} pekar han p{\aa} det faktum att {\it r{\"o}relsen hos en
enskild punktladdning inte rimligen kan tolkas som en str{\"o}m}, d{\aa} den
ena tidpunkten finns p{\aa} plats, medan den {\"o}gonblicket efter{\aa}t inte
finns d{\"a}r.
En s{\aa}dan r{\"o}relse {\"a}r snarast att likna vid en diskret h{\"a}ndelse
inom den klassiska elektrodynamiken, och antagandet om station{\"a}r str{\"o}m
med $d{\bf J}/dt={\bf 0}$ {\"a}r garanterat inte uppfyllt.

Vi {\"a}r med andra ord redan fr{\aa}n b{\"o}rjan tvingade att hantera ett
{\it kontinuum av laddning i r{\"o}relse} f{\"o}r att beskriva hur en str{\"o}m
genererar ett magnetf{\"a}lt, och argumenten f{\"o}r varf{\"o}r Biot--Savarts
lag ser ut som den g{\"o}r blir d{\"a}rmed mycket st{\"o}kigare {\"a}n vad vi
fr{\aa}n en b{\"o}rjan kan f{\"o}rv{\"a}nta oss.
Med detta i bagaget kommer vi nu att g{\aa} in p{\aa} hur station{\"a}ra
str{\"o}mmar och str{\"o}mt{\"a}theter ger upphov till statiska magnetiska
f{\"a}lt.\numberedfootnote{Griffiths v{\"a}ljer redan i ett tidigt stadium
  att direkt fastst{\"a}lla Biot--Savarts kompletta lag p{\aa} integralform.
  Vi v{\"a}ljer h{\"a}r att f{\"o}rst ta ett litet steg i och med
  betraktandet av ett litet {\it linjeelement} l{\"a}ngs med
  str{\"o}mslingan.}

\subsection{Biot--Savarts lag f{\"o}r str{\"o}mslingor}
\sidx{Biot--Savarts lag}
\epsfig{../lect-04/figs/biotsavart.1}\noindent
Om vi betraktar ett linjeelement $d{\bf l}'$ vid k{\"a}llpunkten ${\bf x}'$ med
beloppet $|d{\bf l}'|=dl'$ l{\"a}ngs en str{\"o}mslinga uppb{\"a}rande
str{\"o}mmen ${\bf I}({\bf x}')$ med beloppet $|{\bf I}({\bf x}')|=I$, liksom
tidigare med konventionen att vi s{\"a}tter ett prim p{\aa} vad vi betraktar
som k{\"a}lla i problemet, s{\aa} ger detta (linj{\"a}ra) linjeelement
bidraget\numberedfootnote{Notera {\aa}terigen att vi h{\"a}r l{\"a}tt
  riskerar att f{\"o}rv{\"a}xla {\it linjeelementet} $d{\bf l}'$ (som
  har den fysikaliska dimensionen {\it l{\"a}ngd}) med ett delbidrag
  till den riktade str{\"o}mmen.}{$^{,}$}%
  \numberedfootnote{Det finns ett par underh{\aa}llande
    mini-f{\"o}rel{\"a}sningar p{\aa} YouTube kring hur vi kan tolka kraften
    p{\aa} en laddning i r{\"o}relse n{\"a}ra en str{\"o}mb{\"a}rande ledare,
    det vill s{\"a}ga ett magnetf{\"a}lt genererat enligt Biot--Savarts lag
    s{\aa} som vi h{\"a}r presenterar den, som en direkt effekt av den speciella
    relativitetsteorin applicerad p{\aa} en reservoar av laddning.
    Se till exempel Veritasium, {\it How Special Relativity Makes Magnets Work},
    {\tt https://www.youtube.com/watch?v=1TKSfAkWWN0}, eller
    Fermilab Lectures, {\it How Einstein saved magnet theory},
    {\tt https://www.youtube.com/watch?v=d29cETVUk-0}}
$$
  d{\bf B}({\bf x})={{\mu_0}\over{4\pi}}
    {{{\bf I}({\bf x}')\times({\bf x}-{\bf x}')}
      \over{|{\bf x}-{\bf x}'|^3}}\,dl'
$$
till det magnetiska f{\"a}ltet vid observationspunkten ${\bf x}$. Notera
f{\"o}rekomsten av\numberedfootnote{Detta fixerade v{\"a}rde f{\"o}r den
  magnetiska permeabiliteten definierar i SI enheten Amp\`ere, vilken i
  sin tur (med definitionen av sekund) definierar enheten Coulomb.
  Termen ``permeabilitet'' myntades 1885 av Oliver Heaviside (1850--1925).}
$$
  \mu_0=4\pi\times10^{-7}\ {\rm H}/{\rm m}\quad(\hbox{exakt per definition})
$$
f{\"o}r den {\it magnetiska permeabiliteten i vakuum};
\sidx{Magnetisk permeabilitet}[Vakuumpermeabilitet $\mu_0$]
\sidx{Elektrisk permittivitet}[Vakuumpermittivitet $\varepsilon_0$]
detta {\"a}r f{\"o}rsta g{\aa}ngen som $\mu_0$ dyker upp i denna kurs, p{\aa}
exakt samma s{\"a}tt som $\varepsilon_0$ d{\"o}k upp f{\"o}r f{\"o}rsta
g{\aa}ngen i elektrostatiken i och med att vi introducerades till Coulombs lag.
\sidx{Coulombs kraftlag}
I sj{\"a}lva verket kan vi h{\"a}danefter generellt identifiera
``sl{\"a}kttr{\"a}det'' f{\"o}r v{\aa}ra ekvationer som
\sidx{Elektromagnetiskt sl{\"a}kttr{\"a}d}
\medskip
\item{$\bullet$}{F{\"o}rekomst av {\it enbart} elektrisk
  permittivitet $\varepsilon_0$ $\Rightarrow$ Elektrostatik.}
\item{$\bullet$}{F{\"o}rekomst av {\it enbart} magnetisk
  permeabilitet $\mu_0$ $\Rightarrow$ Magnetostatik.}
\item{$\bullet$}{F{\"o}rekomst av {\it produkten}
  $\varepsilon_0\mu_0$ $\Rightarrow$ Elektromagnetism (elektrodynamik).}
\medskip
\noindent
Om vi summerar upp alla bidrag $d{\bf B}$ till magnetf{\"a}ltet vid
observationspunkten ${\bf x}$, fr{\aa}n alla k{\"a}llor l{\"a}ngs med
str{\"o}mslingan fr{\aa}n ${\bf x}_a$ till ${\bf x}_b$, det vill s{\"a}ga
f{\"o}r alla {\it k{\"a}llpunkter} ${\bf x}'$, s{\aa} erh{\aa}ller vi
direkt Biot--Savarts lag p{\aa} integralform som
linjeintegralen\numberedfootnote{Griffiths Ekv.~(5.34), sid.~224.}
$$
  {\bf B}({\bf x})={{\mu_0}\over{4\pi}}\int^{{\bf x}_b}_{{\bf x}_a}
    {{{\bf I}({\bf x}')\times({\bf x}-{\bf x}')}
      \over{|{\bf x}-{\bf x}'|^3}}\,dl'.
$$
\sidx{Biot--Savarts lag}[F{\"o}r str{\"o}mslinga]
Vi kan redan h{\"a}r se att Biot--Savarts lag kan r{\"a}knas som motsvarigheten
till Coulombs lag i elektrostatiken, dock h{\"a}r ist{\"a}llet relaterande till
{\it r{\"o}relse (dynamik) av laddning}. Som alternativ form av Biot--Savarts
lag s{\aa} kan vi bryta ut den konstanta str{\"o}mmen $I$ som en skal{\"a}r,
och ist{\"a}llet uttrycka som linjeintegralen med linjeelementen $d{\bf l}'$
(l{\"a}ngdelement l{\"a}ngs med str{\"o}mslingan) som
$$
  {\bf B}({\bf x})={{\mu_0 I}\over{4\pi}}\int^{{\bf x}_b}_{{\bf x}_a}
    {{d{\bf l}'\times({\bf x}-{\bf x}')}\over{|{\bf x}-{\bf x}'|^3}}.
$$

\subsection{Biot--Savarts lag f{\"o}r str{\"o}mt{\"a}theten i volymer}
En viss generalisering av Biot--Savarts lag kan g{\"o}ras om vi rekapitulerar
att str{\"o}mmen $I$ ju faktiskt {\"a}r ett specialfall av ett m{\aa}tt av en
str{\"o}mt{\"a}thet ${\bf J}({\bf x}')$ (med ${\bf x}'$ liksom tidigare
varande {\it k{\"a}llpunkter}) som r{\aa}kar vara s{\aa} funtad att den bara
fl{\"o}dar i en enda kurva. I detta fall inneh{\aa}ller ju $I$ redan en
ytintegral {\"o}ver str{\"o}mt{\"a}theten ${\bf J}({\bf x}')$ och man inser
direkt att motsvarande Biot--Savarts lag f{\"o}r {\it str{\"o}mt{\"a}theten}
uttrycks som volymintegralen\numberedfootnote{Griffiths Ekv.~(5.47), sid.~231.}
$$
  {\bf B}({\bf x})={{\mu_0}\over{4\pi}}\iiint_V
  {{{\bf J}({\bf x}')\times({\bf x}-{\bf x}')}\over{|{\bf x}-{\bf x}'|^3}}\,dV'.
$$
\sidx{Biot--Savarts lag}[F{\"o}r str{\"o}mt{\"a}thet]

\section{Divergens f{\"o}r magnetf{\"a}ltet - ``Magnetiska monopoler
  existerar inte''}
\sidx{Magnetiska monopoler}[Icke-existens av]
Notera att integralen som f{\"o}rekommer i Biot--Savarts lag sker {\"o}ver
{\it primmade} koordinater ${\bf x}'$, i vilket vi betraktar observationspunkten
${\bf x}$ som fix. Vi kommer nu att s{\"o}ka uttryck f{\"o}r divergensen och
rotationen f{\"o}r det magnetiska f{\"a}ltet\numberedfootnote{Redan nu kan
  vi g{\"o}ra klart f{\"o}r oss sj{\"a}lva att denna exercis inte {\"a}r
  en exercis f{\"o}r exercisens egen skull, utan f{\"o}r att detta senare,
  i F{\"o}rel{\"a}sning~9 solitt kommer att assistera oss i bygget av
  Maxwells ekvationer, n{\aa}got som i sin tur beskriver all
  elektromagnetisk v{\aa}gutbredning! Med andra ord, {\"a}ven om man
  kan tycka att detta stycke kring magnetostatiken kan vara lite torrt
  och intets{\"a}gande, l{\aa}t oss betrakta detta som en pusselbit
  f{\"o}r vad som komma skall.}
${\bf B}$ vilket vi rekapitulerar {\"a}r {\it operationer som sker i det
icke-primmade} observations-koordinatsystemet ${\bf x}$.

Om vi f{\"o}rst analyserar divergensen f{\"o}r den magnetiska f{\"a}ltet,
s{\aa} som Biot--Savarts lag uttrycker det utifr{\aa}n den generella
beskrivningen av det i termer av str{\"o}mt{\"a}theten ${\bf J}$, s{\aa}
har vi att\numberedfootnote{Griffiths Ekv.~(5.50), sid.~232.}
$$
  \eqalign{
  \nabla\cdot{\bf B}({\bf x})
    &={{\mu_0}\over{4\pi}}\nabla\cdot\iiint_V
        {{{\bf J}({\bf x}')\times({\bf x}-{\bf x}')}
          \over{|{\bf x}-{\bf x}'|^3}}\,dV'\cr
    &=\big\{\hbox{ $\nabla$ opererar p{\aa} ${\bf x}$,
                   {\it inte} p{\aa} ${\bf x}'$ }\big\}\cr
    &={{\mu_0}\over{4\pi}}\iiint_V\nabla\cdot
        \bigg(
        {\bf J}({\bf x}')\times
          {{({\bf x}-{\bf x}')}\over{|{\bf x}-{\bf x}'|^3}}
        \bigg)\,dV'\cr
    &=\big\{\hbox{ Griffiths {\it Product Rule \#6}
                   med ``${\bf a}={\bf J}({\bf x}')$'' }\big\}\cr
    &=\big\{\hbox{ $\nabla\cdot({\bf a}\times{\bf b})
                     ={\bf b}\cdot(\nabla\times{\bf a})
                       -{\bf a}\cdot(\nabla\times{\bf b})$ }\big\}\cr
    &=\big\{
        \hbox{ Notera att ${\bf J}({\bf x}')$ oberoende av ${\bf x}$ }
      \big\}\cr
    &=-{{\mu_0}\over{4\pi}}\iiint_V
        {\bf J}({\bf x}')\cdot
        \underbrace{
        \bigg(
          \nabla\times{{({\bf x}-{\bf x}')}\over{|{\bf x}-{\bf x}'|^3}}
        \bigg)}_{\hbox{$=0, {\rm exercis (1.63)}$}}
        \,dV'\cr
    &=0.
  }
$$
Att vi har\sidx{Tricket
$\displaystyle\nabla{{1}\over{\char124}{\bf x}-{\bf x}'{\char124}}
=-{{({\bf x}-{\bf x}')}\over{{\char124}{\bf x}-{\bf x}'{\char124}^3}}$}
\sidx{Magnetisk fl{\"o}dest{\"a}thet}[Divergens f{\"o}r]
$$
  \nabla\times{{({\bf x}-{\bf x}')}\over{|{\bf x}-{\bf x}'|^3}}\equiv{\bf 0}
$$
kan vi antingen h{\"a}r verifiera genom att utf{\"o}ra differentieringen och
vektoralgebran direkt, eller konstatera att vi i F{\"o}rel{\"a}sning~2, f{\"o}r
rotationen hos det statiska elektriska f{\"a}ltet, fann att just precis det vi
har att g{\"o}ra med h{\"a}r kunde skrivas som en gradient (``tricket'' i
tolkningen av Coulombintegralen), som
$$
  {{({\bf x}-{\bf x}')}\over{|{\bf x}-{\bf x}'|^3}}\equiv
    -\nabla{{1}\over{|{\bf x}-{\bf x}'|}},
$$
och eftersom\numberedfootnote{Se exempelvis innerp{\"a}rmen p{\aa} Griffiths,
  {\it Second derivatives (10)}.}
$$
  \nabla\times(\nabla f)=0
$$
f{\"o}r godtycklig skal{\"a}r funktion $f$, s{\aa} f{\"o}ljer det direkt att
rotationen ovan {\"a}r identiskt noll, och f{\"o}ljdaktligen ocks{\aa} att
divergensen f{\"o}r magnetf{\"a}ltet {\"a}r identiskt noll.

Vad s{\"a}ger oss resultatet att divergensen $\nabla\cdot{\bf B}=0$?
Mer {\"a}n man kan tro, faktiskt. Vi kan direkt j{\"a}mf{\"o}ra detta resultat
med det {\it elektrostatiska} fallet (s{\aa} som vi gick igenom det i
F{\"o}rel{\"a}sning~1) f{\"o}r det elektriska f{\"a}ltet, d{\"a}r vi s{\aa}g
att Gauss lag $\nabla\cdot{\bf E}=\rho/\varepsilon_0$ ger relationen mellan det
elektriska f{\"a}ltet och den lokala elektriska laddningen $\rho$ (eller
{\it laddningst{\"a}theten}, om man skall vara korrekt).
I det magnetostatiska fall som vi h{\"a}r g{\aa}tt igenom har vi med andra ord
att det magnetiska fl{\"o}det ut genom en sluten yta alltid {\"a}r identiskt
noll, och vi har d{\"a}rmed ingen m{\"o}jlighet att innesluta n{\aa}gon
``magnetisk laddning''.

V{\aa}r slutsats blir d{\"a}rmed:
\sidx{Magnetiska monopoler}[Icke-existens av]
\quote{{\it $\nabla\cdot{\bf B}=0$ betyder att det inte existerar
  n{\aa}gon magnetisk laddning!}}
\noindent
Med konstaterandet att ``magnetisk laddning inte existerar'' menar vi h{\"a}r
att {\it magnetiska monopoler} inte existerar, och att {\it magnetiska f{\"a}lt
endast manifesterar sig s{\aa} som om de h{\"a}rr{\"o}r fr{\aa}n magnetiska
dipoler}, det vill s{\"a}ga ``en positiv och negativ laddning p{\aa} ett
avst{\aa}nd fr{\aa}n varandra''.
Denna terminologi anknyter till elektrisk laddning, som p{\aa} samma s{\"a}tt
handlar om {\it elektriska monopoler} (som ju faktiskt existerar) som kan
s{\"a}ttas samman till elektriska dipoler.

\section{Rotationen f{\"o}r magnetf{\"a}ltet}
\sidx{Magnetisk fl{\"o}dest{\"a}thet}[Rotation f{\"o}r]
Med den intressanta observationen att {\it magnetiska monopoler inte existerar}
i bagaget, l{\aa}t oss nu g{\aa} vidare med rotationen av magnetf{\"a}ltet.
Vi anv{\"a}nder {\"a}ven h{\"a}r den generella formen av Biot--Savarts
lag,\numberedfootnote{Notera hur vi {\"a}ven nu tar avstamp i Biot--Savarts
  lag som den bas fr{\aa}n vilket allt inom elektrostatiken tar sin
  b{\"o}rjan, samt hur permeabiliteten $\mu_0$ hakar p{\aa} i allt
  som h{\"a}rleds fr{\aa}n denna.}
vilken d{\aa} vi applicerar rotationen ({\aa}terigen med observationen att
$\nabla$ opererar p{\aa} {\it oprimmade} koordinater ${\bf x}$) ger oss
att\numberedfootnote{F{\"o}r den som endast {\"a}r ute efter resultatet,
  s{\aa} kan man med f{\"o}rdel skippa h{\"a}rledningen nedan och g{\aa}
  vidare direkt till slutresultatet i form av Amp\`eres lag p{\aa} sid.~13.
  H{\"a}rledningen {\"a}r h{\"a}r genomg{\aa}ngen i detalj f{\"o}r att den
  som {\"a}r skeptiskt lagd skall f{\aa} en chans att verifiera just hur
  vi g{\aa}r fr{\aa}n Biot--Savarts lag till Amp\`eres lag p{\aa}
  differentialform.}
$$
  \eqalign{
  \nabla\times{\bf B}({\bf x})
    &={{\mu_0}\over{4\pi}}\iiint_V\nabla\times
      \bigg(
        {\bf J}({\bf x}')\times{{({\bf x}-{\bf x}')}\over{|{\bf x}-{\bf x}'|^3}}
      \bigg)\,dV'\cr
  }
$$
Liksom i fallet med divergensen kommer vi nu att anv{\"a}nda en produktregel
fr{\aa}n innerp{\"a}rmen p{\aa} Griffiths, i detta fall {\it Product Rule \#8},
som med sina fyra termer {\"a}r aningen st{\"o}kigare, dock liksom tidigare
med ${\bf a}={\bf J}({\bf x}')$ (vilken {\"a}r oberoende av ${\bf x}'$ och
d{\"a}rmed ger noll vid differentiering) och med
${\bf b}=({\bf x}-{\bf x}')/|{\bf x}-{\bf x}'|^3$,
$$
  \nabla\times({\bf a}\times{\bf b})
    =\underbrace{{\bf a}(\nabla\cdot{\bf b})}_{\hbox{``Term 1''}}
      -\underbrace{({\bf a}\cdot\nabla){\bf b}}_{\hbox{``Term 2''}}
      +\underbrace{({\bf b}\cdot\nabla){\bf a}}_{
         =0,\ {\bf J}(\underline{\underline{\underline{{\bf x}'}}})},
      -\underbrace{{\bf b}(\nabla\cdot{\bf a})}_{
         =0,\ {\bf J}(\underline{\underline{\underline{{\bf x}'}}})},
$$
vilket {\"o}versatt till integranden ovan resulterar i att\numberedfootnote{Just
  i denna h{\"a}rledning {\"a}r Griffiths lite spretig och bygger mycket
  p{\aa} h{\"a}rledningar som gjorts tidigare i hans {\it Introduction to
  Electrodynamics}. Vi kommer h{\"a}r att f{\"o}rs{\"o}ka h{\aa}lla samman
  den stundvis aningen komplexa h{\"a}rledningen i ett stycke, med hopp om
  att det blir l{\"a}ttare att f{\"o}lja resonemanget.}
$$
  \eqalign{
    \nabla\times
        \bigg(
        {\bf J}({\bf x}')\times
          {{({\bf x}-{\bf x}')}\over{|{\bf x}-{\bf x}'|^3}}
        \bigg)
        =\underbrace{
        {\bf J}({\bf x}')
        \bigg(
        \nabla\cdot
          {{({\bf x}-{\bf x}')}\over{|{\bf x}-{\bf x}'|^3}}
        \bigg)
        }_{\hbox{``Term 1''}}
        -\underbrace{
        \big({\bf J}({\bf x}')\cdot\nabla\big)
        \bigg(
          {{({\bf x}-{\bf x}')}\over{|{\bf x}-{\bf x}'|^3}}
        \bigg)
        }_{\hbox{``Term 2''}}
  }
$$
\vfill\eject

\subsection{Term~1 i rotationen}
H{\"a}r involverar ``Term~1'' en divergens som vi kan {\"o}vers{\"a}tta till
en delta-puls placerad i ${\bf x}'$, eftersom Gauss teorem (divergensteoremet)
$\int(\nabla\cdot{\bf A})\,dV=\oint{\bf A}\cdot d{\bf S}$ ger
att\numberedfootnote{Griffiths har redan gjort f{\"o}rarbetet i kapitlet
  {\it Vector Analysis}, Ekv.~(1.100), sid.~50; se {\"a}ven Sektion 1.5.1,
  sid.~45. Vi kommer h{\"a}r f{\"o}r sakens skull dock att g{\aa} igenom
  denna exercis s{\aa} att vi h{\aa}ller resonemanget kring
  $\nabla\times{\bf B}$ sammanh{\aa}llet och koncist.}
$$
  \eqalign{
    \iiint_V\nabla\cdot{{({\bf x}-{\bf x}')}\over{|{\bf x}-{\bf x}'|^3}}\,dV'
    &=\oiint_S{{({\bf x}-{\bf x}')}
         \over{|{\bf x}-{\bf x}'|^3}}\cdot d{\bf S}'\cr
    &=\big\{\hbox{ Integrera {\"o}ver sf{\"a}r $S$ med
                   radie $|{\bf x}-{\bf x}'|=R$ }\big\}\cr
    &=\big\{\hbox{ Yttryck i sf{\"a}riska kordinater med
                   ${\bf x}'$ som origo }\big\}\cr
    &=\int^{\pi}_{0}\int^{2\pi}_{0} {{R{\bf e}_r}\over{R^3}}\cdot
      \underbrace{{\bf e}_r\,R^2\sin(\vartheta)
        \,d\varphi\,d\vartheta}_{=d{\bf S}\ {\hbox{p{\aa}}}\ S}\cr
    &=\underbrace{\int^{\pi}_{0}\sin(\vartheta)\,d\vartheta}_{=2}
        \underbrace{\int^{2\pi}_{0} d\varphi}_{=4\pi}\cr
    &=4\pi,\cr
  }
$$
f{\"o}r godtyckligt vald radie $R>0$. Samtidigt har vi ju faktiskt att
divergensen i sig ges som
$$
  \eqalign{
    \nabla\cdot{{({\bf x}-{\bf x}')}\over{|{\bf x}-{\bf x}'|^3}}
    &=\bigg(
        {{\partial}\over{\partial x}},
        {{\partial}\over{\partial y}},
        {{\partial}\over{\partial z}}
      \bigg)\cdot
      \bigg(
        {{(x-x',y-y',z-z')}\over{\big[(x-x')^2+(y-y')^2+(z-z')^2\big]^{3/2}}}
      \bigg)\cr
    &={{(1+1+1)}\over{\big[(x-x')^2+\ldots\big]^{3/2}}}
        -{{3}\over{2}}{{(2(x-x'),2(y-y'),2(z-z'))\cdot(x-x',y-y',z-z')}
            \over{\big[(x-x')^2+\ldots\big]^{5/2}}}\cr
    &={{3}\over{\big[(x-x')^2+\ldots\big]^{3/2}}}
        -3{{\big[(x-x')^2+(y-y')^2+(z-z')^2\big]}
            \over{\big[(x-x')^2+(y-y')^2+(z-z')^2\big]^{5/2}}}\cr
    &={{3}\over{\big[(x-x')^2+\ldots\big]^{3/2}}}
        -{{3}\over{\big[(x-x')^2+\ldots\big]^{3/2}}}\cr
    &=0,\cr
  }
$$
f{\"o}r alla observationspunkter ${\bf x}$ i rummet, under
f{\"o}ruts{\"a}ttningen att ${\bf x}\ne{\bf x}'$ (det vill s{\"a}ga att
n{\"a}mnaren $[(x-x')^2+\ldots]^{3/2}\ne0$).
Notera att detta resultat g{\"a}ller {\it oberoende} av v{\"a}rdet p{\aa}
radien $R=|{\bf x}-{\bf x}'|>0$, som vi kan v{\"a}lja godtyckligt liten runt
k{\"a}llpunkten ${\bf x}'$.
Utifr{\aa}n detta argument kan vi dra slutsatsen att {\it divergensen h{\"a}r
kan tolkas som en delta-puls} placerad i ${\bf x}'$, det vill s{\"a}ga {\it att
den {\"a}r noll {\"o}verallt i rummet utom just i k{\"a}llpunkten} ${\bf x}'$,
som
$$
  \nabla\cdot{{({\bf x}-{\bf x}')}\over{|{\bf x}-{\bf x}'|^3}}
    =4\pi\delta({\bf x}-{\bf x}').
$$
Med andra ord kan vi {\"o}vers{\"a}tta ``Term~1'' ovan som\numberedfootnote{Vid
  en anblick p{\aa} detta {\"a}r det paradoxalt att divergensen av en funktion
  som bevisligen {\"o}verallt {\"a}r riktad ut{\aa}t fr{\aa}n k{\"a}llpunkten
  ${\bf x}'$ har v{\"a}rdet noll {\"o}verallt f{\"o}rutom just vid
  k{\"a}llpunkten i sig.}
$$
  \eqalign{
  \iiint_V
  \underbrace{
    {\bf J}({\bf x}')
    \bigg(
    \nabla\cdot
      {{({\bf x}-{\bf x}')}\over{|{\bf x}-{\bf x}'|^3}}
    \bigg)
    }_{\hbox{``Term 1''}}\,dV'
    &=4\pi\iiint_V{\bf J}({\bf x}')\delta({\bf x}-{\bf x}')\,dV'\cr
    &=4\pi{\bf J}({\bf x}).\cr
  }
$$

\subsection{Term~2 i rotationen}
Den andra termen som upptr{\"a}der i integranden som upptr{\"a}der i uttrycket
f{\"o}r $\nabla\times{\bf B}$ kan {\"a}ven den utvecklas
vidare,\numberedfootnote{Griffiths Ekv.~(5.54), sid.~232.} som
$$
  \eqalign{
    \iiint_V
      \underbrace{
        \big({\bf J}({\bf x}')\cdot\nabla\big)
        \bigg({{({\bf x}-{\bf x}')}\over{|{\bf x}-{\bf x}'|^3}}\bigg)
      }_{\hbox{``Term 2''}}\,dV'
    &=\big\{\hbox{ Notera att $\nabla$ opererar p{\aa} ${\bf x}$ }\big\}\cr
    &=\big\{\hbox{ $\nabla\to\nabla'\quad\Rightarrow\quad$ teckenbyte }\big\}\cr
    &=-\iiint_V
        \big({\bf J}({\bf x}')\cdot\nabla'\big)
        \bigg({{({\bf x}-{\bf x}')}\over{|{\bf x}-{\bf x}'|^3}}\bigg)\,dV'
    \cr
  }
$$
Om vi f{\"o}r enkelhets skull tittar p{\aa} detta uttryck komponentvis, med
$x_k=x,y,z$, s{\aa} har vi att med  Griffiths {\it Product Rule (5)} och
``${\bf a}={\bf J}$'',
$$
 \nabla'(f{\bf a})=f(\nabla'\cdot{\bf a})
     +\underbrace{
         {\bf a}\cdot(\nabla'f)
      }_{=({\bf a}\cdot\nabla')f}
 \quad\Leftrightarrow\quad
 ({\bf J}\cdot\nabla')f=\nabla'(f{\bf J})-f(\nabla'\cdot{\bf J})
$$
att d{\aa} vi dessutom observerar att vi f{\"o}r station{\"a}ra str{\"o}mmar
enligt tidigare har att $\nabla'\cdot{\bf J}=0$, so erh{\aa}ller vi
$$
  \eqalign{
        \big({\bf J}({\bf x}')\cdot\nabla'\big)
        \bigg(\underbrace{
          {{x_k-x'_k}\over{|{\bf x}-{\bf x}'|^3}}
        }_{\hbox{=``$f$''}}\bigg)
    &=\nabla'\cdot\bigg(
    \underbrace{
      {{x_k-x'_k}\over{|{\bf x}-{\bf x}'|^3}}
    }_{\hbox{=``$f$''}}{\bf J}({\bf x}')
    \bigg)\cr
  }
$$
D{\aa} vi applicerar Gauss lag p{\aa} detta resultat, under det att vi l{\aa}ter
integrera {\"o}ver en omslutande yta stor nog att innesluta alla
str{\"o}mb{\"a}rande k{\"a}llor till Biot--Savarts lag och att vi d{\"a}rmed
inte har n{\aa}gra in- eller utg{\aa}ende str{\"o}mmar genom denna yta, blir
bidraget fr{\aa}n ``Term~2'' kort och gott f{\"o}r var och en av komponenterna
$x_k=x,y,z$ att
$$
  \eqalign{
    \iiint_V
      \underbrace{
        \big({\bf J}({\bf x}')\cdot\nabla\big)
        \bigg({{(x_k-x'_k)}\over{|{\bf x}-{\bf x}'|^3}}\bigg)
      }_{\hbox{``Term 2''}}\,dV'
    &=-\iiint_V\nabla'\cdot\bigg(
      {{x_k-x'_k}\over{|{\bf x}-{\bf x}'|^3}}{\bf J}({\bf x}')
      \bigg)
      \,dV'\cr
    &=\big\{\hbox{ Gauss lag }\big\}\cr
    &=-\oiint_S\bigg(
      {{x_k-x'_k}\over{|{\bf x}-{\bf x}'|^3}}{\bf J}({\bf x}')
      \bigg)\cdot d{\bf S}\cr
    &=0.
  }
$$
Med andra ord, genom att utnyttja v{\aa}r frihet att definiera v{\aa}r
integrationsdom{\"a}n till att helt innesluta str{\"o}mmarna som utg{\"o}r
k{\"a}llor i Biot--Savarts lag, n{\aa}got som faller sig helt naturligt, s{\aa}
kan vi motivera att bidraget till rotationen av det magnetiska f{\"a}ltet
fr{\aa}n ``Term~2'' {\"a}r noll.

\subsection{Slutligt resultat f{\"o}r rotationen av magnetf{\"a}ltet}
L{\aa}t oss nu s{\"a}tta samman dessa delresultat f{\"o}r ``Term~1'' och
``Term~2'' till ett uttyck f{\"o}r rotationen f{\"o}r magnetf{\"a}ltet,
$$
  \eqalign{
  \nabla\times{\bf B}({\bf x})
    &={{\mu_0}\over{4\pi}}\iiint_V\nabla\times
      \bigg(
        {\bf J}({\bf x}')\times{{({\bf x}-{\bf x}')}\over{|{\bf x}-{\bf x}'|^3}}
      \bigg)\,dV'\cr
    &={{\mu_0}\over{4\pi}}
        \underbrace{
          \iiint_V
          \underbrace{
            {\bf J}({\bf x}')
            \bigg(
            \nabla\cdot
              {{({\bf x}-{\bf x}')}\over{|{\bf x}-{\bf x}'|^3}}
            \bigg)
          }_{\hbox{``Term 1''}}
          \,dV'
        }_{\displaystyle=4\pi{\bf J}({\bf x}),\hbox{ enligt ovan}}
        -{{\mu_0}\over{4\pi}}
        \underbrace{
          \iiint_V\underbrace{
            \big({\bf J}({\bf x}')\cdot\nabla\big)
            \bigg(
              {{({\bf x}-{\bf x}')}\over{|{\bf x}-{\bf x}'|^3}}
            \bigg)
          }_{\hbox{``Term 2''}}
          \,dV'
        }_{\displaystyle=0,\hbox{ enligt ovan}}\cr
    &=\mu_0{\bf J}({\bf x}).
  }
$$

\section{Amp\`eres lag}
\sidx{Amp\`eres lag}[F{\"or}r statiska magnetiska f{\"a}lt]
V{\aa}rt slutliga resultat f{\"o}r rotationen f{\"o}r det magnetiska f{\"a}ltet
p{\aa} differentiell form,
$$
  \nabla\times{\bf B}=\mu_0{\bf J},
$$
kallas f{\"o}r {\it Amp\`eres lag}, och kommer fram{\"o}ver i kursen att ha en
stor betydelse inte bara f{\"o}r hur vi kan ber{\"a}kna kopplingen mellan
str{\"o}mmar och magnetf{\"a}lt, utan {\"a}ven (som det kommer att visa sig
i F{\"o}rel{\"a}sning~9) f{\"o}r hur vi kan formulera elektromagnetisk
v{\aa}gutbredning med Maxwell's ekvationer (med assistans av en
till{\"a}ggsterm till Amp\`eres magnetostatiska lag, som vi d{\aa} kommer att
g{\aa} igenom). {\it Notera h{\"a}r hur vi genomg{\aa}ende kan sp{\aa}ra
f{\"o}rekomsten av $\mu_0$ till Biot--Savarts lag.}

Amp\`eres lag kan enkelt omformuleras p{\aa} integralform genom anv{\"a}ndandet
av Stokes teorem, f{\"o}r en sluten slinga $\Gamma$ inneslutande en yta $S$ i
magnetf{\"a}ltet och str{\"o}mt{\"a}theten, som\numberedfootnote{Se exempelvis
  innerp{\"a}rmen p{\aa} Griffiths, {\it Curl theorem}.}
$$
  \iint_{S}(\nabla\times{\bf B})\cdot d{\bf S}=
  \oint_{\Gamma}{\bf B}\cdot d{\bf l}=
  \mu_0\underbrace{\iint_{S}{\bf J}\cdot d{\bf S}}_{\displaystyle =I_{\rm enc}},
$$
d{\"a}r nu $\iint_{S}{\bf J}\cdot d{\bf S}=I_{\rm enc}$ {\"a}r den av $\Gamma$
inneslutna str{\"o}mmen (med andra ord den totala str{\"o}m som passerar genom
integrationsytan $S$). Med andra ord har vi integralformen av Amp\`eres lag som
$$
  \oint_{\Gamma}{\bf B}\,d{\bf l}=\mu_0I_{\rm enc}.
$$
Denna form {\"a}r ofta att f{\"o}redra i situationer d{\aa} vi kan utnyttja
rent geometriskt--symmetriskt gynn\-samma situationer, p{\aa} precis samma
s{\"a}tt som vi kunnat konstatera med Gauss lag och hur den markant kan
f{\"o}renkla probleml{\"o}sande i elektrostatiska problem med symmetrier
n{\"a}rvarande.

Exempel: Ber{\"a}kning av magnetf{\"a}lt genererade runt str{\"o}mslingor,
i samma geometri som i Biot--Savarts ursprungliga
experiment.\numberedfootnote{Oneliner f{\"o}r magnetf{\"a}lt runt
  o{\"a}ndlig rak ledare b{\"a}rande str{\"o}mmen $I$:
  $$
    \oint_{\Gamma}{\bf B}\,d{\bf l}
      =\int^{2\pi}_{0}B_{\varphi}(r)\,rd\varphi
      =2\pi rB_{\varphi}(r)
      =\mu_0 I_{\rm enc}
      =\mu_0 I\quad\Rightarrow\quad
    B_{\varphi}(r)={{\mu_0 I}\over{2\pi r}}.
  $$}

\section{Vektorpotentialen}
\sidx{Vektorpotential}\sidx{Vektorpotential}[Amp\`eres lag]
Vi erinrar oss att ekvationen $\nabla\times{\bf E}={\bf 0}$ i elektrostatiken
ledde oss till att dra slutsatsen att det existerar en skal{\"a}r potential
definierad genom ${\bf E}=-\nabla\phi$. P{\aa} samma s{\"a}tt inbjuder
$\nabla\cdot{\bf B}=0$ (det vill s{\"a}ga att {\it inga magnetiska monopoler
existerar}) oss till att via {\it vektoridentieteten}\numberedfootnote{Se
  exempelvis innerp{\"a}rmen p{\aa} Griffiths, {\it Second derivatives (9)}.}
$$
  \nabla\cdot(\nabla\times{\bf A})=0
$$
tolka magnetf{\"a}ltet ${\bf B}$ som sprunget ur en {\it vektorpotential}
${\bf A}$ enligt
$$
  {\bf B}=\nabla\times{\bf A}.
$$
Om vi substituerar denna potential f{\"o}r magnetf{\"a}ltet ${\bf B}$ i
Amp\`eres lag p{\aa} differentialform, s{\aa} erh{\aa}ller vi f{\"o}r
v{\"a}nsterledet\numberedfootnote{Se exempelvis innerp{\"a}rmen p{\aa}
  Griffiths, {\it Second derivatives (11)}.}
$$
  \nabla\times{\bf B}=\nabla\times(\nabla\times{\bf A})
    =\nabla\underbrace{(\nabla\cdot{\bf A})}_{=0}-\nabla^2{\bf A}
    =-\nabla^2{\bf A}
$$
Detta g{\"o}r att vi kan formulera Amp\`eres lag i termer av vektorpotentialen
som\numberedfootnote{Griffiths Ekv.~(5.64), sid.~244. Notera att liksom i
  fallet med Poissons ekvation $\nabla^2\phi=-\rho/\varepsilon_0$ f{\"o}r
  den elektrostatiska skal{\"a}ra potentialen fr{\aa}n F{\"o}rel{\"a}sning~3,
  s{\aa} betraktar Griffiths denna form som s{\aa} fundamental att den
  {\"a}r den andra som visas p{\aa} omslaget till {\it Introduction to
  Electrodynamics}.}
\sidx{Amp\`eres lag}[Uttryckt i vektorpotentialen]
$$
  \nabla^2{\bf A}=-\mu_0{\bf J}.
$$

\section{Explicit l{\"o}sning f{\"o}r vektorpotentialen}
\sidx{Vektorpotential}[Explicit l{\"o}sning]
Eftersom vektorpotentialen beskrivs av Poissons ekvation, vilket i grund och
botten {\"a}r en {\it skal{\"a}r} partiell differentialekvation, som kan
projiceras ut komponentvis ${\bf J}=(J_x,J_y,J_z)$ f{\"o}r vektorpotentialen,
$$
  \nabla^2 A_k=-\mu_0 J_k,\quad k=x,y,z,
$$
s{\aa} kan vi direkt notera likheten mellan denna och motsvarande ekvation
f{\"o}r den skal{\"a}ra potentialen
$$
  \nabla^2\phi=-\rho/\varepsilon_0.
$$
Nu r{\aa}kar vi ha s{\aa}dan tur att en explicit l{\"o}sning till Poissons
ekvation f{\"o}r den skal{\"a}ra potentialen har erh{\aa}llits i
F{\"o}rel{\"a}sning~3, som\numberedfootnote{Om vi skall vara riktigt petiga
  h{\"a}r, s{\aa} var det faktiskt s{\aa} att vi i F{\"o}rel{\"a}sning
  {\it definierade} den skal{\"a}ra potentialen $\phi$ som detta uttryck.
  Anledningen till denna bekv{\"a}ma definition (som i m{\aa}ngt och mycket
  handlar om vilket tecken p{\aa} $-\nabla\phi$ v{\"a}ljer att definiera
  det elektriska f{\"a}ltet efter) var dock just att denna via Coulombs
  generaliserade ekvation (``Coulombintegralen'') l{\"o}ser ekvationen
  f{\"o}r det elektriska f{\"a}ltet i n{\"a}rvaro av laddningsf{\"o}rdelning
  $\rho$, s{\aa} vi har visst fog f{\"o}r att denna definition ocks{\aa}
  {\"a}r en formell l{\"o}sning till Poissons ekvation f{\"o}r den
  skal{\"a}ra potentialen.}
d{\"a}r vi definierade den {\it skal{\"a}ra potentialen} $\phi({\bf x})$
explicit som integralen
$$
  \phi({\bf x})={{1}\over{4\pi\varepsilon_0}}\iiint_V
    {{\rho({\bf x}')}\over{|{\bf x}-{\bf x}'|}}\,dV'.
$$
Med andra ord kan vi direkt {\"o}verf{\"o}ra detta resultat {\"a}ven f{\"o}r
vektorpotentialens komponenter, med endast en liten {\"a}ndring i koefficienten
$1/\varepsilon_0\to\mu_0$, s{\aa} att en explicit l{\"o}sning i termer av
str{\"o}mf{\"o}rdelningen $J_k$ erh{\aa}lls som
$$
  {\bf A}({\bf x})={{\mu_0}\over{4\pi}}\iiint_V
    {{{\bf J}({\bf x}')}\over{|{\bf x}-{\bf x}'|}}\,dV',
  \qquad\Leftrightarrow\qquad
  A_k({\bf x})={{\mu_0}\over{4\pi}}\iiint_V
    {{J_k({\bf x}')}\over{|{\bf x}-{\bf x}'|}}\,dV',\quad k=x,y,z,
$$
och motsvarande f{\"o}r en str{\"o}m ${\bf I}=(I_x,I_y,I_z)$ som linjeintegralen
$$
  {\bf A}({\bf x})={{\mu_0}\over{4\pi}}\int_{\Gamma}
    {{{\bf I}({\bf x}')}\over{|{\bf x}-{\bf x}'|}}\,dl'.
$$
Vi kommer i F{\"o}rel{\"a}sning~8 att studera hur denna form f{\"o}r den
skal{\"a}ra potentialen $\phi$ s{\aa}v{\"a}l som vektorpotentialen ${\bf A}$
kan utvecklas i serier i det inversa avst{\aa}ndet $1/|{\bf x}-{\bf x}'|$ mellan
k{\"a}lla och observationspunkt, i den s{\aa} kallade {\it multipolutvecklingen}.
\vfill\eject

\section{Sammanfattning av F{\"o}rel{\"a}sning~4 -- Magnetostatik}
\item{$\bullet$}{Lorentz-kraften p{\aa} fri laddning $q$ med hastighet
  ${\bf v}$ {\"a}r ${\bf F}=q\big({\bf E}+{\bf v}\times{\bf B}\big)$.}
\item{$\bullet$}{Magnetiska krafter (p{\aa} fria laddningar) utf{\"o}r
  inget arbete!}
\item{$\bullet$}{Amp\`eres kraftlag p{\aa} str{\"o}mslinga b{\"a}rande
  str{\"o}mmen $I$,
  $$
    {\bf F}_{\rm mag}=\int^{{\bf x}_b}_{{\bf x}_a} ({\bf I}\times{\bf B})\,dl.
  $$}
\item{$\bullet$}{Str{\"o}mmen $I$ genom en yta $S$ ges av str{\"o}mt{\"a}theten
  ${\bf J}$ som
  $$
    I=\iint_S {\bf J}\cdot d{\bf S},
  $$
\item{$\bullet$}{I en volym med konduktivitet $\sigma$ och ett elektriskt
  f{\"a}lt ${\bf E}$ ges str{\"o}mt{\"a}theten som ${\bf J}=\sigma{\bf E}$.}
\item{$\bullet$}{Lagen om att elektrisk laddning inte kan f{\"o}rsvinna
  beskrivs av sambandet mellan str{\"o}mt{\"a}thet ${\bf J}$ och
  laddningst{\"a}thet $\rho$ som
  $$
    \nabla\cdot{\bf J}=-{{d\rho}\over{dt}}.
  $$}
\item{$\bullet$}{Statiska problem definieras av att
  $$
    {{d\rho}\over{dt}}=0\quad\underline{\hbox{och}}\quad{{d{\bf J}}\over{dt}}=0.
  $$}
\item{$\bullet$}{Divergensen av str{\"o}mt{\"a}theten {\"a}r noll i
  {\it statiska problem}, vilket {\"a}r en direkt f{\"o}ljd av att
  laddnings\-t{\"a}t\-heten {\"a}r tidsoberoende,
  $$
    {{d\rho}\over{dt}}=0\quad\Rightarrow\quad\nabla\cdot{\bf J}=0.
  $$}
\item{$\bullet$}{Biot--Savarts generella lag f{\"o}r samband mellan
  magnetf{\"a}ltet ${\bf B}$ och str{\"o}mt{\"a}theten ${\bf J}$ ges som
  $$
    {\bf B}({\bf x})={{\mu_0}\over{4\pi}}\iiint_V
        {{{\bf J}({\bf x}')\times({\bf x}-{\bf x}')}
          \over{|{\bf x}-{\bf x}'|^3}}\,dV',
  $$
  d{\"a}r $\mu_0=4\pi\times10^{-7}\ {\rm H}/{\rm m}$ (exakt, per definition)
  {\"a}r den magnetiska permeabiliteten i vakuum. Biot--Savarts motsvarande
  lag f{\"o}r str{\"o}mslingor med str{\"o}m $I$ fr{\aa}n en punkt ${\bf x}_a$
  till ${\bf x}_b$ ges som
  $$
    {\bf B}({\bf x})={{\mu_0}\over{4\pi}}\int^{{\bf x}_b}_{{\bf x}_a}
        {{{\bf I}({\bf x}')\times({\bf x}-{\bf x}')}
          \over{|{\bf x}-{\bf x}'|^3}}\,dl'.
  $$}
\item{$\bullet$}{Det g{\"a}ller {\it alltid} att
  $$
    \nabla\cdot{\bf B}=0.
  $$
  Nollan i h{\"o}gerledet har som direkt f{\"o}ljd, via tolkning genom
  Gauss lag,  att ``magnetisk laddning'' inte existerar! (``Magnetisk
  laddning'' {\"a}r h{\"a}r ekvivalent med ``magnetiska monopoler''.)}
\item{$\bullet$}{Amp\`eres magnetostatiska lag p{\aa} integral- respektive
  differentialform,
  $$
    \oint_{\Gamma}{\bf B}\cdot d{\bf l}
      =\mu_0\iint_S{\bf J}\cdot d{\bf S}
      =\mu_0I_{\rm enc}
    \quad\Leftrightarrow\quad
    \nabla\times{\bf B}=\mu_0{\bf J}.
$$\sidx{Amp\`eres lag}[F{\"or}r statiska magnetiska f{\"a}lt]}
\item{$\bullet$}{Ur ``lagen om att inga magnetiska monopoler existerar''
  kan vi direkt formulera vektorpotentialen ${\bf A}$ som
  $$
    \nabla\cdot{\bf B}=0\quad\Leftrightarrow\quad{\bf B}=\nabla\times{\bf A}.
  $$}
\item{$\bullet$}{Amp\`eres lag f{\"o}r vektorpotentialen ${\bf A}$ (andra
  ekvationen p{\aa} omslaget p{\aa} Griffiths!) ges som Poissons ekvation
  med den fria str{\"o}mt{\"a}theten ${\bf J}$ som k{\"a}llterm,
  $$
    \nabla^2{\bf A}=-\mu_0{\bf J},
  $$
  med den explicita l{\"o}sningen
  $$
    {\bf A}({\bf x})={{\mu_0}\over{4\pi}}\iiint_V
      {{{\bf J}({\bf x}')}\over{|{\bf x}-{\bf x}'|}}\,dV',
    \qquad\Leftrightarrow\qquad
    A_k({\bf x})={{\mu_0}\over{4\pi}}\iiint_V
      {{J_k({\bf x}')}\over{|{\bf x}-{\bf x}'|}}\,dV',\quad k=x,y,z.
  $$}

\cleardoublepage
%%% End of auto-extracted text from ../lect-04/lecture-04.tex %%%
%%% Begin of auto-extracted text from ../lect-05/lecture-05.tex %%%
%
% File: teach/elmagii/lect-05/lecture-05.tex [plain TeX code]
% Github: https://github.com/elmagii/lect-04/
% Last change: November 15, 2025
%
% Lecture No 5 in the course ``Elektromagnetism II, 1TE626 (2025)'',
% held November 17, 2025, at Uppsala University, Sweden.
%
% Copyright (C) 2022-2025, Fredrik Jonsson, under Gnu General Public
% License (GPL) v3. See the enclosed LICENSE for details.
%
% This program is free software: you can redistribute it and/or modify
% it under the terms of the GNU General Public License as published by
% the Free Software Foundation, either version 3 of the License, or
% (at your option) any later version.
%
% This program is distributed in the hope that it will be useful,
% but WITHOUT ANY WARRANTY; without even the implied warranty of
% MERCHANTABILITY or FITNESS FOR A PARTICULAR PURPOSE.  See the
% GNU General Public License for more details.
%
% You should have received a copy of the GNU General Public License
% along with this program.  If not, see <https://www.gnu.org/licenses/>.
%
\def\coursename{Elektromagnetism II}
\def\coursecode{1TE626}
\def\courseyear{2025}
\def\courserepo{https://github.com/hp35/elmagii/}
\def\lecturenumber{5}
\def\lecturetitle{Elektrodynamik -- Elektromagnetisk induktion}
\def\lecturesubtitle{}
\def\lectureauthor{Fredrik Jonsson}
\def\lectureplace{Uppsala Universitet}
\def\lecturedate{17 november 2025}
%-------------------- BEGIN OF LOCAL MACROS --------------------
\edef\expandedlecturenumber{5}
\def\ifempty#1{\ifx\relax#1\relax}
\advance\chapno by 1
\secno=0
\footnotenumber=0
\message{==================== Lecture 5 ====================}
\writenumberedtocentry{chapter}{F{\"o}rel{\"a}sning 5 -- {Elektrodynamik -- Elektromagnetisk induktion}}{\thechapno}
\hsize=150mm\hoffset=4.6mm\vsize=230mm\voffset=7mm
\topskip=0pt\baselineskip=12pt\parskip=0pt\leftskip=0pt\parindent=15pt
\ifcolors
  \voffset=-10.2mm\topskip=0pt
\fi
\headline={\ifnum\secno>0\ifodd\pageno\rightheadline\else\leftheadline\fi
  \else\hfill\fi}
\def\rightheadline{\tenrm{\it F\"orel\"asning 5}
  \hfil{\it \coursename, \coursecode\ (\courseyear)}}
\def\leftheadline{\tenrm{\it \coursename, \coursecode\ (\courseyear)}
  \hfil{\it F\"orel\"asning 5}}
\noindent~\vskip-60pt\hskip-40pt{\epsfbox{../lect-01/macros/UU_logo_color.eps}}
\vskip-42pt\hfill\vbox{
    \hbox{{\it \coursename, \coursecode\ (\courseyear)}}
    \hbox{{\it Lecture Notes, \lectureauthor}}
    \hbox{{\it Document Revision \today}}
    \hbox{{\it \courserepo}}}\vskip 36pt
\centerline{\twelvesc F\"orel\"asning 5}
\vskip 24pt\noindent
\centerline{\twelvesc{Elektrodynamik -- Elektromagnetisk induktion}}
\expandafter\ifempty\expandafter{\lecturesubtitle}%
  \else\centerline{\twelvesc\lecturesubtitle}\fi
\bigskip
\centerline{\lectureauthor, \lectureplace, \lecturedate}
\vskip24pt
%--------------------- END OF LOCAL MACROS ---------------------



\plan{{\"A}mnet f{\"o}r f{\"o}rel{\"a}sningen beskrivs kort och koncist
  med Michael Faradays \sidx{Faraday, Michael (1791--1867)}
  egna ord (1831), fritt tolkade som ``Ett varierande magnetf{\"a}lt
  inducerar ett elektriskt f{\"a}lt''. Vi g{\aa}r igenom definitionerna
  av magnetiskt fl{\"o}de $\Phi_{\rm M}$ och det historiskt betingade
  begreppet elektromotorisk ``kraft'' ${\cal E}$.

  Vi h{\"a}rleder Faradays induk\-tions\-lag ${\cal E}=-d\Phi_{\rm M}/dt$ i
  tv{\aa} separata och inb{\"o}rdes sammanh{\aa}llna fall.
  Det f{\"o}rsta fallet introduceras f{\"o}r sin enkelhet och intuitivt
  greppbara geometri, d{\"a}r vi studerar en rektangul{\"a}r str{\"o}mslinga
  som f{\"o}rs genom ett inhomogent magnetiskt f{\"a}lt och kommer p{\aa}
  s{\aa} s{\"a}tt fram till formen p{\aa} Faradays induktionslag via ett
  specialfall.
  Det andra fallet {\"a}r en formell h{\"a}rledning av Faradays induktionslag
  f{\"o}r en godtycklig r{\"o}rlig geometri i form av en slinga med godtycklig
  hastighet utefter sin trajektoria, samt med ett godtyckligt varierande
  inneslutet magnetf{\"a}lt.

  Vi noterar att Faradays induktionslag h{\"a}rleds enbart utifr{\aa}n Lorentz
  kraftlag och ej involverande vare sig Coulombs eller Biot--Savarts lag, eller
  n{\aa}got av deras derivat i det elektromagnetiska ``sl{\"a}kttr{\"a}det''.
  Utifr{\aa}n Faradays induktionslag formulerar vi Lenz lag som slutsatsen
  att en inducerad str{\"o}m alltid har en riktning som motverkar orsaken
  till att den uppkom. Vi g{\aa}r utifr{\aa}n denna princip igenom en
  upps{\"a}ttning av tankeexperiment med b{\"a}ring p{\aa} tolkning av
  elektromagnetisk induktion.
  Med utg{\aa}ngspunkt i Faradays induktionslag h{\"a}rleder vi Faradays lag
  $\nabla\times{\bf E}=-\partial{\bf B}/\partial t$, ofta betecknad
  ``Maxwell--Faradays lag'', p{\aa} differential- och integralform.

  Vi avslutar med att g{\aa} igenom hur tv{\aa} str{\"o}mb{\"a}rande slingor
  p{\aa}verkar varandra genom {\"o}msesidig induktion, och vi h{\"a}rleder
  Neumanns formel f{\"o}r den {\"o}msesidiga induktansen. Specifikt g{\aa}r
  vi igenom tolkningen av Neumanns formel i form av elektromagnetisk
  reciprocitet mellan tv{\aa} slingor, d{\"a}r vi har det sm{\aa}tt
  f{\"o}rbluffande resultatet att det magnetiska fl{\"o}de som uppf{\aa}ngas
  av en slinga fr{\aa}n en str{\"o}m i den andra slingan exakt motsvaras av
  det magnetiska fl{\"o}de som den andra slingan skulle uppf{\aa}nga om
  ist{\"a}llet den f{\"o}rsta slingan drevs med exakt samma str{\"o}m.}

\threepointsummary{%
  Faradays induktionslag,
  $
    {\cal E}=-{{d\Phi_{\rm M}}/{dt}}.
  $
}{%
  Faradays lag,
  $$
    \nabla\times{\bf E}({\bf r},t)
      =-{{\partial{\bf B}({\bf x},t)}\over{\partial t}}
    \quad\Leftrightarrow\quad
    \oint_{\Gamma}{\bf E}({\bf r},t)\cdot d{\bf l}
      =-{{d}\over{dt}}\iint_S{\bf B}({\bf x},t)\cdot d{\bf S}.
  $$
}{%
  {\"O}msesidig induktans enligt Neumanns formel,
  \sidx{Neumanns formel}[{\"O}msesidig induktans]
  $M_{\Gamma\Gamma'}=M_{\Gamma'\Gamma}$.
}
\vfill\eject\copyrights

\section{Introduktion - Faradays induktionslag}
Vi kan i en enda mening sammanfatta {\"a}mnet f{\"o}r dagens
f{\"o}rel{\"a}sning\numberedfootnote{Vi kommer i denna f{\"o}rel{\"a}sning
  att i huvudsak f{\"o}lja Griffiths Kapitel~7, med h{\"a}rledningen av
  Faradays induktionslag p{\aa} Sid.~307--309.}
med Michael Faradays (1791--1867) \sidx{Faraday, Michael (1791--1867)} egen
slutsats, som fritt tolkad lyder:
\quote{\vbox{\hbox{{\it Ett varierande magnetf{\"a}lt inducerar ett
  elektriskt f{\"a}lt.}}\hbox{\hbox to 116pt{}--- Michael Faraday (1831)}}}
\noindent
F{\"o}r {\it elektrostatiska} \sidx{Elektrostatik} system g{\"a}ller det, som
tidigare visats i F{\"o}rel{\"a}sning~2, att som en f{\"o}ljd av att rotationen
av det elektriska f{\"a}ltet alltid {\"a}r noll \sidx{Elektrostatiskt
f{\"a}lt}[Rotation f{\"o}r]  i elektrostatiska system, s{\aa} {\"a}r via Stokes
teorem \sidx{Stokes teorem} integralen av det elektriska f{\"a}ltet {\"o}ver en
godtycklig sluten slinga $\Gamma$ alltid {\"a}r noll, det vill s{\"a}ga
att\numberedfootnote{Recap
  fr{\aa}n F{\"o}rel{\"a}sning~3: F{\"o}r {\it statiska} elektriska f{\"a}lt
  g{\"a}ller alltid att $\nabla\times{\bf E}={\bf 0}$, s{\aa} med Stokes
  teorem ({\it Curl Theorem}) har vi i elektrostatiken trivialt att
  $$
    \oint_{\Gamma}{\bf E}\cdot d{\bf l}
      =\iint_S\underbrace{\nabla\times{\bf E}}_{={\bf 0}}\cdot d{\bf S}
      =0.
  $$}
$$
  \oint_{\Gamma}{\bf E}\cdot d{\bf l}=0.
  \qquad\hbox{(Statiskt!)}
$$
Michael Faraday observerade 1831 att i tidsberoende (icke-statiska = dynamiska)
f{\"a}lt s{\aa} g{\"a}ller inte detta, utan f{\"a}ltet driver en str{\"o}m som
lyder
$$
  \oint_{\Gamma}{\bf E}\cdot d{\bf l}=-{{d\Phi_{\rm M}}\over{dt}},
  \qquad\hbox{(Dynamiskt!)}
$$
d{\"a}r $\Phi_{\rm M}$ {\"a}r det {\it magnetiska fl{\"o}det}, \sidx{Magnetiskt fl{\"o}de $\Phi_{\rm M}$} som vi strax kommer att definiera. Vi brukar i vardagligt tal kalla
denna ekvation {\it Faradays induktionslag}\numberedfootnote{Denna beteckning
  {\"a}r i sig lite olycklig, d{\aa} vi l{\"a}tt kan associera denna
  med Faradays lag p{\aa} differentialform som vi inom kort kommer att
  stifta bekantskap med,
  $$
    \nabla\times{\bf E}=-{{\partial{\bf B}}\over{\partial t}}.
  $$
  Dock kommer vi inom kort att visa hur denna kan h{\"a}rmedas ur just
  \idx{Faradays induktionslag} p{\aa} formen som involverar magnetiska
  fl{\"o}det $\Phi_{\rm M}$.
  Vissa textb{\"o}cker inom elektromagnetism tar Faradays lag p{\aa} denna
  differentialform som ett {\it axiom} (postulat) utan att visa p{\aa} hur
  formen uppkommer, och g{\aa}r helt enkelt till att visa hur induktionslagen
  erh{\aa}lls fr{\aa}n denna form; detta {\"a}r dock fusk och ett
  cirkelresonemang som vi skall h{\aa}lla oss borta fr{\aa}n i m{\"o}jligaste
  m{\aa}n.},
eller kort och gott bara {\it induktionslagen}.\sidx{Induktionslagen}[Faradays
induktionslag]

Vi skall h{\"a}r notera att Faradays induktionslag, fr{\aa}n vilken vi kommer
att h{\"a}rleda Faradays lag (eller ``Maxwell--Faradays lag'', som
engelsk-spr{\aa}kig litteratur ofta betecknar den) {\it inte} {\"a}r m{\"o}jlig
att h{\"a}rleda enbart fr{\aa}n elektrostatiska eller magnetostatiska teorem
som Coulombs \sidx{Coulombs kraftlag} eller Biot--Savarts
lag. \sidx{Biot--Savarts lag}
Induktion {\"a}r ett strikt {\it dynamiskt} fenomen d{\"a}r vi v{\"a}ljer att
antingen
\medskip
\item{A.}{H{\"a}rleda den fr{\aa}n Lorentz-kraften plus information om
  hur magnetf{\"a}ltet ${\bf B}({\bf x},t)$ beror i tiden, eller}
\item{B.}{Helt enkelt godta den empiriskt verifierade ``magnetiska
  fl{\"o}desregeln'' som ett axiom fr{\aa}n vilket vi kan h{\"a}rleda
  Faradays lag.}
\medskip
\noindent
Vi kommer h{\"a}r sj{\"a}lvfallet att formellt h{\"a}rleda fram Faradays lag
fr{\aa}n Lorentz-kraften, \sidx{Lorentz-kraften} d{\aa} det vore n{\"a}stintill
moraliskt f{\"o}rkastligt i en kurs som denna att inte ta tillf{\"a}llet i akt
och en g{\aa}ng f{\"o}r alla visa p{\aa} ursprunget f{\"o}r denna fundamentala
lag inom elektrodynamiken.\numberedfootnote{Det {\"a}r mycket vanligt att
  se ``h{\"a}rledningar'' av Faradays induktionslag (``magnetiska
  fl{\"o}des\-lagen'') ${\cal E}=-d\Phi_{\rm M}/dt$ utg{\aa}ende
  ifr{\aa}n Faradays lag $\nabla\times{\bf E}=-\partial{\bf B}/\partial t$
  som om den senare vore en axiomatisk sanning som inte beh{\"o}ver bevisas.
  (Faradays lag kommer ist{\"a}llet i denna f{\"o}rel{\"a}sning att h{\"a}rledas
  fr{\aa}n induktionslagen.) Denna approach med Faradays lag som ett axiom
  ger endast ett cirkelresonemang, d{\"a}r den ena vyn av induktion
  {\"o}msesidigt bevisar den andra. Om man fr{\aa}gar exempelvis \idx{ChatGPT},
  \idx{Grok} eller n{\aa}gon annan LLM ({\it large language model}) s{\aa} finner
  man sorgligt nog direkt att svaret man f{\aa}r s{\aa} gott som uteslutande
  bygger p{\aa} just Faradays lag som ett axiom.}

\section{Grundl{\"a}ggande begrepp inf{\"o}r Faradays induktionslag}
\subsection{Definition: Magnetiskt fl{\"o}de}
\sidx{Magnetiskt fl{\"o}de $\Phi_{\rm M}$}
I likhet med det elektriska fl{\"o}det $\Phi_{\rm E}$ som vi introducerade i
F{\"o}rel{\"a}sning~1, definierar vi det {\it magnetiska fl{\"o}det}
$\Phi_{\rm M}$ som integralen av normalkomponenten av det magnetiska f{\"a}ltet
${\bf B}$ {\"o}ver den yta $S$ som innesluts av en sluten trajektoria $\Gamma$,
som
$$
  \Phi_{\rm M}=\iint_S{\bf B}\cdot d{\bf S}.
$$
\epsfig{../lect-05/figs/magflow.1}\noindent
L{\aa}t oss nu f{\"o}rst se hur detta magnetiska fl{\"o}de kan t{\"a}nkas ha
ett tidsberoende. Ur definitionen av fl{\"o}det $\Phi_{\rm M}$ ser vi direkt
att om antingen
\medskip
\item{1.}{Det magnetiska f{\"a}ltet som s{\aa}dant {\"a}r tidsberoende,
  ${\bf B}={\bf B}({\bf x},t)$,}
\item{2.}{Ytan hos den slutna slingan $\Gamma$ {\"a}ndras, eller}
\item{3.}{Den slutna slingan $\Gamma$ p{\aa} n{\aa}got s{\"a}tt {\"a}ndras
  i rummet, exempelvis genom att roteras eller translateras,}
\medskip
\noindent
s{\aa} kommer det magnetiska fl{\"o}det $\Phi_{\rm M}=\Phi_{\rm M}(t)$ att f{\aa}
ett tidsberoende.\sidx{Magnetiskt fl{\"o}de $\Phi_{\rm M}$}[Tidsberoende]

\subsection{Magnetisk fl{\"o}dest{\"a}thet och lite terminologi}
\sidx{Magnetisk fl{\"o}dest{\"a}thet}\sidx{Magnetf{\"a}lt}
Den korrekta svenska beteckningen f{\"o}r det som vi lite l{\"o}st kallat
``magnetf{\"a}lt'' (${\bf B}$-f{\"a}ltet) {\"a}r {\it magnetisk
fl{\"o}dest{\"a}thet}. Anledningen till denna terminologi st{\aa}r ganska
klar i och med vad vi diskuterat ovan, d{\aa} det {\it magnetiska fl{\"o}det}
definieras av integralen\sidx{Magnetiskt fl{\"o}de $\Phi_{\rm M}$}
$$
  \Phi_{\rm M}=\iint_S{\bf B}\cdot d{\bf S},
$$
vilket direkt g{\"o}r att vi b{\"o}r associera ``magnetf{\"a}ltet'' ${\bf B}$
med en slags {\it densitet av fl{\"o}de per ytenhet}, eller kort och gott som
just en {\it magnetisk fl{\"o}dest{\"a}thet}. Vi kan p{\aa} s{\"a}tt och vis
s{\"a}ga att den magnetiska fl{\"o}dest{\"a}theten ${\bf B}$ {\"a}r ett
m{\aa}tt p{\aa} hur m{\aa}nga magnetiska f{\"a}ltlinjer
\sidx{Magnetiska f{\"a}ltlinjer} som sk{\"a}r en yta per ytenhet.
En annan beteckning f{\"o}r ${\bf B}$-f{\"a}ltet {\"a}r {\it magnetstyrka},
vilken dock {\"a}r betydligt mer intets{\"a}gande, {\"a}ven om det ger en
parallell association till dess elektriska motsvarighet ${\bf E}$-f{\"a}ltet
i form av {\it elektrisk f{\"a}ltstyrka}.\sidx{Elektrisk f{\"a}ltstyrka}

\subsection{Definition: Elektromotorisk ``kraft'' - EMK}
\sidx{Elektromotorisk ``kraft'' ${\cal E}$}\sidx{Str{\"o}mslinga}[Sluten]
\sidx{Lorentz-kraften}
Vi rekapitulerar att Lorentzkraften\numberedfootnote{Efter Hendrik Antoon
  Lorentz (1853--1928), en Holl{\"a}ndsk teoretisk fysiker.\sidx{Lorentz,
  Hendrik Antoon (1853--1928)}}
p{\aa} en punktladdning $q$ \sidx{Punktladdning} i r{\"o}relse med hastigheten
${\bf v}$ i ett kombi\-nerat elektriskt och magnetiskt f{\"a}lt {\"a}r given som
$$
  {\bf F}=q\big[{\bf E}+({\bf v}\times{\bf B})\big].
$$
Om vi t{\"a}nker oss att vi f{\"o}r en laddad partikel l{\"a}ngs en sluten
slinga $\Gamma$, och att vi under denna r{\"o}relse integrerar den kraft som
verkar p{\aa} punktladdningen och dividerar denna med laddningen i sig, s{\aa}
f{\aa}r vi den resulterande potentialskillnad som agerar f{\"o}r att skicka
laddningen som en str{\"o}m genom slingan.
Den potentialskillnad som ackumulerats runt slingan {\"a}r d{\aa}
$$
  {\cal E}=\oint_{\Gamma}\bigg({{{\bf F}}\over{q}}\bigg)\cdot d{\bf l}
    =\oint_{\Gamma}[{\bf E}+({\bf v}\times{\bf B})\big]\cdot d{\bf l},
$$
vilken brukar betecknas med den olyckligtvis t{\"a}mligen missvisande termen
``elektromotorisk kraft'', eller kort och gott EMK. \sidx{Elektromotorisk
``kraft'' ${\cal E}$}
Att detta inte {\"a}r en ``kraft'' i egentlig bem{\"a}rkelse {\"a}r tydligt
fr{\aa}n den fysikaliska dimensionen hos ${\cal E}$, som {\"a}r~V (volt), men
termen har satt sig fr{\aa}n dess historiska sammanhang, och {\"a}r idag den
allm{\"a}nt vedertagna.\numberedfootnote{Som ett uttryck f{\"o}r v{\aa}r
  allm{\"a}nna irritation {\"o}ver detta spr{\aa}kbruk, som g{\aa}r tv{\"a}rs
  emot den fysikaliska dimensionen, s{\aa} kommer vi fram{\"o}ver genomg{\aa}ende
  att anv{\"a}nda citattecken n{\"a}rhelst denna ``kraft'' n{\"a}mns!}
Likaledes {\"a}r symbolen ${\cal E}$ f{\"o}r den elektromotoriska ``kraften''
olycklig, d{\aa} den ju ger intryck av att vi har att g{\"o}ra med ett
elektriskt f{\"a}lt, vilket ju heller ej {\"a}r fallet, men {\aa}terigen
s{\aa} {\"a}r ``${\cal E}$'' allm{\"a}nt anv{\"a}nt inom litteraturen,
s{\aa} vi h{\aa}ller kvar vid denna.

En annan lite udda aspekt {\"a}r hur vi skall se p{\aa} denna elektromotoriska
``kraft'' f{\"o}r en {\it sluten slinga} $\Gamma$, som ju rimligen har
startpunkten f{\"o}r integralen exakt sammanfallande med slutpunkten.
Med andra ord:
\quote{{\it Hur kan vi ha en potentialskillnad i en och samma punkt i rummet?}}
\noindent
\sidx{Potentialskillnad}
Svaret p{\aa} denna paradox {\"a}r att den elektromotoriska ``kraften'' {\"a}r
ett rent konceptuellt begrepp som inf{\"o}rts som ett {\it skal{\"a}rt m{\aa}tt
p{\aa} hur en noll-skild rotation av ett f{\"a}lt yttrar sig d{\aa} vi
traverserar en sluten trajektoria genom det}.
Vi kan se det som den sp{\"a}nning som skulle alstras i en ledare l{\"a}ngs med
trajektorian $\Gamma$, d{\"a}r vi kan t{\"a}nka oss att vi klippt av slingan
och kopplat in en voltmeter {\"o}ver de l{\"o}sa {\"a}ndarna som h{\aa}lls
mycket n{\"a}ra varandra.\numberedfootnote{Vi kan h{\"a}r p{\aa}minna
  oss om att vi i F{\"o}rel{\"a}sning~2 tog fram att vi f{\"o}r
  {\it elektrostatiska} f{\"a}lt hade att $\nabla\times{\bf E}={\bf 0}$; detta
  noll-resultat {\"a}r n{\aa}got som vi nu l{\"a}mnar bakom oss i och med att
  vi nu kommer att introducera tidsberoende, {\it elektrodynamiska} f{\"a}lt.}
\vfill\eject

\section{Faradays induktionslag h{\"a}rledd f{\"o}r en rektangul{\"a}r
  slinga med konstant hastighet}
\sidx{Faradays induktionslag}[Slinga med konstant hastighet]
Innan vi tar itu en {\it generell} h{\"a}rledning av Faradays induktionslag,
som tyv{\"a}rr riskerar att f{\aa} oss att fastna i vektoralgebra och tappa
fokus p{\aa} sj{\"a}lva k{\"a}rnan i den, l{\aa}t oss f{\"o}rst betrakta ett
f{\"o}renklat specialfall med en rektangul{\"a}r slinga $\Gamma$ som med
konstant hastighet ${\bf v}_0$ dras ut ur ett rektangul{\"a}rt omr{\aa}de med
ett i {\"o}vrigt homogent magnetf{\"a}lt ${\bf B}_0$, ortogonalt mot slingans
plan och ortogonalt mot den konstanta hastigheten ${\bf v}_0$.
\epsfig{../lect-05/figs/faradayrect.1}\noindent
L{\aa}t oss se vad detta h{\"o}gst f{\"o}renklade system kan ge i form av
elektromotorisk ``kraft''. Till att b{\"o}rja med, s{\aa} har vi f{\"o}r
samtliga fyra segment av $\Gamma$ att Lorentz-kraften, om vi antar att inga
statiska elektriska f{\"a}lt {\"a}r med i problemet, beskrivs av
$$
  \eqalign{
    {\bf F}&=q({\bf v}\times{\bf B}_0)\cr
      &=q({\bf e}_z v_0)\times({\bf e}_y B_0))\cr
      &=-q v_0 B_0 {\bf e}_x.\cr
  }
$$
\vfill\eject
N{\"a}r vi integrerar denna kraft, normaliserad med laddningen $q$, l{\"a}ngs
med $\Gamma$, s{\aa} ser vi att segmenten \encircle{2} och \encircle{4} som
{\"a}r ortogonala mot denna kraft ger noll i bidrag, eftersom
skal{\"a}rprodukten ${\bf F}\cdot d{\bf l}$ d{\"a}r {\"a}r identiskt noll.
Med andra ord {\"a}r det bara segmenten \encircle{1} och \encircle{3} som
kan ge bidrag, och eftersom segment \encircle{3} ligger utanf{\"o}r
magnetf{\"a}ltet och d{\"a}rmed {\"a}ven det ger noll i bidrag till
linjeintegralen, s{\aa} {\"a}r det endast segment \encircle{1} som ger ett
nettobidrag till v{\aa}r elektromotoriska ``kraft'',
$$
  \eqalign{
    {\cal E}&=\oint_{\Gamma}\bigg({{{\bf F}}\over{q}}\bigg)\cdot d{\bf l}
       =\oint_{\Gamma}({\bf v}\times{\bf B}_0)\cdot d{\bf l}\cr
      &=\oint_{\hbox{\encircle{1}}}({\bf v}\times{\bf B}_0)\cdot d{\bf l}
          +\oint_{\hbox{\encircle{2}}}\underbrace{
              ({\bf v}\times{\bf B}_0)\cdot d{\bf l}
           }_{=0,\ {\bf v}\parallel d{\bf l}}
          +\oint_{\hbox{\encircle{3}}}
              \underbrace{({\bf v}\times{\bf 0})\cdot d{\bf l}}_{=0}
          +\oint_{\hbox{\encircle{4}}}\underbrace{
              ({\bf v}\times{\bf B}_0)\cdot d{\bf l}
           }_{=0,\ {\bf v}\parallel d{\bf l}}\cr
      &=\int_{\hbox{\encircle{1}}}(-v_0 B_0 {\bf e}_x)\cdot({\bf e}_x dx)
       =-v_0 B_0\int_{\hbox{\encircle{1}}}dx\cr
      &=-v_0 B_0 L.\cr
  }
$$
Samtidigt har vi att det magnetiska fl{\"o}det $\Phi_{\rm M}$ ges av
ytintegralen\numberedfootnote{Vi skall h{\"a}r komma ih{\aa}g
  att integrationsriktningen som vi valt f{\"o}r den slutna linjeintegralen
  {\"o}ver slingan $\Gamma$ dessutom best{\"a}mmer vilken riktning v{\aa}ra
  ytelement $d{\bf S}$ har, med den sedvanliga ``h{\"o}gerhands\-regeln''.
  I detta fall har vi valt en integration som g{\aa}r {\it moturs} i slingans
  plan, s{\aa} som vi ser den i figuren, och ytelementen har s{\aa} en
  {\it normalriktning som pekar ut ur planet}, i negativ ${\bf e}_y$-riktning.}
{\"o}ver den yta $S$ som innesluts av slingan $\Gamma$ och har ett nollskilt
bidrag fr{\aa}n det magnetiska f{\"a}ltet. Det magnetiska fl{\"o}det har en
f{\"o}r{\"a}ndring i tiden som ges av\numberedfootnote{Notera att
  vi i denna f{\"o}rel{\"a}sningsserie genomg{\aa}ende anv{\"a}nder
  notationen ``$dS$'' f{\"o}r ytelement.}
$$
  \eqalign{
    {{d\Phi_{\rm M}}\over{dt}}
       &={{d}\over{dt}}\iint_{S\land{\bf B}\ne{\bf 0}}{\bf B}_0\cdot d{\bf S}
       ={{d}\over{dt}}\iint_{S\land{\bf B}\ne{\bf 0}} (B_0{\bf e}_y)\cdot
         (\underbrace{-{\bf e}_y dS}_{=d{\bf S}})
       =-B_0{{d}\over{dt}}\big((h-v_0 t)L\big)\cr
      &=\underbrace{v_0 B_0 L}_{=-{\cal E}}\cr
  }
$$
Ur detta h{\"o}gst f{\"o}renklade resonemang kan vi dra slutsatsen att den i
slingan $\Gamma$ genererade elektromotoriska ``kraften'' ges som
f{\"o}r{\"a}ndringen i tid av det av slingan inneslutna magnetiska fl{\"o}det,
med omv{\"a}nt tecken, som
$$
  {\cal E}=-{{d\Phi_{\rm M}}\over{dt}}.
$$
Detta samband sammanfattar {\it Faradays induktionslag}\numberedfootnote{Ibland
  kallas denna kort och gott f{\"o}r ``Faradays lag'' eller ``magnetiska
  fl{\"o}deslagen''; vi v{\"a}ljer h{\"a}r dock att beh{\aa}lla den
  formella beteckningen f{\"o}r att inte blanda ihop denna induktionslag
  med den lag p{\aa} differentialform som vi inom kort kommer att
  h{\"a}rleda fr{\aa}n denna.},
vilken vi erinrar oss h{\"a}r har h{\"a}rletts fram f{\"o}r en specifik
geometri och med ett konstant magnetf{\"a}lt ${\bf B}_0$, i vilken det
magnetiska fl{\"o}det endast p{\aa}verkas genom att den slutna slingan
traverserar magnetf{\"a}ltet s{\aa} att fl{\"o}det successivt minskar.

Om vi hade haft det homogena magnetf{\"a}ltet t{\"a}ckande hela den r{\"o}rliga
slingans yta, s{\aa} hade segmentet \encircle{3} gett ett till beloppet lika
stort men motriktat bidrag som segmentet \encircle{1}, och den elektromotoriska
''netto-kraften'' hade blivit noll.
\vfill\eject

\subsection{Observation~I - Avsaknad av permittivitet och permebilitet}
Notera att Faradays induktionslag enligt ovan
``h{\"a}rleddes''\numberedfootnote{I den m{\aa}n vi {\"o}verhuvud taget kan
  tala om en ``h{\"a}rledning'' med anv{\"a}ndande av ett specialfall!}
enbart under antagandet om Lorentz-kraften p{\aa} laddade partiklar i
r{\"o}relse.
Vi har i h{\"a}rledningen inte anv{\"a}nt vare sig Coulombs lag eller n{\aa}gon
av de fr{\aa}n den lagen h{\"a}rledda f{\"o}ljdsatserna, och d{\"a}rmed kan vi
konstatera att den elektriska permittiviteten $\varepsilon_0$ lyser med sin
fr{\aa}nvaro.
Vi har ej heller anv{\"a}nt Biot--Savarts lag eller n{\aa}got av de fr{\aa}n
denna h{\"a}rledda teoremen l{\"a}ngre ner i det ``elektromagnetiska
sl{\"a}kttr{\"a}det'', s{\aa} {\"a}ven den magnetiska permeabiliteten $\mu_0$
lyser med sin fr{\aa}nvaro.
Det enda som vi i h{\"a}rledningen av Faradays lag har tagit som ett axiom
{\"a}r just Lorentz kraftlag, vilket ger oss vid hand att vi {\"a}r n{\aa}got
nytt p{\aa} sp{\aa}ret.

\subsection{Observation~II - Fysisk vs virtuell slinga}
\sidx{Virtuell slinga}
Utifr{\aa}n diskussionen ovan kring alstrande av en elektromotorisk ``kraft''
och integralen av en normaliserad kraft ${\bf F}$ verkande p{\aa} en laddning
$q$, s{\aa} har vi tyv{\"a}rr letts in i tankarna att detta med Faradays
induktionslag i grund och botten bara skulle handla om fysiska str{\"o}mslingor
och hur vi kan generera str{\"o}m induktivt i klassiska generatorer och
liknande. H{\"a}r skall vi dock komma ih{\aa}g att den elektromotoriska
``kraften'' ju {\"a}r definierad i termer av en sluten linjeintegral som
arbetar med {\it f{\"a}lten i sig} som integrand. I grund och botten {\"a}r
ju den elektromotoriska kraften ingenting annat {\"a}n ett {\it rent
matematiskt m{\aa}tt p{\aa} rotationen av det elektriska f{\"a}lt som
{\"a}r inneslutet}, som vi strax skall se.
Med andra ord finns det ingenting som hindrar oss att evaluera detta m{\aa}tt
som en integral {\"o}ver en sluten {\it virtuell slinga} $\Gamma$ som
innesluter n{\aa}got tidsvarierande magnetiskt (eller f{\"o}r den delen
elektriskt) f{\"a}lt.

Samtidigt {\"a}r det magnetiska fl{\"o}det $\Phi_{\rm M}$ {\"a}ven det bara en
matematisk konstruktion i form av en integral {\"o}ver en yta $S$ innesluten
av slingan $\Gamma$, virtuell eller fysisk, och {\"a}ven detta m{\aa}tt kan
sj{\"a}lvfallet integreras {\"o}ver en yta som inte n{\"o}dv{\"a}ndigtvis
m{\aa}ste ringas in av just en {\it fysisk} str{\"o}mslinga.

Med andra ord visar detta p{\aa} att Faradays induktionslag handlar om n{\aa}got
djupare {\"a}n bara generatorer, med b{\"a}ring p{\aa} en tidsberoende
{\it koppling mellan elektriska och magnetiska f{\"a}lt i sig}.
P{\aa} s{\"a}tt och vis kan vi s{\"a}ga att det {\"a}r {\it precis vid denna
punkt i kursen som vi b{\"o}rjar formulera en sammanh{\aa}llen teori f{\"o}r
elektrodynamik och i f{\"o}rl{\"a}ngningen en modell f{\"o}r elektromagnetisk
v{\aa}gutbredning}.\numberedfootnote{Vilket r{\aa}kar vara just den
  f{\"o}rsta tentamensuppgiften, som under F{\"o}rel{\"a}sning~1
  l{\"a}mnades ut p{\aa}
  {\tt https://github.com/hp35/elmagii/blob/main/lect-01/extras/examprob.pdf}}

\subsection{Observation~III kring Faradays induktionslag
  - Negativt tecken och Lenz lag}
\sidx{Lenz lag}
I uttrycket f{\"o}r Faradays induktionslag ser vi {\"a}ven att tidsderivatan av
det magnetiska fl{\"o}det $\Phi_{\rm M}$ f{\"o}rekommer med ett negativt tecken i
h{\"o}gerledet. Detta negativa tecken {\"a}r i grunden en signatur av att den
genererade elektromotoriska ``kraften'', eller om vi s{\aa} vill den i en fysisk
slinga genererade str{\"o}mmen, alltid kommer att genereras {\it s{\aa} att den
har en riktning som motverkar k{\"a}llan till induktionen}.
Denna princip, som brukar betecknas med {\it Lenz lag}, bist{\aa}r med ett
synnerligen kraftfullt verktyg n{\"a}r det kommer till punkten att vi skall
tolka ett resultat eller g{\"o}ra en {\it sanity check} p{\aa} att ett resultat
verkar rimligt.

Specifikt i och med h{\"a}rledningen av Faradays induktionslag enligt ovan,
s{\aa} har vi just r{\aa}kat ta fram formen p{\aa} Lenz lag f{\"o}r
elektromotorisk ``kraft'' genererad i det klassiska problemet f{\"o}r en
ledande stav som r{\"o}r sig ortogonalt mot sin egen axel och ortogonalt mot
ett omgivande konstant magnetf{\"a}lt, som\numberedfootnote{Griffiths
  problem~7.7, sid.~310.}
$$
  {\cal E}=-v_0 B_0 L.
$$
\vfill\eject

\section{Lenz lag som rimlighetsbed{\"o}mning av l{\"o}sningar till
  induktionsproblem}
\sidx{Lenz lag}
{\it Lenz lag s{\"a}ger att en inducerad str{\"o}m har en riktning som
motverkar orsaken till att den uppkom.} Detta inneb{\"a}r att om
magnetf{\"a}ltet genom en ledande slinga (det magnetiska fl{\"o}det
$\Phi_{\rm M}$) {\"o}kar, s{\aa} kommer den i slingan inducerade str{\"o}mmen
att ha en riktning som skapar ett magnetf{\"a}lt som motverkar {\"o}kningen.
Motsatt g{\"a}ller att om det magnetiska fl{\"o}det minskar, s{\aa} kommer den
inducerade str{\"o}mmen att skapa ett magnetf{\"a}lt som motverkar minskningen.

Denna princip {\"a}r oerh{\"o}rt kraftfull n{\"a}r det g{\"a}ller att f{\aa}
en k{\"a}nsla f{\"o}r induktionsproblem i allm{\"a}nhet, och kan med f{\"o}rdel
anv{\"a}ndas som en rimlighetsbed{\"o}mning av l{\"o}sningar som tagits fram i
induktionsproblem. Med detta sagt, l{\aa}t oss applicera Lenz lag p{\aa} ett
antal tankeexperiment enligt f{\"o}ljande.
\epsfig{../lect-05/figs/lenz.1}\noindent
\vfill\eject

\section{Generell h{\"a}rledning av Faradays induktionslag}
\sidx{Faradays induktionslag}[Generell h{\"a}rledning]
\sidx{Magnetiskt fl{\"o}de $\Phi_{\rm M}$}[Tidsberoende]
Utifr{\aa}n en illustrativ men mycket f{\"o}renklad geometri har vi s{\aa}
l{\aa}ngt visat p{\aa} en {\it sannolik} form av Faradays induktionslag, men
det vore n{\"a}stintill skamligt av oss att bara acceptera denna som ett faktum
utan att g{\aa} in p{\aa} hur vi generellt och formellt kan h{\"a}rleda den,
om {\"a}n att det nu blir en aning st{\"o}kigt. Vi kommer nu att introducera
en generell trajektoria $\Gamma(t)$, l{\"a}ngs med vilken varje linjeelement
till{\aa}ts ha en godtycklig hastighet ${\bf v}({\bf x},t)$ som har ett
godtyckligt beroende i tid och rum. Denna generalitet g{\"o}r att {\"a}ven den
omslutna arean och riktningen av ytnormalen till{\aa}ts att variera fritt.
Vi till{\aa}ter vidare att det av trajektorian inneslutna magnetf{\"a}ltet
${\bf B}({\bf x},t)$ till{\aa}ts att variera fritt i tid och
rum.\numberedfootnote{\idx{Leibniz integralregel} n{\"a}mns tyv{\"a}rr inte
  i Griffiths, oturligt nog eftersom det {\"a}r ett statement som i m{\aa}ngt
  och mycket skulle g{\"o}ra att vi bara skulle kunna referera till resultatet
  h{\"a}r.\goodbreak
  \noindent{\tt https://en.wikipedia.org/wiki/Leibniz\_integral\_rule}}
\epsfig{../lect-05/figs/faradayrib.1}\noindent
L{\aa}t oss nu se vad f{\"o}r{\"a}ndringen $d\Phi_{\rm M}$ i magnetiskt fl{\"o}de
fr{\aa}n tiden $t$ till $t+dt$ kan t{\"a}nkas vara f{\"o}r en s{\aa} generell
konstruktion.
D{\aa} trajektorian $\Gamma(t)$ r{\"o}r sig, fr{\aa}n tiden $t$ till tiden
$t+dt$ ``strax d{\"a}refter'', kommer f{\"o}r{\"a}ndringen i magnetiskt
fl{\"o}de att ges som alla delbidrag i den ``remsa'' som i rummet beskrivs
av zonen mellan $\Gamma(t)$ och $\Gamma(t+dt)$,
$$
  d\Phi_{\rm M}=\Phi_{\rm M}(t+dt)-\Phi_{\rm M}(t)
    =\iint_{\hbox{remsa}}{\bf B}({\bf x},t)\cdot d{\bf S},
$$
d{\"a}r areaelementen $d{\bf S}$ l{\"a}ngs med remsan ges som vektorprodukten
mellan den str{\"a}cka som elementet vid punkten ${\bf x}$ p{\aa} trajektorian
tillryggal{\"a}gger under tiden $dt$ samt linjeelementet $d{\bf l}$ l{\"a}ngs
med trajektorian vid tiden $t$,
$$
  d{\bf S}=({\bf v}({\bf x},t)dt)\times(d{\bf l})
    =({\bf v}({\bf x},t)\times d{\bf l})\,dt
$$
Utifr{\aa}n detta, med areaelementen uttryckta som en produkt mellan
linjeelement $d{\bf l}$ och infinitesimala tidssteg $dt$, ser vi direkt att
vi kan ers{\"a}tta areaintegralen {\"o}ver v{\aa}r remsa med en linjeintegral,
som
$$
  \eqalign{
    d\Phi_{\rm M}
      &=\iint_{\hbox{remsa}}{\bf B}({\bf x},t)\cdot d{\bf S}\cr
      &=\oint_{\Gamma(t)}{\bf B}({\bf x},t)
             \cdot\underbrace{
               ({\bf v}({\bf x},t)\times d{\bf l})\,dt
             }_{d{\bf S}\hbox{ l{\"a}ngs }\Gamma(t)}\cr
      &=dt\oint_{\Gamma(t)}{\bf B}({\bf x},t)
             \cdot({\bf v}({\bf x},t)\times d{\bf l}).\cr
  }
$$
Med andra ord kan vi formulera f{\"o}r{\"a}ndringen av det magnetiska fl{\"o}det
\sidx{Magnetiskt fl{\"o}de $\Phi_{\rm M}$} genom trajektorian $\Gamma(t)$ som den
s{\aa} gott som f{\"o}rv{\"a}ntade tidsderivatan
$$
  \eqalign{
    {{d\Phi_{\rm M}}\over{dt}}
      &=\oint_{\Gamma(t)}{\bf B}({\bf x},t)\cdot
           ({\bf v}({\bf x},t)\times d{\bf l}).
  }
$$
L{\aa}t oss med denna deriviata formulerad g{\aa} vidare med att analysera i
mer detalj hur en laddning som f{\"a}rdas l{\"a}ngs den i sig {\it r{\"o}rliga
trajektorian} $\Gamma(t)$ kommer att f{\"o}rflyttas.
\epsfig{../lect-05/figs/faradayelement.1}\noindent
Om vi t{\"a}nker oss att en laddning har hastigheten ${\bf u}({\bf x},t)$
l{\"a}ngs med trajektorian $\Gamma(t)$, s{\aa} {\"a}r sj{\"a}lvfallet denna
hastighet parallell med linjeelementet $d{\bf l}$ i samma punkt.
Trajektorian $\Gamma(t)$ har vid samma tidpunkt en och punkt i rummet
hastigheten ${\bf v}({\bf x},t)$, s{\aa} den t{\"a}nkta laddningens
resulterande hastighet ${\bf w}({\bf x},t)$ ges som summan av dessa tv{\aa}
komposanter som
$$
  {\bf w}({\bf x},t)={\bf u}({\bf x},t)+{\bf v}({\bf x},t).
  \qquad\hbox{(Laddningens hastighet)}
$$
\vfill\eject
Notera att eftersom ${\bf u}({\bf x},t)$ {\"a}r parallell med linjeelementet
$d{\bf l}$, s{\aa} kommer samma linjeelements kryssprodukt med
${\bf v}({\bf x},t)$ att ge en produkt som {\"a}r ortogonal mot just
${\bf u}({\bf x},t)$; med andra ord s{\aa} kan vi precis lika g{\"a}rna
ers{\"a}tta hastigheten i kryssprodukten med den totala resulterande
hastigheten ${\bf w}({\bf x},t)$ f{\"o}r en hypotetisk laddnings r{\"o}relse,
d{\aa}
$$
  \eqalign{
    {{d\Phi_{\rm M}}\over{dt}}
      &=\oint_{\Gamma(t)}{\bf B}({\bf x},t)\cdot
           \underbrace{
             \big(
               (
                 \underbrace{
                   {\bf u}({\bf x},t)
                 }_{\parallel d{\bf l}\to0}+{\bf v}({\bf x},t)
               )\times d{\bf l}
             \big)
           }_{
              {\bf w}\times d{\bf l}={\bf v}\times d{\bf l}
           }\cr
      &=\oint_{\Gamma(t)}{\bf B}({\bf x},t)\cdot
             \big({\bf w}({\bf x},t)\times d{\bf l}\big)\cr
      &=\big\{\hbox{ Griffiths {\it Triple Product (1)} }\big\}\cr
      &=\big\{\hbox{ ${\bf a}\cdot({\bf b}\times{\bf c})
                       ={\bf c}\cdot({\bf a}\times{\bf b})
                       =-({\bf b}\times{\bf a})\cdot{\bf c}$ }\big\}\cr
      &=-\oint_{\Gamma(t)}\underbrace{
        \big({\bf w}({\bf x},t)\times{\bf B}({\bf x},t)\big)
      }_{\equiv{\bf F}_{\rm M}/q}
     \cdot d{\bf l}.\cr
  }
$$
Vi ser nu po{\"a}ngen med att ers{\"a}tta den hypotetiska laddningens hastighet
${\bf v}({\bf x},t)$ l{\"a}ngs trajektorian med den totala hastigheten
${\bf w}({\bf x},t)$ i och med att kryssprodukten med det magnetiska f{\"a}ltet
exakt {\"a}r det magnetiska bidraget till \idx{Lorentz-kraften},
$$
  {\bf F}_{\rm M}=q\big({\bf w}({\bf x},t)\times{\bf B}({\bf x},t)\big),
$$
vilket i sin tur g{\"o}r att vi direkt kan tolka den sista integralen som den
elektromotoriska ``kraft'' som alstras runt trajektorian $\Gamma(t)$,
\sidx{Elektromotorisk ``kraft'' ${\cal E}$}\sidx{Str{\"o}mslinga}[Sluten]
$$
    {{d\Phi_{\rm M}}\over{dt}}
      =-\oint_{\Gamma(t)}\bigg({{{\bf F}_{\rm M}}\over{q}}\bigg)\cdot d{\bf l}
      =-{\cal E}
$$
L{\aa}t oss sammanfatta detta generella resultat med att vi till slut p{\aa}
ett strikt s{\"a}tt h{\"a}rlett Faradays induktionslag f{\"o}r en godtycklig
geometri som\sidx{Faradays induktionslag}[Generell h{\"a}rledning]
$$
  {\cal E}=-{{d\Phi_{\rm M}}\over{dt}}.
$$
Vi erinrar oss {\aa}terigen att denna h{\"a}rletts {\it endast utifr{\aa}n
Lorentz-kraften \sidx{Lorentz-kraften} agerande p{\aa} en hypotetisk laddning},
och att denna lag d{\"a}rmed dels {\"a}r i avsaknad av den elektriska
permittiviteten $\varepsilon_0$, s{\aa}v{\"a}l som den magnetiska
permeabiliteten $\mu_0$.
\sidx{Elektrisk permittivitet}[Vakuumpermittivitet $\varepsilon_0$]
\sidx{Magnetisk permeabilitet}[Vakuumpermeabilitet $\mu_0$]

Vi tar ocks{\aa} tillf{\"a}llet i akt att erinra oss att {\"a}ven om vi h{\"a}r
f{\"o}r enkelhets skull diskuterat Lorentzkraften s{\aa} som den skulle agerat
p{\aa} en fysisk laddning $q$ i r{\"o}relse, s{\aa} {\"a}r resultatet i form av
Faradays lag en {\it relation mellan f{\"a}lt}, specifikt visande att i
n{\"a}rvaro av tidsberoende magnetiska f{\"a}lt s{\aa} {\"a}r rotationen
f{\"o}r det elektriska f{\"a}ltet inte l{\"a}ngre noll.
Detta {\it dynamiska resultat} st{\aa}r i kontrast till det statiska fallet
vilket vi analyserade i F{\"o}rel{\"a}sning~2, d{\"a}r vi konstaterade att
$\nabla\times{\bf E}={\bf 0}$ alltid g{\"a}ller f{\"o}r {\it statiska} f{\"a}lt.
\sidx{Elektrostatiskt f{\"a}lt}[Rotation f{\"o}r]

\section{Faradays induktionslag f{\"o}r station{\"a}ra slingor}
\sidx{Faradays induktionslag}[Station{\"a}r slinga]
\sidx{Str{\"o}mslinga}[Sluten]
\sidx{Str{\"o}mslinga}[Station{\"a}r]
I fall d{\aa} en str{\"o}mslinga $\Gamma$ {\"a}r {\it station{\"a}r i rummet
och har en form som inte beror av tiden}, s{\aa} kan vi direkt uttrycka den
genererade elektromotoriska ``kraften'' utifr{\aa}n definitionen av det
magnetiska fl{\"o}det som\numberedfootnote{Vi betecknar h{\"a}r generellt
  en {\it yttre} verkande tidsderivata som $d/dt$, f{\"o}r p{\aa} s{\aa}
  s{\"a}tt inkludera m{\"o}jligheten att {\"a}ven sj{\"a}lva
  {\it integrationsdom{\"a}nen {\"a}r tidsberoende}.
  N{\"a}r det {\"a}r uppenbart att det endast {\"a}r magnetf{\"a}ltet i
  sig som tidsderiveras kan vi ist{\"a}llet med f{\"o}rdel anv{\"a}nda
  partialderivatan $\partial/\partial t$.}
$$
  {\cal E}=-{{d}\over{dt}}\iint_{S}{\bf B}({\bf x},t)\cdot d{\bf S}
    =-\iint_{S}{{\partial{\bf B}({\bf x},t)}\over{\partial t}}\cdot d{\bf S}
$$
Som exempel p{\aa} applikationer av Faradays induktionslag f{\"o}r
station{\"a}ra slingor kan n{\"a}mnas {\it statorn}, som {\"a}r den
station{\"a}ra, icke r{\"o}rliga komponenten i en generator, som typiskt
inneh{\aa}ller spolar av koppartr{\aa}d (lindningar) som omvandlar det
varierande magnetf{\"a}ltet fr{\aa}n magneter p{\aa} en {\it rotor} till
elektrisk sp{\"a}nning via den inducerade elektromotoriska ``kraften''.
Ett annat exempel {\"a}r pickupen p{\aa} en elgitarr eller elbas, d{\"a}r
station{\"a}ra magneter omgivna av en spole via den vibrerande metalliska
str{\"a}ngen skapar ett varierande magnetiskt f{\"a}lt, och p{\aa} liknande
s{\"a}tt inducerar en signal i form av en sp{\"a}nning.

\section{Spolar och utv{\"a}xling p{\aa} det magnetiska fl{\"o}det}
\sidx{Spole}\sidx{J{\"a}rnk{\"a}rna}
F{\"o}r en spole med $N$ varv, som vart och ett kan anses som identiskt i
sitt t{\"a}ckning av det magnetiska f{\"a}ltet och d{\"a}rmed vart och ett
upplever ett identiskt magnetiskt fl{\"o}de $\Phi_{\rm M}$ s{\aa} blir den
resulterande inducerade elektromotoriska kraften utv{\"a}xlad i samma grad,
som
$$
  {\cal E}=-N{{d\Phi_{\rm M}}\over{dt}}.
$$
F{\"o}r ett tillr{\"a}ckligt h{\"o}gt antal varv $N$ kan mycket sm{\aa}
fluktuationer i det magnetiska f{\"a}ltet detekteras, speciellt om vi
f{\"o}rst{\"a}rker B-f{\"a}ltet genom att introducera en ferrit
(j{\"a}rnk{\"a}rna) inuti spolen. Detta {\"a}r exempelvis principen som typiskt
anv{\"a}nds f{\"o}r att passivt generera signalen fr{\aa}n en elgitarrs eller
elbas pickup. Som exempel kan n{\"a}mnas att en klassisk PAF-style humbucker
p{\aa} en Gibson Les Paul typiskt har cirka $N=5\,000$ varv per spole.

\section{Faradays lag p{\aa} differentialform}
\sidx{Faradays lag}[Differentialform]
Utifr{\aa}n den generella definitionen av den elektromotoriska ``kraften''
kan vi applicera Stokes teorem\numberedfootnote{Se exempelvis innerp{\"a}rmen
  p{\aa} Griffiths, {\it Curl Theorem},\sidx{Stokes teorem}
  $$
    \iint_S\nabla\times{\bf A}\,d{\bf S}=\oint{\bf A}\cdot d{\bf l}.
  $$} p{\aa} den ing{\aa}ende linjeintegralen, som
$$
  {\cal E}
    \equiv\oint_{\Gamma}{\bf E}({\bf r},t)\cdot d{\bf l}
    =\underline{\underline{
        \iint_{S}(\nabla\times{\bf E}({\bf r},t))\cdot d{\bf S}
     }}
    =-{{d\Phi_{\rm M}({\bf x},t)}\over{dt}}
    =\underline{\underline{
        -\iint_S{{\partial{\bf B}({\bf x},t)}\over{\partial t}}\cdot d{\bf S}
     }}.
$$
Eftersom ytintegralen sker {\"o}ver en yta innesluten av en godtycklig
trajektoria $\Gamma$, s{\aa} inneb{\"a}r det att vi f{\"o}r integranderna
erh{\aa}ller det vektoriella sambandet\numberedfootnote{{\"A}ven h{\"a}r
  anv{\"a}nder vi partialderivatan $\partial/\partial t$ f{\"o}r att
  explicit visa att vi h{\aa}ller oss till en beskriv\-ning av en
  motsvarande station{\"a}r slinga $\Gamma$ i rummet.}
$$
  \nabla\times{\bf E}({\bf r},t)
    =-{{\partial{\bf B}({\bf x},t)}\over{\partial t}},
$$
vilket vi betecknar som {\it Faradays lag p{\aa} differentialform}.
Notera att liksom f{\"o}r Faradays induktionslag, s{\aa} {\"a}r denna ekvation
helt oberoende av den elektriska permittiviteten $\varepsilon_0$ eller
magnetiska permeabiliteten $\mu_0$, vilket vi rekapitulerar {\"a}r en effekt
av att denna lag h{\"a}rletts oberoende av Coulombs eller Biot--Savarts lagar
eller n{\aa}gon av deras derivat l{\"a}ngre ner i det elektromagnetiska
sl{\"a}kttr{\"a}det.
\sidx{Coulombs kraftlag}\sidx{Biot--Savarts lag}
\vfill\eject

\section{Faradays lag p{\aa} integralform}
\sidx{Faradays lag}[Integralform]
Vi har i princip redan fastst{\"a}llt Faradays lag p{\aa} integralform i och
med att vi tog fram att inte\-grand\-erna i ytintegralerna m{\aa}ste vara
identiska f{\"o}r en generell trajektoria $\Gamma$, men l{\aa}t oss
{\"a}nd{\aa} f{\"o}r sakens skull ta fram integralformen av Faradays lag
fr{\aa}n differentialformen.
Om vi integrerar differentialformen av Faradays lag {\"o}ver en yta $S$
innesluten av en sluten trajektoria $\Gamma$ och direkt till{\"a}mpar Stokes
teorem, s{\aa} har vi att\sidx{Stokes teorem}
$$
  \iint_S(\nabla\times{\bf E}({\bf r},t))\cdot d{\bf S}
  =\underline{\underline{
     \oint_{\Gamma}{\bf E}({\bf r},t)\cdot d{\bf l}
   }}
  =-\iint_S{{\partial{\bf B}({\bf x},t)}\over{\partial t}}\cdot d{\bf S}
  =\underline{\underline{
     -{{d}\over{dt}}\iint_S{\bf B}({\bf x},t)\cdot d{\bf S}
   }},
$$
vilket vi kan sammanfatta med {\it Faradays lag p{\aa} integralform} som
$$
  \underbrace{
    \oint_{\Gamma}{\bf E}({\bf r},t)\cdot d{\bf l}
  }_{\displaystyle ={\cal E}}
    =\underbrace{
       -{{d}\over{dt}}\iint_S{\bf B}({\bf x},t)\cdot d{\bf S}
     }_{\displaystyle =-{{d\Phi_{\rm M}}\over{dt}}}.
$$
Vi kan i integralformen av Faradays lag direkt identifiera v{\"a}nsterledet som
den elektromotoriska ``kraft'' ${\cal E}$ som alstras i en station{\"a}r slinga,
och h{\"o}gerledet som den negativa tidsderivatan av det magnetiska fl{\"o}det,
$-d\Phi_{\rm M}/dt$; med andra ord {\"a}r denna form i grunden bara
{\it en manifestering av Faradays induktionslag}.

\section{Tre principiella specialfall f{\"o}r induktion}
I Faradays ursprungliga experiment 1831, som vi kan se som tidpunkten d{\"a}r
elektromagnetism och induktion f{\"o}r f{\"o}rsta g{\aa}ngen demonstrerades,
s{\aa} presenterades i huvudsak tre fundamentala
fall.\numberedfootnote{Faradays tre experiment, se Griffiths sektion~7.2,
  sid.~312.}${^{,}}$%
  \numberedfootnote{Fr{\aa}ga till l{\"a}saren: Om man till{\"a}mpar Lenz
    lag p{\aa} dessa tre fall, s{\aa} som de h{\"a}r {\"a}r illustrerade,
    st{\"a}mmer riktningarna p{\aa} de utritade str{\"o}mmarna?}

\epsfig{../lect-05/figs/inductioncases.1}\noindent
Vi kan principiellt s{\"a}rskilja tre grundl{\"a}ggande fall f{\"o}r
elektromagnetisk induktion med Faradays lag:
\medskip
\item{1.}{En r{\"o}rlig slinga $\Gamma$ som traverserar ett inhomogent men
  i {\"o}vrigt konstant magnetf{\"a}lt ${\bf B}_0$, d{\"a}r det magnetiska
  fl{\"o}det $\Phi_{\rm M}$ genom slingan f{\"o}r{\"a}ndras genom slingans
  r{\"o}relse.\sidx{Str{\"o}mslinga}[R{\"o}rlig]}
\item{2.}{En slinga $\Gamma$ som {\"a}r fix i rummet, d{\"a}r ett inhomogent
  men i {\"o}vrigt konstant magnetf{\"a}lt ${\bf B}_0$ r{\"o}r sig {\"o}ver
  slingan, d{\"a}r det magnetiska fl{\"o}det $\Phi_{\rm M}$ f{\"o}r{\"a}ndras
  genom att magnetf{\"a}ltet varierar genom magnetf{\"a}ltets r{\"o}relse.
  \sidx{Str{\"o}mslinga}[Station{\"a}r]}
\item{3.}{En slinga $\Gamma$ som {\"a}r fix i rummet med ett tidsvarierande
  (dynamiskt) magnetf{\"a}lt ${\bf B}(t)$, d{\"a}r det magnetiska fl{\"o}det
  $\Phi_{\rm M}$ f{\"o}r{\"a}ndras genom magnetf{\"a}ltets variation i tiden.
  \sidx{Str{\"o}mslinga}[Varierande magnetiskt fl{\"o}de genom]}
\vfill\eject

\section{{\"O}msesidig induktans}
\sidx{{\"O}msesidig induktans}
Vi har under F{\"o}rel{\"a}sning~4 visat hur \idx{Biot--Savarts lag} beskriver
hur magnetf{\"a}lt kan alstras genom str{\"o}m som traverserar en slinga,
s{\"a}g $\Gamma'$, och vi har nu {\"a}ven visat hur Faradays lag kopplar ett
tidsvarierande magnetiskt f{\"a}lt som genom induktion kan generera str{\"o}m
i en annan slinga, s{\"a}g $\Gamma$.
Uppenbarligen kan vi med andra ord genom att skicka str{\"o}m genom en
prim{\"a}r slinga $\Gamma'$ inducera en str{\"o}m i en sekund{\"a}r slinga
$\Gamma$ utan att dessa slingor har fysisk kontakt med varandra, och beroende
p{\aa} riktning och avst{\aa}nd f{\"o}r magnetf{\"a}ltet som alstras fr{\aa}n
den prim{\"a}ra slingan $\Gamma'$ kommer vi att ha en mer eller mindre effektiv
{\"o}verf{\"o}ring av effekt {\"o}ver till den sekund{\"a}ra slingan $\Gamma$.
\epsfig{../lect-05/figs/inductance.1}\noindent
Fr{\aa}gan blir d{\aa}:\numberedfootnote{Vi f{\"o}ljer h{\"a}r i huvudsak
  Griffiths sid.~321--323.}
\quote{{\it Kan vi p{\aa} n{\aa}got s{\"a}tt sy ihop dessa teorier f{\"o}r
  att extrahera hur stark den induktiva {\"o}msesidiga kopplingen, eller
  den s{\aa} kallade {\it {\"o}msesidiga induktansen}, mellan slingorna
  {\"a}r?}}
\noindent
L{\aa}t oss f{\"o}rst konstatera att det magnetiska f{\"a}ltet som genereras
fr{\aa}n den prim{\"a}ra slingan\numberedfootnote{Eftersom vi i detta problem
  kan betrakta den prim{\"a}ra slingan som en {\it k{\"a}lla}, s{\aa}
  kommer vi genomg{\aa}ende att primma relevanta storheter fr{\aa}n
  denna, f{\"o}ljande den konvention vi h{\aa}ller oss till i denna kurs.}
$\Gamma'$ beskrivs av Biot--Savarts lag, som f{\"o}r detta fall och f{\"o}r
alla observationspunkter ${\bf x}$ formuleras som
linjeintegralen\numberedfootnote{Se F{\"o}rel{\"a}sning~4 eller
  Griffiths Ekv.~(5.34), sid.~224.}
{\"o}ver den prim{\"a}ra loopen $\Gamma'$,
$$
  {\bf B}({\bf x},t)={{\mu_0}\over{4\pi}}\int_{\Gamma'}
    {{{\bf I}'({\bf x}',t)\times({\bf x}-{\bf x}')}
      \over{|{\bf x}-{\bf x}'|^3}}\,dl'
    ={{\mu_0 I'(t)}\over{4\pi}}\int_{\Gamma'}
      {{d{\bf l}'\times({\bf x}-{\bf x}')}\over{|{\bf x}-{\bf x}'|^3}}.
$$
Eftersom detta direkt bist{\aa}r oss med det magnetiska f{\"a}ltet
${\bf B}({\bf x},t)$ som i s{\aa} fall t{\"a}cks av den sekund{\"a}ra loopen,
som vi i detta fall r{\"a}knar som uppbyggd av {\it observationspunkter}, och
eftersom vi vet att ett varierande f{\"a}lt genom sekund{\"a}rloopen kommer
att ge upphov till en elektromotorisk ``kraft'' {\"o}ver denna, l{\aa}t oss
d{\"a}rf{\"o}r formulera det magnetiska fl{\"o}det $\Phi_{\rm M}$ genom just
sekund{\"a}rloopen som ytintegralen av det magnetiska f{\"a}ltet (den
magnetiska fl{\"o}dest{\"a}theten!) {\"o}ver den av sekund{\"a}rloopen
inneslutna ytan,\sidx{Magnetiskt fl{\"o}de $\Phi_{\rm M}$}
$$
  \eqalign{
    \Phi_{\rm M}&=\iint_{S}{\bf B}({\bf x},t)\cdot d{\bf S}
        \qquad\hbox{(Genom sekund{\"a}rloopen $\Gamma$)}\cr
      &=\iint_{S}\bigg(
            \underbrace{
              {{\mu_0 I'(t)}\over{4\pi}}\int_{\Gamma'}
                {{d{\bf l}'\times({\bf x}-{\bf x}')}\over{|{\bf x}-{\bf x}'|^3}}
            }_{={\bf B}({\bf x},t)}
        \bigg)\cdot d{\bf S}\cr
      &=I'(t)\bigg[
          {{\mu_0}\over{4\pi}}\iint_{S}\bigg(
            \int_{\Gamma'}
              {{d{\bf l}'\times({\bf x}-{\bf x}')}\over{|{\bf x}-{\bf x}'|^3}}
          \bigg)\cdot d{\bf S}
        \bigg]\cr
      &=M_{\Gamma\Gamma'} I'(t),\cr
  }
$$
d{\"a}r vi definierade den {\it {\"o}msesidiga induktansen} ({\it mutual
inductance}) mellan slingorna $\Gamma$ och $\Gamma'$ som
$$
  M_{\Gamma\Gamma'}=
    {{\mu_0}\over{4\pi}}\iint_{S}\bigg(
      \int_{\Gamma'}
        {{d{\bf l}'\times({\bf x}-{\bf x}')}\over{|{\bf x}-{\bf x}'|^3}}
    \bigg)\cdot d{\bf S}.
$$
Vi kan h{\"a}r i princip n{\"o}ja oss med detta resultat, d{\aa} vi f{\aa}tt
fram ett paketerat uttryck som relaterar str{\"o}mmen $I'(t)$ i
prim{\"a}rslingan $\Gamma'$ till det magnetiska fl{\"o}det $\Phi_{\rm M}$ i
sekund{\"a}rslingan $\Gamma$, fr{\aa}n vilket vi direkt kan erh{\aa}lla den
resulterande elektromotoriska ``kraften''. Det finns dock ett par moment vi
kan g{\aa} igenom f{\"o}r att f{\"o}ra {\"o}ver den {\"o}msesidiga induktansen
p{\aa} en form som vi enklare kan extrahera fysikaliska samband fr{\aa}n.

\section{Neumanns formel f{\"o}r den {\"o}msesidiga induktansen}
\sidx{{\"O}msesidig induktans}
\sidx{Neumanns formel}[{\"O}msesidig induktans]
Till att b{\"o}rja med backar vi bandet ett steg, och konstaterar att om vi
formulerar den magnetiska fl{\"o}det $\Phi_{\rm M}$ genom sekund{\"a}rslingan
i termer av {\it vektorpotentialen} som
$$
  \eqalign{
    \Phi_{\rm M}&=\iint_{S}{\bf B}({\bf x},t)\cdot d{\bf S}
        \qquad\hbox{(Genom sekund{\"a}rloopen $\Gamma$)}\cr
      &=\big\{\hbox{ Vektorpotential
                     ${\bf B}\equiv\nabla\times{\bf A}$ }\big\}\cr
      &=\iint_{S}(\nabla\times{\bf A}({\bf x},t))\cdot d{\bf S}\cr
      &=\big\{\hbox{ Stokes teorem (Griffiths {\it Curl Theorem}) }\big\}\cr
      &=\oint_{\Gamma}{\bf A}({\bf x},t)\cdot d{\bf l}.\cr
  }
$$
Notera nu att vektorpotentialen ${\bf A}({\bf x},t)$ som ing{\aa}r i
integranden h{\"a}rr{\"o}r fr{\aa}n prim{\"a}rslingan $\Gamma'$, {\"a}ven om
linjeintegralen i sig sker {\"o}ver sekund{\"a}rslingan $\Gamma$.
Vi kan nu utnyttja att den explicita l{\"o}sningen\numberedfootnote{Vi erinrar
  oss att vektorpotentialen ${\bf A}$ lyder \idx{Poissons ekvation} (se omslaget
  p{\aa} Griffiths!) och i och med detta har en l{\"o}sning som {\"a}r p{\aa}
  exakt samma form som l{\"o}sningen f{\"o}r den skal{\"a}ra potentialen
  \idx{Skal{\"a}r potential} $\phi$ fr{\aa}n Coulombs generella ekvation;
  se F{\"o}rel{\"a}sning~4, alternativt Griffiths Ekv.~(5.66), sid.~245.}
f{\"o}r vektorpotentialen fr{\aa}n prim{\"a}rslingan lyder
\idx{Vektorpotential}
$$
  {\bf A}({\bf x},t)={{\mu_0}\over{4\pi}}\oint_{\Gamma'}
    {{{\bf I}({\bf x}',t)}\over{|{\bf x}-{\bf x}'|}}\,dl',
$$
s{\aa} det magnetiska fl{\"o}det genom sekund{\"a}rslingan beskrivs av
\sidx{Magnetiskt fl{\"o}de $\Phi_{\rm M}$}
\sidx{Integral}[Linje-]
$$
  \eqalign{
    \Phi_{\rm M}&=\oint_{\Gamma}
      \bigg(
        {{\mu_0}\over{4\pi}}\oint_{\Gamma'}
        {{{\bf I}({\bf x}',t)}\over{|{\bf x}-{\bf x}'|}}\,dl'
      \bigg)\cdot d{\bf l}.\cr
    &=\big\{\hbox{ I linjeintegralen f{\"o}r prim{\"a}rslingan {\"a}r
                   ${\bf I}'({\bf x})\,dl'=I'(t)\,d{\bf l}'$ }\big\}\cr
    &=\oint_{\Gamma}
      \bigg(
        {{\mu_0}\over{4\pi}}\oint_{\Gamma'}I'(t)
        {{d{\bf l}'}\over{|{\bf x}-{\bf x}'|}}
      \bigg)\cdot d{\bf l}.\cr
    &=\big\{\hbox{ Str{\"o}mmen $I'(t)$ samma {\"o}ver hela
                   prim{\"a}rslingan $\Gamma'$ }\big\}\cr
    &=I'(t) {{\mu_0}\over{4\pi}}\oint_{\Gamma}\oint_{\Gamma'}
        {{d{\bf l}'\cdot d{\bf l}}\over{|{\bf x}-{\bf x}'|}}\cr
    &=M_{\Gamma\Gamma'} I'(t),\cr
  }
$$
d{\"a}r vi nu har formulerat den {\"o}msesifiga induktansen i form av
{\it Neumanns formel}\numberedfootnote{Efter Franz Ernst Neumann (1798--1895),
  \sidx{Neumann, Franz Ernst (1798--1895)} en tysk fysiker som inom
  elektromagnetism fr{\"a}mst {\"a}r k{\"a}nd f{\"o}r att ha formulerat
  vektorpotentialen ${\bf A}$; icke att f{\"o}rv{\"a}xla med Alfred E. Neuman,
  som stavar sitt namn med bara ett ``n''.
  {\tt https://en.wikipedia.org/wiki/Alfred\_E.\_Neuman}}
\sidx{Neumanns formel}[{\"O}msesidig induktans]
$$
  M_{\Gamma\Gamma'}={{\mu_0}\over{4\pi}}\oint_{\Gamma}\oint_{\Gamma'}
        {{d{\bf l}'\cdot d{\bf l}}\over{|{\bf x}-{\bf x}'|}}.
$$

\subsection{Ett par observationer kring Neumanns formel}
Utifr{\aa}n den just h{\"a}rledda formen p{\aa} Neumanns formel kan vi
observera f{\"o}ljande:
\medskip
\item{1.}{Neumanns formel f{\"o}r den {\"o}msesidiga induktansen mellan
  tv{\aa} slingor $\Gamma'$ och $\Gamma$ {\"a}r en {\it rent geometrisk
  konstruktion}, skalad med den magnetiska permeabiliteten $\mu_0/4\pi$.
  \sidx{{\"O}msesidig induktans}
  \sidx{Magnetisk permeabilitet}[Vakuumpermeabilitet $\mu_0$]}
\item{2.}{I den {\"o}msesidiga induktansen kan vi fritt byta beteckning
  mellan prim{\"a}r- och sekund{\"a}rslinga, med f{\"o}ljd att
  $M_{\Gamma\Gamma'}\equiv M_{\Gamma'\Gamma}$ d{\aa} det enda som p{\aa}verkas i
  Neumanns formel {\"a}r integrationsordningen,
  $$
    M_{\Gamma\Gamma'}
       ={{\mu_0}\over{4\pi}}\oint_{\Gamma}\oint_{\Gamma'}
          {{d{\bf l}'\cdot d{\bf l}}\over{|{\bf x}-{\bf x}'|}}
       ={{\mu_0}\over{4\pi}}\oint_{\Gamma'}\oint_{\Gamma}
          {{d{\bf l}\cdot d{\bf l}'}\over{|{\bf x}'-{\bf x}|}}
       \equiv M_{\Gamma'\Gamma}.
  $$
  Detta s{\"a}ger oss att det magnetiska fl{\"o}de $\Phi_{\rm M}$ som vi
  erh{\aa}ller i sekund{\"a}rslingan $\Gamma$ d{\aa} vi driver
  prim{\"a}rslingan med en str{\"o}m $I'(t)$ {\"a}r {\it exakt lika stort}
  som det magnetiska fl{\"o}de $\Phi'_{\rm M}$ som vi skulle detektera i
  prim{\"a}rslingan $\Gamma'$ om vi ist{\"a}llet drev sekund{\"a}rslingan
  $\Gamma$ med samma str{\"o}m $I'(t)$. Detta g{\"a}ller oavsett form eller
  inb{\"o}rdes riktning hos slingorna $\Gamma'$ och  $\Gamma$.
  \sidx{Reciprocitet}}
\item{3.}{P{\aa} ett djupare plan visar denna {\"o}msesidighet mellan
  slingorna $\Gamma'$ och $\Gamma$ p{\aa} {\it elektromagnetisk
  reciprocitet}.\numberedfootnote{Tyv{\"a}rr g{\aa}r inte Griffiths
    igenom denna synnerligen intressanta aspekt av elektromagnetism;
    f{\"o}r en djupare behandling av detta {\"a}mne, se exempelvis
    J.~D.~Jacksons standardverk {\it Classical Electrodynamics}.
    \sidx{Jackson, John David (1925--2016)}[{{\it Classical Electrodynamics}}]}}
\vfill\eject

\section{Kontinuitetsekvationen - Teaser inf{\"o}r Maxwell's ekvationer}
Vi har nu formellt h{\"a}rlett Faradays lag, och vi kan nu fr{\aa}ga oss vad
som egentligen kvarst{\aa}r innan vi har en komplett elektrodynamisk
beskrivning av de elektriska och magnetiska f{\"a}lten. Sanningen {\"a}r att
det fortfarande finns en hel del att g{\"o}ra vad g{\"a}ller v{\"a}xelverkan
mellan f{\"a}lt och materia (de s{\aa} kallade {\it konstitutiva relationerna})
som vi {\"a}nnu inte ens nosat p{\aa} d{\aa} vi helt arbetat i
vakuumdom{\"a}nen.\numberedfootnote{F{\"o}rvisso kan vi fr{\aa}ga oss om
  vakuum i n{\"a}rvaro av str{\"o}mmar av elektroner eller andra laddade
  partiklar verkligen {\"a}r att betrakta som just {\it vakuum} i ordets
  egentliga bem{\"a}rkelse; h{\"a}r ser vi dock dessa t{\"a}theter av
  laddade partiklar som s{\aa} pass l{\aa}g att vi fortfarande helt kan
  f{\"o}rsumma dem i j{\"a}mf{\"o}relse med n{\"a}r vi inom kort kommer
  att g{\aa} in p{\aa} hur ett medium polariseras i av elektriska och
  magnetiska f{\"a}lt.}
Vi har i F{\"o}rel{\"a}sning~4 tagit fram Amp\`eres lag inom magnetostatiken,
$$
  \nabla\times{\bf B}=\mu_0{\bf J}.\qquad\hbox{(Statiskt!)}
$$
Vi har samtidigt i F{\"o}rel{\"a}sning~4 h{\"a}rlett ``lagen om att laddning
inte kan f{\"o}rsvinna'', som uttrycker sambandet mellan divergensen f{\"o}r
str{\"o}mt{\"a}thet ${\bf J}$ och tidsderivatan av laddningst{\"a}theten
$\rho$ som\numberedfootnote{{\it Continuity Equation}; se Griffiths
  Ekv.~(5.29), sid.~222 samt Griffiths Ekv.~(8.4), sid.~356.}
$$
  \nabla\cdot{\bf J}+{{\partial\rho}\over{\partial t}}=0.
$$
Om vi substituerar f{\"o}r str{\"o}mt{\"a}theten i denna lag fr{\aa}n
Amp\`eres statiska lag, s{\aa} har vi med andra ord att
$$
  \nabla\cdot{\bf J}
    ={{1}\over{\mu_0}}\nabla\cdot(\nabla\times{\bf B})
    =\big\{\hbox{ Griffiths vektoridentitet }\big\}
    \equiv 0,
$$
detta trots att vi f{\"o}r att uppfylla den fundamentala lagen om laddningens
bevarande borde ha haft ett ``$-{{\partial\rho}/{\partial t}}$'' i
h{\"o}gerledet ist{\"a}llet f{\"o}r en nolla.

Tricket som James Clerk Maxwell kom p{\aa} i l{\"o}sandet av detta problem
var helt enkelt att l{\"a}gga till en term $\varepsilon_0{{\partial{\bf E}}
/{\partial t}}$ till den fria str{\"o}mt{\"a}theten, en s{\aa} kallad
{\it f{\"o}rskjutningsstr{\"o}m}, s{\aa} att vi helt enkelt ers{\"a}tter
den statiska fria str{\"o}mmen ${\bf J}$ i Amp\`eres lag med
$$
  {\bf J}\quad\to\quad{\bf J}+\varepsilon_0{{\partial{\bf E}}\over{\partial t}},
$$
det vill s{\"a}ga med Amp\`eres lag p{\aa} formen
$$
  \nabla\times{\bf B}=\mu_0\bigg(
    {\bf J}+\varepsilon_0{{\partial{\bf E}}\over{\partial t}}
  \bigg).\qquad\hbox{(Dynamiskt!)}
$$
med f{\"o}ljd att
$$
  \eqalign{
    \nabla\cdot{\bf J}
      &=\nabla\cdot\bigg(
           {{1}\over{\mu_0}}\nabla\times{\bf B}
             -\varepsilon_0{{\partial{\bf E}}\over{\partial t}}
        \bigg)\cr 
      &={{1}\over{\mu_0}}\underbrace{\nabla\cdot(\nabla\times{\bf B})}_{\equiv0}
           -\varepsilon_0{{\partial}\over{\partial t}}
           \underbrace{\nabla\cdot{\bf E}}_{=\rho/\varepsilon_0}\cr
      &=-{{\partial\rho}\over{\partial t}},\cr
 }
$$
det vill s{\"a}ga att vi med den extra termen f{\"o}r
f{\"o}rskjutningsstr{\"o}mmen n{\"a}rvarande direkt l{\"o}ser problemet med
kontinuitetsekvationen f{\"o}r dynamiska f{\"a}lt.
Det finns en liten korrektion som fortfarande beh{\"o}ver g{\"o}ras i
tolkningen av f{\"o}rskjutningsstr{\"o}mmen, som vi kommer att {\aa}terkomma
till i F{\"o}rel{\"a}sning~9 d{\aa} vi till slut kommer att s{\"a}tta samman
alla pusselbitar och till slut formulera den slutliga formen av Maxwells
ekvationer och utifr{\aa}n dem hur elektromagnetiska v{\aa}gor beskrivs
p{\aa} differentialform.\numberedfootnote{Vilket f{\"o}r {\"o}vrigt
  r{\aa}kar vara just den {\it Tentamensuppgift~1} som delades ut
  under F{\"o}rel{\"a}sning~1!}
\vfill\eject

\section{Sammanfattning av F{\"o}rel{\"a}sning~5 -- Elektromagnetisk induktion}
\item{$\bullet$}{Faradays lag kan inte h{\"a}rledas fr{\aa}n Coulombs eller
  Biot--Savarts lag, och inneh{\aa}ller som en f{\"o}ljd av detta ej heller
  de karakteristiska sp{\aa}ren fr{\aa}n dem i form av den elektriska
  permittiviteten $\varepsilon_0$ eller den magnetiska permeabiliteten $\mu_0$.}
\item{$\bullet$}{Magnetiskt fl{\"o}de \sidx{Magnetiskt fl{\"o}de $\Phi_{\rm M}$}
  definieras som
  $$
    \Phi_{\rm M}=\iint_S{\bf B}\cdot d{\bf S}.
  $$}
\item{$\bullet$}{Den elektromotoriska ``kraften'' (en mycket missvisande term)
  runt en sluten slinga $\Gamma$ definieras som
  \sidx{Elektromotorisk ``kraft'' ${\cal E}$}\sidx{Str{\"o}mslinga}[Sluten]
  $$
    {\cal E}=\oint_{\Gamma}\bigg({{{\bf F}}\over{q}}\bigg)\cdot d{\bf l}
      =\oint_{\Gamma}[{\bf E}+({\bf v}\times{\bf B})\big]\cdot d{\bf l},
  $$}
\item{$\bullet$}{Faradays induktionslag (``{\it the flux law}'') relaterar
  den genererade elektromotoriska ``kraften'' i en slinga $\Gamma$ till en
  f{\"o}r{\"a}ndring av det magnetiska fl{\"o}det med omv{\"a}nt tecken som
  $$
    {\cal E}=-{{d\Phi_{\rm M}}\over{dt}}.
  $$
  Faradays induktionslag h{\"a}rleds endast utifr{\aa}n Lorentz-kraften
  (agerande p{\aa} en hypotetisk ladd\-ning), och {\"a}r d{\"a}rmed i
  avsaknad av s{\aa}v{\"a}l den elektriska permittiviteten $\varepsilon_0$
  som den magnetiska permeabiliteten $\mu_0$ d{\aa} varken Coulombs eller
  Biot--Savarts lag anv{\"a}nts.}
\item{$\bullet$}{Lenz lag s{\"a}ger att {\it en inducerad str{\"o}m har en
  riktning som motverkar orsaken till att den uppkom.} Detta inneb{\"a}r att
  om magnetf{\"a}ltet genom en ledande slinga {\it {\"o}kar}, s{\aa} kommer
  den i slingan inducerade str{\"o}mmen att ha en riktning som skapar ett
  magnetf{\"a}lt som {\it motverkar} {\"o}kningen.}
\item{$\bullet$}{F{\"o}r en spole med $N$ varv ($N$ slingor) blir den
  resulterande elektromotoriska ``kraften'' utv{\"a}xlad i motsvarande grad som
  $$
    {\cal E}=-N{{d\Phi_{\rm M}}\over{dt}}.
  $$}
\item{$\bullet$}{Faradays lag p{\aa} differentialform och integralform lyder
  $$
    \nabla\times{\bf E}({\bf r},t)
      =-{{\partial{\bf B}({\bf x},t)}\over{\partial t}}
    \qquad\Leftrightarrow\qquad
    \oint_{\Gamma}{\bf E}({\bf r},t)\cdot d{\bf l}
      =-{{d}\over{dt}}\iint_S{\bf B}({\bf x},t)\cdot d{\bf S}.
  $$}
\item{$\bullet$}{{\"O}msesidig induktans beskrivs av det magnetiska fl{\"o}de
  $\Phi_{\rm M}$ som genereras i en sekund{\"a}rslinga $\Gamma$ fr{\aa}n en
  str{\"o}m $I'(t)$ som drivs genom en prim{\"a}rslinga $\Gamma'$, som
  $$
    \Phi_{\rm M}=M_{\Gamma\Gamma'} I'(t),
  $$
  d{\"a}r den {\it {\"o}msesidiga induktansen} enligt Neumanns formel ges som
  \sidx{Neumanns formel}[{\"O}msesidig induktans]
  $$
    M_{\Gamma\Gamma'}={{\mu_0}\over{4\pi}}\oint_{\Gamma}\oint_{\Gamma'}
          {{d{\bf l}'\cdot d{\bf l}}\over{|{\bf x}-{\bf x}'|}},
  $$
  vilket {\"a}r en rent geometrisk konstruktion, skalad med den magnetiska
  permeabiliteten $\mu_0/4\pi$.}
\item{$\bullet$}{I den {\"o}msesidiga induktansen kan vi godtyckligt v{\"a}lja
  vad vi v{\"a}ljer att beteckna som prim{\"a}r- eller sekund{\"a}rslinga,
  d{\aa}
  $$
    M_{\Gamma\Gamma'}=M_{\Gamma'\Gamma},
  $$
  vilket i sin tur s{\"a}ger oss att det magnetiska fl{\"o}de $\Phi_{\rm M}$
  som vi erh{\aa}ller i sekund{\"a}rslingan $\Gamma$ d{\aa} vi driver
  prim{\"a}rslingan med en str{\"o}m $I'(t)$ {\"a}r exakt lika stort
  som det maknetiska fl{\"o}de $\Phi'_{\rm M}$ som vi skulle detektera
  i prim{\"a}rslingan $\Gamma'$ om vi ist{\"a}llet drev sekund{\"a}rslingan
  $\Gamma$ med samma str{\"o}m $I'(t)$.}

\cleardoublepage
%%% End of auto-extracted text from ../lect-05/lecture-05.tex %%%
%%% Begin of auto-extracted text from ../lect-06/lecture-06.tex %%%
%
% File: teach/elmagii/lect-06/lecture-06.tex [plain TeX code]
% Github: https://github.com/elmagii/lect-06/
% Last change: November 18, 2024
%
% Lecture No 6 in the course ``Elektromagnetism II, 1TE626 (2025)'',
% held November 18, 2025, at Uppsala University, Sweden.
%
% Copyright (C) 2022-2025, Fredrik Jonsson, under Gnu General Public
% License (GPL) v3. See the enclosed LICENSE for details.
%
% This program is free software: you can redistribute it and/or modify
% it under the terms of the GNU General Public License as published by
% the Free Software Foundation, either version 3 of the License, or
% (at your option) any later version.
%
% This program is distributed in the hope that it will be useful,
% but WITHOUT ANY WARRANTY; without even the implied warranty of
% MERCHANTABILITY or FITNESS FOR A PARTICULAR PURPOSE.  See the
% GNU General Public License for more details.
%
% You should have received a copy of the GNU General Public License
% along with this program.  If not, see <https://www.gnu.org/licenses/>.
%
\def\coursename{Elektromagnetism II}
\def\coursecode{1TE626}
\def\courseyear{2025}
\def\courserepo{https://github.com/hp35/elmagii/}
\def\lecturenumber{6}
\def\lecturetitle{Elektriska f\"alt i material}
\def\lecturesubtitle{}
\def\lectureauthor{Fredrik Jonsson}
\def\lectureplace{Uppsala Universitet}
\def\lecturedate{18 november 2025}
%-------------------- BEGIN OF LOCAL MACROS --------------------
\edef\expandedlecturenumber{6}
\def\ifempty#1{\ifx\relax#1\relax}
\advance\chapno by 1
\secno=0
\footnotenumber=0
\message{==================== Lecture 6 ====================}
\writenumberedtocentry{chapter}{F{\"o}rel{\"a}sning 6 -- {Elektriska f\"alt i material}}{\thechapno}
\hsize=150mm\hoffset=4.6mm\vsize=230mm\voffset=7mm
\topskip=0pt\baselineskip=12pt\parskip=0pt\leftskip=0pt\parindent=15pt
\ifcolors
  \voffset=-10.2mm\topskip=0pt
\fi
\headline={\ifnum\secno>0\ifodd\pageno\rightheadline\else\leftheadline\fi
  \else\hfill\fi}
\def\rightheadline{\tenrm{\it F\"orel\"asning 6}
  \hfil{\it \coursename, \coursecode\ (\courseyear)}}
\def\leftheadline{\tenrm{\it \coursename, \coursecode\ (\courseyear)}
  \hfil{\it F\"orel\"asning 6}}
\noindent~\vskip-60pt\hskip-40pt{\epsfbox{../lect-01/macros/UU_logo_color.eps}}
\vskip-42pt\hfill\vbox{
    \hbox{{\it \coursename, \coursecode\ (\courseyear)}}
    \hbox{{\it Lecture Notes, \lectureauthor}}
    \hbox{{\it Document Revision \today}}
    \hbox{{\it \courserepo}}}\vskip 36pt
\centerline{\twelvesc F\"orel\"asning 6}
\vskip 24pt\noindent
\centerline{\twelvesc{Elektriska f\"alt i material}}
\expandafter\ifempty\expandafter{\lecturesubtitle}%
  \else\centerline{\twelvesc\lecturesubtitle}\fi
\bigskip
\centerline{\lectureauthor, \lectureplace, \lecturedate}
\vskip24pt
%--------------------- END OF LOCAL MACROS ---------------------



\plan{I och med denna f{\"o}rel{\"a}sning l{\"a}mnar vi vakuumbeskrivningen av
  f{\"a}lt och g{\aa}r in p{\aa} v{\"a}xelverkan mellan elektriska eller
  magnetiska f{\"a}lt och materia. Vi b{\"o}rjar med en {\"o}vergripande bild
  av hur vi successivt kan g{\aa} fr{\aa}n en kvantmekanisk beskrivning av
  denna v{\"a}xelverkan till makroskopiska begrepp som susceptibiliteter och
  brytningsindex.
  
  Den klassiska dipolmodellen f{\"o}r elektrisk polarisering av ett materials
  molekyler av elektriska f{\"a}lt introduceras, och ett linj{\"a}rt samband
  mellan p{\aa}lagt externt elektriskt f{\"a}lt och det resulterande elektriska
  dipolmomentet hos mediet erh{\aa}lls.
  En medelv{\"a}rdesbildning {\"o}ver molekylernas elektriska dipolmoment ger
  den elektriska polarisa\-tions\-densi\-teten hos det dielektriska mediet,
  och vi formulerar ur denna den elektriska susceptibiliteten $\chi_{\rm e}$
  och den relativa elektriska permittiviteten $\varepsilon_{\rm r}$.
  Vi h{\"a}rleder generaliseringen f{\"o}r Gauss lag f{\"o}r den elektriska
  fl{\"o}dest{\"a}theten ${\bf D}$ som $\nabla\cdot{\bf D}={\rho}_{\rm f}$,
  d{\"a}r ${\rho}_{\rm f}$ {\"a}r den fria laddningst{\"a}theten.
  
  Randvillkor och {\"o}verg{\aa}ngar mellan olika dielektrika h{\"a}rleds
  f{\"o}r den elektriska f{\"a}ltstyrkan ${\bf E}$ och den elektriska
  fl{\"o}dest{\"a}theten ${\bf D}$, f{\"o}r deras normal- och
  tangentialkomponenter.
  Slutligen formulerar vi m{\aa}ttet p{\aa} upplagrad elektrisk energi i
  ett dielektrikum.}

\threepointsummary{%
  Den elektriska polarisationsdensiteten ${\bf P}$ beskriver
  det elektriska dipolmomentet per volymenhet,
  $$
    {\bf P} \equiv \Big\langle{{d{\bf p}}\over{dV}}\Big\rangle
      = \varepsilon_0\chi_{\rm e}{\bf E},
  $$
  d{\"a}r $\chi_{\rm e}$ {\"a}r den {\it elektriska susceptibiliteten}.
}{%
  Den elektriska fl{\"o}dest{\"a}theten,
  $$
    {\bf D}
        \equiv\varepsilon_0{\bf E}+{\bf P}
        =\varepsilon_0(1+\chi_{\rm e}){\bf E}
        \equiv\varepsilon_0\varepsilon_{\rm r}{\bf E},
  $$
  lyder Gauss lag $\nabla\cdot{\bf D}=\rho_{\rm f}$, d{\"a}r $\rho_{\rm f}$
  {\"a}r den fria laddningst{\"a}theten.
}{%
  Den upplagrade energin hos ett dielektrikum i ett externt p{\aa}lagt
  elektriskt f{\"a}lt ${\bf E}$ ges av
  $$
    W = {{1}\over{2}}\iiint{\bf D}\cdot{\bf E}\,dV.
  $$
}
\vfill\eject\copyrights

\section{Introduktion}
Vi kommer i denna f{\"o}rel{\"a}sning att behandla hur ett elektriskt icke
ledande material (dielektrikum\sidx{Dielektrikum}) upptr{\"a}der d{\aa} det
p{\aa}verkas av ett p{\aa}lagt elektriskt f{\"a}lt, \sidx{Elektrisk
f{\"a}ltstyrka} specifikt hur vi elektriskt polariserar mediet
genom att de atom{\"a}ra eller molekyl{\"a}ra elektriska dipolerna
\sidx{Elektrisk dipol} linjeras i f{\"a}ltet.

Om man skall sammanfatta vad vi rent makroskopiskt kan uppfatta g{\"a}llande
elektrisk polarisering i material, s{\aa} kan vi konstatera att denna
polarisering {\"a}r sj{\"a}lva grunden till all form av {\it reflektion},
samt i f{\"o}rl{\"a}ngningen {\"a}ven {\it refraktion} (ljusbrytning) och
{\it diffraktion}.
I och med att allt vi {\"o}verhuvudtaget ser {\"a}r spritt ljus, m{\"o}jligen
undantaget en blick direkt in i en ljusk{\"a}lla, s{\aa} kan man d{\"a}rmed
s{\"a}ga att elektrisk polarisation hos matrial {\"a}r k{\"a}llan till praktiskt
allt vi ser med v{\aa}ra {\"o}gon. Elektriska f{\"a}lt i material och elektrisk
polarisation har med andra ord en mycket konkret effekt p{\aa} v{\aa}rt dagliga
liv.

\section{V{\"a}rldskarta}
V{\"a}xelverkan mellan elektriska f{\"a}lt och materia {\"a}r ett komplext
{\"a}mne som generellt m{\aa}ste utg{\aa} fr{\aa}n de intrinsiska (inneboende)
egenskaperna hos atomerna som utg{\"o}r mediet, hur dessa {\"a}r arrangerade
(exempelvis i gitter, som i kristallina material, eller slumpvis, som i en gas
eller ett amorft medium).
Vi kommer h{\"a}r sj{\"a}lvfallet inte att g{\aa} igenom kvantmekaniken i sig,
men det kan vara bra att ha i huvudet vad vi siktar mot att f{\"o}rs{\"o}ka
f{\aa} till en modell f{\"o}r.

Grunden f{\"o}r hur vi g{\aa}r fr{\aa}n atomer och deras kvantmekniska
v{\aa}gfuktioner upp till makro\-skopiskt observerbara effekter som relativ
permittivitet (elektrisk susceptibilitet) kan generellt sammanfattas med
f{\"o}ljande v{\"a}rldskarta:
$$
  \matrix{
  \psi_j({\bf r},t)
        &\hskip40pt\hbox to 220pt{V{\aa}gfuktion (fr{\aa}n Schr{\"o}dinger;
        beroende av ${\bf E}$)\sidx{Kvantmekanik}[V{\aa}gfunktion]\hfill}\cr
  \downarrow&\cr
  \hat\rho=\sum_j p_j|\psi_j\rangle\langle\psi_j|
        &\hskip40pt\hbox to 220pt{Densitetsmatris/operator
                                  (sannolikheter)\hfill}\cr
  \downarrow&\cr
  {\bf p}=\langle\hat{\bf p}\rangle={\rm Tr}(\hat\rho\hat{\bf p})
        &\hskip40pt\hbox to 220pt{Elektrisk polarisation (dipol,
           ${\rm C}\cdot{\rm m}$)\sidx{Elektrisk dipol}\hfill}\cr
  \downarrow&\cr
  \displaystyle
  {\bf P}={{\Delta{\bf p}}\over{\Delta V}}
        &\hskip40pt\hbox to 220pt{\idx{Elektrisk polarisationsdensitet}
            (makrosk., ${\rm C}/{\rm m}^2$)\hfill}\cr
  \downarrow&\cr
  {\bf P}=\varepsilon_0\chi_{\rm e}{\bf E}
         =\varepsilon_0(1-\varepsilon_{\rm r}){\bf E}
        &\hskip40pt\hbox to 220pt{\sidx{Elektrisk susceptibilitet $\chi_{\rm e}$}
                         (enhetsl{\"o}s)\hfill}\cr
  \downarrow&\cr
  {\bf D}\equiv\varepsilon_0(1+\chi_{\rm e}){\bf E}
                    =\varepsilon_0\varepsilon_{\rm r}{\bf E}
        &\hskip40pt\hbox to 220pt{\idx{Elektrisk fl{\"o}dest{\"a}thet}
                                  {\bf D} (${\rm C}/{\rm m}^2$)\hfill}\cr
  }
$$
Fr{\aa}n den elektriska susceptibiliteten $\chi_{\rm e}$ {\"o}ppnar sig
\sidx{Elektrisk susceptibilitet $\chi_{\rm e}$} d{\"a}refter en m{\"a}ngd olika
discipliner inom fysik, kemi och till{\"a}mpad matematik. N{\aa}gra exempel:
\medskip
\item{$\bullet$}{Brytningsindex $n=\varepsilon^{1/2}_{\rm r}$: Linser, teleskop,
                 optiska fibrer, speglar, radar. ``Allt vi kan se.''
                 \sidx{Brytningsindex}}
\item{$\bullet$}{Radar, ekolod, skiktr{\"o}ntgen, radiov{\aa}gspropagation
                 {\"o}ver l{\aa}nga distanser.}
\item{$\bullet$}{Dispersion med v{\aa}gl{\"a}ngdsberoende brytningsindex,
                 $n=n(\omega)$.\sidx{Dispersion}}
\item{$\bullet$}{Dubbelbrytning med brytningsindex beroende p{\aa} riktning
                 och polarisation hos ljus.\sidx{Dubbelbrytning}}
\item{$\bullet$}{Solitoner och sj{\"a}lvfokuserande ljus, med
                 intensitetsberoende brytningsindex $n=n(I,\omega)$.}
\item{$\bullet$}{Optisk {\"o}vertonsgenerering med ickelinj{\"a}r optik,
                 $\omega_3=\omega_1+\omega_2$.}
\medskip
\noindent
Not: Griffiths anv{\"a}nder generellt symbolen ``$V$'' f{\"o}r att beteckna
potentialer, vilket {\"a}r lite olyckligt, d{\aa} $V$ normalt anv{\"a}nds
f{\"o}r att beteckna s{\aa}v{\"a}l volym som elektrisk sp{\"a}nning (elektrisk
potentialskillnad). Vi kommer h{\"a}r d{\"a}rf{\"o}r att ist{\"a}llet
genomg{\aa}ende anv{\"a}nda $\phi$ f{\"o}r att beteckna en potential.
\sidx{Skal{\"a}r potential $\phi$}
\vfill\eject

\section{Klassisk modell f{\"o}r dipoler}
\sidx{Elektrisk dipol}[Klassisk modell f{\"o}r]
Den enklast m{\"o}jliga modell vi kan f{\"o}rest{\"a}lla oss f{\"o}r hur ett
medium kan polariseras {\"a}r genom att betrakta var och en av de (neutrala)
atomerna som utg{\"o}r materialet som best{\aa}ende av en positiv punktladdning
(atomk{\"a}rnan) och ett sf{\"a}risk-symmetriskt moln av homogen laddning
(elektronmolnet) omg{\"a}rdande k{\"a}rnan. D{\aa} mediet uts{\"a}tts f{\"o}r
ett externt p{\aa}lagt elektriskt f{\"a}lt kommer atomernas negativt laddade
elektronmoln \sidx{Elektronmoln} att dras mot ``$+$'', medan den positivt
laddade k{\"a}rnan \sidx{Atomk{\"a}rna} ist{\"a}llet dras mot ``$-$''.
Detta ger upphov till en elektrisk polarisering av atomen, vilket resulterar i
att vi har en kollektiv polarisering av mediet p{\aa} makroskopisk niv{\aa}.

Observera att den f{\"o}ljande modellen {\"a}r extremt f{\"o}renklad och bara
kan betraktas som en illustrativ modell {\"o}ver v{\"a}xelverkan mellan
f{\"a}lt och materia. I verkligheten skel v{\"a}xelverkan p{\aa} kvantniv{\aa}
d{\"a}r vi projicerar atomens eller molekylens v{\aa}gfunktion p{\aa} en
dipoloperator.\sidx{Kvantmekanik}[Dipoloperator]
Dessutom f{\"o}rsummar denna modell helt och h{\aa}llet eventuella
multipolmoment \sidx{Multipolmoment} hos systemet (mer om detta i kommande
f{\"o}rel{\"a}sningar).
\epsfig{../lect-06/figs/dipole.1}\noindent
Eftersom k{\"a}rnan {\"a}r mycket tyngre {\"a}n elektronmolnet s{\aa} kan vi
dessutom g{\"o}ra det enkelt f{\"o}r oss och helt enkelt se k{\"a}rnan som fix
i rummet och med elektronmolnet r{\"o}rligt kring k{\"a}rnan.
D{\aa} vi applicerar ett elektriskt f{\"a}lt {\"o}ver detta system, vilket vi
s{\aa} l{\aa}ngt antar vara statiskt i den m{\aa}n att f{\"a}ltets
tidsvariation {\"a}r l{\aa}ngsam j{\"a}mf{\"o}rt med naturliga
resonansfrekvenser i det atom{\"a}ra systemet, s{\aa} komer elektronmolnet att
f{\"o}rskjutas relativt atomk{\"a}rnan. Den negativa laddningen kommer att
f{\"o}rskjutas mot positiv elektrisk potential (k{\"a}llan ``$+$'' f{\"o}r
det elektriska f{\"a}ltet), och vi kommer att f{\aa} en resulterande elektrisk
dipol.\sidx{Elektrisk dipol}
\vfill\eject

\epsfig{../lect-06/figs/displace.1}\noindent
F{\"o}r att ta fram relationen mellan det resulterande dipolmomentet ${\bf p}$
\sidx{Dipolmoment}[Elektriskt]
och det externa elektriska f{\"a}ltet ${\bf E}_{\rm ext}$, s{\aa} kan vi
exempelvis betrakta de mekaniska krafterna som verkar p{\aa} k{\"a}rnan
\encircle{+}.
F{\"o}rst av allt ges kraften p{\aa} k{\"a}rnan fr{\aa}n det externa f{\"a}ltet
som
$$
  {\bf F}_{\rm ext}=(+q){\bf E}_{\rm ext}.
$$
Den elektrostatiska kraften \sidx{Elektrostatisk kraft} ${\bf F}_{\rm e}$ p{\aa}
k{\"a}rnan fr{\aa}n det omgivande elektronmolnet, \sidx{Elektronmoln} som
str{\"a}var mot att centrera k{\"a}rnan \sidx{Atomk{\"a}rna} och elektronmolnen
mot varandra, {\"a}r
$$
  {\bf F}_{\rm e}=(+q){\bf E}_{\rm e},
$$
d{\"a}r ${\bf E}_{\rm e}$ {\"a}r det elektriska f{\"a}lt \sidx{Elektrisk
f{\"a}ltstyrka} som k{\"a}rnan upplever fr{\aa}n den negativa elektriska
laddningen hos elektronmolnet.

Det elektriska f{\"a}ltet fr{\aa}n elektronmolnet kan tas fram genom att
utnyttja Gauss lag \sidx{Gauss lag} under sf{\"a}risk symmetri
\sidx{Sf{\"a}risk symmetri} {\"o}ver den sf{\"a}r $\Omega$
som {\"a}r centrerad i elektronmolnet och med k{\"a}rnan p{\aa} ytan,
\sidx{Integral}[Sluten yt-]\sidx{Integral}[Volym-]
$$
  \iiint_V\nabla\cdot{\bf E}_{\rm e}\,dV
    =\oiint_{\Omega}{\bf E}_{\rm e}\cdot d{\bf S}
    ={{Q}\over{\varepsilon_0}},
$$
d{\"a}r $Q$ {\"a}r den av $\Omega$ inneslutna laddningen.
Med laddningsdensiteten $\rho$ \sidx{Laddningst{\"a}thet} f{\"o}r elektronmolnet,
och med detta moln f{\"o}rskjutet str{\"a}ckan $r$, {\"a}r med andra ord den
inneslutna laddningen
$$
  Q={{4\pi r^3}\over{3}}\rho
   =\bigg\{\rho={{(-q)}\over{(4\pi a^3/3)}}\bigg\}
   =-{{r^3}\over{a^3}}q.
$$
Samtidigt har vi fr{\aa}n sf{\"a}risk symmetri \sidx{Sf{\"a}risk symmetri} hos
elektronmolnet att den radiella komponenten av elektriska f{\"a}ltet p{\aa}
sf{\"a}ren med radie $r$ {\"a}r konstant, och ytintegralen \sidx{Integral}[Yt-]
ges d{\"a}rmed som
$$
  \oiint_{\Omega}{\bf E}_{\rm e}\cdot d{\bf S} =E_{\rm e}4\pi r^2.
$$
Om vi s{\"a}tter samman dessa uttryck har vi allts{\aa} att
$$
  E_{\rm e}4\pi r^2 = -{{r^3}\over{a^3}}{{q}\over{\varepsilon_0}}
  \qquad\Leftrightarrow\qquad
  E_{\rm e} = -{{q}\over{4\pi\varepsilon_0}}{{r}\over{a^3}}.
$$
Den {\aa}terst{\"a}llande kraften som verkar p{\aa} k{\"a}rnan fr{\aa}n
elektronmolnet {\"a}r med andra ord
$$
  F_{\rm e} = -{{q^2}\over{4\pi\varepsilon_0}}{{r}\over{a^3}}.
$$
Kraftj{\"a}mvikt inneb{\"a}r d{\"a}rmed att
$$
  F_{\rm ext}+F_{\rm e} = {\bf 0}
  \qquad\Leftrightarrow\qquad
  qE_{\rm ext}-{{q^2}\over{4\pi\varepsilon_0}}{{r}\over{a^3}} = 0
  \qquad\Leftrightarrow\qquad
  r = {{4\pi\varepsilon_0 a^3}\over{q}} E_{\rm ext}.
$$
Det resulterande dipolmomentet \sidx{Dipolmoment}[Elektriskt] {\"a}r d{\"a}rmed
$$
  p = qr
    = \underbrace{4\pi\varepsilon_0 a^3}_{=\alpha} E_{\rm ext}
    = \alpha E_{\rm ext},
$$
d{\"a}r $\alpha$ {\"a}r den s{\aa} kallade {\it atom{\"a}ra
polarisabiliteten}.\numberedfootnote{Exempel p{\aa} v{\"a}rden f{\"o}r denna
    {\aa}terfinns i Griffiths, Tabell 4.1 p{\aa} sidan 168.}
\sidx{Atom{\"a}r polarisabilitet $\alpha$}
Utifr{\aa}n denna extremt f{\"o}renklade modell kan vi observera ett par
intressanta saker:
\medskip
\item{$\bullet$}{Dipolmomentet {\"a}r en linj{\"a}r funktion av det
                 p{\aa}lagda elektriska f{\"a}ltet. Detta brukar vi beteckna
                 med att {\it systemet {\"a}r linj{\"a}rt}.\sidx{Linj{\"a}rt
                 system}\sidx{Dipolmoment}[Elektriskt]}
\item{$\bullet$}{Dipolmomentet {\"a}r {\"a}ven riktat {\it l{\"a}ngs med} det
                 p{\aa}lagda (godtyckligt riktade) elektriska f{\"a}ltet.
                 Detta {\"a}r ett beteende vi associerar vi med ett
                 {\it isotropt} medium.\sidx{Isotropt medium}}
\item{$\bullet$}{Dipolmomentet beror endast p{\aa} eventuell frekvens hos
                 det p{\aa}lagda elektriska f{\"a}ltet. Detta {\"a}r ett
                 beteende vi associerar vi med ett medium fritt fr{\aa}n
                 {\it dispersion} (frekvensoberoende).\sidx{Dispersion}}
\item{$\bullet$}{Polarisabiliteten beror p{\aa} ``atomens''
                 storlek\numberedfootnote{``Atomens'' med citat-tecken,
                 d{\aa} vi ju alla vet att atomer inte {\"a}r isolerade
                 sf{\"a}rer som l{\aa}ter sig beskrivas med klassisk
                 mekanik.} $a$ i sig.}
\item{$\bullet$}{Polarisabiliteten beror d{\"a}remot {\it inte} p{\aa} den
                 totala laddning som k{\"a}rnan uppb{\"a}r (d.v.s. atomnummer).
                 \sidx{Atomnummer}}
\medskip
\section{Varf{\"o}r inte en modell med punktladdningar ist{\"a}llet?}
Retorisk fr{\aa}ga: {\it Varf{\"o}r konstla till det med ett elektronmoln om vi
{\"a}nd{\aa} effektivt har verkan av molnet riktat in mot (det sf{\"a}riska)
molnets centrum? Skulle vi inte lika kunna f{\aa} till en enklare men lika
kvalitativ modell genom att ist{\"a}llet ans{\"a}tta tv{\aa} punktladdningar
med olika tecken?}

Svaret p{\aa} detta {\"a}r att denna modell direkt kommer att resultera i en
direkt ofysikalisk situa\-tion redan utan ett p{\aa}lagt elektriskt f{\"a}lt,
d{\aa} den {\"o}msesidiga attraktionskraften mellan punktladdningarna i s{\aa}
fall skulle g{\aa} mot o{\"a}ndligheten, d{\aa} Coulombkraften\idx{Coulombs
kraftlag}
$$
  F_{\rm c}={{q^2}\over{4\pi\varepsilon_0 r^2}}\to\infty,\qquad r\to 0.
$$
I modellen med elektronmolnet kommer kraften att n{\"a}rma sig noll d{\aa}
$r\to 0$, eftersom vi med Gauss lag s{\aa} att s{\"a}ga ``skalar av'' den
inneslutna laddningen lager f{\"o}r lager d{\aa} radien $r$, det vill s{\"a}ga
avst{\aa}ndet mellan punktladdningen $+q$ f{\"o}r k{\"a}rnan och centrum
f{\"o}r elektronmolnet, successivt minskas.
\vfill\eject

\section{En intressant f{\"o}ljd av Gauss lag f{\"o}r sf{\"a}risk-symmetriska
  laddningsf{\"o}rdelningar}
\sidx{Sf{\"a}risk symmetri}
\sidx{Gravitation}[Analogi med laddningsf{\"o}rdelning]
\sidx{Gravitation}[Ekvivalent punktladdning]
\sidx{Elektrisk f{\"a}ltstyrka ${\bf E}$}[Analogi med gravitation]
\sidx{Elektrisk f{\"a}ltstyrka ${\bf E}$}[Ekvivalent punktladdning]
I f{\"o}reg{\aa}ende analys av den {\aa}terst{\"a}llande kraften f{\"o}r
elektronmolnet, s{\aa} anv{\"a}nde vi Gauss lag f{\"o}r att genom den extrahera
den elektrostatiska kraften ${\bf F}_{\rm e}$. Vi kan notera att vi {\it generellt
har att en sf{\"a}risk laddningsf{\"o}rdelning agerar s{\aa} som om vi hade att
g{\"o}ra med en punktladdning placerad i den sf{\"a}riska f{\"o}rdelningens
centrum, med det elektriska f{\"a}ltet motsvarande den laddning som innesluts
med den radie fr{\aa}n centrum som n{\aa}r observat{\"o}ren}.
\epsfig{../lect-06/figs/gravity.1}\noindent
F{\"o}r ett sf{\"a}riskt-symmetriskt ``moln'' av laddning centrerat i origo, har
vi helt klart att vi vid varje position runtomkring origo, inuti s{\aa}v{\"a}l
som utanf{\"o}r molnet, har att det elektriska f{\"a}ltet endast har en
{\it radiell} komponent, med tecken beroende p{\aa} om den totala inneslutna
laddningen {\"a}r positiv eller negativ. Formulerat med \idx{Gauss lag} har vi
d{\"a}rmed
att
$$
  \oiint_{\Omega:|{\bf x}|=r}{\bf E}\cdot d{\bf S}
    =4\pi r^2 E_r(r)
    ={{1}\over{\varepsilon_0}}\iiint_{V:|{\bf x}|\le r}\rho(r')\,dV'
    ={{1}\over{\varepsilon_0}}\int^r_0\rho(r')
      \underbrace{
        \underbrace{4\pi r'^2}_{=A'}\,dr'
      }_{=dV'},
$$
vilket vi direkt kan l{\"o}sa f{\"o}r det radiella elektriska f{\"a}ltet som
$$
  E_r(r)={{1}\over{4\pi\varepsilon_0 r^2}}
           \underbrace{\int^r_0\rho(r') 4\pi r'^2\,dr'}_{\equiv q_{\rm enc}(r)}
        ={{q_{\rm enc}(r)}\over{4\pi\varepsilon_0 r^2}},
$$
d{\"a}r $q_{\rm enc}(r)$ {\"a}r den laddning som innesluts av sf{\"a}ren med samma
radie $r$ som avst{\aa}ndet till en observat{\"o}r.
\quote{{\it Notera att denna form {\"a}r identisk med den som skulle
  erh{\aa}llas f{\"o}r situationen d{\aa} all innesluten laddning
  $q_{\rm enc}(r)$ ist{\"a}llet skulle placerats som en punktladdning i
  origo.}\sidx{Punktladdning}}
\medskip\noindent
Detta resultat f{\"o}r det elektriska f{\"a}ltet {\"a}r helt och h{\aa}llet i
analogi med exempelvis den gravitation (gravitationskraft dividerat med massa)
som vi skulle uppleva om vi t{\"a}nker oss ett experiment d{\"a}r vi gr{\"a}ver
ett mycket djupt h{\aa}l i jordskorpan och kl{\"a}ttrar nedf{\"o}r detta
h{\aa}l mot centrum av jordklotet.\numberedfootnote{Vi ignorerar det faktum
  att det skulle bli mycket varmt efter ett par kilometer ner, och att vi
  ganska snart skulle brinna upp! Endast 5~km ner i jordskorpan har vi
  typiskt en temperatur{\"o}kning p{\aa} cirka $125$--$150^{\circ}\,{\rm C}$
  relativt yttemperaturen.}
Vi skulle i detta tanke-experiment uppleva en successivt mindre och mindre
tyngdkraft, motsvarande attraktionskraften fr{\aa}n den massa $m_{\rm enc}(r)$
som omsluts av den radie $r$ vid vilken vi r{\aa}kar befinna oss
p{\aa}.\numberedfootnote{Ironiskt nog med exakt samma notation i den
  ing{\aa}ende integralen f{\"o}r $m_{\rm enc}$ (${\rm kg}$), men med
  $\rho(r)$ ist{\"a}llet betecknande {\it densiteten} (${\rm kg}/{\rm m}^3$)
  f{\"o}r jorden.}
\vfill\eject

\section{Terminologi}
F{\"o}r att beskriva grundl{\"a}ggande egenskaper hos ett material, och d{\aa}
inte bara ur elektromagnetisk synvinkel, s{\aa} finns n{\aa}gra
grundl{\"a}ggande begrepp som kan vara bra att ha i bakhuvudet:
\medskip
\item{$\bullet$}{{\it Isotropt:} Invariant (``beter sig p{\aa} samma s{\"a}tt'')
under godtycklig rotation.
      \sidx{Isotropt medium}}
\item{}{{\it ``Mediet beter sig lokalt likadant oavsett i vilket riktning man
      tittar.''}}
\smallskip
\item{}{$\displaystyle \qquad\qquad
    \to P_i=\varepsilon_0\chi_{\rm e}E_i$\hskip20pt (med $\chi_{ij}$ diagonal).}
\smallskip
\item{$\bullet$}{{\it Anisotropt:} Egenskaperna hos mediet i ett fixt
      koordinatsystem (``labbsystem'') {\it {\"a}ndras} under rotation.
      \sidx{Anisotropt medium}}
\smallskip
\item{}{$\displaystyle \qquad\qquad\to P_i=\varepsilon_0\sum_{j}\chi_{ij}E_j$.}
\smallskip
\item{$\bullet$}{{\it Homogent:} Egenskaperna hos mediet {\"a}r samma oavsett
      vilken punkt man observerar.\sidx{Homogent medium}}
\item{}{{\it Notera att ett homogent medium fortfarande kan vara anisotropt!}}
\smallskip
\item{}{$\displaystyle \qquad\qquad\to \chi_{ij}=\hbox{konstant}$,
      oberoende av ${\bf r}$.}
\smallskip
\item{$\bullet$}{{\it Inhomogent:} Egenskaperna hos mediet varierar
      beroende p{\aa} var i rummet man observerar dem.\sidx{Inhomogent medium}}
\item{}{{\it Notera att ett inhomogent men i {\"o}vrigt lokalt isotropt
      medium ger effekter som kan vara beroende p{\aa} riktning!
      Ett vardagligt exempel {\"a}r polarisation hos solljuset som n{\aa}r
      oss vid solnedg{\aa}ng.}}
\smallskip
\item{}{$\displaystyle \qquad\qquad\to \chi_{ij}=\chi_{ij}({\bf r})$.}
\smallskip
\item{$\bullet$}{{\it Dispersion:} Egenskaperna hos mediet varierar beroende
      p{\aa} vilken frekvens (v{\aa}gl{\"a}ngd) ett drivande elektromagnetiskt
      f{\"a}lt har.}
\item{}{{\it Vardagligt exempel: F{\"a}rger fr{\aa}n vitt ljus som bryts
      i en kristall (eller billig lins).\sidx{Dispersion}}
\smallskip
\item{}{$\displaystyle \qquad\qquad\to \chi_{ij}=\chi_{ij}(\omega)$.}
\medskip
\section{Anisotropi}
\sidx{Anisotropt medium}
F{\"o}r molekyler som {\"a}r l{\aa}sta i gitter (kristallina strukturer)
{\"a}r polarisabiliteten generellt en {\it tensor} \sidx{Tensor} som
komponentvis ger hur polarisabiliteten i en viss riktning p{\aa}verkas
av en given komposant hos det externt p{\aa}lagda elektriska f{\"a}ltet
$E_k$, som
$$
  p_j=\alpha_{jk}E_k
  \qquad\Leftrightarrow\qquad
  \pmatrix{p_x\cr p_y\cr p_z}
     =\pmatrix{
         \alpha_{xx}&\alpha_{xy}&\alpha_{xz}\cr
         \alpha_{yx}&\alpha_{yy}&\alpha_{yz}\cr
         \alpha_{zx}&\alpha_{zy}&\alpha_{zz}
         }
      \pmatrix{E_x\cr E_y\cr E_z}.
$$
Denna form svarar f{\"o}r effekter som exempelvis dubbelbrytning och optisk
aktivitet, samt {\"a}ven Faraday-rotation (i vilken ett statiskt p{\aa}lagt
magnetf{\"a}lt skapar en anisotropi).
\sidx{Optisk aktivitet}\sidx{Faraday-rotation}
\vfill\eject

\section{\idx{Elektrisk polarisationsdensitet}}
I ett dielektrikum {\"a}r laddningarna bundna till molekylerna i ett effektivt
``moln'' av elektroner \sidx{Elektronmoln} kring en k{\"a}rna. Om detta moln av
elektroner ligger centrerat kring k{\"a}rnan \sidx{Atomk{\"a}rna} finns inga
permanenta elektriska dipoler i mediet, och d{\"a}rmed heller ingen
netto-polarisering. Om ett elektriskt f{\"a}lt \sidx{Elektrisk f{\"a}ltstyrka}
l{\"a}ggs p{\aa} {\"o}ver detta dielektrikum kommer dock molnet av elektroner
att dras mot den ``positiva'' k{\"a}llan f{\"o}r de elektriska f{\"a}ltlinjerna,
\sidx{F{\"a}ltlinjer}[Elektriska] och kommer d{\"a}rmed ocks{\aa} att orsaka en
elektrisk polarisering av materialet.
I den h{\"a}r modellen blir polariseringen av mediet riktad {\it l{\"a}ngs med}
f{\"a}ltlinjerna f{\"o}r ${\bf E}$-f{\"a}ltet.
\epsfig{../lect-06/figs/poldensity.1}\noindent
Vi m{\"a}ter denna inducerade (``p{\aa}tvingade'') polarisering som en
{\it elektrisk polarisationsdensitet} ${\bf P}$, beskrivande det elektriska
dipolmomentet per volymenhet (d{\"a}rav ``densitet''). Som en f{\"o}renklad
modell kan vi se detta inducerade dipolmoment som linj{\"a}rt beroende av det
p{\aa}lagda elektriska f{\"a}ltet,\numberedfootnote{Uttryckt i den nyligen
  n{\"a}mnda {\it atom{\"a}ra polarisabiliteten}
  \sidx{Atom{\"a}r polarisabilitet $\alpha$} som vi tog fram f{\"o}r
  en rent elektrostatisk/mekanisk modell motsvarar detta f{\"o}r en densitet
  med $N$ ``atomer'' per volym $V$ att
  $$
    {\bf P}=(N/V)\alpha{\bf E}=\varepsilon_0\chi_{\rm e}{\bf E},
  $$
  det vill s{\"a}ga att susceptibiliteten \sidx{Elektrisk susceptibilitet
  $\chi_{\rm e}$} ges i termer av detta oerh{\"o}rt f{\"o}renklade m{\aa}tt
  p{\aa} polarisabiliteten $\alpha$ som
  $$
    \chi_{\rm e}=(N/V)\alpha/\varepsilon_0.
  $$}
$$
  {\bf P} \equiv \Big\langle{{d{\bf p}}\over{dV}}\Big\rangle
    = \varepsilon_0\chi_{\rm e}{\bf E},
$$
d{\"a}r $\chi_{\rm e}$ {\"a}r den {\it elektriska susceptibiliteten} f{\"o}r
materialet ($\chi_{\rm e}$ {\"a}r dimensionsl{\"o}s) och d{\"a}r
$\langle\ldots\rangle$ betecknar medelv{\"a}rdesbildning.

Inom elektromagnetisk teori anv{\"a}nder vi ofta ett f{\"a}lt som beskriver
summan av den effektiva polariseringen som ges av det externt p{\aa}lagda
${\bf E}$-f{\"a}ltet tillsammans med den elektriska polariseringen av
materialet i sig, det s{\aa} kallade {\it elektriska fl{\"o}dest{\"a}theten}
(electric displacement field),\sidx{Elektrisk fl{\"o}dest{\"a}thet}
$$
  \eqalign{
    {\bf D}&\equiv\varepsilon_0{\bf E}+{\bf P}\cr
      &=\varepsilon_0(1+\chi_{\rm e}){\bf E}
       \equiv\varepsilon_0\varepsilon_{\rm r}{\bf E}.\cr
  }
$$
Vi kommer fram{\"o}ver alltsom oftast ist{\"a}llet f{\"o}r den elektriska
susceptibiliteten $\chi_{\rm e}$ att anv{\"a}nda den {\it relativa elektriska
permittiviteten} $\varepsilon_{\rm r}\equiv 1+\chi_{\rm e}$ (liksom
$\chi_{\rm e}$
\sidx{Relativ elektrisk permittivitet $\varepsilon_{\rm r}$}
\sidx{Elektrisk permittivitet}[Relativ permittivitet $\varepsilon_{\rm r}$]
\sidx{Elektrisk permittivitet}[Vakuumpermittivitet $\varepsilon_0$]
{\"a}ven den dimensionsl{\"o}s). F{\"o}r material med
f{\"o}rsumbar magnetisering och i avsaknad av fria laddningar motsvarar
$n=\sqrt{\varepsilon_{\rm r}}$ brytningsindex hos materialet (mer om detta
i kommande f{\"o}rel{\"a}sning om den elektromagnetiska v{\aa}gekvationen).
I kommande analys av Maxwells ekvationer och de fr{\aa}n dessa f{\"o}ljande
elektromagnetiska v{\aa}gekvationerna kommer vi genomg{\aa}ende att anv{\"a}nda
$\varepsilon_{\rm r}$ f{\"o}r att beskriva materialegenskaperna.
\vfill\eject

\section{Gauss teorem f{\"o}r elektriska fl{\"o}dest{\"a}theten {\bf D}}
\sidx{Gauss lag}[F{\"o}r elektrisk fl{\"o}dest{\"a}thet]
Att den elektriska fl{\"o}dest{\"a}theten ${\bf D}$
\sidx{Elektrisk fl{\"o}dest{\"a}thet} {\"a}r konstruerad som den
{\"a}r faller sig naturligt om vi betraktar, till exempel, hur en mer generell
form av Gauss lag kan formuleras i termer av fria s{\aa}v{\"a}l som bundna
laddningsdensiteter.
Genom att anv{\"a}nda att sambandet mellan divergensen av
polarisationsdensiteten ${\bf P}$ \sidx{Elektrisk polarisationsdensitet} och
den {\it bundna} laddningst{\"a}theten \sidx{Bundna laddningar}
\sidx{Bunden laddningst{\"a}thet $\rho_{\rm b}$}
\sidx{Laddningst{\"a}thet}[Bunden, $\rho_{\rm b}$]
$\rho_{\rm b}$,\numberedfootnote{Griffiths Ekv.~(4.12), sidan~174; f{\"o}r
  h{\"a}rledning av denna, se Griffiths, kapitel 4.2.1, sid.~173--174.}
$$
  \qquad
  \qquad
  \qquad
  \nabla\cdot{\bf P} = -\rho_{\rm b},
  \qquad
  \bigg(\quad\Leftrightarrow\qquad
  \iiint_V \rho_{\rm b}\,dV
    = -\iiint_V\nabla\cdot{\bf P}\,dV
    = -\oiint_{\Omega}{\bf P}\,dS
    \quad\bigg)
$$
s{\aa} har vi allts{\aa} f{\"o}r divergensen f{\"o}r det elektriska f{\"a}ltet
att
$$
  \varepsilon_0\nabla\cdot{\bf E}
    =\rho_{\rm tot}
    =\rho_{\rm b}+\rho_{\rm f}
    =-\nabla\cdot{\bf P}+\rho_{\rm f}.
$$
Genom att kombinera divergenserna kan detta skrivas som
$$
  \nabla\cdot(\varepsilon_0{\bf E}+{\bf P})=\rho_{\rm f},
$$
d{\"a}r nu allts{\aa} k{\"a}lltermen i h{\"o}gerledet endast utg{\"o}rs av den
{\it fria} laddningst{\"a}theten $\rho_{\rm f}$,
\sidx{Fri laddningst{\"a}thet $\rho_{\rm f}$}
\sidx{Laddningst{\"a}thet}[Fri, $\rho_{\rm f}$]
och om vi {\it definierar} den elektriska fl{\"o}dest{\"a}theten som
$$
  {\bf D}\equiv\varepsilon_0{\bf E}+{\bf P},
$$
s{\aa} lyder allts{\aa} denna under Gauss lag
\sidx{Gauss lag}[F{\"o}r elektrisk fl{\"o}dest{\"a}thet]
$$
  \nabla\cdot{\bf D}=\rho_{\rm f}.
$$
Eftersom denna nu innefattar {\"a}ven den elektriska polarisationsdensiteten
\idx{Elektrisk polarisationsdensitet} hos materialet, med andra ord
v{\"a}xelverkan mellan det elektriska f{\"a}ltet och mediet, s{\aa} {\"a}r
detta en mycket anv{\"a}ndbar generalisering n{\"a}r vi nu skall analysera
gr{\"a}nsytor mellan olika medier.
\vfill\eject

\section{Randvillkor och {\"o}verg{\aa}ngar mellan olika media}
\sidx{Randvillkor}\sidx{Hoppvillkor}
\sidx{Dielektrikum}[{\"O}verg{\aa}ng mellan olika]
Vid {\"o}verg{\aa}ngar mellan olika media blir {\"a}ndringarna p{\aa}
${\bf E}$- och ${\bf D}$-f{\"a}ltens komposanter olika bero\-ende p{\aa} om
de {\"a}r  normala (vinkelr{\"a}ta) eller tangentiala (parallella) mot
gr{\"a}nsytan mellan media.\numberedfootnote{Se Griffiths s.~185.}

\subsection{Recap p{\aa} hoppvillkor f{\"o}r elektriska f{\"a}lt i vakuum}
\sidx{Hoppvillkor}[F{\"o}r elektriska f{\"a}lt i vakuum]
F{\"o}rst av allt en liten recap p{\aa} randvillkor (``hoppvillkor'') {\"o}ver
gr{\"a}nsytor som uppb{\"a}r en ytladdning $\sigma$!
\epsfig{../lect-06/figs/esurfnorm.1}\noindent
Om vi betraktar en yta med en ytladdningst{\"a}thet
$\sigma$ (${\rm C}/{\rm m}^2$), och innesl{\aa}ter denna yta med en sluten
yta\numberedfootnote{Jackson's {\it Classical Electrodynamics} kallar denna
f{\"o}r ``pillbox'' medan Griffits anv{\"a}nder det fyndiga och tr{\"a}ffande
  ``Gaussian pillbox''.}
$\Omega$ av h{\"o}jd $h\to0$, som har platta ytor ovanf{\"o}r och under,
s{\aa} har vi fr{\aa}n Gauss teorem i elektrostatik att normal-komposanterna
($\perp$) f{\"a}ltet ${\bf E}_1$ under ytan och f{\"a}ltet ${\bf E}_2$ ovan
ytan {\"a}r relaterade genom:\numberedfootnote{Sj{\"a}lvfallet med
  ytladdningsdensiteten som $\sigma=Q/A$ (${\rm C}/{\rm m}^2$).}\sidx{Gauss
  lag}[F{\"o}r elektrisk f{\"a}ltstyrka]
$$
  \hbox{Gauss:}\qquad
  \displaystyle
  \iiint_V\nabla\cdot{\bf E}\,dV=
  \oiint_{\Omega}{\bf E}\cdot d{\bf S}={{Q}\over{\varepsilon_0}}
  \qquad\to\qquad
  {\bf e}_n \cdot({\bf E}_{2}-{\bf E}_{1})A
    ={{Q}\over{\varepsilon_0}}
  \quad\Leftrightarrow\quad
  E^{\perp}_2-E^{\perp}_1={{\sigma}\over{\varepsilon_0}}
$$
\epsfig{../lect-06/figs/esurftang.1}\noindent
P{\aa} liknande s{\"a}tt kan vi analysera tangentiella komposanter
($\parallel$), parallella med ytan, genom att ist{\"a}llet anv{\"a}nda Stokes
teorem i en sluten rektangul{\"a}r slinga $\Sigma$ av h{\"o}jd $h\to0$ och
som innefattar gr{\"a}nsytan:\sidx{Stokes teorem}
$$
  \hbox{Stokes:}\qquad
  \iint_{\Sigma}\underbrace{(\nabla\times{\bf E})}_{=-\partial{\bf B}/\partial t=0}
      \cdot d{\bf S}
    =\oint_{\Sigma}{\bf E}\cdot d{\bf s}
    =0
  \qquad\to\qquad
  {\bf e}_n \times({\bf E}_{2}-{\bf E}_{1})={\bf 0}
  \quad\Leftrightarrow\quad
  E^{\parallel}_2=E^{\parallel}_1
$$
\vfill\eject

\subsection{Hoppvillkor f{\"o}r elektriska f{\"a}lt mellan olika media}
%\sidx{Hoppvillkor}[F{\"o}r elektriska f{\"a}lt mellan olika media]
Vi kan nu anv{\"a}nda en analogi till ovanst{\aa}ende f{\"o}r att analysera vad
som h{\"a}nder d{\aa} materialen under och ovanf{\"o}r gr{\"a}nsytan har olika
relativ elektrisk permittivitet. \sidx{Elektrisk fl{\"o}dest{\"a}thet}
F{\"o}r den elektriska fl{\"o}dest{\"a}theten har \idx{Gauss lag} (utifr{\aa}n
$\nabla\cdot{\bf D}=\rho_{\rm f}$) samma form som f{\"o}r det elektriska
f{\"a}ltet, med skillnaden att vi nu har den {\it fria} elektriska
laddningst{\"a}theten $\rho_{\rm f}$ som en k{\"a}llterm i h{\"o}gerledet.
Med exakt samma geometri som tidigare f{\"o}ljer d{\"a}rmed att vi
f{\aa}r\numberedfootnote{{\AA}terigen, sj{\"a}lvfallet med den {\it fria}
  ytladdningsdensiteten som $\sigma_{\rm f}=Q_{\rm f}/A$ (${\rm C}/{\rm m}^2$).
  \sidx{Laddningst{\"a}thet}[Yt-]\sidx{Ytladdningst{\"a}thet}}
$$
  \hbox{Gauss:}\qquad
  \displaystyle
  \iiint_V\nabla\cdot{\bf D}\,dV=
  \oiint_{\Omega}{\bf D}\cdot d{\bf S} = Q_{\rm f}
  \qquad\to\qquad
  {\bf e}_n \cdot({\bf D}_{2}-{\bf D}_{1})A = Q_{\rm f}
  \quad\Leftrightarrow\quad
  D^{\perp}_{2}-D^{\perp}_{1} = \sigma_{\rm f},
$$
d{\"a}r $\sigma_{\rm f}$ {\"a}r den {\it fria} elektriska
ytladdningst{\"a}theten. Med andra ord, s{\aa} tar den normalkomposanten av
den elektriska fl{\"o}dest{\"a}theten ${\bf D}$ ett hopp motsvarande den fria
ytladdningst{\"a}theten $\sigma_{\rm f}$ d{\aa} vi passerar gr{\"a}nsytan.

Utifr{\aa}n detta resultat {\"a}r det mycket l{\"a}tt att f{\"o}rledas att tro
att ett ``universalrecept'' f{\"o}r att behandla elektrostatik i ett medium:
\medskip
\hbox{\hskip60pt ``Vi beh{\"o}ver ju bara byta ut ${\bf E}$ mot ${\bf D}$ och
  ist{\"a}llet anv{\"a}nda de {\it fria}}
\hbox{\hskip60pt laddningarna, s{\aa} har vi ju precis samma resultat.
  Busenkelt!''}
\medskip
\noindent
Detta antagande {\"a}r dock falskt, vilket vi kan se genom att exempelvis
betrakta rotationen av den elektriska fl{\"o}dest{\"a}theten,
$$
  \nabla\times{\bf D}
    =\varepsilon_0\underbrace{\nabla\times{\bf E}}_{=0}+\nabla\times{\bf P}
    =\nabla\times{\bf P},
$$
vilken generellt {\it inte} alltid {\"a}r identiskt
noll.\numberedfootnote{Griffiths har en bra genombelysning av detta i
  kapitel 4.3.2, sidan~184.}
Om vi anv{\"a}nder Stokes teorem p{\aa} $\nabla\times{\bf P}$ s{\aa}
erh{\aa}ller vi ist{\"a}llet
$$
  \eqalign{
  \hbox{Stokes:}\qquad
  \iint_{\Sigma}\underbrace{(\nabla\times{\bf D})}_{\ne0}\cdot d{\bf S}
    =\oint_{\Sigma}{\bf D}\cdot d{\bf s}
    =\oint_{\Sigma}{\bf P}\cdot d{\bf s}
  \quad\to\quad&
  {\bf e}_n \times({\bf D}_{2}-{\bf D}_{1})=
  {\bf e}_n \times({\bf P}_{2}-{\bf P}_{1})\cr
  &\quad\Leftrightarrow\quad
  D^{\parallel}_2-D^{\parallel}_1 = P^{\parallel}_2-P^{\parallel}_1\cr
  }
$$
Om vi har tv{\aa} medier av olika relativ permittivitet
$\varepsilon^{(1)}_{\rm r}=1+\chi^{(1)}_{\rm e}$ och
$\varepsilon^{(2)}_{\rm r}=1+\chi^{(2)}_{\rm e}$,
s{\aa} kan vi sammanfatta detta med att
$$
  \underbrace{
    \varepsilon_0(1+\chi^{(2)}_{\rm e})E^{\parallel}_2
      -\varepsilon_0(1+\chi^{(2)}_{\rm e})E^{\parallel}_2
  }_{=D^{\parallel}_2-D^{\parallel}_1}
  =
  \underbrace{
    \varepsilon_0\chi^{(2)}_{\rm e} E^{\parallel}_2
      -\varepsilon_0\chi^{(2)}_{\rm e} E^{\parallel}_2
  }_{=P^{\parallel}_2-P^{\parallel}_1}
  \qquad\Leftrightarrow\qquad
  E^{\parallel}_2=E^{\parallel}_1,
$$
alternativt uttryckt i ${\bf D}$-f{\"a}ltet och de relativa permittiviteterna
som \sidx{Elektrisk permittivitet}[Relativ permittivitet $\varepsilon_{\rm r}$]
$$
  {{D^{\parallel}_2}\over{\varepsilon^{(2)}_{\rm r}}}
    ={{D^{\parallel}_1}\over{\varepsilon^{(1)}_{\rm r}}}
$$
\vfill\eject

\section{Upplagrad elektrisk energi}
\sidx{Upplagrad energi}[I elektriskt f{\"a}lt]
D{\aa} vi applicerar ett elektriskt f{\"a}lt {\"o}ver ett dielektrikun, s{\aa}
kommer vi att lagra upp energi i det system av mikrosopiska
(atom{\"a}ra/molekyl{\"a}ra) dipoler som d{\"a}rmed kommer att linjeras upp i
det externt p{\aa}lagda f{\"a}ltet. Som an analogi kan vi se detta som
motsvarigheten till en upplagring av mekanisk energi i ett distribuerat system
av mycket sm{\aa} men m{\aa}nga fj{\"a}drar. Det som h{\"a}r tillkommer {\"a}r
att vi dessutom har fria laddningar som kommer att justera position
allteftersom j{\"a}mvikt i systemet uppn{\aa}s.

Uttrycket f{\"o}r den upplagrade energin (i Joule) hos ett dielektrikun under
ett externt p{\aa}lagt elektriskt f{\"a}lt ${\bf E}$ ges av
$$
  W = {{1}\over{2}}\iiint{\bf D}\cdot{\bf E}\,dV
$$
H{\"a}rledningen av denna form ges av Griffiths p{\aa} sidorna 197--198, och
{\"a}r en nyttig {\"o}vningsuppgift att f{\"o}lja.

\section{Dielektrisk sf{\"a}r i elektriskt f{\"a}lt}
\sidx{Dielektrisk sf{\"a}r}
Vi har just sett hur en extremt f{\"o}renklad modell av en atom kan anv{\"a}ndas
f{\"o}r att kvalitativt beskriva uppkomsten av dipolmoment d{\aa} systemet
uts{\"a}tts f{\"o}r ett externt p{\aa}lagt elektriskt f{\"a}lt. Om vi lyfter
blicken en aning, s{\aa} {\"a}r ett annat intressant objekt en sf{\"a}r
best{\aa}ende av ett homogent och isotropt medium, f{\"o}r enkelhets skull
placerad i ett medium med $\varepsilon_{\rm e}=1$ (h{\"a}r antar vi att ``luft
$\approx$ vakuum''). F{\"o}rutom att vara en mycket anv{\"a}ndbar modell
f{\"o}r spridning av elektromagnetiska f{\"a}lt, s{\aa} utg{\"o}r detta enkla
system {\"a}ven en intressant exercis i partiella differentialekvationer och
elektrostatik.

Om en dielektrisk sf{\"a}r av radie $R$ och relativ permittivitet
(dielektricitetskonstant) $\varepsilon_{\rm r}$ placeras i ett elektriskt
f{\"a}lt ${\bf E}$, s{\aa} kommer dipolmomenten i sf{\"a}ren att addera upp
till att ge sf{\"a}ren i sig ett netto-dipolmoment ${\bf p}$. Man kan visa att
detta dipolmoment ges som\numberedfootnote{F{\"o}r en h{\"a}rledning av det
  elektriska f{\"a}ltet inuti en dielektrisk sf{\"a}r, se Griffiths Exempel~4.7
  p{\aa} sidan 193 (vilket f{\"o}r {\"o}vrigt {\"a}r ett mycket bra exempel);
  alternativt Jackson, s.~151. Se {\"a}ven Griffiths, Problem~4.41, sidan 208
  f{\"o}r ett exempel p{\aa} relationen mellan atom{\"a}r polarisabilitet
  $\alpha$ och relativ permittivitet $\varepsilon_{\rm r}$.
  Clausius--Mossotti-relationen g{\"a}ller generellt f{\"o}r icke-pol{\"a}ra
  media, medan uttryck enligt Langevin g{\"a}ller f{\"o}r pol{\"a}ra media.}
$$
  {\bf p}=\underbrace{4\pi\varepsilon_0
    \bigg({{\varepsilon_{\rm r}-1}\over{\varepsilon_{\rm r}+2}}\bigg)
      R^3}_{=``\alpha''}{\bf E}_{\infty}
$$
d{\"a}r ${\bf E}_{\infty}$ {\"a}r det elektriska f{\"a}ltet l{\aa}ngt ifr{\aa}n
sj{\"a}lva sf{\"a}ren. Faktorn $(\varepsilon_{\rm r}-1)/(\varepsilon_{\rm r}+1)$
kallas {\it Clausius--Mossotti}-relationen\numberedfootnote{Efter
Ottaviano-Fabrizio Mossotti (1791--1863) \sidx{Mossotti, Ottaviano-Fabrizio
(1791--1863)} och Rudolf Clausius (1822--1888).\sidx{Clausius, Rudolf
(1822--1888)}}.
Ur denna relation har vi att den effektiva polarisabiliteten
\sidx{Atom{\"a}r polarisabilitet $\alpha$} hos sf{\"a}ren
i sig ges som\sidx{Clausius--Mossotti-relationen}
$$
  \alpha=4\pi\varepsilon_0
    \bigg({{\varepsilon_{\rm r}-1}\over{\varepsilon_{\rm r}+2}}\bigg) R^3
$$
\vfill\eject

\section{Sammanfattning av F{\"o}rel{\"a}sning~6
  -- Elektriska f\"alt i material}
\item{$\bullet$}{Den elektriska polarisationsdensiteten ${\bf P}$ beskriver
  det elektriska dipolmomentet per volymenhet. I en f{\"o}renklad modell av
  v{\"a}xelverkan mellan f{\"a}lt och materia kan vi se dipolmomentet som
  linj{\"a}rt beroende av det p{\aa}lagda elektriska f{\"a}ltet, med resultatet
  $$
    {\bf P} \equiv \Big\langle{{d{\bf p}}\over{dV}}\Big\rangle
      = \varepsilon_0\chi_{\rm e}{\bf E},
  $$
  d{\"a}r $\chi_{\rm e}$ {\"a}r den {\it elektriska susceptibiliteten}
  f{\"o}r materialet och d{\"a}r $\langle\ldots\rangle$ betecknar
  medelv{\"a}rdes\-bildning.}
\item{$\bullet$}{Den elektriska fl{\"o}dest{\"a}theten ({\it electric
  displacement field}) definieras som
  $$
      {\bf D}
        \equiv\varepsilon_0{\bf E}+{\bf P}
        =\varepsilon_0(1+\chi_{\rm e}){\bf E}
        \equiv\varepsilon_0\varepsilon_{\rm r}{\bf E},
  $$
  d{\"a}r $\varepsilon_{\rm r}\equiv 1+\chi_{\rm e}$ {\"a}r den {\it relativa
  elektriska permittiviteten}.}
\item{$\bullet$}{Gauss lag f{\"o}r den elektriska fl{\"o}dest{\"a}theten lyder
  $$
    \nabla\cdot{\bf D}=\rho_{\rm f},
  $$
  d{\"a}r $\rho_{\rm f}$ {\"a}r den {\it fria laddningst{\"a}theten}.}
\item{$\bullet$}{Recap p{\aa} hoppvillkor f{\"o}r elektriska f{\"a}lt i vakuum.
  Normalkomponenter av det elektriska f{\"a}ltet lyder
  $$
    E^{\perp}_2-E^{\perp}_1={{\sigma}\over{\varepsilon_0}}
  $$
  {\"o}ver en yta med ytladdningst{\"a}theten som $\sigma$, medan
  tangentialkomponenter lyder
  $$
    E^{\parallel}_2=E^{\parallel}_1.
  $$}
\item{$\bullet$}{Hoppvillkor f{\"o}r elektriska f{\"a}lt mellan olika
  dielektrika. Normalkomponenter av det elektriska f{\"a}ltet lyder
  $$
    D^{\perp}_{2}-D^{\perp}_{1} = \sigma_{\rm f},
  $$
  d{\"a}r $\sigma_{\rm f}$ {\"a}r den {\it fria} elektriska
  ytladdningst{\"a}theten, medan tangentialkomponenter lyder
  $$
    D^{\parallel}_2-D^{\parallel}_1 = P^{\parallel}_2-P^{\parallel}_1.
  $$
  Om vi har tv{\aa} medier av olika relativ permittivitet
  $\varepsilon^{(1)}_{\rm r}$ och $\varepsilon^{(2)}_{\rm r}$, s{\aa} har vi
  $$
    E^{\parallel}_2=E^{\parallel}_1
    \qquad\Leftrightarrow\qquad
    {{D^{\parallel}_2}/{\varepsilon^{(2)}_{\rm r}}}
      ={{D^{\parallel}_1}/{\varepsilon^{(1)}_{\rm r}}}.
  $$}
\item{$\bullet$}{Den upplagrade energin (Joule) hos ett dielektrikum i ett
  externt p{\aa}lagt elektriskt f{\"a}lt ${\bf E}$ ges av
  $$
    W = {{1}\over{2}}\iiint{\bf D}\cdot{\bf E}\,dV.
  $$}

\cleardoublepage
%%% End of auto-extracted text from ../lect-06/lecture-06.tex %%%
%%% Begin of auto-extracted text from ../lect-07/lecture-07.tex %%%
%
% File: teach/elmagii/lect-07/lecture-07.tex [plain TeX code]
% Github: https://github.com/elmagii/lect-07/
% Last change: November 19, 2025
%
% Lecture No 7 in the course ``Elektromagnetism II, 1TE626 (2025)'',
% held November 19, 2025, at Uppsala University, Sweden.
%
% Copyright (C) 2022-2025, Fredrik Jonsson, under Gnu General Public
% License (GPL) v3. See the enclosed LICENSE for details.
%
% This program is free software: you can redistribute it and/or modify
% it under the terms of the GNU General Public License as published by
% the Free Software Foundation, either version 3 of the License, or
% (at your option) any later version.
%
% This program is distributed in the hope that it will be useful,
% but WITHOUT ANY WARRANTY; without even the implied warranty of
% MERCHANTABILITY or FITNESS FOR A PARTICULAR PURPOSE.  See the
% GNU General Public License for more details.
%
% You should have received a copy of the GNU General Public License
% along with this program.  If not, see <https://www.gnu.org/licenses/>.
%
\def\coursename{Elektromagnetism II}
\def\coursecode{1TE626}
\def\courseyear{2025}
\def\courserepo{https://github.com/hp35/elmagii/}
\def\lecturenumber{7}
\def\lecturetitle{Magnetiska f{\"a}lt i material}
\def\lecturesubtitle{}
\def\lectureauthor{Fredrik Jonsson}
\def\lectureplace{Uppsala Universitet}
\def\lecturedate{19 november 2025}
%-------------------- BEGIN OF LOCAL MACROS --------------------
\edef\expandedlecturenumber{7}
\def\ifempty#1{\ifx\relax#1\relax}
\advance\chapno by 1
\secno=0
\footnotenumber=0
\message{==================== Lecture 7 ====================}
\writenumberedtocentry{chapter}{F{\"o}rel{\"a}sning 7 -- {Magnetiska f{\"a}lt i material}}{\thechapno}
\hsize=150mm\hoffset=4.6mm\vsize=230mm\voffset=7mm
\topskip=0pt\baselineskip=12pt\parskip=0pt\leftskip=0pt\parindent=15pt
\ifcolors
  \voffset=-10.2mm\topskip=0pt
\fi
\headline={\ifnum\secno>0\ifodd\pageno\rightheadline\else\leftheadline\fi
  \else\hfill\fi}
\def\rightheadline{\tenrm{\it F\"orel\"asning 7}
  \hfil{\it \coursename, \coursecode\ (\courseyear)}}
\def\leftheadline{\tenrm{\it \coursename, \coursecode\ (\courseyear)}
  \hfil{\it F\"orel\"asning 7}}
\noindent~\vskip-60pt\hskip-40pt{\epsfbox{../lect-01/macros/UU_logo_color.eps}}
\vskip-42pt\hfill\vbox{
    \hbox{{\it \coursename, \coursecode\ (\courseyear)}}
    \hbox{{\it Lecture Notes, \lectureauthor}}
    \hbox{{\it Document Revision \today}}
    \hbox{{\it \courserepo}}}\vskip 36pt
\centerline{\twelvesc F\"orel\"asning 7}
\vskip 24pt\noindent
\centerline{\twelvesc{Magnetiska f{\"a}lt i material}}
\expandafter\ifempty\expandafter{\lecturesubtitle}%
  \else\centerline{\twelvesc\lecturesubtitle}\fi
\bigskip
\centerline{\lectureauthor, \lectureplace, \lecturedate}
\vskip24pt
%--------------------- END OF LOCAL MACROS ---------------------



\plan{I denna f{\"o}rel{\"a}sning analyserar vi vad som h{\"a}nder d{\aa}
  det magnetiska spinnet hos material linjeras och magnetiserar materialet,
  antingen genom ett externt p{\aa}lagt magnetiskt f{\"a}lt eller genom att
  spinnen {\"a}r naturligt linjerade i materialet, i s{\aa} kallade
  permanentmagneter.

  Krafter och moment p{\aa} magnetiska dipoler extraheras fr{\aa}n upplagrade
  energier f{\"o}r dipoler i elektriska eller magnetiska f{\"a}lt, och vi
  introducerar magnetisering som en ``magnetisk polarisationsdensitet'', i
  analogi med den elektriska motsvarigheten.
  Vi diskuterar {\"o}versiktligt den upplagrade energin i ett magnetiserat
  medium och effekten av hysteres som effekt av ett varierande externt
  p{\aa}lagt magnetf{\"a}lt.

  Slutligen g{\aa}r vi igenom hur vektorpotentialen ${\bf A}({\bf x},t)$
  uttrycks f{\"o}r ett objekt med godtycklig magnetisering
  ${\bf M}({\bf x},t)$, och finner att uttrycket f{\"o}r vektorpotentialen
  {\"a}r ekvivalent med om det magnetiserade objektet ist{\"a}llet hade
  varit konstruerat som ekvivalenta volyms- och ytstr{\"o}mmar
  ${\bf J}_{\rm b}$ och ${\bf K}_{\rm b}$ av {\it bundna laddningar}.}

\threepointsummary{%
  Magnetisering
  $$
    {\bf M} \equiv \Big\langle{{d{\bf m}}\over{dV}}\Big\rangle
      = {{1}\over{\mu_0}}\Big(1-{{1}\over{\mu_{\rm r}}}\Big){\bf B}
  $$
  och magnetiseringsstyrkan
  $$
    {\bf H}\equiv{{\bf B}\over{\mu_0}}-{\bf M}
      ={{\bf B}\over{\mu_0\mu_{\rm r}}}
      \qquad\Leftrightarrow\qquad
      {\bf B}=\mu_0\mu_{\rm r}{\bf H}.
  $$
}{%
  Vektorpotential ${\bf A}({\bf x})$ fr{\aa}n magnetisk dipol
  $$
    {\bf A}({\bf x})={{\mu_0}\over{4\pi}}
      {{{\bf m}\times({\bf x}-{\bf x}')}\over{|{\bf x}-{\bf x}'|^3}}.
  $$
}{%
  Vektorpotentialen ${\bf A}({\bf x})$, och d{\"a}rmed {\"a}ven det
  magnetiska f{\"a}ltet ${\bf B}({\bf x})=\nabla\times{\bf A}({\bf x})$,
  fr{\aa}n ett magnetiserat objekt g{\aa}r ej att s{\"a}rskilja fr{\aa}n
  fallet d{\"a}r objektet ist{\"a}llet hade uppburit l{\"a}mpligt
  konstruerade volyms- och en ytstr{\"o}mmar ${\bf J}_{\rm b}$ och
  ${\bf K}_{\rm b}$. F{\"o}r en observat{\"o}r kan vi alltid beskriva
  f{\"a}ltet fr{\aa}n ett magnetiserat objekt i termer av ekvivalenta
  str{\"o}mmar.
}
\vfill\eject\copyrights

\section{Dipolmodellen av magneter}
\idx{Dipolmodellen av magneter}
L{\aa}t oss anta att det fanns en magnetisk motsvarighet till elektrisk
laddning, motsvarande magnetiska monopoler. Fr{\aa}n $\nabla\cdot{\bf B}=0$
f{\"o}ljer det direkt via Gauss lag att om vi skulle f{\"o}rs{\"o}ka isolera
en s{\aa}dan magnetisk laddning $Q_{\rm m}$, s{\aa} skulle vi ha att
$$
  \iiint_V\underbrace{\nabla\cdot{\bf B}}_{=0}\,dV=Q_{\rm m}=0,
$$
det vill s{\"a}ga att den enda m{\"o}jligheten {\"a}r att laddningen {\"a}r
just noll. {\it Med andra ord s{\aa} existerar inga magnetiska monopoler.}
\sidx{Magnetiska monopoler}[Icke-existens av]
Trots detta, s{\aa} kan vi {\"a}nd{\aa} t{\"a}nka oss magnetisering som
best{\aa}ende av en magnetisk ``nord-monopol'' och en ``syd-monopol''.
Denna modell betecknas med ``Gilberts modell''.
\sidx{Elektrisk dipol ${\bf p}$}\sidx{Gilberts modell f{\"o}r magneter}
\epsfig{../lect-07/figs/magdipole.1}\noindent
En mer fysikalisk modell {\"a}r dock att ans{\"a}tta varje magnetisk dipol som
best{\aa}ende av en str{\"o}mslinga uppb{\"a}rande en str{\"o}m, skapandes en
inducerad magnetisk dipol. Denna modell kan vi kalla ``Amp\`ere-modellen'',
\sidx{Amp\`eres modell f{\"o}r magneter} d{\aa} dipolmomentet i den direkt
f{\"o}ljer av Amp\`eres klassiska lag.
I Amp\`eres modell {\"a}r det magnetiska moment som genereras kort och
gott\numberedfootnote{Notera att vi i denna f{\"o}rel{\"a}sningsserie
  genomg{\aa}ende anv{\"a}nder notationen ``$dS$'' eller ``$d{\bf S}$''
  f{\"o}r {\it ytelement}; de ``$d{\bf A}$'' som vi strax kommer att
  anv{\"a}nda i andra halvan av denna f{\"o}rel{\"a}sning {\"a}r bidrag till
  {\it vektorpotentialen} ${\bf A}$. Just denna risk f{\"o}r
  f{\"o}rv{\"a}xling mellan yta $A$ och vektorpotential ${\bf A}$
  {\"a}r sj{\"a}lva anledningen till att vi genomg{\aa}ende h{\aa}ller
  oss just till ``$dS$'' eller ``$d{\bf S}$'' f{\"o}r ytelement!}
$$
  {\bf m}=I\iint_A\,d{\bf S}=IA{\bf e}_{\bf n},
$$
d{\"a}r $A$ {\"a}r arean som innesluts av den str{\"o}mb{\"o}rande loopen, med
riktningen ${\bf e}_n$ normal mot loopens plan. F{\"o}r att sammanfatta, s{\aa}
f{\"o}ljer magnetism inte av n{\aa}gra magnetiska monopoler, utan kommer
fr{\aa}n {\it r{\"o}relse av elektrisk laddning}.
\sidx{Elektrisk laddning}[I r{\"o}relse]
\vfill\eject
\section{Krafter och moment p{\aa}\ dipoler}
\sidx{Elektrisk dipol ${\bf p}$}[Upplagrad energi $W_{\rm e}$]
\sidx{Elektrisk dipol ${\bf p}$}[Kraft ${\bf F}_{\rm e}$ verkande p{\aa}]
\sidx{Elektrisk dipol ${\bf p}$}[Vridmoment $\tau_{\rm e}$ verkande p{\aa}]
\sidx{Magnetisk dipol ${\bf m}$}[Upplagrad energi $W_{\rm m}$]
\sidx{Magnetisk dipol ${\bf m}$}[Kraft ${\bf F}_{\rm m}$ verkande p{\aa}]
\sidx{Magnetisk dipol ${\bf m}$}[Vridmoment $\tau_{\rm m}$ verkande p{\aa}]
Elektriska och magnetiska dipoler i externt p{\aa}lagda elektriska respektive
magnetiska f{\"a}lt upplagrar energierna
$$
  W_{\rm e} = -{\bf p}\cdot{\bf E}_{\rm ext},\qquad\qquad
  W_{\rm m} = -{\bf m}\cdot{\bf B}_{\rm ext}
$$
J{\"a}mf{\"o}r med elektrisk dipol placerad i f{\"a}lt med sin positiva
laddning n{\"a}rmst k{\"a}llan f{\"o}r elektriska f{\"a}ltet (med linjer
fr{\aa}n ``$+$''). Den elektriska dipolen str{\"a}var efter att linjera sig
(med axeln pekande fr{\aa}n negative till positiv laddning) l{\"a}ngs med
det externa elektriska f{\"a}ltet.

Kraften p{\aa} den elektriska respektiva magnetiska dipolen ges d{\aa} av
gradienten av den upplagrade energin, som (observera tecken p{\aa} $\nabla$!)%
\numberedfootnote{Fys.~dimension:
$[\nabla({\bf p}\cdot{\bf E}_{\rm ext})]={\rm m}^{-1}{\rm C}{\rm m}{\rm V/m}
={\rm J/m}={\rm N}$;
$[\nabla({\bf m}\cdot{\bf B}_{\rm ext})]={\rm m}^{-1}{\rm A}{\rm m}^2{\rm N/Am}
={\rm N}$.}
$$
  {\bf F}_{\rm e} = -\nabla W_{\rm e}
    =\nabla({\bf p}\cdot{\bf E}_{\rm ext}),\qquad\qquad
  {\bf F}_{\rm m} = -\nabla W_{\rm m}
    =\nabla({\bf m}\cdot{\bf B}_{\rm ext}).
$$
Notera att f{\"o}r att ha en kraft p{\aa} en elektrisk dipol som saknar
netto-laddning, s{\aa} kr{\"a}vs det att det p{\aa}lagda elektriska f{\"a}ltet
${\bf E}$ har en gradient vid dipolen ${\bf p}$. Omv{\"a}nt, om det p{\aa}lagda
elektriska f{\"a}ltet {\"a}r homogent (utan gradienter), s{\aa} m{\aa}ste
dipolen ha en nettoladdning f{\"o}r att kunna p{\aa}verkas av en kraft.

F{\"o}r det magnetiska fallet har vi att det inte existerar n{\aa}gon magnetisk
nettoladdning (eftersom $\nabla\cdot{\bf B}=0$ alltid g{\"a}ller;
j{\"a}mf{\"o}r med $\nabla\cdot{\bf E}=\rho/\varepsilon_0$), s{\aa} d{\"a}r
blir kravet att det magnetiska f{\"a}ltet ${\bf B}$ ovillkorligen m{\aa}ste ha
en gradient f{\"o}r att kunna ut{\"o}va en nettokraft p{\aa} den magnetiska
dipolen ${\bf m}$.

Vridmomentet $\tau$ (SI-enhet ${\rm N}{\rm m}$) p{\aa} en elektrisk respektive
magnetis dipol i externt p{\aa}lagda f{\"a}lt {\"a}r
$$
  {\bf \tau}_{\rm e}={\bf p}\times{\bf E}_{\rm ext},\qquad\qquad
  {\bf \tau}_{\rm m}={\bf m}\times{\bf B}_{\rm ext}.
$$
Notera att vridmomenten blir noll d{\aa} dipolerna {\"a}r parallella med
f{\"a}lten (${\bf e}\times{\bf e}\equiv{\bf 0}$); dock {\"a}r endast fallen
d{\aa} dipolerna dessutom pekar i samma riktning som f{\"a}lten stabila (detta
f{\"o}ljer fr{\aa}n att energierna $W_{\rm e}$ och $W_{\rm m}$ d{\aa} minimeras;
att dessutom visa att energimaximum {\"a}r instabilt {\"a}r enkelt att visa
genom att uttrycka skal{\"a}rprodukten som en faktor $\cos\theta$ i sf{\"a}riska
koordinater och analysera andraderivatan av energierna med avseende p{\aa}
$\theta$).

\section{Magnetisering - ``magnetisk polarisationsdensitet''}
\idx{Magnetisering ${\bf M}$}
\sidx{Magnetisering ${\bf M}$}[``Magnetisk polarisationsdensitet'']
\sidx{Dipolmoment}[Magnetiskt]
\sidx{Elektrisk polarisationsdensitet ${\bf P}$}
Liksom f{\"o}r den elektriska polarisationsdensiteten ${\bf P}$ f{\"o}ljer att
ett material som regel linjerar sina mikroskopiska magnetiska dipoler
(molekyl{\"a}ra spinn) efter ett externt p{\aa}lagt magnetiskt f{\"a}lt.
{\"a}ven om man kan t{\"a}nka sig en bild av detta som en dipol best{\aa}ende
av en moln av positiva och negativa ``magnetiska laddningar'', s{\aa} {\"a}r
detta en felaktig bild d{\aa} det inte existerar magnetiska laddningar
({\aa}terigen, fr{\aa}n $\nabla\cdot{\bf B}=0$). Ist{\"a}llet kan vi
f{\"o}rest{\"a}lla oss detta som en ensemble av mikroskopiska
str{\"o}mb{\"a}rande slutna slingor som var och en bidrar med ett magnetiskt
dipolmoment.\sidx{Str{\"o}mslinga}[Sluten]
\epsfig{../lect-07/figs/magdensity.1}\noindent
Vi beskriver detta som en magnetisering ${\bf M}$ av magnetiska dipolmoment
per volymenhet (d{\"a}rav en slags ``magnetisk polarisationsdensitet'').
Som en f{\"o}renklad modell kan vi se denna inducerade magnetisering
som linj{\"a}rt beroende av det p{\aa}lagda magnetiska f{\"a}ltet,
$$
  {\bf M} \equiv \Big\langle{{d{\bf m}}\over{dV}}\Big\rangle
    = {{1}\over{\mu_0}}\Big(1-{{1}\over{\mu_{\rm r}}}\Big){\bf B}
$$
d{\"a}r $\mu_{\rm r}$ {\"a}r den {\it relativa magnetiska permeabiliteten}
\sidx{Magnetisk permeabilitet}[Relativ permeabilitet $\mu_{\rm r}$]
f{\"o}r materialet ($\mu_{\rm r}$ {\"a}r dimensionsl{\"o}s).

Liksom f{\"o}r den elektriska polarisationsdensiteten ${\bf P}$ {\"a}r det
inom elektromagnetisk f{\"a}ltteori bekv{\"a}mt att baka ihop det magnetiska
f{\"a}ltet ${\bf B}$ och magnetiseringen ${\bf M}$ till
{\it magnetiseringsstyrkan}\sidx{Magnetisk fl{\"o}dest{\"a}thet ${\bf B}$}
\sidx{Magnetiseringsstyrka ${\bf H}$}
\sidx{Magnetisk permeabilitet}[Vakuumpermeabilitet $\mu_0$]
$$
  {\bf H}\equiv{{\bf B}\over{\mu_0}}-{\bf M}
    ={{\bf B}\over{\mu_0\mu_{\rm r}}},
$$
en konstitutiv relation som {\"a}r lite avigt definierad d{\aa} vi ser ${\bf B}$
som den prim{\"a}ra beskrivningen av magnetf{\"a}ltet. Normalt uttrycker vi inom
elektromagnetisk f{\"a}ltteori denna p{\aa} formen\numberedfootnote{Griffiths
  betecknar $H$-f{\"a}ltet som ``auxiliary field''; se Griffiths sidan~279.}
$$
  {\bf B}=\mu_0\mu_{\rm r}{\bf H},
$$
d{\aa} detta {\"a}r den naturliga tolkningen (${\bf B}$ som funktion av
${\bf H}$ och inte tv{\"a}rtom) d{\aa} vi formulerar de elektromagnetiska
v{\aa}gekvationerna fr{\aa}n Maxwells ekvationer. Dock, om vi utg{\aa}r
fr{\aa}n Faradays induktionslag,\sidx{Faradays lag}
$$
  \nabla\times{\bf E}=-{{\partial{\bf B}}\over{\partial t}},
$$
s{\aa} faller sig paret $[{\bf E},{\bf B}]$ som det mest naturliga att
anv{\"a}nda, trots att ${\bf E}$ r{\"a}knas som en (elektrisk)
{\it f{\"a}ltstyrka} och ${\bf B}$ som en (magnetisk) {\it fl{\"o}dest{\"a}thet}.
\medskip
\item{$\bullet$}{F{\"o}r {\it diamagnetiska} material \sidx{Diamagnetism}
   har vi att $\mu_{\rm r}\le 1$, f{\"o}r vilka magnetiseringen i materialet
   {\"a}r riktat i {\it motsatt riktning} mot det p{\aa}lagda magnetf{\"a}ltet.
   Supraledare {\"a}r exempel p{\aa} perfekta diamagneter, d{\"a}r de helt
   repellerar the p{\aa}lagda f{\"a}ltet fr{\aa}n det interna magnetf{\"a}ltet
   (som i en supraledare {\"a}r noll).}
\item{$\bullet$}{F{\"o}r {\it paramagnetiska} material \sidx{Paramagnetism}
   {\"a}r $\mu_{\rm r}>1$, f{\"o}r vilka magnetiseringen i materialet {\"a}r
   riktat i {\it samma riktning} som det p{\aa}lagda magnetf{\"a}ltet.}
\item{$\bullet$}{I {\it ferromagnetiska} material \sidx{Ferromagnetism}
   {\"a}r de magnetiska momenten (molekyl{\"a}ra spinn) permanent linjerade
   i en huvudsaklig riktning. {\it Curietemperaturen} \idx{Curietemperatur}
   f{\"o}r ett ferromagnetiskt material {\"a}r den temperatur d{\"a}r
   materialet {\"o}verg{\aa}r fr{\aa}n att vara ferromagnetiskt till att
   bli paramagnetiskt.}
\medskip
\noindent
I elektromagnetisk v{\aa}gpropagation och analys av v{\aa}gekvationen kan vi
s{\aa} gott som alltid anta att mediet {\"a}r ickemagnetiskt, med
$\mu_{\rm r}=1$, det vill s{\"a}ga att vi ur magnetisk synpunkt kan betrakta
materialet som om det vore vakuum. (Detta antagande g{\"a}ller dock alltsom
oftast {\it ej} f{\"o}r den elektriska polarisationsdensiteten.)

\section{Magnetisering sedd som bundna yt- och volymstr{\"o}mmar}
\sidx{Magnetisering ${\bf M}$}[Som bundna yt- och volymstr{\"o}mmar]
\sidx{Bunden ytstr{\"o}m ${\bf K}_{\rm b}$}
\sidx{Bunden volymstr{\"o}m ${\bf J}_{\rm b}$}
Amp\`eres modell av magnetiska dipoler som sm{\aa} str{\"o}mslingor
\sidx{Str{\"o}mslinga}[Sluten] har en intressant och generellt giltig
tolkning i ett makroskopiskt perspektiv.
Notera att str{\"o}mmen som i modellen ses generera ett magnetiskt dipolmoment
\sidx{Dipolmoment}[Magnetiskt] {\"a}r {\it bunden}, det vill s{\"a}ga att
laddningarna inte {\"a}r fria att r{\"o}ra sig i mediet, bara i ``sm{\aa}''
cirklar runt den punkt d{\"a}r vi betraktar dipolmomentet. Dessa str{\"o}mmar
kan vi p{\aa} ytan av ett magnetiserat objekt beskriva som bundna
{\it ytstr{\"o}mmar}, betecknade ${\bf K}_{\rm b}$ (enhet ${\rm A}/{\rm m}$),
samt i volymen av objektet som bundna {\it volymstr{\"o}mmar}, betecknade
${\bf J}_{\rm b}$ (enhet ${\rm A}/{\rm m}^2$).
\epsfig{../lect-07/figs/magcurrent.1}\noindent
Yttryckt i de bundna yt- och volymstr{\"o}mmarna ${\bf K}_{\rm b}$ och
${\bf J}_{\rm b}$ kan magnetiseringen lokalt p{\aa} och i mediet uttryckas
som\numberedfootnote{Griffiths Ekv.~(6.13), sid.~275.}
$$
  {\bf J}_{\rm b}=\nabla\times{\bf M}
                 \qquad(\hbox{volymstr{\"o}m},\ {\rm A}/{\rm m}^2),\qquad\qquad
  {\bf K}_{\rm b}={\bf M}\times{\bf e}_n
                 \qquad(\hbox{ytstr{\"o}m},\ {\rm A}/{\rm m}).
$$
d{\"a}r ${\bf e}_n$ {\"a}r normalvektorn ut fr{\aa}n den slutna ytan hos det
magnetiserade objektet.
\vfill\eject

\section{Upplagrad energi i magnetf{\"a}lt}
\sidx{Upplagrad energi}[I magnetf{\"a}lt]\sidx{Hysteres}
I likhet med ett dielektrikum kommer ett p{\aa}lagt magnetiskt f{\"a}lt
${\bf B}$ d{\aa} det magnetiserar mediet att lagra upp energi. Uttrycket
f{\"o}r den upplagrade energin (i Joule) hos ett magnetiskt material under
ett externt p{\aa}lagt magnetiskt f{\"a}lt ${\bf B}$ ges av
$$
  W = {{1}\over{2}}\iiint{\bf B}\cdot{\bf H}\,dV
$$
Vid hysteres\numberedfootnote{Griffiths sid.~291.}, som {\"a}r en vanligt
f{\"o}rekommande effekt i magnetiska material, finns det ett ``minne'' av
historiken hur magnetf{\"a}ltet har varit riktat och med vilket styrka.
N{\"a}r ett externt p{\aa}lagt magnetf{\"a}lt st{\"a}ngs av kan en kvarvarande
magnetisering finnas kvar, och om vi cykliskt varierar det p{\aa}lagda
f{\"a}ltet kommer eftersl{\"a}pningen att inneb{\"a}ra en tr{\"o}ghet d{\"a}r
materialets magnetisering arbetar mot f{\"o}r{\"a}ndringar av det p{\aa}lagda
f{\"a}ltet. Med andra ord s{\aa} utf{\"o}rs vid hysteres ett internt arbete som
leder till att v{\"a}rme utvecklas.
\epsfig{../lect-07/figs/hysteres.1}\noindent
Vi kan se denna utvecklade v{\"a}rme som att vi i hystereskurvan f{\"o}r
materialet t{\"a}cker in en viss ``area'' $B\cdot H$ som har dimensionen
energi per volymsenhet, och ju st{\"o}rre area vi t{\"a}cker in d{\aa} vi
g{\aa}r runt med en kurva i hysteresdiagrammet, desto st{\"o}rre
v{\"a}rmeutveckling.
\vfill\eject

\section{Vektorpotential fr{\aa}n ett magnetiserat objekt}
\sidx{Vektorpotential ${\bf A}$}\sidx{Magnetisering ${\bf M}$}[Statisk]
Vi kommer nu att studera ett objekt som har den rumsberoende statiska
magnetiseringen ${\bf M}({\bf x})$ given. Vi har sedan tidigare att
s{\aa}v{\"a}l elektriska som magnetiska f{\"a}lt kan uttryckas i termer av
skal{\"a}r potential och vektorpotential, d{\"a}r speciellt magnetf{\"a}ltet
${\bf B}({\bf x})$ direkt kan uttryckas i
vektorpotentialen\numberedfootnote{Recap: Existensen av en
   vektorpotential ${\bf A}$ motiveras av ``teoremet om att inga magnetiska
   dipoler existerar'',
   $$
     \nabla\cdot{\bf B}=0
     \qquad\Leftrightarrow\qquad
     {\bf B}=\nabla\times{\bf A},
   $$
   detta eftersom vi har vektoridentiteten $\nabla\cdot(\nabla\times{\bf A})
   \equiv 0$ f{\"o}r {\it godtycklig} vektorfunktion ${\bf A}$.}
${\bf A}({\bf x})$ som
$$
  {\bf B}({\bf x})=\nabla\times{\bf A}({\bf x}).
$$
Det {\"a}r d{\"a}rf{\"o}r av intresse att se om vi utifr{\aa}n en magnetisering
${\bf M}({\bf x})$ kan f{\aa} fram vektorpotentialen ${\bf A}({\bf x})$ eftersom
vi via den kan extrahera {\"o}vriga f{\"a}lt.\numberedfootnote{Vi f{\"o}ljer
  h{\"a}r i huvudsak Griffiths kapitel 6.2.1, sidorna 274--275.}

Som en f{\"o}rsta b{\"o}rjan i den kommande vektoralgebran, s{\aa} kan vi
konstatera att vektorpotentialen i en {\it observationspunkt} ${\bf x}$
\sidx{Observationspunkt ${\bf x}$} fr{\aa}n en enda, isolerad magnetisk dipol
${\bf m}$ \sidx{Dipolmoment}[Magnetiskt] i {\it k{\"a}llpunkten} ${\bf x}'$
\sidx{K{\"a}llpunkt ${\bf x}'$} ges av\numberedfootnote{Se Griffiths
  Ekv.~(5.85), sidan 253.}
$$
  {\bf A}({\bf x})={{\mu_0}\over{4\pi}}
    {{{\bf m}\times({\bf x}-{\bf x}')}\over{|{\bf x}-{\bf x}'|^2}}.
$$
Med denna enkla relation kan vi enkelt g{\aa} vidare med en generalisering om
vi ser sm{\aa} volymelement $dV$ som en ensemble av k{\"a}llpunkter, var och
en uppb{\"a}rande en rumsberoende magnetisering (``magnetisk
polarisationsdensitet'') ${\bf M}({\bf x}')$.\sidx{Magnetisering ${\bf M}$}
\epsfig{../lect-07/figs/vectpot.1}\noindent
Bidraget $d{\bf A}({\bf x})$ till vektorpotentialen vid observationspunkten
${\bf x}$ fr{\aa}n ett enskilt magnetiskt dipolmoment $d{\bf m}$ vid
k{\"a}llpunkten ${\bf x}'$ har tagits fram tidigare, och ges
av\numberedfootnote{{\AA}terigen, notera att vi genomg{\aa}ende anv{\"a}nder
  notationen ``$dS$'' eller ``$d{\bf S}$'' f{\"o}r {\it ytelement}; de
  ``$d{\bf A}$'' som vi just h{\"a}r anv{\"a}nder {\"a}r bidrag till
  {\it vektorpotentialen} ${\bf A}$.}
$$
  d{\bf A}({\bf x})={{\mu_0}\over{4\pi}}
     {{d{\bf m}\times({\bf x}-{\bf x}')}\over{|{\bf x}-{\bf x}'|^3}}.
$$
Varje volymelement $dV'$ av det magnetiserade mediet uppb{\"a}r ett magnetiskt
moment
$$
  d{\bf m}({\bf x}')={\bf M}({\bf x}')dV',
$$
och den totala vektorpotentialen vid observationspunkten ${\bf x}$ ges d{\aa}
genom att summera (integrera) samtliga delbidrag $d{\bf A}({\bf x})$, enligt
$$
  {\bf A}({\bf x}) = \iiint_V d{\bf A}({\bf x})
  ={{\mu_0}\over{4\pi}}
     \iiint_V{{{\bf M}({\bf x}')\times({\bf x}-{\bf x}')}
                       \over{|{\bf x}-{\bf x}'|^3}}\,dV'.
$$
I princip {\"a}r detta uttryck en helt acceptabel slutdestination; dock finns
det ett ``trick'' som vi kan till{\"a}mpa, och som vi {\"a}ven kommer att
anv{\"a}nda senare i kursen f{\"o}r att f{\"o}renkla uttryck f{\"o}r
mutipolutveckling f{\"o}r elektriska f{\"a}lt och skal{\"a}r potential.
Detta ``trick'' g{\aa}r ut p{\aa} att utnyttja relationen\numberedfootnote{Detta
  trick {\"a}r identiskt med det trick vi utnyttjade i s{\aa}v{\"a}l
  F{\"o}rel{\"a}sning~2 (d{\"a}r vi explicit visade p{\aa} formen av den
  skal{\"a}ra potentialen $\phi$ fr{\aa}n en generell laddningsf{\"o}rdelning
  $\rho$) som F{\"o}rel{\"a}sning~4 (d{\"a}r vi visade att
  $\nabla\cdot{\bf B}=0$ generellt, och att det d{\"a}rmed existerar en
  vektorpotential ${\bf A}$ som uppfyller ${\bf B}=\nabla\times{\bf A}$),
  och kan i huvudsak sammanfattas med en kort h{\"a}rledning som\sidx{Tricket
    $\displaystyle\nabla{{1}\over{\char124}{\bf x}-{\bf x}'{\char124}}
    =-{{({\bf x}-{\bf x}')}\over{{\char124}{\bf x}-{\bf x}'{\char124}^3}}$}
  $$
    \eqalign{
      \nabla'{{1}\over{|{\bf x}-{\bf x}'|}}
      &\equiv\bigg(
         {{\partial}\over{\partial x'}},
         {{\partial}\over{\partial y'}},
         {{\partial}\over{\partial z'}}
       \bigg)
       {{1}\over{[(x-x')^2+(y-y')^2+(z-z')^2]^{1/2}}}\cr
      &= -{{1}\over{2}}{{\Big(-2(x-x'),-2(y-y'),-2(z-z')\Big)}
       \over{[(x-x')^2+(y-y')^2+(z-z')^2]^{3/2}}}\cr
      &= {{{\bf x}-{\bf x}'}\over{|{\bf x}-{\bf x}'|^3}}.\cr
    }
  $$}
$$
  \nabla'{{1}\over{|{\bf x}-{\bf x}'|}}
  \equiv
     \bigg(
       {{\partial}\over{\partial x'}},
       {{\partial}\over{\partial y'}},
       {{\partial}\over{\partial z'}}
     \bigg)
     {{1}\over{|{\bf x}-{\bf x}'|}}
  ={{{\bf x}-{\bf x}'}\over{|{\bf x}-{\bf x}'|^3}},
$$
vilket till{\aa}ter oss att formulera vektorpotentialen som
$$
  {\bf A}({\bf x}) = {{\mu_0}\over{4\pi}}\iiint_V
     \bigg[
       {\bf M}({\bf x}')\times\bigg(
         \nabla'{{1}\over{|{\bf x}-{\bf x}'|}}
       \bigg)
     \bigg]
     \,dV'.
$$
Vi kan h{\"a}r till{\"a}mpa partiell integration\numberedfootnote{Av n{\aa}gon
  outgrundlig anledning kallar Griffiths denna f{\"o}r ``integration by parts,
  using product rule 7'' p{\aa} sidan 274. Varf{\"o}r? Jo, den finns p{\aa}
  insidan av p{\"a}rmen som en {\it Product Rule}.}\sidx{Partiell integration}
p{\aa} detta uttryck, genom att observera att rotationen av en produkt mellan
en skal{\"a}r funktion $f$ och vektor ${\bf G}$ ges som
$$
  \eqalign{
    \nabla\times(f{\bf G})=f\nabla&\times{\bf G}-{\bf G}\times(\nabla f)\cr
      &\Updownarrow\cr
    {\bf G}\times(\nabla f) = f\nabla&\times{\bf G} - \nabla\times(f{\bf G})\cr
  }
$$
vilket ger oss att vektorpotentialen kan uttryckas som
$$
  {\bf A}({\bf x}) =
     {{\mu_0}\over{4\pi}}\iiint_V
     {{\big(\nabla'\times{\bf M}({\bf x}')\big)}\over{|{\bf x}-{\bf x}'|}}\,dV'
    -{{\mu_0}\over{4\pi}}
     \underbrace{
       \iiint_V
       \nabla'\times
       \bigg(
         {{{\bf M}({\bf x}')}\over{|{\bf x}-{\bf x}'|}}
       \bigg)
       \,dV'
     }_{\hbox{Stokes! Gauss? (Eller?)}}
$$
\vfill\eject
\noindent
Den sista termen har en form som {\"a}r misst{\"a}nkt lik grunden f{\"o}r
\idx{Stokes teorem}; detta {\"a}r dock en {\aa}terv{\"a}ndsgr{\"a}nd i och
med att vi h{\"a}r har att g{\"o}ra med en {\it volymsintegral} av en
ing{\aa}ende rotation (och inte en ytintegral, som annars {\"a}r integranden
i Stokes teorem). (Med andra ord varken Stokes eller Gauss teorem, i den
bem{\"a}rkelsen vi normalt anv{\"a}nder terminologin i denna kurs.)
Ist{\"a}llet kan vi h{\"a}r anv{\"a}nda en liknande
variant,\numberedfootnote{Denna form finns som ett
   h{\"a}rlednings-problem 1.61 (b) i Griffiths, sidan 56. L{\"o}sningen {\"a}r
   som f{\"o}ljer, enligt Griffiths ledtr{\aa}dar: Ans{\"a}tt f{\"o}rst
   ${\bf v}\to{\bf v}\times{\bf c}$, d{\"a}r ${\bf c}$ {\"a}r en konstant
   godtycklig vektor, och anv{\"a}nd denna form i Gauss teorem:
   $$
      \iiint_V\nabla\cdot({\bf v}\times{\bf c}) dV
        =\oiint_S({\bf v}\times{\bf c})\cdot d{\bf S}.
   $$
   Genom anv{\"a}ndande av produktregeln (``product rule \#6'' i p{\"a}rmen i
   Griffiths),
   $$
     \nabla\cdot({\bf v}\times{\bf c})
       ={\bf c}\cdot(\nabla\times{\bf v})
         -{\bf v}\cdot\underbrace{(\nabla\times{\bf c})}_{=0}
       ={\bf c}\cdot(\nabla\times{\bf v}),
   $$
   samt att vi genom vektoridentiteten (1) i p{\"a}rmen i Griffiths har att
   $$
     d{\bf S}\cdot({\bf v}\times{\bf c})
       = {\bf c}\cdot(d{\bf S}\times{\bf v})
       = -{\bf c}\cdot({\bf v}\times d{\bf S}),
   $$
   s{\aa} omformuleras Gauss teorem \sidx{Gauss lag} enligt ovan till
   $$
     \iiint_V {\bf c}\cdot(\nabla\times{\bf v})\,dV
       =\oiint_S -{\bf c}\cdot({\bf v}\times d{\bf S})
     \qquad\Leftrightarrow\qquad
     \iiint_V (\nabla\times{\bf v})\,dV
       =-\oiint_S ({\bf v}\times d{\bf S}).
     \qquad\hbox{(Q.E.D.)}
   $$}
$$
  \iiint_V(\nabla\times{\bf v})\,dV = -\oiint_{S} {\bf v}\times d{\bf S}.
$$
Den andra integralen i h{\"o}gerledet kan nu omformuleras med hj{\"a}lp av
detta ``kvasi-Gauss-teorem'' som
$$
  \iiint_V\nabla'\times\bigg({{{\bf M}({\bf x}')}
    \over{|{\bf x}-{\bf x}'|}}\bigg)\,dV'
  = -\oiint_{S}\bigg({{{\bf M}({\bf x}')}
       \over{|{\bf x}-{\bf x}'|}}\bigg)\times d{\bf S}'.
$$
\sidx{Vektorpotential ${\bf A}$}
Vektorpotentialen enligt ovan uttrycks d{\"a}rmed (n{\"a}stan) slutligen som
$$
  {\bf A}({\bf x}) = {{\mu_0}\over{4\pi}}\iiint_V
    {{\big(\nabla'\times{\bf M}({\bf x}')\big)}\over{|{\bf x}-{\bf x}'|}}\,dV'
  +{{\mu_0}\over{4\pi}}\oiint_S
     \bigg({{{\bf M}({\bf x}')}\over{|{\bf x}-{\bf x}'|}}\bigg)\times d{\bf S}'.
$$
\vfill\eject

Den f{\"o}rsta termen kan tolkas som ett potentialbidrag fr{\aa}n en
{\it volymsstr{\"o}m} orsakad av {\it bundna
laddningar},\numberedfootnote{Griffiths Ekv.~(6.13), sid.~275.}
\sidx{Bunden volymstr{\"o}m ${\bf J}_{\rm b}$}
$$
  {\bf J}_{\rm b}=\nabla\times{\bf M},
$$
medan den andra termen ist{\"a}llet kan tolkas som ett potentialbidrag fr{\aa}n
en {\it ytstr{\"o}m}, {\"a}ven den orsakad av bundna
laddningar,\numberedfootnote{Griffiths Ekv.~(6.14), sid.~275.}
\sidx{Bunden ytstr{\"o}m ${\bf K}_{\rm b}$}
$$
  {\bf K}_{\rm b}={\bf M}\times{\bf e}_n,
$$
d{\"a}r ${\bf e}_n$ {\"a}r normalvektorn ut fr{\aa}n den slutna ytan, det
vill s{\"a}ga samma normalvektor som ing{\aa}r i ytelementets normal som
$d{\bf S}'={\bf e}_n\,dS'$. Summa summarum kan allts{\aa} vektorpotentialen
skrivas i termer av ekvivalenta volyms- och ytstr{\"o}mmar som
\sidx{Vektorpotential ${\bf A}$}[Ekvivalenta volym- och ytstr{\"o}mmar]
$$
  {\bf A}({\bf x}) = {{\mu_0}\over{4\pi}}\iiint_V
    {{{\bf J}_{\rm b}({\bf x}')}\over{|{\bf x}-{\bf x}'|}}\,dV'
  +{{\mu_0}\over{4\pi}}\oiint_S
    {{{\bf K}_{\rm b}({\bf x}')}\over{|{\bf x}-{\bf x}'|}}\,dS'.
$$
Detta betyder att vektorpotentialen ${\bf A}({\bf x})$, och d{\"a}rmed {\"a}ven
det magnetiska f{\"a}ltet ${\bf B}({\bf x})=\nabla\times{\bf A}({\bf x})$
fr{\aa}n det magnetiserade objektet ges som om objektet ist{\"a}llet hade
uppburit en motsvarande {\it bunden volyms-str{\"o}m} ${\bf J}_{\rm b}$ och
en {\it bunden ytstr{\"o}m} ${\bf K}_{\rm b}$. Detta illustrerar den t{\"a}ta
kopplingen mellan magnetiska f{\"a}lt och ekvivalenta str{\"o}mmar.
\vfill\eject

\section{Sammanfattning av F{\"o}rel{\"a}sning~7
  -- Magnetiska f{\"a}lt i material}
\item{$\bullet$}{Att $\nabla\cdot{\bf B}=0$ betyder att det inte existerar
  n{\aa}gra magnetiska monopoler.}
\item{$\bullet$}{Magnetiskt moment ${\bf m}$ enligt Gilbert-modellen ges som
$$
{\bf m}=I\iint_A\,d{\bf A}=IA{\bf e}_{\bf n}.
$$}
\item{$\bullet$}{Upplagrad energi f{\"o}r elektrisk dipol ${\bf p}$ i
  elektriskt f{\"a}lt ${\bf E}_{\rm ext}$ respektive magnetisk dipol ${\bf m}$
  i magnetiskt f{\"a}lt ${\bf B}_{\rm ext}$,
  $$
    W_{\rm e} = -{\bf p}\cdot{\bf E}_{\rm ext},\qquad\qquad
    W_{\rm m} = -{\bf m}\cdot{\bf B}_{\rm ext}.
  $$}
\item{$\bullet$}{Kraft p{\aa} elektrisk dipol ${\bf p}$ repektive magnetisk
  dipol ${\bf m}$,
  $$
    {\bf F}_{\rm e} = -\nabla W_{\rm e}
      =\nabla({\bf p}\cdot{\bf E}_{\rm ext}),\qquad\qquad
    {\bf F}_{\rm m} = -\nabla W_{\rm m}
      =\nabla({\bf m}\cdot{\bf B}_{\rm ext}).
  $$}
\item{$\bullet$}{Vridmoment ${\bf\tau}$ p{\aa} elektrisk dipol ${\bf p}$
  repektive magnetisk dipol ${\bf m}$,
  $$
    {\bf \tau}={\bf p}\times{\bf E}_{\rm ext},\qquad\qquad
    {\bf \tau}={\bf m}\times{\bf B}_{\rm ext}.
  $$}
\item{$\bullet$}{Magnetisering (``magnetisk polarisationsdensitet''),
  $$
    {\bf M} \equiv \Big\langle{{d{\bf m}}\over{dV}}\Big\rangle
      = {{1}\over{\mu_0}}\Big(1-{{1}\over{\mu_{\rm r}}}\Big){\bf B}
  $$
  d{\"a}r $\mu_{\rm r}$ {\"a}r den {\it relativa magnetiska permeabiliteten}
  f{\"o}r materialet.}

\item{$\bullet$}{Lokal magnetisering ${\bf M}$ kan uttryckas i termer av
  bundna yt- och volymstr{\"o}mmar $K_{\rm b}$ och $J_{\rm b}$ som
  $$
    {\bf J}_{\rm b}=\nabla\times{\bf M}
                   \quad(\hbox{volymstr{\"o}m},\ {\rm C}/{\rm m}^2),\qquad\qquad
    {\bf K}_{\rm b}={\bf M}\times{\bf e}_n
                   \quad(\hbox{ytstr{\"o}m},\ {\rm C}/{\rm m}).
  $$
  d{\"a}r ${\bf e}_n$ {\"a}r normalvektorn ut fr{\aa}n den slutna ytan hos den
  magnetiserade dom{\"a}nen.}
\item{$\bullet$}{Magnetiseringsstyrkan ${\bf H}$ definieras som
  $$
    {\bf H}\equiv{{\bf B}\over{\mu_0}}-{\bf M}={{\bf B}\over{\mu_0\mu_{\rm r}}}
    \qquad\Leftrightarrow\qquad
    {\bf B}=\mu_0\mu_{\rm r}{\bf H}.
  $$}
\item{$\bullet$}{Upplagrad energi i magnetf{\"a}lt,
  $$
    W = {{1}\over{2}}\iiint{\bf B}\cdot{\bf H}\,dV.
  $$}
\item{$\bullet$}{Vektorpotentialen i en {\it observationspunkt} ${\bf x}$
  fr{\aa}n en isolerad magnetisk dipol ${\bf m}$ i {\it k{\"a}llpunkten}
  ${\bf x}'$ ges av
  $$
    {\bf A}({\bf x})={{\mu_0}\over{4\pi}}
      {{{\bf m}\times({\bf x}-{\bf x}')}\over{|{\bf x}-{\bf x}'|^2}}.
  $$}
\item{$\bullet$}{Vektorpotentialen fr{\aa}n ett magnetiserat objekt kan
  {\it alltid} skrivas i termer av ekvivalenta volyms- och ytstr{\"o}mmar
  ${\bf J}_{\rm b}$ och ${\bf K}_{\rm b}$ av {\it bundna laddningar} som
  $$
    {\bf A}({\bf x}) = {{\mu_0}\over{4\pi}}\iiint_V
      {{{\bf J}_{\rm b}({\bf x}')}\over{|{\bf x}-{\bf x}'|}}\,dV'
    +{{\mu_0}\over{4\pi}}\oiint_S
      {{{\bf K}_{\rm b}({\bf x}')}\over{|{\bf x}-{\bf x}'|}}\,dS'.
  $$}
\item{$\bullet$}{F{\"o}r en observat{\"o}r g{\aa}r ett {\it magnetiserat}
  objekt ej att s{\"a}rskilja fr{\aa}n fallet d{\"a}r objektet ist{\"a}llet
  hade uppburit l{\"a}mpligt konstruerade interna och bundna volyms- och
  ytstr{\"o}mmar ${\bf J}_{\rm b}$ och ${\bf K}_{\rm b}$.}

\cleardoublepage
%%% End of auto-extracted text from ../lect-07/lecture-07.tex %%%
%%% Begin of auto-extracted text from ../lect-08/lecture-08.tex %%%
%
% File: teach/elmagii/lect-08/lecture-08.tex [plain TeX code]
% Github: https://github.com/elmagii/lect-08/
% Last change: December 20, 2025
%
% Lecture No 8 in the course ``Elektromagnetism II, 1TE626 (2023)'',
% held November 24, 2025, at Uppsala University, Sweden.
%
% Copyright (C) 2022-2025, Fredrik Jonsson, under Gnu General Public
% License (GPL) v3. See the enclosed LICENSE for details.
%
% This program is free software: you can redistribute it and/or modify
% it under the terms of the GNU General Public License as published by
% the Free Software Foundation, either version 3 of the License, or
% (at your option) any later version.
%
% This program is distributed in the hope that it will be useful,
% but WITHOUT ANY WARRANTY; without even the implied warranty of
% MERCHANTABILITY or FITNESS FOR A PARTICULAR PURPOSE.  See the
% GNU General Public License for more details.
%
% You should have received a copy of the GNU General Public License
% along with this program.  If not, see <https://www.gnu.org/licenses/>.
%
\def\coursename{Elektromagnetism II}
\def\coursecode{1TE626}
\def\courseyear{2025}
\def\courserepo{https://github.com/hp35/elmagii/}
\def\lecturenumber{8}
\def\lecturetitle{Multipolutvecklingen}
\def\lecturesubtitle{}
\def\lectureauthor{Fredrik Jonsson}
\def\lectureplace{Uppsala Universitet}
\def\lecturedate{24 november 2025}
%-------------------- BEGIN OF LOCAL MACROS --------------------
\edef\expandedlecturenumber{8}
\def\ifempty#1{\ifx\relax#1\relax}
\advance\chapno by 1
\secno=0
\footnotenumber=0
\message{==================== Lecture 8 ====================}
\writenumberedtocentry{chapter}{F{\"o}rel{\"a}sning 8 -- {Multipolutvecklingen}}{\thechapno}
\hsize=150mm\hoffset=4.6mm\vsize=230mm\voffset=7mm
\topskip=0pt\baselineskip=12pt\parskip=0pt\leftskip=0pt\parindent=15pt
\ifcolors
  \voffset=-10.2mm\topskip=0pt
\fi
\headline={\ifnum\secno>0\ifodd\pageno\rightheadline\else\leftheadline\fi
  \else\hfill\fi}
\def\rightheadline{\tenrm{\it F\"orel\"asning 8}
  \hfil{\it \coursename, \coursecode\ (\courseyear)}}
\def\leftheadline{\tenrm{\it \coursename, \coursecode\ (\courseyear)}
  \hfil{\it F\"orel\"asning 8}}
\noindent~\vskip-60pt\hskip-40pt{\epsfbox{../lect-01/macros/UU_logo_color.eps}}
\vskip-42pt\hfill\vbox{
    \hbox{{\it \coursename, \coursecode\ (\courseyear)}}
    \hbox{{\it Lecture Notes, \lectureauthor}}
    \hbox{{\it Document Revision \today}}
    \hbox{{\it \courserepo}}}\vskip 36pt
\centerline{\twelvesc F\"orel\"asning 8}
\vskip 24pt\noindent
\centerline{\twelvesc{Multipolutvecklingen}}
\expandafter\ifempty\expandafter{\lecturesubtitle}%
  \else\centerline{\twelvesc\lecturesubtitle}\fi
\bigskip
\centerline{\lectureauthor, \lectureplace, \lecturedate}
\vskip24pt
%--------------------- END OF LOCAL MACROS ---------------------



\plan{Vi l{\"a}mnar f{\"o}r denna f{\"o}rel{\"a}sning dipolmodellen f{\"o}r ett
  tag, och visar p{\aa} att andra konstruktioner av laddningsf{\"o}rdelningar,
  exempelvis kvadrupoler och oktopoler, ger bidrag till f{\"a}lt som avklingar
  med en annan takt en det klassiska ``$1/r^2$''-upptr{\"a}dandet.
  Vi utg{\aa}r fr{\aa}n de klassiska integralerna f{\"o}r den skal{\"a}ra
  potentialen $\phi$ och vektorpotentialen ${\bf A}$ och {\"a}gnar oss {\aa}t
  en statisk modell f{\"o}r multipolutvecklingen.

  Vi g{\aa}r igenom begreppen dipol, kvadrupol, oktopol och h{\"o}gre
  ordningar; specifikt visar vi p{\aa} att {\"a}ven det enklast t{\"a}nkbara
  fallet av en elektrisk dipol har h{\"o}gre ordningars multipolmoment
  n{\"a}rvarande, men att dessa oftast f{\"o}rsummas vid betraktelse p{\aa}
  stora avst{\aa}nd.

  F{\"o}rel{\"a}sningen avslutas med att demonstrera multipolutvecklingen
  f{\"o}r generella laddningsf{\"o}rdelningar i termer av skal{\"a}r potential
  (elektrostatik) och vektorpotential (magnetostatik), samt fastst{\"a}llande
  av dipolapproximationen f{\"o}r elektrostatiska och magnetostatiska f{\"a}lt.
}

\threepointsummary{%
  Multipolutvecklingen {\"a}r i grund och botten en serieutveckling av
  potentialen (skal{\"a}r eller vektor) i inversa potenser av distansen
  mellan k{\"a}lla och observationspunkt.
}{%
  \sidx{Multipolutveckling}[Klassificering av moment]
  Multipolutvecklingen klassificeras term f{\"o}r term utifr{\aa}n hur
  beroendet av avst{\aa}ndet $r=|{\bf x}-{\bf x}'|$ mellan k{\"a}llpunkt
  ${\bf x}'$ och f{\"a}ltpunkt ${\bf x}$ ser ut:
  \litem[1.]{\hbox to110pt{Monopol (1-pol):\hfil}
    \sidx{Monopol}$\phi({\bf x})\sim 1/r$}
  \litem[2.]{\hbox to110pt{Dipol (2-pol):\hfil}
    \sidx{Dipol}$\phi({\bf x})\sim 1/r^2$}
  \litem[3.]{\hbox to110pt{Kvadrupol (4-pol):\hfil}
    \sidx{Kvadrupol}$\phi({\bf x})\sim 1/r^3$}
  \litem[4.]{\hbox to110pt{Oktopol (8-pol):\hfil}
    \sidx{Oktopol}$\phi({\bf x})\sim 1/r^4$}
  \litem[5.]{\hbox to110pt{Hexadecapol (16-pol):\hfil}
    \sidx{Hexadecapol}$\phi({\bf x})\sim 1/r^5$}
  \litem[6.]{\hbox to110pt{Dotriacontapol (32-pol):\hfil}
    \sidx{Dotriacontapol}$\phi({\bf x})\sim 1/r^6$}
}{%
  \idx{Dipolapproximationen} f{\"o}r elektrostatiska (${\bf E}=-\nabla\phi$)
  och magnetostatiska (${\bf B}=\nabla\times{\bf A}$) f{\"a}lt p{\aa} l{\aa}ngt
  avst{\aa}nd $|{\bf x}|$ fr{\aa}n laddningsf{\"o}rdelningen (k{\"a}llan)
  $\rho$ lyder
  $$
    \phi({\bf x})\approx{{1}\over{4\pi\varepsilon_0}}\bigg(
        {{q}\over{|{\bf x}|}} + {{{\bf p}\cdot{\bf x}}\over{|{\bf x}|^3}}
      \bigg),\qquad\qquad
    {\bf A}({\bf x})\approx{{\mu_0}\over{4\pi}}
        {{{\bf m}\times{\bf x}}\over{|{\bf x}|^3}},
  $$
  d{\"a}r $q$ {\"a}r nettoladdningen och ${\bf p}$ dipolmomentet hos en
  magnetisk dipol samt ${\bf m}$ det magnetiska dipolmomentet.
  \sidx{Dipolmoment}[Elektriskt]\sidx{Dipolmoment}[Magnetiskt]
}
\vfill\eject\copyrights

\section{Multipolutveckling av f{\"a}lt}
\sidx{Multipolutveckling}
Vi kan sammanfatta {\"a}mnet f{\"o}r dagens f{\"o}rel{\"a}sning, {\it multipolutvecklingen},
med f{\"o}ljande punkter:
\item{$\bullet$}{Multipolutvecklingen bist{\aa}r oss med en formalism f{\"o}r
   att projicera ut olika bidrag (s{\aa} kallade ``moment'') till
   elektromagnetiska f{\"a}lt.\sidx{Elektrisk f{\"a}ltstyrka ${\bf E}$}}
\item{$\bullet$}{Denna klassificering {\"a}r i grund och botten bara en
   t{\"a}mligen enkel serieutveckling, d{\"a}r de olika momenten (monopol,
   dipol, kvadrupol, oktopol, etc.) klassificeras utifr{\aa}n hur deras
   avklingande sker i form av potenser i det inversa avst{\aa}ndet $1/r^n$
   mellan k{\"a}lla och observationspunkt.}
\item{$\bullet$}{Ofta f{\"o}rekommande f{\"o}r karakterisering av antenner.
                 \sidx{Antenner}}
\item{$\bullet$}{\idx{Molekyler} som interagerar med elekromagnetiska f{\"a}lt
   agerar som ``antenner'' vilka kan beskrivas utifr{\aa}n deras
   multipolutvecklingar.}
\item{$\bullet$}{Basen f{\"o}r beskrivning av $\varepsilon_{\rm r}$ i konstitutiva
   relationen ${\bf P}=\varepsilon_0\varepsilon_{\rm r}{\bf E}$ (hur material
   polariseras av elektriska f{\"a}lt) eller ${\bf B}=\mu_0\mu_{\rm r}{\bf H}$
   (hur material magnetiseras av magnetiska f{\"a}lt).
   \sidx{Magnetisk permeabilitet}[Vakuumpermeabilitet $\mu_0$]
   \sidx{Magnetisk permeabilitet}[Relativ permeabilitet $\mu_{\rm r}$]}
\item{$\bullet$}{Oftast fokus p{\aa} elektrisk dipol-approximation, men vissa
   fenomen, som optisk aktivitet, kan bara beskrivas om h{\"o}gre ordningars
   termer tas med.}
\medskip
\noindent
Vi kommer h{\"a}r att i huvudsak anv{\"a}nda de generella uttrycken f{\"o}r
elektrisk skal{\"a}r potential och vektorpotentialen,
som\numberedfootnote{F{\"o}r formulering av skal{\"a}ra potentialen
  $\phi({\bf x})$ samt vektorpotentialen ${\bf A}({\bf x})$ fr{\aa}n en
  generell laddningsdistribution, se F{\"o}rel{\"a}sning~2 (f{\"o}r den
  skal{\"a}ra potentialen $\phi$) samt F{\"o}rel{\"a}sning~4 (f{\"o}r
  vektorpotentialen ${\bf A}$); alternativt Griffiths Ekv.~(10.26), s.~445.
  Notera h{\"a}r hur vektorpotentialen ${\bf A}({\bf x})$ har en form
  som {\it exakt matchar det uttryck som vi under f{\"o}rra
  f{\"o}rel{\"a}sningen tog fram f{\"o}r samma vektorpotential f{\"o}r
  ett magnetiserat objekt} med magnetiseringen ${\bf M}({\bf x})$, i
  termer av en {\it bunden volymstr{\"o}m}
  ${\bf J}_{\rm b}=\nabla\times{\bf M}$ (${\rm A}/{\rm m}^2$) och en
  {\it bunden ytstr{\"o}m} ${\bf K}_{\rm b}={\bf M}\times{\bf e}_n$
  (${\rm A}/{\rm m}$).
  Detta uttrycker p{\aa} ett mer generellt plan att vi med f{\"o}rra
  f{\"o}rel{\"a}sningens slutresultat {\it kan applicera den
  multipolutveckling som vi nu kommer att g{\aa} igenom p{\aa} ett
  godtyckligt magnetiserat objekt!}}
$$
  \phi({\bf x})={{1}\over{4\pi\varepsilon_0}}
    \iiint_{{\Bbb R}^3}{{\rho({\bf x}')}\over{|{\bf x}-{\bf x}'|}}\,dV',
    \qquad\qquad
  {\bf A}({\bf x})={{\mu_0}\over{4\pi}}
    \iiint_{{\Bbb R}^3}{{{\bf J}({\bf x}')}\over{|{\bf x}-{\bf x}'|}}\,dV',
$$
d{\"a}r $\varepsilon_0= 8.854\times10^{-12}\ {\rm F}/{\rm m}$ {\"a}r den
{\it elektriska permittiviteten f{\"o}r vakuum} (electric permittivity of
vacuum), samt $\mu_0=1.257\times10^{-6}\ {\rm N}/{\rm A}^2$ den {\it magnetiska
permeabiliteten f{\"o}r vakuum} (vacuum magnetic permeability). I dessa uttryck
{\"a}r $\rho$ den elektriska laddningst{\"a}theten\numberedfootnote{Eller
  om vi s{\aa} vill, {\it laddningsdensiteten}, f{\"o}r att anknyta till
  massdensitet.}
(${\rm C}/{\rm m}^3$) och ${\bf J}$ den elektriska str{\"o}mt{\"a}theten
(${\rm A}/{\rm m}^2$).
Vi anv{\"a}nder h{\"a}r de generella, tredimensionella uttrycken f{\"o}r
potentialerna f{\"o}r att {\"o}va p{\aa} deras till{\"a}mpningen, samt f{\"o}r
att bygga upp en generell verktygsl{\aa}da f{\"o}r att probleml{\"o}sning inom
elektromagnetisk f{\"a}ltteori.

F{\"o}r att rekapitulera sj{\"a}lva vitsen med att anv{\"a}nda den skal{\"a}ra
potentialen och vektorpotentialen, s{\aa} kan vi ur dessa extrahera de
{\it statiska} (ej tidsberoende) elektriska och magnetiska f{\"a}lten
som\numberedfootnote{Recap p{\aa} ursprunget f{\"o}r vektorpotentialen:
  $\nabla\cdot{\bf B}=0\quad\Leftrightarrow
    \quad\exists{\bf A}: {\bf B}=\nabla\times{\bf A}$.
  Notera att uttrycket som h{\"a}r anv{\"a}nds f{\"o}r den elektriska
  f{\"a}ltstyrkan {\it endast} {\"a}r giltig f{\"o}r {\it statiska} f{\"a}lt.
  Som vi kommer att se i F{\"o}rel{\"a}sning~11 p{\aa} retarderade
  (tidsf{\"o}rdr{\"o}jda) potentialer, s{\aa} {\"a}r den egentliga
  elektrodynamiska formen
  $$
    \hskip140pt
    {\bf E}=-\nabla\phi-{{\partial{\bf A}}\over{\partial t}}.\qquad\qquad
    \bigg(\hbox{h{\"a}r statiskt: }{{\partial{\bf A}}\over{\partial t}}=0\bigg)
  $$}
$$
  {\bf E}=-\nabla\phi,\qquad
  {\bf B}=\nabla\times{\bf A}.
$$
\sidx{Skal{\"a}r potential $\phi$}
\sidx{Vektorpotential ${\bf A}$}
\sidx{Elektrisk f{\"a}ltstyrka ${\bf E}$}
\sidx{Magnetisk fl{\"o}dest{\"a}thet ${\bf B}$}
Vi kan ocks{\aa} rekapitulera att SI-enheterna f{\"o}r den skal{\"a}ra
potentialen och vektorpotentialen {\"a}r
$$
  \eqalign{
    [\phi]&={{[\rho][dV']}\over{[\varepsilon_0][{\bf x}]}}
        ={{({\rm C}/{\rm m}^3){\rm m}^3}\over{({\rm F}/{\rm m}){\rm m}}}
        =\{\ {\rm F}={\rm C}/{\rm V}\ \}
        ={\rm V},\cr
    [{\bf A}]&={{[\mu_0][{\bf J}][dV']}\over{[{\bf x}]}}
        ={{({\rm N}/{\rm A}^2)({\rm A}/{\rm m}^2){\rm m}^3}\over{{\rm m}}}
        ={\rm N}/{\rm A}.\cr
  }
$$
\vfill\eject

\section{Multipolutveckling f{\"o}r skal{\"a}r potential f{\"o}r en
         elektrisk dipol}
\sidx{Dipol}[Elektrisk]
Ett av de enklaste testobjekten inom elektromagnetism {\"a}r den elektriska
dipolen,\numberedfootnote{The English prefixes {\it bi-}, derived from Latin,
  and its Greek variant {\it di-} both mean ``two''. The Latin prefix is far
  more prevalent in common words, such as {\it bilingual}, {\it biceps}, and
  {\it biped}; the more technical Greek {\it di-} appears in such words as
  {\it diphthong} and {\it dilemma}.} best{\aa}ende av en positiv och negativ
laddning separerade ett avst{\aa}nd $L$.
{\"A}ven f{\"o}r en elektrisk {\it di}pol finns det (paradoxalt) termer av
{\it multi}polmoment d{\aa} vi l{\"a}mnar approximationen att den skal{\"a}ra
elektriska potentialen enbart ges av skal{\"a}rprodukten mellan dipolmomentet
och ortsvektorn till observationspunkten.

Vi betraktar en klassisk elektrisk dipol med tv{\aa} laddningar $+q$ or $-q$
separerade avst{\aa}ndet $L=2b$. Dipolen kan beskrivas med en
laddningsf{\"o}rdelning enligt {\it distributionen}
\sidx{Laddningsf{\"o}rdelning}[Distribution f{\"o}r punktladdning]
\sidx{Punktladdning}
$$
  \rho({\bf x})=(+q)\delta({\bf x}-b{\bf e}_z)
     +(-q)\delta({\bf x}+b{\bf e}_z)
$$
\epsfig{../lect-08/figs/eldipole.1}\noindent
\vfill\eject
Den {\it elektriska skal{\"a}ra potentialen} \sidx{Skal{\"a}r potential $\phi$}
fr{\aa}n denna specifika distribution av tv{\aa}
punktk{\"a}llor\numberedfootnote{Vi utg{\aa}r h{\"a}r ifr{\aa}n den generella
  tre-dimensionella volymintegralen \sidx{Integral}[Volym-] av en
  laddningst{\"a}thet $\rho({\bf x})$, bara f{\"o}r att illustrera hur vi kan
  till{\"a}mpa denna {\"a}ven f{\"o}r lokaliserade punktladdningar.}
blir d{\"a}rmed
$$
  \eqalign{
    \phi({\bf x})&={{1}\over{4\pi\varepsilon_0}}
        \iiint_{V}{{\rho({\bf x}')}\over{|{\bf x}-{\bf x}'|}}\,dV'\cr
      &={{1}\over{4\pi\varepsilon_0}}\left(
         {{(+q)}\over{|{\bf x}-b{\bf e}_z|}}+{{(-q)}\over{|{\bf x}+b{\bf e}_z|}}
        \right)\cr
      &={{q}\over{4\pi\varepsilon_0}}\left(
          {{1}\over{r-b\cos\theta}}-{{1}\over{r+b\cos\theta}}
        \right)\cr
      &=\big\{\ \hbox{Definiera}\ \varepsilon \equiv (b/r)\cos\theta\ \big\}\cr
      &={{q}\over{4\pi\varepsilon_0 r}}\left(
          {{1}\over{1-\varepsilon}}-{{1}\over{1+\varepsilon}}
        \right)\cr
      &=\big\{\ \hbox{Taylor-utveckling f{\"o}r litet}\ \varepsilon\ \big\}\cr
      &={{q}\over{4\pi\varepsilon_0 r}}\left(
          (1+\varepsilon+\varepsilon^2+\varepsilon^3+\ldots)
            -(1-\varepsilon+\varepsilon^2-\varepsilon^3+\ldots)
        \right)\cr
      &={{q}\over{2\pi\varepsilon_0 r}}\left(
          \varepsilon+\varepsilon^3+\varepsilon^5+\ldots
        \right)\cr
      &={{q}\over{2\pi\varepsilon_0}}\Bigg(
         \underbrace{{{b\cos\theta}\over{r^2}}}_{\rm dipol}
         +\underbrace{{{(b\cos\theta)^3}\over{r^4}}}_{\rm oktopol}
         +\underbrace{{{(b\cos\theta)^5}\over{r^6}}}_{\rm dotriacontapol}
         +\ldots
        \Bigg)\cr
  }
$$
\sidx{Dipol}\sidx{Oktopol}\sidx{Dotriacontapol}

\vfill\eject
\centerline{\epsfxsize=144mm\epsfbox{../lect-08/multipoles/multipoles/lindipole.eps}}
\noindent
{\captionwide Skal{\"a}r potential $\phi(x,y)$ f{\"o}r en elektrisk dipol.}
\sidx{Skal{\"a}r potential $\phi$}\sidx{Dipol}[Elektrisk]
\medskip
\centerline{\epsfxsize=144mm\epsfbox{../lect-08/multipoles/multipoles/lindipole-str.eps}}
\noindent
{\captionwide Skal{\"a}r potential $\phi(x,y)$ f{\"o}r en elektrisk dipol med
f{\"a}ltlinjer f{\"o}r ${\bf E}(x,y)=-\nabla\phi(x,y)$.}
\vfill\eject

\section{Exempel p{\aa} multipoler}
Som en relativt enkel illustration av multipoler kan vi konstruera olika
multipolmoment utifr{\aa}n diskreta laddningar.
\epsfig{../lect-08/figs/quadrupole.1}
\epsfig{../lect-08/figs/octopole.1}
\noindent
En intressant {\"o}vning {\"a}r att g{\aa} igenom dessa specifika multipoler
och upprepa den geometriska analysen f{\"o}r dipolen, f{\"o}r att p{\aa} s{\aa}
s{\"a}tt extrahera styrkan p{\aa} de olika multipolmomenten.

\section{Multipolutveckling f{\"o}r skal{\"a}r potential f{\"o}r en
         linj{\"a}r elektrisk kvadrupol}
L{\aa}t oss g{\"o}ra en liten {\"a}ndring p{\aa} den elektriska dipolen i
f{\"o}reg{\aa}ende exempel, och ist{\"a}llet l{\"a}gga en punktladdning $-2q$
i centrum med tv{\aa} punktladdningar $+q$ p{\aa} diametralt motsatt sida om
denna. Denna konfiguration kan beskrivas med {\it distributionen}
$$
  \rho({\bf x})=(+q)\delta({\bf x}-b{\bf e}_z)
  +(-2q)\delta({\bf x})
  +(+q)\delta({\bf x}+b{\bf e}_z),
$$
precis analogt med det f{\"o}reg{\aa}ende fallet f{\"o}r den elektriska dipolen.
\epsfig{../lect-08/figs/linelquadpole.1}\noindent
P{\aa} exakt samma s{\"a}tt som tidigare, med den enda skillnaden att vi nu
har {\it tre} laddningar (k{\"a}ll\-termer) f{\"o}r den skal{\"a}ra potentialen
$\phi({\bf x})$, s{\aa} har vi att
$$
  \eqalign{
    \phi({\bf x})&={{1}\over{4\pi\varepsilon_0}}
        \iiint_{V}{{\rho({\bf x}')}\over{|{\bf x}-{\bf x}'|}}\,dV'\cr
      &={{1}\over{4\pi\varepsilon_0}}\left(
          {{(+q)}\over{|{\bf x}-b{\bf e}_z|}}
            +{{(-2q)}\over{|{\bf x}|}}
            +{{(+q)}\over{|{\bf x}+b{\bf e}_z|}}
        \right)\cr
      &={{q}\over{4\pi\varepsilon_0}}\left(
          {{1}\over{r-b\cos\theta}}
            -{{2}\over{r}}
            +{{1}\over{r+b\cos\theta}}
        \right)\cr
      &=\big\{\ \hbox{Definiera}\ \varepsilon \equiv (b/r)\cos\theta\ \big\}\cr
      &={{q}\over{4\pi\varepsilon_0 r}}\left(
          {{1}\over{1-\varepsilon}}
            -2
            +{{1}\over{1+\varepsilon}}
        \right)\cr
      &=\big\{\ \hbox{Taylor-utveckling f{\"o}r litet}\ \varepsilon\ \big\}\cr
      &={{q}\over{4\pi\varepsilon_0 r}}\left(
          (1+\varepsilon+\varepsilon^2+\varepsilon^3+\ldots)
            -2
            +(1-\varepsilon+\varepsilon^2-\varepsilon^3+\ldots)
        \right)\cr
      &={{q}\over{2\pi\varepsilon_0 r}}\left(
          \varepsilon^2+\varepsilon^4+\varepsilon^6+\ldots
        \right)\cr
      &={{q}\over{2\pi\varepsilon_0}}\Bigg(
         \underbrace{{{(b\cos\theta)^2}\over{r^3}}}_{\rm kvadrupol}
         +\underbrace{{{(b\cos\theta)^4}\over{r^5}}}_{\rm hexadecapol}
         +\underbrace{{{(b\cos\theta)^6}\over{r^7}}}_{\rm hexacontatetrapol}
         +\ldots
        \Bigg)\cr
  }
$$
\sidx{Kvadrupol}\sidx{Hexadecapol}\sidx{Hexacontatetrapol}
I denna serieutveckling {\aa}terfinns nu endast {\it udda} potenser av
avst{\aa}ndet $r$, varav den dominerande termen p{\aa} stora avst{\aa}nd
kommer att vara den som g{\aa}r som $O(1/r^3)$, svarande mot en elektrisk
{\it kvadrupol}. V{\"a}rt att notera {\"a}r att i denna serieutveckling
{\aa}terfinns {\it ingen dipolterm}.\sidx{Dipol}[Elektrisk]

\section{Klassificering av multipolmoment}
\sidx{Multipolutveckling}[Klassificering av moment]
Multipolutvecklingen klassificeras term f{\"o}r term av hur beroendet av
avst{\aa}ndet $r=|{\bf x}-{\bf x}'|$ mellan k{\"a}llpunkt ${\bf x}'$ och
f{\"a}ltpunkt ${\bf x}$ ser ut:
\medskip
\litem[1.]{\hbox to140pt{Monopol (1-pol):\hfil} $\phi({\bf x})\sim 1/r$}
\litem[2.]{\hbox to140pt{Dipol (2-pol):\hfil} $\phi({\bf x})\sim 1/r^2$}
\litem[3.]{\hbox to140pt{Kvadrupol (4-pol):\hfil} $\phi({\bf x})\sim 1/r^3$}
\litem[4.]{\hbox to140pt{Oktopol (8-pol):\hfil} $\phi({\bf x})\sim 1/r^4$}
\litem[5.]{\hbox to140pt{Hexadecapol (16-pol):\hfil} $\phi({\bf x})\sim 1/r^5$}
\litem[6.]{\hbox to140pt{Dotriacontapol (32-pol):\hfil}
  $\phi({\bf x})\sim 1/r^6$}
\litem[7.]{\hbox to140pt{Hexacontatetrapol (64-pol):\hfil}
  $\phi({\bf x})\sim 1/r^7$}
\litem[8.]{\hbox to110pt{$\ldots$\hfil} $\ldots$}
\sidx{Monopol}
\sidx{Dipol}
\sidx{Kvadrupol}
\sidx{Oktopol}
\sidx{Hexadecapol}
\sidx{Dotriacontapol}
\sidx{Hexacontatetrapol}
\medskip
\noindent
Sammanfattningsvis s{\aa} har en klassisk elektrisk dipol samtliga
multipolmoment som {\"a}r {\it j{\"a}mna}, det vill s{\"a}ga {\it dipol}
($\phi\sim 1/r^2$), {\it oktopol} ($\phi\sim 1/r^4$), {\it dotriacontapole}
($\phi\sim 1/r^6$), etc.
\vfill\eject
\section{Skal{\"a}r potential och elektriskt f{\"a}lt f{\"o}r en linj{\"a}r
         elektrisk kvadrupol}
\centerline{\epsfxsize=142mm
  \epsfbox{../lect-08/multipoles/multipoles/linquadrupole.eps}}
\noindent
{\captionwide Skal{\"a}r potential $\phi(x,y)$ f{\"o}r en linj{\"a}r elektrisk
kvadrupol.}
\medskip
\centerline{\epsfxsize=142mm
  \epsfbox{../lect-08/multipoles/multipoles/linquadrupole-str.eps}}
\noindent
{\captionwide Skal{\"a}r potential $\phi(x,y)$ f{\"o}r en elektrisk linj{\"a}r
kvadrupol med f{\"a}ltlinjer f{\"o}r ${\bf E}(x,y)=-\nabla\phi(x,y)$.}
\vfill\eject

\section{Skal{\"a}r potential och elektriskt f{\"a}lt f{\"o}r en kvadratisk
         elektrisk kvadrupol}
\centerline{\epsfxsize=142mm
  \epsfbox{../lect-08/multipoles/multipoles/quadquadrupole.eps}}
\noindent
{\captionwide Skal{\"a}r potential $\phi(x,y)$ f{\"o}r en kvadratisk elektrisk
kvadrupol.}
\medskip
\centerline{\epsfxsize=142mm
  \epsfbox{../lect-08/multipoles/multipoles/quadquadrupole-str.eps}}
\noindent
{\captionwide Skal{\"a}r potential $\phi(x,y)$ f{\"o}r en kvadratisk
elektrisk kvadrupol med f{\"a}ltlinjer ${\bf E}(x,y)=-\nabla\phi(x,y)$.}

\section{Multipolutveckling av skal{\"a}r potential f{\"o}r generella laddningsf{\"o}rdelningar}
\sidx{Multipolutveckling}[F{\"o}r skal{\"a}r potential]
\sidx{Multipolutveckling}[Generell laddnings\-f{\"o}r\-del\-ning]
Vi betraktar en generell laddningsf{\"o}rdelning enligt figur.
Laddningsf{\"ordelningen} kan vara en generell distribution i 3D, men {\"a}ven
i kombinationer av 2D (ytladdningar, sk{\"a}rmar, jordplan),
1D (linjeladdningar, antenner) eller 0D (punktladdningar). Vi kommer i det
f{\"o}ljande att anv{\"a}nda en geometri i vilken laddningst{\"a}theten
$\rho({\bf x})$ {\"a}r lokaliserad i n{\"a}rheten av origo (vilket g{\"o}r
det enklare rent algebraiskt att r{\"a}tt av till{\"a}mpa en serieutveckling
f{\"o}r laddningsdensiteten i en Maclaurin-utveckling, snarare {\"a}n i en
generell Taylor-utveckling, {\"a}ven om slutresultatet naturligtvis {\"a}r
detsamma), enligt figur.
\sidx{Taylor-utveckling}\sidx{Maclaurin-utveckling}
\sidx{Skal{\"a}r potential $\phi$}
\epsfig{../lect-08/figs/chargedist.1}\noindent
Den skal{\"a}ra (elektriska) potentialen $\phi({\bf x})$ fr{\aa}n en
punktk{\"a}lla $q$ placerad i ${\bf x}'$ ges som
$$
  \phi({\bf x})={{q}\over{4\pi\varepsilon_0|{\bf x}-{\bf x}'|}},
$$
eller om vi s{\aa} vill, det infinitesimala bidraget $d\phi$ till den
skal{\"a}ra potentialen vid ${\bf x}$ fr{\aa}n k{\"a}llan
$dq'=\rho({\bf x}')dV'$ vid k{\"a}llpunkten ${\bf x}'$.
\noindent
Den skal{\"a}ra potentialen fr{\aa}n distributionen $\rho({\bf x})$ ges
d{\"a}rmed analogt genom att helt enkelt summera upp alla bidrag fr{\aa}n
samtliga $\rho({\bf x}')dV'$ i volymen $V$, som
$$
  \phi({\bf x})={{1}\over{4\pi\varepsilon_0}}
     \iiint_{V}{{\rho({\bf x}')}\over{|{\bf x}-{\bf x}'|}}\,dV'
$$
I denna integral s{\aa} kan vi se det som att vi summerar upp alla element
$dV$'s infinitesimala laddningar $dq=\rho({\bf x}')dV'$ (eller
str{\"o}mt{\"a}theter ${\bf J}({\bf x}')dV$ i fallet med vektorpotentialen
${\bf A}$), viktade med en skal{\"a}r faktor
$$
  f({\bf x}')={{1}\over{|{\bf x}-{\bf x}'|}}
$$
som kort och gott {\"a}r det inversa geometriska avst{\aa}ndet mellan
k{\"a}llpunkten ${\bf x}'$ och f{\"a}ltpunkten (observationspunkten) ${\bf x}$.
Vi kan, om vi s{\aa} vill, se detta som en summation av potentialbidragen
fr{\aa}n alla infinitesimala elektriska {\it monopoler} som ryms i volymen
$V$.

\subsection{Maclaurin-utveckling av den geometriska viktfunktionen}
Liksom i det enkla fallet med dipolen, siktar vi h{\"a}r mot en serieutveckling
av denna viktfunktion f{\"o}r att enklare kunna tolka de termer som blir
resultatet. F{\"o}r att rekapitulera, s{\aa} var serieutvecklingen i fallet
med dipolen i grund och botten en serieutveckling av position f{\"o}r
k{\"a}lltermerna, med ``litet separations-avst{\aa}nd $L$ i f{\"o}rh{\aa}llande
till avst{\aa}ndet till observationspunkten ${\bf x}$''. Vi {\"o}nskar
d{\"a}rmed att uttrycka serieutvecklingen i koordinaten ${\bf x}'$ f{\"o}r
{\it k{\"a}llan} $\rho({\bf x}')$. Generellt har vi att en Maclaurin-utveckling
(Taylor-utveckling kring origo) i tre dimensioner ges av
\sidx{Taylor-utveckling}\sidx{Maclaurin-utveckling}
\sidx{Skal{\"a}r potential $\phi$}
$$
  f({\bf x}') = f({\bf 0})
  +\sum^{3}_{k=1} x'_k
     {{\partial f({\bf x}')}
       \over{\partial x'_k}}\bigg|_{{\bf x}'={\bf 0}}
  +{{1}\over{2}}\sum^{3}_{j=1}\sum^{3}_{k=1} x'_j x'_k
     {{\partial^2 f({\bf x}')}
       \over{\partial x'_j\partial x'_k}}\bigg|_{{\bf x}'={\bf 0}}
  +\ldots
$$
F{\"o}r ``viktfunktionen'' $f({\bf x}')$ har vi att
$$
  \eqalign{
    f({\bf x}')&={{1}\over{|{\bf x}-{\bf x}'|}}
      ={{1}\over{\sqrt{(x-x')^2+(y-y')^2+(z-z')^2}}},\cr
    {{\partial f({\bf x}')}\over{\partial x'_k}}
      &={{\partial}\over{\partial x'_k}}
      {{1}\over{\sqrt{(x-x')^2+(y-y')^2+(z-z')^2}}}\cr
      &=-{{1}\over{2}}
      {{-2(x_k-x'_k)}\over{({(x-x')^2+(y-y')^2+(z-z')^2})^{3/2}}}
      =\ldots
      ={{x_k-x'_k}\over{|{\bf x}-{\bf x}'|^3}},\cr
    {{\partial^2 f({\bf x}')}\over{\partial x'_j \partial x'_k}}
      &={{\partial^2}\over{\partial x'_j \partial x'_k}}
          {{1}\over{\sqrt{(x-x')^2+(y-y')^2+(z-z')^2}}}\cr
      &={{\partial}\over{\partial x'_j}}
          {{x_k-x'_k}\over{((x-x')^2+(y-y')^2+(z-z')^2)^{3/2}}}\cr
      &={{{{\partial(x_k-x'_k)}\over{\partial x'_j}}
          ((x-x')^2+(y-y')^2+(z-z')^2)^{3/2}-
            (x_k-x'_k){{\partial ((x-x')^2+(y-y')^2+(z-z')^2)^{3/2}}
              \over{\partial x'_j}}}\over{((x-x')^2+(y-y')^2+(z-z')^2)^{3}}}\cr
      &=\ldots\cr
      &={{3(x_j-x'_j)(x_k-x'_k)
      -\delta_{jk}((x-x')^2+(y-y')^2+(z-z')^2)}
             \over{((x-x')^2+(y-y')^2+(z-z')^2)^{5/2}}}\cr
      &={{3(x_j-x'_j)(x_k-x'_k)-\delta_{jk}|{\bf x}-{\bf x}'|^2}
          \over{|{\bf x}-{\bf x}'|^5}}\cr
  }
$$
F{\"o}r koefficienterna i Maclaurin-utvecklingen av den skal{\"a}ra potentialen
inneb{\"a}r detta specifikt att
$$
    f({\bf 0})
      ={{1}\over{|{\bf x}|}},\qquad\qquad
    {{\partial f({\bf x}')}
      \over{\partial x'_k}}\bigg|_{{\bf x}'={\bf 0}}
        ={{x_k}\over{|{\bf x}|^3}},\qquad\qquad
    {{\partial^2 f({\bf x}')}
      \over{\partial x'_j \partial x'_k}}\bigg|_{{\bf x}'={\bf 0}}
        ={{3 x_j x_k - \delta_{jk}|{\bf x}|^2}\over{|{\bf x}|^5}},
$$
och v{\aa}r Maclaurin-utveckling av viktsfunktionen $f({\bf x}')$ bist{\aa}r
direkt med respektive multipol-termer i uttrycket f{\"o}r skal{\"a}ra elektriska
potentialen som
\sidx{Taylor-utveckling}\sidx{Maclaurin-utveckling}
\sidx{Skal{\"a}r potential $\phi$}[Multipolutveckling]
$$
  \eqalign{
    \phi({\bf x})
      &={{1}\over{4\pi\varepsilon_0}}
        \iiint_{V} 
        \bigg(
          {{1}\over{|{\bf x}|}}
            +\sum^{3}_{k=1} {{x_k}\over{|{\bf x}|^3}} x'_k
            +{{1}\over{2}}\sum^{3}_{j=1}\sum^{3}_{k=1}
              {{3 x_j x_k - \delta_{jk}|{\bf x}|^2}\over{|{\bf x}|^5}} x'_j x'_k
            +\ldots
        \bigg) \rho({\bf x}')\,dV'\cr
      &={{1}\over{4\pi\varepsilon_0}}
        \Bigg(
        {{1}\over{|{\bf x}|}}
        \underbrace{
          \iiint_{V} \rho({\bf x}')\,dV'
        }_{{\rm monopol},\ q}
      +\sum^{3}_{k=1} {{x_k}\over{|{\bf x}|^3}}
        \underbrace{
          \iiint_{V} x'_k \rho({\bf x}')\,dV'
        }_{{\rm dipolmoment},\ p_k}
        \cr&\hskip150pt
      +{{1}\over{2}}\sum^{3}_{j=1}\sum^{3}_{k=1}
          {{3 x_j x_k - \delta_{jk}|{\bf x}|^2}\over{|{\bf x}|^5}}
        \underbrace{
          \iiint_{V} x'_j x'_k \rho({\bf x}')\,dV'
        }_{{\rm kvadrupolmoment},\ Q_{jk}}
      +\ldots
        \Bigg).\cr
  }
$$
Notera att f{\"o}r dipoltermen utg{\"o}rs skal{\"a}ra potentialen av en
{\it skal{\"a}rprodukt} ${\bf x}\cdot{\bf p}$, medan kvadrupoltermen svarar mot
en {\it matrisprodukt} fr{\aa}n vilken sp{\aa}ret (``trace'') subtraheras,
$3{\bf x}^{\rm T}{\Bbb Q}{\bf x}-|{\bf x}|^2\Tr[{\Bbb Q}]$.
\vfill\eject

\subsection{Sammanfattning av multipolutvecklingen av den skal{\"a}ra
  potentialen}
F{\"o}r att sammanfatta detta, s{\aa} har vi allts{\aa} kommit fram till att
den skal{\"a}ra potentialen resulterande fr{\aa}n en godtycklig
laddningsf{\"o}rdelning $\rho({\bf x})$ kan skrivas som en serieutveckling i
olika {\it moment} som
$$
  \eqalign{
    \phi({\bf x})
      &={{1}\over{4\pi\varepsilon_0}}
        \Bigg(
        {{1}\over{|{\bf x}|}}q
      +\sum^{3}_{k=1} {{x_k}\over{|{\bf x}|^3}}p_k
      +{{1}\over{2}}\sum^{3}_{j=1}\sum^{3}_{k=1}
          {{(3 x_j x_k - \delta_{jk}|{\bf x}|^2)}\over{|{\bf x}|^5}}Q_{jk}
      +\ldots
        \Bigg),
  }
$$
d{\"a}r
\halign{\hskip80pt#\hfil&\hskip40pt(#)\hfil\cr
  $\displaystyle q=\iiint_{V} \rho({\bf x}')\,dV'$
    &elektrisk monopol; enhet: ${\rm C}$\cr
  $\displaystyle p_k=\iiint_{V} x'_k \rho({\bf x}')\,dV'$
    &elektriskt dipolmoment; enhet: ${\rm C}\cdot{\rm m}$\cr
  $\displaystyle Q_{jk}=\iiint_{V} x'_j x'_k \rho({\bf x}')\,dV'$
    &elektriskt kvadrupolmoment; enhet: ${\rm C}\cdot{\rm m}^2$\cr
}
Analysen f{\"o}r vektorpotentialen ${\bf A}({\bf x})$ f{\"o}ljer p{\aa} samma
s{\"a}tt, med en helt och h{\aa}llet analog serieutveckling av
``viktfunktionen'' $f({\bf x}')$ f{\"o}r den elektriska str{\"o}mdensiteten
${\bf J}({\bf x})$.
\vfill\eject

\section{Multipolutveckling av vektorpotentialen f{\"o}r generella
  str{\"o}mt{\"a}theter}
\sidx{Multipolutveckling}[F{\"o}r vektorpotential]
\sidx{Multipolutveckling}[Generell str{\"o}mt{\"a}thet]
Motsvarande serieutveckling f{\"o}r vektorpotentialen f{\"o}ljer direkt fr{\aa}n
exakt samma serietveckling av ``viktfunktionen'' $f({\bf x}')$ i
k{\"a}llkoordinater som f{\"o}r den skal{\"a}ra potentialen, med den enda
skillnaden att vi nu tar fram vektorpotentialen ${\bf A}({\bf x})$ i
observationspunkten komponentvis fr{\aa}n str{\"o}mt{\"a}theten (k{\"a}llan)
${\bf J}({\bf x}')$, som
\sidx{Taylor-utveckling}\sidx{Maclaurin-utveckling}
\sidx{Vektorpotential ${\bf A}$}[Multipolutveckling]
$$
  \eqalign{
    {\bf A}({\bf x})
      &={{\mu_0}\over{4\pi}}
        \iiint_{V} 
        \bigg(
          {{1}\over{|{\bf x}|}}
            +\sum^{3}_{k=1} {{x_k}\over{|{\bf x}|^3}} x'_k
            +{{1}\over{2}}\sum^{3}_{j=1}\sum^{3}_{k=1}
              {{3 x_j x_k - \delta_{jk}|{\bf x}|^2}\over{|{\bf x}|^5}} x'_j x'_k
            +\ldots
        \bigg) {\bf J}({\bf x}')\,dV'\cr
      &={{\mu_0}\over{4\pi}}
        \Bigg(
        {{1}\over{|{\bf x}|}}
        \underbrace{
          \iiint_{V} {\bf J}({\bf x}')\,dV'
        }_{\vbox{\hbox{magnetisk}\vskip-5pt\hbox{``monopol''}}}
      +\sum^{3}_{k=1} {{x_k}\over{|{\bf x}|^3}}
        \underbrace{
          \iiint_{V} x'_k {\bf J}({\bf x}')\,dV'
        }_{\vbox{\hbox{magnetiskt}\vskip-5pt\hbox{dipolmoment, ${\bf m}$}}}
        \cr&\hskip140pt
      +{{1}\over{2}}\sum^{3}_{j=1}\sum^{3}_{k=1}
          {{3 x_j x_k - \delta_{jk}|{\bf x}|^2}\over{|{\bf x}|^5}}
        \underbrace{
          \iiint_{V} x'_j x'_k {\bf J}({\bf x}')\,dV'
        }_{\vbox{\hbox{magnetiskt kvadrupol-}\vskip-5pt\hbox{moment, $m_{jk}$}}}
      +\ldots
        \Bigg).\cr
  }
$$
Tolkningen av de ing{\aa}ende magnetiska momenten, som nu involverar
str{\"o}mt{\"a}theten ${\bf J}$ snarare {\"a}n den statiska
laddningst{\"a}theten $\rho$ som f{\"o}r den skal{\"a}ra potentialen,
blir dock lite annorlunda som vi nu skall se.

Ett antagande som vi beh{\"o}ver g{\"o}ra {\"a}r att str{\"o}mt{\"a}theten
${\bf J}$, likt den statiska laddnings\-f{\"o}r\-delningen $\rho$ i fallet
f{\"o}r den skal{\"a}ra potentialen $\phi$, m{\aa}ste uppfylla att den {\"a}r
{\it lokaliserad}, det vill s{\"a}ga att den g{\aa}r mot noll f{\"o}r ett
tillr{\"a}ckligt stort avst{\aa}nd fr{\aa}n origo (som vi valt att lokalisera
n{\aa}gonstans inuti k{\"a}llans utstr{\"a}ckning).

\subsection{Magnetisk monopolterm}
\sidx{Magnetiska monopoler}[Icke-existens av, statiskt]
\sidx{Magnetiska monopoler}[Fr{\aa}n multipolutveckling av vektorpotential]
\sidx{Vektorpotential ${\bf A}$}[Magnetisk monopolterm]
L{\aa}t oss f{\"o}rst av allt ta oss an den magnetiska
``monopolen''.\numberedfootnote{Fr{\aa}n
  F{\"o}rel{\"a}sning~4 och Gauss lag f{\"o}r den magnetiska
  fl{\"o}dest{\"a}theten $\nabla\cdot{\bf B}=0$ s{\aa} vet vi
  {\it a priori} att denna term kommer att vara noll, s{\aa}
  i princip skulle vi h{\"a}r bara kunna konstatera detta faktum;
  dock finns det ett egenv{\"a}rde i att faktiskt visa detta {\"a}ven
  utifr{\aa}n egenskaper hos str{\"o}mt{\"a}theten ${\bf J}$.}
\sidx{Lagen om att laddning inte kan f{\"o}rsvinna}
Fr{\aa}n ``lagen om att laddning inte kan f{\"o}rsvinna''\numberedfootnote{Se
  F{\"o}rel{\"a}sning~4 eller Griffiths Ekv.~(5.29), sid.~222.}
har vi i statiska problem att
$$
  \nabla\cdot{\bf J}=-{{d\rho}\over{dt}}=0,
$$
vilket i sin tur, genom att tolka $\nabla\cdot{\bf J}=0$ via
vektoridentiteten\numberedfootnote{Se exempelvis innerp{\"a}rmen p{\aa}
  Griffiths, {\it Second Derivatives (9)}.}
$\nabla\cdot(\nabla\times{\bf a})=0$, ger att vi kan tolka den
{\it magnetostatiska}\numberedfootnote{Notera att detta endast generellt
  g{\"a}ller f{\"o}r {\it statiska} problem, och ej generellt f{\"o}r
  (elektro-){\it dynamiska} fall.}
str{\"o}mt{\"a}theten som en rotation
$$
  {\bf J}=\nabla\times{\bf K},
$$
f{\"o}r n{\aa}gon vektorv{\"a}rd funktion ${\bf K}={\bf K}({\bf x})$, som vi
f{\"o}r stunden l{\aa}ter bli att definiera i detalj.
Denna form av den magnetostatiska str{\"o}mt{\"a}theten betyder att
monopoltermen antar formen
$$
  \eqalign{
    \iiint_{V} {\bf J}({\bf x}')\,dV'
      &=\iiint_{V} \nabla\times{\bf K}({\bf x}')\,dV'\cr
      &=\big\{\hbox{ Rotations-versionen av Gauss teorem }\big\}\cr
      &=\oiint_{S} d{\bf S}'\times{\bf K}({\bf x}').\cr
  }
$$
Eftersom denna identitet g{\"a}ller f{\"o}r en godtycklig volum $V$ med
omslutande gr{\"a}nsyta $S$, s{\aa} kan vi v{\"a}lja denna som exempelvis
en sf{\"a}r med radie $r\to\infty$. Med antagandet att vi har en str{\"o}m
som {\"a}r lokaliserad (ej o{\"a}ndligt utspridd) s{\aa} kan vi alltid v{\"a}lja
funktionen ${\bf K}({\bf x})$ s{\aa} att den liksom ${\bf J}({\bf x})$ g{\aa}r
mot noll d{\aa} $r\to\infty$. Den enda m{\"o}jligheten {\"a}r d{\"a}rmed att
den inneslutna magnetiska laddningen, eller om man s{\aa} vill magnetiska
monopolen, m{\aa}ste vara noll,
$$
  \iiint_{V} {\bf J}({\bf x}')\,dV'=0.
$$
Denna h{\"a}rledning l{\"a}mnar dock en viss k{\"a}nsla av icke-stringens efter
sig, d{\aa} vi ju kan fr{\aa}ga oss om vi verkligen har visat att magnetiska
monopoler inte kan existera bara f{\"o}r att den {\it inneslutna magnetiska
netto-laddningen} alltid {\"a}r noll. Trots allt, s{\aa} kan vi ju exempelvis
mycket v{\"a}l ha en innesluten elektrisk laddning som till sin total {\"a}r
noll, trots att vi mycket v{\"a}l vet att elektriska monopoler
(de elementarladdningar som {\"a}r de diskreta byggblocken som bygger upp en
elektrisk totalladdning) existerar i h{\"o}gsta grad.

Det argumentet ovan visar {\"a}r att hur vi {\"a}n v{\"a}nder och vrider p{\aa}
problemet, s{\aa} visar slutsatsen att den inst{\"a}ngda magnetiska
``laddningen'' alltid summerar till exakt noll p{\aa} att magnetiska
``laddningar'' (magnetiska monopoler) alltid m{\aa}ste upptr{\"a}da parvis,
vilket vi f{\"o}rvisso kan f{\"o}rest{\"a}lla oss som Gilberts
modell\numberedfootnote{Se F{\"o}rel{\"a}sning~7, {\it Magnetiska
  f{\"a}lt i material}.}
av magnetiska dipoler med associerade nord- och sydpoler, men som i en mer
grundl{\"a}ggande form snarare b{\"o}r ses som individuella magnetiska
{\it dipolmoment} som i Amp\`eres modell.

\subsection{Fysikalisk tolkning av magnetisk monopolterm}
\sidx{Vektorpotential ${\bf A}$}[Magnetiska monopolterm]
Eftersom vi i magneto-statiken har att $\nabla\cdot{\bf J}=0$, det vill
s{\"a}ga att str{\"o}mt{\"a}theten {\"o}verallt {\"a}r {\it divergensfri},
s{\aa} inneb{\"a}r detta att {\it varje str{\"o}mlinje l{\"a}ngs
str{\"o}mt{\"a}theten bildar en sluten slinga}.
Detta betyder i sin tur att varje str{\"o}mslingas bidrag m{\aa}ste vara i
form av ett magnetiskt dipolmoment, med en perfekt matchning mellan ``negativ
och positiv magnetisk laddning'', vilket i sin tur betyder att ingen isolerad
magnetisk laddning kan existera. Med andra ord, {\it inga magnetiska monopoler
kan existera}, och den magnetiska monopoltermen m{\aa}ste vara identiskt noll
{\"o}verallt.

\subsection{Magnetisk dipolterm}
\sidx{Magnetiskt dipolmoment}[Fr{\aa}n multipolutveckling av vektorpotential]
\sidx{Vektorpotential ${\bf A}$}[Magnetiskt dipolmoment]
Den magnetiska dipoltermen kan uttryckas som
$$
  \eqalign{
%    {\bf m}&=\iiint_{V} x'_k {\bf J}({\bf x}')\,dV'\cr
%      &=\iiint_{V} x'_k {\bf J}({\bf x}')\,dV'\cr
    \sum^{3}_{k=1} {{x_k}\over{|{\bf x}|^3}}\iiint_{V} x'_k {\bf J}({\bf x}')\,dV'
      &={{1}\over{|{\bf x}|^3}}\iiint_{V}
        \underbrace{
          ({\bf x}\cdot{\bf x}'){\bf J}({\bf x}')}
          _{\hbox{``$({\bf a}\cdot{\bf b}){\bf c}$''}}\,dV'\cr
      &=\big\{\hbox{ Vektoridentitet ${\bf a}\times({\bf b}\times{\bf c})=
          {\bf b}({\bf a}\cdot{\bf c})-{\bf c}({\bf a}\cdot{\bf b})$ }\big\}\cr
      &={{1}\over{|{\bf x}|^3}}\iiint_{V}\big[
              {\bf x}\times({\bf x}'\times{\bf J}({\bf x}'))
                +{\bf x}'({\bf x}\cdot{\bf J}({\bf x}'))
            \big]\,dV'\cr
  }
$$






\subsection{Magnetisk kvadrupolterm}
\sidx{Magnetiskt kvadrupolmoment}[Fr{\aa}n multipolutveckling av
  vektorpotential]
\sidx{Vektorpotential ${\bf A}$}[Magnetiskt kvadrupolmoment]
[TO BE CONTINUED]

\vfill\eject

\section{Kan dynamiska magnetiska monopoler existera?}
\sidx{Magnetiska monopoler}[Icke-existens av, dynamiskt]
Vi har just precis p{\aa} ett alternativt s{\"a}tt, ut{\"o}ver det tidigare
$\nabla\cdot{\bf B}=0$ fr{\aa}n exempelvis F{\"o}rel{\"a}sning~4, visat att
{\it magnetiska monopoler inte existerar i magneto-statiska problem}.%
\numberedfootnote{Vi kan rekapitulera att vi inom magnetostatiken,
  i F{\"o}rel{\"a}sning~4, fann $\nabla\cdot{\bf B}=0$ som en direkt
  f{\"o}ljd av formen p{\aa} Biot--Savarts lag f{\"o}r en generell
  str{\"o}mt{\"a}thet, samt fr{\aa}n identiteten
  $$
    \nabla\times{{({\bf x}-{\bf x}')}\over{|{\bf x}-{\bf x}'|^3}}
      =\underbrace{
         \nabla\times\bigg(-\nabla{{1}\over{|{\bf x}-{\bf x}'|}}\bigg)
       }_{\nabla\times(\nabla f)\equiv 0}
      =0.
  $$}
Fr{\aa}gan infinner sig d{\aa} f{\"o}rst{\aa}s om det finns m{\"o}jlighet att
magnetiska monopoler existerar om vi till{\aa}ter statiken att g{\aa} {\"o}ver
till {\it dynamik}, d{\"a}r vi till{\aa}ter ett tidsberoende hos
laddningsf{\"o}rdelningen $\rho$?

Det enklaste argumentet f{\"o}r att magnetiska monopoler ej heller kan existera
i en dynamisk regim f{\"o}ljer fr{\aa}n det klassiska statiska argumentet att
$\nabla\cdot{\bf B}=0$. Vi kan generalisera detta argument till en lika giltig
{\it elektrodynamisk} lag genom att applicera divergensen p{\aa} Faradays
lag,\numberedfootnote{Se F{\"o}rel{\"a}sning~5, {\it Faradays lag p{\aa}
  differentialform}, eller Griffiths Ekv.~(7.16),
  sid.~313.}
$$
  \underbrace{
    \nabla\cdot\big(\nabla\times{\bf E}({\bf x},t)\big)
  }_{\nabla\cdot(\nabla\times{\bf a})\equiv0}
  =-\nabla\cdot\Big({{\partial{\bf B}({\bf x},t)}\over{\partial t}}\Big)
  =-{{\partial}\over{\partial t}}\nabla\cdot{\bf B}({\bf x},t)
  =0.
$$
Den sista likheten kan trivialt integreras till
$$
  \nabla\cdot{\bf B}({\bf x},t)=\hbox{konstant},
$$
och om denna relation skall g{\"a}lla f{\"o}r alla tider $t$, s{\aa} m{\aa}ste
den {\"a}ven g{\"a}lla f{\"o}r n{\aa}gon start-tid (eller f{\"o}r den delen
senare tid) d{\aa} vi r{\aa}kar ha en {\it statisk} konfiguration vid vilken
$\nabla\cdot{\bf B}({\bf x},t)=0$. Enda m{\"o}jligheten {\"a}r d{\"a}rmed att
vi {\"a}ven {\it elektrodynamiskt} har att
$$
  \nabla\cdot{\bf B}({\bf x},t)=0,
$$
vilket direkt ger vid hand att {\it magnetiska monopoler inte heller kan
existera  dynamiskt.} Argumentet ovan betyder ocks{\aa} att Gauss lag f{\"o}r
den magnetiska fl{\"o}dest{\"a}theten {\"a}r identisk i magnetostatiska
s{\aa}v{\"a}l som i elektrodynamiska situationer, n{\aa}got som vi kommer att
anv{\"a}nda i F{\"o}rel{\"a}sning~9 d{\aa} vi s{\"a}tter samman den generella
formen av {\it Maxwell's ekvationer.}
\vfill\eject

\section{Dipolapproximationen f{\"o}r station{\"a}ra laddningsf{\"o}rdelningar
   och str{\"o}mmar}
\sidx{Dipolapproximationen}
F{\"o}r att sammanfatta ges den station{\"a}ra skal{\"a}ra potentialen
$\phi({\bf x})$ och vektorpotentialen ${\bf A}({\bf x})$ fr{\aa}n en
laddnings- och str{\"o}mt{\"a}thet i n{\"a}rheten av origo, i dipolapproximation
och p{\aa} l{\aa}ngt avst{\aa}nd fr{\aa}n k{\"a}llan, som
\sidx{Skal{\"a}r potential $\phi$}
\sidx{Vektorpotential ${\bf A}$}
\sidx{Elektrisk f{\"a}ltstyrka ${\bf E}$}
\sidx{Magnetisk fl{\"o}dest{\"a}thet ${\bf B}$}
$$
  \phi({\bf x})\approx{{1}\over{4\pi\varepsilon_0}}\bigg(
      {{q}\over{|{\bf x}|}} + {{{\bf p}\cdot{\bf x}}\over{|{\bf x}|^3}}
    \bigg),\qquad\qquad
  {\bf A}({\bf x})\approx{{\mu_0}\over{4\pi}}
      {{{\bf m}\times{\bf x}}\over{|{\bf x}|^3}},
$$
d{\"a}r
$$
  q=\underbrace{
      \iiint_{V} \rho({\bf x}')\,dV'
    }_{{\rm nettoladdning}\ [{\rm C}]},\qquad
  {\bf p}=\underbrace{
      \iiint_{V} x'_k \rho({\bf x}')\,dV'
    }_{{\rm elektriskt\ dipolmoment}\ [{\rm C}{\rm m}]},\qquad
  {\bf m}=\underbrace{
      {{1}\over{2}}\iiint_{V} {\bf x}'\times{\bf J}({\bf x}')\,dV'
    }_{{\rm magnetiskt\ dipolmoment}\ [{\rm A}{\rm m}^2]}.
$$
Fr{\aa}n dessa ges de {\it statiska} elektriska och magnetiska f{\"a}lten
som\numberedfootnote{Notera att Ekv.~(8.1.5) i Olov {\AA}grens
  {\it Elektromagnetism} (Studentlitteratur, 2014) definierar magnetiska
  f{\"a}ltet utifr{\aa}n en konstruerad {\it magnetisk skal{\"a}r potential}
  ist{\"a}llet f{\"o}r vektorpotentialen. Vektorpotentialen {\"a}r h\"ar dock
  mest naturlig att anv{\"a}nda, utifr{\aa}n den grundl{\"a}ggande egenskapen
  $\nabla\cdot{\bf B}=0\Leftrightarrow{\bf B}=\nabla\times{\bf A}$ hos
  magnetiska f{\"a}ltet, med ${\bf A}$ som vektorpotentialen.}
$$
  {\bf E}({\bf x})=-\nabla\phi({\bf x}),\qquad\qquad
  {\bf B}({\bf x})=\nabla\times{\bf A}({\bf x}).
$$
\sidx{Magnetiska monopoler}[Icke-existens av, statiskt]
Man kan h{\"a}r fr{\aa}ga sig varf{\"o}r multipolutvecklingen f{\"o}r
vektorpotentialen ${\bf A}$ b{\"o}rjar med dipolmomentet, som g{\aa}r som
$\sim 1/|{\bf x}|^3$, och inte som den skal{\"a}ra potentialen inneh{\aa}ller
n{\aa}gon term som g{\aa}r som $\sim 1/|{\bf x}|$? Svaret p{\aa} denna
fr{\aa}ga {\"a}r sj{\"a}lvfallet att vektorpotentialen, som {\"a}r direkt
l{\"a}nkad till magnetf{\"a}ltet genom ${\bf B}=\nabla\times{\bf A}$, till
skillnad fr{\aa}n det elektrostatiska fallet {\it aldrig kan involvera
magnetiska monopoler}, varf{\"o}r monopoltermer ocks{\aa} saknas f{\"o}r
just vektorpotentialen.
\vfill\eject

\section{Sammanfattning av F{\"o}rel{\"a}sning~8 -- Multipolutvecklingen}
\item{$\bullet$}{Multipolutvecklingen {\"a}r i grund och botten bara en
  serieutveckling av potentialen (skal{\"a}r eller vektor) i inversa
  potenser av distansen mellan k{\"a}lla och observationspunkt.}
\item{$\bullet$}{
  Vi tar som vanligt fram uttrycken f{\"o}r potentialerna som
  $$
    \phi({\bf x})={{1}\over{4\pi\varepsilon_0}}
      \iiint_{{\Bbb R}^3}{{\rho({\bf x}')}\over{|{\bf x}-{\bf x}'|}}\,dV',
      \qquad\qquad
    {\bf A}({\bf x})={{\mu_0}\over{4\pi}}
      \iiint_{{\Bbb R}^3}{{{\bf J}({\bf x}')}\over{|{\bf x}-{\bf x}'|}}\,dV'.
  $$}
\item{$\bullet$}{Multipolutvecklingen klassificeras term f{\"o}r term
  utifr{\aa}n hur beroendet av avst{\aa}ndet $r=|{\bf x}-{\bf x}'|$ mellan
  k{\"a}llpunkt ${\bf x}'$ och f{\"a}ltpunkt ${\bf x}$ ser ut:
  \litem[1.]{\hbox to110pt{Monopol (1-pol):\hfil}
    $\phi({\bf x})\sim 1/r$}
  \litem[2.]{\hbox to110pt{Dipol (2-pol):\hfil}
    $\phi({\bf x})\sim 1/r^2$}
  \litem[3.]{\hbox to110pt{Kvadrupol (4-pol):\hfil}
    $\phi({\bf x})\sim 1/r^3$}
  \litem[4.]{\hbox to110pt{Oktopol (8-pol):\hfil}
    $\phi({\bf x})\sim 1/r^4$}
  \litem[5.]{\hbox to110pt{Hexadecapol (16-pol):\hfil}
    $\phi({\bf x})\sim 1/r^5$}
  \litem[6.]{\hbox to110pt{Dotriacontapol (32-pol):\hfil}
    $\phi({\bf x})\sim 1/r^6$}}
\item{$\bullet$}{Multipolutvecklingen f{\"o}r den skal{\"a}ra potentialen
  f{\"o}r en linj{\"a}r elektrisk {\it dipol} ges som
  $$
    \phi({\bf x})
       ={{q}\over{2\pi\varepsilon_0}}\Bigg(
         \underbrace{{{b\cos\theta}\over{r^2}}}_{\rm dipol}
         +\underbrace{{{(b\cos\theta)^3}\over{r^4}}}_{\rm oktopol}
         +\underbrace{{{(b\cos\theta)^5}\over{r^6}}}_{\rm dotriacontapol}
         +\ldots
        \Bigg).
  $$
  {\it Notera att {\"a}ven den elektriska dipolen inneh{\aa}ller
  n{\"a}rf{\"a}lts-termer svarande mot h{\"o}gre ordningars moment!}}
\item{$\bullet$}{Multipolutvecklingen f{\"o}r den skal{\"a}ra potentialen
  f{\"o}r en linj{\"a}r elektrisk {\it kvadrupol} ges som
  $$
    \phi({\bf x})
      ={{q}\over{2\pi\varepsilon_0}}\Bigg(
         \underbrace{{{(b\cos\theta)^2}\over{r^3}}}_{\rm kvadrupol}
         +\underbrace{{{(b\cos\theta)^4}\over{r^5}}}_{\rm hexadecapol}
         +\underbrace{{{(b\cos\theta)^6}\over{r^7}}}_{\rm hexacontatetrapol}
         +\ldots
        \Bigg).
  $$
  {\it Notera att den elektriska kvadrupolen saknar dipolmoment!}}
\item{$\bullet$}{Dipolapproximationen f{\"o}r elektrostatiska
  (${\bf E}=-\nabla\phi$) och magnetostatiska (${\bf B}=\nabla\times{\bf A}$)
  f{\"a}lt p{\aa} l{\aa}ngt avst{\aa}nd $|{\bf x}|$ fr{\aa}n
  laddningsf{\"o}rdelningen (k{\"a}llan) $\rho$ lyder
  $$
    \phi({\bf x})\approx{{1}\over{4\pi\varepsilon_0}}\bigg(
        {{q}\over{|{\bf x}|}} + {{{\bf p}\cdot{\bf x}}\over{|{\bf x}|^3}}
      \bigg),\qquad\qquad
    {\bf A}({\bf x})\approx{{\mu_0}\over{4\pi}}
        {{{\bf m}\times{\bf x}}\over{|{\bf x}|^3}},
  $$
  d{\"a}r
  $$
    q=\underbrace{
        \iiint_{V} \rho({\bf x}')\,dV'
      }_{{\rm nettoladdning}\ [{\rm C}]},\qquad
    {\bf p}=\underbrace{
        \iiint_{V} x'_k \rho({\bf x}')\,dV'
      }_{{\rm elektriskt\ dipolmoment}\ [{\rm C}{\rm m}]},\qquad
    {\bf m}=\underbrace{
        {{1}\over{2}}\iiint_{V} {\bf x}'\times{\bf J}({\bf x}')\,dV'
      }_{{\rm magnetiskt\ dipolmoment}\ [{\rm A}{\rm m}^2]}.
  $$}

\cleardoublepage
%%% End of auto-extracted text from ../lect-08/lecture-08.tex %%%
%%% Begin of auto-extracted text from ../lect-09/lecture-09.tex %%%
%
% File: teach/elmagii/lect-09/lecture-09.tex [plain TeX code]
% Github: https://github.com/elmagii/lect-09/
% Last change: November 21, 2024
%
% Lecture No 9 in the course ``Elektromagnetism II, 1TE626 (2023)'',
% held November 25, 2025, at Uppsala University, Sweden.
%
% Copyright (C) 2022-2025, Fredrik Jonsson, under Gnu General Public
% License (GPL) v3. See the enclosed LICENSE for details.
%
% This program is free software: you can redistribute it and/or modify
% it under the terms of the GNU General Public License as published by
% the Free Software Foundation, either version 3 of the License, or
% (at your option) any later version.
%
% This program is distributed in the hope that it will be useful,
% but WITHOUT ANY WARRANTY; without even the implied warranty of
% MERCHANTABILITY or FITNESS FOR A PARTICULAR PURPOSE.  See the
% GNU General Public License for more details.
%
% You should have received a copy of the GNU General Public License
% along with this program.  If not, see <https://www.gnu.org/licenses/>.
%
\def\coursename{Elektromagnetism II}
\def\coursecode{1TE626}
\def\courseyear{2025}
\def\courserepo{https://github.com/hp35/elmagii/}
\def\lecturenumber{9}
\def\lecturetitle{Maxwells ekvationer och v{\aa}gutbredning}
\def\lecturesubtitle{}
\def\lectureauthor{Fredrik Jonsson}
\def\lectureplace{Uppsala Universitet}
\def\lecturedate{25 november 2025}
%-------------------- BEGIN OF LOCAL MACROS --------------------
\edef\expandedlecturenumber{9}
\def\ifempty#1{\ifx\relax#1\relax}
\advance\chapno by 1
\secno=0
\footnotenumber=0
\message{==================== Lecture 9 ====================}
\writenumberedtocentry{chapter}{F{\"o}rel{\"a}sning 9 -- {Maxwells ekvationer och v{\aa}gutbredning}}{\thechapno}
\hsize=150mm\hoffset=4.6mm\vsize=230mm\voffset=7mm
\topskip=0pt\baselineskip=12pt\parskip=0pt\leftskip=0pt\parindent=15pt
\ifcolors
  \voffset=-10.2mm\topskip=0pt
\fi
\headline={\ifnum\secno>0\ifodd\pageno\rightheadline\else\leftheadline\fi
  \else\hfill\fi}
\def\rightheadline{\tenrm{\it F\"orel\"asning 9}
  \hfil{\it \coursename, \coursecode\ (\courseyear)}}
\def\leftheadline{\tenrm{\it \coursename, \coursecode\ (\courseyear)}
  \hfil{\it F\"orel\"asning 9}}
\noindent~\vskip-60pt\hskip-40pt{\epsfbox{../lect-01/macros/UU_logo_color.eps}}
\vskip-42pt\hfill\vbox{
    \hbox{{\it \coursename, \coursecode\ (\courseyear)}}
    \hbox{{\it Lecture Notes, \lectureauthor}}
    \hbox{{\it Document Revision \today}}
    \hbox{{\it \courserepo}}}\vskip 36pt
\centerline{\twelvesc F\"orel\"asning 9}
\vskip 24pt\noindent
\centerline{\twelvesc{Maxwells ekvationer och v{\aa}gutbredning}}
\expandafter\ifempty\expandafter{\lecturesubtitle}%
  \else\centerline{\twelvesc\lecturesubtitle}\fi
\bigskip
\centerline{\lectureauthor, \lectureplace, \lecturedate}
\vskip24pt
%--------------------- END OF LOCAL MACROS ---------------------



\plan{Faradays och Amp\`eres lagar sys tillsammans med Gauss lag f{\"o}r det
  elektriska och magnetiska f{\"a}ltet slutligen ihop till Maxwells ekvationer.
  Nyckeln till dessa kommer ifr{\aa}n Maxwells generalisering av Amp\`eres lag,
  d{\"a}r Maxwell kom p{\aa} att l{\"o}sningen till problemet med att
  kontinuitetsekvationen f{\"o}r elektrisk laddning inte uppfylldes f{\"o}r
  tidsberoende f{\"a}lt var att l{\"a}gga till en term i den fria
  str{\"o}mt{\"a}theten ${\bf J}_{\rm f}$, motsvarande
  f{\"o}rskjutningsstr{\"o}mmen $\partial{\bf D}/\partial t$.
  Med denna f{\"o}rskjutningsstr{\"o}m n{\"a}rvarande i Amp\`eres lag
  $\nabla\times{\bf H}={\bf J}_{\rm f}+\partial{\bf D}/\partial t$ satisfieras
  {\"a}ven kontinuitetsekvationen $\nabla\cdot{\bf J}=-\partial\rho/\partial t$.
  \sidx{Lagen om att laddning inte kan f{\"o}rsvinna}
  \sidx{Kontinuitetsekvationen}

  Vi sammanfattar Maxwells ekvationer en g{\aa}ng f{\"o}r alla, och visar hur
  dessa kan omformuleras till tv{\aa} v{\aa}gekvationer f{\"o}r de elektriska
  och magnetiska f{\"a}lten.}

\threepointsummary{%
  Maxwells ekvationer lyder
  \sidx{Maxwells ekvationer}[Differentialform]
  \sidx{Maxwells ekvationer}[Integralform]
  \sidx{Amp\`eres lag}\sidx{Faradays lag}
  \sidx{Gauss lag}[F{\"o}r elektrisk fl{\"o}dest{\"a}thet ${\bf D}$]
  \sidx{Gauss lag}[F{\"o}r magnetisk fl{\"o}dest{\"a}thet ${\bf B}$]
  $$\hbox{~\hskip33pt}
    \matrix{
    &
      \displaystyle\oiint{\bf D}\cdot d{\bf S}=\iiint\rho\,dV\hfill&
      \quad\Leftrightarrow\quad&
      \displaystyle\nabla\cdot{\bf D}=\rho\hfill&
      &\cr
    &
      \displaystyle\oiint{\bf B}\cdot d{\bf S}=0\hfill&
      \quad\Leftrightarrow\quad&
      \displaystyle\nabla\cdot{\bf B}=0\hfill&
      &\cr
    &
      \displaystyle\oint{\bf E}\cdot d{\bf l}
         =-{{\partial}\over{\partial t}}\iint{\bf B}\cdot d{\bf S}\hfill&
      \quad\Leftrightarrow\quad&
      \displaystyle\nabla\times{\bf E}
         =-{{\partial{\bf B}}\over{\partial t}}\hfill&
      &\cr
    &
      \displaystyle\oint{\bf H}\cdot d{\bf l}=\iint{\bf J}_{\rm f}\cdot d{\bf S}
         +{{\partial}\over{\partial t}}\iint{\bf D}\cdot d{\bf S}\hfill&
      \quad\Leftrightarrow\quad&
      \displaystyle\nabla\times{\bf H}={\bf J}_{\rm f}
         +{{\partial{\bf D}}\over{\partial t}}\hfill&
      &\cr
  }
$$
}{%
  De elektromagnetiska v{\aa}gekvationerna f{\"o}r de elektriska och
  magnetiska f{\"a}lten lyder\sidx{Elektromagnetiska v{\aa}gekvationer}
  \sidx{Elektrisk f{\"a}ltstyrka ${\bf E}$}
  \sidx{Magnetisk fl{\"o}dest{\"a}thet ${\bf B}$}
  $$
    \eqalign{
      \nabla\times\nabla\times{\bf E}
        +\mu_0\varepsilon_0{{\partial^2{\bf E}}\over{\partial t^2}}&=
           -\mu_0{{\partial}\over{\partial t}}
            \underbrace{
               \bigg({\bf J}_{\rm f}
                  +{{\partial{\bf P}}\over{\partial t}}
                  +\nabla\times{\bf M}\bigg)}_{\hbox{gemensam k{\"a}llterm}},\cr
      \nabla\times\nabla\times{\bf B}
        +\mu_0\varepsilon_0{{\partial^2{\bf B}}\over{\partial t^2}}&=
            \mu_0\nabla\times
            \underbrace{
               \bigg({\bf J}_{\rm f}
                  +{{\partial{\bf P}}\over{\partial t}}
                  +\nabla\times{\bf M}\bigg)}_{\hbox{gemensam k{\"a}llterm}}.\cr
  }
$$
}{%
  Fr{\aa}n de elektromagnetiska v{\aa}gekvationerna kan samtliga fall som
  hittills behandlas i kursen h{\"a}rledas, beroende p{\aa} vilka termer
  som kan s{\"a}ttas till noll och p{\aa} s{\aa} s{\"a}tt reducera systemet
  till den statiska eller dynamiska situation man {\"o}nskar analysera.
}
\vfill\eject\copyrights

\section{Maxwells generalisering av Amp\`eres lag}
\sidx{Amp\`eres lag}[Maxwells generalisering av]
S{\aa} l{\aa}ngt i kursen\numberedfootnote{Den f{\"o}ljande behandlingen i
  denna f{\"o}rel{\"a}sning, d{\"a}r vi till slut kommer fram till en mer
  generell form av Maxwell's ekvationer {\"a}n vad man som oftast st{\"o}ter
  p{\aa} i standard-textb{\"o}cker, f{\"o}ljer i huvudsak Griffiths kapitel
  7.3. Dock tycker jag att Griffiths saknar po{\"a}ngen att k{\"a}lltermerna
  f{\"o}r s{\aa}v{\"a}l det elektriska som magnetiska f{\"a}ltet i sj{\"a}lva
  verket har en gemensam form, best{\aa}ende av den fria str{\"o}mt{\"a}theten,
  tidsderivatan av den elektriska polarisationsdensiteten, samt rotationen av
  magnetf{\"a}ltet. Min f{\"o}rhoppning {\"a}r att analysen h{\"a}r, om {\"a}n
  n{\aa}got mer omfattande, skall belysa denna {\it fundamenta} och v{\"a}cka
  intresset f{\"o}r denna oerh{\"o}rt vackra del av elektromagnetisk teori.}
har Maxwells ekvationer baserats p{\aa} frirymds-formen
$$
  \nabla\cdot{\bf E}={{\rho}\over{\varepsilon_0}},\qquad
  \nabla\times{\bf E}=-{{\partial{\bf B}}\over{\partial t}},\qquad
  \nabla\cdot{\bf B}=0,
$$
samt med Amp\`eres lag (s{\aa} l{\aa}ngt) p{\aa} den {\it statiska} formen
$$
  \hskip70pt\nabla\times{\bf B}=\mu_0{\bf J}.\hskip30pt(\hbox{Statisk!})
$$
Utifr{\aa}n en {\it elektrostatisk} betraktelse, s{\aa} ser vi direkt att det
elektriska f{\"a}ltet har sin k{\"a}lla i statiska elektriska laddningar, via
Gauss lag f{\"o}r det elektriska f{\"a}ltet, medan det {\it statiska}
magnetf{\"a}ltet i sin tur har sin k{\"a}lla i motsvarande {\it r{\"o}relse}
av de elektriska laddningarna (det vill s{\"a}ga {\it str{\"o}m}).
Redan i Faradays lag ovan har vi dock en f{\"o}rsta koppling till en
{\it elektrodynamisk} koppling i och med att tidsvariationen hos det magnetiska
f{\"a}ltet kopplar till rotationen hos det elektriska f{\"a}ltet.
Denna koppling sker dessutom via ett omv{\"a}nt tecken, vilket i grund och
botten fastst{\"a}ller {\it Lenz lag} (som s{\"a}ger att en inducerad
elektromotorisk kraft alltid riktas s{\aa} att den motverkar k{\"a}llan som
inducerade den).

Som ett grundl{\"a}ggande krav f{\"o}r att dessa tre ekvationer skall vara
fysikaliskt korrekta m{\aa}ste dessutom {\it lagen om att elektrisk laddning
inte kan f{\"o}rsvinna}\numberedfootnote{Notera att man talade om elektrisk
  laddning l{\aa}ngt innan elektronen 1897 uppt{\"a}cktes experimentellt av
  brittiske fysikern och nobelpristagaren Joseph John Thomson. Elektronens
  existens framf{\"o}rdes dock som hypotes 1838 av den brittiske naturfilosofen
  Richard Laming, d.v.s. mindre {\"a}n 20 {\aa}r innan Maxwell konsoliderade
  ``sina'' ekvationer.}
vara uppfylld,\numberedfootnote{Vi kan rekapitulera att denna h{\"a}rleddes i
  F{\"o}rel{\"a}sning~4, utifr{\aa}n Gauss lag applicerad p{\aa} den totala
  str{\"o}m som passerar ut genom en {\it sluten} yta $S$ omslutande en
  volym $V$,
  $$
    I=\Big[\hbox{str{\"o}mmen ut genom ytan $S$}\Big]
     =\oiint_S {\bf J}\cdot d{\bf S}
     =\iiint_V (\nabla\cdot{\bf J})\,dV.
  $$
  Eftersom ingen laddning kan skapas eller f{\"o}rintas internt i volymen (vi
  erinrar oss att all laddnings\-transport in eller ut fr{\aa}n volymen sker
  genom den slutna ytan $S$, s{\aa} m{\aa}ste den laddning som fl{\"o}dar
  {\it ut genom ytan} g{\"o}ra att den i volymen $V$ {\it inneslutna laddningen
  minskar} i motsvarande grad, det vill s{\"a}ga
  $$
    \iiint_V (\nabla\cdot{\bf J})\,dV
      =-{{d}\over{dt}}\Big[\hbox{Innesluten laddning}\Big]
      =-{{d}\over{dt}}\iiint_V \rho\,dV
      =-\iiint_V {{d\rho}\over{dt}}\,dV
  $$
  \sidx{Lagen om att laddning inte kan f{\"o}rsvinna}
  Notera att tecknet f{\"o}r $-\partial\rho/\partial t$ i
  ``lagen om att elektrisk laddning inte kan f{\"o}rsvinna''
  h{\"a}nger ihop med att str{\"o}mmen ${\bf J}$ r{\"a}knas som
  positiv {\it ut} fr{\aa}n punkten f{\"o}r {\it positiv} laddningstransport.
  Med $\nabla\cdot{\bf J}({\bf x})>0$ m{\aa}ste vi d{\"a}rf{\"o}r ha en
  {\it ackumulering av negativ laddning} i punkten ${\bf x}$, och d{\"a}rmed
  att $\partial\rho({\bf x})/\partial t < 0$.}
$$
  {{\partial\rho}\over{\partial t}}+\nabla\cdot{\bf J}=0.
$$
\sidx{Lagen om att laddning inte kan f{\"o}rsvinna}
\epsfig{../lect-09/figs/chargeconsv.1}\noindent
L{\aa}t oss d{\"a}rf{\"o}r en g{\aa}ng f{\"o}r alla kontrollera att den
elektriska laddningen {\"a}r bevarad utifr{\aa}n ekvationerna ovan!
Fr{\aa}n den {\it statiska} formen av Amp\`eres lag har vi dessv{\"a}rre att
$$
  \nabla\cdot{\bf J}
    = {{1}\over{\mu_0}}\nabla\cdot\underbrace{(\nabla\times{\bf B})}_{=\mu_0{\bf J}}
    \equiv \{\ \hbox{vektoridentitet}\ \}
    \equiv 0,
$$
{\it fast vi egentligen skulle beh{\"o}vt ett ``$-\partial\rho/\partial t$'' i
h{\"o}gerledet ist{\"a}llet f{\"o}r en nolla.} Detta visar tydligt hur Amp\`eres
lag p{\aa} formen ovan (i sin statiska form) {\it ej} kan till{\"a}mpas p{\aa}
generella tidsberoende problem.

Ett av James Clerk Maxwells (1831--1879, skotsk matematiker och fysiker)
\sidx{Maxwell, James Clerk (1831--1879)} viktigare bidrag till elektromagnetisk
f{\"a}ltteori var n{\"a}r han 1856 ins{\aa}g hur Amp\`eres lag b{\"o}r
modifieras f{\"o}r att dessutom uppfylla kravet p{\aa} bevaring av elektrisk
laddning.
Maxwells argument {\"a}r anm{\"a}rkningsv{\"a}rt enkelt och rakt p{\aa} sak,
och lyder i stort som f{\"o}ljer: Om vi fr{\aa}n kravet p{\aa} bevaring av
laddning har att ``$-\partial\rho/\partial t$'' saknas i h{\"o}gerledet ovan,
{\it kan vi d{\aa} inte helt enkelt l{\"a}gga till en s{\aa}dan term och se
vad som h{\"a}nder?}
$$
    \nabla\cdot{\bf J}
       = {{1}\over{\mu_0}}\nabla\cdot(\nabla\times{\bf B})
       \eqq -{{\partial\rho}\over{\partial t}}
       = -{{\partial}\over{\partial t}}(\varepsilon_0\nabla\cdot{\bf E})
       = -\varepsilon_0\nabla\cdot{{\partial{\bf E}}\over{\partial t}}.
$$
Vi anv{\"a}nde i mellansteget ovan Gauss lag f{\"o}r elektrisk laddning.
Uppenbarligen s{\aa} skulle en modifiering av Amp\`eres ``statiska'' lag med
den (fria) str{\"o}mt{\"a}theten ersatt med
\sidx{Str{\"o}mt{\"a}thet}[Fria laddningar]
$$
  {\bf J}\to{\bf J}+\varepsilon_0{{\partial{\bf E}}\over{\partial t}}
$$
direkt l{\"o}sa problemet med bevaring av laddning f{\"o}r tidsberoende problem,
och detta {\"a}r ocks{\aa} sj{\"a}lva k{\"a}rnan i den modifiering som Maxwell
introducerade.

I n{\"a}rvaron av fria str{\"o}mmar (men utan att befinna sig i ett medium som
kan polariseras elektriskt) {\"a}r d{\"a}rf{\"o}r den generella (tidsberoende)
formen av Amp\`eres lag
$$
  \nabla\times{\bf B}=\mu_0\bigg({\bf J}
    +\varepsilon_0{{\partial{\bf E}}\over{\partial t}}\bigg),
$$
d{\"a}r\sidx{F{\"o}rskjutningsstr{\"o}m}
$$
  \varepsilon_0{{\partial{\bf E}}\over{\partial t}}
    =\hbox{``f{\"o}rskjutningsstr{\"o}mmen'' (i vakuum, paradoxalt nog!)}
$$
\sidx{Elektrisk permittivitet}[Vakuumpermittivitet $\varepsilon_0$]
Att det {\"a}r just $\varepsilon_0{\bf E}$ som dyker upp i derivatan, och att
denna term samtidigt finns naturligt i uttrycket f{\"o}r den elektriska
polarisationsdensiteten \sidx{Elektrisk polarisationsdensitet ${\bf P}$} hos
ett material\numberedfootnote{Se exempelvis F{\"o}rel{\"a}sning~8.} leder oss
till att fundera p{\aa} om det i det generella fallet (inuti ett elektriskt
polariserbart medium, och inte bara i vakuum) i sj{\"a}lva verket inte
snarare {\"a}r den elektriska fl{\"o}dest{\"a}theten ${\bf D}=\varepsilon_0
{\bf E}+{\bf P}$ som borde ing{\aa}.
\sidx{Elektrisk fl{\"o}dest{\"a}thet ${\bf D}$}
I sj{\"a}lva verket {\"a}r det exakt s{\aa} som den
korrekta generella formen {\"a}r, som\numberedfootnote{Vi inf{\"o}r h{\"a}r
   beteckningen ${\bf J}_{\rm f}$ f{\"o}r den {\it fria} str{\"o}mt{\"a}theten,
   identisk med all str{\"o}mt{\"a}thet som diskuterats tidigare i kursen, bara
   f{\"o}r att vara noga med att s{\"a}rskilja denna fr{\aa}n r{\"o}relsen hos
   {\it bundna} laddningar hos mediet. Notera att f{\"o}r statiska problem
   {\"a}r ${\bf E}$ tidsoberoende och Amp\`eres statiska lag op{\aa}verkad
   av denna modifikation.\sidx{Str{\"o}mt{\"a}thet}[Bundna laddningar]}
$$
  {\bf J}_{\rm tot} = {\bf J}_{\rm f}+{{\partial{\bf D}}\over{\partial t}},
$$
d{\"a}r ${\bf J}_{\rm f}$ {\"a}r str{\"o}mmen av {\it fria laddningar}, medan
${{\partial{\bf D}}/{\partial t}}$ {\"a}r ``str{\"o}mmen'' av {\it bundna
laddningar}. Maxwell sj{\"a}lv kallade denna form f{\"o}r ``{\it The Law of
Total Currents}''.
\sidx{Maxwell, James Clerk (1831--1879)}[{{\it The Law of Total Currents}}]

S{\aa} varf{\"o}r kallar vi ${{\partial{\bf D}}/{\partial t}}$ f{\"o}r just
``f{\"o}rskjutningsstr{\"o}m''? Till att b{\"o}rja med dyker denna term upp
som en naturlig korrektion till str{\"o}mmen ${\bf J}_{\rm f}$ av fria
laddningar, och har sj{\"a}lvfallet samma fysikaliska dimension som
str{\"o}mt{\"a}thet (${\rm C}/{\rm m}^2)$. Ut{\"o}ver detta, s{\aa} involverar
${\bf D}$ {\"a}ven den elektriska polarisationen av mediet, med en
{\it f{\"o}rskjutning} av elektrisk laddning som en direkt f{\"o}ljd av det
p{\aa}lagda elektriska f{\"a}ltet. Tidsderivatan av denna f{\"o}rskjutning av
elektrisk laddning blir d{\aa} en effektiv elektrisk {\it str{\"o}m}
(f{\"o}rflyttning av elektrisk laddning per tidsenhet), och d{\"a}rav att
${{\partial{\bf D}}/{\partial t}}$ {\"a}r en effektiv
``f{\"o}rskjutningsstr{\"o}m''. Viktigt h{\"a}r {\"a}r att notera att en
{\it str{\"o}m} kan utg{\"o}ras {\"a}ven av {\it bundna} laddningar,
exempelvis i ett annars icke elektriskt ledande dielektrikum.
\epsfig{../lect-09/figs/poldensity.1}\noindent
Exempelvis {\"a}r en oscillerande molekyl{\"a}r elektrisk dipol ett exempel
p{\aa} denna f{\"o}rskjutningsstr{\"o}m, vilket vi kan se som en elektriskt
driven antenn p{\aa} mikroskopisk niv{\aa}, d{\"a}r en medelv{\"a}rdesbildning
som omfattar antennen ger vid hand att ingen netto-str{\"o}m genom volymen sker
trots att ``antennen'' lokalt uppb{\"a}r en periodiskt oscillerande effektiv
(och h{\"o}gst lokal) str{\"o}m.
\vfill\eject

\section{Maxwells ekvationer}
\sidx{Maxwells ekvationer}
Maxwells fyra ekvationer\numberedfootnote{Det var faktiskt inte
   f{\"o}rr{\"a}n 1884 som Oliver Heaviside \sidx{Heaviside, Oliver (1850--1925)}
   (samme Heaviside som ligger bakom namnet f{\"o}r Heavisides stegfunktion
   $H(x)$), samtidigt med liknande liknande arbeten av Josiah Willard Gibbs
   (1839--1903) \sidx{Gibbs, Josiah Willard (1839--1903)} och Heinrich Hertz
   (1857--1894), \sidx{Hertz, Heinrich (1857--1894)} f{\"o}renklade och
   grupperade ihop de ursprungligen 20 ekvationerna till endast fyra, under
   anv{\"a}nd\-an\-de av modern vektornotation.
   Denna grupp av fyra vektor-ekvationer genom historien har kallats
   s{\aa}v{\"a}l {\it Hertz--Heavisides ekvationer} som {\it Maxwell--Hertz
   ekvationer}, men {\"a}r idag kort och gott k{\"a}nda som {\it Maxwells
   ekvationer}.}
kan uttryckas antingen i en differentiell form eller i integralform.
Ekvationerna involverar de tre elektriska f{\"a}lten (${\bf E}$, ${\bf D}$
och polarisationsdensiteten ${\bf P}$) och de tre magnetiska f{\"a}lten
(${\bf B}$, ${\bf H}$ och magnetiseringen ${\bf M}$).

\sidx{Elektrisk permittivitet}[Relativ permittivitet $\varepsilon_{\rm r}$]
\sidx{Magnetisk permeabilitet}[Relativ permeabilitet $\mu_{\rm r}$]
F{\"o}r v{\aa}gpropagation i fri rymd (vakuum) kan vi anta att den relativa
elektriska permittiviteten $\varepsilon_{\rm r}=1$ och att den relativa
magnetiska permeabiliteten $\mu_{\rm r}=1$, f{\"o}r vilket fall vi frikopplar
ekvationerna f{\"o}r ${\bf E}$ och ${\bf B}$ fr{\aa}n konstitutiva relationer.
Vi kommer h{\"a}r att ist{\"a}llet f{\"o}r att utg{\aa} ifr{\aa}n ett
f{\"o}renklat fall anta att vi har en mer generell situation i ett material.
Maxwells ekvationer kan i sin generella form sammanfattas
med\numberedfootnote{Vi kan enkelt verifiera att ``lagen om att laddning
  inte kan f{\"o}rsvinna'' {\"a}r uppfylld med den dynamiska formen av
  Amp\`eres lag, genom
  $$
    \nabla\cdot{\bf J}_{\rm f}
      =\nabla\cdot\Big(\nabla\times{\bf H}
           -{{\partial{\bf D}}\over{\partial t}}\Big)
      =\big\{\hbox{ $\nabla\cdot(\nabla\times{\bf a})\equiv0$ }\big\}
      =-{{\partial}\over{\partial t}}\nabla\cdot{\bf D}
      =-{{\partial\rho}\over{\partial t}}.
  $$}
\sidx{Maxwells ekvationer}[Differentialform]
\sidx{Maxwells ekvationer}[Integralform]
\sidx{Amp\`eres lag}\sidx{Faradays lag}
\sidx{Gauss lag}[F{\"o}r elektrisk fl{\"o}dest{\"a}thet ${\bf D}$]
\sidx{Gauss lag}[F{\"o}r magnetisk fl{\"o}dest{\"a}thet ${\bf B}$]
$$\hbox{~\hskip33pt}
  \matrix{
  &
    \displaystyle\oiint{\bf D}\cdot d{\bf S}=\iiint\rho\,dV\hfill&
    \quad\Leftrightarrow\quad&
    \displaystyle\nabla\cdot{\bf D}=\rho\hfill&
    &\hfill\hbox{(Gauss lag)}\cr
  &
    \displaystyle\oiint{\bf B}\cdot d{\bf S}=0\hfill&
    \quad\Leftrightarrow\quad&
    \displaystyle\nabla\cdot{\bf B}=0\hfill&
    &\hfill\hbox{(Gauss lag)}\cr
  &
    \displaystyle\oint{\bf E}\cdot d{\bf l}
       =-{{\partial}\over{\partial t}}\iint{\bf B}\cdot d{\bf S}\hfill&
    \quad\Leftrightarrow\quad&
    \displaystyle\nabla\times{\bf E}
       =-{{\partial{\bf B}}\over{\partial t}}\hfill&
    &\hfill\hbox{(Faradays lag)}\cr
  &
    \displaystyle\oint{\bf H}\cdot d{\bf l}=\iint{\bf J}_{\rm f}\cdot d{\bf S}
       +{{\partial}\over{\partial t}}\iint{\bf D}\cdot d{\bf S}\hfill&
    \quad\Leftrightarrow\quad&
    \displaystyle\nabla\times{\bf H}={\bf J}_{\rm f}
       +{{\partial{\bf D}}\over{\partial t}}\hfill&
    &\hfill\hbox{(Amp\`eres lag)}\cr
  }
$$
I dessa ekvationer {\"a}r $\rho$ den {\it fria elektriska
laddningst{\"a}theten}\numberedfootnote{Till skillnad fr{\aa}n den fria
   str{\"o}mt{\"a}theten l{\aa}ter vi h{\"a}r bli att s{\"a}tta ett explicit
   index ``f'' p{\aa} laddningsdensiteten, eftersom det alltid {\"a}r klart
   att $\rho$ relaterar just till {\it fria} laddningar, till skillnad
   fr{\aa}n ${\bf J}_{\rm f}$ som {\"a}r separerad fr{\aa}n
   f{\"o}rskjutningsstr{\"o}mmen $\partial{\bf D}/\partial t$.}
${\rm C}/{\rm m}^3$) och ${\bf J}_{\rm f}$ den {\it fria elektriska
str{\"o}mt{\"a}theten} (${\rm A}/{\rm m}^2$). Ut{\"o}ver dessa ekvationer
har vi (fr{\aa}n exempelvis f{\"o}reg{\aa}ende f{\"o}rel{\"a}sning) de
{\it konstitutiva relationerna}
\sidx{Konstitutiva relationer}
$$
  \eqalign{
    {\bf D}&=\varepsilon_0{\bf E}+{\bf P}
            =\varepsilon_0\varepsilon_{\rm r}{\bf E},\cr
    {\bf B}&=\mu_0({\bf H}+{\bf M})
            =\mu_0\mu_{\rm r}{\bf H}.\cr
  }
$$
Notera att vi har en lite ``avig'' relation f{\"o}r magnetiska f{\"a}lten
om vi betraktar paret (${\bf E}$,${\bf B}$) som v{\aa}ra prim{\"a}ra
f{\"a}ltvariabler. I m{\aa}nga fall {\"a}r det en fr{\aa}ga om tycke och
smak om man v{\"a}ljer att definiera f{\"a}ltproblemet utifr{\aa}n paret
(${\bf E}$,${\bf B}$) eller (${\bf E}$,${\bf H}$), och speciellt i fallet
med propagation i fri rymd {\"a}r valet egalt, men alltsom oftast {\"a}r det
mest bekv{\"a}mt att inom elektromagnetisk f{\"a}ltteori h{\aa}lla sig till
(${\bf E}$,${\bf B}$). Vi skall strax se varf{\"o}r.

De ing{\aa}ende f{\"a}lten och deras respektive \idx{SI-enheter} {\"a}r, f{\"o}r
att rekapitulera,
$$
  \eqalign{
    {\bf E} &= \hbox{Elektrisk f{\"a}ltstyrka (``elektriskt\ f{\"a}lt'')}
               \ ({\rm V}/{\rm m})
               \sidx{Elektrisk f{\"a}ltstyrka ${\bf E}$}\cr
    {\bf D} &= \hbox{Elektrisk fl{\"o}dest{\"a}thet}
               \ ({\rm C}/{\rm m}^2)
               \sidx{Elektrisk fl{\"o}dest{\"a}thet ${\bf D}$}\cr
    {\bf P} &= \hbox{Elektrisk polarisationsdensitet}
               \ ({\rm C}/{\rm m}^2)
               \sidx{Elektrisk polarisationsdensitet ${\bf P}$}\cr
    {\bf B} &= \hbox{Magnetisk fl{\"o}dest{\"a}thet (``B-f{\"a}lt'')}
               \ ({\rm T})
               \sidx{Magnetisk fl{\"o}dest{\"a}thet ${\bf B}$}\cr
    {\bf H} &= \hbox{Magnetisk f{\"a}ltstyrka (``H-f{\"a}lt'')}
               \ ({\rm A}/{\rm m})
               \sidx{Magnetisk f{\"a}ltstyrka ${\bf H}$}\cr
    {\bf M} &= \hbox{Magnetisering}
               \ ({\rm A}/{\rm m})
               \sidx{Magnetisering ${\bf M}$}\cr
  }
$$
\vfill\eject

\section{Fr{\aa}n Maxwells ekvationer till elektromagnetisk v{\aa}gekvation}
\sidx{Elektromagnetiska v{\aa}gekvationer}[H{\"a}rledning fr{\aa}n Maxwells
ekvationer]
{\"O}verg{\aa}ngen fr{\aa}n Maxwells ekvationer till de tv{\aa} vektoriella
elektromagnetiska v{\aa}gekvationerna {\"a}r onekligen en av de stiligaste
h{\"a}rledningarna i klassisk elektrodynamik. Det kan starkt rekommenderas att
en g{\aa}ng f{\"o}r alla g{\aa} igenom denna h{\"a}rledning med papper och
penna, om s{\aa} inte annat bara f{\"o}r att f{\"o}lja den aningen ov{\"a}ntade
kopplingen mellan klassisk induktion till elektromagnetisk v{\aa}gutbredning.
(Dessutom {\"a}r det en ganska kul och enkel exercis i vektoralgebra!)

Maxwells ekvationer i den form som de st{\aa}r i den klassiska beskrivningen
kan tolkas ganska direkt som relationer f{\"o}r induktions- och k{\"a}llagar.
De beskriver dock inte sj{\"a}lvklart v{\aa}gekvationer, {\aa}tminstone inte
vid en f{\"o}rsta anblick. Vi kommer h{\"a}r att h{\"a}rleda v{\aa}gekvationerna
f{\"o}r ${\bf E}$- och ${\bf B}$-f{\"a}lten genom att eliminera den elektriska
fl{\"o}dest{\"a}theten ${\bf D}$ och den magnetiska f{\"a}ltstyrkan ${\bf H}$,
till f{\"o}rm{\aa}n f{\"o}r en beskrivning direkt i termer av fri
str{\"o}mt{\"a}thet ${\bf J}$, elektrisk polarisationsdensitet ${\bf P}$ samt
magnetiseringen ${\bf M}$.

Vi b{\"o}rjar med den elektriska f{\"a}ltstyrkan ${\bf E}$ genom att applicera
$\nabla\times$ (``ta rotationen'') p{\aa} Faradays generella induktionslag,
\sidx{Amp\`eres lag}\sidx{Faradays lag}
\sidx{Gauss lag}[F{\"o}r elektrisk fl{\"o}dest{\"a}thet ${\bf D}$]
\sidx{Gauss lag}[F{\"o}r magnetisk fl{\"o}dest{\"a}thet ${\bf B}$]
\sidx{Elektrisk permittivitet}[Relativ permittivitet $\varepsilon_{\rm r}$]
$$
  \eqalign{
    \nabla\times\nabla\times{\bf E}
      &=\nabla\times\bigg(-{{\partial{\bf B}}\over{\partial t}}\bigg)\cr
      &=\{\ \hbox{Konstitutiv relation}
             \ {\bf B}=\mu_0({\bf H}+{\bf M})\ \}\cr
      &=-{{\partial}\over{\partial t}}
          \nabla\times\bigg(\mu_0({\bf H}+{\bf M})\bigg)\cr
      &=-\mu_0{{\partial}\over{\partial t}}\nabla\times{\bf H}
          -\mu_0{{\partial}\over{\partial t}}\nabla\times{\bf M}\cr
      &=\{\ \hbox{Till{\"a}mpa Amp\`eres lag}\ \}\cr
      &=-\mu_0{{\partial}\over{\partial t}}
          \bigg({\bf J}_{\rm f}+{{\partial{\bf D}}\over{\partial t}}\bigg)
          -\mu_0{{\partial}\over{\partial t}}\nabla\times{\bf M}\cr
      &=\{\ \hbox{Konstitutiv relation}
             \ {\bf D}=\varepsilon_0{\bf E}+{\bf P}\ \}\cr
      &=-\mu_0{{\partial^2}\over{\partial t^2}}
          \big(\varepsilon_0{\bf E}+{\bf P}\big)
          -\mu_0{{\partial}\over{\partial t}}
	     \big({\bf J}_{\rm f}+\nabla\times{\bf M}\big)\cr
      &=\{\ \hbox{Kombinera polarisationsdensiteten in i k{\"a}llterm}\ \}\cr
      &=-\mu_0\varepsilon_0{{\partial^2{\bf E}}\over{\partial t^2}}
          -\underbrace{
	     \mu_0{{\partial}\over{\partial t}}
	     \bigg({\bf J}_{\rm f}
	        +{{\partial{\bf P}}\over{\partial t}}
	        +\nabla\times{\bf M}\bigg)}_{\hbox{k{\"a}llterm}}\cr
  }
$$
Notera att denna ekvation f{\"o}r ${\bf E}$ g{\"a}ller {\it oavsett} eventuella
spatiala variationer hos relativa permittiviteten eller permeabiliteten, det
vill s{\"a}ga oavsett om relationen mellan de exciterande f{\"a}lten och den
resulterande elektriska polarisationsdensiteten eller magnetiseringen
{\"a}ndras. P{\aa} samma s{\"a}tt har vi f{\"o}r magnetiska
fl{\"o}dest{\"a}theten ${\bf B}=\mu_0({\bf H}+{\bf M})$ att
\sidx{Amp\`eres lag}\sidx{Faradays lag}
\sidx{Gauss lag}[F{\"o}r elektrisk fl{\"o}dest{\"a}thet ${\bf D}$]
\sidx{Gauss lag}[F{\"o}r magnetisk fl{\"o}dest{\"a}thet ${\bf B}$]
\sidx{Magnetisk permeabilitet}[Relativ permeabilitet $\mu_{\rm r}$]
$$
  \eqalign{
    \nabla\times\nabla\times{\bf B}
      &=\mu_0\nabla\times\nabla\times{\bf H}
           +\mu_0\nabla\times\nabla\times{\bf M}\cr
      &=\{\ \hbox{Till{\"a}mpa Amp\`eres lag}\ \}\cr
      &=\mu_0\nabla\times\bigg({\bf J}_{\rm f}
           +{{\partial{\bf D}}\over{\partial t}}\bigg)
           +\mu_0\nabla\times\nabla\times{\bf M}\cr
      &=\mu_0{{\partial}\over{\partial t}}\nabla\times{\bf D}
           +\mu_0\nabla\times\bigg({\bf J}_{\rm f}+\nabla\times{\bf M}\bigg)\cr
      &=\{\ \hbox{Konstitutiv relation}
             \ {\bf D}=\varepsilon_0{\bf E}+{\bf P}\ \}\cr
      &=\mu_0\varepsilon_0{{\partial}\over{\partial t}}\nabla\times{\bf E}
           +\mu_0\nabla\times\bigg({\bf J}_{\rm f}
	   +{{\partial{\bf P}}\over{\partial t}}
	   +\nabla\times{\bf M}\bigg)\cr
      &=\{\ \hbox{Till{\"a}mpa Faradays lag}\ \}\cr
      &=-\mu_0\varepsilon_0{{\partial{\bf B}}^2\over{\partial t^2}}
           +\mu_0\nabla\times\bigg({\bf J}_{\rm f}
	   +{{\partial{\bf P}}\over{\partial t}}
	   +\nabla\times{\bf M}\bigg)\cr
  }
$$
{\"A}ven denna ekvation f{\"o}r ${\bf B}$ g{\"a}ller {\it oavsett} eventuella
spatiala variationer hos relativa permittiviteten eller permeabiliteten.
F{\"o}r att sammanfatta resultaten f{\"o}r den elektriska f{\"a}ltstyrkan
${\bf E}$ och den magnetiska fl{\"o}dest{\"a}theten ${\bf B}$:
\sidx{Elektromagnetiska v{\aa}gekvationer}[H{\"a}rledning fr{\aa}n Maxwells
ekvationer]
\sidx{Elektromagnetiska v{\aa}gekvationer}[Gemensam k{\"a}llterm]
\sidx{Elektrisk f{\"a}ltstyrka ${\bf E}$}[V{\aa}gekvation f{\"o}r]
\sidx{Magnetisk fl{\"o}dest{\"a}thet ${\bf B}$}[V{\aa}gekvation f{\"o}r]
$$
  \eqalign{
    \nabla\times\nabla\times{\bf E}
      +\mu_0\varepsilon_0{{\partial^2{\bf E}}\over{\partial t^2}}&=
         -\mu_0{{\partial}\over{\partial t}}
          \underbrace{
             \bigg({\bf J}_{\rm f}
	        +{{\partial{\bf P}}\over{\partial t}}
	        +\nabla\times{\bf M}\bigg)}_{\hbox{gemensam k{\"a}llterm}},\cr
    \nabla\times\nabla\times{\bf B}
      +\mu_0\varepsilon_0{{\partial^2{\bf B}}\over{\partial t^2}}&=
          \mu_0\nabla\times
          \underbrace{
	     \bigg({\bf J}_{\rm f}
	        +{{\partial{\bf P}}\over{\partial t}}
	        +\nabla\times{\bf M}\bigg)}_{\hbox{gemensam k{\"a}llterm}}.\cr
  }
$$
I dessa ekvationer svarar v{\"a}nsterleden mot de tre-dimensionella
v{\aa}gekvationerna for elektriska och magnetiska f{\"a}lt, medan
h{\"o}gerleden svarar mot k{\"a}lltermer inkluderande fria str{\"o}mmar,
f{\"o}rskjut\-nings\-str{\"o}mmens bidrag fr{\aa}n materialet i sig (via
polarisationsdensiteten ${\bf P}$, som regel en funktion beroende av elektriska
f{\"a}ltet), samt magnetiseringen ${\bf M}$ (som regel en funktion beroende av
magnetf{\"a}ltet).

V{\"a}rt att notera i ekvationerna f{\"o}r de elektromagnetiska f{\"a}lten
{\"a}r den gemensamma formen av k{\"a}lltermerna, d{\"a}r den enda skillnaden
(f{\"o}rutom omv{\"a}nt tecken) {\"a}r tidsderivatan respektive rotationen.
Dessa ekvationer har i sig ingen inneboende best{\"a}md frekvens eller liknande
f{\"o}r f{\"a}lten, utan detta best{\"a}ms av randv{\"a}rden (exempelvis en
drivande antenn eller laser) samt materialegenskaperna som modelleras via
${\bf J}_{\rm f}$, ${\bf P}$ och ${\bf M}$.
\sidx{Elektrisk polarisationsdensitet ${\bf P}$}
\sidx{Magnetisering ${\bf M}$}
\vfill\eject

\section{Aprop{\aa} s{\"a}rskiljning av polarisationsdensitet och magnetisering}
\sidx{Elektrisk polarisationsdensitet ${\bf P}$}[Kontra magnetisering]
\sidx{Magnetisering ${\bf M}$}[Kontra elektrisk polarisationsdensitet]
En intressant f{\"o}ljd av den gemensamma formen p{\aa} k{\"a}lltermerna i
h{\"o}gerleden i ekvationerna f{\"o}r ${\bf E}$ och ${\bf B}$ {\"a}r att
effekterna av den fria str{\"o}mmen, polarisationsdensiteten och magnetiseringen
ej {\"a}r entydigt best{\"a}mda relativt varandra, utan har en viss grad av
godtycklighet mellan sig. Som ett exempel, om vi antar att
polarisationsdensiteten har en godtycklig term i sig som beskrivs av en
rotation, s{\"a}g
$$
  {\bf P}={\bf P}'+\nabla\times{\bf G},
$$
s{\aa} {\"a}r det egalt om termen $\nabla\times{\bf G}$ formellt ing{\aa}r i
polarisationsdensiteten ${\bf P}$ eller magnetiseringen ${\bf M}$, eftersom
$$
  {\partial\over{\partial t}}\big({\bf P}'+\nabla\times{\bf G}\big)
    +\nabla\times{\bf M}
  = {{\partial{\bf P}}'\over{\partial t}}
      +\nabla\times\big({\bf M}+{\bf G}\big)
$$
Eftersom ingen av ekvationerna f{\"o}r f{\"a}lten ${\bf E}$ eller ${\bf B}$
ovan p{\aa}verkas av om vi har antingen en modell f{\"o}r
polarisationsdensiteten som ${\bf P}={\bf P}'+\nabla\times{\bf G}$ eller
magnetiseringen som ${\bf M}={\bf M}'+{\bf G}$, s{\aa} blir fr{\aa}gan vad
som egenttligen {\"a}r polarisationsdensitet eller magnetisering till viss
del en fr{\aa}ga om godtycklighet och konvention, {\aa}tminstone vad
betr{\"a}ffar den elektromagnetiska f{\"a}ltteorin.

Ett exempel p{\aa} en polarisationsdensitet som inneh{\aa}ller just
$\nabla\times{\bf E}$ {\"a}r fallet d{\aa} ett medium {\"a}r optiskt aktivt,
under vilket det roterar polarisationstillst{\aa}ndet hos ljuset runt axeln
l{\"a}ngs vilken ljuset propagerar. I detta fall kan inte effekten av optisk
aktivitet s{\"a}rskiljas fr{\aa}n fallet med Faraday-effekt, d{\"a}r vi har
ett tatiskt magnetf{\"a}lt p{\aa}lagt l{\"a}ngs med axeln f{\"o}r ljusets
propagationsriktning. (F{\"o}r att s{\"a}rskilja dessa fall kr{\"a}vs att vi
{\"a}ven studerar motsvarande effekter vid motpropagerande f{\"a}lt, men detta
{\"a}r l{\aa}ngt utanf{\"o}r vad denna kurs t{\"a}cker.)

Ett annat s{\"a}tt att se p{\aa} k{\"a}lltermerna i h{\"o}gerleden {\"a}r som
{\it effektiva str{\"o}mt{\"a}theter} ${\bf J}_{\rm eff}$ involverande
tidsderivatan av ${\bf P}$ (f{\"o}rskjutningsstr{\"o}mmen) och rotationen
av ${\bf M}$ som till{\"a}ggstermer till den fria str{\"o}mt{\"a}theten
${\bf J}_{\rm f}$, som
$$
  {\bf J}_{\rm eff}= {\bf J}_{\rm f}
    +{{\partial{\bf P}}\over{\partial t}} + \nabla\times{\bf M},
$$
och helt enkelt reducera f{\"a}ltekvationerna till
$$
  \eqalign{
    \nabla\times\nabla\times{\bf E}
      +\mu_0\varepsilon_0{{\partial^2{\bf E}}\over{\partial t^2}}&=
         -\mu_0{{\partial{\bf J}_{\rm eff}}\over{\partial t}},\cr
    \nabla\times\nabla\times{\bf B}
      +\mu_0\varepsilon_0{{\partial^2{\bf B}}\over{\partial t^2}}&=
          \mu_0\nabla\times{\bf J}_{\rm eff}.\cr
  }
$$
\vfill\eject
\section{V{\aa}gekvation, induktion, elektrostatik, elektrodynamik, vad
         g{\"a}ller detta egentligen?}
I h{\"a}rledandet av formen p{\aa} v{\aa}gekvationerna f{\"o}r f{\"a}lten
${\bf E}$ och ${\bf B}$ ovan, s{\aa} har vi varken lagt till eller tagit bort
n{\aa}got. Allt utg{\aa}r ifr{\aa}n Faradays och Amp\`eres lagar, inklusive
Maxwells korrektion f{\"o}r att uppfylla villkoret f{\"o}r den elektriska
laddningens bevarande, och ekvationerna beskriver i en form eller annan
d{\"a}rmed {\it samtliga} omr{\aa}den som vi hittills behandlat i kursen.

Allt handlar om att reducera v{\aa}gekvationerna till relevant situation,
exempelvis om vi har att g{\"o}ra med ett statiskt problem, om vi har avsaknad
av magnetisering eller om vi rent av har att g{\"o}ra med ett frirymdsproblem.
Allt handlar med andra ord om vilka f{\"o}ruts{\"a}ttningar vi s{\aa} att
s{\"a}ga ``matar in som antaganden'' i v{\aa}gekvationerna ovan. Om man s{\aa}
vill, s{\aa} kan man s{\"a}ga att grunden f{\"o}r samtliga elektrodynamiska
problem {\"a}r tv{\aa} v{\aa}gekvationer, f{\"o}r ${\bf E}$- och ${\bf B}$-%
f{\"a}lten, och att statiska problem helt enkelt reducerar v{\aa}gekvationerna
till enbart spatial dom{\"a}n genom att tidsderivator f{\"o}rsvinner.

\subsection{Exempel I: Elektrostatik}
\sidx{Elektrostatik}[Fr{\aa}n elektromagnetisk v{\aa}gekvation]
Om vi till exempel behandlar fallet elektrostatik i n{\"a}rvaro av konstanta
str{\"o}mmar, s{\aa} reducerar ekvationen ovan for det (statiska) elektriska
f{\"a}ltet till
$$
  \nabla\times\nabla\times{\bf E}=0,
$$
vilket under anv{\"a}ndande av vektoridentiteten
$\nabla\times\nabla\times{\bf A}=\nabla\cdot(\nabla{\bf A})-\nabla^2{\bf A}$
reducerar till
$$
  \nabla(\nabla\cdot{\bf E})-\nabla^2{\bf E}=0.
$$
Det {\"a}r h{\"a}r l{\"a}tt att f{\"o}rledas att tro att $\nabla\cdot{\bf E}=0$
om vi inte har n{\aa}gra fria laddningar. Dock, vi m{\aa}ste komma ih{\aa}g att
detta argument bara h{\aa}ller f{\"o}r frirymdsformen av Gauss lag
($\nabla\cdot{\bf E}=\rho/\varepsilon_0$). Om materialet {\"a}r {\it inhomogent}
men i {\"o}vrigt fritt fr{\aa}n fria laddningar ($\rho=0$), s{\aa} m{\aa}ste vi
g{\aa} p{\aa} den egentliga formen av Gauss lag som g{\"a}ller den elektriska
fl{\"o}dest{\"a}theten,
$$
  \nabla\cdot{\bf D}
    =\varepsilon_0\nabla\cdot(\varepsilon_{\rm r}{\bf E})=0
  \qquad\Leftrightarrow\qquad
  {\bf E}\cdot\nabla\varepsilon_{\rm r}+
    \varepsilon_{\rm r}\nabla\cdot{\bf E}=0
  \qquad\Leftrightarrow\qquad
  \nabla\cdot{\bf E}=
    -{{1}\over{\varepsilon_{\rm r}}}{\bf E}\cdot\nabla\varepsilon_{\rm r}.
$$
Allts{\aa}, elektrostatik i {\it inhomogena media} beskrivs av ekvationen
$$
  \nabla^2{\bf E}=-\nabla\bigg(
      {{1}\over{\varepsilon_{\rm r}}}{\bf E}\cdot\nabla\varepsilon_{\rm r}
    \bigg).
$$

\subsection{Exempel II: Magnetostatik}
\sidx{Magnetostatik}[Fr{\aa}n elektromagnetisk v{\aa}gekvation]
P{\aa} samma s{\"a}tt har vi f{\"o}r det magnetiska f{\"a}ltet ${\bf B}$ i
n{\"a}rvaro av en konstant str{\"o}mt{\"a}thet ${\bf J}_{\rm f}$, men i
fr{\aa}nvaro av magnetisering ${\bf M}$, att ekvationen ovan for det (statiska)
magnetiska f{\"a}ltet enligt ovan reducerar till
$$
  \nabla\times\nabla\times{\bf B}=\mu_0\nabla\times{\bf J}_{\rm f}
  \qquad\Leftrightarrow\qquad
  \nabla\times{\bf B}=\mu_0{\bf J}_{\rm f},
$$
det vill s{\"a}ga svarandes mot frirymdsformen av Amp\'eres lag, precis som
f{\"o}rv{\"a}ntat utifr{\aa}n premisserna f{\"o}r reduktionen fr{\aa}n de
generella v{\aa}gekvationerna ovan.
\vfill\eject

\section{Sammanfattning av F{\"o}rel{\"a}sning~8
  -- Maxwells ekvationer och v{\aa}gutbredning}
\item{$\bullet$}{Maxwells genidrag f{\"o}r att f{\aa} kontinuitetsekvationen
  f{\"o}r elektrisk laddning att g{\aa} ihop var att i uttrycket f{\"o}r den
  totala str{\"o}mt{\"a}theten l{\"a}gga till en
  {\it f{\"o}rskjutningsstr{\"o}m} ${{\partial{\bf D}}/{\partial t}}$ till
  den fria str{\"o}mt{\"a}theten ${\bf J}_{\rm f}$, som
  $$
    {\bf J}_{\rm tot} = {\bf J}_{\rm f}+{{\partial{\bf D}}\over{\partial t}}.
  $$}
  \sidx{Maxwells ekvationer}[Differentialform]
  \sidx{Maxwells ekvationer}[Integralform]
  \sidx{Amp\`eres lag}\sidx{Faradays lag}
  \sidx{Gauss lag}[F{\"o}r elektrisk fl{\"o}dest{\"a}thet ${\bf D}$]
  \sidx{Gauss lag}[F{\"o}r magnetisk fl{\"o}dest{\"a}thet ${\bf B}$]
\item{$\bullet$}{
  Maxwells ekvationer p{\aa} integral- och differentialform lyder
  $$\hbox{~\hskip33pt}
    \matrix{
    &
      \displaystyle\oiint{\bf D}\cdot d{\bf S}=\iiint\rho\,dV\hfill&
      \quad\Leftrightarrow\quad&
      \displaystyle\nabla\cdot{\bf D}=\rho\hfill&
      &\cr
    &
      \displaystyle\oiint{\bf B}\cdot d{\bf S}=0\hfill&
      \quad\Leftrightarrow\quad&
      \displaystyle\nabla\cdot{\bf B}=0\hfill&
      &\cr
    &
      \displaystyle\oint{\bf E}\cdot d{\bf l}
         =-{{\partial}\over{\partial t}}\iint{\bf B}\cdot d{\bf S}\hfill&
      \quad\Leftrightarrow\quad&
      \displaystyle\nabla\times{\bf E}
         =-{{\partial{\bf B}}\over{\partial t}}\hfill&
      &\cr
    &
      \displaystyle\oint{\bf H}\cdot d{\bf l}=\iint{\bf J}_{\rm f}\cdot d{\bf S}
         +{{\partial}\over{\partial t}}\iint{\bf D}\cdot d{\bf S}\hfill&
      \quad\Leftrightarrow\quad&
      \displaystyle\nabla\times{\bf H}={\bf J}_{\rm f}
         +{{\partial{\bf D}}\over{\partial t}}\hfill&
      &\cr
    }
  $$}
\item{$\bullet$}{De konstitutiva relationerna f{\"o}r elektriska och magnetiska
  f{\"a}lt lyder
  $$
    \eqalign{
      {\bf D}&=\varepsilon_0{\bf E}+{\bf P}
              =\varepsilon_0\varepsilon_{\rm r}{\bf E},\cr
      {\bf B}&=\mu_0({\bf H}+{\bf M})
              =\mu_0\mu_{\rm r}{\bf H}.\cr
    }
  $$\sidx{Konstitutiva relationer}}
\item{$\bullet$}{De elektromagnetiska v{\aa}gekvationerna f{\"o}r de
  elektriska och magnetiska f{\"a}lten lyder
  $$
    \eqalign{
      \nabla\times\nabla\times{\bf E}
        +\mu_0\varepsilon_0{{\partial^2{\bf E}}\over{\partial t^2}}&=
           -\mu_0{{\partial}\over{\partial t}}
            \underbrace{
               \bigg({\bf J}_{\rm f}
                  +{{\partial{\bf P}}\over{\partial t}}
                  +\nabla\times{\bf M}\bigg)}_{\hbox{gemensam k{\"a}llterm}},\cr
      \nabla\times\nabla\times{\bf B}
        +\mu_0\varepsilon_0{{\partial^2{\bf B}}\over{\partial t^2}}&=
            \mu_0\nabla\times
            \underbrace{
               \bigg({\bf J}_{\rm f}
                  +{{\partial{\bf P}}\over{\partial t}}
                  +\nabla\times{\bf M}\bigg)}_{\hbox{gemensam k{\"a}llterm}}.\cr
  }
  $$}
\item{$\bullet$}{Fr{\aa}n de elektromagnetiska v{\aa}gekvationerna kan
  samtliga fall som hittills behandlas i kursen h{\"a}rledas, beroende
  p{\aa} vilka termer som kan s{\"a}ttas till noll och p{\aa} s{\aa} s{\"a}tt
  reducera systemet till den statiska eller dynamiska situation man {\"o}nskar
  analysera.}

\cleardoublepage
%%% End of auto-extracted text from ../lect-09/lecture-09.tex %%%
%%% Begin of auto-extracted text from ../lect-10/lecture-10.tex %%%
%
% File: teach/elmagii/lect-10/lecture-10.tex [plain TeX code]
% Github: https://github.com/elmagii/lect-10/
% Last change: November 25, 2025
%
% Lecture No 10 in the course ``Elektromagnetism II, 1TE626 (2023)'',
% held November 26, 2025, at Uppsala University, Sweden.
%
% Copyright (C) 2022-2025, Fredrik Jonsson, under Gnu General Public
% License (GPL) v3. See the enclosed LICENSE for details.
%
% This program is free software: you can redistribute it and/or modify
% it under the terms of the GNU General Public License as published by
% the Free Software Foundation, either version 3 of the License, or
% (at your option) any later version.
%
% This program is distributed in the hope that it will be useful,
% but WITHOUT ANY WARRANTY; without even the implied warranty of
% MERCHANTABILITY or FITNESS FOR A PARTICULAR PURPOSE.  See the
% GNU General Public License for more details.
%
% You should have received a copy of the GNU General Public License
% along with this program.  If not, see <https://www.gnu.org/licenses/>.
%
\def\coursename{Elektromagnetism II}
\def\coursecode{1TE626}
\def\courseyear{2025}
\def\courserepo{https://github.com/hp35/elmagii/}
\def\lecturenumber{10}
\def\lecturetitle{V{\aa}gutbredning i homogena och isotropa dielektrika}
\def\lecturesubtitle{}
\def\lectureauthor{Fredrik Jonsson}
\def\lectureplace{Uppsala Universitet}
\def\lecturedate{26 november 2025}
%-------------------- BEGIN OF LOCAL MACROS --------------------
\edef\expandedlecturenumber{10}
\def\ifempty#1{\ifx\relax#1\relax}
\advance\chapno by 1
\secno=0
\footnotenumber=0
\message{==================== Lecture 10 ====================}
\writenumberedtocentry{chapter}{F{\"o}rel{\"a}sning 10 -- {V{\aa}gutbredning i homogena och isotropa dielektrika}}{\thechapno}
\hsize=150mm\hoffset=4.6mm\vsize=230mm\voffset=7mm
\topskip=0pt\baselineskip=12pt\parskip=0pt\leftskip=0pt\parindent=15pt
\ifcolors
  \voffset=-10.2mm\topskip=0pt
\fi
\headline={\ifnum\secno>0\ifodd\pageno\rightheadline\else\leftheadline\fi
  \else\hfill\fi}
\def\rightheadline{\tenrm{\it F\"orel\"asning 10}
  \hfil{\it \coursename, \coursecode\ (\courseyear)}}
\def\leftheadline{\tenrm{\it \coursename, \coursecode\ (\courseyear)}
  \hfil{\it F\"orel\"asning 10}}
\noindent~\vskip-60pt\hskip-40pt{\epsfbox{../lect-01/macros/UU_logo_color.eps}}
\vskip-42pt\hfill\vbox{
    \hbox{{\it \coursename, \coursecode\ (\courseyear)}}
    \hbox{{\it Lecture Notes, \lectureauthor}}
    \hbox{{\it Document Revision \today}}
    \hbox{{\it \courserepo}}}\vskip 36pt
\centerline{\twelvesc F\"orel\"asning 10}
\vskip 24pt\noindent
\centerline{\twelvesc{V{\aa}gutbredning i homogena och isotropa dielektrika}}
\expandafter\ifempty\expandafter{\lecturesubtitle}%
  \else\centerline{\twelvesc\lecturesubtitle}\fi
\bigskip
\centerline{\lectureauthor, \lectureplace, \lecturedate}
\vskip24pt
%--------------------- END OF LOCAL MACROS ---------------------



\plan{Maxwells ekvationer och den elektromagnetiska v{\aa}gekvationen
  sammanfattas och vi introducerar fyra t{\"a}mligen allm{\"a}nt giltiga
  approximationer:
    (i)   ${\bf J}_{\rm f}={\bf 0}$ (inga fria str{\"o}mmar),
          \sidx{Str{\"o}mt{\"a}thet}[Fria laddningar]
    (ii)  $\rho=0$ (inga fria laddningar),
          \sidx{Laddningst{\"a}thet}[Fria laddningar $\rho_{\rm f}$]
    (iii) ${\bf P}=\varepsilon_0(\varepsilon_{\rm r}-1){\bf E}$
              (linj{\"a}rt medium med homogent $\varepsilon_{\rm r}({\bf x})
               =\hbox{konstant}$), samt
          \sidx{Elektrisk polarisationsdensitet ${\bf P}$}
    (iv)  ${\bf M}={\bf 0}$ (f{\"o}rsumbar magnetisering, $\mu_{\rm r}\approx1$).
          \sidx{Magnetisering ${\bf M}$}
  Med dessa approximationer omformulerar vi v{\aa}gekvationerna f{\"o}r det
  elektromagnetiska f{\"a}ltet i form av tv{\aa} identiska ekvationer, f{\"o}r
  det elektriska f{\"a}ltet ${\bf E}$ och det magnetiska f{\"a}ltet ${\bf B}$
  respektive, vilka vidare direkt kan reduceras till en skal{\"a}r
  v{\aa}gekvation som $(\partial^2/\partial z^2
  -\mu_0\varepsilon_0\varepsilon_{\rm r}\partial^2/\partial t^2) E(z,t)=0$.
  Utifr{\aa}n denna form identifierar vi att ljushastigheten i vakuum $c_0$
  ges av $c^{-2}_0=\mu_0\varepsilon_0$ samt brytningsindex
  $n=\varepsilon^{1/2}_{\rm r}$, skalandes utbredningshastigheten f{\"o}r de
  elektromagnetiska v{\aa}gorna som $c=c_0/n$.
  Med hj{\"a}lp av d'Alemberts metod tar vi fram den generella l{\"o}sningen
  f{\"o}r v{\aa}g\-ekva\-tionen och visar p{\aa} hur denna i en dimension har
  motpropagerande komponenter som alltid alltid uppfyller
  $E(z,t)=f(z-ct)+g(z+ct)$.

  F{\"o}r tidsharmoniska f{\"a}lt ans{\"a}tter vi komplexv{\"a}rda envelopper
  som ${\bf E}({\bf x},t)=\sum_{\bf k}\Re[{\bf E}_{\bf k}\exp(i{\bf k}
  \cdot{\bf x}-i\omega({\bf k}) t)]$, med samma form f{\"o}r det magnetiska
  f{\"a}ltet, och visar p{\aa} hur denna planv{\aa}gsuppdelning leder till
  en ortogonal koppling mellan det elektriska och magnetiska f{\"a}ltet som
  ${\bf B}_{\bf k}={\bf k}\times{\bf E}_{\bf k}/\omega$.
  \sidx{Planv{\aa}guppdelning}}

\threepointsummary{%
  \sidx{Elektromagnetisk v{\aa}gekvation}
  Den elektromagnetiska v{\aa}gekvationen i homogena dielektrika utan
  n{\"a}rvaro av fria laddningar, str{\"o}mmar eller magnetisering {\"a}r
  identisk f{\"o}r det elektriska och magnetiska f{\"a}ltet,
  $$
    \nabla^2{\bf E}-{{1}\over{c^2}}{{\partial^2{\bf E}}\over{\partial t^2}}=0,
    \qquad\quad
    \nabla^2{\bf B}-{{1}\over{c^2}}{{\partial^2{\bf B}}\over{\partial t^2}}=0,
  $$
  d{\"a}r $c=c_0/n$ {\"a}r den hastighet som v{\aa}gen har i mediet, med
  ljus\-hastigheten i vakuum $c_0=(\mu_0\varepsilon_0)^{-1/2}$ och
  brytnings\-index $n=\varepsilon^{1/2}_{\rm r}$.
}{%
  \sidx{d'Alembert, Jean le Rond (1717--1783)}[d'Alemberts l{\"o}sningsmetod
  (1747)]
  Med d'Alemberts l{\"o}sningsmetod ges generella l{\"o}sningar till
  v{\aa}gekvationen i en dimension som
  $$
    E(z,t) = f(z-ct) + g(z+ct).
  $$
}{%
  Planv{\aa}gsuppdelning av den elektromagnetiska v{\aa}gen med
  komplex-v{\"a}rda envelopper,\sidx{Planv{\aa}guppdelning}
  $$
    \eqalign{
      {\bf E}({\bf x},t)&=\sum_{\bf k}
         \Re[{\bf E}_{\bf k}\exp(i{\bf k}\cdot{\bf x}-i\omega({\bf k}) t)],\cr
      {\bf B}({\bf x},t)&=\sum_{\bf k}
         \Re[{\bf B}_{\bf k}\exp(i{\bf k}\cdot{\bf x}-i\omega({\bf k}) t)].\cr
    }
  $$
}
\vfill\eject\copyrights

\section{Elektromagnetiska v{\aa}gekvationen i homogena dielektrika}
\sidx{Elektromagnetisk v{\aa}gekvation}[I homogent dielektrikum]
Under den f{\"o}rra f{\"o}rel{\"a}sningen tog vi ur Maxwells ekvationer fram
de generella v{\aa}gekvationerna f{\"o}r ${\bf E}$- och ${\bf B}$-f{\"a}lten.
Dessa visade sig ha en gemensam form p{\aa} k{\"a}lltermerna i h{\"o}gerledet
som\sidx{Homogent medium}\sidx{Elektromagnetisk v{\aa}gekvation}
$$
  \eqalign{
    \nabla\times\nabla\times{\bf E}
      +\mu_0\varepsilon_0{{\partial^2{\bf E}}\over{\partial t^2}}&=
         -\mu_0{{\partial}\over{\partial t}}
          \underbrace{
             \bigg({\bf J}_{\rm f}
	        +{{\partial{\bf P}}\over{\partial t}}
	        +\nabla\times{\bf M}\bigg)
                }_{\hbox{gemensam k{\"a}llterm}},\cr
    \nabla\times\nabla\times{\bf B}
      +\mu_0\varepsilon_0{{\partial^2{\bf B}}\over{\partial t^2}}&=
          \mu_0\nabla\times
          \underbrace{
	     \bigg({\bf J}_{\rm f}
	        +{{\partial{\bf P}}\over{\partial t}}
	        +\nabla\times{\bf M}\bigg)
                }_{\hbox{gemensam k{\"a}llterm}}.\cr
  }
$$
{\"A}ven om de generella v{\aa}gekvationerna f{\"o}r ${\bf E}$ och ${\bf B}$
enligt tidigare har en viktig betydelse i sig, i och med deras generalitet och
fr{\aa}nvaro av approximationer, s{\aa} {\"a}r de i rent praktisk mening av
begr{\"a}nsad betydelse eftersom de inte {\"a}nnu klarg{\"o}r hur effekten av
mediet f{\"o}r v{\aa}gpropagationen sl{\aa}r in. Vi kommer d{\"a}rf{\"o}r att
nu g{\"o}ra f{\"o}ljande f{\"o}renklingar:
\sidx{Laddningst{\"a}thet}[Fria laddningar $\rho_{\rm f}$]
\sidx{Str{\"o}mt{\"a}thet}[Fria laddningar]
\sidx{Elektrisk polarisationsdensitet ${\bf P}$}
\sidx{Elektrisk polarisationsdensitet ${\bf P}$}[Linearitet]
\sidx{Polarisationsdensitet {\bf P}}[Elektrisk]
\sidx{Magnetisering ${\bf M}$}
\medskip
\item{$\bullet$}{${\bf J}_{\rm f}={\bf 0}$ (inga fria str{\"o}mmar)}
\item{$\bullet$}{$\rho=0$ (inga fria laddningar)}
\item{$\bullet$}{${\bf P}=\varepsilon_0(\varepsilon_{\rm r}-1){\bf E}$
   ({\it linj{\"a}rt} medium, och vi antar {\"a}ven {\it homogent}
   $\varepsilon_{\rm r}({\bf x})=\hbox{konstant}$)}
\item{$\bullet$}{${\bf M}={\bf 0}$ (f{\"o}rsumbar magnetisering,
   $\mu_{\rm r}\approx1$)}
\medskip
\noindent
Med ``linj{\"a}rt'' medium menar vi h{\"a}r att polarisationsdensiteten
${\bf P}$ kort och gott {\"a}r en linj{\"a}r funktion av det p{\aa}lagda
elektriska f{\"a}ltet ${\bf E}$. Genom att inf{\"o}ra dessa f{\"o}renklingar,
speciellt f{\"o}r polarisationsdensiteten
${\bf P}=\varepsilon_0(1-\varepsilon_{\rm r}){\bf E}$, har vi att
$$
  \eqalign{
    \nabla\times\nabla\times{\bf E}
      +\mu_0\varepsilon_0{{\partial^2{\bf E}}\over{\partial t^2}}
        &=-\mu_0{{\partial^2{\bf P}}\over{\partial t^2}}\cr
        &=-\mu_0\varepsilon_0(\varepsilon_{\rm r}-1)
            {{\partial^2{\bf E}}\over{\partial t^2}},\cr
    \nabla\times\nabla\times{\bf B}
      +\mu_0\varepsilon_0{{\partial^2{\bf B}}\over{\partial t^2}}
        &= \mu_0\nabla\times{{\partial{\bf P}}\over{\partial t}}\cr
        &= \mu_0\varepsilon_0(\varepsilon_{\rm r}-1)
	  \nabla\times{{\partial{\bf E}}\over{\partial t}}\cr
    &= \{\ \hbox{Byt ordning p{\aa}}\ \nabla\times\ \hbox{och}
           \ \partial/\partial t\ \}\cr
    &= \mu_0\varepsilon_0(\varepsilon_{\rm r}-1)
          {{\partial}\over{\partial t}}\nabla\times{\bf E}\cr
    &= \{\ \hbox{Faradays lag}\ \}\cr
    &= -\mu_0\varepsilon_0(\varepsilon_{\rm r}-1)
          {{\partial^2{\bf B}}\over{\partial t^2}}.\cr
  }
$$
Vi ser redan h{\"a}r en fundamental egenskap i elektrodynamiken, n{\"a}mligen
att ekvationerna f{\"o}r v{\aa}gutbredningen av elektriska och magnetiska
f{\"a}lt i fr{\aa}nvaro av p{\aa}verkan av externa k{\"a}llor {\"a}r
{\it identiska}. Detta f{\aa}r till f{\"o}ljd att vi (efter att ha definierat
respektive randvillkor och initialv{\"a}rden till de partiella
differentialekvationerna) kan behandla elektriska och magnetiska f{\"a}lt
p{\aa} samma s{\"a}tt, och om f{\"a}lten har ett gemensamt ursprung, exempelvis
fr{\aa}n en antenn, laser eller liknande, s{\aa} kommer ${\bf E}$- och
${\bf B}$-f{\"a}lten att f{\"o}ljas {\aa}t d{\aa} de f{\"o}ljer {\it exakt}
samma ekvationer.

Vi kan f{\"o}renkla v{\"a}nsterledets spatiala differentialtermer n{\aa}got,
genom att f{\"o}rst konstatera att
$$
  \nabla\times\nabla\times{\bf E}
    =\nabla(\nabla\cdot{\bf E})-\nabla^2{\bf E}.
$$
F{\"o}r v{\aa}gutbredning i {\it homogena} material {\it utan fria laddningar}
blir termen $\nabla(\nabla\cdot{\bf E})$ noll, eftersom
\sidx{Elektrisk f{\"a}ltstyrka ${\bf E}$}[Divergens f{\"o}r]
$$
  \nabla\cdot{\bf E}
    =\nabla\cdot\bigg({{{\bf D}}\over{\varepsilon_0\varepsilon_{\rm r}}}\bigg)
    ={{1}\over{\varepsilon_0\varepsilon_{\rm r}}}\nabla\cdot{\bf D}
    ={{\rho}\over{\varepsilon_0\varepsilon_{\rm r}}}
    =0.
$$
Notera dock att Gauss lag f{\"o}r elektriska f{\"a}ltet formellt kommer fr{\aa}n
$\nabla\cdot{\bf D}=0$, och {\it ej} fr{\aa}n ``$\nabla\cdot{\bf E}=0$'' (som
i grund och botten bara {\"a}r ett specialfall, om {\"a}n vanligt
f{\"o}rekommande).

F{\"o}r det magnetiska f{\"a}ltet g{\"a}ller samma sak, speciellt eftersom vi
{\it alltid} har att $\nabla\cdot{\bf B}=0$. Med andra ord, f{\"o}r
{\it homogena} media utan fria laddningar kan vi ers{\"a}tta
$$
  \nabla\times\nabla\times\qquad\to\qquad-\nabla^2,
$$
och ekvationerna f{\"o}r elektromagnetisk v{\aa}gutbredning antar d{\aa} formen
$$
  \nabla^2{\bf E} - \mu_0\varepsilon_0\varepsilon_{\rm r}
    {{\partial^2{\bf E}}\over{\partial t^2}} = 0,
  \qquad\qquad
  \nabla^2{\bf B} - \mu_0\varepsilon_0\varepsilon_{\rm r}
    {{\partial^2{\bf B}}\over{\partial t^2}} = 0.
$$
Med detta i bagaget kan vi s{\aa} l{\"a}nge g{\aa} vidare med att enbart
studera det elektriska f{\"a}ltet, och konstatera att en analys av det
magnetiska f{\"a}ltet f{\"o}ljer helt analogt. Notera dock att {\"a}ven om
dessa ekvationer skenbart ger vid hand att ${\bf E}$- och ${\bf B}$-f{\"a}lten
ser ut att vara helt frikopplade fr{\aa}n varandra, s{\aa} har de fortfarande
kopplingen sinsemellan via exempelvis Faradays induktionslag. D{\"a}rf{\"o}r
{\"a}r det oftast en mer korrekt v{\"a}g, med f{\"a}rre f{\"a}llor att ramla
ner i vad g{\"a}ller korrekt formulerade randvillkor och initialv{\"a}rden,
om man f{\"o}rst l{\"o}ser f{\"a}ltproblemet f{\"o}r det elektriska f{\"a}ltet
och d{\"a}refter (om man beh{\"o}ver det magnetiska f{\"a}ltet i analysen)
anv{\"a}nda Faradays eller Amp\`eres lag f{\"o}r att direkt l{\"a}nka in
l{\"o}sningen f{\"o}r magnetf{\"a}ltet.

\section{Plana elektromagnetiska f{\"a}lt och d'Alemberts generella l{\"o}sning
         till v{\aa}gekvationen}
\sidx{Plana elektromagnetiska f{\"a}lt}
\sidx{d'Alembert, Jean le Rond (1717--1783)}[d'Alemberts l{\"o}sningsmetod
  (1747)]
Formen med ``$\nabla^2$'' f{\"o}r de elektromagnetiska f{\"a}ltekvationerna
{\"a}r viktig och kan i m{\aa}nga fall l{\"o}sas {\"a}ven analytiskt, givet
att geometrin {\"a}r tillr{\"a}ckligt enkel. Denna form kan anv{\"a}ndas
s{\aa}v{\"a}l f{\"o}r att beskriva diffraktion som tidsberoende fenomen som
dispersion och pulsbreddning. Denna generella form {\"a}r sj{\"a}lvfallet
ocks{\aa} anv{\"a}ndbar som utg{\aa}ngspunkt f{\"o}r analys i cylindriska
eller sf{\"a}riska koordinater, i vilket fall vi f{\aa}r uttrycka $\nabla^2$
i respektive system.

Vi kommer nu dock att ytterligare f{\"o}renkla beskrivningen genom att anta att
v{\aa}gutbredningen sker i form av {\it o{\"a}ndligt utstr{\"a}ckta plana
v{\aa}gor}, f{\"o}r vilka vi kan f{\"o}rsumma spatial breddning eller
fokusering tv{\"a}rs axeln l{\"a}ngs med vilken v{\aa}gorna breder ut sig.
I detta fall ers{\"a}tter vi, om vi antar v{\aa}gutbredning l{\"a}ngs med
$z$-axeln i ett kartesiskt koordinatsystem,
$$
  \nabla^2\qquad\to\qquad{{\partial^2}\over{\partial z^2}}
$$
Eftersom den elektromagnetiska v{\aa}gekvationen s{\aa} l{\aa}ngt i
approximationen f{\"o}rvisso {\"a}r p{\aa} vektoriell form, men med samtliga
koefficienter och operatorer som skal{\"a}rer ({\"a}ven $\nabla^2$!), s{\aa}
kan vi {\"a}ven v{\"a}lja att studera endast en komponent av respektive
f{\"a}lt. F{\"o}r att g{\aa} vidare endast med det elektriska f{\"a}ltet,
eftersom magnetiska f{\"a}ltet som vi sett dessutom f{\"o}ljer analogt, s{\aa}
blir den partiella differentialekvationen omformulerad som
$$
  {{\partial^2 E(z,t)}\over{\partial z^2}}
    -\mu_0\varepsilon_0\varepsilon_{\rm r}
      {{\partial^2 E(z,t)}\over{\partial t^2}} = 0.
$$
Redan h{\"a}r kan vi med viss grad av kunskap om partiella
differentialekvationer dra slutsatsen att hastigheten f{\"o}r v{\aa}gor som
beskrivs av denna ekvation m{\aa}ste ha beloppet
$(\mu_0\varepsilon_0\varepsilon_{\rm r})^{-1/2}$.
Vi kommer d{\"a}rf{\"o}r att, enbart f{\"o}r att f{\"o}renkla notationen,
{\it definiera}
$$
  c \equiv {{1}\over{(\mu_0\varepsilon_0\varepsilon_{\rm r})^{1/2}}},
$$
{\"a}ven om vi s{\aa} l{\"a}nge kommer att h{\aa}lla sj{\"a}lva tolkningen av
$c$ {\"o}ppen fram till dess att vi explicit visat hur l{\"o}sningen ser ut.
\vfill\eject

\subsection{d'Alemberts variabelsubstitution}
Homogena partiella differentialekvationer med termer d{\"a}r derivatorna med
avseende p{\aa} olika variabler {\"a}r separerade {\"a}r skolexempel p{\aa}
fall d{\"a}r vi kan till{\"a}mpa {\it variabelseparation}. Vi kommer dock
h{\"a}r att beskriva ``standards{\"a}ttet'' att angripa v{\aa}gekvationen
p{\aa}.\numberedfootnote{Griffiths v{\"a}ljer tyv{\"a}rr att inte g{\aa}
  igenom d'Alemberts f{\"o}rh{\aa}llandevis enkla f{\"o}rklaring till att
  v{\aa}gekvationen st{\"o}djer godtycklig v{\aa}gform s{\aa} l{\"a}nge som
  vi har argument av formen $z\pm ct$, utan v{\"a}ljer att bara visa att
  dessa l{\"o}sningar uppfyller v{\aa}gekvationen. Se Griffiths sid.~382--385.}
F{\"o}rst av allt, byt variabler till\numberedfootnote{Detta
   tillv{\"a}gag{\aa}ngss{\"a}tt f{\"o}ljer i stort sett den h{\"a}rledning av
   den generella l{\"o}sningen till v{\aa}g\-ekva\-tionen som gjordes 1747 av
   den franske matematikern Jean le Rond d'Alembert, fransk matematiker och
   upplysningsfilosof (1717--1783). Faktum {\"a}r att d'Alembert-operatorn
   \sidx{d'Alembert, Jean le Rond (1717--1783)}[d'Alembert-operator]
   $$
     \square\equiv{{1}\over{c^2}}{{\partial^2}\over{\partial t^2}}-\nabla^2
   $$
   {\"a}r uppkallad efter honom. F{\"o}r en detaljerad om {\"a}n n{\aa}got
   ostrukturerad beskrivning av metodiken, se Wikipedia-artikeln
   {\tt https://en.wikipedia.org/wiki/D\%27Alembert\%27s\_formula}.}
$$
  \xi = z - ct,
  \qquad\qquad
  \eta = z + ct.
$$
F{\"o}r derivatorna med avseende p{\aa} $\xi$ och $\eta$ har vi med
anv{\"a}ndande av kedjeregeln i tv{\aa} variabler att
$$
  \eqalign{
    {{\partial}\over{\partial z}}
      &=\underbrace{{{\partial\xi}\over{\partial z}}}_{=1}
           {{\partial}\over{\partial\xi}}
         +\underbrace{{{\partial\eta}\over{\partial z}}}_{=1}
	   {{\partial}\over{\partial\eta}}
       ={{\partial}\over{\partial\xi}}
         +{{\partial}\over{\partial\eta}}\cr
  &\hskip20pt\Rightarrow
    {{\partial^2 E}\over{\partial z^2}}
       =\bigg({{\partial}\over{\partial\xi}}
         +{{\partial}\over{\partial\eta}}\bigg)
        \bigg({{\partial E}\over{\partial\xi}}
         +{{\partial E}\over{\partial\eta}}\bigg)
       ={{\partial^2 E}\over{\partial\xi^2}}
	 +2{{\partial^2 E}\over{\partial\xi\partial\eta}}
         +{{\partial^2 E}\over{\partial\eta^2}},\cr
    {{\partial}\over{\partial t}}
      &=\underbrace{{{\partial\xi}\over{\partial t}}}_{=-c}
           {{\partial}\over{\partial\xi}}
         +\underbrace{{{\partial\eta}\over{\partial t}}}_{=c}
	   {{\partial}\over{\partial\eta}}
       =-c\bigg({{\partial}\over{\partial\xi}}
         -{{\partial}\over{\partial\eta}}\bigg)\cr
  &\hskip20pt\Rightarrow
    {{\partial^2 E}\over{\partial t^2}}
       =c^2\bigg({{\partial}\over{\partial\xi}}
         -{{\partial}\over{\partial\eta}}\bigg)
        \bigg({{\partial E}\over{\partial\xi}}
         -{{\partial E}\over{\partial\eta}}\bigg)
       =c^2\bigg({{\partial^2 E}\over{\partial\xi^2}}
	 -2{{\partial^2 E}\over{\partial\xi\partial\eta}}
         +{{\partial^2 E}\over{\partial\eta^2}}\bigg).\cr
  }
$$
Detta f{\"o}r {\"o}ver den partiella differentialekvationen f{\"o}r $E$ till
formen
$$
  \underbrace{
  {{\partial^2 E}\over{\partial\xi^2}}
     +2{{\partial^2 E}\over{\partial\xi\partial\eta}}
     +{{\partial^2 E}\over{\partial\eta^2}}
     }_{\displaystyle={{\partial^2E}\over{\partial z^2}}}
   -{{1}\over{c^2}}
  \underbrace{
     c^2\bigg({{\partial^2 E}\over{\partial\xi^2}}
     -2{{\partial^2 E}\over{\partial\xi\partial\eta}}
     +{{\partial^2 E}\over{\partial\eta^2}}\bigg)
     }_{\displaystyle={{\partial^2E}\over{\partial t^2}}} = 0
  \qquad\Leftrightarrow\qquad
  {{\partial^2 E}\over{\partial\xi\partial\eta}} = 0.
$$
Denna partiella differentialekvation (PDE) {\"a}r {\it linj{\"a}r}, s{\aa}
superpositionsprincipen -- som s{\"a}ger att l{\"o}sningar som individuellt
satisfierar ekvationen kan adderas till varandra och fortfarande (som summa
betraktad) bist{\aa} med en l{\"o}sning till ekvationen -- g{\"a}ller generellt.
Den speciella formen p{\aa} ekvationen s{\"a}ger oss ocks{\aa} att om vi finner
en l{\"o}sning s{\aa} kan den {\it enbart} vara en funktion av {\it antingen}
$\xi$ {\it eller} $\eta$, eftersom andraderivatan i den andra, parvisa,
variabeln garanterar att endast l{\"o}sningar av denna form kommer att
satisfiera ekvationen.
\vfill\eject

\subsection{Integration of d'Alemberts transformerade v{\aa}gekvation}
\sidx{d'Alembert, Jean le Rond (1717--1783)}[Integration of d'Alemberts
  transformerade v{\aa}gekvation]
Vi v{\"a}ljer att f{\"o}rst integrera d'Alemberts ekvation i de transformerade
variablerna med avseende p{\aa} $\xi$, vilket resulterar i ett h{\"o}gerled med
en ``integrationskonstant'' i form av en funktion $h$ som {\it endast kan bero
p{\aa} variabeln $\eta$} (eftersom partialderivatan $\partial/\partial\xi$ av
denna ``konstant'' m{\aa}ste bli identiskt noll), som
$$
  (\hbox{``$\int d\xi$`` }\to)\qquad
  {{\partial E}\over{\partial\eta}}=h(\eta).
$$
Om vi nu integrerar denna med avseende p{\aa} den kvarvarande variabeln $\eta$,
s{\aa} kommer det ing{\aa}ende h{\"o}gerledet $h(\eta)$ ist{\"a}llet att bli
den primitiva funktionen, s{\"a}g $g(\eta)=\int h(\eta)\,d\eta=H(\eta)$, samt
att en till ``integrationskonstant'' som endast beror av $\xi$ m{\aa}ste
l{\"a}ggas till, s{\"a}g en funktion $f(\xi)$, med f{\"o}ljd att
$$
  (\hbox{``$\int d\eta$`` }\to)\qquad
  E=f(\xi)+\underbrace{g(\eta)}_{=H(\eta)}.
$$
Om vi ers{\"a}tter de transformerade variablerna med de ursprungliga,
$\xi=z-ct$ och $\eta=z+ct$, s{\aa} erh{\aa}ller vi allts{\aa} den generella
l{\"o}sningen som
$$
  E = f(\underbrace{z-ct}_{=\xi}) + g(\underbrace{z+ct}_{=\eta}).
$$
Denna generella l{\"o}sning beskriver dels en v{\aa}g $f(z-ct)$ som r{\"o}r sig
i {\it positiv} $z$-led med hastigheten $c$, dels en v{\aa}g $g(z+ct)$ som
r{\"o}r sig i {\it negativ} $z$-led med samma hastighet. B{\aa}da dessa
v{\aa}gor {\"a}r generella till sin form, och beh{\"o}ver ej vara ``klassiska
harmoniska sinusv{\aa}gor''. V{\aa}gorna beh{\"o}ver ej heller vara repetitiva,
utan kan lika v{\"a}l beskriva solit{\"a}ra pulser.\sidx{Motpropagerande
v{\aa}gor}

\subsection{Tolkning av reducerad fashastighet och brytningsindex}
\sidx{Brytningsindex $n$}
\sidx{Ljushastighet}[I dielektrikum, $c$]
\sidx{Ljushastighet}[I vakuum, $c_0$]
\sidx{Magnetisk permeabilitet}[Vakuumpermeabilitet $\mu_0$]
\sidx{Elektrisk permittivitet}[Vakuumpermittivitet $\varepsilon_0$]
\sidx{Elektrisk permittivitet}[Relativ permittivitet $\varepsilon$]
Det intressanta h{\"a}r {\"a}r att vi nu har en tolkning av
$$
  c \equiv {{1}\over{(\mu_0\varepsilon_0\varepsilon_{\rm r})^{1/2}}}
$$
som varandes den hastighet som den elektromagnetiska v{\aa}gr{\"o}relsen har i
mediet, och d{\aa} ljus\-hastig\-heten i vakuum {\"a}r
$$
  c_0 \equiv {{1}\over{(\mu_0\varepsilon_0)^{1/2}}}
      = 299\,792\,458\ {\rm m}/{\rm s}\qquad(\hbox{exakt})
$$
s{\aa} inneb{\"a}r det att vi i $c$ har en v{\aa}g som propagerar
l{\aa}ngsammare med en faktor $n$ enligt
$$
  c = c_0/n,
$$
d{\"a}r
$$
  n \equiv\varepsilon^{1/2}_{\rm r}
$$
{\"a}r {\it brytningsindex} f{\"o}r mediet.

\subsection{Initialvillkor}
\sidx{Initialvillkor f{\"o}r v{\aa}gpropagation}
F{\"o}r den formella l{\"o}sningen f{\"o}r $E(z,t)$ {\"o}ver en dom{\"a}n,
s{\"a}g, $0\le z\le L$, beh{\"o}ver vi dessutom dels initialv{\"a}rdet
$$
  E(z,0) = f(z) + g(z),
$$
men {\"a}ven tidsderivatan
$$
  {{\partial E(z,t)}\over{\partial t}}\bigg|_{t=0}
    = -c{{df(z)}\over{dz}} + c{{dg(z)}\over{dz}},
$$
varefter vi kan integrera l{\"o}sningen och f{\aa} fram tidsberoendet hos de
fram{\aa}t- och bak{\aa}tpropagerande v{\aa}gorna. F{\"o}r detaljer kring
d'Alemberts l{\"o}sningsmetodik f{\"o}r dessa initialv{\"a}rdesproblem, se
praktiskt taget vilken standardbok som helst f{\"o}r en grundkurs i partiella
differentialekvationer.\numberedfootnote{Exempelvis D.~W.~Trim, {\it Applied
Partial Differential Equations} (PWS-Kent, 1990), s.~48.}

\section{Tidsharmoniska f{\"a}lt och planv{\aa}guppdelning}
\sidx{Tidsharmoniska f{\"a}lt}
\sidx{Planv{\aa}guppdelning}
I f{\"o}reg{\aa}ende sektion visade vi f{\"o}r en plan v{\aa}g att vi har en
generell l{\"o}sning i form av en fram{\aa}tg{\aa}ende och en
bak{\aa}tg{\aa}ende v{\aa}g med samma fashastighet $c$. Detta visade vi f{\"o}r
en {\it generell v{\aa}gform}, som inte beh{\"o}ver vara av harmonisk
karakt{\"a}r (med vilket vi l{\"o}st formulerat menar av typen
``sinus-l{\"o}sning''). Samtidigt vet vi fr{\aa}n teorin bakom
differentialekvationer ({\"a}ven partiella s{\aa}dana) att
superpositionsprincipen g{\"a}ller, varvid vi kan analysera l{\"o}sningar
f{\"o}r olika delar av ett elektromagnetiskt f{\"a}lt, och d{\"a}refter
sammanfoga den totala l{\"o}sningen f{\"o}r f{\"a}ltet.

\subsection{Komplexv{\"a}rda f{\"a}ltenvelopper}
\sidx{Komplexv{\"a}rda f{\"a}ltenvelopper}
\sidx{Elektrisk f{\"a}ltstyrka ${\bf E}$}[Komplexv{\"a}rd envelopp]
\sidx{Magnetisk fl{\"o}dest{\"a}thet ${\bf B}$}[Komplexv{\"a}rd envelopp]
Formen p{\aa} den homogena differentialekvationen,\numberedfootnote{Den
  f{\"o}ljande behandlingen av plana v{\aa}gor f{\"o}ljer i huvudsak
  Griffiths Kap.~9.2, sid.~393--415.}
$$
  \nabla^2{\bf E} - {{1}\over{c^2}} % \mu_0\varepsilon_0\varepsilon_{\rm r}
    {{\partial^2{\bf E}}\over{\partial t^2}} = 0,
  \qquad\qquad
  \nabla^2{\bf B} - {{1}\over{c^2}} % \mu_0\varepsilon_0\varepsilon_{\rm r}
    {{\partial^2{\bf B}}\over{\partial t^2}} = 0,
$$
ger dels vid hand att det i det h{\"a}r antagna {\it isotropa} och
{\it homogena} mediet g{\aa}r att frikoppla olika polarisationstillst{\aa}nd
fr{\aa}n varandra (eftersom ekvationerna till{\aa}ter att vi helt enkelt
skal{\"a}r±-multipli\-cerar med en enhetsvektor {\aa}t n{\aa}got h{\aa}ll
och direkt f{\aa}r en frikopplad skal{\"a}r partiell differentialekvation
f{\"o}r just det polarisationstillst{\aa}ndets f{\"a}lt), men {\"a}ven att
vi kan dela f{\"a}lten i deras respektive harmoniska frekvensinneh{\aa}ll,
enligt\numberedfootnote{Ett litet sidosp{\aa}r som kan vara v{\"a}rt att
  n{\"a}mna: Om vi i formen f{\"o}r de homogena differentialekvationerna
  stannar vid att enbart projicera ut den tidsharmoniska delen, genom
  $$
    {\bf E}({\bf x},t)=\sum_{\bf k}\Re[\tilde{\bf E}\exp(-i\omega({\bf k}) t)],
    \qquad\qquad
    {\bf B}({\bf x},t)=\sum_{\bf k}\Re[\tilde{\bf B}\exp(-i\omega({\bf k}) t)],
  $$
  s{\aa} erh{\aa}ller vi egenv{\"a}rdesproblemen f{\"o}r Laplace-operatorn som
  $$
    \nabla^2\tilde{\bf E} - {{\omega^2}\over{c^2}}\tilde{\bf E} = 0,
    \qquad\qquad
    \nabla^2\tilde{\bf B} - {{\omega^2}\over{c^2}}\tilde{\bf B} = 0,
  $$
  i form av {\it Helmholtz ekvationer} f{\"o}r de komplexv{\"a}rda
  tids-envelopperna $\tilde{\bf E}$ och $\tilde{\bf B}$. Dessa ekvationer
  {\"a}r uppkallade efter Hermann von Helmholtz, tysk fysiker (1821--1894).
  \sidx{von Helmholtz, Hermann (1821--1894)}
  \sidx{Laplace-operatorn}[Egenv{\"a}rdesproblem f{\"o}r]}
$$
  {\bf E}({\bf x},t)=\sum_{\bf k}
     \Re[{\bf E}_{\bf k}\exp(i{\bf k}\cdot{\bf x}-i\omega({\bf k}) t)],
  \qquad\qquad
  {\bf B}({\bf x},t)=\sum_{\bf k}
     \Re[{\bf B}_{\bf k}\exp(i{\bf k}\cdot{\bf x}-i\omega({\bf k}) t)],
$$
d{\"a}r ${\bf k}$ {\"a}r {\it v{\aa}gvektorn} f{\"o}r komponenten med komplex
amplitud ${\bf E}_{\bf k}$ samt d{\"a}r $\omega({\bf k})$ {\"a}r
vinkelhastigheten (rad/s) f{\"o}r oscillationen hos f{\"a}lten i tid.
Om vi har generella v{\aa}gformer som inte {\"a}r av ``sinus-typ'', s{\aa} kan
vi alltid f{\aa} fram dessa genom addition av harmoniska f{\"a}lt till en
{\it Fourier-utveckling}, med respektive amplitud hos varje frekvenskomponent
som {\it Fourier-koefficienter}.\sidx{V{\aa}gvektor ${\bf k}$}

Denna tids- och rums-harmoniska form f{\"o}r {\"o}ver de partiella
differentialekvationerna f{\"o}r f{\"a}lten p{\aa}
formen\numberedfootnote{Eftersom $\nabla\to{\bf k}\cdot$ och
  ${{\partial}\over{\partial t}}\to -i\omega$.}
$$
  \Big(k^2 - {{\omega^2}\over{c^2}}\Big) {\bf E} = 0,
  \qquad\qquad
  \Big(k^2 - {{\omega^2}\over{c^2}}\Big) {\bf B} = 0,
$$
fr{\aa}n vilket vi direkt (f{\"o}r noll-skilda f{\"a}lt) f{\"o}r beloppet
$k=|{\bf k}|$ f{\"o}r v{\aa}gvektorn f{\aa}r att
$$
  k = {{\omega}\over{c}}
    = {{\omega}\over{(c_0/n)}},
$$
d{\"a}r, f{\"o}r att rekapitulera,
$$
  \hskip100pt\hphantom{ljushastigheten}
  c_0 = {{1}\over{\sqrt{\mu_0\varepsilon_0}}},
  \hskip60pt(\hbox{ljushastigheten i vakuum})
$$
samt
$$
  \hskip69pt\hphantom{brytningsindex}
  n = \sqrt{\varepsilon_{\rm r}}.
  \hskip69pt(\hbox{brytningsindex})
$$

\subsection{Ortogonalitet mellan f{\"a}lt och v{\aa}gvektor}
\sidx{Komplexv{\"a}rda f{\"a}ltenvelopper}
\sidx{Ortogonalitet mellan f{\"a}lt och v{\aa}gvektor}
Med den tids- och rums-harmoniska formen f{\aa}r vi {\"a}ven direkt ett par
andra karakteristiska egenskaper f{\"o}r elektromagnetisk v{\aa}gutbredning
i homogena och isotropa material, exempelvis fr{\aa}n Faradays lag, att
$$
  \nabla\times{\bf E}=-{{\partial{\bf B}}\over{\partial t}}
  \qquad\Rightarrow\qquad
  i{\bf k}\times{\bf E}_{\bf k} = i\omega{\bf B}_{\bf k}
  \qquad\Leftrightarrow\qquad
  {\bf B}_{\bf k}={\bf k}\times{\bf E}_{\bf k}/\omega,
$$
fr{\aa}n vilket vi direkt har ortogonalitet mellan den elektriska
f{\"a}ltstyrkan ${\bf E}_{\bf k}$ och magnetiska fl{\"o}des\-t{\"a}t\-heten
${\bf B}_{\bf k}$ som
$$
  {\bf E}_{\bf k}\cdot{\bf B}_{\bf k}=0,
$$
men {\"a}ven ortogonatlitet mellan magnetiska fl{\"o}dest{\"a}theten och
v{\aa}gvektorn, d{\aa}\sidx{V{\aa}gvektor ${\bf k}$}
$$
  {\bf k}\cdot{\bf B}_{\bf k}=0.
$$
Fr{\aa}n Gauss lag\numberedfootnote{I fr{\aa}nvaro av fria elektriska
  laddningar, $\nabla\cdot{\bf D}=\nabla\cdot(\varepsilon_0\varepsilon_{\rm r}
  {\bf E})=\varepsilon_0\varepsilon_{\rm r}\nabla\cdot{\bf E}=0$ med
  $\varepsilon_{\rm r}$ som oberoende av rumskoordinater.}
har vi {\"a}ven motsvarande ortogonalitet mellan den elektriska f{\"a}ltstyrkan
och v{\aa}gvektorn som
$$
  \nabla\cdot{\bf E}=0
  \qquad\Rightarrow\qquad
  {\bf k}\cdot{\bf E}_{\bf k}=0.
$$
Detta sammantaget inneb{\"a}r att vektorerna $({\bf E}_{\bf k},{\bf B}_{\bf k},
{\bf k})$ bildar ett {\it h{\"o}gerhands-orienterat system}. Med andra ord,
s{\aa} kan vi sammanfatta detta med att de elektriska och magnetiska f{\"a}lten
i homogena och isotropa media alltid {\"a}r ortogonala mot varandra samt
ortogonala mot v{\aa}gvektorn (riktningen) f{\"o}r v{\aa}gutbredningen.
\epsfig{../lect-10/figs/ebk.1}

\section{Poynting-vektorn}
\sidx{Poynting-vektorn ${\bf S}$}[Effekt transporterat av elektromagnetiskt
  f{\"a}lt]
Som ett m{\aa}tt p{\aa} den effekt som transporteras per ytenhet i
tv{\"a}rsnitt av ett elektromagnetiskt f{\"a}lt, definieras
{\it Poynting-vektorn} ${\bf S}$ som
$$
  {\bf S}={\bf E}\times{\bf H},
$$
med enheten ${\rm W}/{\rm m}^2$.
\vfill\eject

\section{Sammanfattning av F{\"o}rel{\"a}sning~10
  -- V{\aa}gutbredning i homogena och isotropa dielektrika}
\item{$\bullet$}{Den generella elektromagnetiska v{\aa}gekvationen,
  $$
    \eqalign{
      \nabla\times\nabla\times{\bf E}
        +\mu_0\varepsilon_0{{\partial^2{\bf E}}\over{\partial t^2}}&=
           -\mu_0{{\partial}\over{\partial t}}
            \underbrace{
               \bigg({\bf J}_{\rm f}
  	        +{{\partial{\bf P}}\over{\partial t}}
  	        +\nabla\times{\bf M}\bigg)
                  }_{\hbox{gemensam k{\"a}llterm}},\cr
      \nabla\times\nabla\times{\bf B}
        +\mu_0\varepsilon_0{{\partial^2{\bf B}}\over{\partial t^2}}&=
            \mu_0\nabla\times
            \underbrace{
  	     \bigg({\bf J}_{\rm f}
  	        +{{\partial{\bf P}}\over{\partial t}}
  	        +\nabla\times{\bf M}\bigg)
                  }_{\hbox{gemensam k{\"a}llterm}}.\cr
    }
  $$}
\item{$\bullet$}{Fyra allm{\"a}nna f{\"o}renklingar,
  \litem[1.]{${\bf J}_{\rm f}={\bf 0}$ (inga fria str{\"o}mmar)}
  \litem[2.]{$\rho=0$ (inga fria laddningar)}
  \litem[3.]{${\bf P}=\varepsilon_0(\varepsilon_{\rm r}-1){\bf E}$
             ({\it linj{\"a}rt} medium, och vi antar {\"a}ven homogent
             $\varepsilon_{\rm r}({\bf x})=\hbox{konstant}$)}
  \litem[4.]{${\bf M}={\bf 0}$ (f{\"o}rsumbar magnetisering,
             $\mu_{\rm r}\approx1$)}
  \smallskip
  leder till att v{\aa}gekvationerna antar formen
  $$
    \nabla^2{\bf E}-{{1}\over{c^2}}{{\partial^2{\bf E}}\over{\partial t^2}}=0,
    \qquad\quad
    \nabla^2{\bf B}-{{1}\over{c^2}}{{\partial^2{\bf B}}\over{\partial t^2}}=0,
  $$
  d{\"a}r $c=c_0/n$ {\"a}r den hastighet som v{\aa}gen har i mediet, med
  ljus\-hastigheten i vakuum som $c_0=1/\sqrt{\mu_0\varepsilon_0}$ och
  brytnings\-index $n=\sqrt{\varepsilon_{\rm r}}$.}
\item{$\bullet$}{Med d'Alemberts l{\"o}sningsmetod ges generella l{\"o}sningar
  till v{\aa}gekvationen i en dimension som
  $$
    E(z,t) = f(z-ct) + g(z+ct).
  $$}
\item{$\bullet$}{Vi utf{\"o}r en planv{\aa}gsuppdelning av den elektromagnetiska
  v{\aa}gen med assistans av komplex-v{\"a}rda envelopper ${\bf E}_{\bf k}$ och
  ${\bf B}_{\bf k}$ som
  $$
    \eqalign{
      {\bf E}({\bf x},t)&=\sum_{\bf k}
         \Re[{\bf E}_{\bf k}\exp(i{\bf k}\cdot{\bf x}-i\omega({\bf k}) t)],\cr
      {\bf B}({\bf x},t)&=\sum_{\bf k}
         \Re[{\bf B}_{\bf k}\exp(i{\bf k}\cdot{\bf x}-i\omega({\bf k}) t)].\cr
    }
  $$}
\item{$\bullet$}{
  De komplexv{\"a}rda envelopperna i planv{\aa}gsuppdelningen {\"a}r ortogonala
  enligt
  $$
    {\bf B}_{\bf k}={\bf k}\times{\bf E}_{\bf k}/\omega,\qquad
    {\bf E}_{\bf k}\cdot{\bf B}_{\bf k}=0,\qquad
    {\bf k}\cdot{\bf B}_{\bf k}=0,\qquad
    {\bf k}\cdot{\bf E}_{\bf k}=0.
  $$}
\item{$\bullet$}{Den effekt som per ytenhet transporteras i tv{\"a}rsnitt av
  en elektromagnetisk v{\aa}g ges av {\it Poynting-vektorn},
  $$
    {\bf S}={\bf E}\times{\bf H},
  $$
  med enheten ${\rm W}/{\rm m}^2$.}

\cleardoublepage
%%% End of auto-extracted text from ../lect-10/lecture-10.tex %%%
%%% Begin of auto-extracted text from ../lect-11/lecture-11.tex %%%
%
% File: teach/elmagii/lect-11/lecture-11.tex [plain TeX code]
% Github: https://github.com/elmagii/lect-11/
% Last change: December 15, 2025
%
% Lecture No 11 in the course ``Elektromagnetism II, 1TE626 (2023)'',
% held December 1, 2025, at Uppsala University, Sweden.
%
% Copyright (C) 2022-2025, Fredrik Jonsson, under Gnu General Public License
% (GPL) v3. See the enclosed LICENSE for details.
%
% This program is free software: you can redistribute it and/or modify
% it under the terms of the GNU General Public License as published by
% the Free Software Foundation, either version 3 of the License, or
% (at your option) any later version.
%
% This program is distributed in the hope that it will be useful,
% but WITHOUT ANY WARRANTY; without even the implied warranty of
% MERCHANTABILITY or FITNESS FOR A PARTICULAR PURPOSE.  See the
% GNU General Public License for more details.
%
% You should have received a copy of the GNU General Public License
% along with this program.  If not, see <https://www.gnu.org/licenses/>.
%
\def\coursename{Elektromagnetism II}
\def\coursecode{1TE626}
\def\courseyear{2025}
\def\courserepo{https://github.com/hp35/elmagii/}
\def\lecturenumber{11}
\def\lecturetitle{Retarderade potentialer som l{\"o}sningar till Maxwells ekvationer}
\def\lecturesubtitle{}
\def\lectureauthor{Fredrik Jonsson}
\def\lectureplace{Uppsala Universitet}
\def\lecturedate{1 december 2025}
%-------------------- BEGIN OF LOCAL MACROS --------------------
\edef\expandedlecturenumber{11}
\def\ifempty#1{\ifx\relax#1\relax}
\advance\chapno by 1
\secno=0
\footnotenumber=0
\message{==================== Lecture 11 ====================}
\writenumberedtocentry{chapter}{F{\"o}rel{\"a}sning 11 -- {Retarderade potentialer som l{\"o}sningar till Maxwells ekvationer}}{\thechapno}
\hsize=150mm\hoffset=4.6mm\vsize=230mm\voffset=7mm
\topskip=0pt\baselineskip=12pt\parskip=0pt\leftskip=0pt\parindent=15pt
\ifcolors
  \voffset=-10.2mm\topskip=0pt
\fi
\headline={\ifnum\secno>0\ifodd\pageno\rightheadline\else\leftheadline\fi
  \else\hfill\fi}
\def\rightheadline{\tenrm{\it F\"orel\"asning 11}
  \hfil{\it \coursename, \coursecode\ (\courseyear)}}
\def\leftheadline{\tenrm{\it \coursename, \coursecode\ (\courseyear)}
  \hfil{\it F\"orel\"asning 11}}
\noindent~\vskip-60pt\hskip-40pt{\epsfbox{../lect-01/macros/UU_logo_color.eps}}
\vskip-42pt\hfill\vbox{
    \hbox{{\it \coursename, \coursecode\ (\courseyear)}}
    \hbox{{\it Lecture Notes, \lectureauthor}}
    \hbox{{\it Document Revision \today}}
    \hbox{{\it \courserepo}}}\vskip 36pt
\centerline{\twelvesc F\"orel\"asning 11}
\vskip 24pt\noindent
\centerline{\twelvesc{Retarderade potentialer som l{\"o}sningar till Maxwells ekvationer}}
\expandafter\ifempty\expandafter{\lecturesubtitle}%
  \else\centerline{\twelvesc\lecturesubtitle}\fi
\bigskip
\centerline{\lectureauthor, \lectureplace, \lecturedate}
\vskip24pt
%--------------------- END OF LOCAL MACROS ---------------------



\plan{Vi inleder med en rekapitulation av den generella formen p{\aa} den
  elektromagnetiska v{\aa}gekvationen f{\"o}r elektriska och magnetiska
  f{\"a}lt, f{\"o}ljt av en analogi mellan potentialerna och mekaniska system.
  Utifr{\aa}n definitionen av vektorpotentialen ${\bf A}$ som
  ${\bf B}=\nabla\times{\bf A}$ fr{\aa}n identiteten $\nabla\cdot{\bf B}=0$,
  s{\aa} erh{\aa}ller vi via Faradays lag att det elektriska f{\"a}ltet i
  elektrodynamiska problem m{\aa}ste inneh{\aa}lla {\"a}ven vektorpotentialen
  som ${\bf E}=-\nabla\phi-{{\partial{\bf A}}/{\partial t}}$.

  V{\aa}gekvationen f{\"o}r de elektrodynamiska potentialerna formuleras,
  f{\"o}ljt av en genomg{\aa}ng av en viss form av godtycke som existerar
  i valet av potentialerna, under den s{\aa} kallade gauge-transformen.
  Vi finner under antagande om Lorenz-villkoret (Lorenz gauge) att
  ekvationerna f{\"o}r potentialermna frikopplas fr{\aa}n varandra.
  Formen f{\"o}r de retarderade potentialerna, d{\"a}r ett objekts verkan
  p{\aa} avst{\aa}nd analyseras, formuleras p{\aa} integralform f{\"o}r
  den skal{\"a}ra potentialen och vektorpotentialen, och exemplifieras
  slutligen f{\"o}r en tunn dipolantenn.}

\threepointsummary{%
  Den elektrodynamiska formen f{\"o}r den skal{\"a}ra potentialen $\phi$ och
  vektorpotentialen ${\bf A}$ lyder
  $$
    {\bf E}({\bf x},t)
      =-\nabla\phi({\bf x},t)-{{\partial{\bf A}({\bf x},t)}\over{\partial t}},
      \qquad
    {\bf B}({\bf x},t)
      =\nabla\times{\bf A}({\bf x},t),
  $$
  f{\"o}r vilka de s{\aa} kallade gauge-transformationen av potentialerna,
  $$
    {\bf A}'({\bf x},t)={\bf A}({\bf x},t)+\nabla\psi({\bf x},t),\qquad
    \phi'({\bf x},t)=\phi({\bf x},t)
      -{{\partial\psi({\bf x},t)}\over{\partial t}},
  $$
  l{\"a}mnar de elektromagnetiska f{\"a}lten of{\"o}r{\"a}ndrade. Vi kan
  med dessa till viss del v{\"a}lja vilken form av v{\aa}gekvation vi
  {\"o}nskar f{\"o}r potentialerna.
  \sidx{Skal{\"a}r potential $\phi$}[Elektrodynamisk]
  \sidx{Vektorpotential ${\bf A}$}[Elektrodynamisk]
}{%
  Specifikt erh{\aa}ller vi under Lorenz-villkoret, eller ``Lorenz gauge'',
  d{\"a}r
  $$
    \nabla\cdot{\bf A}'({\bf x},t)+
      {{1}\over{c^2}}
      {{\partial\phi'({\bf x},t)}\over{\partial t}} = 0.
  $$
  att v{\aa}gekvationerna f{\"o}r potentialerna frikopplas fr{\aa}n varandra,
  till att lyda\sidx{Lorenz-villkoret}
  \sidx{Lorenz-villkoret}[{{\it Lorenz gauge}}]
  $$
     \Big(
       \nabla^2-{{1}\over{c^2}}{{\partial^2}\over{\partial t^2}}
     \Big)\phi({\bf x},t)
        =-{{\rho({\bf x},t)}\over{\varepsilon_0\varepsilon_{\rm r}}},\quad
     \Big(
       \nabla^2-{{1}\over{c^2}}{{\partial^2}\over{\partial t^2}}
     \Big){\bf A}({\bf x},t)
        =-\mu_0{\bf J}_{\rm f}({\bf x},t).
  $$
}{%
  Retarderade potentialer i den retarderade tiden $t'=t-|{\bf x}-{\bf x}'|/c$
  ges som volymintegralerna
  \sidx{Retarderad potential}[Tidsf{\"o}rdr{\"o}jd potential]
  $$
    \phi({\bf x},t)={{1}\over{4\pi\varepsilon_0}}\iiint_{{\Bbb R}^3}
      {{\rho({\bf x}',t')}\over{|{\bf x}-{\bf x}'|}}\,dV',\qquad
    {\bf A}({\bf x},t)={{\mu_0}\over{4\pi}}\iiint_{{\Bbb R}^3}
      {{{\bf J}({\bf x}',t')}\over{|{\bf x}-{\bf x}'|}}\,dV'.
  $$
}
\vfill\eject\copyrights

\section{Elektrodynamiska f{\"a}lt och retarderade potentialer}
Vi har i tidigare f{\"o}rel{\"a}sningar kommit in p{\aa} hur Maxwells
ekvationer kan omformuleras till tv{\aa} v{\aa}gekvationer f{\"o}r den
elektriska f{\"a}ltstyrkan ${\bf E}$ och den magnetiska fl{\"o}dest{\"a}theten
${\bf B}$, med v{\"a}xel\-verkan mellan mediet och de elektromagnetiska
f{\"a}lten beskrivna av k{\"a}lltermer i h{\"o}gerledet enligt
\sidx{Elektromagnetisk v{\aa}gekvation}
$$
  \eqalign{
    \nabla\times\nabla\times{\bf E}
      +{{1}\over{c^2_0}}{{\partial^2{\bf E}}\over{\partial t^2}}&=
         -\mu_0{{\partial}\over{\partial t}}
          \underbrace{
             \bigg({\bf J}_{\rm f}
	        +{{\partial{\bf P}}\over{\partial t}}
	        +\nabla\times{\bf M}\bigg)}_{\hbox{gemensam k{\"a}llterm}},\cr
    \nabla\times\nabla\times{\bf B}
      +{{1}\over{c^2_0}}{{\partial^2{\bf B}}\over{\partial t^2}}&=
          \mu_0\nabla\times
          \underbrace{
	     \bigg({\bf J}_{\rm f}
	        +{{\partial{\bf P}}\over{\partial t}}
	        +\nabla\times{\bf M}\bigg)}_{\hbox{gemensam k{\"a}llterm}},\cr
  }
$$
\sidx{Str{\"o}mt{\"a}thet}[Fria laddningar, ${\bf J}_{\rm f}$]
\sidx{Elektrisk polarisationsdensitet ${\bf P}$}
\sidx{Magnetisering ${\bf M}$}
\sidx{Elektrisk f{\"a}ltstyrka ${\bf E}$}
\sidx{Magnetisk fl{\"o}dest{\"a}thet ${\bf B}$}
\sidx{Elektromagnetisk v{\aa}gekvation}[Gemensam k{\"a}llterm]
d{\"a}r ${\bf J}_{\rm f}$ {\"a}r den fria str{\"o}mt{\"a}theten, ${\bf P}$ den
elektriska polarisationsdensiteten och ${\bf M}$ magnetiseringen hos mediet.
Tidigare har vi anv{\"a}nt den skal{\"a}ra potentialen $\phi({\bf x})$ och
vektorpotentialen ${\bf A}({\bf x})$ i en {\it elektrostatisk} respektive
{\it magnetostatisk} analys. Specifikt har vi visat hur det statiska elektriska
f{\"a}ltet ${\bf E}$ och statiska magnetiska fl{\"o}dest{\"a}theten ${\bf B}$
direkt kan f{\aa}s fram ur dessa {\it statiska} potentialer som
${\bf E}=-\nabla\phi$ och ${\bf B}=\nabla\times{\bf A}$, och hur vi p{\aa}
ett strukturerat s{\"a}tt kan f{\aa} fram de {\it statiska} potentialerna genom
relativt enkla integraler ({\"o}ver volymer, ytor eller linjer).

Fr{\aa}gan infinner sig d{\aa} naturligtvis om det finns tidsberoende
motsvarigheter till dessa potentialer som kan appliceras p{\aa} {\it dynamiska}
(tidsberoende) elektromagnetiska f{\"a}lt? Vidare, finns det d{\aa} dessutom
m{\"o}jlighet att formulera en potentialteori d{\"a}r man tar h{\"a}nsyn till
den f{\"o}rdr{\"o}jning som onekligen m{\aa}ste ske fr{\aa}n det att en laddning
eller str{\"o}m manifesteras {\it lokalt vid en k{\"a}llpunkt} till dess att vi
verkligen observerar dess effekt vid en observationspunkt?
I s{\aa} fall skulle vi kunna g{\aa} via dessa potentialer och ur dessa extrahera
de tidsvarierande elektriska och magnetiska f{\"a}lten p{\aa} ett strikt
s{\"a}tt, inkluderande s{\aa}v{\"a}l dynamiken som tidsf{\"o}rdr{\"o}jningen
fr{\aa}n olika delar av k{\"a}llan eller k{\"a}llorna.

Svaret {\"a}r att det finns s{\aa}dana potentialer, s{\aa} kallade
{\it retarderade potentialer}, \sidx{Retarderad potential}[Tidsf{\"o}rdr{\"o}jd
potential] vilkas existens kan h{\"a}rledas fram (n{\"a}stan) analogt med det
elektrostatiska fallet.
\vfill\eject

\section{Recap p{\aa} vad skal{\"a}ra potentialen och vektorpotentialen
         {\"a}r bra f{\"o}r}
\sidx{Skal{\"a}r potential $\phi$}[Analogi med gravitation]
I en analogi med klassisk mekanik kan vi betrakta en punktmassa i ett
gravitationsf{\"a}lt ${\bf G}$ med gravitationskonstanten $g$ (N/kg).
Partikeln startar med en horisontell hastighet $v_0$ vid h{\"o}jden $z=h$
och f{\"a}rdas nedf{\"o}r en backe beskriven av funktionen $z=f(x)$.
En typisk uppgift skulle h{\"a}r kunna vara att r{\"a}kna ut sluthastigheten
$v$ vid $z=0$.

\epsfig{../lect-11/figs/analogi.1}\noindent
Vi kan naturligtvis rent principiellt st{\"a}lla upp r{\"o}relseekvationerna
f{\"o}r denna massa, g{\"o}ra vissa antaganden om huruvida hastigheten {\"a}r
l{\aa}g nog f{\"o}r att vi ej skall f{\aa} lyft fr{\aa}n underlaget med mera,
och d{\"a}refter integrera r{\"o}relseekvationerna fram till ett resultat, men
alla f{\"o}rst{\aa}r nog att en betydligt enklare approach (om vi inte s{\"o}ker
funktionen som beskriver hastigheten som funktion av tid) helt enkelt {\"a}r
att ist{\"a}llet konstatera att partikeln tappar i {\it potential}. Denna
f{\"o}rlust i potential kan via partikelns massa $m$ enkelt {\"o}vers{\"a}ttas
till en f{\"o}rlust i {\it potentiell energi}, vilken ist{\"a}llet adderas till
den {\it kinetiska energin}.

Ett annat syns{\"a}tt {\"a}r att se partikeln som att den befinner sig i en
skal{\"a}r mekanisk potential $\phi(z)=gz$, resulterande i ett
{\it gravitationsf{\"a}lt}
$$
  {\bf G}=-\nabla\phi(z)=-{\bf e}_z g,
  \qquad\Bigg(\quad\Leftrightarrow\qquad
  {\bf E}=-\nabla\phi(z)
  \quad\Bigg)
$$
i analogi med ett statiskt elektriskt f{\"a}lt ${\bf E}$. Kraften (N) p{\aa}
punktmassan $m$ (i analogi med kraften ${\bf F}=q{\bf E}$ p{\aa} en
punkt\-laddning $q$ i ett elektrostatiskt f{\"a}lt) blir
\sidx{Elektrostatiskt f{\"a}lt}[Analogi med gravitationsf{\"a}lt]
$$
  {\bf F}=m{\bf G}.
  \qquad\Bigg(\quad\Leftrightarrow\qquad
  {\bf F}=q{\bf E}
  \quad\Bigg)
$$
Om vi utg{\aa}r ifr{\aa}n planet $z=0$ som referens, s{\aa} blir den
potentiella {\it energin} (J) f{\"o}r partikeln p{\aa} h{\"o}jden $z$
d{\"a}rmed
\sidx{Potentiell energi}[Elektrostatisk]
$$
  W = -\int^z_{z=0}{\bf F}\cdot d{\bf x} = mgz
  \qquad\Bigg(\quad\Leftrightarrow\qquad
  W = -\int_{\Gamma}{\bf F}\cdot d{\bf x}
    = -q\int_{\Gamma}{\bf E}\cdot d{\bf x}
  \quad\Bigg)
$$
Denna aningen naivistiska analogi illustrerar hur vi rent elektrostatiskt,
f{\"o}r {\it statiska} elektriska laddningar, kan se p{\aa} den skal{\"a}ra
potentialen $\phi$ och hur vi kan anv{\"a}nda den i elektrostatiska problem
genom att vi kan extrahera det elektriska f{\"a}ltet genom
${\bf E}=-\nabla\phi$. Inom magnetism, som i grund och botten handlar om
laddningars {\it dynamik} (r{\"o}relse), har vi ist{\"a}llet vektorpotentialen
${\bf A}$, fr{\aa}n vilken vi ist{\"a}llet f{\aa}r den magnetiska
fl{\"o}dest{\"a}theten som ${\bf B}=\nabla\times{\bf A}$.
\vfill\eject

\section{Konservativa kontra icke-konservativa f{\"a}lt}
\sidx{Skal{\"a}r potential $\phi$}[Konservativa f{\"a}lt]
\sidx{Vektorpotential ${\bf A}$}[Icke-konservativa f{\"a}lt]
Vi ser hur begreppet potential {\"a}r synnerligen anv{\"a}ndbart i s{\aa} kallade {\it konservativa f{\"a}lt}, som exempelvis gravitation och elektrostatik. L{\aa}t oss med en handfull enkla fr{\aa}gor reda ut n{\aa}gra begrepp som kan vara aningen f{\"o}rvirrande i sammanhanget.

\subsection{Fr{\aa}ga 1: Vad {\"a}r ett konservativt f{\"a}lt?}
\sidx{Konservativt f{\"a}lt}
Ett {\it konservativt f{\"a}lt}\numberedfootnote{Eller om vi skall vara petnoga,
  {\it konservativt vektorf{\"a}lt}.}
{\"a}r ett f{\"a}lt d{\"a}r arbetet mellan tv{\aa} punkter inte beror p{\aa} v{\"a}gen, bara p{\aa} start- och slutpunkt. Rent matematiskt {\"a}r arbetet
$$
  W=-\int^{{\bf x}_b}_{{\bf x}_a}{\bf F}\cdot d{\bf l}
$$
i ett konservativt f{\"a}lt detsamma oavsett vilken v{\"a}g vi v{\"a}ljer att ta mellan
${\bf x}_a$ och ${\bf x}_b$. Vi kan h{\"a}r typiskt f{\"o}rest{\"a}lla oss att ${\bf F}=m{\bf G}$ (som i gravitation) eller ${\bf F}=q{\bf E}$ (som i elektrostatik). Lite l{\"o}st kan vi s{\"a}ga att ``allt som str{\aa}lar ut fr{\aa}n en punktk{\"a}lla'' (typ elektriska f{\"a}ltlinjer fr{\aa}n en statisk laddning) kommer att beskriva ett konservativt f{\"a}lt.

\subsection{Fr{\aa}ga 2: Vad k{\"a}nnetecknar ett konservativt f{\"a}lt?}
Definitionen enligt ovan {\"a}r f{\"o}rvisso tillr{\"a}cklig, men rent praktiskt vore det bra om vi kunde k{\"a}nna igen ett konservativt f{\"a}lt utifr{\aa}n sin form. Det absolut enklaste n{\"o}dv{\"a}ndiga och tillr{\"a}ckliga villkoret f{\"o}r att ett f{\"a}lt skall vara konservativt {\"a}r att dess rotation {\"a}r noll, exempelvis som f{\"o}r s{\aa}v{\"a}l elektrostatiska f{\"a}lt ${\bf E}$ med $\nabla\times{\bf E}={\bf 0}$ som f{\"o}r gravitationsf{\"a}ltet ${\bf G}$ med $\nabla\times{\bf G}={\bf 0}$.
Vi kan erinra oss att detta i grunden ju var sj{\"a}lva {\it anledningen} att vi i F{\"o}rel{\"a}sning~2 kunde utnyttja vektorteoremet $\nabla\times(\nabla f)={\bf 0}$ till att dra slutsatsen om existens av en skal{\"a}r potential\numberedfootnote{Rekapitulera att vi v{\"a}ljer att {\it definiera} denna potential med negativt tecken s{\aa} att vi f{\aa}r den elektrostatiska kraften ${\bf F}=q{\bf E}$ i samma riktning som f{\"a}ltet ${\bf E}$ f{\"o}r positiva laddningar $q$.} ${\bf E}=-\nabla\phi$.

\subsection{Fr{\aa}ga 3: Vad {\"a}r ett icke-konservativt f{\"a}lt?}
\sidx{Icke-konservativt f{\"a}lt}

\subsection{Fr{\aa}ga 4: Vad k{\"a}nnetecknar ett icke-konservativt f{\"a}lt?}


\subsection{Fr{\aa}ga 5: {\"A}r ett elektrostatiskt f{\"a}lt alltid konservativt?}

\subsection{Fr{\aa}ga 6: {\"A}r ett magnetostatiskt f{\"a}lt alltid konservativt?}

\subsection{Fr{\aa}ga 7: {\"A}r ett elektrodynamiskt f{\"a}lt alltid konservativt?}


\section{Finns det absoluta m{\aa}tt p{\aa} en potential?}
\sidx{Skal{\"a}r potential $\phi$}[Absolut m{\aa}tt f{\"o}r]
\sidx{Vektorpotential ${\bf A}$}[Absolut m{\aa}tt f{\"o}r]


\section{Retarderade potentialer och kopplingen till elektromagnetiska f{\"a}lt}
\sidx{Retarderad potential}[Tidsf{\"o}rdr{\"o}jd potential]
Gauss lag f{\"o}r magnetiska fl{\"o}dest{\"a}theten ${\bf B}({\bf x},t)$
g{\"a}ller alltid generellt, och ger direkt vid hand att {\"a}ven en
{\it tidsberoende} vektorpotential ${\bf A}({\bf x},t)$ har exakt samma l{\"a}nk
till ${\bf B}({\bf x},t)$ som tidigare, eftersom vi vid godtycklig
observationspunkt ${\bf x}$ och godtycklig tid $t$ har
att\numberedfootnote{V{\"a}n av ordning kan h{\"a}r fr{\aa}ga sig hur vi
  verkligen kan s{\"a}ga att $\nabla\cdot{\bf B}=0$ {\"a}ven f{\"o}r
  {\it dynamiska} f{\"a}lt. Trots allt s{\aa} h{\"a}rledde vi ju detta
  i F{\"o}rel{\"a}sning~4 utifr{\aa}n en {\it statisk} form av Biot--Savarts
  lag, och att bara s{\"a}ga att detta {\"a}ven g{\"a}ller {\it dynamiskt}
  {\"a}r ju lite fusk, eller hur? Det enklaste s{\"a}ttet att visa p{\aa}
  att $\nabla\cdot{\bf B}=0$ {\"a}ven i ett dynamiskt fall {\"a}r att
  konstatera att Faradays lag (som ju {\"a}r dynamisk och generellt giltig)
  ger vid hand att
  $$
    \nabla\times{\bf E}=-{{\partial{\bf B}}\over{\partial t}}
    \quad\Rightarrow\quad
    \underbrace{
      \nabla\cdot(\nabla\times{\bf E})
    }_{\nabla\cdot(\nabla\times{\bf a})\equiv0}
    =-\nabla\cdot\Big({{\partial{\bf B}}\over{\partial t}}\Big)
    =-{{\partial}\over{\partial t}}\nabla\cdot{\bf B}
    =0.
  $$
  Om vi integrerar den sista likheten med avseende p{\aa} tid, s{\aa} betyder
  detta kort och gott att $\nabla\cdot{\bf B}=\hbox{konstant}$ i tiden.
  Oberoende av vilken tid vid vilken vi startar denna identitet, s{\aa}
  m{\aa}ste d{\"a}rmed $\nabla\cdot{\bf B}$ evaluera till samma konstant,
  specifikt {\"a}ven d{\aa} vi passerar eller startar fr{\aa}n n{\aa}gon
  tid d{\aa} vi r{\aa}kar ha en {\it statisk} situation d{\aa} uppenbarligen
  $\nabla\cdot{\bf B}=0$. Med andra ord {\"a}r enda m{\"o}jligheten att
  integrationskonstanten i fr{\aa}ga {\"a}r noll, och att det {\"a}ven
  {\it dynamiskt} g{\"a}ller att $\nabla\cdot{\bf B}=0$.
  (Ut{\"o}ver att det vore synnerligen m{\"a}rklig fysik om magnetiska
  monopoler, om de skulle existera, skulle b{\"o}rja upptr{\"a}da enbart
  i dynamiska experiment.)\sidx{Magnetiska monopoler}[Icke-existens av]}
$$
  \nabla\cdot{\bf B}({\bf x},t)=0
  \qquad\Leftrightarrow\qquad
  {\bf B}({\bf x},t)=\nabla\times{\bf A}({\bf x},t).
$$
Om vi s{\"a}tter in vektorpotentialen ${\bf A}({\bf x},t)$ i Faradays
induktionslag,\numberedfootnote{Notera hur Faradays lag, som vi erinrar
  oss h{\"a}rleddes ``rent'' i F{\"o}rel{\"a}sning~5 {\it enbart
  utifr{\aa}n antagandet om Lorentz-kraften}, {\aa}terigen kommer
  till assistans!}
s{\aa} erh{\aa}ller vi
$$
  \nabla\times{\bf E}({\bf x},t)=
    -{{\partial}\over{\partial t}}
       \underbrace{\nabla\times{\bf A}({\bf x},t)}_{={\bf B}({\bf x},t)}
  \qquad\Leftrightarrow\qquad
  \nabla\times\underbrace{
      \bigg({\bf E}({\bf x},t)
        +{{\partial{\bf A}({\bf x},t)}\over{\partial t}}
      \bigg)
    }_{=-\nabla\phi({\bf x},t)}=0.
$$
Eftersom rotationen av argumentet {\"a}r noll,\numberedfootnote{Se Griffiths
  sammanfattade vektoridentiteter p{\aa} insidan av p{\"a}rmen,
  $\nabla\times(\nabla f)=0$.}
s{\aa} betyder detta att argumentet kan skrivas som gradienten av en skal{\"a}r
potential, s{\"a}g som\sidx{Skal{\"a}r potential $\phi$}[Elektrodynamisk]
\sidx{Vektorpotential ${\bf A}$}[Elektrodynamisk]
$$
  {\bf E}({\bf x},t)
    +{{\partial{\bf A}({\bf x},t)}\over{\partial t}}
  =-\nabla\phi({\bf x},t)
  \qquad\Leftrightarrow\qquad
  {\bf E}({\bf x},t)
    =-\nabla\phi({\bf x},t)-{{\partial{\bf A}({\bf x},t)}\over{\partial t}}.
$$
Valet av negativt tecken f{\"o}r gradienten i potentialen kommer fr{\aa}n
v{\aa}r konvention f{\"o}r krafter p{\aa} laddningar i elektriska f{\"a}lt,
och att positiva laddningar str{\"a}var mot minsta potential.
F{\"o}r att f{\"o}rtydliga vad vi h{\"a}r gjort, s{\aa} har vi {\it endast}
anv{\"a}nt Gauss lag f{\"o}r magnetiska f{\"a}lt samt Faradays induktionslag,
f{\"o}r vilka paret $({\bf E}({\bf x},t),{\bf B}({\bf x},t))$ {\"a}r oberoende
av materialegenskaperna. Med andra ord {\"a}r existensen av den skal{\"a}ra
potentialen $\phi({\bf x},t)$ och vektorpotentialen ${\bf A}({\bf x},t)$
{\it oberoende av det medium i vilket de analyseras}, {\"a}ven i det
elektrodynamiska (tidsberoende) fallet.

Vi kan h{\"a}r ocks{\aa} notera att f{\"a}lten evalueras p{\aa} exakt samma
punkt spatialt som vi uttrycker potentialerna i, med andra ord s{\aa} har vi
{\"a}nnu inte inf{\"o}rt n{\aa}got ``retarderat'' eller ``f{\"o}rdr{\"o}jt''
i ekvationerna. Detta kommer dock att inkluderas h{\"a}rn{\"a}st.
\vfill\eject

\section{V{\aa}gekvationen f{\"o}r retarderade potentialer}
\sidx{Elektromagnetisk v{\aa}gekvation}[f{\"o}r retarderade potentialer]
\subsection{Plan f{\"o}r hur vi angriper problemet med tidsf{\"o}rdr{\"o}jning
  hos potentialer}
S{\aa} l{\aa}ngt i kursen har vi analyserat f{\"a}lt och deras kopplingar till
varandra och externa k{\"a}llor som laddningsf{\"o}rdelningar och
str{\"o}mt{\"a}theter som fenomen som sker {\it instantant utan n{\aa}gon
f{\"o}rdr{\"o}jning i tid}. I m{\aa}nga situationer kan vi dock inte
f{\"o}rsumma den f{\"o}rdr{\"o}jning som sker, exempelvis om vi r{\aa}kar ha
en tidsvarierande laddningsf{\"o}rdelning, s{\"a}g p{\aa} en antenn, d{\"a}r
de olika delarna f{\"o}rdelningen har ett s{\aa} pass varierande avst{\aa}nd
som inte kan r{\"a}knas som litet i f{\"o}rh{\aa}llande till
observa\-tions\-punkten.
Vi har i F{\"o}rel{\"a}sningarna~2 och~ sett hur de {\it statiska} potentialerna,
som lyder \idx{Poissons ekvation}
$$
  \nabla^2\phi({\bf x})=-\rho({\bf x})/\varepsilon_0,\qquad
  \nabla^2{\bf A}=-\mu_0{\bf J}_{\rm f},
$$
direkt leder till explicita l{\"o}sningar p{\aa} integralform som\sidx{Poissons
ekvation}[L{\"o}sningar p{\aa} integralform]
$$
  \phi({\bf x})={{1}\over{4\pi\varepsilon_0}}\iiint_V
    {{\rho({\bf x}')}\over{|{\bf x}-{\bf x}'|}}\,dV',\qquad
  {\bf A}({\bf x})={{\mu_0}\over{4\pi}}\iiint_V
    {{{\bf J}_{\rm f}({\bf x}')}\over{|{\bf x}-{\bf x}'|}}\,dV',
$$
vilket vi erinrar oss {\"a}r en direkt f{\"o}ljd fr{\aa}n formen p{\aa} det
elektriska f{\"a}ltet fr{\aa}n den generaliserade formen av Gauss lag f{\"o}r
statisk laddning, samt formen p{\aa} magnetf{\"a}ltet enligt \idx{Amp\`eres lag}
(vilken i sin tur {\"a}r h{\"a}rledd fr{\aa}n \idx{Biot--Savarts lag}).
Vi kan med t{\"a}mligen stor sannolikhet konstatera att potentialerna, liksom den
elektromagnetiska v{\aa}g de beskriver, f{\"a}rdas med ljusets hastighet $c=c_0$
\sidx{Ljushastighet}[I dielektrikum, $c$] \sidx{Ljushastighet}[I vakuum, $c_0$]
(eller $c=c_0/n$, skalat med brytningsindex $n$ \sidx{Brytningsindex $n$} om vi
betraktar icke-vakuum), och det {\"a}r rimligt att anta att varje delbidrag i
integralformerna ovan d{\"a}rmed kommer att detekteras en viss tid
$|{\bf x}-{\bf x}'|/c$ efter den tid $t$ d{\aa} k{\"a}llan hade det
tillst{\aa}nd (laddning eller str{\"o}m) som en observat{\"o}r observerar
en str{\"a}cka $|{\bf x}-{\bf x}'|$ bort.

Med andra ord s{\aa} faller det sig helt naturligt att {\it gissa} att en
generaliserad form av dessa integralformer, som tar h{\"a}nsyn till den tid
$|{\bf x}-{\bf x}'|/c$ som potentialerna vid k{\"a}llpunkterna ${\bf x}'$ tar
p{\aa} sig f{\"o}r att n{\aa} fram till en observat{\"o}r vid punkten ${\bf x}$
skulle ges som
$$
  \phi({\bf x},t)={{1}\over{4\pi\varepsilon_0}}\iiint_{{\Bbb R}^3}
    {{\rho({\bf x}',t')}\over{|{\bf x}-{\bf x}'|}}\,dV',\qquad
  {\bf A}({\bf x},t)={{\mu_0}\over{4\pi}}\iiint_{{\Bbb R}^3}
    {{{\bf J}({\bf x}',t')}\over{|{\bf x}-{\bf x}'|}}\,dV',
$$
d{\"a}r
$$
  t' = t - |{\bf x}-{\bf x}'|/c
$$
{\"a}r den {\it retarderade tiden}, eller {\it f{\"o}rdr{\"o}jda tiden}, som
beror av avst{\aa}ndet $|{\bf x}-{\bf x}'|$ mellan k{\"a}llpunkt ${\bf x}'$ och
observationspunkt ${\bf x}$.\sidx{Retarderad tid}[Tidsf{\"o}rdr{\"o}jning]

Detta argument {\"a}r dock {\"a}n s{\aa} l{\"a}nge av t{\"a}mligen handviftande
karakt{\"a}r, och vi har i strikt mening inte {\"a}nnu formulerat hur de
{\it dynamiska} potentialerna propagerar. Vi kan f{\"o}rvisso ha en kvalificerad
gissning att potentialerna, ur vilka de elektriska och magnetiska f{\"a}lten
direkt kan erh{\aa}llas, m{\aa}ste f{\"o}lja exakt samma hastighet som det
elektromagnetiska f{\"a}ltet i sig (det vill s{\"a}ga ljushastigheten i mediet),
men det finns n{\aa}gra viktiga po{\"a}nger som {\"a}r knutna till potentialerna
i sig som vi i sin tur kan utnyttja.

Kort och gott, vi kommer nu att f{\"o}rst analysera hur v{\aa}gekvationen
f{\"o}r potentialerna i sig ser ut, och d{\"a}refter f{\"o}lja upp resultatet
med att formulera integralformen f{\"o}r de retarderade
potentialerna.\numberedfootnote{Spoiler: Den alldeles nyss presenterade
  gissningen f{\"o}r de retarderade potentialernas integralform {\"a}r
  helt korrekt!}
\vfill\eject

\subsection{Den generella formen av v{\aa}gekvationen f{\"o}r vektorpotentialen}
Om vi substituerar f{\"o}r den skal{\"a}ra potentialen $\phi({\bf x},t)$ och
vektorpotentialen ${\bf A}({\bf x},t)$ i Gauss och Amp\`eres
lagar\numberedfootnote{Notera att Gauss och Amp\`eres lagar som involverar
    mediets egenskaper (eftersom vi h{\"a}r betraktar ${\bf E}$ och ${\bf B}$
    som v{\aa}ra elektrodynamiska ``basf{\"a}lt'') s{\aa} l{\aa}ngt ej {\"a}nnu
    anv{\"a}nts; detta {\"a}r punkten d{\aa} vi f{\"o}rst introducerar
    mediets egenskaper.}
s{\aa} f{\aa}r vi, under antagandet om en linj{\"a}r elektrisk
fl{\"o}dest{\"a}thet i homogent medium,
$$
  {\bf D}({\bf x},t)=\varepsilon_0\varepsilon_{\rm r}{\bf E}({\bf x},t),
$$
fr{\aa}n Gauss lag $\nabla\cdot{\bf D}({\bf x},t)=\rho({\bf x},t)$ f{\"o}r den
elektriska fl{\"o}dest{\"a}theten att\sidx{Gauss lag}[F{\"o}r elektrisk fl{\"o}dest{\"a}thet
${\bf D}$]
$$
  \nabla\cdot{\bf D}({\bf x},t)
    =-\varepsilon_0\varepsilon_{\rm r}\nabla\cdot
      \underbrace{
        \bigg(
          \nabla\phi({\bf x},t)+{{\partial{\bf A}({\bf x},t)}\over{\partial t}}
        \bigg)
      }_{\equiv-{\bf E}({\bf x},t)}
    =\rho({\bf x},t)
  \quad\Leftrightarrow\quad
    \nabla^2\phi({\bf x},t)
      +{{\partial}\over{\partial t}}\nabla\cdot{\bf A}({\bf x},t)
    =-{{\rho({\bf x},t)}\over{\varepsilon_0\varepsilon_{\rm r}}}
$$
vilket vi {\aa}tminstone fr{\aa}n f{\"o}rekomsten av ``$\nabla^2$'' kan
b{\"o}rja gissa oss till kommer att handla om en v{\aa}gekvation.
Det som saknas h{\"a}r {\"a}r hur vi skall tolka tidsderivatan av
$\nabla\cdot{\bf A}({\bf x},t)$.

Fr{\aa}n \idx{Amp\`eres lag} i dess grundform (notera att detta {\"a}r punkten
d{\aa} vi i princip f{\"o}r in kopplingen med magnetiska egenskaper i problemet
via de konstitutiva relationerna \sidx{Konstitutiva relationer} f{\"o}r
${\bf H}$ och ${\bf D}$),
$$
  \nabla\times{\bf H}={\bf J}_{\rm f}+{{\partial{\bf D}}\over{\partial t}},
$$
f{\"o}r enkelhets skull under antagandet om ett ickemagnetiskt medium med
${\bf B}({\bf x},t)=\mu_0{\bf H}({\bf x},t)$, har vi samtidigt att
$$
  \eqalign{
    \nabla\times{\bf B}({\bf x},t)
      &=%\underbrace{
         \nabla\times
           \underbrace{
             \nabla\times{\bf A}({\bf x},t)
           }_{={\bf B}({\bf x},t)}
        %}_{=\nabla(\nabla\cdot{\bf A})-\nabla^2{\bf A}}
        \cr
     &=\nabla(\nabla\cdot{\bf A})-\nabla^2{\bf A}\cr
     &=\big\{\hbox{ Amp\`eres lag,
                    $\nabla\times{\bf B}=\mu_0\nabla\times{\bf H}$ }\big\}\cr
     &=\mu_0{\bf J}_{\rm f}({\bf x},t)
      +\mu_0{{\partial{\bf D}({\bf x},t)}\over{\partial t}}\cr
     &=\mu_0{\bf J}_{\rm f}({\bf x},t)
      +\underbrace{\mu_0\varepsilon_0\varepsilon_{\rm r}}_{\equiv 1/c^2}
      {{\partial{\bf E}({\bf x},t)}\over{\partial t}}\cr
     &=\Big\{\ \hbox{Uttryck i vektorpotentialer,}
         \ {\bf E}=-\nabla\phi-{{\partial{\bf A}}\over{\partial t}}\ \Big\}\cr
     &=\mu_0{\bf J}_{\rm f}({\bf x},t)
      -{{1}\over{c^2}}
        {{\partial}\over{\partial t}}
        \bigg(
          \nabla\phi({\bf x},t)+{{\partial{\bf A}({\bf x},t)}\over{\partial t}}
        \bigg).\cr
  }
$$
Om vi stuvar om termerna lite, s{\aa} har vi under utnyttjandet av $\nabla
\times\nabla\times{\bf A}\equiv\nabla(\nabla\cdot{\bf A})-\nabla^2{\bf A}$
att\numberedfootnote{Recap fr{\aa}n F{\"o}rel{\"a}sning~10, d{\"a}r den
  homogena v{\aa}gekvationen i tre dimensioner (``d'Alemberts v{\aa}gekvation'')
  skrevs p{\aa} formen
  $$
    \Big(\nabla^2-{{1}\over{c^2}}{{\partial^2}\over{\partial t^2}}\Big)
      f({\bf x},t)=0.
  $$}
$$
  \underbrace{
  \bigg(
    \nabla^2-{{1}\over{c^2}}
      {{\partial^2}\over{\partial t^2}}
  \bigg)
  {\bf A}({\bf x},t)
  }_{\hbox{Typisk v{\aa}gekvation!}}
  -\underbrace{
    \nabla\bigg(
    \nabla\cdot{\bf A}({\bf x},t)+
    {{1}\over{c^2}}{{\partial\phi({\bf x},t)}\over{\partial t}}
  \bigg)}_{\hbox{Fr{\aa}ga: ``Hur bli av med denna?''}}
    =-\mu_0{\bf J}_{\rm f}({\bf x},t)
$$
\vfill\eject

\subsection{Den generella formen av v{\aa}gekvationen f{\"o}r skal{\"a}ra
  potentialen}
\sidx{Skal{\"a}r potential $\phi$}[V{\aa}gekvation f{\"o}r]
Embryot till en v{\aa}gekvation f{\"o}r den skal{\"a}ra potentialen $\phi$ kan
tas fram p{\aa} ett liknande s{\"a}tt, om {\"a}n betydligt enklare {\"a}n
f{\"o}r vektorpotentialen, genom att utveckla divergensen f{\"o}r den
{\it elektriska fl{\"o}dest{\"a}theten} ${\bf D}=\varepsilon_0\varepsilon_{\rm r}
{\bf E}$ under i {\"o}vrigt samma f{\"o}ruts{\"a}ttningar som f{\"o}r
vektorpotentialen, som
$$
  \nabla\cdot{\bf D}
    =\varepsilon_0\varepsilon_{\rm r}\nabla\cdot{\bf E}
    =\varepsilon_0\varepsilon_{\rm r}\nabla\cdot
        \Big(-\nabla\phi-{{\partial{\bf A}}\over{\partial t}}\Big)
    =\rho
  \qquad\Leftrightarrow\qquad
    \nabla^2\phi+{{\partial}\over{\partial t}}\nabla\cdot{\bf A}
    =-{{\rho}\over{\varepsilon_0\varepsilon_{\rm r}}}.
$$
Det som f{\"o}r vektorpotentialen tog ett antal rader i anspr{\aa}k f{\"o}r
h{\"a}rledning kunde f{\"o}r den skal{\"a}ra potentialen g{\"o}ras med en
{\it one-liner}, med ``$\nabla^2\phi$'' som {\aa}tminstone en bra b{\"o}rjan
p{\aa} en ``halv v{\aa}gekvation''.
Fr{\aa}gan h{\"a}r blir ist{\"a}llet hur termen $({{\partial}/{\partial t}})
\nabla\cdot{\bf A}$ skall tolkas.
Som vi strax kommer att se, s{\aa} kommer denna fr{\aa}ga att praktiskt taget
trivialt f{\aa} en l{\"o}sning s{\aa} snart som den mer trixiga termen f{\"o}r
vektorpotentialen eliminerats.

\subsection{Gauge-transformen}
\sidx{Gauge-transform}
Om vi skall sammanfatta detta halvv{\"a}gs, s{\aa} har vi i och med detta
reducerat Maxwells fyra ekvationer till tv{\aa}, en f{\"o}r den skal{\"a}ra
potentialen $\phi$ och en f{\"o}r vektorpotentialen ${\bf A}$, vilka dock
fortfarande {\"a}r kopplade. Att de fortfarande {\"a}r kopplade g{\"o}r det
sv{\aa}rt f{\"o}r oss att formulera hur dessa potentialer skall extraheras
fr{\aa}n k{\"a}nda variabler s{\aa} som laddningsf{\"o}rdelningar och
str{\"o}mt{\"a}theter.
Is{\"a}rkopplingen av dessa tv{\aa} ekvationer f{\"o}r potentialerna kan dock
utf{\"o}ras genom att utnyttja en viss grad av godtycklighet som fortfarande
finns inneboende i ekvationerna, som vi nu skall visa.

Vi noterar att eftersom vektorpotentialen ${\bf B}({\bf x},t)$ definieras
utifr{\aa}n rotationen av vektorpotentialen som\sidx{Vektorpotential ${\bf A}$}
$$
  {\bf B}({\bf x},t)=\nabla\times{\bf A}({\bf x},t),
$$
och eftersom vi alltid har vektoridentiteten $\nabla\times\nabla f\equiv 0$,
s{\aa} {\"a}r vektorpotentialen godtycklig i den bem{\"a}rkelsen att det
magnetiska f{\"a}ltet alltid l{\"a}mnas invariant d{\aa} vi adderar en
gradient av n{\aa}gon skal{\"a}r funktion, s{\"a}g $\psi({\bf x},t)$, till den,
$$
  {\bf A}({\bf x},t)
  \quad\to\quad
  {\bf A}'({\bf x},t)={\bf A}({\bf x},t)+\nabla\psi({\bf x},t),
$$
d{\"a}r $\psi({\bf x},t)$ {\"a}r en {\it godtycklig tv{\aa} g{\aa}nger
kontinuerligt differentierbar funktion i rum och tid},\numberedfootnote{En
  funktion {\"a}r tv{\aa} g{\aa}nger kontinuerligt differentierbar om den
  uppfyller att dess f{\"o}rsta- och andraderivator existerar, samt att
  b{\aa}de f{\"o}rsta- och andraderivatorna {\"a}r kontinuerliga.
  Fr{\aa}gan {\"a}r d{\aa} naturligtvis: Varf{\"o}r st{\"a}ller vi detta
  specifika krav p{\aa} {\it gauge-funktionen} $\psi$?
  Det enkla svaret p{\aa} denna fr{\aa}ga {\"a}r att vi st{\"a}ller kravet
  att det elektromagnetiska f{\"a}ltet {\"a}r {\"o}verallt kontinuerligt
  d{\"a}r potentiall{\"o}sningarna tas fram, i omgivningar fria fr{\aa}n
  diskontinuiteter, och att detta f{\"a}lt utg{\"o}rs av f{\"o}rstaderivator
  av potentialerna.
  Om vi dessutom i fallet med vektorpotentialen ${\bf A}$ {\it adderar en
  gradient} $\nabla\psi$ dessutom kommer att st{\"a}lla kravet p{\aa} $\psi$
  att den {\"a}r just en tv{\aa} g{\aa}nger kontinuerligt differentierbar
  funktion.
  Detta enkla svar haltar dock en aning, speciellt d{\aa} detta inte
  f{\"o}rklarar varf{\"o}r $\psi$ dessutom m{\aa}ste vara tv{\aa} g{\aa}nger
  kontinuerligt differentierbar {\"a}ven i tid. Vi kommer strax att se ett
  b{\"a}ttre sk{\"a}l till varf{\"o}r vi kr{\"a}ver av $\psi$ att den {\"a}r
  tv{\aa} g{\aa}nger kontinuerligt differentierbar s{\aa}v{\"a}l i rum som
  i tid.}
beroende av rums-koordinater och tid. Med andra ord har vi full frihet att
addera en gradient av en l{\"a}mpligt vald skal{\"a}r funktion $\psi$ till
vektorpotentialen och {\it fortfarande ha exakt samma magnetiska f{\"a}lt
kopplat till denna}.
\vfill\eject

F{\"o}r att det {\it elektriska f{\"a}ltet} skall vara of{\"o}r{\"a}ndrat av
denna transformation, det vill s{\"a}ga att
$$
  {\bf E}({\bf x},t)
    =-\nabla\phi({\bf x},t)-{{\partial}\over{\partial t}}
        \big[
          \underbrace{
            {\bf A}'({\bf x},t)-\nabla\psi({\bf x},t)
          }_{={\bf A}({\bf x},t)}
        \big]
    =-\nabla\bigg(
          \underbrace{
            \phi({\bf x},t)
              -{{\partial\psi({\bf x},t)}\over{\partial t}}
          }_{=\phi'({\bf x},t)}
         \bigg)
    -{{\partial{\bf A}'({\bf x},t)}\over{\partial t}}
$$
skall f{\"o}rbli of{\"o}r{\"a}ndrad, s{\aa} {\"a}r kravet att den skal{\"a}ra
potentialen {\it samtidigt} transformeras som
$$
  \phi({\bf x},t)
  \quad\to\quad
  \phi'({\bf x},t)=\phi({\bf x},t)-{{\partial\psi({\bf x},t)}\over{\partial t}}.
$$
Sammanfattningsvis, s{\aa} har vi resultatet att den parvisa transformationen
$$
  \eqalign{
    {\bf A}'({\bf x},t)
      &={\bf A}({\bf x},t)+\nabla\psi({\bf x},t),\cr
    \phi'({\bf x},t)
      &=\phi({\bf x},t)-{{\partial\psi({\bf x},t)}\over{\partial t}},\cr
  }
$$\sidx{Gauge-transformation}
som kallas f{\"o}r {\it gauge-transformation} och anv{\"a}nds inte bara inom
klassisk elektrodynamik, utan {\"a}ven ofta inom kvantmekanik, l{\"a}mnar de
elektriska och magnetiska f{\"a}lten\numberedfootnote{Om man skall vara petig,
  den {\it elektriska f{\"a}ltstyrkan} respektive den {\it magnetiska
  fl{\"o}dest{\"a}theten}.}
of{\"o}r{\"a}ndrade,\numberedfootnote{Som en liten illustrativ verifiering av
  detta grundl{\"a}ggande resultat, s{\aa} {\"a}r allts{\aa}
  $$
    {\bf E}=-\nabla\phi'-{{\partial{\bf A}'}\over{\partial t}}
      =-\nabla\Big(\phi-{{\partial\psi}\over{\partial t}}\Big)
         -{{\partial}\over{\partial t}}\Big({\bf A}+\nabla\psi\Big)
      =-\nabla\phi-{{\partial{\bf A}}\over{\partial t}}.
$$}
$$
  \eqalign{
    {\bf E}({\bf x},t)
      &\equiv-\nabla\phi'({\bf x},t)
          -{{\partial{\bf A}'({\bf x},t)}\over{\partial t}},\qquad
          \bigg(\ =-\nabla\phi({\bf x},t)
             -{{\partial{\bf A}({\bf x},t)}\over{\partial t}}\ \bigg)\cr
    {\bf B}({\bf x},t)
      &\equiv\nabla\times{\bf A}'({\bf x},t).\hskip68.5pt
      \bigg(\ =\nabla\times{\bf A}({\bf x},t)\ \bigg)\cr
  }
$$
Vi kommer nu att utnyttja denna m{\"o}jlighet till {\it aktivt val} av
gaugefunktionen $\psi$ f{\"o}r att g{\"o}ra oss av med det som s{\aa} att
s{\"a}ga ``inte passar in'' i v{\aa}gekvationen f{\"o}r vektorpotentialen.
\vfill\eject

\subsection{Lorenz-villkoret (``Lorenz gauge'')}
\sidx{Lorenz-villkoret}[{{\it Lorenz gauge}}]
\sidx{Gauge-funktion $\psi$}
Om vi nu {\aa}terv{\"a}nder till grundproblemet med ekvationen f{\"o}r
vektorpotentialen ovan, s{\aa} kan vi se att vi med de {\it gauge-transformerade}
potentialerna $\phi'({\bf x},t)$ och ${\bf A}'({\bf x},t)$ har
att\numberedfootnote{Notera hur den slutliga termen som inneh{\aa}ller
  enbart gauge-funktionen $\psi$ {\"a}ven den har formen av {\"a}nnu en
  v{\aa}gfunktion! Man kan p{\aa} s{\"a}tt och vis se detta som en
  i tv{\aa} led inb{\"a}ddad v{\aa}gfunktion fr{\aa}n det elektromagnetiska
  f{\"a}ltet till potentialerna till gauge-funktionen $\psi$.}
$$
  \eqalign{
      \underbrace{
      \nabla\cdot{\bf A}'({\bf x},t)+
      {{1}\over{c^2}}{{\partial\phi'({\bf x},t)}\over{\partial t}}
      }_{\hbox{``Det vi f{\"o}rs{\"o}ker bli av med''}}
    &=\nabla\cdot\big({\bf A}({\bf x},t)+\nabla\psi({\bf x},t)\big)+
      {{1}\over{c^2}}{{\partial}\over{\partial t}}
      \bigg(
        \phi({\bf x},t)-{{\partial\psi({\bf x},t)}\over{\partial t}}
      \bigg)\cr
    &=\{\ \hbox{Stuva om termer}\ \}\cr
    &=\underbrace{
      \nabla\cdot{\bf A}({\bf x},t)
        +{{1}\over{c^2}}{{\partial\phi({\bf x},t)}\over{\partial t}}
      +\underbrace{
         \bigg(
           \nabla^2\psi({\bf x},t)
             -{{1}\over{c^2}}{{\partial^2\psi({\bf x},t)}\over{\partial t^2}}
         \bigg)
       }_{\hbox{Frihet att v{\"a}lja $\psi$}}
      }_{\vbox{
          \hbox{Vi har frihet att v{\"a}lja $\psi$ s{\aa}}
          \hbox{att h{\"o}gerledet blir noll!}
          }}.\cr
  }
$$
Eftersom funktionen\numberedfootnote{Enheten f{\"o}r $\psi({\bf x},t)$ {\"a}r
  ${\rm V}\cdot{\rm s}$.}
$\psi({\bf x},t)$ {\"a}r {\it godtycklig} (och vi ser nu dessutom i formen ovan
varf{\"o}r vi kr{\"a}ver att funktionen {\"a}r just {\it tv{\aa}} g{\aa}nger
differentierbar b{\aa}de i rum och tid), s{\aa} inneb{\"a}r detta specifikt att
vi kan v{\"a}lja den s{\aa} att
$$
  \nabla\cdot{\bf A}({\bf x},t)
    +{{1}\over{c^2}}{{\partial\phi({\bf x},t)}\over{\partial t}}
   +\bigg(
      \nabla^2\psi({\bf x},t)
        -{{1}\over{c^2}}{{\partial^2\psi({\bf x},t)}\over{\partial t^2}}
    \bigg)
    =0
$$
vilket i sin tur betyder att ``det vi f{\"o}rs{\"o}ker bli av med'' blir noll,
$$
  \nabla\cdot{\bf A}'({\bf x},t)+
    {{1}\over{c^2}}
    {{\partial\phi'({\bf x},t)}\over{\partial t}} = 0.
$$
Denna relation mellan vektorpotential och skal{\"a}r potential kallas
allm{\"a}nt {\it Lorenz-villkoret}, och inneb{\"a}r ut{\"o}ver att eliminera
``det vi f{\"o}rs{\"o}ker bli av med'' fr{\aa}n ekvationen f{\"o}r
vektorpotentialen {\"a}ven att ekvationen f{\"o}r den skal{\"a}ra potentialen
under Lorenz-villkoret frikopplas fr{\aa}n vektorpotentialen, d{\aa}
$$
  \nabla\cdot{\bf A}'=-{{1}\over{c^2}}{{\partial\phi'}\over{\partial t}}
  \quad\Rightarrow\quad
  \nabla^2\phi'+
    {{\partial}\over{\partial t}}\nabla\cdot{\bf A}'
  =\nabla^2\phi'-
      {{1}\over{c^2}}{{\partial^2\phi'}\over{\partial t^2}}
    =-{{\rho}\over{\varepsilon_0\varepsilon_{\rm r}}}.
$$
Den frihet som gauge-transformen av den skal{\"a}ra potentialen och
vektorpotentialen bist{\aa}r med inneb{\"a}r allts{\aa} att vi specifikt har
{\it friheten att v{\"a}lja potentialer $\phi({\bf x},t)$ och
${\bf A}({\bf x},t)$ s{\aa} att ekvationerna f{\"o}r dem frikopplas},
resulterande i tv{\aa} inhomogena partiella differentialekvationer
$$
  \eqalign{
    \nabla^2\phi({\bf x},t)-{{1}\over{c^2}}
      {{\partial^2\phi({\bf x},t)}\over{\partial t^2}}
      &=-{{\rho({\bf x},t)}\over{\varepsilon_0\varepsilon_{\rm r}}},\cr
    \nabla^2{\bf A}({\bf x},t)-{{1}\over{c^2}}
      {{\partial^2{\bf A}({\bf x},t)}\over{\partial t^2}}
      &=-\mu_0{\bf J}_{\rm f}({\bf x},t),\cr
  }
$$
d{\"a}r vi tog oss friheten att ``droppa primmen'' p{\aa} $\phi'$ och ${\bf A}'$.
Det {\"a}r forts{\"a}ttningsvis underf{\"o}rst{\aa}tt att v{\aa}gekvationerna
f{\"o}r potentialerna $\phi$ och ${\bf A}$ i sig inneb{\"a}r att vi redan
best{\"a}mt oss f{\"o}r en viss {\it gauge} och att detta {\"a}r inkluderat i
variablerna, vilket vi erinrar oss i alla h{\"a}ndelser ej kommer att
p{\aa}verka l{\"o}sningarna f{\"o}r det elektromagnetiska f{\"a}ltet.

\subsection{Tre viktiga observationer}
Fr{\aa}n dessa tv{\aa} {\it v{\aa}gekvationer} ser vi direkt tv{\aa} saker:
\medskip
\litem[1.]{Potentialerna propagerar enligt dessa v{\aa}gekvationer med
    {\it ljusets hastighet}\numberedfootnote{Vi erinrar
      oss att $c=c_0/n$ {\"a}r ljushastigheten i mediet med brytningsindex
      $n=\sqrt{\varepsilon_{\rm r}}$, och d{\"a}rmed inte begr{\"a}nsar
      slutsatsen om potentialernas utbredningshastighet till vakuum.}
    $c$.}
\litem[2.]{L{\"o}sningarna till v{\aa}gekvationerna f{\"o}r potentialerna
    kan f{\aa}s fram exakt p{\aa} analogt s{\"a}tt som f{\"o}r det
    elektrostatiska fallet, s{\aa} l{\"a}nge som vi {\"a}r noga med att ta
    i beaktande {\it tidsf{\"o}r\-dr{\"o}j\-ningen
    $\Delta t=|{\bf x}-{\bf x}'|/c$ fr{\aa}n k{\"a}lla till
    observationspunkt}, resulterande i s{\aa} kallade {\it retarderade
    potentialer}.\sidx{Retarderad potential}[Tidsf{\"o}rdr{\"o}jd potential]}
\litem[3.]{I fall d{\aa} vi har att g{\"o}ra med tidsoberoende, statiska
    problem s{\aa} f{\"o}rsvinner sj{\"a}lvfallet tidsderivatorna och vi
    erh{\aa}ller {\it exakt samma form} p{\aa} Poissons ekvationer f{\"o}r
    den {\it elektro\-statiska} skal{\"a}ra potentialen (som vi behandlade
    under F{\"o}rel{\"a}sning~2 och~3) och den {\it magneto\-statiska}
    vektorpotentialen (som vi behandlade under F{\"o}rel{\"a}sning~4),
    $$
      \eqalign{
        \nabla^2\phi({\bf x},t)
          &=-{{\rho({\bf x},t)}\over{\varepsilon_0\varepsilon_{\rm r}}},\cr
        \nabla^2{\bf A}({\bf x},t)
          &=-\mu_0{\bf J}_{\rm f}({\bf x},t).\cr
      }
    $$
    Notera att ljushastigheten $c$ i och med denna {\aa}terg{\aa}ng,
    fr{\aa}n {\it dynamik} till {\it statik}, automagiskt f{\"o}rsvinner
    fr{\aa}n ekvationerna!}
\medskip
\noindent
Det b{\"o}r h{\"a}r understrykas att {\it utbredningshastigheten f{\"o}r potentialerna {\"a}r beroende av vilken gauge vi v{\"a}ljer att behandla dem i}, och att vi i Coulomb-gauge ist{\"a}llet betraktar potentialerna som instantana (samtidiga) {\"o}verallt!

Den som t{\"a}nker n{\aa}gorlunda kritiskt b{\"o}r h{\"a}r st{\"a}lla sig fr{\aa}gan: Hur i hela friden h{\"a}nger detta ihop? Vi kan v{\"a}l inte godtyckligt, bara s{\aa}d{\"a}r, v{\"a}lja utbredningshastigheten f{\"o}r elektromagnetiska f{\"a}lt genom att v{\"a}lja n{\aa}gon l{\"a}mplig gauge-funktion $\psi$? En matematisk konstruktion kan ju inte g{\"o}ra v{\aa}ld p{\aa} fysiken i sig!

Gaugeteori {\"a}r onekligen ett av de mer sv{\aa}rgreppbara koncepten inom elektrodynamik s{\aa} snart vi b{\"o}rjar granska fundamenta f{\"o}r det elektromagnetiska f{\"a}ltet. Som 




\section{[{\"O}verkurs] Gauge-transformen: Lorenz och Coulomb gauge}
\subsection{Lorenz gauge}
\sidx{Lorenz-villkoret}[{{\it Lorenz gauge}}]
Gauge-transformationen ovan, vilken s{\"a}gs fixera potentialerna i den s{\aa}
kallade {\it Lorenz gauge},\numberedfootnote{Om Lorenz gauge (Wikipedia):
    ``It is unique among the constraint gauges in retaining manifest Lorentz
    invariance. Note, however, that this gauge was originally named after the
    Danish physicist Ludvig Lorenz and not after Hendrik Lorentz; it is often
    misspelled ``Lorentz gauge''. (Neither was the first to use it in
    calculations; it was introduced in 1888 by George F.~FitzGerald.)''}
s{\"a}ger egentligen bara att s{\aa} l{\"a}nge som vi utnyttjar valfriheten att
v{\"a}lja den (tv{\aa} g{\aa}nger differentierbara i rum och tid) funktionen
$\psi({\bf x},t)$ s{\aa} att den uppfyller
$$
  \nabla^2\psi({\bf x},t)
    -{{1}\over{c^2}}{{\partial^2\psi({\bf x},t)}\over{\partial t^2}}
  =-\bigg(
      \nabla\cdot{\bf A}({\bf x},t)
        +{{1}\over{c^2}}{{\partial\phi({\bf x},t)}\over{\partial t}}
     \bigg),
$$
s{\aa} kommer ekvationerna f{\"o}r $\phi({\bf x},t)$ och ${\bf A}({\bf x},t)$
att frikopplas till tv{\aa} (i allm{\"a}nhet inhomogena) v{\aa}gekvationer.
Lorenz-villkoret {\"a}r vanligast f{\"o}rekommande n{\"a}r man arbetar med
retarderade potentialer, eftersom vi d{\aa} f{\aa}r frikopplade ekvationer
f{\"o}r den skal{\"a}ra potentialen och
vektorpotentialen.\numberedfootnote{F{\"o}r en intressant utl{\"a}ggning kring
  historiken och utvecklingen av teorin bakom gauge-trans\-forma\-tionen,
  inklusive den felaktiga termen {\it Lorentz gauge} (ist{\"a}llet f{\"o}r
  det korrekta {\it Lorenz gauge}), se till exempel J.~D. Jackson,
  {\it Historical roots of gauge invariance}, Rev. Mod. Phys. {\bf 73},
  663 (2001);
  {\tt https://journals.aps.org/rmp/abstract/10.1103/RevModPhys.73.663}.}

\subsection{Coulomb gauge}
\sidx{Coulomb-villkoret}[{{\it Coulomb gauge}}]
Den andra ofta f{\"o}rekommande varianten {\"a}r {\it Coulomb-villkoret}
f{\"o}r gauge-transformen, i vilket vi fixerar potentialerna i det s{\aa}
kallade {\it Coulomb gauge}. I detta fall v{\"a}ljer vi $\psi({\bf x},t)$
s{\aa} att
$$
  \nabla\cdot{\bf A}({\bf x},t)=0.
$$
Observera att $\nabla\cdot{\bf A}({\bf x},t)=0$ sj{\"a}lvfallet {\it inte}
p{\aa} n{\aa}got s{\"a}tt inneb{\"a}r att $\nabla\times{\bf A}({\bf x},t)$
(som ju {\"a}r detsamma som magnetiska fl{\"o}dest{\"a}theten
${\bf B}({\bf x},t)$) n{\"o}dv{\"a}ndigtvis {\"a}r noll.

I detta fall blir v{\aa}rt krav p{\aa} $\psi({\bf x},t)$ att funktionen
ist{\"a}llet uppfyller
$$
  \nabla^2\psi({\bf x},t)
    -{{1}\over{c^2}}{{\partial^2\psi({\bf x},t)}\over{\partial t^2}}
  =-{{1}\over{c^2}}{{\partial\phi({\bf x},t)}\over{\partial t}},
$$
och vi ser att de partiella differentialekvationerna f{\"o}r potentialerna
ist{\"a}llet antar formen
$$
  \eqalign{
    \nabla^2\phi({\bf x},t)
    &=-{{\rho({\bf x},t)}\over{\varepsilon_0\varepsilon_{\rm r}}},\cr
    \bigg(
      \nabla^2-{{1}\over{c^2}}{{\partial^2}\over{\partial t^2}}
    \bigg){\bf A}({\bf x},t)
    &=-\mu_0{\bf J}_{\rm f}({\bf x},t)
      +{{1}\over{c^2}}\nabla{{\partial\phi({\bf x},t)}\over{\partial t}},\cr
  }
$$
det vill s{\"a}ga att den skal{\"a}ra potentialen $\phi({\bf x},t)$ nu
(t{\"a}mligen ov{\"a}ntat) uppfyller den vanliga {\it statiska}
Poisson-ekvationen (notera att vi i denna ekvation saknar tidsderivata, och
att h{\"o}ger- och v{\"a}nsterled {\"a}r direkt kopplade utan n{\aa}gon
tidsf{\"o}rdr{\"o}jning mellan k{\"a}lla och observationspunkt), med
{\it omedelbar} verkan fr{\aa}n en laddningsf{\"o}rdelning $\rho({\bf x},t)$,
med l{\"o}sningen
$$
  \phi({\bf x},t)={{1}\over{4\pi\varepsilon_0\varepsilon_{\rm r}}}
      \iiint_{{\Bbb R}^3}{{\rho({\bf x},t)}\over{|{\bf x}-{\bf x}'|}}\,dV'.
$$
Ett annat s{\"a}tt att se p{\aa} detta {\"a}r att vi i denna gauge har den
skal{\"a}ra potentialen som en {\it direkt och omedelbar} Coulomb-potential
fr{\aa}n laddningst{\"a}theten $\rho({\bf x},t)$, d{\"a}rav att detta betecknas
som {\it Coulomb gauge}. I denna gauge l{\"o}ser vi principiellt f{\"o}rst
f{\"o}r den skal{\"a}ra potentialen (och ignorerar det faktum att potentialer
likt de elektromagnetiska f{\"a}lten i verkligheten sj{\"a}lvfallet propagerar
med ljusets hastighet), varvid vi anv{\"a}nder l{\"o}sningen $\phi({\bf x},t)$
som en k{\"a}llterm i den inhomogena v{\aa}gekvationen f{\"o}r
vektorpotentialen ${\bf A}({\bf x},t)$.

Coulomb-villkoret (Coulomb gauge) anv{\"a}nds ofta f{\"o}r f{\"a}ltproblem
d{\aa} vi har avsaknad av fria laddningar eller str{\"o}mmar. I detta fall
{\"a}r den skal{\"a}ra potentialen $\phi({\bf x},t)=0$, och vektorpotentialen
uppfyller d{\aa} den homogena v{\aa}gekvationen
$$
  \bigg(
    \nabla^2-{{1}\over{c^2}}{{\partial^2}\over{\partial t^2}}
  \bigg){\bf A}({\bf x},t)=0.
$$
Det elektriska f{\"a}ltet ${\bf E}({\bf x},t)$ och magnetiska
fl{\"o}dest{\"a}theten ${\bf B}({\bf x},t)$ f{\aa}s d{\aa} som
$$
  {\bf E}({\bf x},t)=-{{\partial{\bf A}({\bf x},t)}\over{\partial t}},
  \qquad\qquad
  {\bf B}({\bf x},t)=\nabla\times{\bf A}({\bf x},t).
$$
Eftersom vi uppenbarligen har en ofysikalisk situation i det att den
skal{\"a}ra potentialen i Coulomb gauge {\it omedelbart} f{\aa}r en effekt
p{\aa} en observationspunkt p{\aa} ett avst{\aa}nd fr{\aa}n k{\"a}llan (och
d{\"a}rmed bryter mot den grundl{\"a}ggande fysikaliska pricipen att ingenting
kan f{\"a}rdas fortare {\"a}n ljusets hastighet), s{\aa} infinner sig
fr{\aa}gan om detta verkligen kan vara korrekt?

H{\"a}r m{\aa}ste vi h{\aa}lla is{\"a}r begreppet potential och f{\"a}lt, och
f{\"o}rst av allt konstatera att vektorpotentialen ${\bf A}({\bf x},t)$, som i
Coulomb gauge bist{\aa}r med b{\aa}de det elektriska och magnetiska f{\"a}ltet,
f{\"o}ljer en v{\aa}gekvation som propagerar l{\"o}sningen med ljushastigheten.
I denna v{\aa}gekvation ing{\aa}r den omedelbara skal{\"a}ra potentialen som en
slags {\it artificiell k{\"a}llterm}, fr{\aa}n vilken vi aldrig direkt
extraherar n{\aa}gra f{\"a}lt. Med andra ord, s{\aa} {\"a}r Coulomb gauge
anv{\"a}ndbar i fr{\aa}nvaro av fria laddningar och str{\"o}mmar, men vi
m{\aa}ste d{\aa} vara noga med att inte tolka in n{\aa}gra elektrodynamiska
effekter direkt fr{\aa}n den skal{\"a}ra potentialen.

Om inga speciella omst{\"a}ndigheter r{\aa}der, {\"a}r en generell
rekommendation att anv{\"a}nda Lorenz gauge n{\"a}rhelst det {\"a}r
m{\"o}jligt, d{\aa} detta inte kan ge upphov till eventuella misstolkningar
av de retarderade potentialerna som bist{\aa}r med l{\"o}sningarna.
\vfill\eject

\section{Retarderade potentialer p{\aa} integralform}
\sidx{Retarderad potential}[Tidsf{\"o}rdr{\"o}jd potential p{\aa} integralform]
Vi kommer nu att forts{\"a}tta under antagandet att vi fixerar potentialerna
under Lorenz-villkoret. Som vi sett kan vi om vi fixerar den skal{\"a}ra
potentialen och vektorpotentialen i {\it Lorenz gauge} frikoppla ekvationerna
fr{\aa}n varandra, till tv{\aa} inhomogena v{\aa}gekvationer.

Ofta kan f{\"a}ltekvationerna f{\"o}r ${\bf E}({\bf x},t)$ och
${\bf B}({\bf x},t)$ l{\"o}sas genom att precis som i det elektrostatiska
fallet f{\"o}rst ber{\"a}kna integralerna f{\"o}r (den genom Lorenz-villkoret
frikopplade) skal{\"a}ra potentialen och vektorpotentialen. Dessa potentialer
{\"a}r dock f{\"o}r det elektrodynamiska fallet tidsberoende, oavsett att de i
{\it Lorenz gauge} {\"a}r frikopplade, och deras v{\"a}rden $\phi({\bf x},t)$
respektive ${\bf A}({\bf x},t)$ vid tidpunkten $t$ kommer att bero p{\aa}
summan av deras infinitesimala bidrag\numberedfootnote{{\AA}terigen, notera
  att vi i denna f{\"o}rel{\"a}sningsserie genomg{\aa}ende anv{\"a}nder
  notationen ``$dS$'' eller ``$d{\bf S}$'' f{\"o}r {\it ytelement};
  de ``$d{\bf A}$'' som vi h{\"a}r anv{\"a}nder {\"a}r bidrag till
  {\it vektorpotentialen} ${\bf A}$. Just denna risk f{\"o}r
  f{\"o}rv{\"a}xling mellan yta $A$ och vektorpotential ${\bf A}$
  {\"a}r sj{\"a}lva anledningen till att vi genomg{\aa}ende h{\aa}ller
  oss just till ``$dS$'' eller ``$d{\bf S}$'' f{\"o}r ytelement!}
$d\phi({\bf x},t')$ och $d{\bf A}({\bf x},t')$ fr{\aa}n en {\it tidigare}
tidpunkt $t'$.

\epsfig{../lect-11/figs/chargedist.1}\noindent
Den tid som det tar f{\"o}r den elektromagnetiska potentialen (eller f{\"o}r
den delen, det elektromagnetiska f{\"a}ltet) att n{\aa} observationspunkten
${\bf x}$ fr{\aa}n k{\"a}llpunkten ${\bf x}'$ {\"a}r helt enkelt
$\Delta t = |{\bf x}-{\bf x}'|/c$, vilket inneb{\"a}r att n{\"a}r vi summerar
alla infinitesimala bidrag i den vanliga volymsintegralen f{\"o}r potentialerna,
s{\aa} m{\aa}ste vi vara noga med att inte bara summera {\"o}ver rummet, utan
{\"a}ven ta h{\"a}nsyn till den tid som det tar f{\"o}r potentialen att n{\aa}
observationspunkten. H{\"a}rav att vi kallar den elektrodynamiska varianten av
den skal{\"a}ra potentialen och vektorpotentialen f{\"o}r {\it retarderade
potentialer} (eller ``f{\"o}rdr{\"o}jda'' potentialer, om man s{\aa} vill).
Dessa potentialer formuleras som volymintegralerna
$$
  \phi({\bf x},t)={{1}\over{4\pi\varepsilon_0}}\iiint_{{\Bbb R}^3}
    {{\rho({\bf x}',t')}\over{|{\bf x}-{\bf x}'|}}\,dV',\qquad\qquad
  {\bf A}({\bf x},t)={{\mu_0}\over{4\pi}}\iiint_{{\Bbb R}^3}
    {{{\bf J}({\bf x}',t')}\over{|{\bf x}-{\bf x}'|}}\,dV',
$$
d{\"a}r
$$
  t' = t - |{\bf x}-{\bf x}'|/c
$$
{\"a}r den {\it retarderade tiden} som beror av avst{\aa}ndet mellan k{\"a}lla
${\bf x}'$ och observationspunkt ${\bf x}$. Utifr{\aa}n dessa tv{\aa} integraler
f{\aa}s d{\"a}refter de elektromagnetiska f{\"a}lten ${\bf E}({\bf x},t)$ och
${\bf B}({\bf x},t)$ direkt fr{\aa}n relationerna
$$
  {\bf B}({\bf x},t)=\nabla\times{\bf A}({\bf x},t),\qquad\qquad
  {\bf E}({\bf x},t)
    =-\nabla\phi({\bf x},t)-{{\partial{\bf A}({\bf x},t)}\over{\partial t}}.
$$
Notera att denna evaluering av potentialerna sj{\"a}lvfallet sker vid den
{\it aktuella} tiden $t$ (vid vilken vid evaluerar f{\"a}lten); all tidigare
historik hos alla delbidragande volymselement f{\"o}r de retarderade
potentialerna har ju inkluderats genom sj{\"a}lva integrationen.
\vfill\eject

\section{Exempel: Halvv{\aa}gsantenn och emitterade elektromagnetiska f{\"a}lt}
\sidx{Halvv{\aa}gsantenn}\sidx{Antenn}[Halvv{\aa}gsantenn]
Som ett handfast exempel p{\aa} den praktiska till{\"a}mpningen av de
retarderade potentialerna $\phi({\bf x},t)$ och ${\bf A}({\bf x},t)$ skall vi
nu genom dessa potentialer ber{\"a}kna de emitterade elektriska och magnetiska
f{\"a}lten ${\bf E}({\bf x},t)$ och ${\bf B}({\bf x},t)$ fr{\aa}n en
halvv{\aa}gsantenn (s{\aa} kallad ``dipolantenn''), riktad l{\"a}ngs
${\bf e}_z$ och matad med str{\"o}mmen
$$
  I(z,t) = I_0\cos(2\pi z/\lambda)\sin(\omega t),
$$
f{\"o}r $-\lambda/4\le z\le \lambda/4$ och med $\omega/c=2\pi/\lambda$.
Evaluera f{\"a}lten vid en punkt ${\bf x}=x{\bf e}_x$ d{\"a}r $x\gg\lambda$.

\epsfig{../lect-11/figs/example.1}\noindent
{\it F{\"o}rst av allt: N{\aa}gra kvalificerade gissningar}
\smallskip
\noindent
\item{$\bullet$}{F{\"a}lten b{\"o}r i punkten ${\bf x}=x{\bf e}_x$ rimligen
    breda ut sig med en v{\aa}gvektor riktad l{\"a}ngs $x$-axeln.}
\item{$\bullet$}{I punkten ${\bf x}=x{\bf e}_x$ b{\"o}r det elektriska
    f{\"a}ltet av symmetrisk{\"a}l hos k{\"a}llan vara riktat l{\"a}ngs
    med $z$-axeln.}
\item{$\bullet$}{Eftersom den magnetiska fl{\"o}destt{\"a}theten {\"a}r
    ortogonal mot b{\aa}de v{\aa}gvektorn och det elektriska f{\"a}ltet,
    s{\aa} b{\"o}r det vara riktat l{\"a}ngs $y$-axeln (in{\aa}t i figurens
    plan).}
\item{$\bullet$}{Vi kan gissa oss till att en rimlig arbetsg{\aa}ng {\"a}r
    att fr{\aa}n str{\"o}mmen s{\"o}ka potentialerna och ur dessa f{\"a}lten,
    $$
      I(z,t)\quad\to\quad
      \underbrace{\phi({\bf x},t),{\bf A}({\bf x},t)}_{\hbox{retarderade
          potentialer}}
      \quad\to\qquad
      \underbrace{
        \vbox{
          \hbox{$\displaystyle
	    {\bf E}({\bf x},t)=-\nabla\phi({\bf x},t)
	        -{{\partial{\bf A}({\bf x},t)}\over{\partial t}}$}
          \hbox{$\displaystyle
	    {\bf B}({\bf x},t)=\nabla\times{\bf A}({\bf x},t)$}
	}
      }_{\hbox{f{\"a}lt}}
    $$}
\item{$\bullet$}{Vektorpotentialen kan ber{\"a}knas ur den givna str{\"o}mmen;
    dock beh{\"o}ver vi f{\"o}r den skal{\"a}ra potentialen {\"a}ven den
    elektriska laddningen p{\aa} antennen. {\"A}ven om denna inte {\"a}r
    given, s{\aa} kan vi gissa oss till att denna i alla h{\"a}ndelser kan
    tas fram ur det allm{\"a}nt giltiga kontinuitetssambandet
    (konserveringslagen) f{\"o}r elektrisk laddning)
    $$
      {{\partial\rho}\over{\partial t}}+\nabla\cdot{\bf J}_{\bf f}=0.
    $$}
\vfill\eject
\noindent
{\it L{\"o}sning}
\smallskip
\noindent
Vi vet redan p{\aa} f{\"o}rhand att vektorpotentialen kan tas fram direkt ur
den givna str{\"o}mmen i antennen. F{\"o}r den skal{\"a}ra potentialen
beh{\"o}ver vi dock {\"a}ven laddningsdensiteten l{\"a}ngs antennen, n{\aa}got
som ej {\"a}r givet direkt i uppgiften. Fr{\aa}n den generella lagen om
konservering av laddning i tre dimensioner f{\aa}r vi dock fram
laddningsdensiteten {\it per l{\"a}ngdenhet} $\rho_{\ell}(z,t)$ p{\aa}
antennen fr{\aa}n den k{\"a}nda str{\"o}mmen, som:
$$
  {{\partial\rho}\over{\partial t}}+\nabla\cdot{\bf J}_{\bf f}=0
    \qquad\Rightarrow\qquad
  {{\partial\rho_{\ell}(z,t)}\over{\partial t}}
    +{{\partial I(z,t)}\over{\partial z}}=0,
$$
det vill s{\"a}ga, om vi nu integrerar detta i tiden (fr{\aa}n, s{\"a}g, $t=0$)
f{\"o}r att f{\aa} fram laddningen per l{\"a}ngdenhet l{\"a}ngs antennen,
$$
  \eqalign{
    \underbrace{\rho_{\ell}(z,t)}_{\hbox{(C/m)}}
     &=-\int^{t}_{0}{{\partial I(z,\tau)}\over{\partial z}}\,d\tau\cr
     &=\{\ I(z,t) = I_0\cos(2\pi z/\lambda)\sin(\omega t)\ \}\cr
     &=-I_0 \underbrace{{{\partial\cos(2\pi z/\lambda)}
            \over{\partial z}}}_{
	       \displaystyle\bigg(-{{2\pi}
	                     \over{\lambda}}\sin(2\pi z/\lambda)\bigg)}
	    \underbrace{\int^{t}_{0}\sin(\omega\tau)\,d\tau}_{
	       \displaystyle\bigg(-{{(\cos(\omega t)-1)}\over{\omega}}\bigg)}\cr
     &={{2\pi I_0}\over{\lambda\omega}}\sin(2\pi z/\lambda)
            (1-\cos(\omega t))\cr
     &=\{\ \lambda\omega=\lambda(2\pi c/\lambda)=2\pi c\ \}\cr
     &=\underbrace{(I_0/c)}_{\hbox{(C/m)}}\sin(2\pi z/\lambda)
            (1-\cos(\omega t))\cr
  }
$$
Ur detta f{\aa}r vi d{\"a}refter direkt den retarderade skal{\"a}ra potentialen
som
$$
  \eqalign{
    \phi({\bf x},t)
      &={{1}\over{4\pi\varepsilon_0}}\iiint_{{\Bbb R}^3}
        {{\rho({\bf x}',t')}\over{|{\bf x}-{\bf x}'|}}\,dV'\cr
      &=\{\ \hbox{linjeladdning}\ \to\ \hbox{en-dimensionell integral}\ \}\cr
      &={{1}\over{4\pi\varepsilon_0}}\int^{\lambda/4}_{-\lambda/4}
        {{\rho_{\ell}({\bf x}',t')}\over{|{\bf x}-{\bf x}'|}}\,dz'\cr
      &={{I_0}\over{4\pi\varepsilon_0 c}}\int^{\lambda/4}_{-\lambda/4}
        {{\sin(2\pi z'/\lambda)(1-\cos(\omega t'))}
	    \over{|x{\bf e}_x-z'{\bf e}_z|}}\,dz'\cr
      &=\{\ \hbox{retarderad tid,}\ t'=t-|{\bf x}-{\bf x}'|/c
            =t-\sqrt{x^2+z'^2}/c\ \}\cr
      &={{I_0}\over{4\pi\varepsilon_0 c}}\int^{\lambda/4}_{-\lambda/4}
        {{\sin(2\pi z'/\lambda)(1-\cos(\omega(t-\sqrt{x^2+z'^2}/c)))}
	    \over{\sqrt{x^2+z'^2}}}\,dz'\cr
      &=\{\ \hbox{Antagande i problemet:}\ x\gg\lambda\ \Leftrightarrow\
      x\gg z\in[-\lambda/4,\lambda/4]\ \}\cr
      &={{I_0}\over{4\pi\varepsilon_0 c x}}(1-\cos(\omega(t-x/c)))
          \underbrace{
	    \int^{\lambda/4}_{-\lambda/4}\sin(2\pi z'/\lambda)\,dz'}_{=0}\cr
      &=0\cr
  }
$$
Detta n{\aa}got sn{\"o}pliga resultat {\"a}r n{\aa}got vi faktiskt borde ha
anat redan utifr{\aa}n den symmetriska integralen l{\"a}ngs $z$ och den
{\it anti-symmetriska} formen p{\aa} integranden $\sin(2\pi z/\lambda)$.
Att den skal{\"a}ra potentialen $\phi({\bf x},t)$ {\"a}r identiskt noll betyder
dock {\it inte} att det elektriska f{\"a}ltet {\"a}r noll, vilket vi l{\"a}tt
kan f{\"o}rledas att tro ifr{\aa}n det {\it elektrostatiska} sambandet
${\bf E}=-\nabla\phi$, vilket inte {\"a}r applicerbart r{\"a}tt av f{\"o}r
{\it elektrodynamiska} (tidsberoende) problem som detta. Ist{\"a}llet {\"a}r
det tidsderivatan av vektorpotentialen som kommer att bist{\aa} med detta.
\sidx{Elektrodynamik}

Den retarderade vektorpotentialen f{\aa}s (som vi tidigare n{\"a}mnt) direkt
fr{\aa}n den givna str{\"o}m\-f{\"o}r\-del\-ningen $I(z,t)$ som
\sidx{Retarderad potential}[Tidsf{\"o}rdr{\"o}jd potential]
$$
  \eqalign{
    {\bf A}({\bf x},t)
      &={{\mu_0}\over{4\pi}}\iiint_{{\Bbb R}^3}
        {{{\bf J}(z',t')}\over{|{\bf x}-{\bf x}'|}}\,dV'\cr
      &=\{\ \hbox{linjestr{\"o}m; {\"a}ven noga att vi anv{\"a}nder
                 {\it retarderad} tid $t'$}\ \}\cr
      &={{\mu_0}\over{4\pi}}\int^{\lambda/4}_{-\lambda/4}
        {{I(z',t'){\bf e}_z}\over{|x{\bf e}_x-z'{\bf e}_z|}}\,dz'\cr
      &={{\mu_0 I_0}\over{4\pi}}{\bf e}_z
        \int^{\lambda/4}_{-\lambda/4}
        {{\cos(2\pi z'/\lambda)\sin(\omega t')}\over{\sqrt{x^2+z'^2}}}\,dz'\cr
      &=\{\ \hbox{retarderad tid,}\ t'=t-|{\bf x}-{\bf x}'|/c
            =t-\sqrt{x^2+z'^2}/c\ \}\cr
      &={{\mu_0 I_0}\over{4\pi}}{\bf e}_z
        \int^{\lambda/4}_{-\lambda/4}
        {{\cos(2\pi z'/\lambda)\sin(\omega(t-\sqrt{x^2+z'^2}/c))}
          \over{\sqrt{x^2+z'^2}}}\,dz'\cr
      &=\{\ x\gg\lambda\ \}\cr
      &\approx{{\mu_0 I_0 \sin(\omega(t-x/c))}\over{4\pi x}}{\bf e}_z
        \underbrace{
	    \int^{\lambda/4}_{-\lambda/4}\cos(2\pi z'/\lambda)\,dz'
	}_{=\lambda/\pi}\cr
      &={{\mu_0 I_0 \lambda \sin(\omega(t-x/c))}\over{4\pi^2 x}}{\bf e}_z\cr
  }
$$
Fr{\aa}n detta erh{\aa}lls den magnetiska fl{\"o}dest{\"a}theten vid
observationspunkten ${\bf x}$ som
$$
  \eqalign{
    {\bf B}({\bf x},t)
      &=\nabla\times{\bf A}({\bf x},t)\cr
      &=\bigg(
          {\bf e}_x{{\partial}\over{\partial x}}
	  +\underbrace{
             {\bf e}_y{{\partial}\over{\partial y}}
            +{\bf e}_z{{\partial}\over{\partial z}}}_{\to0}\bigg)\times
      {{\mu_0 I_0 \lambda \sin(\omega(t-x/c))}\over{4\pi^2 x}}{\bf e}_z\cr
      &=\underbrace{({\bf e}_x\times{\bf e}_z)}_{=-{\bf e}_y}
        {{\mu_0 I_0 \lambda}\over{4\pi^2}}
        {{\partial}\over{\partial x}}
          \bigg({{\sin(\omega(t-x/c))}\over{x}}\bigg)\cr
      &=-{\bf e}_y{{\mu_0 I_0 \lambda}\over{4\pi^2}}
          \bigg({{-(\omega/c)\cos(\omega(t-x/c))\cdot x
	             -\sin(\omega(t-x/c))\cdot 1}\over{x^2}}\bigg)\cr
      &={\bf e}_y{{\mu_0 I_0 \lambda}\over{4\pi^2}}
          \bigg({{(\omega x/c)\cos(\omega(t-x/c))
	             +\sin(\omega(t-x/c))}\over{x^2}}\bigg)\cr
      &={\bf e}_y{{\mu_0 I_0 \lambda}\over{4\pi^2}}
          \bigg({{\omega}\over{c}}{{\cos(\omega(t-x/c))}\over{x}}
	             +{{\sin(\omega(t-x/c))}\over{x^2}}\bigg)\cr
  }
$$
P{\aa} samma s{\"a}tt erh{\aa}lles den elektriska f{\"a}ltstyrkan vid
observationspunkten ${\bf x}$ som
$$
  \eqalign{
    {\bf E}({\bf x},t)
      &=-\underbrace{\nabla\phi({\bf x},t)}_{=0}
          -{{\partial {\bf A}({\bf x},t)}\over{\partial t}}\cr
      &=-{\bf e}_z{{\partial}\over{\partial t}}\bigg(
           {{\mu_0 I_0 \lambda \sin(\omega(t-x/c))}\over{4\pi^2 x}}
	   \bigg)\cr
      &=-{\bf e}_z{{\mu_0 I_0 \lambda\omega}\over{4\pi^2}}
           {{\cos(\omega(t-x/c))}\over{x}}\cr
      &=\{\ \omega\lambda=2\pi c\ \}\cr
      &=-{\bf e}_z{{\mu_0 c I_0 \lambda\omega}\over{2\pi}}
           {{\cos(\omega(t-x/c))}\over{x}}\cr
  }
$$
\vfill\eject
\noindent
{\it N{\aa}gra intressanta saker att observera ur l{\"o}sningen}
\smallskip
\noindent
\item{$\bullet$}{De elektriska och magnetiska f{\"a}lten {\"a}r riktade precis
    s{\aa} som vi f{\"o}rv{\"a}ntade oss, med ${\bf E}$ l{\"a}ngs ${\bf e}_z$
    och med ${\bf B}$ l{\"a}ngs ${\bf e}_y$.}
\item{$\bullet$}{Vi kan enkelt verifiera l{\"o}sningarna (utf{\"o}r g{\"a}rna
    denna exercis!) f{\"o}r f{\"a}lten ${\bf E}({\bf x},t)$ och
    ${\bf B}({\bf x},t)$ mot varandra genom att utv{\"a}rdera h{\"o}ger-
    och v{\"a}nsterledet i Faradays lag, $$\nabla\times{\bf E}({\bf x},t)
    =-{{\partial{\bf B}({\bf x},t)}\over{\partial t}}.$$}
\item{$\bullet$}{N{\aa}got mer f{\"o}rv{\aa}nande {\"a}r att det elektriska
    f{\"a}ltet avtar med avst{\aa}ndet fr{\aa}n antennen som ${\bf E}\sim 1/x$,
    medan magnetf{\"a}ltet ${\bf B}$ best{\aa}r av tv{\aa} termer med den ena
    $\sim 1/x$ och den andra $\sim 1/x^2$. Vi m{\aa}ste h{\"a}r dock h{\aa}lla
    i minnet att vi d{\aa} vi n{\"a}rmar oss n{\"a}rf{\"a}ltet, s{\aa}
    g{\"a}ller inte antagandet om en direkt proportionalitet mellan det
    elektriska och magnetiska f{\"a}ltet. Faradays lag, som {\"a}r grunden
    {\"a}ven f{\"o}r denna avvikelse mellan f{\"a}ltens avtagande med $x$ i
    n{\"a}rf{\"a}ltet, g{\"a}ller dock alltid.}
\item{$\bullet$}{Eftersom ${\bf E}$ f{\"o}r $x\gg\lambda$ avtar som $\sim 1/x$,
    s{\aa} avtar f{\"a}ltstyrkan effektivt p{\aa} samma s{\"a}tt som en
    motsvarande {statisk elektrisk monopol}. Att vi vi har avtagandet som
    ${\bf E}\sim 1/x$ {\"a}r faktiskt f{\"o}ruts{\"a}ttningen f{\"o}r
    l{\aa}ngdistanskommunikation med radiofrekvenser! (Man skulle m{\"o}jligen,
    i en analogi med det elektrostatiska fallet, annars kunna f{\"o}rv{\"a}nta
    sig att f{\"a}ltet spreds som fr{\aa}n en elektrostatisk dipol, som
    ${\bf E}\sim 1/x^2$, men detta {\"a}r allts{\aa} felaktigt.)}
\item{$\bullet$}{Allts{\aa}: Det {\it elektrodynamiska} (tidsberoende)
    f{\"a}lten skiljer sig radikalt fr{\aa}n de {\it elektrostatiska}
    (tidsoberoende)!}
\vfill\eject

\section{Sammanfattning av F{\"o}rel{\"a}sning~11 -- Retarderade potentialer}
\item{$\bullet$}{Den elektro{\it dynamiska} formen f{\"o}r den skal{\"a}ra
  potentialen $\phi$ och vektorpotentialen ${\bf A}$ lyder
  \sidx{Skal{\"a}r potential $\phi$}[Elektrodynamisk]
  \sidx{Vektorpotential ${\bf A}$}[Elektrodynamisk]
  $$
    \eqalign{
      {\bf E}({\bf x},t)
        &=-\nabla\phi({\bf x},t)
            -{{\partial{\bf A}({\bf x},t)}\over{\partial t}},\cr
      {\bf B}({\bf x},t)
        &=\nabla\times{\bf A}({\bf x},t).\cr
    }
  $$}
\item{$\bullet$}{Den s{\aa} kallade {\it gauge-transformationen} av
  potentialerna,\sidx{Gauge-transform}\sidx{Gauge-funktion}
  $$
    \eqalign{
      \phi'({\bf x},t)&=\phi({\bf x},t)
        -{{\partial\psi({\bf x},t)}\over{\partial t}},\cr
      {\bf A}'({\bf x},t)&={\bf A}({\bf x},t)+\nabla\psi({\bf x},t),\cr
    }
  $$
  d{\"a}r $\psi$ {\"a}r en funktion som {\"a}r tv{\aa} g{\aa}nger kontinuerligt
  differentierbar i rum och tid, l{\"a}mnar de elektromagnetiska f{\"a}lten
  of{\"o}r{\"a}ndrade. Vi kan med denna transform till viss del v{\"a}lja
  vilken form av v{\aa}gekvation vi {\"o}nskar f{\"o}r potentialerna.}
\item{$\bullet$}{Under {\it Lorenz-villkoret}, d{\"a}r vi v{\"a}ljer $\psi$
  s{\aa} att\sidx{Lorenz-villkoret}
  $$
    \nabla\cdot{\bf A}'({\bf x},t)+
      {{1}\over{c^2}}
      {{\partial\phi'({\bf x},t)}\over{\partial t}} = 0,
  $$
  s{\aa} frikopplar ekvationerna f{\"o}r potentialerna fr{\aa}n varandra,
  resulterande i v{\aa}gekvationerna
  $$
    \eqalign{
      \Big(
        \nabla^2-{{1}\over{c^2}}{{\partial^2}\over{\partial t^2}}
      \Big)\phi({\bf x},t)
          &=-{{\rho({\bf x},t)}\over{\varepsilon_0\varepsilon_{\rm r}}},\cr
      \Big(
        \nabla^2-{{1}\over{c^2}}{{\partial^2}\over{\partial t^2}}
      \Big){\bf A}({\bf x},t)
          &=-\mu_0{\bf J}_{\rm f}({\bf x},t).\cr
    }
  $$}
\item{$\bullet$}{De {\it retarderade potentialerna} (f{\"o}rdr{\"o}jda
  potentialerna) tar h{\"a}nsyn till den tidsf{\"o}rdr{\"o}jning som olika
  k{\"a}llor vid k{\"a}llpunkterna ${\bf x}'$ har fram till att deras
  respektive bidrag observeras vid en observationspunkt ${\bf x}$, och
  formuleras p{\aa} integralform som
  \sidx{Retarderad potential}[Tidsf{\"o}rdr{\"o}jd potential]
  $$
    \eqalign{
      \phi({\bf x},t)&={{1}\over{4\pi\varepsilon_0}}\iiint_{{\Bbb R}^3}
        {{\rho({\bf x}',t')}\over{|{\bf x}-{\bf x}'|}}\,dV',\cr
      {\bf A}({\bf x},t)&={{\mu_0}\over{4\pi}}\iiint_{{\Bbb R}^3}
        {{{\bf J}({\bf x}',t')}\over{|{\bf x}-{\bf x}'|}}\,dV',\cr
    }
  $$
  d{\"a}r $t'=t-|{\bf x}-{\bf x}'|/c$ {\"a}r den {\it retarderade tiden}
  fr{\aa}n ${\bf x}'$ till ${\bf x}$.
  \sidx{Retarderad tid}[Tidsf{\"o}rdr{\"o}jning]}

\cleardoublepage
%%% End of auto-extracted text from ../lect-11/lecture-11.tex %%%
%%% Begin of auto-extracted text from ../lect-12/lecture-12.tex %%%
%
% File: teach/elmagii/lect-12/lecture-12.tex [plain TeX code]
% Github: https://github.com/elmagii/lect-12/
% Last change: January 1, 2026
%
% Lecture No 12 in the course ``Elektromagnetism II, 1TE626 (2023)'',
% held December 12, 2025, at Uppsala University, Sweden.
%
% Copyright (C) 2022-2025, Fredrik Jonsson, under Gnu General Public License
% (GPL) v3. See the enclosed LICENSE for details.
%
% This program is free software: you can redistribute it and/or modify
% it under the terms of the GNU General Public License as published by
% the Free Software Foundation, either version 3 of the License, or
% (at your option) any later version.
%
% This program is distributed in the hope that it will be useful,
% but WITHOUT ANY WARRANTY; without even the implied warranty of
% MERCHANTABILITY or FITNESS FOR A PARTICULAR PURPOSE.  See the
% GNU General Public License for more details.
%
% You should have received a copy of the GNU General Public License
% along with this program.  If not, see <https://www.gnu.org/licenses/>.
%
\def\coursename{Elektromagnetism II}
\def\coursecode{1TE626}
\def\courseyear{2025}
\def\courserepo{https://github.com/hp35/elmagii/}
\def\lecturenumber{12}
\def\lecturetitle{Grundl{\"a}ggande antennteori}
\def\lecturesubtitle{}
\def\lectureauthor{Fredrik Jonsson}
\def\lectureplace{Uppsala Universitet}
\def\lecturedate{12 december 2025}
%-------------------- BEGIN OF LOCAL MACROS --------------------
\edef\expandedlecturenumber{12}
\def\ifempty#1{\ifx\relax#1\relax}
\advance\chapno by 1
\secno=0
\footnotenumber=0
\message{==================== Lecture 12 ====================}
\writenumberedtocentry{chapter}{F{\"o}rel{\"a}sning 12 -- {Grundl{\"a}ggande antennteori}}{\thechapno}
\hsize=150mm\hoffset=4.6mm\vsize=230mm\voffset=7mm
\topskip=0pt\baselineskip=12pt\parskip=0pt\leftskip=0pt\parindent=15pt
\ifcolors
  \voffset=-10.2mm\topskip=0pt
\fi
\headline={\ifnum\secno>0\ifodd\pageno\rightheadline\else\leftheadline\fi
  \else\hfill\fi}
\def\rightheadline{\tenrm{\it F\"orel\"asning 12}
  \hfil{\it \coursename, \coursecode\ (\courseyear)}}
\def\leftheadline{\tenrm{\it \coursename, \coursecode\ (\courseyear)}
  \hfil{\it F\"orel\"asning 12}}
\noindent~\vskip-60pt\hskip-40pt{\epsfbox{../lect-01/macros/UU_logo_color.eps}}
\vskip-42pt\hfill\vbox{
    \hbox{{\it \coursename, \coursecode\ (\courseyear)}}
    \hbox{{\it Lecture Notes, \lectureauthor}}
    \hbox{{\it Document Revision \today}}
    \hbox{{\it \courserepo}}}\vskip 36pt
\centerline{\twelvesc F\"orel\"asning 12}
\vskip 24pt\noindent
\centerline{\twelvesc{Grundl{\"a}ggande antennteori}}
\expandafter\ifempty\expandafter{\lecturesubtitle}%
  \else\centerline{\twelvesc\lecturesubtitle}\fi
\bigskip
\centerline{\lectureauthor, \lectureplace, \lecturedate}
\vskip24pt
%--------------------- END OF LOCAL MACROS ---------------------



\plan{I denna f{\"o}rel{\"a}sning knyter vi ihop den potentialteori som vi under kursens g{\aa}ng utvecklat, och applicerar den p{\aa} ett konkret exempel i form av en antenn.}

\threepointsummary{%
}{%
}{%
}
\vfill\eject\copyrights

\section{Retarderade potentialer f{\"o}r antenner}
\sidx{Retarderade potentialer}[f{\"o}r cylindrisk rak antenn]
Vi betraktar en dipolantenn best{\aa}ende av tv{\aa} identiska cylindriska
element av l{\"a}ngd $L$, separerade med ett litet luftgap $g$. Med ``litet''
luftgap menar vi h{\"a}r ett gap som {\"a}r litet i f{\"o}rh{\aa}llande till
antennelementen ($g\ll L$) s{\aa}v{\"a}l som i f{\"o}rh{\aa}llande till
v{\aa}gl{\"a}ngden ($g\ll\lambda$).
Utan att g{\"o}ra ett alltf{\"o}r stort avsteg fr{\aa}n det generella fallet,
kan vi anta att str{\"o}mt{\"a}theten i antennelementen i huvudsak kommer att
l{\"o}pa i elementens mantelyta, och vi kommer att anta att eventuella effekter
fr{\aa}n elementens plana {\"a}ndytor kan f{\"o}rsummas.
I detta avseende kan vi d{\"a}rf{\"o}r modellera antennen som best{\aa}ende av
tv{\aa} cylindriska r{\"o}r, var och ett b{\"a}rande en str{\"o}mt{\"a}thet som
{\"a}r homogen i varje tv{\"a}rsnitt $z$ av antennen.
\epsfig{../lect-12/figs/dipole-antenna.1}\noindent
\sidx{Dipolantenn}

\vfill\eject

\section{Sammanfattning av F{\"o}rel{\"a}sning~12 -- Grundl{\"a}ggande antennteori}
\item{$\bullet$}{}
\item{$\bullet$}{}
\item{$\bullet$}{}
\item{$\bullet$}{}

\cleardoublepage
%%% End of auto-extracted text from ../lect-12/lecture-12.tex %%%
\index
\bye
