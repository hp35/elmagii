%
% File: teach/elmagii/book/book.tex [plain TeX code]
% Github: https://github.com/elmagii/book/
% Last change: November 28, 2025
%
% Compilation of all Lecture Notes in the course "Elektromagnetism II, 1TE626",
% held 2022-2025 at Uppsala University, Sweden.
%
% Copyright (C) 2022-2025, Fredrik Jonsson, under Gnu General Public
% License (GPL) v3. See the enclosed LICENSE for details.
%
% This program is free software: you can redistribute it and/or modify
% it under the terms of the GNU General Public License as published by
% the Free Software Foundation, either version 3 of the License, or
% (at your option) any later version.
%
% This program is distributed in the hope that it will be useful,
% but WITHOUT ANY WARRANTY; without even the implied warranty of
% MERCHANTABILITY or FITNESS FOR A PARTICULAR PURPOSE.  See the
% GNU General Public License for more details.
%
% You should have received a copy of the GNU General Public License
% along with this program.  If not, see <https://www.gnu.org/licenses/>.
%
\input ../lect-01/macros/epsf.tex
\input ../lect-01/macros/eplain.tex
\input amssym % to get the {\Bbb E} font (strikethrough E)
\input src/fonts.tex
\input src/misc.tex
\input src/toc.tex
\input src/index.tex
\input src/copyrights.tex
\input src/frontpage.tex
\input src/preface.tex

%
% Define language constants. We are here only concerned with Swedish
% or English, and we for this document set the language to be used
% throughout to Swedish.
%
\newcount\swedish \swedish=0
\newcount\english \english=1
\newcount\lang \lang=\swedish  % Change to \english for English

\frontpage                             % Title page of compilation
\pageno=-2\footline{\hfil\folio\hfil}  % Roman pagenumbering for intro
\copyrights                            % Generic copyrights page
\tableofcontents                       % Four pages, manual page advance
\pageno=-9\footline{\hfil\folio\hfil}  % Roman pagenumbering for preface
\preface{{\it First things first}\medskip\noindent
  L{\aa}t oss b{\"o}rja med det viktigaste av allt: Vad av detta kommer p{\aa}
  tentan? I detta har f{\"o}rfattaren en {\it bra} och en {\it d{\aa}lig} nyhet:
  \medskip
  \item{$\bullet$}{Den {\it bra} nyheten {\"a}r att {\it praktiskt taget
                   ingenting} av det som h{\"a}r presenteras kommer p{\aa}
                   tentan.}
  \item{$\bullet$}{Den {\it d{\aa}liga} nyheten {\"a}r att {\it praktiskt
                   taget ingenting} av det som h{\"a}r presenteras kommer
                   p{\aa} tentan.}
  \medskip
  \noindent
  M{\aa}let med dessa {\it Lecture Notes} {\"a}r att bist{\aa} med
  {\it v{\"a}gledning} i den ganska sn{\aa}riga skog som utg{\"o}rs av
  alla teorem inom elektrostatik, magnetostatik och elektrodynamik.
  M{\aa}ls{\"a}ttningen {\"a}r att kondensera de viktigaste passagerna
  fr{\aa}n standardlitteraturen och f{\aa} ihop det sammanhang som utg{\"o}r
  den klassiska elektromagnetiska teorin, med f{\"o}rhoppningen att detta
  underl{\"a}ttar f{\"o}rst{\aa}elsen -- och kreativiteten! -- n{\"a}r det
  senare kommer till praktiskt probleml{\"o}sande. Vi kan p{\aa} s{\"a}tt
  och vis s{\"a}ga att vi har f{\"o}ljande trestegsraket framf{\"o}r oss:
  \medskip
  $$
    \underbrace{
      \hbox{Elektromagnetisk f{\"a}ltteori}
    }_{\vbox{
        \hbox{Kommer (n{\"a}stan) inte}
        \hbox{alls p{\aa} tentan; f{\"o}rst{\aa}else}
        \hbox{hj{\"a}lper dock praktisk}
        \hbox{probleml{\"o}sning!}
       }
      }
    \quad\to\quad
    \underbrace{
      \hbox{Elektromagnetisk probleml{\"o}sning}
    }_{\vbox{
        \hbox{Kommer {\it garanterat} p{\aa} tentan!}
        \hbox{L{\"o}s massor med uppgifter och}
        \hbox{f{\"o}rs{\"o}k att f{\"o}rst{\aa} vad den teore-}
        \hbox{tiska grunden handlar om, s{\aa}}
        \hbox{kommer det att g{\aa} bra!}
       }
      }
    \quad\to\quad
    \hbox{Godk{\"a}nd tenta!}
  $$
  \medskip
  \noindent
  Det praktiska probleml{\"o}sandet {\"a}r n{\aa}got som ni som g{\aa}r denna
  kurs f{\aa}r {\"o}va i lektionssalar p{\aa} tillf{\"a}llen utanf{\"o}r
  f{\"o}rel{\"a}sningstid, och det {\"a}r ocks{\aa} praktiskt probleml{\"o}sande
  som utg{\"o}r fokus i tentamensuppgifterna. Att avkr{\"a}va av teknologer att
  kunna dra fram en godtycklig och ofta algebraiskt komplex h{\"a}rledning ur
  byxfickan vore alldeles f{\"o}r mycket beg{\"a}rt, det f{\"o}rst{\aa}r vem
  som helst.
  \bigskip\noindent{\it Ett undantag}\medskip\noindent
  Med detta sagt s{\aa} finns det {\it ett specifikt undantag} till denna
  praktiska probleml{\"o}sning p{\aa} tentan, n{\"a}mligen h{\"a}rledningen
  av de generella elektromagnetiska v{\aa}gekvationerna som vi h{\"a}rleder
  fram i F{\"o}rel{\"a}sning~9. K{\"a}nn dock ingen oro f{\"o}r denna uppgift;
  tentauppgiften i sig samt l{\"o}sningsf{\"o}rslaget delas ut under f{\"o}rsta
  f{\"o}rel{\"a}sningen, samt att vi har en gemensam genomg{\aa}ng av
  l{\"o}sningen under F{\"o}rel{\"a}sning~9.
  Den som vill kan {\"a}gna denna uppgift {\aa}t att dissekera metodiken och
  hur vi v{\"a}ver in centrala begrepp fr{\aa}n kursen, via Maxwells ekvationer
  samt de konstitutiva relationerna -- eller s{\aa} kan man helt enkelt sl{\aa}
  in detta medelst mekanisk korvstoppning. {\it B{\aa}da varianterna {\"a}r helt
  okej}, och det {\"a}r f{\"o}rfattarens uppgift att {\"a}ven den som inte
  brinner f{\"o}r elektrodynamikens sk{\"o}nhet faktiskt kommer att f{\aa}
  utbyte av korvstoppningsmetoden.
  Ytinl{\"a}rning kommer f{\"o}re djupinl{\"a}rning.
  S{\aa}, med detta sp{\"o}rsm{\aa}l ur v{\"a}gen, l{\aa}t oss g{\aa} vidare!
  \bigskip\noindent{\it Let's move on}\medskip\noindent
  Detta {\"a}r den kompletta serien av f{\"o}rel{\"a}sningar i kursen
  {\it Elektromagnetism II} (1TE626, 5hp), givna 2022--2025 f{\"o}r teknologer
  i andra {\aa}rskursen f{\"o}r Teknisk Fysik p{\aa} Masterniv{\aa} (syftande
  mot civilingenj{\"o}rsexamen) p{\aa} {\AA}ngstr{\"o}mlaboratoriet, Uppsala
  Universitet, med varje kapitel motsvarande en f{\"o}rel{\"a}sning om tv{\aa}
  timmar ($2\times45$ minuter i svensk akademisk standard).
  Dessa f{\"o}rel{\"a}sningar publiceras h{\"a}rmed publikt med hopp om att
  de kan v{\"a}gleda i den stundtals sn{\aa}riga och matematiskt t{\"a}mligen
  intensiva disciplin som elektrostatik, magnetostatik och elektrodynamik
  utg{\"o}r.
  Vissa h{\"a}rledningar {\"a}r till sin natur t{\"a}mligen omst{\"a}ndliga
  och kan med f{\"o}rdel l{\"a}mnas till den intresserade l{\"a}saren att
  f{\"o}lja  p{\aa} egen kammare med papper och penna tillhands.

  F{\"o}rel{\"a}sningarna som s{\aa}dana str{\"a}var mot att f{\"o}rklara de
  grundl{\"a}ggande {\it principerna} p{\aa} vilka elektromagnetismen {\"a}r
  baserade, snarare {\"a}n att bist{\aa} med praktisk probleml{\"o}sning.
  F{\"o}rhoppningen {\"a}r dock att den teori som h{\"a}r presenteras p{\aa}
  ett djupare plan, genom h{\"a}rledning av de samband som d{\"a}refter kommer
  till anv{\"a}ndning just i probleml{\"o}sandet, kan v{\"a}gleda l{\"a}saren
  till en k{\"a}nsla f{\"o}r fysikaliska samband och tumregler.
  Som ett viktigt komplement till f{\"o}rel{\"a}sningarna ges l{\"a}ngs med
  kursens g{\aa}ng gott om {\"o}vningstillf{\"a}llen d{\"a}r kursdeltagarna
  f{\aa}r {\"o}va sig i oms{\"a}ttandet av teori i praktiskt probleml{\"o}sande.

  L{\"a}mpliga f{\"o}rkunskaper f{\"o}r att kunna tillg{\"o}ra sig
  inneh{\aa}llet som t{\"a}cks av f{\"o}rel{\"a}sningarna som h{\"a}rmed
  l{\"a}ggs fram {\"a}r en- och flervariabelanalys samt vektoranalys.
  Grundl{\"a}ggande kunskaper i elektrostatik och magnetostatik kan givetvis
  underl{\"a}tta, men {\"a}r inte n{\"o}dv{\"a}ndiga d{\aa} grunderna g{\aa}s
  igenom i F{\"o}rel{\"a}sning~1 och~4.
  {\"A}ven om f{\"o}rel{\"a}sningarna som h{\"a}r presenteras {\"a}r upplagda
  enligt en specifik kursplan, s{\aa} {\"a}r det f{\"o}rfattarens
  f{\"o}rhoppning att de kan bidra {\"a}ven utanf{\"o}r den akademiska
  sf{\"a}ren p{\aa} {\AA}ngstr{\"o}mlaboratoriet och Uppsala Universitet.

  F{\"o}r att l{\"a}saren skall f{\aa} en uppfattning om riktning och
  v{\"a}gen fram{\aa}t under varje tv{\aa}timmarspass introduceras varje
  f{\"o}rel{\"a}sning med en kortare sammanfattning samt en sammanst{\"a}llning
  av det som f{\"o}rfattaren anser vara de tre viktigaste punkterna som
  l{\"a}saren kan fokusera p{\aa} att ta med sig.
  En kompakt formelsamling som sammanfattar f{\"o}rel{\"a}sningarna {\"a}r
  bifogad p{\aa} sidorna ix--xii, vilka kan vara l{\"a}mpliga att skriva ut
  och ha till hands i fysisk form under det att l{\"a}saren l{\"o}ser problem
  under lektioner eller p{\aa} egen kammare.

  I kursen {\it Elektromagnetism II} {\"a}r den rekommenderade kurslitteraturen
  David J.~Griffiths {\it Introduction to Electrodynamics}, och f{\"o}rekommande
  sidreferenser till denna i f{\"o}rel{\"a}sningarna till denna bok g{\"a}ller
  den fj{\"a}rde utg{\aa}van (2017). Den som {\"o}nskar inf{\"o}rskaffa detta
  standardverk inom elektromagnetism kan dock med f{\"o}rdel passa p{\aa} att
  ist{\"a}llet rikta in sig p{\aa} den uppdaterade femte utg{\aa}van (2023).

  Denna f{\"o}rel{\"a}sningsserie publiceras dels som ett elektroniskt dokument
  p{\aa} {\it Digitala Vetenskapliga Arkivet} (DiVA),
  {\tt https://www.diva-portal.org/smash/record.jsf?pid=diva2:25333}, men
  {\"a}ven som helt {\"o}ppen k{\"a}ll\-kod p{\aa}
  {\tt https://github.com/hp35/elmagii/}, med all text i ren \TeX\ samt kod
  i \MP\ och Python f{\"o}r generering av samtliga figurer.
  \vskip32pt
  \noindent Fredrik Jonsson,\par
  \noindent Februari 2026}

%
% Collection of formulas in electromagnetism.
%
\input ../formulas/formulas-twocol.tex % Two-column collection of formulas

%
% Portrait gallery in electromagnetism.
%
\input ../gallery/src/portrait.tex
\input ../gallery/src/gallery-src.tex

\pageno=1

%%%%%%%%%%%%%%%%%%%%%%%%%%%%%%%%%%%%%%%%%%%%%%%%%%%%%%%%%%%%%%%%%%%%%%%%
%%% The rest of this document is autoloaded and formatted externally %%%
%%%%%%%%%%%%%%%%%%%%%%%%%%%%%%%%%%%%%%%%%%%%%%%%%%%%%%%%%%%%%%%%%%%%%%%%
