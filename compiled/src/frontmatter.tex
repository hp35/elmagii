%
% File: teach/elmagii/book/book.tex [plain TeX code]
% Github: https://github.com/elmagii/book/
% Last change: November 28, 2025
%
% Compilation of all Lecture Notes in the course "Elektromagnetism II, 1TE626",
% held 2022-2025 at Uppsala University, Sweden.
%
% Copyright (C) 2022-2025, Fredrik Jonsson, under Gnu General Public
% License (GPL) v3. See the enclosed LICENSE for details.
%
% This program is free software: you can redistribute it and/or modify
% it under the terms of the GNU General Public License as published by
% the Free Software Foundation, either version 3 of the License, or
% (at your option) any later version.
%
% This program is distributed in the hope that it will be useful,
% but WITHOUT ANY WARRANTY; without even the implied warranty of
% MERCHANTABILITY or FITNESS FOR A PARTICULAR PURPOSE.  See the
% GNU General Public License for more details.
%
% You should have received a copy of the GNU General Public License
% along with this program.  If not, see <https://www.gnu.org/licenses/>.
%
\input ../lect-01/macros/epsf.tex
\input ../lect-01/macros/eplain.tex
% \input macros/bookmacros.tex
\font\ninerm=cmr9
\font\tenssbx=cmssbx10
\font\twelvesc=cmcsc10 at 12 truept
\input amssym % to get the {\Bbb E} font (strikethrough E)

%
% Various font definitions
%
\font\fivecmr=cmr7 at 5 truept
\font\sevencmr=cmr7
\font\eightcmr=cmr8
\font\eightcmssq=cmssq8
\font\eightcmssqi=cmssqi8
\font\ninecmti=cmti9
\font\tencmti=cmti10
\font\tencmss=cmss10
\font\tencmcsc=cmcsc10
\font\tencmssi=cmssi10
\font\twelvecmbx=cmbx12
\font\sixteencmbx=cmbx12 at 16 truept
\font\sixteencmsc=cmcsc10 at 16 truept
\font\fourteencmssbx=cmssbx10 at 14 truept
\font\twentyeightcmssbx=cmssbx10 at 28 truept

%
% Make sure that the METAPOST logo can be loaded even in plain TeX.
%
\ifx\MP\undefined
    \ifx\manfnt\undefined
            \font\manfnt=logo10
    \fi
    \ifx\manfntsl\undefined
            \font\manfntsl=logosl10
    \fi
    \def\MP{{\ifdim\fontdimen1\font>0pt \let\manfnt = \manfntsl \fi
      {\manfnt META}\-{\manfnt POST}}\spacefactor1000 }%
\fi
\ifx\METAPOST\undefined \let\METAPOST=\MP \fi

\countdef\chapno=28         % register to keep track of chapter numbers
\countdef\secno=29          % register to keep track of section numbers
\countdef\subsecno=30       % register to keep track of subsection numbers
\countdef\chapstartpage=31  % register to keep track of start page of chapter
\countdef\appno=32          % register to keep track of appendix numbers
\countdef\figno=32          % register to keep track of figure numbers
\chapno=0
\appno=0
\figno=0
\pageno=-1                  % start with roman page numbering

%
% Define a convenient macro for the expansion of the chapter numbers.
% This macro is introduced in order to provide for an easy change of
% the chapter labeling (from numbers to capital letters) whenever the
% appendices appear, via a redefinition of '\thechapno' in the '\appendix'
% macro definition.
%
\def\thechapno{\the\chapno}

%
% Define a macro for the expansion of page numbers, in similar to the
% '\thechapno' macro for expansion of chapter numbering. In this case
% we are in particular interested in typsetting the pages preceding the
% first chapter with roman numerals; this is done by associating them
% with negative page numbers.
%
\def\thepageno{\ifnum\pageno<0\romannumeral-\pageno \else\number\pageno\fi}
\def\abs#1{\ifnum#1<0 \number-#1\else\number#1\fi}

%
% The following redefinitions of the default eplain TeX table-of-contents
% line styles causes the generated table of contents to be of the form:
%
%        Preface . . . . . . . . . . . . . . . . . . . . . . .  v
%     1  Introduction                                           1
%     2  The classical picture                                  9
%        2.1  The mechanical spring oscillator . . . . . . . . 10
%        2.2  Damped motion  . . . . . . . . . . . . . . . . . 12
%             2.2.1 Longitudinal relaxation  . . . . . . . . . 14
%             2.2.2 Transverse relaxation  . . . . . . . . . . 15
%     3  Lagrangian formulation                                17
%     Bibliography                                             27
%     Index                                                    35
%
\newdimen\tocchapindent \tocchapindent=20pt
\newdimen\tocsecindent \tocsecindent=20pt
\newdimen\tocsubsecindent \tocsubsecindent=27pt
\def\tocpreentry#1#2#3{\line{
  \hbox to\tocchapindent{\hfil}
  \hbox{#1}\dotfill{#3}}}%
\def\tocchapterentry#1#2#3{\bigskip\goodbreak\line{
  \hbox to\tocchapindent{\bf #2\hfil}
  \hbox{\bf #1}\hfill{\bf #3}}}%
\def\tocsectionentry#1#2#3{\line{
  \hbox to\tocchapindent{\hfil}
  \hbox to\tocsecindent{#2\hfil}
  \hbox{#1}\dotfill\ {#3}}}%
\def\tocsubsectionentry#1#2#3{\line{
  \hbox to\tocchapindent{\hfil}
  \hbox to\tocsecindent{\hfil}
  \hbox to\tocsubsecindent{#2\hfil}
  \hbox{#1}\dotfill\ {#3}}}%
\def\tocappendixentry#1#2#3{\bigskip\line{
  \hbox to\tocchapindent{\bf #2\hfil}
  \hbox{\bf #1}\hfill{\bf #3}}}%
\def\tocbibentry#1#2#3{\bigskip\line{
  \hbox{\bf #1}\hfill{\bf #3}}}%
\def\tocindexentry#1#2#3{\tocbibentry{#1}{#2}{#3}}% identical to '\tocbibentry'

%
% The following redefinition of the '\writenumberedcontentsentry' macro of
% the eplain macro package has been done in order to avoid the '\sanitize'
% operation on the secondary number. If '\sanitized', any multiple appearance
% of chapter number, section number, etc. would otherwise be destroyed in the
% secondary number.
%
\def\writecontentsentry#1#2#3{\writenumberedcontentsentry{#1}{#2}{#3}{}}%
\def\writenumberedcontentsentry#1#2#3#4{%
  \csname ifrewrite#1file\endcsname
    \csname open#1file\endcsname
    \toks0 = {\expandafter\noexpand \csname #1#2entry\endcsname}%
    \def\temp{#3}%
    \toks2 = \expandafter{#4}%
    \edef\cs{\the\toks2}%
    \edef\@wr{%
      \write\csname #1file\endcsname{%
        \the\toks0 % the \toc...entry control sequence
        {\sanitize\temp}% the text
        \ifx\empty\cs\else {\cs}\fi % A secondary number, or nothing
        {\noexpand\folio}% the page number
      }%
    }%
    \@wr
  \fi
  \ignorespaces
}%

%
% The '\tableofcontents' macro typesets the table of contents, based on
% the entries written to the .toc file by the '\writenumberedtocentry'
% blocks in the '\chapter', '\section', and '\subsection' macros. The
% page layout and placement of the `Contents' label is identical to the
% preface.
%
\def\tableofcontents{
  \chapstartpage=\pageno
  \message{Contents}
  ~\vskip 5pc\goodbreak\noindent{\twelvecmbx Inneh{\aa}ll}\par
  \nobreak\vskip 48pt\noindent\ignorespaces
  \readtocfile\vfill\eject}


\def\section #1 {\advance\secno by 1
  \subsecno=0
  \writenumberedtocentry{section}{#1}{\thechapno.\the\secno}
  \medskip\goodbreak\noindent{\tenssbx{\thechapno.\the\secno.} #1}
  \par\nobreak\smallskip\noindent}
\def\subsection #1 {\advance\subsecno by 1
  \writenumberedtocentry{subsection}{#1}{\thechapno.\the\secno.\the\subsecno}
  \medskip\goodbreak\noindent{\it{\thechapno.\the\secno.\the\subsecno.} #1}
  \par\nobreak\smallskip\noindent}
\def\iint{\mathop{\int\kern-8pt\int}}
\def\iiint{\mathop{\int\kern-8pt\int\kern-8pt\int}}
\def\oiint{\mathop{\int\kern-8pt\int\kern-13.2pt{\bigcirc}}}
\def\sgn{\mathop{\rm sgn}\nolimits} % sign
\def\Re{\mathop{\rm Re}\nolimits}   % real part
\def\Im{\mathop{\rm Im}\nolimits}   % imaginary part
\def\Tr{\mathop{\rm Tr}\nolimits}   % quantum mechanical trace
\def\eqq{\mathop{\vbox{\hbox{\hskip2pt?}\vskip-6pt\hbox{=}}}}
\def\encircle#1{\kern3pt#1\kern-7.6pt$\bigcirc$\kern4pt}
\def\boxit#1{\vbox{\hrule\hbox{\vrule\kern3pt
  \vbox{\kern3pt#1\kern3pt}\kern3pt\vrule}\hrule}}
\def\quote#1{\leftskip=36pt\rightskip=36pt\smallskip\noindent#1\par
  \leftskip=0pt\rightskip=0pt\smallskip}
\def\plan#1{\par\leftskip=36pt\rightskip=36pt\bigskip%
  \noindent{\it Sammanfattning av f{\"o}rel{\"a}sningen}\smallskip
  \noindent{\it #1}\par\leftskip=0pt\rightskip=0pt} %\vfill\eject}
\def\threepointsummary#1#2#3{\par\leftskip=36pt\rightskip=36pt\bigskip
  \noindent{\it Sammanfattning i tre punkter}\smallskip
  \leftskip=48pt\rightskip=36pt\hangindent=20pt
  \noindent{\it\hbox to 20pt{1. }#1}\smallskip
  \leftskip=48pt\rightskip=36pt\hangindent=20pt
  \noindent{\it\hbox to 20pt{2. }#2}\smallskip
  \leftskip=48pt\rightskip=36pt\hangindent=20pt
  \noindent{\it\hbox to 20pt{3. }#3}\par%\medskip
  \leftskip=0pt\rightskip=0pt\vfill\eject}
\def\epsfig#1{\bigskip\centerline{\epsfbox{#1}}\medskip}
\def\captionwide{\advance\leftskip by 60pt
  \advance\rightskip by 60pt}
\newdimen\itemindent \itemindent=28pt
\newdimen\hangitemindent \hangitemindent=46pt
\def\litem[#1]{\smallbreak\noindent%
  \hbox to\itemindent{\hfil}\hbox to\itemindent{#1\hfill}%
  \hangindent\hangitemindent\ignorespaces}
\def\cleardoublepage{\ifodd\pageno\vfill\eject\fi~\vskip260pt
  \centerline{[Page intentionally left blank]} ~\vfill\eject}

\tableofcontents
\pageno=1

%%%%%%%%%%%%%%%%%%%%%%%%%%%%%%%%%%%%%%%%%%%%%%%%%%%%%%%%%%%%%%%%%%%%%%%%
%%% The rest of this document is autoloaded and formatted externally %%%
%%%%%%%%%%%%%%%%%%%%%%%%%%%%%%%%%%%%%%%%%%%%%%%%%%%%%%%%%%%%%%%%%%%%%%%%
