%
% File: teaching/elmag/lect-11/lecture-11.tex [plain TeX code]
% Last change: December 10, 2024
%
% Lecture No 11 in the course ``Elektromagnetism II, 1TE626 (2023)'',
% held December 13, 2024, at Uppsala University, Sweden.
%
% Copyright (C) 2024, Fredrik Jonsson, under Gnu General Public License
% (GPL) v3. See the enclosed LICENSE for details.
%
% This program is free software: you can redistribute it and/or modify
% it under the terms of the GNU General Public License as published by
% the Free Software Foundation, either version 3 of the License, or
% (at your option) any later version.
%
% This program is distributed in the hope that it will be useful,
% but WITHOUT ANY WARRANTY; without even the implied warranty of
% MERCHANTABILITY or FITNESS FOR A PARTICULAR PURPOSE.  See the
% GNU General Public License for more details.
%
% You should have received a copy of the GNU General Public License
% along with this program.  If not, see <https://www.gnu.org/licenses/>.
%
\input macros/epsf.tex
\input macros/eplain.tex
\font\ninerm=cmr9
\font\twentyrm=cmr12 at 20 truept
\font\twelvesc=cmcsc10
\input amssym % to get the {\Bbb E} font (strikethrough E)
\def\lecture #1 {\hsize=150mm\hoffset=4.6mm\vsize=230mm\voffset=7mm
  \topskip=0pt\baselineskip=12pt\parskip=0pt\leftskip=0pt\parindent=15pt
  \headline={\ifnum\pageno>1\ifodd\pageno\rightheadline\else\leftheadline\fi
    \else\hfill\fi}
  \def\rightheadline{\tenrm{\it F\"orel\"asning #1}
    \hfil{\it Elektromagnetism II, 1TE626 (2024)}}
  \def\leftheadline{\tenrm{\it Elektromagnetism II, 1TE626 (2024)}
    \hfil{\it F\"orel\"asning #1}}
  \noindent~\vskip-60pt\hskip-40pt{\epsfbox{macros/UU_logo_color.eps}}
  \vskip-42pt\hfill\vbox{\hbox{{\it Elektromagnetism II, 1TE626 (2024)}}
  \hbox{{\it Lecture Notes, Fredrik Jonsson}}}\vskip 36pt
    \centerline{\twelvesc F\"orel\"asning #1}
  \vskip 24pt\noindent}
\def\section #1 {\medskip\goodbreak\noindent{\bf #1}
  \par\nobreak\smallskip\noindent}
\def\subsection #1 {\smallskip\goodbreak\noindent{\it #1}
  \par\nobreak\smallskip\noindent}
\def\iint{\mathop{\int\kern-8pt\int}}
\def\iiint{\mathop{\int\kern-8pt\int\kern-8pt\int}}
\def\oiint{\mathop{\int\kern-8pt\int\kern-13.2pt{\bigcirc}}}
\def\Re{\mathop{\rm Re}\nolimits} % real part
\def\Im{\mathop{\rm Im}\nolimits} % imaginary part
\def\Tr{\mathop{\rm Tr}\nolimits} % quantum mechanical trace
\def\eqq{\mathop{\vbox{\hbox{\hskip2pt?}\vskip-6pt\hbox{=}}}}

\lecture{11}
\centerline{\twelvesc Retarderade potentialer som l{\"o}sningar till Maxwells
  ekvationer}
\centerline{Fredrik Jonsson, Uppsala Universitet, 13 december 2024}
\vskip24pt

\section{Elektrodynamiska f{\"a}lt och retarderade potentialer}
Vi har i tidigare f{\"o}rel{\"a}sningar kommit in p{\aa} hur Maxwells
ekvationer kan omformuleras till tv{\aa} v{\aa}gekvationer f{\"o}r den
elektriska f{\"a}ltstyrkan ${\bf E}$ och den magnetiska fl{\"o}dest{\"a}theten
${\bf B}$, med v{\"a}xel\-verkan mellan mediet och de elektromagnetiska
f{\"a}lten beskrivna av k{\"a}lltermer i h{\"o}gerledet enligt
$$
  \eqalign{
    \nabla\times\nabla\times{\bf E}
      +{{1}\over{c^2_0}}{{\partial^2{\bf E}}\over{\partial t^2}}&=
         -\mu_0{{\partial}\over{\partial t}}
          \underbrace{
             \bigg({\bf J}_{\rm f}
	        +{{\partial{\bf P}}\over{\partial t}}
	        +\nabla\times{\bf M}\bigg)}_{\hbox{gemensam k{\"a}llterm}},\cr
    \nabla\times\nabla\times{\bf B}
      +{{1}\over{c^2_0}}{{\partial^2{\bf B}}\over{\partial t^2}}&=
          \mu_0\nabla\times
          \underbrace{
	     \bigg({\bf J}_{\rm f}
	        +{{\partial{\bf P}}\over{\partial t}}
	        +\nabla\times{\bf M}\bigg)}_{\hbox{gemensam k{\"a}llterm}}.\cr
  }
$$
Tidigare har vi anv{\"a}nt den skal{\"a}ra potentialen $\Phi({\bf x})$ och
vektorpotentialen ${\bf A}({\bf x})$ i en {\it elektrostatisk} respektive
{\it magnetostatisk} analys. Specifikt har vi visat hur det statiska elektriska
f{\"a}ltet ${\bf E}$ och statiska magnetiska fl{\"o}dest{\"a}theten ${\bf B}$
direkt kan f{\aa}s fram ur dessa potentialer, och hur vi p{\aa} ett strukturerat
s{\"a}tt kan f{\aa} fram potentialerna genom relativt enkla integraler
({\"o}ver volymer, ytor eller linjer).

Fr{\aa}gan infinner sig d{\aa} naturligtvis om det finns tidsberoende
motsvarigheter till dessa potentialer som kan appliceras p{\aa} tidsberoende
elektromagnetiska f{\"a}lt? I s{\aa} fall skulle vi kunna g{\aa} via dessa
potentialer och ur dessa extrahera de tidsvarierande elektriska och magnetiska
f{\"a}lten. Svaret {\"a}r att det finns s{\aa}dana potentialer, s{\aa} kallade
{\it retarderade potentialer}, vilkas existens kan h{\"a}rledas fram
(n{\"a}stan) analogt med det elektrostatiska fallet.

\section{Recap p{\aa} vad skal{\"a}ra potentialen och vektorpotentialen
         {\"a}r bra f{\"o}r}
I en analogi med klassisk mekanik kan vi betrakta en punktmassa i ett
gravitationsf{\"a}lt ${\bf G}$ med gravitationskonstanten $g$ (N/kg).
Partikeln startar med en horisontell hastighet $v_0$ vid h{\"o}jden $z=h$
och f{\"a}rdas nedf{\"o}r en backe beskriven av funktionen $z=f(x)$.
En typisk uppgift skulle h{\"a}r kunna vara att r{\"a}kna ut sluthastigheten
$v$ vid $z=0$.
\smallskip\centerline{\epsfbox{figs/analogi.1}}
\medskip
\noindent
Vi kan naturligtvis rent principiellt st{\"a}lla upp r{\"o}relseekvationerna
f{\"o}r denna massa, g{\"o}ra vissa antaganden om huruvida hastigheten {\"a}r
l{\aa}g nog f{\"o}r att vi ej skall f{\aa} lyft fr{\aa}n underlaget med mera,
och d{\"a}refter integrera r{\"o}relseekvationerna fram till ett resultat, men
alla f{\"o}rst{\aa}r nog att en betydligt enklare approach (om vi inte s{\"o}ker
funktionen som beskriver hastigheten som funktion av tid) helt enkelt {\"a}r
att ist{\"a}llet konstatera att partikeln tappar i {\it potential}. Denna
f{\"o}rlust i potential kan via partikelns massa $m$ enkelt {\"o}vers{\"a}ttas
till en f{\"o}rlust i {\it potentiell energi}, vilken ist{\"a}llet adderas till
den {\it kinetiska energin}.

Ett annat syns{\"a}tt {\"a}r att se partikeln som att den befinner sig i en
skal{\"a}r mekanisk potential $\Phi(z)=gz$, resulterande i ett
{\it gravitationsf{\"a}lt}
$$
  {\bf G}=-\nabla\Phi(z)=-{\bf e}_z g,
  \qquad\Bigg(\quad\Leftrightarrow\qquad
  {\bf E}=-\nabla\Phi(z)
  \quad\Bigg)
$$
i analogi med ett statiskt elektriskt f{\"a}lt ${\bf E}$. Kraften (N) p{\aa}
punktmassan $m$ (i analogi med kraften ${\bf F}=q{\bf E}$ p{\aa} en
punkt\-laddning $q$ i ett elektrostatikt f{\"a}lt) blir
$$
  {\bf F}=m{\bf G}.
  \qquad\Bigg(\quad\Leftrightarrow\qquad
  {\bf F}=q{\bf E}
  \quad\Bigg)
$$
Om vi utg{\aa}r ifr{\aa}n planet $z=0$ som referens, s{\aa} blir den
potentiella {\it energin} (J) f{\"o}r partikeln p{\aa} h{\"o}jden $z$
d{\"a}rmed
$$
  W = -\int^z_{z=0}{\bf F}\cdot d{\bf x} = mgz
  \qquad\Bigg(\quad\Leftrightarrow\qquad
  W = -\int_{\Gamma}{\bf F}\cdot d{\bf x}
    = -q\int_{\Gamma}{\bf E}\cdot d{\bf x}
  \quad\Bigg)
$$
Denna aningen naivistiska analogi illustrerar hur vi rent elektrostatiskt,
f{\"o}r {\it statiska} elektriska laddningar, kan se p{\aa} den skal{\"a}ra
potentialen $\Phi$ och hur vi kan anv{\"a}nda den i elektrostatiska problem
genom att vi kan extrahera det elektriska f{\"a}ltet genom
${\bf E}=-\nabla\Phi$. Inom magnetism, som i grund och botten handlar om
laddningars {\it dynamik} (r{\"o}relse), har vi ist{\"a}llet vektorpotentialen
${\bf A}$, fr{\aa}n vilken vi ist{\"a}llet f{\aa}r den magnetiska
fl{\"o}dest{\"a}theten som ${\bf B}=\nabla\times{\bf A}$.

\section{Retarderade potentialer och kopplingen till elektromagnetiska f{\"a}lt}
Gauss lag f{\"o}r magnetiska fl{\"o}dest{\"a}theten ${\bf B}({\bf x},t)$
g{\"a}ller alltid generellt, och ger direkt vid hand att {\"a}ven en {\it tidsberoende} vektorpotential ${\bf A}({\bf x},t)$ har exakt samma l{\"a}nk till ${\bf B}({\bf x},t)$ som tidigare, eftersom vi vid godtycklig observationspunkt ${\bf x}$ och godtycklig tid $t$ har att
$$
  \nabla\cdot{\bf B}({\bf x},t)=0
  \qquad\Leftrightarrow\qquad
  {\bf B}({\bf x},t)=\nabla\times{\bf A}({\bf x},t).
$$
Om vi s{\"a}tter in vektorpotentialen ${\bf A}({\bf x},t)$ i Faradays
induktionslag, s{\aa} erh{\aa}ller vi
$$
  \nabla\times{\bf E}({\bf x},t)=
    -{{\partial}\over{\partial t}}
       \underbrace{\nabla\times{\bf A}({\bf x},t)}_{={\bf B}({\bf x},t)}
  \qquad\Leftrightarrow\qquad
  \nabla\times\underbrace{
      \bigg({\bf E}({\bf x},t)
        +{{\partial{\bf A}({\bf x},t)}\over{\partial t}}
      \bigg)
    }_{=-\nabla\Phi({\bf x},t)}=0.
$$
Eftersom rotationen av argumentet {\"a}r noll,\numberedfootnote{Se Griffiths
  sammanfattade vektoridentiteter p{\aa} insidan av p{\"a}rmen,
  $\nabla\times(\nabla f)=0$.}
s{\aa} betyder detta att argumentet kan skrivas som gradienten av en skal{\"a}r
potential, s{\"a}g som
$$
  {\bf E}({\bf x},t)
    +{{\partial{\bf A}({\bf x},t)}\over{\partial t}}
  =-\nabla\Phi({\bf x},t)
  \qquad\Leftrightarrow\qquad
  {\bf E}({\bf x},t)
    =-\nabla\Phi({\bf x},t)-{{\partial{\bf A}({\bf x},t)}\over{\partial t}}.
$$
Valet av negativt tecken f{\"o}r gradienten i potentialen kommer fr{\aa}n
v{\aa}r konvention f{\"o}r krafter p{\aa} laddningar i elektriska f{\"a}lt,
och att positiva laddningar str{\"a}var mot minsta potential.
F{\"o}r att f{\"o}rtydliga vad vi h{\"a}r gjort, s{\aa} har vi {\it endast}
anv{\"a}nt Gauss lag f{\"o}r magnetiska f{\"a}lt samt Faradays induktionslag,
f{\"o}r vilka paret $({\bf E}({\bf x},t),{\bf B}({\bf x},t))$ {\"a}r oberoende
av materialegenskaperna. Med andra ord {\"a}r existensen av den skal{\"a}ra
potentialen $\Phi({\bf x},t)$ och vektorpotentialen ${\bf A}({\bf x},t)$
{\it oberoende av det medium i vilket de analyseras}, {\"a}ven i det
elektrodynamiska (tidsberoende) fallet.

Vi kan h{\"a}r ocks{\aa} notera att f{\"a}lten evalueras p{\aa} exakt samma
punkt spatialt som vi uttrycker potentialerna i, med andra ord s{\aa} har vi
{\"a}nnu inte inf{\"o}rt n{\aa}got ``retarderat'' eller ``f{\"o}rdr{\"o}jt''
i ekvationerna. Detta kommer dock att inkluderas h{\"a}rn{\"a}st.

\section{V{\aa}gekvationen f{\"o}r retarderade potentialer}
Om vi substituerar f{\"o}r den skal{\"a}ra potentialen $\Phi({\bf x},t)$ och
vektorpotentialen ${\bf A}({\bf x},t)$ i Gauss och Amp\`eres
lagar\numberedfootnote{Notera att Gauss och Amp\`eres lagar som involverar
    mediets egenskaper (eftersom vi h{\"a}r betraktar ${\bf E}$ och ${\bf B}$
    som v{\aa}ra elektrodynamiska ``basf{\"a}lt'') s{\aa} l{\aa}ngt ej {\"a}nnu
    anv{\"a}nts; detta {\"a}r punkten d{\aa} vi f{\"o}rst introducerar
    mediets egenskaper.}
s{\aa} f{\aa}r vi, under antagandet om en linj{\"a}r elektrisk
fl{\"o}dest{\"a}thet i homogent medium,
$$
  {\bf D}({\bf x},t)=\varepsilon_0\varepsilon_{\rm r}{\bf E}({\bf x},t),
$$
fr{\aa}n Gauss lag $\nabla\cdot{\bf D}({\bf x},t)=\rho({\bf x},t)$ f{\"o}r den
elektriska fl{\"o}dest{\"a}theten att
$$
  \nabla\cdot{\bf E}({\bf x},t)
    =-\nabla\cdot\bigg(
        \nabla\Phi({\bf x},t)+{{\partial{\bf A}({\bf x},t)}\over{\partial t}}
      \bigg)
    ={{\rho({\bf x},t)}\over{\varepsilon_0\varepsilon_{\rm r}}}
  \qquad\Leftrightarrow\qquad
    \nabla^2\Phi({\bf x},t)
      +{{\partial}\over{\partial t}}\nabla\cdot{\bf A}({\bf x},t)
    =-{{\rho({\bf x},t)}\over{\varepsilon_0\varepsilon_{\rm r}}}
$$
vilket vi {\aa}tminstone fr{\aa}n f{\"o}rekomsten av ``$\nabla^2$'' kan b{\"o}rja gissa oss till kommer att handla om en v{\aa}gekvation. Det som saknas h{\"a}r {\"a}r hur vi skall tolka tidsderivatan av $\nabla\cdot{\bf A}({\bf x},t)$.

Fr{\aa}n Amp\`eres lag i dess grundform (notera att detta {\"a}r punkten d{\aa} vi f{\"o}r in de magnetiska materialegenskaperna i problemet via de konstitutiva relationerna f{\"o}r ${\bf H}$ och ${\bf D}$),
$$
  \nabla\times{\bf H}={\bf J}_{\rm f}+{{\partial{\bf D}}\over{\partial t}},
$$
f{\"o}r enkelhets skull under antagandet om ett ickemagnetiskt medium med
${\bf B}({\bf x},t)=\mu_0{\bf H}({\bf x},t)$, har vi samtidigt att
$$
  \eqalign{
    \nabla\times{\bf B}({\bf x},t)
      &=\underbrace{
         \nabla\times
           \underbrace{
             \nabla\times{\bf A}({\bf x},t)
           }_{={\bf B}({\bf x},t)}
        }_{=\nabla(\nabla\cdot{\bf A})-\nabla^2{\bf A}}\cr
     &=\{\ \hbox{Amp\`eres lag}\ \}\cr
     &=\mu_0{\bf J}_{\rm f}({\bf x},t)
      +\mu_0{{\partial{\bf D}({\bf x},t)}\over{\partial t}}\cr
     &=\mu_0{\bf J}_{\rm f}({\bf x},t)
      +\mu_0\varepsilon_0\varepsilon_{\bf r}
      {{\partial{\bf E}({\bf x},t)}\over{\partial t}}\cr
     &=\{\ \hbox{Uttryck i vektorpotentialer,}
         \ {\bf E}=-\nabla\Phi-{{\partial{\bf A}}\over{\partial t}}\ \}\cr
     &=\mu_0{\bf J}_{\rm f}({\bf x},t)
      -\mu_0\underbrace{\varepsilon_0\varepsilon_{\bf r}
        {{\partial}\over{\partial t}}
        \bigg(
          \nabla\Phi({\bf x},t)+{{\partial{\bf A}({\bf x},t)}\over{\partial t}}
        \bigg)
      }_{\displaystyle={{\partial{\bf D}({\bf x},t)}\over{\partial t}}
         =\varepsilon_0\varepsilon_{\rm r}
            {{\partial{\bf E}({\bf x},t)}\over{\partial t}}}.\cr
  }
$$
Om vi stuvar om termerna lite, s{\aa} har vi under utnyttjandet av $\nabla
\times\nabla\times{\bf A}\equiv\nabla(\nabla\cdot{\bf A})-\nabla^2{\bf A}$ att
$$
  \underbrace{
  \bigg(
    \nabla^2-\underbrace{\mu_0\varepsilon_0\varepsilon_{\bf r}}_{=1/c^2}
      {{\partial^2}\over{\partial t^2}}
  \bigg)
  {\bf A}({\bf x},t)
  }_{\hbox{Typisk v{\aa}gekvation!}}
  -\underbrace{
    \nabla\bigg(
    \nabla\cdot{\bf A}({\bf x},t)+
    \underbrace{\mu_0\varepsilon_0\varepsilon_{\bf r}}_{=1/c^2}
    {{\partial\Phi({\bf x},t)}\over{\partial t}}
  \bigg)}_{\hbox{Fr{\aa}ga: ``Hur bli av med denna?''}}
    =-\mu_0{\bf J}_{\rm f}({\bf x},t)
$$
Om vi skall sammanfatta detta halvv{\"a}gs, s{\aa} har vi i och med detta
reducerat Maxwells fyra ekvationer till tv{\aa}, vilka dock fortfarande {\"a}r
kopplade. Att de fortfarande {\"a}r kopplade g{\"o}r det sv{\aa}rt f{\"o}r oss
att formulera hur dessa potentialer skall extraheras fr{\aa}n k{\"a}nda
variabler s{\aa} som laddningsf{\"o}rdelningar och str{\"o}mt{\"a}theter.
Is{\"a}rkopplingen av dessa tv{\aa} ekvationer kan dock g{\"o}ras genom att
utnyttja en viss grad av godtycklighet som fortfarande finns inneboende i
ekvationerna.

Vi noterar att eftersom vektorpotentialen ${\bf B}({\bf x},t)$ definieras
utifr{\aa}n rotationen
$$
  {\bf B}({\bf x},t)=\nabla\times{\bf A}({\bf x},t)
$$
av vektorpotentialen (och eftersom vi alltid har att $\nabla\times\nabla f
\equiv 0$), s{\aa} {\"a}r den senare godtycklig i den bem{\"a}rkelsen att
magnetf{\"a}ltet l{\"a}mnas invariant d{\aa} vi adderar en gradient av
n{\aa}gon funktion, s{\"a}g $\psi({\bf x},t)$, till den,
$$
  {\bf A}({\bf x},t)
  \quad\to\quad
  {\bf A}'({\bf x},t)={\bf A}({\bf x},t)+\nabla\psi({\bf x},t),
$$
d{\"a}r $\psi({\bf x},t)$ {\"a}r en {\it godtycklig tv{\aa} g{\aa}nger
differentierbar funktion}, beroende av rums-koordinater och tid. F{\"o}r att
det {\it elektriska f{\"a}ltet skall vara of{\"o}r{\"a}ndrat av denna
transformation,} det vill s{\"a}ga att
$$
  {\bf E}({\bf x},t)
    =-\nabla\Phi({\bf x},t)-{{\partial}\over{\partial t}}
        \big[{\bf A}({\bf x},t)+\nabla\psi({\bf x},t)\big]
    =-\nabla\bigg(\Phi({\bf x},t)
         +{{\partial\psi({\bf x},t)}\over{\partial t}}\bigg)
    -{{\partial{\bf A}({\bf x},t)}\over{\partial t}}
$$
skall f{\"o}rbli of{\"o}r{\"a}ndrad, s{\aa} {\"a}r kravet att den skal{\"a}ra
potentialen {\it samtidigt} transformeras som
$$
  \Phi({\bf x},t)
  \quad\to\quad
  \Phi'({\bf x},t)=\Phi({\bf x},t)-{{\partial\psi({\bf x},t)}\over{\partial t}}.
$$
Den parvisa transformationen
$$
  \eqalign{
    {\bf A}'({\bf x},t)
      &={\bf A}({\bf x},t)+\nabla\psi({\bf x},t),\cr
    \Phi'({\bf x},t)
      &=\Phi({\bf x},t)-{{\partial\psi({\bf x},t)}\over{\partial t}},\cr
  }
$$
kallas f{\"o}r {\it gauge-transformation}, och anv{\"a}nds inte bara inom
klassisk elektrodynamik, utan {\"a}ven ofta inom kvantmekanik.

Om vi nu {\aa}terv{\"a}nder till grundproblemet med ekvationen f{\"o}r
vektorpotentialen ovan, s{\aa} kan vi se att vi med de
{\it gauge-transformerade} potentialerna har att
$$
  \eqalign{
      \underbrace{
      \nabla\cdot{\bf A}'({\bf x},t)+
      {{1}\over{c^2}}{{\partial\Phi'({\bf x},t)}\over{\partial t}}
      }_{\hbox{``Det vi f{\"o}rs{\"o}ker bli av med''}}
    &=\nabla\cdot\big({\bf A}({\bf x},t)+\nabla\psi({\bf x},t)\big)+
      {{1}\over{c^2}}{{\partial}\over{\partial t}}
      \bigg(
        \Phi({\bf x},t)-{{\partial\psi({\bf x},t)}\over{\partial t}}
      \bigg)\cr
    &=\{\ \hbox{Stuva om termer}\ \}\cr
    &=\nabla\cdot{\bf A}({\bf x},t)
        +{{1}\over{c^2}}{{\partial\Phi({\bf x},t)}\over{\partial t}}
      +\underbrace{
         \bigg(
           \nabla^2\psi({\bf x},t)
             -{{1}\over{c^2}}{{\partial^2\psi({\bf x},t)}\over{\partial t^2}}
         \bigg)
       }_{\vbox{
          \hbox{Vi har frihet att v{\"a}lja $\psi$}
          \hbox{s{\aa} att denna blir noll!}
         }}.\cr
  }
$$
Eftersom funktionen $\psi({\bf x},t)$ {\"a}r {\it godtycklig} (och vi ser nu
dessutom i formen ovan varf{\"o}r vi kr{\"a}ver att funktionen {\"a}r just
{\it tv{\aa}} g{\aa}nger differentierbar b{\aa}de i rum och tid), s{\aa}
inneb{\"a}r detta att vi kan v{\"a}lja den s{\aa} att
$$
  \nabla\cdot{\bf A}'({\bf x},t)+
    {{1}\over{c^2}}
    {{\partial\Phi'({\bf x},t)}\over{\partial t}} = 0.
$$
Denna relation mellan vektorpotential och skal{\"a}r potential kallas
allm{\"a}nt {\it Lorenz-villkoret}.
Den frihet som gauge-transformen av den skal{\"a}ra potentialen och
vektorpotentialen bist{\aa}r med inneb{\"a}r allts{\aa} att vi har friheten
att v{\"a}lja potentialer $\Phi({\bf x},t)$ och ${\bf A}({\bf x},t)$ s{\aa}
att ekvationerna f{\"o}r dem frikopplas, resulterande i tv{\aa} inhomogena
partiella differentialekvationer
$$
  \eqalign{
    \nabla^2\Phi({\bf x},t)-{{1}\over{c^2}}
      {{\partial^2\Phi({\bf x},t)}\over{\partial t^2}}
      &={{\rho({\bf x},t)}\over{\varepsilon_0\varepsilon_{\rm r}}},\cr
    \nabla^2{\bf A}({\bf x},t)-{{1}\over{c^2}}
      {{\partial^2{\bf A}({\bf x},t)}\over{\partial t^2}}
      &=-\mu_0{\bf J}_{\rm r}({\bf x},t).\cr
  }
$$
Fr{\aa}n dessa tv{\aa} {\it v{\aa}gekvationer} ser vi direkt tv{\aa} saker:
\medskip
\item{$\bullet$}{Potentialerna propagerar med ljusets hastighet $c$ (notera
    att $c=c_0/n$ h{\"a}r fortfarande inkluderar brytningsindex
    $n=\sqrt{\varepsilon_{\rm r}}$), s{\aa} teorin bakom gauge-transformationen
    {\"a}r till{\"a}mplig {\"a}ven f{\"o}r elektromagnetisk v{\aa}gpropagation
    i ett dielektrikum.}
\item{$\bullet$}{L{\"o}sningarna till v{\aa}gekvationerna f{\"o}r potentialerna
    kan f{\aa}s fram exakt p{\aa} analogt s{\"a}tt som f{\"o}r det
    elektrostatiska fallet, s{\aa} l{\"a}nge som vi {\"a}r noga med att ta
    i beaktande {\it tidsf{\"o}rdr{\"o}jningen fr{\aa}n k{\"a}lla till
    observationspunkt}, resulterande i s{\aa} kallade {\it retarderade
    potentialer}.}
\medskip
\vfill\eject

\section{[{\"O}verkurs] Gauge-transformen: Lorenz och Coulomb gauge}
\medskip
\noindent
{\it Lorenz gauge}
\medskip
\noindent
Gauge-transformationen ovan, vilken s{\"a}gs fixera potentialerna i den s{\aa}
kallade {\it Lorenz gauge},\numberedfootnote{Om Lorenz gauge (Wikipedia):
    ``It is unique among the constraint gauges in retaining manifest Lorentz
    invariance. Note, however, that this gauge was originally named after the
    Danish physicist Ludvig Lorenz and not after Hendrik Lorentz; it is often
    misspelled ``Lorentz gauge''. (Neither was the first to use it in
    calculations; it was introduced in 1888 by George F.~FitzGerald.)''}
s{\"a}ger egentligen bara att s{\aa} l{\"a}nge som vi utnyttjar valfriheten att
v{\"a}lja den (tv{\aa} g{\aa}nger differentierbara i rum och tid) funktionen
$\psi({\bf x},t)$ s{\aa} att den uppfyller
$$
  \nabla^2\psi({\bf x},t)
    -{{1}\over{c^2}}{{\partial^2\psi({\bf x},t)}\over{\partial t^2}}
  =-\bigg(
      \nabla\cdot{\bf A}({\bf x},t)
        +{{1}\over{c^2}}{{\partial\Phi({\bf x},t)}\over{\partial t}}
     \bigg),
$$
s{\aa} kommer ekvationerna f{\"o}r $\Phi({\bf x},t)$ och ${\bf A}({\bf x},t)$
att frikopplas till tv{\aa} (i allm{\"a}nhet inhomogena) v{\aa}gekvationer.
Lorenz-villkoret {\"a}r vanligast f{\"o}rekommande n{\"a}r man arbetar med
retarderade potentialer, eftersom vi d{\aa} f{\aa}r frikopplade ekvationer
f{\"o}r den skal{\"a}ra potentialen och
vektorpotentialen.\numberedfootnote{F{\"o}r
    en intressant utl{\"a}ggning kring historiken och utvecklingen av teorin
    bakom gauge-trans\-forma\-tionen, se till exempel J.~D. Jackson,
    {\it Historical roots of gauge invariance}, Rev. Mod. Phys. {\bf 73},
    663 (2001);
    {\tt https://journals.aps.org/rmp/abstract/10.1103/RevModPhys.73.663}.}
\medskip
\noindent
{\it Coulomb gauge}
\medskip
\noindent
Den andra ofta f{\"o}rekommande varianten {\"a}r {\it Coulomb-villkoret}
f{\"o}r gauge-transformen, i vilket vi fixerar potentialerna i det s{\aa}
kallade {\it Coulomb gauge}. I detta fall v{\"a}ljer vi $\psi({\bf x},t)$
s{\aa} att
$$
  \nabla\cdot{\bf A}({\bf x},t)=0.
$$
Observera att $\nabla\cdot{\bf A}({\bf x},t)=0$ sj{\"a}lvfallet {\it inte}
p{\aa} n{\aa}got s{\"a}tt inneb{\"a}r att $\nabla\times{\bf A}({\bf x},t)$
(som ju {\"a}r detsamma som magnetiska fl{\"o}dest{\"a}theten
${\bf B}({\bf x},t)$) n{\"o}dv{\"a}ndigtvis {\"a}r noll.

I detta fall blir v{\aa}rt krav p{\aa} $\psi({\bf x},t)$ att funktionen
ist{\"a}llet uppfyller
$$
  \nabla^2\psi({\bf x},t)
    -{{1}\over{c^2}}{{\partial^2\psi({\bf x},t)}\over{\partial t^2}}
  =-{{1}\over{c^2}}{{\partial\Phi({\bf x},t)}\over{\partial t}},
$$
och vi ser att de partiella differentialekvationerna f{\"o}r potentialerna
ist{\"a}llet antar formen
$$
  \eqalign{
    \nabla^2\Phi({\bf x},t)
    &=-{{\rho({\bf x},t)}\over{\varepsilon_0\varepsilon_{\rm r}}},\cr
    \bigg(
      \nabla^2-{{1}\over{c^2}}{{\partial^2}\over{\partial t^2}}
    \bigg){\bf A}({\bf x},t)
    &=-\mu_0{\bf J}_{\rm f}({\bf x},t)
      +{{1}\over{c^2}}\nabla{{\partial\Phi({\bf x},t)}\over{\partial t}},\cr
  }
$$
det vill s{\"a}ga att den skal{\"a}ra potentialen $\Phi({\bf x},t)$ nu
(t{\"a}mligen ov{\"a}ntat) uppfyller den vanliga {\it statiska}
Poisson-ekvationen (notera att vi i denna ekvation saknar tidsderivata, och
att h{\"o}ger- och v{\"a}nsterled {\"a}r direkt kopplade utan n{\aa}gon
tidsf{\"o}rdr{\"o}jning mellan k{\"a}lla och observationspunkt), med
{\it omedelbar} verkan fr{\aa}n en laddningsf{\"o}rdelning $\rho({\bf x},t)$,
med l{\"o}sningen
$$
  \Phi({\bf x},t)={{1}\over{4\pi\varepsilon_0\varepsilon_{\rm r}}}
      \iiint_{{\Bbb R}^3}{{\rho({\bf x},t)}\over{|{\bf x}-{\bf x}'|}}\,dV'.
$$
Ett annat s{\"a}tt att se p{\aa} detta {\"a}r att vi i denna gauge har den
skal{\"a}ra potentialen som en {\it direkt och omedelbar} Coulomb-potential
fr{\aa}n laddningst{\"a}theten $\rho({\bf x},t)$, d{\"a}rav att detta betecknas
som {\it Coulomb gauge}. I denna gauge l{\"o}ser vi principiellt f{\"o}rst
f{\"o}r den skal{\"a}ra potentialen (och ignorerar det faktum att potentialer
likt de elektromagnetiska f{\"a}lten i verkligheten sj{\"a}lvfallet propagerar
med ljusets hastighet), varvid vi anv{\"a}nder l{\"o}sningen $\Phi({\bf x},t)$
som en k{\"a}llterm i den inhomogena v{\aa}gekvationen f{\"o}r
vektorpotentialen ${\bf A}({\bf x},t)$.

Coulomb-villkoret (Coulomb gauge) anv{\"a}nds ofta f{\"o}r f{\"a}ltproblem
d{\aa} vi har avsaknad av fria laddningar eller str{\"o}mmar. I detta fall
{\"a}r den skal{\"a}ra potentialen $\Phi({\bf x},t)=0$, och vektorpotentialen
uppfyller d{\aa} den homogena v{\aa}gekvationen
$$
  \bigg(
    \nabla^2-{{1}\over{c^2}}{{\partial^2}\over{\partial t^2}}
  \bigg){\bf A}({\bf x},t)=0.
$$
Det elektriska f{\"a}ltet ${\bf E}({\bf x},t)$ och magnetiska
fl{\"o}dest{\"a}theten ${\bf B}({\bf x},t)$ f{\aa}s d{\aa} som
$$
  {\bf E}({\bf x},t)=-{{\partial{\bf A}({\bf x},t)}\over{\partial t}},
  \qquad\qquad
  {\bf B}({\bf x},t)=\nabla\times{\bf A}({\bf x},t).
$$
Eftersom vi uppenbarligen har en ofysikalisk situation i det att den
skal{\"a}ra potentialen i Coulomb gauge {\it omedelbart} f{\aa}r en effekt
p{\aa} en observationspunkt p{\aa} ett avst{\aa}nd fr{\aa}n k{\"a}llan (och
d{\"a}rmed bryter mot den grundl{\"a}ggande fysikaliska pricipen att ingenting
kan f{\"a}rdas fortare {\"a}n ljusets hastighet), s{\aa} infinner sig
fr{\aa}gan om detta verkligen kan vara korrekt?

H{\"a}r m{\aa}ste vi h{\aa}lla is{\"a}r begreppet potential och f{\"a}lt, och
f{\"o}rst av allt konstatera att vektorpotentialen ${\bf A}({\bf x},t)$, som i
Coulomb gauge bist{\aa}r med b{\aa}de det elektriska och magnetiska f{\"a}ltet,
f{\"o}ljer en v{\aa}gekvation som propagerar l{\"o}sningen med ljushastigheten.
I denna v{\aa}gekvation ing{\aa}r den omedelbara skal{\"a}ra potentialen som en
slags {\it artificiell k{\"a}llterm}, fr{\aa}n vilken vi aldrig direkt
extraherar n{\aa}gra f{\"a}lt. Med andra ord, s{\aa} {\"a}r Coulomb gauge
anv{\"a}ndbar i fr{\aa}nvaro av fria laddningar och str{\"o}mmar, men vi
m{\aa}ste d{\aa} vara noga med att inte tolka in n{\aa}gra elektrodynamiska
effekter direkt fr{\aa}n den skal{\"a}ra potentialen.

Om inga speciella omst{\"a}ndigheter r{\aa}der, {\"a}r en generell
rekommendation att anv{\"a}nda Lorenz gauge n{\"a}rhelst det {\"a}r
m{\"o}jligt, d{\aa} detta inte kan ge upphov till eventuella misstolkningar
av de retarderade potentialerna som bist{\aa}r med l{\"o}sningarna.
\vfill\eject

\section{Retarderade potentialer p{\aa} integralform}
Vi kommer nu att forts{\"a}tta under antagandet att vi fixerar potentialerna
under Lorenz-villkoret. Som vi sett kan vi om vi fixerar den skal{\"a}ra
potentialen och vektorpotentialen i {\it Lorenz gauge} frikoppla ekvationerna
fr{\aa}n varandra, till tv{\aa} inhomogena v{\aa}gekvationer.

Ofta kan f{\"a}ltekvationerna f{\"o}r ${\bf E}({\bf x},t)$ och
${\bf B}({\bf x},t)$ l{\"o}sas genom att precis som i det elektrostatiska
fallet f{\"o}rst ber{\"a}kna integralerna f{\"o}r (den genom Lorenz-villkoret
frikopplade) skal{\"a}ra potentialen och vektorpotentialen. Dessa potentialer
{\"a}r dock f{\"o}r det elektrodynamiska fallet tidsberoende, oavsett att de i
{\it Lorenz gauge} {\"a}r frikopplade, och deras v{\"a}rden $\Phi({\bf x},t)$
respektive ${\bf A}({\bf x},t)$ vid tidpunkten $t$ kommer att bero p{\aa}
summan av deras infinitesimala bidrag $d\Phi({\bf x},t')$ och
$d{\bf A}({\bf x},t')$ fr{\aa}n en {\it tidigare} tidpunkt $t'$.

\smallskip\centerline{\epsfbox{figs/chargedist.1}}
\medskip
\noindent
Den tid som det tar f{\"o}r den elektromagnetiska potentialen (eller f{\"o}r
den delen, det elektromagnetiska f{\"a}ltet) att n{\aa} observationspunkten
${\bf x}$ fr{\aa}n k{\"a}llpunkten ${\bf x}'$ {\"a}r helt enkelt
$\Delta t = |{\bf x}-{\bf x}'|/c$, vilket inneb{\"a}r att n{\"a}r vi summerar
alla infinitesimala bidrag i den vanliga volymsintegralen f{\"o}r potentialerna,
s{\aa} m{\aa}ste vi vara noga med att inte bara summera {\"o}ver rummet, utan
{\"a}ven ta h{\"a}nsyn till den tid som det tar f{\"o}r potentialen att n{\aa}
observationspunkten. H{\"a}rav att vi kallar den elektrodynamiska varianten av
den skal{\"a}ra potentialen och vektorpotentialen f{\"o}r {\it retarderade
potentialer} (eller ``f{\"o}rdr{\"o}jda'' potentialer, om man s{\aa} vill).
Dessa potentialer formuleras som volymintegralerna
$$
  \Phi({\bf x},t)={{1}\over{4\pi\varepsilon_0}}\iiint_{{\Bbb R}^3}
    {{\rho({\bf x}',t')}\over{|{\bf x}-{\bf x}'|}}\,dV',\qquad\qquad
  {\bf A}({\bf x},t)={{\mu_0}\over{4\pi}}\iiint_{{\Bbb R}^3}
    {{{\bf J}({\bf x}',t')}\over{|{\bf x}-{\bf x}'|}}\,dV',
$$
d{\"a}r
$$
  t' = t - |{\bf x}-{\bf x}'|/c
$$
{\"a}r den {\it retarderade tiden} som beror av avst{\aa}ndet mellan k{\"a}lla
${\bf x}'$ och observationspunkt ${\bf x}$. Utifr{\aa}n dessa tv{\aa} integraler
f{\aa}s d{\"a}refter de elektromagnetiska f{\"a}lten ${\bf E}({\bf x},t)$ och
${\bf B}({\bf x},t)$ direkt fr{\aa}n relationerna
$$
  {\bf B}({\bf x},t)=\nabla\times{\bf A}({\bf x},t),\qquad\qquad
  {\bf E}({\bf x},t)
    =-\nabla\Phi({\bf x},t)-{{\partial{\bf A}({\bf x},t)}\over{\partial t}}.
$$
Notera att denna evaluering av potentialerna sj{\"a}lvfallet sker vid den
{\it aktuella} tiden $t$ (vid vilken vid evaluerar f{\"a}lten); all tidigare
historik hos alla delbidragande volymselement f{\"o}r de retarderade
potentialerna har ju inkluderats genom sj{\"a}lva integrationen.
\vfill\eject

\section{Exempel: Halvv{\aa}gsantenn och emitterade elektromagnetiska f{\"a}lt}
Som ett handfast exempel p{\aa} den praktiska till{\"a}mpningen av de
retarderade potentialerna $\Phi({\bf x},t)$ och ${\bf A}({\bf x},t)$ skall vi
nu genom dessa potentialer ber{\"a}kna de emitterade elektriska och magnetiska
f{\"a}lten ${\bf E}({\bf x},t)$ och ${\bf B}({\bf x},t)$ fr{\aa}n en
halvv{\aa}gsantenn (s{\aa} kallad ``dipolantenn''), riktad l{\"a}ngs
${\bf e}_z$ och matad med str{\"o}mmen
$$
  I(z,t) = I_0\cos(2\pi z/\lambda)\sin(\omega t),
$$
f{\"o}r $-\lambda/4\le z\le \lambda/4$ och med $\omega/c=2\pi/\lambda$.
Evaluera f{\"a}lten vid en punkt ${\bf x}=x{\bf e}_x$ d{\"a}r $x\gg\lambda$.
\smallskip\centerline{\epsfbox{figs/example.1}}
\medskip
\noindent
{\it F{\"o}rst av allt: N{\aa}gra kvalificerade gissningar}
\smallskip
\noindent
\item{$\bullet$}{F{\"a}lten b{\"o}r i punkten ${\bf x}=x{\bf e}_x$ rimligen
    breda ut sig med en v{\aa}gvektor riktad l{\"a}ngs $x$-axeln.}
\item{$\bullet$}{I punkten ${\bf x}=x{\bf e}_x$ b{\"o}r det elektriska
    f{\"a}ltet av symmetrisk{\"a}l hos k{\"a}llan vara riktat l{\"a}ngs
    med $z$-axeln.}
\item{$\bullet$}{Eftersom den magnetiska fl{\"o}destt{\"a}theten {\"a}r
    ortogonal mot b{\aa}de v{\aa}gvektorn och det elektriska f{\"a}ltet,
    s{\aa} b{\"o}r det vara riktat l{\"a}ngs $y$-axeln (in{\aa}t i figurens
    plan).}
\item{$\bullet$}{Vi kan gissa oss till att en rimlig arbetsg{\aa}ng {\"a}r
    att fr{\aa}n str{\"o}mmen s{\"o}ka potentialerna och ur dessa f{\"a}lten,
    $$
      I(z,t)\quad\to\quad
      \underbrace{\Phi({\bf x},t),{\bf A}({\bf x},t)}_{\hbox{retarderade
          potentialer}}
      \quad\to\qquad
      \underbrace{
        \vbox{
          \hbox{$\displaystyle
	    {\bf E}({\bf x},t)=-\nabla\Phi({\bf x},t)
	        -{{\partial{\bf A}({\bf x},t)}\over{\partial t}}$}
          \hbox{$\displaystyle
	    {\bf B}({\bf x},t)=\nabla\times{\bf A}({\bf x},t)$}
	}
      }_{\hbox{f{\"a}lt}}
    $$}
\item{$\bullet$}{Vektorpotentialen kan ber{\"a}knas ur den givna str{\"o}mmen;
    dock beh{\"o}ver vi f{\"o}r den skal{\"a}ra potentialen {\"a}ven den
    elektriska laddningen p{\aa} antennen. {\"A}ven om denna inte {\"a}r
    given, s{\aa} kan vi gissa oss till att denna i alla h{\"a}ndelser kan
    tas fram ur det allm{\"a}nt giltiga kontinuitetssambandet
    (konserveringslagen) f{\"o}r elektrisk laddning)
    $$
      {{\partial\rho}\over{\partial t}}+\nabla\cdot{\bf J}_{\bf f}=0.
    $$}
\vfill\eject
\noindent
{\it L{\"o}sning}
\smallskip
\noindent
Vi vet redan p{\aa} f{\"o}rhand att vektorpotentialen kan tas fram direkt ur
den givna str{\"o}mmen i antennen. F{\"o}r den skal{\"a}ra potentialen
beh{\"o}ver vi dock {\"a}ven laddningsdensiteten l{\"a}ngs antennen, n{\aa}got
som ej {\"a}r givet direkt i uppgiften. Fr{\aa}n den generella lagen om
konservering av laddning i tre dimensioner f{\aa}r vi dock fram
laddningsdensiteten {\it per l{\"a}ngdenhet} $\rho_{\ell}(z,t)$ p{\aa}
antennen fr{\aa}n den k{\"a}nda str{\"o}mmen, som:
$$
  {{\partial\rho}\over{\partial t}}+\nabla\cdot{\bf J}_{\bf f}=0
    \qquad\Rightarrow\qquad
  {{\partial\rho_{\ell}(z,t)}\over{\partial t}}
    +{{\partial I(z,t)}\over{\partial z}}=0,
$$
det vill s{\"a}ga, om vi nu integrerar detta i tiden (fr{\aa}n, s{\"a}g, $t=0$)
f{\"o}r att f{\aa} fram laddningen per l{\"a}ngdenhet l{\"a}ngs antennen,
$$
  \eqalign{
    \underbrace{\rho_{\ell}(z,t)}_{\hbox{(C/m)}}
     &=-\int^{t}_{0}{{\partial I(z,\tau)}\over{\partial z}}\,d\tau\cr
     &=\{\ I(z,t) = I_0\cos(2\pi z/\lambda)\sin(\omega t)\ \}\cr
     &=-I_0 \underbrace{{{\partial\cos(2\pi z/\lambda)}
            \over{\partial z}}}_{
	       \displaystyle\bigg(-{{2\pi}
	                     \over{\lambda}}\sin(2\pi z/\lambda)\bigg)}
	    \underbrace{\int^{t}_{0}\sin(\omega\tau)\,d\tau}_{
	       \displaystyle\bigg(-{{(\cos(\omega t)-1)}\over{\omega}}\bigg)}\cr
     &={{2\pi I_0}\over{\lambda\omega}}\sin(2\pi z/\lambda)
            (1-\cos(\omega t))\cr
     &=\{\ \lambda\omega=\lambda(2\pi c/\lambda)=2\pi c\ \}\cr
     &=\underbrace{(I_0/c)}_{\hbox{(C/m)}}\sin(2\pi z/\lambda)
            (1-\cos(\omega t))\cr
  }
$$
Ur detta f{\aa}r vi d{\"a}refter direkt den retarderade skal{\"a}ra potentialen
som
$$
  \eqalign{
    \Phi({\bf x},t)
      &={{1}\over{4\pi\varepsilon_0}}\iiint_{{\Bbb R}^3}
        {{\rho({\bf x}',t')}\over{|{\bf x}-{\bf x}'|}}\,dV'\cr
      &=\{\ \hbox{linjeladdning}\ \to\ \hbox{en-dimensionell integral}\ \}\cr
      &={{1}\over{4\pi\varepsilon_0}}\int^{\lambda/4}_{-\lambda/4}
        {{\rho_{\ell}({\bf x}',t')}\over{|{\bf x}-{\bf x}'|}}\,dz'\cr
      &={{I_0}\over{4\pi\varepsilon_0 c}}\int^{\lambda/4}_{-\lambda/4}
        {{\sin(2\pi z'/\lambda)(1-\cos(\omega t'))}
	    \over{|x{\bf e}_x-z'{\bf e}_z|}}\,dz'\cr
      &=\{\ \hbox{retarderad tid,}\ t'=t-|{\bf x}-{\bf x}'|/c
            =t-\sqrt{x^2+z'^2}/c\ \}\cr
      &={{I_0}\over{4\pi\varepsilon_0 c}}\int^{\lambda/4}_{-\lambda/4}
        {{\sin(2\pi z'/\lambda)(1-\cos(\omega(t-\sqrt{x^2+z'^2}/c)))}
	    \over{\sqrt{x^2+z'^2}}}\,dz'\cr
      &=\{\ \hbox{Antagande i problemet:}\ x\gg\lambda\ \Leftrightarrow\
      x\gg z\in[-\lambda/4,\lambda/4]\ \}\cr
      &={{I_0}\over{4\pi\varepsilon_0 c x}}(1-\cos(\omega(t-x/c)))
          \underbrace{
	    \int^{\lambda/4}_{-\lambda/4}\sin(2\pi z'/\lambda)\,dz'}_{=0}\cr
      &=0\cr
  }
$$
Detta n{\aa}got sn{\"o}pliga resultat {\"a}r n{\aa}got vi faktiskt borde ha
anat redan utifr{\aa}n den symmetriska integralen l{\"a}ngs $z$ och den
{\it anti-symmetriska} formen p{\aa} integranden $\sin(2\pi z/\lambda)$.
Att den skal{\"a}ra potentialen $\Phi({\bf x},t)$ {\"a}r identiskt noll betyder
dock {\it inte} att det elektriska f{\"a}ltet {\"a}r noll, vilket vi l{\"a}tt
kan f{\"o}rledas att tro ifr{\aa}n det {\it elektrostatiska} sambandet
${\bf E}=-\nabla\Phi$, vilket inte {\"a}r applicerbart r{\"a}tt av f{\"o}r
{\it elektrodynamiska} (tidsberoende) problem som detta. Ist{\"a}llet {\"a}r
det tidsderivatan av vektorpotentialen som kommer att bist{\aa} med detta.

Den retarderade vektorpotentialen f{\aa}s (som vi tidigare n{\"a}mnt) direkt
fr{\aa}n den givna str{\"o}m\-f{\"o}r\-del\-ningen $I(z,t)$ som
$$
  \eqalign{
    {\bf A}({\bf x},t)
      &={{\mu_0}\over{4\pi}}\iiint_{{\Bbb R}^3}
        {{{\bf J}(z',t')}\over{|{\bf x}-{\bf x}'|}}\,dV'\cr
      &=\{\ \hbox{linjestr{\"o}m; {\"a}ven noga att vi anv{\"a}nder
                 {\it retarderad} tid $t'$}\ \}\cr
      &={{\mu_0}\over{4\pi}}\int^{\lambda/4}_{-\lambda/4}
        {{I(z',t'){\bf e}_z}\over{|x{\bf e}_x-z'{\bf e}_z|}}\,dz'\cr
      &={{\mu_0 I_0}\over{4\pi}}{\bf e}_z
        \int^{\lambda/4}_{-\lambda/4}
        {{\cos(2\pi z'/\lambda)\sin(\omega t')}\over{\sqrt{x^2+z'^2}}}\,dz'\cr
      &=\{\ \hbox{retarderad tid,}\ t'=t-|{\bf x}-{\bf x}'|/c
            =t-\sqrt{x^2+z'^2}/c\ \}\cr
      &={{\mu_0 I_0}\over{4\pi}}{\bf e}_z
        \int^{\lambda/4}_{-\lambda/4}
        {{\cos(2\pi z'/\lambda)\sin(\omega(t-\sqrt{x^2+z'^2}/c))}
          \over{\sqrt{x^2+z'^2}}}\,dz'\cr
      &=\{\ x\gg\lambda\ \}\cr
      &\approx{{\mu_0 I_0 \sin(\omega(t-x/c))}\over{4\pi x}}{\bf e}_z
        \underbrace{
	    \int^{\lambda/4}_{-\lambda/4}\cos(2\pi z'/\lambda)\,dz'
	}_{=\lambda/\pi}\cr
      &={{\mu_0 I_0 \lambda \sin(\omega(t-x/c))}\over{4\pi^2 x}}{\bf e}_z\cr
  }
$$
Fr{\aa}n detta erh{\aa}lls den magnetiska fl{\"o}dest{\"a}theten vid
observationspunkten ${\bf x}$ som
$$
  \eqalign{
    {\bf B}({\bf x},t)
      &=\nabla\times{\bf A}({\bf x},t)\cr
      &=\bigg(
          {\bf e}_x{{\partial}\over{\partial x}}
	  +\underbrace{
             {\bf e}_y{{\partial}\over{\partial y}}
            +{\bf e}_z{{\partial}\over{\partial z}}}_{\to0}\bigg)\times
      {{\mu_0 I_0 \lambda \sin(\omega(t-x/c))}\over{4\pi^2 x}}{\bf e}_z\cr
      &=\underbrace{({\bf e}_x\times{\bf e}_z)}_{=-{\bf e}_y}
        {{\mu_0 I_0 \lambda}\over{4\pi^2}}
        {{\partial}\over{\partial x}}
          \bigg({{\sin(\omega(t-x/c))}\over{x}}\bigg)\cr
      &=-{\bf e}_y{{\mu_0 I_0 \lambda}\over{4\pi^2}}
          \bigg({{-(\omega/c)\cos(\omega(t-x/c))\cdot x
	             -\sin(\omega(t-x/c))\cdot 1}\over{x^2}}\bigg)\cr
      &={\bf e}_y{{\mu_0 I_0 \lambda}\over{4\pi^2}}
          \bigg({{(\omega x/c)\cos(\omega(t-x/c))
	             +\sin(\omega(t-x/c))}\over{x^2}}\bigg)\cr
      &={\bf e}_y{{\mu_0 I_0 \lambda}\over{4\pi^2}}
          \bigg({{\omega}\over{c}}{{\cos(\omega(t-x/c))}\over{x}}
	             +{{\sin(\omega(t-x/c))}\over{x^2}}\bigg)\cr
  }
$$
P{\aa} samma s{\"a}tt erh{\aa}lles den elektriska f{\"a}ltstyrkan vid
observationspunkten ${\bf x}$ som
$$
  \eqalign{
    {\bf E}({\bf x},t)
      &=-\underbrace{\nabla\Phi({\bf x},t)}_{=0}
          -{{\partial {\bf A}({\bf x},t)}\over{\partial t}}\cr
      &=-{\bf e}_z{{\partial}\over{\partial t}}\bigg(
           {{\mu_0 I_0 \lambda \sin(\omega(t-x/c))}\over{4\pi^2 x}}
	   \bigg)\cr
      &=-{\bf e}_z{{\mu_0 I_0 \lambda\omega}\over{4\pi^2}}
           {{\cos(\omega(t-x/c))}\over{x}}\cr
      &=\{\ \omega\lambda=2\pi c\ \}\cr
      &=-{\bf e}_z{{\mu_0 c I_0 \lambda\omega}\over{2\pi}}
           {{\cos(\omega(t-x/c))}\over{x}}\cr
  }
$$
\vfill\eject
\noindent
{\it N{\aa}gra intressanta saker att observera ur l{\"o}sningen}
\smallskip
\noindent
\item{$\bullet$}{De elektriska och magnetiska f{\"a}lten {\"a}r riktade precis
    s{\aa} som vi f{\"o}rv{\"a}ntade oss, med ${\bf E}$ l{\"a}ngs ${\bf e}_z$
    och med ${\bf B}$ l{\"a}ngs ${\bf e}_y$.}
\item{$\bullet$}{Vi kan enkelt verifiera l{\"o}sningarna (utf{\"o}r g{\"a}rna
    denna exercis!) f{\"o}r f{\"a}lten ${\bf E}({\bf x},t)$ och
    ${\bf B}({\bf x},t)$ mot varandra genom att utv{\"a}rdera h{\"o}ger-
    och v{\"a}nsterledet i Faradays lag, $$\nabla\times{\bf E}({\bf x},t)
    =-{{\partial{\bf B}({\bf x},t)}\over{\partial t}}.$$}
\item{$\bullet$}{N{\aa}got mer f{\"o}rv{\aa}nande {\"a}r att det elektriska
    f{\"a}ltet avtar med avst{\aa}ndet fr{\aa}n antennen som ${\bf E}\sim 1/x$,
    medan magnetf{\"a}ltet ${\bf B}$ best{\aa}r av tv{\aa} termer med den ena
    $\sim 1/x$ och den andra $\sim 1/x^2$. Vi m{\aa}ste h{\"a}r dock h{\aa}lla
    i minnet att vi d{\aa} vi n{\"a}rmar oss n{\"a}rf{\"a}ltet, s{\aa}
    g{\"a}ller inte antagandet om en direkt proportionalitet mellan det
    elektriska och magnetiska f{\"a}ltet. Faradays lag, som {\"a}r grunden
    {\"a}ven f{\"o}r denna avvikelse mellan f{\"a}ltens avtagande med $x$ i
    n{\"a}rf{\"a}ltet, g{\"a}ller dock alltid.}
\item{$\bullet$}{Eftersom ${\bf E}$ f{\"o}r $x\gg\lambda$ avtar som $\sim 1/x$,
    s{\aa} avtar f{\"a}ltstyrkan effektivt p{\aa} samma s{\"a}tt som en
    motsvarande {statisk elektrisk monopol}. Att vi vi har avtagandet som
    ${\bf E}\sim 1/x$ {\"a}r faktiskt f{\"o}ruts{\"a}ttningen f{\"o}r
    l{\aa}ngdistanskommunikation med radiofrekvenser! (Man skulle m{\"o}jligen,
    i en analogi med det elektrostatiska fallet, annars kunna f{\"o}rv{\"a}nta
    sig att f{\"a}ltet spreds som fr{\aa}n en elektrostatisk dipol, som
    ${\bf E}\sim 1/x^2$, men detta {\"a}r allts{\aa} felaktigt.)}
\item{$\bullet$}{Allts{\aa}: Det {\it elektrodynamiska} (tidsberoende)
    f{\"a}lten skiljer sig radikalt fr{\aa}n de {\it elektrostatiska}
    (tidsoberoende)!}
\bye
