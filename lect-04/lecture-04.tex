%
% File: teach/elmagii/lect-04/lecture-04.tex [plain TeX code]
% Github: https://github.com/elmagii/lect-04/
% Last change: November 10, 2025
%
% Lecture No 4 in the course ``Elektromagnetism II, 1TE626 (2025)'',
% held November 11, 2025, at Uppsala University, Sweden.
%
% Copyright (C) 2022-2025, Fredrik Jonsson, under Gnu General Public
% License (GPL) v3. See the enclosed LICENSE for details.
%
% This program is free software: you can redistribute it and/or modify
% it under the terms of the GNU General Public License as published by
% the Free Software Foundation, either version 3 of the License, or
% (at your option) any later version.
%
% This program is distributed in the hope that it will be useful,
% but WITHOUT ANY WARRANTY; without even the implied warranty of
% MERCHANTABILITY or FITNESS FOR A PARTICULAR PURPOSE.  See the
% GNU General Public License for more details.
%
% You should have received a copy of the GNU General Public License
% along with this program.  If not, see <https://www.gnu.org/licenses/>.
%
\def\coursename{Elektromagnetism II}
\def\coursecode{1TE626}
\def\courseyear{2025}
\def\courserepo{https://github.com/hp35/elmagii/}
\def\lecturenumber{4}
\def\lecturetitle{Magnetostatik}
\def\lecturetitlesub{}
\def\lectureauthor{Fredrik Jonsson}
\def\lectureplace{Uppsala Universitet}
\def\lecturedate{11 november 2025}
%-------------------- BEGIN OF LOCAL MACROS --------------------
\input macros/epsf.tex
\input macros/eplain.tex
\input amssym % to get the {\Bbb E} font (strikethrough E)
\font\ninerm=cmr9
\font\tenssbx=cmssbx10
\font\twelvesc=cmcsc10 at 12 truept

\def\ifempty#1{\ifx\relax#1\relax}
\def\initlecture{
  \hsize=150mm\hoffset=4.6mm\vsize=230mm\voffset=7mm
  \topskip=0pt\baselineskip=12pt\parskip=0pt\leftskip=0pt\parindent=15pt
  \headline={\ifnum\pageno>1\ifodd\pageno\rightheadline\else\leftheadline\fi
    \else\hfill\fi}
  \def\rightheadline{\tenrm{\it F\"orel\"asning \lecturenumber}
    \hfil{\it \coursename, \coursecode\ (\courseyear)}}
  \def\leftheadline{\tenrm{\it \coursename, \coursecode\ (\courseyear)}
    \hfil{\it F\"orel\"asning \lecturenumber}}
  \noindent~\vskip-60pt\hskip-40pt{\epsfbox{macros/UU_logo_color.eps}}
  \vskip-42pt\hfill\vbox{
      \hbox{{\it \coursename, \coursecode\ (\courseyear)}}
      \hbox{{\it Lecture Notes, \lectureauthor}}
      \hbox{{\it Document Revision \today}}
      \hbox{{\it \courserepo}}}\vskip 36pt
  \centerline{\twelvesc F\"orel\"asning \lecturenumber}
  \vskip 24pt\noindent
  \centerline{\twelvesc\lecturetitle}
  \expandafter\ifempty\expandafter{\lecturetitlesub}%
    \else\centerline{\twelvesc\lecturetitlesub}\fi
  \bigskip
  \centerline{\lectureauthor, \lectureplace, \lecturedate}
  \vskip24pt}
\def\section #1 {\medskip\goodbreak\noindent{\tenssbx #1}
  \par\nobreak\smallskip\noindent}
\def\subsection #1 {\medskip\goodbreak\noindent{\it #1}
  \par\nobreak\smallskip\noindent}
\def\iint{\mathop{\int\kern-8pt\int}}
\def\iiint{\mathop{\int\kern-8pt\int\kern-8pt\int}}
\def\oiint{\mathop{\int\kern-8pt\int\kern-13.2pt{\bigcirc}}}
\def\sgn{\mathop{\rm sgn}\nolimits} % sign
\def\Re{\mathop{\rm Re}\nolimits}   % real part
\def\Im{\mathop{\rm Im}\nolimits}   % imaginary part
\def\Tr{\mathop{\rm Tr}\nolimits}   % quantum mechanical trace
\def\eqq{\mathop{\vbox{\hbox{\hskip2pt?}\vskip-6pt\hbox{=}}}}
\def\quote#1{\par\leftskip=36pt\rightskip=36pt\smallskip\noindent#1\par
  \leftskip=0pt\rightskip=0pt\smallskip}
\long\def\plan#1{\par\leftskip=36pt\rightskip=36pt\bigskip
  \noindent{\it Sammanfattning}\smallskip
  \noindent{\it #1}\par\leftskip=0pt\rightskip=0pt}
\def\threepointsummary#1#2#3{\par\leftskip=36pt\rightskip=36pt\bigskip
  \noindent{\it Tre h{\aa}llpunkter i f{\"o}rel{\"a}sningen}\smallskip
  \leftskip=48pt\rightskip=36pt\hangindent=20pt
  \noindent{\it\hbox to 20pt{1. }#1}\smallskip
  \leftskip=48pt\rightskip=36pt\hangindent=20pt
  \noindent{\it\hbox to 20pt{2. }#2}\smallskip
  \leftskip=48pt\rightskip=36pt\hangindent=20pt
  \noindent{\it\hbox to 20pt{3. }#3}\par
  \leftskip=0pt\rightskip=0pt\vfill\eject}
\def\epsfig#1{\bigskip\centerline{\epsfbox{#1}}\medskip}
\def\captionwide{\advance\leftskip by 60pt
  \advance\rightskip by 60pt}
\newif\ifshowindex
\showindextrue  % Use \showindextrue and \showindexfalse to enable/disable index
\def\index{\ifshowindex\vfill\eject\section{Index} \readindexfile{i}\fi}
%--------------------- END OF LOCAL MACROS ---------------------

\initlecture

\plan{Med en kort sammanfattning av historiken bakom uppt{\"a}ckandet av
  magnetiska f{\"a}lt g{\aa}r vi in p{\aa} sj{\"a}lva definitionen av ett
  magnetiskt f{\"a}lt som den kraft som  via Lorentz kraftlag ut{\"o}vas
  p{\aa} en laddad partikel i r{\"o}relse. Utifr{\aa}n denna formuleras
  Amp\`eres kraftlag f{\"o}r str{\"o}mslingor, samt att vi kan dra slutsatsen
  att kraften p{\aa} fria laddningar aldrig utf{\"o}r n{\aa}got arbete.

  D{\aa} vi generaliserar str{\"o}mbegreppet till en str{\"o}mt{\"a}thet
  ${\bf J}$ kan vi under anv{\"a}n\-dande av Gauss lag ta fram
  kontinuitets\-ekvationen f{\"o}r laddning, $\nabla\cdot{\bf J}=-d\rho/dt$,
  som l{\"a}nkar ihop divergensen hos str{\"o}m\-t{\"a}theten med
  tidsderivatan av laddnings\-t{\"a}t\-heten. Biot--Savarts lag introduceras
  som ett axiom f{\"o}r det magnetf{\"a}lt som genereras av en str{\"o}m
  traverserande en str{\"o}mslinga, vilket f{\"o}r {\"o}vrigt {\"a}r
  f{\"o}rsta momentet d{\"a}r den magnetiska permea\-bili\-teten $\mu_0$
  introduceras.
  Utifr{\aa}n formen p{\aa} Biot--Savarts lag f{\"o}r generering av
  magnetf{\"a}lt visar vi att $\nabla\cdot{\bf B}=0$ alltid {\"a}r uppfyllt,
  vilket p{\aa}visar att magnetiska monopoler (magnetisk laddning) ej existerar,
  och att magnetism alltid endast yttrar sig i form av magnetiska dipoler eller
  h{\"o}gre ordningar i multipolutvecklingen.
  Vi visar att i magnetostatik kan rotationen av magnetf{\"a}ltet erh{\aa}llas
  som $\nabla\times{\bf B}=\mu_0{\bf J}$, kallad Amp\`eres lag.

  Slutligen visar vi p{\aa} att icke-existensen av magnetiska monopoler
  direkt har som f{\"o}ljd att vi kan tolka det magnetiska f{\"a}ltet
  som h{\"a}rr{\"o}rande fr{\aa}n en vektorpotential ${\bf A}$, i analogi
  med den skal{\"a}ra potentialen $\phi$ inom elektrostatik, som
  ${\bf B}=\nabla\times{\bf A}$.
  Amp\`eres lag kan tolkas i termer av denna vektorpotential som Poissons
  ekvation $\nabla^2{\bf A}=-\mu_0{\bf J}$ med str{\"o}mt{\"a}theten som
  k{\"a}llterm.}

\threepointsummary{%
  Lagen om att elektrisk laddning inte kan f{\"o}rsvinna,
  $$
    \nabla\cdot{\bf J}=-{{d\rho}\over{dt}}.
  $$
}{%
  Ur ``lagen om att inga magnetiska monopoler existerar''
  kan vi direkt formulera vektorpotentialen ${\bf A}$ som
  $\nabla\cdot{\bf B}=0\ \Leftrightarrow\ {\bf B}=\nabla\times{\bf A}$.
}{%
  Amp\`eres lag f{\"o}r den magnetostatiska vektorpotentialen ${\bf A}$ ges
  som Poissons ekvation med den fria str{\"o}mt{\"a}theten ${\bf J}$ som
  k{\"a}llterm,
  $$
    \nabla^2{\bf A}=-\mu_0{\bf J},
  $$
  med den explicita l{\"o}sningen
  $$
    {\bf A}({\bf x})={{\mu_0}\over{4\pi}}\iiint_V
      {{{\bf J}({\bf x}')}\over{|{\bf x}-{\bf x}'|}}\,dV'.
  $$
}
%----------------------- END OF PREAMBLE -----------------------

\section{Introduktion}
\sidx{Elektrostatik}\sidx{Magnetostatik}\sidx{Elektron}\sidx{Str{\"o}m}
Vi kommer i denna f{\"o}rel{\"a}sning att behandla konstanta str{\"o}mmar, i
vilka laddningar (l{\"a}s: elektroner) r{\"o}r sig l{\"a}ngs givna f{\"a}lt
eller trajektorior som i sig {\"a}r konstanta i rum och tid.\numberedfootnote{Vi
  kommer i denna f{\"o}rel{\"a}sning att i huvudsak f{\"o}lja Griffiths
  kapitel~5, sid.~210--.}
Vi kan som en rekapitulation fr{\aa}n F{\"o}rel{\"a}sning~1 sammanfatta
begreppen elektrostatik och magnetostatik enligt f{\"o}ljande:
\medskip
\item{$\bullet$}{Station{\"a}ra laddningar $\Rightarrow$
  Konstanta elektriska f{\"a}lt (elektrostatik)}
\item{$\bullet$}{Konstanta str{\"o}mmar $\Rightarrow$
  Konstanta magnetiska f{\"a}lt (magnetostatik)}
\medskip
\noindent
Magnetostatik {\"a}r studiet av magnetf{\"a}lt i system d{\"a}r n{\"a}rvarande
elektriska str{\"o}mmar {\"a}r konstanta. Detta {\"a}r den magnetiska analogin
med elektrostatik, d{\"a}r ist{\"a}llet de elektriska laddningarna {\"a}r
station{\"a}ra och fixa i tid och rum.
Att vi h{\"a}r har att g{\"o}ra med statiska magnetf{\"a}lt betyder inte att
teorin inte g{\aa}r att applicera p{\aa} tidsberoende problem, bara att det
m{\aa}ste g{\"o}ras under f{\"o}ruts{\"a}ttningen att f{\"o}rloppen {\"a}r
l{\aa}ngsamma nog att f{\"o}r att den elektromagnetiska v{\aa}gl{\"a}ngden
\sidx{Elektromagnetisk v{\aa}gl{\"a}ngd} i problemet rej{\"a}lt {\"o}verstiger
storleken p{\aa} dom{\"a}nen som analyseras.
Till exempel kan vanliga elektriska generatorer och motorer, exempelvis
startmotorn i en bil eller generatorn i ett vindkraftverk, med mycket god
approximation behandlas som just magnetostatiska problem, trots den uppenbara
r{\"o}relsen och tidsberoendet. Nyckeln {\"a}r att det {\"a}r de {\it olika
tidsskalorna} mellan den mekaniska r{\"o}relsen och f{\"o}r utbredningen av
elektromagnetiska v{\aa}gor som avg{\"o}r om vi kan betrakta problemet som
magnetostatiskt eller ej.

\section{Historik}
Magnetostatiken kan s{\"a}gas ha uppt{\"a}ckts 1269 av fransmannen Petrus
Peregrinus de Maricourt,\sidx{de Maricourt, Petrus Peregrinus} som
unders{\"o}kte det magnetiska f{\"a}ltet p{\aa} ytan av en sf{\"a}risk
magnet\sidx{Permanentmagnet} med n{\aa}lar av j{\"a}rn. Han noterade att de
resulterande f{\"a}ltlinjerna som n{\aa}larna beskrev korsades p{\aa} tv{\aa}
p{\aa} sf{\"a}ren motsatta punkter, som han betecknade som ``poler''
\sidx{Magnetiska poler} i analogi
med jordens poler.\numberedfootnote{Intressant nog, som en litet sidosp{\aa}r
  till Maricourts observationer kring detta med poler, s{\aa} formulerade
  grekiska filosofer som \idx{Empedocles (494--434~BC)} och \idx{Anaxagoras
  (500--428~BC)} redan kring 500~BC hypotesen att jorden sannolikt var rund,
  utifr{\aa}n den runda skugga som jorden gav p{\aa} m{\aa}nen vid en
  m{\aa}nf{\"o}rm{\"o}rkelse.
  Kring 350~BC bistod \idx{Aristoteles (384--322~BC)} med observationen att
  d{\aa} skepp som seglar iv{\"a}g f{\"o}rsvinner skrovet f{\"o}rst ur sikte,
  f{\"o}re masten, och att detta pekar p{\aa} att jorden har en rund form.
  F{\"o}rst 1522~AD fick vi dock ``h{\aa}rt bevis'' p{\aa} att jorden
  {\"a}r rund genom Magellan--Elcanos expedition som genomf{\"o}rde den
  f{\"o}rsta v{\"a}rldsomseglingen och d{\"a}rmed en g{\aa}ng f{\"o}r
  alla bevisade att vi kan resa hela v{\"a}gen runt jordklotet.}

Maricourt formulerade ocks{\aa} den synnerligen intressanta observationen att
{\it oavsett hur fint vi skivar en magnet, s{\aa} har den alltid en nord- och
sydpol}. Som vi skall se fram{\"o}ver i denna f{\"o}rel{\"a}sningsserie
h{\"a}nger detta intimt samman med att magnetism alltid manifesterar sig
som dipoler (eller h{\"o}gre ordningars multipoler), och {\it aldrig som
magnetisk laddning (monopoler)},\sidx{Magnetisk monopol} detta som en markant
skillnad gentemot elektrisk laddning.\sidx{Elektrisk monopol}

Den som r{\"a}knas som uppt{\"a}ckaren av att elektriska str{\"o}mmar genererar
magnetiska f{\"a}lt {\"a}r Hans Christian {\OE}rsted,\sidx{{\OE}rsted, Hans
Christian (1777--1851)} som 1820 publicerade sin uppt{\"a}ckt att orienteringen
hos en kompassn{\aa}l p{\aa}verkas av en elektrisk str{\"o}m i n{\"a}rheten av
n{\aa}len.
F{\"o}r sin uppt{\"a}ckt bel{\"o}nades {\OE}rsted av {\it The Royal Society} i
England med Copley-medaljen, samt att den \idx{Franska Akademien} bel{\"o}nade
honom med $3\,000$ franc.
{\OE}rsteds uppt{\"a}ckt kan s{\"a}gas vara startskottet f{\"o}r arbetet med
att formulera den moderna elektromagnetismen, speciellt inspirerade detta den
franske fysikern Andr\'e-Marie Amp\`ere\sidx{Amp\`ere, Andr\'e-Marie
(1775--1836)} till att formulera en matematisk formel f{\"o}r att beskriva den
magnetiska kraften mellan str{\"o}mslingor.
\vfill\eject

\section{Vad {\"a}r ett magnetf{\"a}lt?}
\sidx{Magnetf{\"a}lt}\sidx{Magnetisk fl{\"o}dest{\"a}thet}
\sidx{Elektrisk str{\"o}m}
\quote{{\bf Definition.} Vi definierar ett magnetf{\"a}lt\numberedfootnote{I
    denna kurs betecknar vi magnetf{\"a}ltet ocks{\aa} som $B$-f{\"a}lt,
    men i andra sammanhang betecknas f{\"a}ltet {\"a}ven som $H$-f{\"a}lt,
    beroende vilken ing{\aa}ng och historisk konvention man r{\aa}kar ha.
    Den korrekta svenska termen f{\"o}r $B$-f{\"a}ltet {\"a}r egentligen
    {\it magnetisk fl{\"o}dest{\"a}thet}.}
  som ett {\it fysiskt f{\"a}lt som beskriver magnetisk p{\aa}verkan p{\aa}
  r{\"o}rliga elektriska laddningar, elektriska str{\"o}mmar och magnetiska
  material.}}
\noindent
I n{\aa}gon m{\aa}n kan vi s{\"a}ga att dessa tre m{\"o}jligheter egentligen
kokar ner till en enda sak, n{\"a}mligen p{\aa}verkan av r{\"o}rliga elektriska
laddningar,\sidx{Elektrisk laddning}  eftersom elektrisk str{\"o}m utg{\"o}rs
av r{\"o}rliga laddningar samt att magnetiska material handlar om hur
materialet p{\aa} en mikroskopisk niv{\aa}, eller snarare kvantmekanisk
niv{\aa}, beter sig i linjering av spinn och magnetiska moment som kan ses
som mikroskopiska slutna str{\"o}mslingor.\sidx{Str{\"o}mslinga}[Sluten]
Den kvantmekaniska\sidx{Kvantmekanik} behandlingen av spinn och magnetiska
moment\sidx{Magnetiskt dipolmoment} ligger dock utanf{\"o}r omfattningen av
denna kurs.

\section{Lorentz-kraften - R{\"o}rliga laddningar i statiska elektriska och
  magnetiska f{\"a}lt}
\sidx{Lorentz-kraften}
\epsfig{figs/lorentz.1}\noindent
Problemet med magnetiska f{\"a}lt {\"a}r att det fr{\aa}n ett klassiskt
angreppss{\"a}tt {\"a}r om{\"o}jligt att strikt h{\"a}rleda dem {\it a priori}
fr{\aa}n klassiska elektromagnetiska modeller. Vi kommer h{\"a}r att rent
axiomatiskt konstatera att ett magnetf{\"a}lt ${\bf B}$ {\"a}r det f{\"a}lt
som ger kraften p{\aa} en r{\"o}rlig och elektriskt laddad partikel med
laddningen $q$ och hastigheten ${\bf v}$ som
$$
  {\bf F}=q({\bf v}\times{\bf B}).
$$
Notera h{\"a}r hur vi i likhet med den elektrostatik som vi behandlade
i F{\"o}rel{\"a}sning~1 kan se det magnetiska f{\"a}ltet ${\bf B}$ som
{\it definierat} av den kraft ${\bf F}$ som ut{\"o}vas p{\aa} en laddning,
bara det att denna kraft nu relaterar till laddningens {\it r{\"o}relse}
och inte bara till dess belopp.
Om vi l{\"a}gger till kraften p{\aa} laddningen fr{\aa}n ett statiskt elektriskt
f{\"a}lt, s{\aa} erh{\aa}ller vi {\it Lorentz-kraften}\numberedfootnote{Som,
  liksom Griffiths korrekt p{\aa}pekar, ursprungligen formulerades av
  Oliver Heaviside,\sidx{Heaviside, Oliver (1850--1925)} som senare kom att
  ha en instrumentell del i formulerandet av den moderna formen av
  {\it Maxwell's ekvationer}\sidx{Maxwells ekvationer} s{\aa} som vi idag
  k{\"a}nner dem. F{\"o}r en intressant sammanfattning
  av Heavisides arbete med att sl{\aa} samman de fr{\aa}n b{\"o}rjan tjugo
  ekvationerna beskrivande elektrodynamik till de fyra som utg{\"o}r den
  moderna formen av Maxwells ekvationer, se exempelvis Damian P. Hampshire,
  {\it A derivation of Maxwell's equations using the Heaviside notation},
  Phil. Trans. Royal Society A {\bf 376}, 20170447 (2017).
  {\tt https://royalsocietypublishing.org/doi/10.1098/rsta.2017.0447}}
$$
  {\bf F}=q\big({\bf E}+{\bf v}\times{\bf B}\big).
$$
{\AA}terigen, vi h{\"a}vdar h{\"a}r inte att vi i denna kurs p{\aa} n{\aa}got
vis kommer att h{\"a}rleda denna relation; vi kommer h{\"a}r ist{\"a}llet att
helt luta oss mot att denna form {\"a}r experimentellt verifierad efter alla
konstens regler och d{\"a}rmed n{\"o}ja oss med det.

\subsection{Magnetisk kraft utf{\"o}r inget arbete}
\sidx{Magnetisk kraft utf{\"o}r inget arbete}
Utifr{\aa}n formen p{\aa} Lorentz-kraften, som {\"a}r ortogonal mot
s{\aa}v{\"a}l det magnetiska f{\"a}ltet ${\bf B}$ som hastigheten ${\bf v}$,
kan vi dra en viktig och generell slutsats:
\quote{{\it Magnetiska krafter utf{\"o}r inget arbete.}}
\noindent
Detta kan spontant tyckas vara mots{\"a}gelsefullt; trots allt vet vi ju att
generatorer och elektriska motorer bygger just p{\aa} magnetf{\"a}lt (och som
vi konstaterat s{\aa} kan vi i dessa fall dessutom behandla de magnetiska
f{\"a}ltproblemen som just {\it magnetostatiska}), s{\aa} hur skulle dessa
inte utf{\"o}ra n{\aa}got arbete?

Argumentet h{\"a}r g{\"a}ller dock att {\it magnetiska krafter} faktiskt inte
utf{\"o}r arbete p{\aa} {\it fri} elektrisk laddning, eftersom om laddningen
$q$ f{\"o}rflyttas en str{\"a}cka
$$
  d{\bf l}={\bf v}dt
  \qquad\Rightarrow\qquad\hbox{Utf{\"o}rt arbete:}\quad
  dW={\bf F}\cdot d{\bf l}
    =q\underbrace{({\bf v}\times{\bf B})}_{\perp{\bf v}}\cdot{\bf v}dt
    =0,
$$
det vill s{\"a}ga att hur vi {\"a}n f{\"o}rflyttar laddningen i ett magnetiskt
f{\"a}lt, s{\aa} kommer den resulterande Lorentz-kraften att vara ortogonal mot
f{\"o}rflyttningen och det resulterande {\it magnetiska} arbetet kommer att
vara noll.

S{\aa} hur kan generatorer och elektriska motorer fungera om den magnetiska
kraften inte utf{\"o}r n{\aa}got arbete? L{\"o}sningen till denna paradox
{\"a}r att ovanst{\aa}ende argument h{\aa}ller f{\"o}r en {\it fri} laddning
som {\it inte {\"a}r l{\aa}st till en viss trajektoria}.
F{\"o}r en fri laddning kommer den ortogonala kraften att kontinuerligt
{\"a}ndra riktningen p{\aa} laddningen, typiskt resulterande i cirkul{\"a}ra
eller helixformade banor\numberedfootnote{Vilket exempelvis {\"a}r vad som
  h{\"a}nder med de laddade partiklar som n{\"a}r de infaller i jordens
  magnetf{\"a}lt resulterar i h{\"o}gfrekventa helix-formade trajektorior
  som genererar synligt ljus, s{\aa} kallat {\it norrsken}.}
Om vi exempelvis har ett fl{\"o}de av elektroner i en str{\"o}mslinga, s{\aa}
{\"a}r dessa l{\aa}sta i sin r{\"o}relse, och genom att de inte kan l{\"a}mna
ledaren bidrar de kollektivt till att ut{\"o}va en kraft p{\aa} ledaren.
Denna kraft p{\aa} r{\"o}relse av laddningar som genom str{\"o}mslingor {\"a}r
{\it begr{\"a}nsade} i sin r{\"o}relse utf{\"o}r sj{\"a}lvfallet arbete.

\section{Amp\`eres kraftlag - Kraften p{\aa} str{\"o}mslingor i magnetf{\"a}lt}
\sidx{Amp\`eres kraftlag}
Vi skall nu g{\aa} in p{\aa} hur krafter verkar p{\aa} laddningar som
transporteras i f{\"o}rutbest{\"a}mda banor, det vill s{\"a}ga elektrisk
str{\"o}m i {\it str{\"o}mslingor}.\sidx{Str{\"o}mslinga}
\quote{{\bf Definition:} Vi definierar str{\"o}m \sidx{Str{\"o}m} som den
  laddning, vanligtvis elektroner, som transporteras genom en givet
  tv{\"a}rsnitt per tidsenhet.}
\noindent
Kort och gott kommer vi i praktiken att definiera str{\"o}m som det antal
Coulomb som passerar en ledares tv{\"a}rsnitt per sekund, som
\sidx{Amp\`ere, enhet f{\"o}r str{\"o}m}\sidx{Coulomb, enhet f{\"o}r laddning}
$$
  1~\hbox{A} = 1~\hbox{C}/\hbox{s}.
$$
Antag att vi har en linjeladdning $\lambda$ (${\rm C}/{\rm m}$), det vill
s{\"a}ga en viss laddning $q$ utsmetad l{\"a}ngs en viss str{\"a}cka, och att
denna linjeladdning r{\"o}r sig l{\"a}ngs en fix trajektoria (ledare) $\Gamma$
i rummet med farten $v$. Under ett {\"o}gonblick $\Delta t$ r{\"o}r sig med
andra ord denna linjeladdning en str{\"a}cka $v\Delta t$ l{\"a}ngs trajektorian.
I ett tv{\"a}rsnitt av ledaren har vi d{\"a}rmed str{\"o}mmen
$$
  I={{\hbox{(Laddning)}}\over{\hbox{(tid)}}}
   ={{\lambda v \Delta t}\over{\Delta t}}
   =\lambda v.
$$
\epsfig{figs/ampereforce.1}\noindent
Den magnetiska kraften p{\aa} en str{\"o}mslinga l{\"a}ngs en trajektoria
fr{\aa}n ${\bf x}_a$ till ${\bf x}_b$ b{\"a}randes denna str{\"o}m, erh{\aa}lls
d{\"a}rmed genom att summera upp alla delbidrag fr{\aa}n de infinitesimala
laddningarna i r{\"o}relse enligt\numberedfootnote{Griffiths g{\aa}r
  i Ekv.~(5.16), sid.~217, vidare med denna form och konstaterar att
  str{\"o}mmen $I$ {\"o}verallt l{\"a}ngs str{\"o}mslingan {\"a}r
  riktad l{\"a}ngs linjelementen $d{\bf l}$, och att vi d{\"a}rmed
  f{\"o}r en konstant str{\"o}m $I$ l{\"a}ngs str{\"o}mslingan kan
  skriva om detta som
  $$
    {\bf F}_{\rm mag}=I\int^{{\bf x}_b}_{{\bf x}_a} (d{\bf l}\times{\bf B}).
  $$
  Denna form {\"a}r f{\"o}r v{\aa}ra {\"a}ndam{\aa}l dock lite
  f{\"o}rvirrande i notationen, d{\aa} det {\it linjeelement} $d{\bf l}$
  som ing{\aa}r i kryssprodukten {\"a}r f{\"o}rvillande likt ett
  {\it str{\"o}melement} $d{\bf I}$. Vi f{\"o}rs{\"o}ker h{\"a}r
  d{\"a}rf{\"o}r att i m{\"o}jligaste m{\aa}n undvika denna form.}
$$
  \eqalign{
    {\bf F}_{\rm mag}
      &=\lim_{\Delta l\to0}\sum_k
          \underbrace{
          ({\bf v}_k\times{\bf B}_k)
          \underbrace{\lambda\Delta l}_{=\Delta q}
          }_{=\Delta{\bf F}_{\rm mag}}\cr
      &=\int^{{\bf x}_b}_{{\bf x}_a} ({\bf v}\times{\bf B})\lambda\,dl\cr
      &=\big\{\hbox{ Str{\"o}mmen {\"a}r en vektor,
                     ${\bf I}=\lambda{\bf v}$ }\big\}\cr
      &=\int^{{\bf x}_b}_{{\bf x}_a} ({\bf I}\times{\bf B})\,dl.\cr
  }
$$
Vi brukar beteckna detta som {\it Amp\`eres kraftlag} f{\"o}r str{\"o}mslingor,
vilket inte skall f{\"o}rv{\"a}xlas med {\it Amp\`eres lag} som vi strax skall
h{\"a}rleda, och som beskriver hur magnetiska f{\"a}ltet i sig kopplar till en
str{\"o}mt{\"a}thet.\sidx{Amp\`eres kraftlag}\sidx{Str{\"o}mt{\"a}thet}
\sidx{Str{\"o}mslinga}
\vfill\eject

\section{Volymstr{\"o}mmar och ``lagen om att laddning inte kan f{\"o}rsvinna''}
\sidx{Str{\"o}mt{\"a}thet}\sidx{Lagen om att laddning inte kan f{\"o}rsvinna}
Enligt definitionen som vi h{\"a}r anv{\"a}nder f{\"o}r str{\"o}mmen $I$, s{\aa}
{\"a}r denna definierad som den laddning som per tidsenhet passerar {\it genom
ett givet tv{\"a}rsnitt}. Vi kan formulera detta som att vi genom en yta $S$,
mad randen i form av en (sluten) trajektoria $\Gamma$, har str{\"o}mmen given
i termer av en {\it str{\"o}mt{\"a}thet} ${\bf J}$, med den senare m{\"a}tt i
den laddning som transporteras per ytenhet och per tidsenhet, eller
${\rm C}/({\rm m}^2{\rm s})$,
$$
  I=\iint_S {\bf J}\cdot d{\bf S}.
$$
Vi kan sj{\"a}lvfallet utveckla detta till att g{\"a}lla den totala str{\"o}m
som passerar genom en {\it sluten} yta $S$ som omsluter en volym $V$, genom att
anv{\"a}nda Gauss lag\numberedfootnote{Se exempelvis innerp{\"a}rmen p{\aa}
  Griffiths, {\it Divergence theorem}:\sidx{Gauss lag}
  $$
    \int(\nabla\cdot{\bf A})\,dV=\oiint{\bf A}\,d{\bf S}.
  $$
  {\AA}terigen, notera att Griffiths olyckligtvis anv{\"a}nder den udda
  och vilseledande notationen $d\tau$ f{\"o}r volymelement. Normalt
  anv{\"a}nder vi $\tau$ som integrationsvariabel i {\it tid}. F{\"o}r
  att inte f{\"o}rvirra oss ytterligare v{\"a}ljer vi dessutom att
  anv{\"a}nda notationen $d{\bf S}$ f{\"o}r ytelement (``S'' f{\"o}r
  {\it surface}) inkluderande normalriktning.}
$$
  I=\Big[\hbox{str{\"o}mmen ut genom ytan $S$}\Big]
   =\oiint_S {\bf J}\cdot d{\bf S}
   =\iiint_V (\nabla\cdot{\bf J})\,dV.
$$
\sidx{Elektrisk laddning}
Eftersom ingen laddning kan skapas eller f{\"o}rintas internt i volymen (vi
erinrar oss att all laddnings\-transport in eller ut fr{\aa}n volymen sker
genom den slutna ytan $S$, s{\aa} m{\aa}ste den laddning som fl{\"o}dar
{\it ut genom ytan} g{\"o}ra att den i volymen $V$ {\it inneslutna laddningen
minskar} i motsvarande grad, det vill s{\"a}ga
$$
  \iiint_V (\nabla\cdot{\bf J})\,dV
    =-{{d}\over{dt}}\Big[\hbox{Innesluten laddning}\Big]
    =-{{d}\over{dt}}\iiint_V \rho\,dV
    =-\iiint_V {{d\rho}\over{dt}}\,dV
$$
Fl{\"o}det av laddning i den slutna ytintegralen definieras som fl{\"o}det
{\it ut genom ytan}, och vi kan som en liten {\it sanity check} konstatera att
minustecknet i h{\"o}gerledet d{\"a}rmed betyder att positiv laddning som
fl{\"o}dar {\it ut genom ytan} motsvaras av en motsvarande {\it minskning av
positiv laddning i volymen} $V$, helt enligt f{\"o}rv{\"a}ntan.

Eftersom detta argument kring fl{\"o}de av laddning ut genom en sluten yta
{\"a}r giltigt f{\"o}r en {\it godtycklig} volym $V$, s{\aa} betyder detta att
$$
  \nabla\cdot{\bf J}=-{{d\rho}\over{dt}},
$$
vilket vi betecknar\numberedfootnote{Det m{\aa}ste medges att detta {\"a}r
  en av de trixigare termerna att uttrycka p{\aa} svenska, d{\aa} man
  g{\"a}rna vill uttrycka detta som ``konservering av laddning'', vilket
  leder tanken till inl{\"a}ggningar av sill och frukt, eller ``bevarande
  av laddning'', vilket ist{\"a}llet har en air av bevarande av n{\aa}gon
  kulturhistorisk artefakt. Vi h{\aa}ller oss h{\"a}r till det tydliga om
  {\"a}n lite klumpigare ``lagen om att laddning inte kan f{\"o}rsvinna''.}
som {\it lagen om att laddning inte kan f{\"o}rsvinna}.\numberedfootnote{{\it
  Continuity Equation}; se Griffiths Ekv.~(5.29), sid.~222 samt
  Griffiths Ekv.~(8.4), sid.~356.}

\section{Aprop{\aa} detta med magnetostatik vs elektrostatik}
\sidx{Magnetostatik}\sidx{Elektrostatik}
L{\aa}t oss g{\"o}ra en liten utvikning kring detta med elektrostatik och
magnetostatik, och konstatera att vi i dessa {\it statiska} problem formellt
har att laddningst{\"a}theten $\rho$ och str{\"o}mt{\"a}theten ${\bf J}$
{\"a}r tidsoberoende {\"o}verallt i problemet, eller
$$
  {{d\rho}\over{dt}}=0
  \qquad\underline{\hbox{och}}\qquad
  {{d{\bf J}}\over{dt}}=0.
$$
{\AA}terigen, s{\aa} kan vi dock med god precision betrakta alla problem som
har en s{\aa} pass l{\aa}g associerad frekvens att den elektromagnetiska
v{\aa}gl{\"a}ngden $\lambda=c/f$ vida {\"o}verstiger problemets spatiala
utstr{\"a}ckning\numberedfootnote{Fina ord: ``spatial'' = ``i rummet'',
  ``temporal'' = ``i tiden'', ``spatiotemporal'' = ``i rumtid''.}
som just statiska problem. Exempel p{\aa} saker som vi inte kan behandla som
statiska {\"a}r typiskt radioantenner (som per definition har en
utstr{\"a}ckning i storleksordningen av en halv v{\aa}gl{\"a}ngd av den
elektromagnetiska str{\aa}lning som skall f{\aa}ngas upp eller skickas ut)
eller elektronisk apparatur i GHz-omr{\aa}det ($\lambda\sim0.3\ {\rm m}$)
och upp{\aa}t.

Ett annat s{\"a}tt att se p{\aa} statiska problem {\"a}r till exempel att vi
inte till{\aa}ter str{\"o}mmen $I$ att variera l{\"a}ngs en str{\"o}mslinga,
eftersom det ju direkt skulle betyda att vi l{\"a}ngs slingan ackumulerar
elektrisk laddning n{\aa}gonstans. Eftersom vi i magnetostatiken (enligt
observationen ovan) dessutom kr{\"a}ver att laddningst{\"a}theten $\rho$
{\"a}r konstant i tiden, s{\aa} {\"a}r divergensen av str{\"o}mt{\"a}theten
noll i statiska problem,
$$
  {{d\rho}\over{dt}}=0\qquad\Rightarrow\qquad\nabla\cdot{\bf J}=0.
$$
En tolkning av divergensen $\nabla\cdot{\bf J}=0$ {\"a}r att vi i statiska
problem (inom gr{\"a}nsen av en giltighet f{\"o}r statik som vi nyss
konstaterat {\"a}r t{\"a}mligen vid) i strikt mening inte till{\aa}ter
laddning att ackumuleras n{\aa}gonstans i problemet.

Vi n{\"a}rmar oss nu pudelns k{\"a}rna i problemet med att f{\aa} fram hur
magnetiska f{\"a}lt genereras av str{\"o}mmar.
Vi kunde i elektrostatiken se att Coulombs lag f{\"o}r v{\"a}xelverkan mellan
statiska punktladdningar\numberedfootnote{En lag f{\"o}r v{\"a}xelverkan som,
  {\it nota bene}, vi helt sonika har stadsf{\"a}st som varandes en
  fundamentalt giltig beskrivning mellan statiska laddningar utan att
  vi f{\"o}r den skull ens skissat p{\aa} ett bevis f{\"o}r den!}
via superpositionsprincipen kunde generaliseras till kontinuerliga elektriska
laddningsf{\"o}rdelningar, och att vi utifr{\aa}n dessa kunde visa p{\aa}
existensen av en skal{\"a}r, elektrostatisk potential $\phi$ definierad av
${\bf E}=-\nabla\phi$. S{\"a}kerligen kan vi nu dra fram hur {\it r{\"o}relsen}
hos en punktladdning genererar n{\aa}gon sorts ``svallv{\aa}gor'', som vi i
analogi med elektrostatiken kan generalisera och analysera f{\"o}r
str{\"o}mslingor med ett kontinuum av laddning. Eller?
\vfill\eject

\section{Biot--Savarts lag - Magnetf{\"a}lt fr{\aa}n str{\"o}mslingor}
\subsection{Historiken f{\"o}r Biot--Savarts lag}
\sidx{Biot--Savarts lag}
Efter att Hans Christian {\OE}rsted {\aa}r 1820 hade gjort sin banbrytande
uppt{\"a}ckt att en magnetn{\aa}l p{\aa}verkas av elektriska str{\"o}mmar, tog
de franska fysikerna Jean-Baptiste Biot (1774--1862) \sidx{Biot, Jean-Baptiste
(1774--1862)} och F\'elix Savart (1791--1841) \sidx{Savart, F\'elix
(1791--1841)} (b{\aa}da i Paris) samma {\aa}r upp f{\"o}rs{\"o}k med att
fysikaliskt m{\"a}ta hur stort det genererade magnetf{\"a}ltet var och vilka
lagar som kunde t{\"a}nkas styra det.\numberedfootnote{Herman Erlichson,
  {\it The experiments of Biot and Savart concerning the force exerted by
  a current on a magnetic needle}, Am.~J.~Phys. {\bf 66} (1998).}
I sina experiment sp{\"a}nde Biot och Savart upp en l{\aa}ng ledande tr{\aa}d
genom vilken de kunde leda en elektrisk str{\"o}m vertikalt och upph{\"a}ngde
vid sidan av tr{\aa}dens mitt en liten horisontell magnetn{\aa}l, som skyddades
mot luftstr{\"o}mmar av ett glasomh{\"o}lje.
F{\"o}r att s{\aa} mycket som m{\"o}jligt undg{\aa} p{\aa}verkan fr{\aa}n
jordens magnetf{\"a}lt p{\aa} experimentet anv{\"a}nde de en v{\"a}xlande
str{\"o}m genom tr{\aa}den, och kunde p{\aa} s{\aa} s{\"a}tt anv{\"a}nda
amplituden p{\aa} n{\aa}lens r{\"o}relse som m{\aa}tt p{\aa} styrkan hos
magnetf{\"a}ltet fr{\aa}n str{\"o}mb{\"a}rande tr{\aa}den.

\subsection{Sv{\aa}righeten med att formellt h{\"a}rleda Biot--Savarts lag}
Griffiths pekar i sin {\it Introduction to Electrodynamics} p{\aa} ett mycket
m{\aa}lande s{\"a}tt hur han sj{\"a}lv som f{\"o}rfattare {\"a}r mycket
frustrerad {\"o}var att ingen enkel modell kan g{\"o}ras f{\"o}r
magnetf{\"a}ltet fr{\aa}n en punktladdning i r{\"o}relse utan att ta till ett
maskineri som ligger l{\aa}ngt utanf{\"o}r omfattningen av hans bok och f{\"o}r
den delen denna kurs.
Specifikt s{\aa} pekar han p{\aa} det faktum att {\it r{\"o}relsen hos en
enskild punktladdning inte rimligen kan tolkas som en str{\"o}m}, d{\aa} den
ena tidpunkten finns p{\aa} plats, medan den {\"o}gonblicket efter{\aa}t inte
finns d{\"a}r.
En s{\aa}dan r{\"o}relse {\"a}r snarast att likna vid en diskret h{\"a}ndelse
inom den klassiska elektrodynamiken, och antagandet om station{\"a}r str{\"o}m
med $d{\bf J}/dt={\bf 0}$ {\"a}r garanterat inte uppfyllt.

Vi {\"a}r med andra ord redan fr{\aa}n b{\"o}rjan tvingade att hantera ett
{\it kontinuum av laddning i r{\"o}relse} f{\"o}r att beskriva hur en str{\"o}m
genererar ett magnetf{\"a}lt, och argumenten f{\"o}r varf{\"o}r Biot--Savarts
lag ser ut som den g{\"o}r blir d{\"a}rmed mycket st{\"o}kigare {\"a}n vad vi
fr{\aa}n en b{\"o}rjan kan f{\"o}rv{\"a}nta oss.
Med detta i bagaget kommer vi nu att g{\aa} in p{\aa} hur station{\"a}ra
str{\"o}mmar och str{\"o}mt{\"a}theter ger upphov till statiska magnetiska
f{\"a}lt.\numberedfootnote{Griffiths v{\"a}ljer redan i ett tidigt stadium
  att direkt fastst{\"a}lla Biot--Savarts kompletta lag p{\aa} integralform.
  Vi v{\"a}ljer h{\"a}r att f{\"o}rst ta ett litet steg i och med
  betraktandet av ett litet {\it linjeelement} l{\"a}ngs med
  str{\"o}mslingan.}

\subsection{Biot--Savarts lag f{\"o}r str{\"o}mslingor}
\sidx{Biot--Savarts lag}
\epsfig{figs/biotsavart.1}\noindent
Om vi betraktar ett linjeelement $d{\bf l}'$ vid k{\"a}llpunkten ${\bf x}'$ med
beloppet $|d{\bf l}'|=dl'$ l{\"a}ngs en str{\"o}mslinga uppb{\"a}rande
str{\"o}mmen ${\bf I}({\bf x}')$ med beloppet $|{\bf I}({\bf x}')|=I$, liksom
tidigare med konventionen att vi s{\"a}tter ett prim p{\aa} vad vi betraktar
som k{\"a}lla i problemet, s{\aa} ger detta (linj{\"a}ra) linjeelement
bidraget\numberedfootnote{Notera {\aa}terigen att vi h{\"a}r l{\"a}tt
  riskerar att f{\"o}rv{\"a}xla {\it linjeelementet} $d{\bf l}'$ (som
  har den fysikaliska dimensionen {\it l{\"a}ngd}) med ett delbidrag
  till den riktade str{\"o}mmen.}{$^{,}$}%
  \numberedfootnote{Det finns ett par underh{\aa}llande
    mini-f{\"o}rel{\"a}sningar p{\aa} YouTube kring hur vi kan tolka kraften
    p{\aa} en laddning i r{\"o}relse n{\"a}ra en str{\"o}mb{\"a}rande ledare,
    det vill s{\"a}ga ett magnetf{\"a}lt genererat enligt Biot--Savarts lag
    s{\aa} som vi h{\"a}r presenterar den, som en direkt effekt av den speciella
    relativitetsteorin applicerad p{\aa} en reservoar av laddning.
    Se till exempel Veritasium, {\it How Special Relativity Makes Magnets Work},
    {\tt https://www.youtube.com/watch?v=1TKSfAkWWN0}, eller
    Fermilab Lectures, {\it How Einstein saved magnet theory},
    {\tt https://www.youtube.com/watch?v=d29cETVUk-0}}
$$
  d{\bf B}({\bf x})={{\mu_0}\over{4\pi}}
    {{{\bf I}({\bf x}')\times({\bf x}-{\bf x}')}
      \over{|{\bf x}-{\bf x}'|^3}}\,dl'
$$
till det magnetiska f{\"a}ltet vid observationspunkten ${\bf x}$. Notera
f{\"o}rekomsten av\numberedfootnote{Detta fixerade v{\"a}rde f{\"o}r den
  magnetiska permeabiliteten definierar i SI enheten Amp\`ere, vilken i
  sin tur (med definitionen av sekund) definierar enheten Coulomb.
  Termen ``permeabilitet'' myntades 1885 av Oliver Heaviside (1850--1925).}
$$
  \mu_0=4\pi\times10^{-7}\ {\rm H}/{\rm m}\quad(\hbox{exakt per definition})
$$
f{\"o}r den {\it magnetiska permeabiliteten i vakuum};
\sidx{Magnetisk permeabilitet}[Vakuumpermeabilitet $\mu_0$]
\sidx{Elektrisk permittivitet}[Vakuumpermittivitet $\varepsilon_0$]
detta {\"a}r f{\"o}rsta g{\aa}ngen som $\mu_0$ dyker upp i denna kurs, p{\aa}
exakt samma s{\"a}tt som $\varepsilon_0$ d{\"o}k upp f{\"o}r f{\"o}rsta
g{\aa}ngen i elektrostatiken i och med att vi introducerades till Coulombs lag.
\sidx{Coulombs kraftlag}
I sj{\"a}lva verket kan vi h{\"a}danefter generellt identifiera
``sl{\"a}kttr{\"a}det'' f{\"o}r v{\aa}ra ekvationer som
\sidx{Elektromagnetiskt sl{\"a}kttr{\"a}d}
\medskip
\item{$\bullet$}{F{\"o}rekomst av {\it enbart} elektrisk
  permittivitet $\varepsilon_0$ $\Rightarrow$ Elektrostatik.}
\item{$\bullet$}{F{\"o}rekomst av {\it enbart} magnetisk
  permeabilitet $\mu_0$ $\Rightarrow$ Magnetostatik.}
\item{$\bullet$}{F{\"o}rekomst av {\it produkten}
  $\varepsilon_0\mu_0$ $\Rightarrow$ Elektromagnetism (elektrodynamik).}
\medskip
\noindent
Om vi summerar upp alla bidrag $d{\bf B}$ till magnetf{\"a}ltet vid
observationspunkten ${\bf x}$, fr{\aa}n alla k{\"a}llor l{\"a}ngs med
str{\"o}mslingan fr{\aa}n ${\bf x}_a$ till ${\bf x}_b$, det vill s{\"a}ga
f{\"o}r alla {\it k{\"a}llpunkter} ${\bf x}'$, s{\aa} erh{\aa}ller vi
direkt Biot--Savarts lag p{\aa} integralform som
linjeintegralen\numberedfootnote{Griffiths Ekv.~(5.34), sid.~224.}
$$
  {\bf B}({\bf x})={{\mu_0}\over{4\pi}}\int^{{\bf x}_b}_{{\bf x}_a}
    {{{\bf I}({\bf x}')\times({\bf x}-{\bf x}')}
      \over{|{\bf x}-{\bf x}'|^3}}\,dl'.
$$
\sidx{Biot--Savarts lag}[F{\"o}r str{\"o}mslinga]
Vi kan redan h{\"a}r se att Biot--Savarts lag kan r{\"a}knas som motsvarigheten
till Coulombs lag i elektrostatiken, dock h{\"a}r ist{\"a}llet relaterande till
{\it r{\"o}relse (dynamik) av laddning}. Som alternativ form av Biot--Savarts
lag s{\aa} kan vi bryta ut den konstanta str{\"o}mmen $I$ som en skal{\"a}r,
och ist{\"a}llet uttrycka som linjeintegralen med linjeelementen $d{\bf l}'$
(l{\"a}ngdelement l{\"a}ngs med str{\"o}mslingan) som
$$
  {\bf B}({\bf x})={{\mu_0 I}\over{4\pi}}\int^{{\bf x}_b}_{{\bf x}_a}
    {{d{\bf l}'\times({\bf x}-{\bf x}')}\over{|{\bf x}-{\bf x}'|^3}}.
$$

\subsection{Biot--Savarts lag f{\"o}r str{\"o}mt{\"a}theten i volymer}
En viss generalisering av Biot--Savarts lag kan g{\"o}ras om vi rekapitulerar
att str{\"o}mmen $I$ ju faktiskt {\"a}r ett specialfall av ett m{\aa}tt av en
str{\"o}mt{\"a}thet ${\bf J}({\bf x}')$ (med ${\bf x}'$ liksom tidigare
varande {\it k{\"a}llpunkter}) som r{\aa}kar vara s{\aa} funtad att den bara
fl{\"o}dar i en enda kurva. I detta fall inneh{\aa}ller ju $I$ redan en
ytintegral {\"o}ver str{\"o}mt{\"a}theten ${\bf J}({\bf x}')$ och man inser
direkt att motsvarande Biot--Savarts lag f{\"o}r {\it str{\"o}mt{\"a}theten}
uttrycks som volymintegralen\numberedfootnote{Griffiths Ekv.~(5.47), sid.~231.}
$$
  {\bf B}({\bf x})={{\mu_0}\over{4\pi}}\iiint_V
  {{{\bf J}({\bf x}')\times({\bf x}-{\bf x}')}\over{|{\bf x}-{\bf x}'|^3}}\,dV'.
$$
\sidx{Biot--Savarts lag}[F{\"o}r str{\"o}mt{\"a}thet]

\section{Divergens f{\"o}r magnetf{\"a}ltet - ``Magnetiska monopoler
  existerar inte''}
\sidx{Magnetiska monopoler}[Icke-existens av]
Notera att integralen som f{\"o}rekommer i Biot--Savarts lag sker {\"o}ver
{\it primmade} koordinater ${\bf x}'$, i vilket vi betraktar observationspunkten
${\bf x}$ som fix. Vi kommer nu att s{\"o}ka uttryck f{\"o}r divergensen och
rotationen f{\"o}r det magnetiska f{\"a}ltet\numberedfootnote{Redan nu kan
  vi g{\"o}ra klart f{\"o}r oss sj{\"a}lva att denna exercis inte {\"a}r
  en exercis f{\"o}r exercisens egen skull, utan f{\"o}r att detta senare,
  i F{\"o}rel{\"a}sning~9 solitt kommer att assistera oss i bygget av
  Maxwells ekvationer, n{\aa}got som i sin tur beskriver all
  elektromagnetisk v{\aa}gutbredning! Med andra ord, {\"a}ven om man
  kan tycka att detta stycke kring magnetostatiken kan vara lite torrt
  och intets{\"a}gande, l{\aa}t oss betrakta detta som en pusselbit
  f{\"o}r vad som komma skall.}
${\bf B}$ vilket vi rekapitulerar {\"a}r {\it operationer som sker i det
icke-primmade} observations-koordinatsystemet ${\bf x}$.

Om vi f{\"o}rst analyserar divergensen f{\"o}r den magnetiska f{\"a}ltet,
s{\aa} som Biot--Savarts lag uttrycker det utifr{\aa}n den generella
beskrivningen av det i termer av str{\"o}mt{\"a}theten ${\bf J}$, s{\aa}
har vi att\numberedfootnote{Griffiths Ekv.~(5.50), sid.~232.}
$$
  \eqalign{
  \nabla\cdot{\bf B}({\bf x})
    &={{\mu_0}\over{4\pi}}\nabla\cdot\iiint_V
        {{{\bf J}({\bf x}')\times({\bf x}-{\bf x}')}
          \over{|{\bf x}-{\bf x}'|^3}}\,dV'\cr
    &=\big\{\hbox{ $\nabla$ opererar p{\aa} ${\bf x}$,
                   {\it inte} p{\aa} ${\bf x}'$ }\big\}\cr
    &={{\mu_0}\over{4\pi}}\iiint_V\nabla\cdot
        \bigg(
        {\bf J}({\bf x}')\times
          {{({\bf x}-{\bf x}')}\over{|{\bf x}-{\bf x}'|^3}}
        \bigg)\,dV'\cr
    &=\big\{\hbox{ Griffiths {\it Product Rule \#6}
                   med ``${\bf a}={\bf J}({\bf x}')$'' }\big\}\cr
    &=\big\{\hbox{ $\nabla\cdot({\bf a}\times{\bf b})
                     ={\bf b}\cdot(\nabla\times{\bf a})
                       -{\bf a}\cdot(\nabla\times{\bf b})$ }\big\}\cr
    &=\big\{
        \hbox{ Notera att ${\bf J}({\bf x}')$ oberoende av ${\bf x}$ }
      \big\}\cr
    &=-{{\mu_0}\over{4\pi}}\iiint_V
        {\bf J}({\bf x}')\cdot
        \underbrace{
        \bigg(
          \nabla\times{{({\bf x}-{\bf x}')}\over{|{\bf x}-{\bf x}'|^3}}
        \bigg)}_{\hbox{$=0, {\rm exercis (1.63)}$}}
        \,dV'\cr
    &=0.
  }
$$
Att vi har\sidx{Tricket
$\displaystyle\nabla{{1}\over{\char124}{\bf x}-{\bf x}'{\char124}}
=-{{({\bf x}-{\bf x}')}\over{{\char124}{\bf x}-{\bf x}'{\char124}^3}}$}
\sidx{Magnetisk fl{\"o}dest{\"a}thet}[Divergens f{\"o}r]
$$
  \nabla\times{{({\bf x}-{\bf x}')}\over{|{\bf x}-{\bf x}'|^3}}\equiv{\bf 0}
$$
kan vi antingen h{\"a}r verifiera genom att utf{\"o}ra differentieringen och
vektoralgebran direkt, eller konstatera att vi i F{\"o}rel{\"a}sning~2, f{\"o}r
rotationen hos det statiska elektriska f{\"a}ltet, fann att just precis det vi
har att g{\"o}ra med h{\"a}r kunde skrivas som en gradient (``tricket'' i
tolkningen av Coulombintegralen), som
$$
  {{({\bf x}-{\bf x}')}\over{|{\bf x}-{\bf x}'|^3}}\equiv
    -\nabla{{1}\over{|{\bf x}-{\bf x}'|}},
$$
och eftersom\numberedfootnote{Se exempelvis innerp{\"a}rmen p{\aa} Griffiths,
  {\it Second derivatives (10)}.}
$$
  \nabla\times(\nabla f)=0
$$
f{\"o}r godtycklig skal{\"a}r funktion $f$, s{\aa} f{\"o}ljer det direkt att
rotationen ovan {\"a}r identiskt noll, och f{\"o}ljdaktligen ocks{\aa} att
divergensen f{\"o}r magnetf{\"a}ltet {\"a}r identiskt noll.

Vad s{\"a}ger oss resultatet att divergensen $\nabla\cdot{\bf B}=0$?
Mer {\"a}n man kan tro, faktiskt. Vi kan direkt j{\"a}mf{\"o}ra detta resultat
med det {\it elektrostatiska} fallet (s{\aa} som vi gick igenom det i
F{\"o}rel{\"a}sning~1) f{\"o}r det elektriska f{\"a}ltet, d{\"a}r vi s{\aa}g
att Gauss lag $\nabla\cdot{\bf E}=\rho/\varepsilon_0$ ger relationen mellan det
elektriska f{\"a}ltet och den lokala elektriska laddningen $\rho$ (eller
{\it laddningst{\"a}theten}, om man skall vara korrekt).
I det magnetostatiska fall som vi h{\"a}r g{\aa}tt igenom har vi med andra ord
att det magnetiska fl{\"o}det ut genom en sluten yta alltid {\"a}r identiskt
noll, och vi har d{\"a}rmed ingen m{\"o}jlighet att innesluta n{\aa}gon
``magnetisk laddning''.

V{\aa}r slutsats blir d{\"a}rmed:
\sidx{Magnetiska monopoler}[Icke-existens av]
\quote{{\it $\nabla\cdot{\bf B}=0$ betyder att det inte existerar
  n{\aa}gon magnetisk laddning!}}
\noindent
Med konstaterandet att ``magnetisk laddning inte existerar'' menar vi h{\"a}r
att {\it magnetiska monopoler} inte existerar, och att {\it magnetiska f{\"a}lt
endast manifesterar sig s{\aa} som om de h{\"a}rr{\"o}r fr{\aa}n magnetiska
dipoler}, det vill s{\"a}ga ``en positiv och negativ laddning p{\aa} ett
avst{\aa}nd fr{\aa}n varandra''.
Denna terminologi anknyter till elektrisk laddning, som p{\aa} samma s{\"a}tt
handlar om {\it elektriska monopoler} (som ju faktiskt existerar) som kan
s{\"a}ttas samman till elektriska dipoler.

\section{Rotationen f{\"o}r magnetf{\"a}ltet}
\sidx{Magnetisk fl{\"o}dest{\"a}thet}[Rotation f{\"o}r]
Med den intressanta observationen att {\it magnetiska monopoler inte existerar}
i bagaget, l{\aa}t oss nu g{\aa} vidare med rotationen av magnetf{\"a}ltet.
Vi anv{\"a}nder {\"a}ven h{\"a}r den generella formen av Biot--Savarts
lag,\numberedfootnote{Notera hur vi {\"a}ven nu tar avstamp i Biot--Savarts
  lag som den bas fr{\aa}n vilket allt inom elektrostatiken tar sin
  b{\"o}rjan, samt hur permeabiliteten $\mu_0$ hakar p{\aa} i allt
  som h{\"a}rleds fr{\aa}n denna.}
vilken d{\aa} vi applicerar rotationen ({\aa}terigen med observationen att
$\nabla$ opererar p{\aa} {\it oprimmade} koordinater ${\bf x}$) ger oss
att\numberedfootnote{F{\"o}r den som endast {\"a}r ute efter resultatet,
  s{\aa} kan man med f{\"o}rdel skippa h{\"a}rledningen nedan och g{\aa}
  vidare direkt till slutresultatet i form av Amp\`eres lag p{\aa} sid.~13.
  H{\"a}rledningen {\"a}r h{\"a}r genomg{\aa}ngen i detalj f{\"o}r att den
  som {\"a}r skeptiskt lagd skall f{\aa} en chans att verifiera just hur
  vi g{\aa}r fr{\aa}n Biot--Savarts lag till Amp\`eres lag p{\aa}
  differentialform.}
$$
  \eqalign{
  \nabla\times{\bf B}({\bf x})
    &={{\mu_0}\over{4\pi}}\iiint_V\nabla\times
      \bigg(
        {\bf J}({\bf x}')\times{{({\bf x}-{\bf x}')}\over{|{\bf x}-{\bf x}'|^3}}
      \bigg)\,dV'\cr
  }
$$
Liksom i fallet med divergensen kommer vi nu att anv{\"a}nda en produktregel
fr{\aa}n innerp{\"a}rmen p{\aa} Griffiths, i detta fall {\it Product Rule \#8},
som med sina fyra termer {\"a}r aningen st{\"o}kigare, dock liksom tidigare
med ${\bf a}={\bf J}({\bf x}')$ (vilken {\"a}r oberoende av ${\bf x}'$ och
d{\"a}rmed ger noll vid differentiering) och med
${\bf b}=({\bf x}-{\bf x}')/|{\bf x}-{\bf x}'|^3$,
$$
  \nabla\times({\bf a}\times{\bf b})
    =\underbrace{{\bf a}(\nabla\cdot{\bf b})}_{\hbox{``Term 1''}}
      -\underbrace{({\bf a}\cdot\nabla){\bf b}}_{\hbox{``Term 2''}}
      +\underbrace{({\bf b}\cdot\nabla){\bf a}}_{
         =0,\ {\bf J}(\underline{\underline{\underline{{\bf x}'}}})},
      -\underbrace{{\bf b}(\nabla\cdot{\bf a})}_{
         =0,\ {\bf J}(\underline{\underline{\underline{{\bf x}'}}})},
$$
vilket {\"o}versatt till integranden ovan resulterar i att\numberedfootnote{Just
  i denna h{\"a}rledning {\"a}r Griffiths lite spretig och bygger mycket
  p{\aa} h{\"a}rledningar som gjorts tidigare i hans {\it Introduction to
  Electrodynamics}. Vi kommer h{\"a}r att f{\"o}rs{\"o}ka h{\aa}lla samman
  den stundvis aningen komplexa h{\"a}rledningen i ett stycke, med hopp om
  att det blir l{\"a}ttare att f{\"o}lja resonemanget.}
$$
  \eqalign{
    \nabla\times
        \bigg(
        {\bf J}({\bf x}')\times
          {{({\bf x}-{\bf x}')}\over{|{\bf x}-{\bf x}'|^3}}
        \bigg)
        =\underbrace{
        {\bf J}({\bf x}')
        \bigg(
        \nabla\cdot
          {{({\bf x}-{\bf x}')}\over{|{\bf x}-{\bf x}'|^3}}
        \bigg)
        }_{\hbox{``Term 1''}}
        -\underbrace{
        \big({\bf J}({\bf x}')\cdot\nabla\big)
        \bigg(
          {{({\bf x}-{\bf x}')}\over{|{\bf x}-{\bf x}'|^3}}
        \bigg)
        }_{\hbox{``Term 2''}}
  }
$$
\vfill\eject

\subsection{Term~1 i rotationen}
H{\"a}r involverar ``Term~1'' en divergens som vi kan {\"o}vers{\"a}tta till
en delta-puls placerad i ${\bf x}'$, eftersom Gauss teorem (divergensteoremet)
$\int(\nabla\cdot{\bf A})\,dV=\oint{\bf A}\cdot d{\bf S}$ ger
att\numberedfootnote{Griffiths har redan gjort f{\"o}rarbetet i kapitlet
  {\it Vector Analysis}, Ekv.~(1.100), sid.~50; se {\"a}ven Sektion 1.5.1,
  sid.~45. Vi kommer h{\"a}r f{\"o}r sakens skull dock att g{\aa} igenom
  denna exercis s{\aa} att vi h{\aa}ller resonemanget kring
  $\nabla\times{\bf B}$ sammanh{\aa}llet och koncist.}
$$
  \eqalign{
    \iiint_V\nabla\cdot{{({\bf x}-{\bf x}')}\over{|{\bf x}-{\bf x}'|^3}}\,dV'
    &=\oiint_S{{({\bf x}-{\bf x}')}
         \over{|{\bf x}-{\bf x}'|^3}}\cdot d{\bf S}'\cr
    &=\big\{\hbox{ Integrera {\"o}ver sf{\"a}r $S$ med
                   radie $|{\bf x}-{\bf x}'|=R$ }\big\}\cr
    &=\big\{\hbox{ Yttryck i sf{\"a}riska kordinater med
                   ${\bf x}'$ som origo }\big\}\cr
    &=\int^{\pi}_{0}\int^{2\pi}_{0} {{R{\bf e}_r}\over{R^3}}\cdot
      \underbrace{{\bf e}_r\,R^2\sin(\vartheta)
        \,d\varphi\,d\vartheta}_{=d{\bf S}\ {\hbox{p{\aa}}}\ S}\cr
    &=\underbrace{\int^{\pi}_{0}\sin(\vartheta)\,d\vartheta}_{=2}
        \underbrace{\int^{2\pi}_{0} d\varphi}_{=4\pi}\cr
    &=4\pi,\cr
  }
$$
f{\"o}r godtyckligt vald radie $R>0$. Samtidigt har vi ju faktiskt att
divergensen i sig ges som
$$
  \eqalign{
    \nabla\cdot{{({\bf x}-{\bf x}')}\over{|{\bf x}-{\bf x}'|^3}}
    &=\bigg(
        {{\partial}\over{\partial x}},
        {{\partial}\over{\partial y}},
        {{\partial}\over{\partial z}}
      \bigg)\cdot
      \bigg(
        {{(x-x',y-y',z-z')}\over{\big[(x-x')^2+(y-y')^2+(z-z')^2\big]^{3/2}}}
      \bigg)\cr
    &={{(1+1+1)}\over{\big[(x-x')^2+\ldots\big]^{3/2}}}
        -{{3}\over{2}}{{(2(x-x'),2(y-y'),2(z-z'))\cdot(x-x',y-y',z-z')}
            \over{\big[(x-x')^2+\ldots\big]^{5/2}}}\cr
    &={{3}\over{\big[(x-x')^2+\ldots\big]^{3/2}}}
        -3{{\big[(x-x')^2+(y-y')^2+(z-z')^2\big]}
            \over{\big[(x-x')^2+(y-y')^2+(z-z')^2\big]^{5/2}}}\cr
    &={{3}\over{\big[(x-x')^2+\ldots\big]^{3/2}}}
        -{{3}\over{\big[(x-x')^2+\ldots\big]^{3/2}}}\cr
    &=0,\cr
  }
$$
f{\"o}r alla observationspunkter ${\bf x}$ i rummet, under
f{\"o}ruts{\"a}ttningen att ${\bf x}\ne{\bf x}'$ (det vill s{\"a}ga att
n{\"a}mnaren $[(x-x')^2+\ldots]^{3/2}\ne0$).
Notera att detta resultat g{\"a}ller {\it oberoende} av v{\"a}rdet p{\aa}
radien $R=|{\bf x}-{\bf x}'|>0$, som vi kan v{\"a}lja godtyckligt liten runt
k{\"a}llpunkten ${\bf x}'$.
Utifr{\aa}n detta argument kan vi dra slutsatsen att {\it divergensen h{\"a}r
kan tolkas som en delta-puls} placerad i ${\bf x}'$, det vill s{\"a}ga {\it att
den {\"a}r noll {\"o}verallt i rummet utom just i k{\"a}llpunkten} ${\bf x}'$,
som
$$
  \nabla\cdot{{({\bf x}-{\bf x}')}\over{|{\bf x}-{\bf x}'|^3}}
    =4\pi\delta({\bf x}-{\bf x}').
$$
Med andra ord kan vi {\"o}vers{\"a}tta ``Term~1'' ovan som\numberedfootnote{Vid
  en anblick p{\aa} detta {\"a}r det paradoxalt att divergensen av en funktion
  som bevisligen {\"o}verallt {\"a}r riktad ut{\aa}t fr{\aa}n k{\"a}llpunkten
  ${\bf x}'$ har v{\"a}rdet noll {\"o}verallt f{\"o}rutom just vid
  k{\"a}llpunkten i sig.}
$$
  \eqalign{
  \iiint_V
  \underbrace{
    {\bf J}({\bf x}')
    \bigg(
    \nabla\cdot
      {{({\bf x}-{\bf x}')}\over{|{\bf x}-{\bf x}'|^3}}
    \bigg)
    }_{\hbox{``Term 1''}}\,dV'
    &=4\pi\iiint_V{\bf J}({\bf x}')\delta({\bf x}-{\bf x}')\,dV'\cr
    &=4\pi{\bf J}({\bf x}).\cr
  }
$$

\subsection{Term~2 i rotationen}
Den andra termen som upptr{\"a}der i integranden som upptr{\"a}der i uttrycket
f{\"o}r $\nabla\times{\bf B}$ kan {\"a}ven den utvecklas
vidare,\numberedfootnote{Griffiths Ekv.~(5.54), sid.~232.} som
$$
  \eqalign{
    \iiint_V
      \underbrace{
        \big({\bf J}({\bf x}')\cdot\nabla\big)
        \bigg({{({\bf x}-{\bf x}')}\over{|{\bf x}-{\bf x}'|^3}}\bigg)
      }_{\hbox{``Term 2''}}\,dV'
    &=\big\{\hbox{ Notera att $\nabla$ opererar p{\aa} ${\bf x}$ }\big\}\cr
    &=\big\{\hbox{ $\nabla\to\nabla'\quad\Rightarrow\quad$ teckenbyte }\big\}\cr
    &=-\iiint_V
        \big({\bf J}({\bf x}')\cdot\nabla'\big)
        \bigg({{({\bf x}-{\bf x}')}\over{|{\bf x}-{\bf x}'|^3}}\bigg)\,dV'
    \cr
  }
$$
Om vi f{\"o}r enkelhets skull tittar p{\aa} detta uttryck komponentvis, med
$x_k=x,y,z$, s{\aa} har vi att med  Griffiths {\it Product Rule (5)} och
``${\bf a}={\bf J}$'',
$$
 \nabla'(f{\bf a})=f(\nabla'\cdot{\bf a})
     +\underbrace{
         {\bf a}\cdot(\nabla'f)
      }_{=({\bf a}\cdot\nabla')f}
 \quad\Leftrightarrow\quad
 ({\bf J}\cdot\nabla')f=\nabla'(f{\bf J})-f(\nabla'\cdot{\bf J})
$$
att d{\aa} vi dessutom observerar att vi f{\"o}r station{\"a}ra str{\"o}mmar
enligt tidigare har att $\nabla'\cdot{\bf J}=0$, so erh{\aa}ller vi
$$
  \eqalign{
        \big({\bf J}({\bf x}')\cdot\nabla'\big)
        \bigg(\underbrace{
          {{x_k-x'_k}\over{|{\bf x}-{\bf x}'|^3}}
        }_{\hbox{=``$f$''}}\bigg)
    &=\nabla'\cdot\bigg(
    \underbrace{
      {{x_k-x'_k}\over{|{\bf x}-{\bf x}'|^3}}
    }_{\hbox{=``$f$''}}{\bf J}({\bf x}')
    \bigg)\cr
  }
$$
D{\aa} vi applicerar Gauss lag p{\aa} detta resultat, under det att vi l{\aa}ter
integrera {\"o}ver en omslutande yta stor nog att innesluta alla
str{\"o}mb{\"a}rande k{\"a}llor till Biot--Savarts lag och att vi d{\"a}rmed
inte har n{\aa}gra in- eller utg{\aa}ende str{\"o}mmar genom denna yta, blir
bidraget fr{\aa}n ``Term~2'' kort och gott f{\"o}r var och en av komponenterna
$x_k=x,y,z$ att
$$
  \eqalign{
    \iiint_V
      \underbrace{
        \big({\bf J}({\bf x}')\cdot\nabla\big)
        \bigg({{(x_k-x'_k)}\over{|{\bf x}-{\bf x}'|^3}}\bigg)
      }_{\hbox{``Term 2''}}\,dV'
    &=-\iiint_V\nabla'\cdot\bigg(
      {{x_k-x'_k}\over{|{\bf x}-{\bf x}'|^3}}{\bf J}({\bf x}')
      \bigg)
      \,dV'\cr
    &=\big\{\hbox{ Gauss lag }\big\}\cr
    &=-\oiint_S\bigg(
      {{x_k-x'_k}\over{|{\bf x}-{\bf x}'|^3}}{\bf J}({\bf x}')
      \bigg)\cdot d{\bf S}\cr
    &=0.
  }
$$
Med andra ord, genom att utnyttja v{\aa}r frihet att definiera v{\aa}r
integrationsdom{\"a}n till att helt innesluta str{\"o}mmarna som utg{\"o}r
k{\"a}llor i Biot--Savarts lag, n{\aa}got som faller sig helt naturligt, s{\aa}
kan vi motivera att bidraget till rotationen av det magnetiska f{\"a}ltet
fr{\aa}n ``Term~2'' {\"a}r noll.

\subsection{Slutligt resultat f{\"o}r rotationen av magnetf{\"a}ltet}
L{\aa}t oss nu s{\"a}tta samman dessa delresultat f{\"o}r ``Term~1'' och
``Term~2'' till ett uttyck f{\"o}r rotationen f{\"o}r magnetf{\"a}ltet,
$$
  \eqalign{
  \nabla\times{\bf B}({\bf x})
    &={{\mu_0}\over{4\pi}}\iiint_V\nabla\times
      \bigg(
        {\bf J}({\bf x}')\times{{({\bf x}-{\bf x}')}\over{|{\bf x}-{\bf x}'|^3}}
      \bigg)\,dV'\cr
    &={{\mu_0}\over{4\pi}}
        \underbrace{
          \iiint_V
          \underbrace{
            {\bf J}({\bf x}')
            \bigg(
            \nabla\cdot
              {{({\bf x}-{\bf x}')}\over{|{\bf x}-{\bf x}'|^3}}
            \bigg)
          }_{\hbox{``Term 1''}}
          \,dV'
        }_{\displaystyle=4\pi{\bf J}({\bf x}),\hbox{ enligt ovan}}
        -{{\mu_0}\over{4\pi}}
        \underbrace{
          \iiint_V\underbrace{
            \big({\bf J}({\bf x}')\cdot\nabla\big)
            \bigg(
              {{({\bf x}-{\bf x}')}\over{|{\bf x}-{\bf x}'|^3}}
            \bigg)
          }_{\hbox{``Term 2''}}
          \,dV'
        }_{\displaystyle=0,\hbox{ enligt ovan}}\cr
    &=\mu_0{\bf J}({\bf x}).
  }
$$

\section{Amp\`eres lag}
\sidx{Amp\`eres lag}
V{\aa}rt slutliga resultat f{\"o}r rotationen f{\"o}r det magnetiska f{\"a}ltet
p{\aa} differentiell form,
$$
  \nabla\times{\bf B}=\mu_0{\bf J},
$$
kallas f{\"o}r {\it Amp\`eres lag}, och kommer fram{\"o}ver i kursen att ha en
stor betydelse inte bara f{\"o}r hur vi kan ber{\"a}kna kopplingen mellan
str{\"o}mmar och magnetf{\"a}lt, utan {\"a}ven (som det kommer att visa sig
i F{\"o}rel{\"a}sning~9) f{\"o}r hur vi kan formulera elektromagnetisk
v{\aa}gutbredning med Maxwell's ekvationer (med assistans av en
till{\"a}ggsterm till Amp\`eres magnetostatiska lag, som vi d{\aa} kommer att
g{\aa} igenom). {\it Notera h{\"a}r hur vi genomg{\aa}ende kan sp{\aa}ra
f{\"o}rekomsten av $\mu_0$ till Biot--Savarts lag.}

Amp\`eres lag kan enkelt omformuleras p{\aa} integralform genom anv{\"a}ndandet
av Stokes teorem, f{\"o}r en sluten slinga $\Gamma$ inneslutande en yta $S$ i
magnetf{\"a}ltet och str{\"o}mt{\"a}theten, som\numberedfootnote{Se exempelvis
  innerp{\"a}rmen p{\aa} Griffiths, {\it Curl theorem}.}
$$
  \iint_{S}(\nabla\times{\bf B})\cdot d{\bf S}=
  \oint_{\Gamma}{\bf B}\cdot d{\bf l}=
  \mu_0\underbrace{\iint_{S}{\bf J}\cdot d{\bf S}}_{\displaystyle =I_{\rm enc}},
$$
d{\"a}r nu $\iint_{S}{\bf J}\cdot d{\bf S}=I_{\rm enc}$ {\"a}r den av $\Gamma$
inneslutna str{\"o}mmen (med andra ord den totala str{\"o}m som passerar genom
integrationsytan $S$). Med andra ord har vi integralformen av Amp\`eres lag som
$$
  \oint_{\Gamma}{\bf B}\,d{\bf l}=\mu_0I_{\rm enc}.
$$
Denna form {\"a}r ofta att f{\"o}redra i situationer d{\aa} vi kan utnyttja
rent geometriskt--symmetriskt gynn\-samma situationer, p{\aa} precis samma
s{\"a}tt som vi kunnat konstatera med Gauss lag och hur den markant kan
f{\"o}renkla probleml{\"o}sande i elektrostatiska problem med symmetrier
n{\"a}rvarande.

Exempel: Ber{\"a}kning av magnetf{\"a}lt genererade runt str{\"o}mslingor,
i samma geometri som i Biot--Savarts ursprungliga
experiment.\numberedfootnote{Oneliner f{\"o}r magnetf{\"a}lt runt
  o{\"a}ndlig rak ledare b{\"a}rande str{\"o}mmen $I$:
  $$
    \oint_{\Gamma}{\bf B}\,d{\bf l}
      =\int^{2\pi}_{0}B_{\varphi}(r)\,rd\varphi
      =2\pi rB_{\varphi}(r)
      =\mu_0 I_{\rm enc}
      =\mu_0 I\quad\Rightarrow\quad
    B_{\varphi}(r)={{\mu_0 I}\over{2\pi r}}.
  $$}

\section{Vektorpotentialen}
\sidx{Vektorpotential}\sidx{Vektorpotential}[Amp\`eres lag]
Vi erinrar oss att ekvationen $\nabla\times{\bf E}={\bf 0}$ i elektrostatiken
ledde oss till att dra slutsatsen att det existerar en skal{\"a}r potential
definierad genom ${\bf E}=-\nabla\phi$. P{\aa} samma s{\"a}tt inbjuder
$\nabla\cdot{\bf B}=0$ (det vill s{\"a}ga att {\it inga magnetiska monopoler
existerar}) oss till att via {\it vektoridentieteten}\numberedfootnote{Se
  exempelvis innerp{\"a}rmen p{\aa} Griffiths, {\it Second derivatives (9)}.}
$$
  \nabla\cdot(\nabla\times{\bf A})=0
$$
tolka magnetf{\"a}ltet ${\bf B}$ som sprunget ur en {\it vektorpotential}
${\bf A}$ enligt
$$
  {\bf B}=\nabla\times{\bf A}.
$$
Om vi substituerar denna potential f{\"o}r magnetf{\"a}ltet ${\bf B}$ i
Amp\`eres lag p{\aa} differentialform, s{\aa} erh{\aa}ller vi f{\"o}r
v{\"a}nsterledet\numberedfootnote{Se exempelvis innerp{\"a}rmen p{\aa}
  Griffiths, {\it Second derivatives (11)}.}
$$
  \nabla\times{\bf B}=\nabla\times(\nabla\times{\bf A})
    =\nabla\underbrace{(\nabla\cdot{\bf A})}_{=0}-\nabla^2{\bf A}
    =-\nabla^2{\bf A}
$$
Detta g{\"o}r att vi kan formulera Amp\`eres lag i termer av vektorpotentialen
som\numberedfootnote{Griffiths Ekv.~(5.64), sid.~244. Notera att liksom i
  fallet med Poissons ekvation $\nabla^2\phi=-\rho/\varepsilon_0$ f{\"o}r
  den elektrostatiska skal{\"a}ra potentialen fr{\aa}n F{\"o}rel{\"a}sning~3,
  s{\aa} betraktar Griffiths denna form som s{\aa} fundamental att den
  {\"a}r den andra som visas p{\aa} omslaget till {\it Introduction to
  Electrodynamics}.}
\sidx{Amp\`eres lag}[Uttryckt i vektorpotentialen]
$$
  \nabla^2{\bf A}=-\mu_0{\bf J}.
$$

\section{Explicit l{\"o}sning f{\"o}r vektorpotentialen}
\sidx{Vektorpotential}[Explicit l{\"o}sning]
Eftersom vektorpotentialen beskrivs av Poissons ekvation, vilket i grund och
botten {\"a}r en {\it skal{\"a}r} partiell differentialekvation, som kan
projiceras ut komponentvis ${\bf J}=(J_x,J_y,J_z)$ f{\"o}r vektorpotentialen,
$$
  \nabla^2 A_k=-\mu_0 J_k,\quad k=x,y,z,
$$
s{\aa} kan vi direkt notera likheten mellan denna och motsvarande ekvation
f{\"o}r den skal{\"a}ra potentialen
$$
  \nabla^2\phi=-\rho/\varepsilon_0.
$$
Nu r{\aa}kar vi ha s{\aa}dan tur att en explicit l{\"o}sning till Poissons
ekvation f{\"o}r den skal{\"a}ra potentialen har erh{\aa}llits i
F{\"o}rel{\"a}sning~3, som\numberedfootnote{Om vi skall vara riktigt petiga
  h{\"a}r, s{\aa} var det faktiskt s{\aa} att vi i F{\"o}rel{\"a}sning
  {\it definierade} den skal{\"a}ra potentialen $\phi$ som detta uttryck.
  Anledningen till denna bekv{\"a}ma definition (som i m{\aa}ngt och mycket
  handlar om vilket tecken p{\aa} $-\nabla\phi$ v{\"a}ljer att definiera
  det elektriska f{\"a}ltet efter) var dock just att denna via Coulombs
  generaliserade ekvation (``Coulombintegralen'') l{\"o}ser ekvationen
  f{\"o}r det elektriska f{\"a}ltet i n{\"a}rvaro av laddningsf{\"o}rdelning
  $\rho$, s{\aa} vi har visst fog f{\"o}r att denna definition ocks{\aa}
  {\"a}r en formell l{\"o}sning till Poissons ekvation f{\"o}r den
  skal{\"a}ra potentialen.}
d{\"a}r vi definierade den {\it skal{\"a}ra potentialen} $\phi({\bf x})$
explicit som integralen
$$
  \phi({\bf x})={{1}\over{4\pi\varepsilon_0}}\iiint_V
    {{\rho({\bf x}')}\over{|{\bf x}-{\bf x}'|}}\,dV'.
$$
Med andra ord kan vi direkt {\"o}verf{\"o}ra detta resultat {\"a}ven f{\"o}r
vektorpotentialens komponenter, med endast en liten {\"a}ndring i koefficienten
$1/\varepsilon_0\to\mu_0$, s{\aa} att en explicit l{\"o}sning i termer av
str{\"o}mf{\"o}rdelningen $J_k$ erh{\aa}lls som
$$
  {\bf A}({\bf x})={{\mu_0}\over{4\pi}}\iiint_V
    {{{\bf J}({\bf x}')}\over{|{\bf x}-{\bf x}'|}}\,dV',
  \qquad\Leftrightarrow\qquad
  A_k({\bf x})={{\mu_0}\over{4\pi}}\iiint_V
    {{J_k({\bf x}')}\over{|{\bf x}-{\bf x}'|}}\,dV',\quad k=x,y,z,
$$
och motsvarande f{\"o}r en str{\"o}m ${\bf I}=(I_x,I_y,I_z)$ som linjeintegralen
$$
  {\bf A}({\bf x})={{\mu_0}\over{4\pi}}\int_{\Gamma}
    {{{\bf I}({\bf x}')}\over{|{\bf x}-{\bf x}'|}}\,dl'.
$$
\vfill\eject

\section{Sammanfattning av F{\"o}rel{\"a}sning~4 -- Magnetostatik}
\item{$\bullet$}{Lorentz-kraften p{\aa} fri laddning $q$ med hastighet
  ${\bf v}$ {\"a}r ${\bf F}=q\big({\bf E}+{\bf v}\times{\bf B}\big)$.}
\item{$\bullet$}{Magnetiska krafter (p{\aa} fria laddningar) utf{\"o}r
  inget arbete!}
\item{$\bullet$}{Amp\`eres kraftlag p{\aa} str{\"o}mslinga b{\"a}rande
  str{\"o}mmen $I$,
  $$
    {\bf F}_{\rm mag}=\int^{{\bf x}_b}_{{\bf x}_a} ({\bf I}\times{\bf B})\,dl.
  $$}
\item{$\bullet$}{Str{\"o}mmen $I$ genom en yta $S$ ges av str{\"o}mt{\"a}theten
  ${\bf J}$ som
  $$
    I=\iint_S {\bf J}\cdot d{\bf S},
  $$
\item{$\bullet$}{I en volym med konduktivitet $\sigma$ och ett elektriskt
  f{\"a}lt ${\bf E}$ ges str{\"o}mt{\"a}theten som ${\bf J}=\sigma{\bf E}$.}
\item{$\bullet$}{Lagen om att elektrisk laddning inte kan f{\"o}rsvinna
  beskrivs av sambandet mellan str{\"o}mt{\"a}thet ${\bf J}$ och
  laddningst{\"a}thet $\rho$ som
  $$
    \nabla\cdot{\bf J}=-{{d\rho}\over{dt}}.
  $$}
\item{$\bullet$}{Statiska problem definieras av att
  $$
    {{d\rho}\over{dt}}=0\quad\underline{\hbox{och}}\quad{{d{\bf J}}\over{dt}}=0.
  $$}
\item{$\bullet$}{Divergensen av str{\"o}mt{\"a}theten {\"a}r noll i
  {\it statiska problem}, vilket {\"a}r en direkt f{\"o}ljd av att
  laddnings\-t{\"a}t\-heten {\"a}r tidsoberoende,
  $$
    {{d\rho}\over{dt}}=0\quad\Rightarrow\quad\nabla\cdot{\bf J}=0.
  $$}
\item{$\bullet$}{Biot--Savarts generella lag f{\"o}r samband mellan
  magnetf{\"a}ltet ${\bf B}$ och str{\"o}mt{\"a}theten ${\bf J}$ ges som
  $$
    {\bf B}({\bf x})={{\mu_0}\over{4\pi}}\iiint_V
        {{{\bf J}({\bf x}')\times({\bf x}-{\bf x}')}
          \over{|{\bf x}-{\bf x}'|^3}}\,dV',
  $$
  d{\"a}r $\mu_0=4\pi\times10^{-7}\ {\rm H}/{\rm m}$ (exakt, per definition)
  {\"a}r den magnetiska permeabiliteten i vakuum. Biot--Savarts motsvarande
  lag f{\"o}r str{\"o}mslingor med str{\"o}m $I$ fr{\aa}n en punkt ${\bf x}_a$
  till ${\bf x}_b$ ges som
  $$
    {\bf B}({\bf x})={{\mu_0}\over{4\pi}}\int^{{\bf x}_b}_{{\bf x}_a}
        {{{\bf I}({\bf x}')\times({\bf x}-{\bf x}')}
          \over{|{\bf x}-{\bf x}'|^3}}\,dl'.
  $$}
\item{$\bullet$}{Det g{\"a}ller {\it alltid} att
  $$
    \nabla\cdot{\bf B}=0.
  $$
  Nollan i h{\"o}gerledet har som direkt f{\"o}ljd, via tolkning genom
  Gauss lag,  att ``magnetisk laddning'' inte existerar! (``Magnetisk
  laddning'' {\"a}r h{\"a}r ekvivalent med ``magnetiska monopoler''.)}
\item{$\bullet$}{Amp\`eres magnetostatiska lag p{\aa} integral- respektive
  differentialform,
  $$
    \oint_{\Gamma}{\bf B}\cdot d{\bf l}
      =\mu_0\iint_S{\bf J}\cdot d{\bf S}
      =\mu_0I_{\rm enc}
    \quad\Leftrightarrow\quad
    \nabla\times{\bf B}=\mu_0{\bf J}.
$$}
\item{$\bullet$}{Ur ``lagen om att inga magnetiska monopoler existerar''
  kan vi direkt formulera vektorpotentialen ${\bf A}$ som
  $$
    \nabla\cdot{\bf B}=0\quad\Leftrightarrow\quad{\bf B}=\nabla\times{\bf A}.
  $$}
\item{$\bullet$}{Amp\`eres lag f{\"o}r vektorpotentialen ${\bf A}$ (andra
  ekvationen p{\aa} omslaget p{\aa} Griffiths!) ges som Poissons ekvation
  med den fria str{\"o}mt{\"a}theten ${\bf J}$ som k{\"a}llterm,
  $$
    \nabla^2{\bf A}=-\mu_0{\bf J},
  $$
  med den explicita l{\"o}sningen
  $$
    {\bf A}({\bf x})={{\mu_0}\over{4\pi}}\iiint_V
      {{{\bf J}({\bf x}')}\over{|{\bf x}-{\bf x}'|}}\,dV',
    \qquad\Leftrightarrow\qquad
    A_k({\bf x})={{\mu_0}\over{4\pi}}\iiint_V
      {{J_k({\bf x}')}\over{|{\bf x}-{\bf x}'|}}\,dV',\quad k=x,y,z.
  $$}
\index
\bye
