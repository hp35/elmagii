%
% File: teach/elmagii/lect-01/lecture-01.tex [plain TeX code]
% Github: https://github.com/elmagii/lect-01/
% Last change: November 2, 2025
%
% Lecture No 1 in the course ``Elektromagnetism II, 1TE626 (2025)'',
% held November 3, 2025, at Uppsala University, Sweden.
%
% Copyright (C) 2022-2025, Fredrik Jonsson, under Gnu General Public
% License (GPL) v3. See the enclosed LICENSE for details.
%
% This program is free software: you can redistribute it and/or modify
% it under the terms of the GNU General Public License as published by
% the Free Software Foundation, either version 3 of the License, or
% (at your option) any later version.
%
% This program is distributed in the hope that it will be useful,
% but WITHOUT ANY WARRANTY; without even the implied warranty of
% MERCHANTABILITY or FITNESS FOR A PARTICULAR PURPOSE.  See the
% GNU General Public License for more details.
%
% You should have received a copy of the GNU General Public License
% along with this program.  If not, see <https://www.gnu.org/licenses/>.
%
\def\coursename{Elektromagnetism II}
\def\coursecode{1TE626}
\def\courseyear{2025}
\def\courserepo{https://github.com/hp35/elmagii/}
\def\lecturenumber{1}
\def\lecturetitle{Elektrostatik, superpositionsprincipen och Gauss lag}
\def\lecturetitlesub{}
\def\lectureauthor{Fredrik Jonsson}
\def\lectureplace{Uppsala Universitet}
\def\lecturedate{3 november 2025}
%-------------------- BEGIN OF LOCAL MACROS --------------------
\input macros/epsf.tex
\input macros/eplain.tex
\input amssym % to get the {\Bbb E} font (strikethrough E)
\font\ninerm=cmr9
\font\tenssbx=cmssbx10
\font\twelvesc=cmcsc10 at 12 truept
\def\ifempty#1{\ifx\relax#1\relax}
\def\initlecture{
  \hsize=150mm\hoffset=4.6mm\vsize=230mm\voffset=7mm
  \topskip=0pt\baselineskip=12pt\parskip=0pt\leftskip=0pt\parindent=15pt
  \headline={\ifnum\pageno>1\ifodd\pageno\rightheadline\else\leftheadline\fi
    \else\hfill\fi}
  \def\rightheadline{\tenrm{\it F\"orel\"asning \lecturenumber}
    \hfil{\it \coursename, \coursecode\ (\courseyear)}}
  \def\leftheadline{\tenrm{\it \coursename, \coursecode\ (\courseyear)}
    \hfil{\it F\"orel\"asning \lecturenumber}}
  \noindent~\vskip-60pt\hskip-40pt{\epsfbox{macros/UU_logo_color.eps}}
  \vskip-42pt\hfill\vbox{
      \hbox{{\it \coursename, \coursecode\ (\courseyear)}}
      \hbox{{\it Lecture Notes, \lectureauthor}}
      \hbox{{\it Document Revision \today}}
      \hbox{{\it \courserepo}}}\vskip 36pt
  \centerline{\twelvesc F\"orel\"asning \lecturenumber}
  \vskip 24pt\noindent
  \centerline{\twelvesc\lecturetitle}
  \expandafter\ifempty\expandafter{\lecturetitlesub}%
    \else\centerline{\twelvesc\lecturetitlesub}\fi
  \bigskip
  \centerline{\lectureauthor, \lectureplace, \lecturedate}
  \vskip24pt}
\def\section #1 {\medskip\goodbreak\noindent{\tenssbx #1}
  \par\nobreak\smallskip\noindent}
\def\subsection #1 {\medskip\goodbreak\noindent{\it #1}
  \par\nobreak\smallskip\noindent}
\def\iint{\mathop{\int\kern-8pt\int}}
\def\iiint{\mathop{\int\kern-8pt\int\kern-8pt\int}}
\def\oiint{\mathop{\int\kern-8pt\int\kern-13.2pt{\bigcirc}}}
\def\sgn{\mathop{\rm sgn}\nolimits} % sign
\def\Re{\mathop{\rm Re}\nolimits}   % real part
\def\Im{\mathop{\rm Im}\nolimits}   % imaginary part
\def\Tr{\mathop{\rm Tr}\nolimits}   % quantum mechanical trace
\def\eqq{\mathop{\vbox{\hbox{\hskip2pt?}\vskip-6pt\hbox{=}}}}
\def\quote#1{\par\leftskip=36pt\rightskip=36pt\smallskip\noindent#1\par
  \leftskip=0pt\rightskip=0pt\smallskip}
\long\def\plan#1{\par\leftskip=36pt\rightskip=36pt\bigskip
  \noindent{\it Sammanfattning}\smallskip
  \noindent{\it #1}\par\leftskip=0pt\rightskip=0pt}
\def\threepointsummary#1#2#3{\par\leftskip=36pt\rightskip=36pt\bigskip
  \noindent{\it Tre h{\aa}llpunkter i f{\"o}rel{\"a}sningen}\smallskip
  \leftskip=48pt\rightskip=36pt\hangindent=20pt
  \noindent{\it\hbox to 20pt{1. }#1}\smallskip
  \leftskip=48pt\rightskip=36pt\hangindent=20pt
  \noindent{\it\hbox to 20pt{2. }#2}\smallskip
  \leftskip=48pt\rightskip=36pt\hangindent=20pt
  \noindent{\it\hbox to 20pt{3. }#3}\par
  \leftskip=0pt\rightskip=0pt\vfill\eject}
\def\epsfig#1{\bigskip\centerline{\epsfbox{#1}}\medskip}
\def\captionwide{\advance\leftskip by 60pt
  \advance\rightskip by 60pt}
\newif\ifshowindex
\showindextrue  % Use \showindextrue and \showindexfalse to enable/disable index
\def\index{\ifshowindex\vfill\eject\section{Index} \readindexfile{i}\fi}
%--------------------- END OF LOCAL MACROS ---------------------

\initlecture

\plan{Med en kort sammanfattning av historiken bakom elektrostatik och
  uppt{\"a}ckten av elektronen som elementarladdning g{\aa}r vi direkt in
  p{\aa} Coulombs lag f{\"o}r v{\"a}xelverkan mellan laddade partiklar.
  Coulombs lag, som i grund och botten kan h{\"a}rledas fr{\aa}n utbyte av
  virtuella fotoner mellan laddade elementarpartiklar, tas h{\"a}r som ett
  axiom, fr{\aa}n vilket vi h{\"a}rleder fram motsvarande kraft p{\aa} en
  testladdning fr{\aa}n ett system av laddningar, ur vilken det elektriska
  f{\"a}ltet definieras.
  
  En genomg{\aa}ng av superpositionsprincipen f{\"o}r elektriska f{\"a}lt
  f{\"o}ljs av en h{\"a}r\-led\-ning av Coulomb-v{\"a}xelverkan f{\"o}r en
  kontinuerlig distribution av laddningst{\"a}thet, den s{\aa} kallade
  Coulombs generaliserade lag, eller kort och gott ``Coulombintegralen''.
  Vi intro\-ducerar det elektriska fl{\"o}det som integralen av det elektriska
  f{\"a}ltet {\"o}ver en godtycklig yta, utifr{\aa}n vilket vi h{\"a}rleder
  Gauss lag f{\"o}r elektriska f{\"a}lt, till{\"a}mpbar p{\aa} godtyckliga
  laddningsf{\"o}rdelningar i form av punkt- linje- yt- eller volymladdningar.
  Slutligen avslutar vi med en h{\"a}rledning av Gauss lag p{\aa}
  differentialform.}

\threepointsummary{%
  Superpositionsprincipen, som {\"a}r central i hela f{\"o}rel{\"a}sningsserien,
  inneb{\"a}r att vi kan addera separata l{\"o}sningar f{\"o}r elektriska
  f{\"a}lt fr{\aa}n separata ladd\-nin\-gar och laddningsf{\"o}rdelningar
  till en l{\"o}sning f{\"o}r det totala f{\"a}ltet.
}{%
  Det elektriska f{\"a}ltet fr{\aa}n ett system av punktladdningar $q'_k$,
  placerade i k{\"a}llpunkter ${\bf x}'_k$, kan f{\"o}r en kontinuerlig
  laddningsf{\"o}rdelning (laddnings\-t{\"a}thet) $\rho({\bf x})$ formuleras
  som Coulombs generaliserade lag (``Coulombintegralen''),
  $$
    {\bf E}({\bf x})={{1}\over{4\pi\varepsilon_0}}\sum^N_{k=1}
       q'_k {{({\bf x}-{\bf x}'_k)}\over{|{\bf x}-{\bf x}'_k|^3}},
       \quad\Leftrightarrow\quad
    {\bf E}({\bf x})={{1}\over{4\pi\varepsilon_0}}\iiint_V\rho({\bf x}')
      {{({\bf x}-{\bf x}')}\over{|{\bf x}-{\bf x}'|^3}}\,dV'.
  $$
  d{\"a}r laddningst{\"a}theten i sin tur kan beskriva punktladdningar som
  $$
    \rho({\bf x})=\sum_k q_k \delta({\bf x}-{\bf x}_k)
  $$
}{%
  Gauss lag p{\aa} integral- respektive differentialform:
  $$
    \oiint_S {\bf E}\cdot d{\bf A}
       ={{1}\over{\varepsilon_0}}\iiint_V\rho({\bf x})\,dV
    \qquad\Leftrightarrow\qquad
    \nabla\cdot{\bf E}={{\rho({\bf x})}\over{\varepsilon_0}}.
  $$
}
%----------------------- END OF PREAMBLE -----------------------

\section{Introduktion till elektrostatik}
Vi kommer i denna f{\"o}rsta f{\"o}rel{\"a}sning\numberedfootnote{Detta
  avsnitt har sannolikt ett visst {\"o}verlapp med tidigare kurser;
  anledningen till att vi trots allt v{\"a}ljer att inkludera fundamentan
  av v{\"a}xelverkan mellan laddningar {\"a}r att notation och beteckningar
  kommer att {\aa}terkomma frekvent genom kursen, samt att vissa detaljer
  som superpositionsprincipen {\"a}r basen f{\"o}r senare, mer avancerade
  till{\"a}mpningar.}
att behandla  {\it elektrostatik}, som {\"a}r l{\"a}ran om hur
{\it station{\"a}ra} elektriska laddningar v{\"a}xelverkar.
Vi passar redan nu p{\aa} att sammanfatta begreppen elektrostatik och
magnetostatik enligt f{\"o}ljande:
\medskip
\item{$\bullet$}{Station{\"a}ra, tidsoberoende laddningar
  $\Rightarrow$ Konstanta elektriska f{\"a}lt (elektro{\it statik})}
\item{$\bullet$}{Station{\"a}ra, tidsoberoende str{\"o}mmar
  $\Rightarrow$ Konstanta magnetiska f{\"a}lt (magneto{\it statik})}

\section{Historik -- Elektrostatik fr{\aa}n de gamla grekerna till Coulomb}
\sidx{Elektrostatik}
\sidx{Elektrostatik}[Tidig historik]
Elektrostatikens historik kan s{\"a}gas ha b{\"o}rjat med ``de gamla grekerna'',
som observerade att b{\"a}rnsten som gnuggades med skinn efter{\aa}t attraherade
l{\"a}tta objekt som damm eller h{\aa}rstr{\aa}n. Detta fenomen d{\"o}ptes till
``elektricitet'' fr{\aa}n det gammalgrekiska (klassiskt grekiska) ordet f{\"o}r
b{\"a}rnsten, ``elektron''.\numberedfootnote{%
  $\eta\lambda\varepsilon\kern-.4pt\kappa\tau\kern-1.4pt\rho o$;
  fr{\aa}n klassisk grekiska
  $\eta\lambda\varepsilon\kern-.4pt\kappa\tau\kern-1.4pt\rho o\nu$
  (elektron, ``b{\"a}rnsten'').}
Under 1600- och 1700-talen gjordes framsteg prim{\"a}rt i och med William
Gilberts\sidx{Gilbert, William (1544--1603)} (1544--1603)
{\it On Magnetism}\sidx{Gilbert, William (1544--1603)}[{\it On Magnetism}]
(1600), d{\"a}r elektrisk attraktionskraft\sidx{Attraktionskraft}[Elektrisk]
separerades fr{\aa}n magnetism, samt
Fran\c{c}ois de Cisternay du Fays (1698--1739)\sidx{de Cisternay du Fays,
Fran\c{c}ois (1698--1739)}\numberedfootnote{Vilket namn!} uppt{\"a}ckt (1730)
av {\it tv{\aa} typer} av laddning, ``vitreous'' (``glasaktig'', associerat
med att den uppstod d{\aa} glas gnidits med silke) och ``resinous''
(``k{\aa}daktig'', ``hartsartad'', associerat med att den uppstod d{\aa} harts
eller b{\"a}rnsten gnidits med p{\"a}ls).

Benjamin Franklin\sidx{Franklin, Benjamin (1706--1790)} (1706--1790) utf{\"o}rde
under {\aa}ren 1747--1749 experiment f{\"o}r att bevisa sin teori kring
``en-v{\"a}tske\-hypo\-tesen'', d{\"a}r han argumenterade f{\"o}r att du Fays
``vitreous och resinuous'' dubbel-v{\"a}tsketori var felaktig och att det bara
fanns en typ av ``elektrisk v{\"a}tska'', och att alla elektrostatiska effekter
kunde sp{\aa}ras till ett {\"o}verfl{\"o}d respektive underskott av denna
``v{\"a}tska''.
Franklin betecknade dessa som {\it positivt} ({\"o}verfl{\"o}d) respektive
{\it negativt} (underskott) av elektrisk v{\"a}tska, termer som fortfarande
idag lever kvar f{\"o}r att beteckna laddningens polaritet.
\sidx{Elektrisk laddning}[Polaritet]

Charles-Augustin de Coulomb\sidx{de Coulomb, Charles-Augustin (1736--1806)}
(1736--1806) f{\"o}rtj{\"a}nar ett eget kapitel i elektrostatikens historia, i
och med publiceringen av hans {\it Second M\'emoire sur l'\'Electricit\'e et le
Magn\'etisme}\sidx{de Coulomb, Charles-Augustin (1736--1806)}[{{\it Second
M\'emoire sur l'\'Electricit\'e\break et le Magn\'etisme}}] [{\it Histoire de
l'Acad\'emie Royale des Sciences}, sid.~578--611 (1785)].\numberedfootnote{%
  V{\"a}l v{\"a}rd att ta en titt p{\aa}:
  {\tt https://books.google.se/books?id=by5EAAAAcAAJ\&pg=PA578}}
I denna publikation, p{\aa} sidan 579, fastst{\"a}lls det f{\"o}r f{\"o}rsta
g{\aa}ngen att den attraktiva kraften\sidx{Attraktionskraft}[Elektrisk]
mellan tv{\aa} motsatt laddade sf{\"a}rer {\"a}r proportionell mot produkten av
laddningarna p{\aa} sf{\"a}rerna, samt inverst proportionell mot kvadraten p{\aa}
avst{\aa}ndet mellan sf{\"a}rerna;\numberedfootnote{Coulombs lag var
  sj{\"a}lvfallet ej uttryckt i moderna enheter i denna tidiga artikel;
  SI-systemet skulle inte komma att tr{\"a}da i kraft f{\"o}rr{\"a}n 1954.}
med andra ord Coulombs lag\sidx{Coulombs kraftlag} s{\aa} som vi idag k{\"a}nner
den, och som dessutom {\"a}r den prim{\"a}ra basen f{\"o}r {\"a}mnet f{\"o}r
denna f{\"o}rel{\"a}sning.

Rent historiskt {\"a}r det anm{\"a}rkningsv{\"a}rt att det tog mer {\"a}n hundra
{\aa}r innan den Brittiska fysikern Joseph John ``J.~J.'' Thomson\sidx{Thomson,
Joseph John (1856--1940)} (1856--1940, Nobelpris i fysik 1906) visade p{\aa}
existensen av elektroner\sidx{Elektron} 1897, genom experiment med elektroner i
katodstr{\aa}ler{\"o}r. Genom att m{\"a}ta hur katodstr{\aa}len kunde f{\aa}s
att avvika fr{\aa}n en r{\"a}t linje med elektriska och magnetiska f{\"a}lt drog
han slutsatsen att katodstr{\aa}len bestod av vad han fr{\aa}n b{\"o}rjan
kallade ``korpuskler'' \sidx{Korpuskler} med negativ laddning och en storlek
som var v{\"a}sentligt mindre {\"a}n atomer.
Den vetenskapliga v{\"a}rlden hade dock redan b{\"o}rjat anamma ordet
``elektron'' f{\"o}r dessa elementarladdningar,\sidx{Elementarladdning $e$} och
Thomson anpassade sin terminologi d{\"a}refter.

J.~J.~Thomsons uppt{\"a}ckt av elektronen hade effekt inte bara p{\aa}
elektrostatiken i sig, utan p{\aa} vetenskapen i stort d{\aa} han p{\aa}visade
att atomer inte var de fundamentala partiklar som man tidigare hade trott, i
form av homogena sf{\"a}rer, utan kan s{\"a}gas vara startskottet f{\"o}r nya
atommodeller, som till exempel Ernest Rutherfords\sidx{Rutherford, Ernest
(1871--1937)} (1871--1937, Nobelpris i kemi 1908) uppt{\"a}ckt av atomk{\"a}rnan
1912, och Niels Bohrs\sidx{Bohr, Niels (1885--1962)} (1885--1962, Nobelpris i
fysik 1922) skalmodell av atomen 1913.
\vfill\eject

\section{Coulombs lag f{\"o}r punktladdningar}
\sidx{Coulombs kraftlag}
\epsfig{figs/coulomb.1}\noindent
Som den mest fundamentala byggstenen i elektrostatiken har vi att tv{\aa}
punktladdningar\sidx{Punktladdning}\sidx{Elektrisk laddning}
\sidx{Elektrisk laddning}[Polaritet] $q$ och $q'$, r{\"a}knade med sina
respektive tecken f{\"o}r positiv eller negativ laddning i enheter av Coulomb
(C) \sidx{Coulomb, enhet f{\"o}r laddning} och placerade i
respektive {\it observationspunkten}\sidx{Observationspunkt} ${\bf x}$ och
{\it k{\"a}llpunkten}\sidx{K{\"a}llpunkt} ${\bf x}'$,
attraherar\sidx{Attraktionskraft}[Elektrisk] eller repellerar varandra med en
kraft ${\bf F}$ genom Coulombs lag\numberedfootnote{Griffiths Ekv.~(2.1),
  sid.~60; laddningen i observationspunkten betecknas som ``test charge''.
  Observera ocks{\aa} Griffiths lite udda stil i notationen av
  ``$({\bf x}-{\bf x}')$'' som ``scriptat ${\bf x}$''
  (se Griffiths Ekv.~(2.2) p{\aa} sid.~60). Den notation som
  Griffiths anv{\"a}nder {\"a}r lite olycklig i det att tolkningen
  av en ortsvektor ${\bf x}$ d{\"a}rmed blir beroende av vilken
  stil p{\aa} typsnittet som anv{\"a}nts; i denna f{\"o}rel{\"a}sningsserie
  kommer vi att helt undvika denna f{\"o}rbryllande notation och
  ist{\"a}llet genomg{\aa}ende att i klartext skriva ut
  ``$({\bf x}-{\bf x}')$''.}
$$
  {\bf F}={{qq'}\over{4\pi\varepsilon_0}}
  \underbrace{
     {{({\bf x}-{\bf x}')}\over{|{\bf x}-{\bf x}'|^3}}
  }_{\sim 1/r^2},
$$
d{\"a}r
$$
  \varepsilon_0\approx8.854\times10^{12}\ {\rm F}/{\rm m}
$$
{\"a}r konstanten f{\"o}r den {\it elektriska permittiviteten i vakuum}, eller
kort och gott {\it vakuumpermittiviteten}.\sidx{Elektrisk
permittivitet}[Vakuumpermittivitet $\varepsilon_0$] Sj{\"a}lvfallet agerar denna
kraft reciprokt p{\aa} k{\"a}lladdningen\sidx{K{\"a}lladdning} $q'$, och vi kan
i uttrycket f{\"o}r \idx{Coulombs kraftlag} ovan helt sonika v{\"a}xla primmet
{\"o}ver till den andra positionen och direkt erh{\aa}lla
$$
  {\bf F}'={{q'q}\over{4\pi\varepsilon_0}}
     {{({\bf x}'-{\bf x})}\over{|{\bf x}'-{\bf x}|^3}}.
$$
\vfill\eject

\epsfig{figs/coulombsys.1}\noindent
Om vi ist{\"a}llet f{\"o}r en enskild punktladdning vid
k{\"a}llpunkten\sidx{K{\"a}llpunkt} betraktar ett
{\it system}\sidx{Punktladdning}[System av punktladdningar] av $N$ ladd\-ningar
$q'_k$ vid respektive k{\"a}llpositioner ${\bf x}'_k$ (med prim f{\"o}r
konsekvent notation f{\"o}r k{\"a}llpunkter), {\"a}r den totala kraften som
verkar p{\aa} laddningen $q$ vid observationspunkten uppenbarligen
$$
  \eqalign{
  {\bf F}&=\sum^N_{k=1} {\bf F}_k\cr
    &={{1}\over{4\pi\varepsilon_0}}\sum^N_{k=1}
     q q'_k {{({\bf x}-{\bf x}'_k)}\over{|{\bf x}-{\bf x}'_k|^3}}\cr
    &=q\bigg({{1}\over{4\pi\varepsilon_0}}\sum^N_{k=1}
     q'_k {{({\bf x}-{\bf x}'_k)}\over{|{\bf x}-{\bf x}'_k|^3}}\bigg)\cr
     &=q{\bf E}({\bf x}),
  }
$$
d{\"a}r vi {\it definierade} det elektriska f{\"a}ltet\sidx{Elektrisk
f{\"a}ltstyrka} ${\bf E}({\bf x})$ fr{\aa}n de $N$ punktladdningarna
som\numberedfootnote{Griffiths Ekv.~(2.4), sid.~61.}
$$
  {\bf E}({\bf x})\equiv{{1}\over{4\pi\varepsilon_0}}\sum^N_{k=1}
     q'_k {{({\bf x}-{\bf x}'_k)}\over{|{\bf x}-{\bf x}'_k|^3}}.
$$
Ett par observationer kring det elektriska f{\"a}ltet:
\medskip
\item{$\bullet$}{Formuleringen av uttrycket f{\"o}r det elektriska f{\"a}ltet
  {\"a}r {\it helt oberoende av testladdningen}\sidx{Testladdning} $q$. Detta
  kan tyckas
  sj{\"a}lvklart, men har en fundamental betydelse n{\"a}r vi alldeles strax
  kommer att generalisera f{\"a}ltbeskrivningen som Gauss lag. Specifikt, s{\aa}
  kan vi till ett elektriskt f{\"a}lt associerat med en viss grupp av laddningar
  det elektriska f{\"a}ltet associerat med en komplement{\"a}r grupp av
  laddningar; exempelvis kan vi betrakta ett totalt f{\"a}lt som uppbyggt
  dels av k{\"a}ll-laddningarna $q'_k$ dels av f{\"a}ltet som {\"a}r
  associerat till testladdningen $q$.}
\item{$\bullet$}{Summeringen av alla delbidrag vilar p{\aa} att vi kan
  betrakta varje laddning som oberoende av alla andra laddningar.
  I grund och botten antar vi att detta {\"a}r ett {\it linj{\"a}rt}
  problem. Mer om detta strax.}
\item{$\bullet$}{Denna m{\"o}jlighet att addera individuella del-l{\"o}sningar
  till en l{\"o}sning f{\"o}r det totala problemet brukar vi beteckna med
  {\it superpositionsprincipen},\sidx{Superpositionsprincipen} vilken {\"a}r
  generellt giltig enbart f{\"o}r {\it linj{\"a}ra problem}.}
\item{$\bullet$}{I detta antagande ligger implicit antagandet om att
  samtliga laddningar i problemet har fixa positioner som inte {\"a}ndras
  genom n{\"a}rvaro av andra laddningar. Vi kommer senare i kursen att se
  hur exempelvis ytladdningar p{\aa} ledande material justeras utifr{\aa}n
  elektrostatiken till att bilda f{\"o}rdelningar beroende p{\aa} externa
  faktorer (externa laddningar); i dessa fall {\"a}r dock den station{\"a}ra
  l{\"o}sningen i {\it steady-state} fortfarande giltig under
  superpositionsprincipen.}

\section{Vad {\"a}r ursprunget f{\"o}r denna v{\"a}xelverkan?
  Kan vi h{\"a}rleda Coulombs lag?}
Vi kan alla alltsedan sedan skoltiden i gymnasiet formen p{\aa} Coulombs
lag,\sidx{Coulombs kraftlag} och till slut s{\"a}tter sig denna k{\"a}nsla
f{\"o}r ``proportionalitet mot produkten av laddningarnas v{\"a}rde och
inversen av deras avst{\aa}nd i kvadrat'' i ryggm{\"a}rgen som en given,
praktiskt taget axiomatisk\numberedfootnote{{\it Axiomatisk}:
  ``sj{\"a}lvklart sann'' eller baserad p{\aa} ett eller flera axiom.
  Ett axiom {\"a}r en grund\-l{\"a}ggande sats eller ett p{\aa}st{\aa}ende
  som antas vara sant {\it utan bevis} och som anv{\"a}nds f{\"o}r att
  h{\"a}rleda andra sanningar i ett system.}
naturlag. Om man b{\"o}rjar fundera lite p{\aa} det och l{\"a}mnar den
intuition som vi erh{\aa}llit kring Coulombs lag, s{\aa} blir det dock
aningen konstigt.
Varf{\"o}r skulle laddningar {\"o}verhuvudtaget ha egenskaper som g{\"o}r att
de attraheras eller repelleras av varandra? Vi vet ju dessutom att dessa
laddningar (typiskt elek\-tron\-er och protoner) {\"a}r mycket sm{\aa}, och
{\"a}ven om vi har m{\"a}ngder av dessa elementarpartiklar i ett material,
varf{\"o}r skulle dessa sub-mikroskopiska partiklar ha en r{\"a}ckvidd som
str{\"a}cker sig {\"o}ver makroskopiska avst{\aa}nd?
\medskip
\quote{{\it Fr{\aa}gan blir d{\"a}rmed: Kan vi h{\"a}rleda Coulombs lag?}}
\medskip
\noindent
Svaret p{\aa} denna fr{\aa}ga {\"a}r utan tvekan ``ja'', men tyv{\"a}rr inte
med de klassiska verktyg som vi f{\"o}rfogar {\"o}ver i denna kurs.
Om vi skall f{\"o}rs{\"o}ka sammanfatta grunden i v{\"a}xelverkan mellan
laddade elementarpartiklar i ett par punkter, s{\aa} bygger denna p{\aa}
f{\"o}ljande.
\medskip
\item{$\bullet$}{V{\"a}xelverkan mellan elementarpartiklar som b{\"a}r laddning
  sker via s{\aa} kallade {\it virtuella fotoner}\sidx{Virtuella fotoner}, vilka
  skiljer sig fr{\aa}n v{\aa}ra ``vardagliga'' fotoner som {\"a}r synliga kvanta
  av ljus.
  Virtuella fotoner {\"a}r tempor{\"a}ra och existerar endast under det att
  v{\"a}xelverkan sker, och {\"a}r ``b{\"a}rarna av kraft'' f{\"o}r
  elektromagnetiska krafter.}
\item{$\bullet$}{Laddade partiklar som elektroner\sidx{Elektron} och
  protoner\sidx{Proton} utbyter virtuella fotoner mellan sig hela tiden,
  men med starkare intensitet n{\"a}r de kommer n{\"a}ra varandra.
  Detta konstanta utbyte till{\aa}ter partiklarna att ut{\"o}va krafter p{\aa}
  varandra, s{\aa} som attraktion\sidx{Attraktionskraft}[Elektrisk] eller
  repulsion.}
\item{$\bullet$}{Viktig po{\"a}ng: {\it Virtuella fotoner kan inte observeras
  direkt, till skillnad fr{\aa}n vanliga fotoner.}}
\item{$\bullet$}{Virtuella fotoner existerar och kan p{\aa}visas inom ramverket
  f{\"o}r kvantf{\"a}ltteori ({\it quantum field theory}). De {\"a}r dock inte
  v{\"a}ldefinierade utanf{\"o}r f{\"a}ltet f{\"o}r v{\"a}xelverkan mellan
  partiklar, och de kan (i likhet med vanliga fotoner) ``l{\aa}na'' eneri och
  moment under korta tidsrymder, helt enligt Heisenbergs os{\"a}kerhetsteori
  i kvantmekaniken.}
\medskip
\noindent
Med detta sagt, s{\aa} kommer vi fram{\"o}ver i kursen helt och h{\aa}llet att
betrakta Coulombs lag\sidx{Coulombs kraftlag} som en axiomatiskt given naturlag.
\vfill\eject

\section{Superpositionsprincipen}
\sidx{Superpositionsprincipen}
Att separat framtagna elektriska f{\"a}lt kan ses som och adderas som
komponenter av ett totalt elektriskt f{\"a}lt {\"a}r basen i vad vi kallar
{\it superpositionsprincipen}.\numberedfootnote{Generellt g{\"a}ller det
  under {\it superposition} att vi rent matematiskt har att g{\"o}ra med
  en funktion $F$ som har egenskaperna att $F(x+y)=F(x)+F(y)$ (additivitet)
  samt att $F(ax)=aF(x)$ (homogenitet). Ordet {\it superposition}
  h{\"a}rr{\"o}r i sig fr{\aa}n det senlatinska {\it superpositionem},
  med betydelse {\it att placera {\"o}ver}. Ordet kommer sig av {\it super}
  (``ovanf{\"o}r'') och {\it ponere} (``att placera'').}
Vi kan illustrera detta genom att godtyckligt dela upp det elektriska f{\"a}lt
vi nyss tog fram i tv{\aa} delar, med tv{\aa} separata grupper av de $N$
k{\"a}lladdningarna\sidx{K{\"a}lladdning}, enligt
$$
  {\bf E}({\bf x})\equiv
  \underbrace{
    {{1}\over{4\pi\varepsilon_0}}\sum^M_{k=1}
    q'_k {{({\bf x}-{\bf x}'_k)}\over{|{\bf x}-{\bf x}'_k|^3}}
  }_{\vbox{\hbox{Grupp 1 $\rightarrow\ {\bf E}_1({\bf x})$}\hbox{($M$ laddningar)}}}
  +\underbrace{
     {{1}\over{4\pi\varepsilon_0}}\sum^{N}_{k=M+1}
     q'_k {{({\bf x}-{\bf x}'_k)}\over{|{\bf x}-{\bf x}'_k|^3}}
   }_{\vbox{\hbox{Grupp 2 $\rightarrow\ {\bf E}_2({\bf x})$}\hbox{($N-M$ laddningar)}}}
   ={\bf E}_1({\bf x})+{\bf E}_2({\bf x}).
$$
\epsfig{figs/coulombsyspart.1}\noindent
Med andra ord, s{\aa} l{\"a}nge som inte de inb{\"o}rdes positionerna eller
laddningarna p{\aa}verkas av varandra (inga fria r{\"o}relser eller
tillf{\"o}rsel av str{\"o}mmar) och om vi r{\aa}kar ha en geometri som
p{\aa} n{\aa}got s{\"a}tt gynnar framtagandet av komponenter f{\"o}r det
totala elektriska f{\"a}ltet, s{\aa} st{\aa}r det oss fritt att {\it ber{\"a}kna
komponenterna separat och d{\"a}refter sammanfoga dessa till en total
l{\"o}sning f{\"o}r det elektriska f{\"a}ltet}. Denna superpositionsprincip
g{\"a}ller generellt f{\"o}r s{\aa} kallade {\it linj{\"a}ra problem}.

\section{Coulombintegralen - Coulombs generaliserade lag}
\sidx{Coulombs generaliserade lag}[Coulombintegralen]
S{\aa} l{\aa}ngt har vi endast betraktat system av {\it punktladdningar}, men
med superpositionsprincipen i beaktande {\"a}r naturligtvis steget minimalt att
att {\"o}verf{\"o}ra diskussionen till ett {\it kontinuum av laddning i rummet},
f{\"o}rdelad enligt en laddningst{\"a}thet\sidx{Elektrisk laddningst{\"a}thet}
$\rho({\bf x})$ (${\rm C}/{\rm m}^3$). I detta fall kan vi se det som att varje
volymelement $\Delta V_k$ i k{\"a}llpunkterna ${\bf x}'_k$ i rummet uppb{\"a}r
en laddning som vi kan se som en punktladdning, vilket i ett kontinuum
{\"o}verg{\aa}r till
$$
  q'_k=\rho({\bf x}')\Delta V_k\to\rho({\bf x}')dV'.
$$
D{\aa} alla dessa infinitesimala volymelement summeras upp erh{\aa}ller
vi {\it Coulombintegralen}\sidx{Integral}[Volymintegral]\sidx{Coulombs
generaliserade lag}[Coulombintegralen] f{\"o}r det elektriska f{\"a}ltet i en
observationspunkt\sidx{Observationspunkt} ${\bf x}$ som\numberedfootnote{Notera
  att Griffiths Ekv.~(2.8), sid.~63, betecknar {\it inte} denna som
  ``{\it Coulomb integral}'', vilket {\"a}r lite synd d{\aa} denna term
  p{\aa} pricken beskriver vad det handlar om.}
$$
  {\bf E}({\bf x})={{1}\over{4\pi\varepsilon_0}}\iiint_V\rho({\bf x}')
    {{({\bf x}-{\bf x}')}\over{|{\bf x}-{\bf x}'|^3}}\,dV'.
$$

\section{Vad {\"a}r egentligen po{\"a}ngen med att anv{\"a}nda elektriska
  f{\"a}lt?}
N{\"a}r vi nu framst{\"a}llt tv{\aa} olika representationer f{\"o}r att behandla
elektrostatiska f{\"a}ltproblem (och vi kommer alldeles strax att dessutom
introducera en {\it tredje} variant i och med potentialer), s{\aa} infinner
sig naturligtvis fr{\aa}gan varf{\"o}r vi inte bara kan n{\"o}ja oss med
Coulombs lag?\sidx{Coulombs kraftlag} Denna ger ju direkt kraften, s{\aa} vad
{\"a}r {\"o}verhuvud vitsen med att (till synes) bara komplicera saker och ting
genom att inf{\"o}ra elektriska f{\"a}lt och potentialer?

L{\aa}t oss med anledning av denna retoriska fr{\aa}ga sammanfatta ett par
anledningar till varf{\"o}r konceptet med f{\"a}lt underl{\"a}ttar f{\"o}r
oss.\numberedfootnote{Fundera g{\"a}rna igenom argumenten och kom p{\aa}
  fler exempel!}

\subsection{F{\"a}ltkonceptet eliminerar ``verkan p{\aa} distans''
  (``{\it action at a distance}'')}
\item{$\bullet$}{Enbart utifr{\aa}n Coulombs lag s{\aa} som den h{\"a}r
  fastst{\"a}llts, s{\aa} verkar varje enskild laddning p{\aa} alla andra
  laddningar momentant (direkt utan n{\aa}gon f{\"o}rdr{\"o}jning) oavsett
  avst{\aa}nd.}
\item{$\bullet$}{F{\"a}ltmodellen till{\aa}ter oss att l{\aa}ta en laddnings
  p{\aa}verkan att propagera genom rummet, med en fix hastighet som normalt
  {\"a}r ljushastigheten\sidx{Ljushastighet} eller denna nedskalad med
  motsvarande brytningsindex\sidx{Brytningsindex} f{\"o}r mediet i rummet.}

\subsection{Lokalitet och kausalitet}
\item{$\bullet$}{Vi kan med f{\"a}ltkonceptet analysera kraften p{\aa} en
  laddning {\it lokalt} vid punkten d{\"a}r den {\"a}r placerad, utan att
  beh{\"o}va bry oss om sj{\"a}lva k{\"a}llan. (Detta n{\"a}r vi v{\"a}l
  har ber{\"a}knat sj{\"a}lva f{\"a}ltet, sj{\"a}lvfallet!)}
\item{$\bullet$}{Detta g{\"o}r f{\"a}ltkonceptet kompatibelt med
  relativitetsteori\sidx{Relativitetsteori} och modern
  f{\"a}ltteori\sidx{F{\"a}ltteori}.}

\subsection{F{\"o}renkling av m{\aa}ngkropps-problem}
\item{$\bullet$}{Med $N$ laddningar blir ber{\"a}kningen av alla parvisa
  krafter som verkar i systemet snabbt mycket omst{\"a}ndligt, med $N(N-1)$
  v{\"a}xelverkningar.}
\item{$\bullet$}{Med f{\"a}ltkonceptet kan vi ber{\"a}kna f{\"a}ltet fr{\aa}n
  de $N$ laddningarna en g{\aa}ng f{\"o}r alla (med superpositionsprincipen!)
  och d{\"a}refter ta fram kraften p{\aa} varje laddning $q_k$ som
  ${\bf F}_k=q_k{\bf E}$.}

\subsection{F{\"a}lt kan existera oberoende av laddning}
\item{$\bullet$}{F{\"a}lt kan existera och breda ut sig utan (direkt)
  n{\"a}rvaro av laddning, till exempel elektromagnetiska v{\aa}gor.}
\item{$\bullet$}{Detta visar p{\aa} att f{\"a}lt {\"a}r inte bara ett
  bekv{\"a}mt matematiskt verktyg, de har en fysikalisk realitet bortom
  enbart varandes ett s{\"a}tt att dela upp \sidx{Coulombs kraftlag} i faktorer.}
\vfill\eject

\section{Gauss lag h{\"a}rledd i fyra steg}
Vi skulle h{\"a}r i princip bara kunna h{\"a}nvisa till den generella formen
av Gauss\numberedfootnote{Efter Carl Friedrich Gauss (1777--1855)\sidx{Gauss,
  Carl Friedrich (1777--1855)}, tysk matematiker och naturvetare, ofta kallad
  {\it Princeps Mathematicorum} (``matematikernas furste'').}
lag,\sidx{Gauss lag}[F{\"o}r elektrisk f{\"a}ltstyrka] men skulle d{\aa} riskera
att missa n{\aa}gra intressanta po{\"a}nger.
L{\aa}t oss d{\"a}rf{\"o}r ta detta fr{\aa}n grunden, med ut\-g{\aa}ngs\-punkt
i Coulombs lag, i fyra enkla steg som f{\"o}rhoppningsvis ger oss en djupare
fysikalisk f{\"o}rst{\aa}else f{\"o}r vad Gauss lag inneb{\"a}r.

\subsection{Definition: Elektriskt fl{\"o}de}
\sidx{Elektriskt fl{\"o}de}
F{\"o}r att diskutera resultaten i dessa fyra steg kommer vi att anv{\"a}nda
konceptet {\it elektriskt fl{\"o}de} $\Phi_{\rm E}$
(enhet: ${\rm V}\cdot{\rm m}$), som definieras som den integrerade
normalkomponenten av den elektriska f{\"a}ltstyrkan {\"o}ver en yta~$S$,
\epsfig{figs/elecflow.1}\noindent
Notera att det {\it inte finns n{\aa}got fysikaliskt fl{\"o}de associerat med
ett elektriskt f{\"a}lt}, men att vi i analogi med andra vektorf{\"a}lt inom
``riktiga fl{\"o}den'' hos gaser eller v{\"a}tskor t{\"a}nker oss ett fl{\"o}de
{\"a}ven f{\"o}r det elektriska f{\"a}ltet.\numberedfootnote{Notera {\"a}ven
  den lite olyckliga associationen man l{\"a}tt g{\"o}r till ``elektriskt
  fl{\"o}de'' som en slags str{\"o}m; det elektriska f{\"a}ltet i sig
  inneb{\"a}r ju dock ej n{\aa}gon explicit transport av laddning, vilket
  i s{\aa} fall skulle betecknas som en elektrisk str{\"o}m. F{\"o}rst i
  n{\"a}rvaro av fria laddningar har vi ett fysikaliskt fl{\"o}de i form
  av en str{\"o}m associerad med det elektriska fl{\"o}det.}
Vi kan lite handviftande s{\"a}ga att det elektriska fl{\"o}det {\"a}r ett
m{\aa}tt p{\aa} ``hur m{\aa}nga elektriska f{\"a}ltlinjer som passerar ut
genom ytan'', r{\"a}knat med tecken utifr{\aa}n ytans
normalvektor\sidx{Normalvektor}~${\bf n}$.
\vfill\eject

\subsection{Steg~1: Sf{\"a}risk symmetrisk omslutande yta och punktladdning}
\sidx{Gauss lag}[H{\"a}rledning fr{\aa}n Coulombs lag]
Antag att vi har en punktladdning\sidx{Punktladdning} $q'$ placerad i position
vid ortsvektorn ${\bf x}'$. D{\aa} vi ber{\"a}knar det elektriska
fl{\"o}det\sidx{Elektriskt fl{\"o}de} ut fr{\aa}n denna laddning, s{\aa} kan vi
se det som att fl{\"o}det h{\"a}rr{\"o}r fr{\aa}n en {\it k{\"a}lla}, och vi
kommer fram{\"o}ver i kursen ofta att relatera till ``k{\"a}lladdningar'' som
ger upphov till elektriska f{\"a}lt och
fl{\"o}den.\numberedfootnote{Genomg{\aa}ende kommer vi i kursen att
  s{\"a}tta ett prim ($'$) p{\aa} de objekt som vi betraktar som
  k{\"a}llor,\sidx{K{\"a}lladdning} som ${\bf x}'$, $dA'$ eller $dV'$,
  och l{\aa}ta observationspunkt och m{\"a}tetal vid denna vara oprimmade.
  Vi kommer {\"a}ven att genomg{\aa}ende explicit anv{\"a}nda
  Leibniz\sidx{Leibniz, Gottfried Wilhelm (1646--1716)} notation
  (${{\partial}\over{\partial x}}$, ${{d}\over{dx}}$) f{\"o}r derivator,
  f{\"o}r att inte r{\aa}ka in i situationer d{\"a}r ett prim kan misstas
  f{\"o}r en derivata.}

Vi l{\"a}gger en hypotetisk sf{\"a}r, eller ``virtuell sf{\"a}r'',\sidx{Virtuell
sf{\"a}r} $S$ med radien $|{\bf x}-{\bf x}'|=r=\hbox{konstant}$ runt
punktladdningen, med syfte att f{\"o}rs{\"o}ka ber{\"a}kna det totala elektriska
fl{\"o}det\sidx{Elektriskt fl{\"o}de} ut genom ytan.
\epsfig{figs/gauss-step-1.1}\noindent
Det elektriska fl{\"o}det ut genom den slutna sf{\"a}ren $S$, r{\"a}knat
gentemot ytans normalvektor ${\bf n}$, ges med utnyttjandet av den sf{\"a}riska
symmetrin som\sidx{Integral}[Ytintegral]
$$
  \eqalign{
    \Phi_{\rm E}&=\oiint_S {\bf E}({\bf x})\cdot d{\bf A}\cr
      &=\oiint_S
      \underbrace{
        {{q'}\over{4\pi\varepsilon_0}}
        {{({\bf x}-{\bf x}')}\over{|{\bf x}-{\bf x}'|^3}}
      }_{=E_r(r){\bf e}_r}\cdot
      \underbrace{
        {{({\bf x}-{\bf x}')}\over{|{\bf x}-{\bf x}'|}}dA
      }_{={\bf e}_r dA}
      =\big\{\hbox{Tag $r\equiv|{\bf x}-{\bf x}'|$}\big\}\cr
      &=\oiint_S{{q'}\over{4\pi\varepsilon_0 r^2}}\,dA
      ={{q'}\over{4\pi\varepsilon_0 r^2}}\oiint_S\,dA
      ={{q'}\over{4\pi\varepsilon_0 r^2}}4\pi r^2
      ={{q'}\over{\varepsilon_0}}\cr
  }
$$
Notera att det elektriska fl{\"o}det $\Phi_{\rm E}$ tolkat som ``hur m{\aa}nga
f{\"a}ltlinjer\sidx{Elektrisk f{\"a}ltlinje} som passerar ytan'' {\"a}r
obe\-ro\-ende av radien p{\aa} den omslutande sf{\"a}ren, och {\it enbart
beror p{\aa} styrkan av den inneslutna laddningen} (som kan vara positivt
eller negativt) samt vakuumpermittiviteten $\varepsilon_0$.\sidx{Elektrisk
permittivitet}[Vakuumpermittivitet $\varepsilon_0$]
Vi kommer att utnyttja detta faktum i n{\"a}sta steg.
\vfill\eject

\subsection{Steg~2: Godtycklig omslutande yta och punktladdning}
L{\aa}t oss nu generalisera Steg~1 genom att ers{\"a}tta den sf{\"a}riska
referensytan mot en yta av godtycklig form, med det enda kravet att vi
fortfarande omsluter punktladdningen $q'$. Denna nya yta beh{\"o}ver inte
vara, s{\"a}g, {\"o}verallt konvex, och vi till{\aa}ter {\"a}ven att ytan
exempelvis f{\aa}r vika sig runt sig sj{\"a}lv.
\epsfig{figs/gauss-step-2.1}\noindent
Om vi nu betraktar det elektriska fl{\"o}det som m{\aa}ttet p{\aa} ``hur
m{\aa}nga fl{\"o}deslinjer som passerar ytan, r{\"a}knat gentemot ytans
normalvektor'', s{\aa} ser vi att oavsett hur den omslutande ytan {\"a}r
formad s{\aa} kommer exakt lika m{\aa}nga sk{\"a}rningspunkter att erh{\aa}llas
som i Steg~1 under sf{\"a}risk symmetri. I de fall d{\"a}r en f{\"a}ltlinje
sk{\"a}r en del av den omslutande ytan som {\"a}r vikt omlott, s{\aa} kommer
varje sk{\"a}rning in i volymen att exakt motsvaras av en sk{\"a}rning ut
ifr{\aa}n ytan, med f{\"o}ljd att samtliga f{\"a}ltlinjer kommer att ha ett
resultat av exakt en sk{\"a}rning ut{\aa}t.

Ett s{\"a}tt att bokf{\"o}ringsm{\"a}ssigt s{\"a}tt hantera antalet
sk{\"a}rningar mellan varje f{\"a}ltlinje och ytan omslutande punktladdningen
{\"a}r att associera varje sk{\"a}rning {\it ut} fr{\aa}n ytan med v{\"a}rdet
$+1$ (${\bf n}\cdot{\bf E}>0$) och varje sk{\"a}rning {\it in} mot ytan med
v{\"a}rdet $-1$ (${\bf n}\cdot{\bf E}<0$). Om man f{\"o}ljer varje f{\"a}ltlinje
fr{\aa}n k{\"a}llan ut mot o{\"a}ndligheten s{\aa} inser man direkt att summan
av alla sk{\"a}rningar, och d{\"a}rmed varje f{\"a}ltlinjes bidrag till det
totala elektriska fl{\"o}det, {\"a}r {\it exakt en ekvivalent sk{\"a}rning
ut{\aa}t genom ytan}.

Notera att laddningar som ligger {\it utanf{\"o}r} volymen alltid kommer att ha
f{\"a}ltlinjer som har ett {\it j{\"a}mnt antal sk{\"a}rningar} med den
godtyckliga ytan $S$ (inklusive m{\"o}jligheten att en f{\"a}ltlinje inte
sk{\"a}r ytan alls). Slutsatsen av detta {\"a}r att {\it laddningar utanf{\"o}r
volymen alltid kommer att ha exakt noll i sina bidrag till det totala elektriska
fl{\"o}det genom den slutna ytan~$S$}.

Resultatet av detta resonemang {\"a}r att vi {\it fortfarande har exakt samma
totala elektriska fl{\"o}de} $\Phi_{\rm E}$ ut genom ytan som omsluter
punktladdningen $q'$. Med andra ord g{\"a}ller det {\"a}ven f{\"o}r en
{\it godtycklig} omslutande yta $S$ att\sidx{Integral}[Ytintegral]
$$
  \Phi_{\rm E}=\oiint_S {\bf E}({\bf x})\cdot d{\bf A}={{q'}\over{\varepsilon_0}},
$$
det vill s{\"a}ga att {\it det totala elektriska fl{\"o}det ut genom den slutna
ytan enbart best{\"a}ms av v{\"a}rdet p{\aa} den inneslutna punktladdningen}.
Vi kommer nu att anv{\"a}nda detta resultat i Steg~3.
\vfill\eject

\subsection{Steg~3: Godtycklig omslutande yta och system av punktladdningar}
Vi kommer nu att ytterligare generalisera f{\"o}reg{\aa}ende resultat genom att
betrakta ett system av $N$ statiska punktladdningar, fixerade i rummet och
liksom tidigare inneslutna av en hypotetisk godtyckig yta $S$ med samma
egenskaper som tidigare.
\epsfig{figs/gauss-step-3.1}\noindent
Vi ser att situationen f{\"o}r varje enskild laddning i sig {\"a}r identisk med
situationen som vi analyserade i Steg~2. Varje enskild innesluten punktladdning
(``k{\"a}lla'') $q'_k$ skulle d{\"a}rmed ge ett bidrag till det totala
elektriska fl{\"o}det som $q'_k/\varepsilon_0$.

Rent formellt {\"a}r det totala elektriska fl{\"o}det ut genom den slutna
generella ytan $S$ enligt superpositionsprincipen given
som\sidx{Integral}[Ytintegral]
$$
  \eqalign{
    \Phi_{\rm E}&=\oiint_S {\bf E}({\bf x})\cdot d{\bf A}
      =\big\{\hbox{Superpositionsprincipen}\big\}\cr
      &=\oiint_S \Bigg(\sum^N_{k=1}{\bf E}_k({\bf x})\Bigg)\cdot d{\bf A}
       =\big\{\hbox{Bryt ut summationen}\big\}\cr
      &=\sum^N_{k=1}\underbrace{
             \oiint_S {\bf E}_k({\bf x})\cdot d{\bf A}
           }_{=q'_k/\varepsilon_0}
       ={{1}\over{\varepsilon_0}}\sum^N_{k=1}q'_k
       ={{q'_{\rm tot}}/{\varepsilon_0}}.\cr
  }
$$
Slutsatsen av detta resultat {\"a}r att {\it det totala elektriska fl{\"o}det
$\Phi_{\rm E}$ ut genom den slutna ytan fortfarande enbart beror av den inneslutna
laddningen $q'_{\rm tot}$.}\sidx{Elektriskt fl{\"o}de} I det sista steget kommer
vi nu att gene\-ra\-li\-sera detta till godtyckliga kontinuerliga
laddningsf{\"o}rdelningar.
\vfill\eject

\subsection{Steg~4: Godtycklig omslutande yta och kontinuerlig
  laddningsf{\"o}rdelning}
Antag att vi nu ist{\"a}llet f{\"o}r diskreta punktladdningar $q'_k$ har en
laddningsf{\"o}rdelning $\rho({\bf x})$\sidx{Elektrisk laddningst{\"a}thet}
(enhet ${\rm C}/{\rm m}^3$) i en sluten (i rummet begr{\"a}nsad) volym $V$.
Denna laddningsf{\"o}rdelning kan variera kontinuerligt (j{\"a}mnt) i rummet
s{\aa}v{\"a}l som diskontinuerligt (stegvis), och vi l{\"a}mnar {\"a}ven
{\"o}ppet f{\"o}r att $\rho({\bf x})$ skall kunna tolkas som en f{\"o}rdelning
inneh{\aa}llande (Dirac-)delta-funktioner\sidx{Diracs delta-distribution
$\delta$} $\delta({\bf x}-{\bf x}'_k)$, med betydelsen av diskreta
punktladdningar placerade vid k{\"a}llpositioner ${\bf x}'_k$.

I termer av laddningsf{\"o}rdelningen $\rho({\bf x})$ uppb{\"a}r d{\aa} varje
infinitesimalt {\it k{\"a}llelement} med volymen $dV'$ vid k{\"a}llpositionen
${\bf x}'$ laddningen $dq'=\rho({\bf x}')dV'$, vilken vi kan betrakta som en
infinitesimal punktladdning.
\epsfig{figs/gauss-step-4.1}\noindent
Med det tidigare resultatet f{\"o}r framtagandet av det elektriska f{\"a}ltet
fr{\aa}n diskreta laddningar s{\aa} f{\"o}ljer det kontinuerliga fallet helt
analogt, och med anv{\"a}ndande av superpositionsprincipen f{\aa}r vi direkt
att den tidigare summan {\"o}ver diskreta laddningar i rummet ers{\"a}tts av
\sidx{Integral}[Volymintegral]\sidx{Integral}[Ytintegral]
volymintegralen\numberedfootnote{Notera att Griffiths (sid.~70)
  olyckligtvis anv{\"a}nder den udda och vilseledande notationen
  $Q_{\rm enc}=\int_V\rho\,d\tau$ f{\"o}r volymintegralen. Normalt
  anv{\"a}nder vi $\tau$ som integrationsvariabel i {\it tid}.}
$$
  \eqalign{
    \Phi_{\rm E}=\oiint_S {\bf E}({\bf x})\cdot d{\bf A}
       =\big\{
          \hbox{ Steg 3: ``${{1}\over{\varepsilon_0}}\sum^N_{k=1} dq'_k$'' }
        \big\}
       ={{1}\over{\varepsilon_0}}\iiint_V\rho({\bf x}')\,dV'
       =q_{\rm tot}/\varepsilon_0.
  }
$$
Vi har d{\"a}rmed kommit fram till den generella formen av {\it Gauss lag p{\aa}
integralform},\sidx{Gauss lag}[Integralform] vilken vi sam\-man\-fattar med
$$
  \oiint_S {\bf E}({\bf x})\cdot d{\bf A}
     ={{1}\over{\varepsilon_0}}\iiint_V\rho({\bf x}')\,dV'.
$$
Vi rekapitulerar att {\it Gauss lag har h{\"a}rletts enbart utifr{\aa}n
Coulombs klassiska lag f{\"o}r punktladd\-ningar samt superpositionsprincipen}.

D{\aa} Coulombs lag bygger p{\aa} ett (f{\"o}r v{\aa}r del) mer eller mindre
heuristiskt $1/r^2$-beroende f{\"o}r det elektriska f{\"a}ltets avtagande
fr{\aa}n punktladdningen, s{\aa} {\"a}r det l{\"a}tt att tro att detta beroende
p{\aa} n{\aa}got s{\"a}tt ocks{\aa} kommer att sl{\aa} in p{\aa} det elektriska
f{\"a}ltet fr{\aa}n en godtycklig laddningsf{\"o}rdelning.
Detta {\"a}r dock mer komplicerat till sin natur, och som vi senare kommer att
se i F{\"o}rel{\"a}sning~8 kring multipolutveckling av laddningsf{\"o}rdelningar
s{\aa} finns det en uppsj{\"o} av olika s{\aa} kallade {\it multipoler} med
olika grad av avklingande.
\vfill\eject

\section{Kontinuerliga laddningsf{\"o}rdelningar}
\sidx{Elektrisk laddningst{\"a}thet}[Volym-, yt-, linje-, punkt-]
Konceptet kontinuerlig laddningsf{\"o}rdelning kan sj{\"a}lvfallet appliceras
p{\aa} {\"a}ven trajektorior i rummet (linjeladdningar), ytor (ytladdningar).
I de fall d{\"a}r man har att g{\"o}ra med punktladdningar p{\aa} linjer, ytor
eller i volymer, s{\aa} kan dessa modelleras som spatiala delta-pulser
$\delta({\bf x}-{\bf x}')$ d{\"a}r ${\bf x}'$ {\"a}r positionen f{\"o}r
punktladdningen.
\epsfig{figs/chargetypes.1}

\section{Fr{\aa}n Gauss lag till Coulombs lag}
\sidx{Coulombs kraftlag}[H{\"a}rledd fr{\aa}n Gauss lag]
En enkel exercis f{\"o}r att f{\aa} en k{\"a}nsla f{\"o}r Gauss lag och vad det
inneb{\"a}r {\"a}r att g{\aa} andra v{\"a}gen, och fr{\aa}n v{\aa}rt sista
resultat h{\"a}rleda Coulombs lag f{\"o}r v{\"a}xelverkan mellan
punktladdningar.
Antag att vi placerat en ``k{\"a}lla'' i form av en punktladdning $q'$ i
k{\"a}llpunkten ${\bf x}'$. Sett som en distribution motsvarar detta
laddningst{\"a}theten\numberedfootnote{Griffiths anv{\"a}nder notationen
  $\delta^3({\bf x})$ f{\"o}r den skal{\"a}ra (Dirac-)delta-funktionen
  i tre dimensioner; exponentl{\"a}gets ``3'' {\"a}r dock on{\"o}digt
  d{\aa} det utifr{\aa}n argumentet {\"a}r uppenbart att det handlar
  om just tre dimensioner.}
$$
  \rho({\bf x})=q'\delta({\bf x}-{\bf x}').
$$
Antag vidare att vi l{\"a}gger en hypotetisk sf{\"a}r,\sidx{Virtuell sf{\"a}r}
vilket vi {\"a}ven kan beteckna som en ``virtuell sf{\"a}r''\sidx{Virtuell
sf{\"a}r} i detta tankeexperiment, med radien $r$ centrerad runt denna
punktladdning. Gauss lag f{\"o}r laddnings\-f{\"o}rdel\-ningar ger oss d{\aa}
med anv{\"a}ndande av sf{\"a}risk symmetri att
$$
  \underbrace{
    \oiint_S {\bf E}({\bf x})\cdot d{\bf A}
  }_{=E_r(r)4\pi r^2}
  ={{1}\over{\varepsilon_0}}
  \underbrace{
       \iiint_V\underbrace{
           q'\delta({\bf x}-{\bf x}')
       }_{\rho({\bf x})}\,dV'
  }_{=q'}
  \qquad\Rightarrow\qquad
  E_r(r) = {{q'}\over{4\pi\varepsilon_0 r^2}}.
$$
Ut{\"o}ver att visa p{\aa} hur man kan ``g{\aa} bakl{\"a}nges'' fr{\aa}n Gauss
lag till Coulombs lag, med det v{\"a}lk{\"a}nda $1/r^2$-beroendet p{\aa}
avst{\aa}nd fr{\aa}n punktk{\"a}llan, s{\aa} ger denna h{\"a}rledning ocks{\aa}
vid hand hur koefficienten ``$4\pi\varepsilon_0$'' dyker upp.
Vi kommer fram{\"o}ver i kursen att se hur denna koefficient dyker upp praktiskt
taget {\"o}verallt i elektrostatiken.
\vfill\eject

\section{Gauss lag p{\aa} differentialform}
S{\aa} som vi formulerat Gauss lag hittills {\"a}r den p{\aa} {\it integralform}.
Denna form f{\"o}ljer mer eller mindre intuitivt utifr{\aa}n s{\"a}ttet vi
h{\"a}rlett den, genom successiva generaliseringar d{\"a}r vi adderar
(integrerar) infinitesimala laddningar och via superpositionsprincipen
l{\"a}gger ihop delresultaten f{\"o}r f{\"a}lt och fl{\"o}den till en total
l{\"o}sning.
I m{\aa}nga fall {\"a}r det dock anv{\"a}ndbart att ist{\"a}llet ha Gauss lag
p{\aa} differentialform till hands, och en f{\"o}rdel med denna form {\"a}r att
vi samtidigt enklare ser hur vi kan se laddningst{\"a}theten $\rho({\bf x})$
som en {\it k{\"a}llf{\"o}rdelning} i elektrostatiska (och elektrodynamiska)
problem.

Vi applicerar f{\"o}rst divergensteoremet\numberedfootnote{Se exempelvis
  innerp{\"a}rmen p{\aa} Griffiths, ``{\it Divergence Theorem}''.
  {\AA}terigen, notera att Griffiths anv{\"a}nder den aningen olyckliga
  notationen ``$d\tau$'' f{\"o}r volymelement.}
p{\aa} det elektriska f{\"a}ltet, som f{\"o}r en godtycklig volym $V$ omsluten
av en yta $S$ lyder
$$
  \iiint_V\nabla\cdot{\bf E}\,dV=\oiint_S{\bf E}\cdot d{\bf A}.
$$
Vi har samtidigt visat att detta uttryck enligt Gauss lag uppenbarligen ges
som\sidx{Integral}[Sluten ytintegral]\numberedfootnote{Vi tar oss h{\"a}r
  friheten att skippa primmet p{\aa} k{\"a}llorna; det {\"a}r i sammanhanget
  uppenbart {\"o}ver vilka dom{\"a}ner integralerna skall tolkas.}
$$
  \oiint_S {\bf E}\cdot d{\bf A}
     ={{1}\over{\varepsilon_0}}\iiint_V\rho({\bf x})\,dV,
$$
vilket i sin tur betyder att
$$
  \iiint_V\nabla\cdot{\bf E}\,dV
     ={{1}\over{\varepsilon_0}}\iiint_V\rho({\bf x})\,dV.
$$
Eftersom denna relation g{\"a}ller f{\"o}r en {\it godtyckligt} vald volym $V$
och f{\"o}r en {\it godtycklig} laddnings\-t{\"a}t\-het $\rho({\bf x})$, s{\aa}
betyder detta att integranderna i v{\"a}nster- och h{\"o}gerledet m{\aa}ste vara
identiska, det vill s{\"a}ga att
$$
  \nabla\cdot{\bf E}={{\rho({\bf x})}\over{\varepsilon_0}},
$$
vilket sammanfattar {\it Gauss lag p{\aa} differentialform}.\sidx{Gauss lag}[Differentialform]
\vfill\eject

\section{Sammanfattning av F{\"o}rel{\"a}sning~1 -- Elektrostatik,
  superpositionsprincipen och Gauss lag}
\item{$\bullet$}{I elektrostatiken, och {\"a}ven senare i elektrodynamiken,
  har vi i huvudsak tre s{\"a}tt att betrakta v{\"a}xelverkan mellan
  station{\"a}ra laddningar: Som krafter mellan v{\"a}xelverkande laddningar,
  som f{\"a}lt och som potentialer. Mer om potentialer i kommande
  f{\"o}rel{\"a}sningar.}
\item{$\bullet$}{Coulombs kraftlag f{\"o}r punktladdningar lyder
  $$
    {\bf F}
      ={{qq'}\over{4\pi\varepsilon_0}}
        {{({\bf x}-{\bf x}')}\over{|{\bf x}-{\bf x}'|^3}}
      ={{qq'}\over{4\pi\varepsilon_0}}{{1}\over{r^2}}{\bf e}_r,
  $$
  d{\"a}r
  $$
    \varepsilon_0\approx8.854\times10^{12}\ {\rm F}/{\rm m}
  $$
  {\"a}r konstanten f{\"o}r den {\it elektriska permittiviteten i
  vakuum},\sidx{Elektrisk permittivitet}[Vakuumpermittivitet $\varepsilon_0$]
  eller kort och gott {\it vakuumpermittiviteten}. Denna permittivitet
  kommer genom kursen att h{\"a}nga med som en signatur p{\aa} allt som
  h{\"a}danefter kommer att h{\"a}rledas fr{\aa}n Coulombs lag.}
\item{$\bullet$}{Superpositionsprincipen inneb{\"a}r att vi kan
  {\it addera separata l{\"o}sningar} f{\"o}r elektriska f{\"a}lt
  och fl{\"o}den fr{\aa}n separata laddningar och laddningsf{\"o}rdelningar
  till en l{\"o}sning f{\"o}r det {\it totala} f{\"a}ltet och fl{\"o}det.
  Superpositionsprincipen g{\"a}ller {\it enbart f{\"o}r linj{\"a}ra
  problem}, i vilka inga potenser av det elektriska f{\"a}ltet finns
  i de grund\-l{\"a}ggande ekvationerna.}
\item{$\bullet$}{Det elektriska f{\"a}ltet fr{\aa}n ett system av
  punktladdningar $q'_k$ placerade i k{\"a}llpunkter ${\bf x}'_k$ {\"a}r
  $$
    {\bf E}({\bf x})={{1}\over{4\pi\varepsilon_0}}\sum^N_{k=1}
       q'_k {{({\bf x}-{\bf x}'_k)}\over{|{\bf x}-{\bf x}'_k|^3}},
  $$
  med kraften p{\aa} en testladdning (punktladdning) $q$ placerad i
  observationspunkten ${\bf x}$ given som
  $$
    {\bf F}=q{\bf E}({\bf x}).
  $$}
\item{$\bullet$}{Coulomb-integralen f{\"o}r ett kontinuum av laddning
  f{\"o}rdelad enligt en laddningst{\"a}thet $\rho({\bf x})$ lyder
  $$
    {\bf E}({\bf x})={{1}\over{4\pi\varepsilon_0}}\iiint_V\rho({\bf x}')
      {{({\bf x}-{\bf x}')}\over{|{\bf x}-{\bf x}'|^3}}\,dV'.
  $$}
\item{$\bullet$}{F{\"a}lt kan existera och breda ut sig {\it utan
  (direkt) n{\"a}rvaro av laddning}, till exempel elektromagnetiska
  v{\aa}gor. Detta visar p{\aa} att f{\"a}lt {\"a}r inte bara ett
  bekv{\"a}mt matematiskt verktyg, de har en fysikalisk realitet
  bortom enbart varandes ett s{\"a}tt att dela upp Coulombs kraftlag
  i faktorer.}
\item{$\bullet$}{Punktladdning $q$ (${\rm C}$),
  linjeladdning $\lambda$ (${\rm C}/{\rm m}$),
  ytladdning $\sigma$ (${\rm C}/{\rm m}^2$),
  volymladdning $\rho$ (${\rm C}/{\rm m}^3$).}
\item{$\bullet$}{Gauss lag p{\aa} integral- respektive differentialform:
  $$
    \oiint_S {\bf E}\cdot d{\bf A}
       ={{1}\over{\varepsilon_0}}\iiint_V\rho({\bf x})\,dV
    \qquad\Leftrightarrow\qquad
    \nabla\cdot{\bf E}={{\rho({\bf x})}\over{\varepsilon_0}}.
  $$\sidx{Gauss lag}[Integralform]\sidx{Gauss lag}[Differentialform]}
\item{$\bullet$}{Tolkningen av Gauss lag {\"a}r att {\it det elektriska
  fl{\"o}det $\Phi_{\rm E}$ ut genom en sluten yta $S$ ges som den av ytan
  inneslutna laddningen $q_{\rm tot}$ dividerad med vakuumpermittiviteten}
  \sidx{Elektrisk permittivitet}[Vakuumpermittivitet $\varepsilon_0$]
  $\varepsilon_0$, som
  $$
    \Phi_{\rm E}
       \equiv\oiint_S {\bf E}({\bf x})\cdot d{\bf A}
       ={{1}\over{\varepsilon_0}}\iiint_V\rho({\bf x})\,dV
       \equiv q_{\rm tot}/\varepsilon_0.
  $$}
\index
\bye
