%
% File: teaching/elmag/lect-01/lecture-01.tex [plain TeX code]
% Last change: October 6, 2025
%
% Lecture No 1 in the course ``Elektromagnetism II, 1TE626 (2025)'',
% held October 3, 2025, at Uppsala University, Sweden.
%
% Copyright (C) 2025, Fredrik Jonsson, under Gnu General Public License
% (GPL) v3. See the enclosed LICENSE for details.
%
% This program is free software: you can redistribute it and/or modify
% it under the terms of the GNU General Public License as published by
% the Free Software Foundation, either version 3 of the License, or
% (at your option) any later version.
%
% This program is distributed in the hope that it will be useful,
% but WITHOUT ANY WARRANTY; without even the implied warranty of
% MERCHANTABILITY or FITNESS FOR A PARTICULAR PURPOSE.  See the
% GNU General Public License for more details.
%
% You should have received a copy of the GNU General Public License
% along with this program.  If not, see <https://www.gnu.org/licenses/>.
%
\input macros/epsf.tex
\input macros/eplain.tex
\font\ninerm=cmr9
\font\twelvesc=cmcsc10 at 12 truept
\input amssym % to get the {\Bbb E} font (strikethrough E)
\def\lecture #1 {\hsize=150mm\hoffset=4.6mm\vsize=230mm\voffset=7mm
  \topskip=0pt\baselineskip=12pt\parskip=0pt\leftskip=0pt\parindent=15pt
  \headline={\ifnum\pageno>1\ifodd\pageno\rightheadline\else\leftheadline\fi
    \else\hfill\fi}
  \def\rightheadline{\tenrm{\it F\"orel\"asning #1}
    \hfil{\it Elektromagnetism II, 1TE626 (2025)}}
  \def\leftheadline{\tenrm{\it Elektromagnetism II, 1TE626 (2025)}
    \hfil{\it F\"orel\"asning #1}}
  \noindent~\vskip-60pt\hskip-40pt{\epsfbox{macros/UU_logo_color.eps}}
  \vskip-42pt\hfill\vbox{\hbox{{\it Elektromagnetism II, 1TE626 (2025)}}
  \hbox{{\it Lecture Notes, Fredrik Jonsson}}}\vskip 36pt
    \centerline{\twelvesc F\"orel\"asning #1}
  \vskip 24pt\noindent}
\def\section #1 {\medskip\goodbreak\noindent{\bf #1}
  \par\nobreak\smallskip\noindent}
\def\subsection #1 {\medskip\goodbreak\noindent{\it #1}
  \par\nobreak\smallskip\noindent}
\def\iint{\mathop{\int\kern-8pt\int}}
\def\iiint{\mathop{\int\kern-8pt\int\kern-8pt\int}}
\def\oiint{\mathop{\int\kern-8pt\int\kern-13.2pt{\bigcirc}}}
\def\Re{\mathop{\rm Re}\nolimits} % real part
\def\Im{\mathop{\rm Im}\nolimits} % imaginary part
\def\Tr{\mathop{\rm Tr}\nolimits} % quantum mechanical trace
\def\eqq{\mathop{\vbox{\hbox{\hskip2pt?}\vskip-6pt\hbox{=}}}}
%
% The 'boxit' macro from D.E. Knuths "The TeXbook", Exercise 21.3.
%
\def\boxit#1{\vbox{\hrule\hbox{\vrule\kern3pt
  \vbox{\kern3pt#1\kern3pt}\kern3pt\vrule}\hrule}}

\lecture{1}
\centerline{\twelvesc Elektrostatik, superpositionsprincipen och Gauss lag}
\centerline{Fredrik Jonsson, Uppsala Universitet, 3 november 2025}
\vskip24pt

\section{Elektrostatik}
Vi kommer i denna f{\"o}rsta f{\"o}rel{\"a}sning\numberedfootnote{Detta
  avsnitt har h{\"o}gst sannolikt ett visst {\"o}verlapp med tidigare
  kurser; anledningen till att vi trots allt v{\"a}ljer att inkludera
  fundamentan av v{\"a}xelverkan mellan laddningar {\"a}r att notation
  och beteckningar kommer att {\aa}terkomma frekvent genom kursen, samt
  att vissa detaljer som superpositionsprincipen kommer att vara essentiella
  f{\"o}r att kunna tillgodog{\"o}ra sig mer avancerade till{\"a}mpningar
  fram{\"o}ver i kursen.}
att behandla  elektrostatik, som {\"a}r l{\"a}ran om hur {\it station{\"a}ra}
elektriska laddningar v{\"a}xelverkar.
Som den mest fundamentala byggstenen i elektrostatiken har vi att tv{\aa} punktladdningar $q$ och $q'$, r{\"a}knade med sina respektive tecken f{\"o}r positiv eller negativ laddning och placerade i respektive {\it observationspunkten} ${\bf x}$ och {\it k{\"a}llpunkten} ${\bf x}'$, attraherar eller repellerar varandra med en kraft ${\bf F}$ genom Coulombs lag\numberedfootnote{Griffiths Ekv.~(2.1), sid.~60; laddningen i observationspunkten betecknas som ``test charge''. Observera ocks{\aa} Griffiths lite udda stil i notationen av ``${\bf x}-{\bf x}'$'' som ``scriptat ${\bf x}$'' (se Griffiths Ekv.~(2.2) p{\aa} sid.~60. Den notation som Griffiths anv{\"a}nder {\"a}r lite olycklig i det att tolkningen av en ortsvektor ${\bf x}$ d{\"a}rmed blir beroende av vilken stil p{\aa} typsnittet som anv{\"a}nts; i denna f{\"o}rel{\"a}sningsserie kommer vi att helt undvika denna f{\"o}rbryllande notation och ist{\"a}llet genomg{\aa}ende att i klartext skriva ut ``${\bf x}-{\bf x}'$''.}
$$
  {\bf F}={{1}\over{4\pi\varepsilon_0}}\underbrace{
     {{q q'}\over{|{\bf x}-{\bf x}'|^3}}({\bf x}-{\bf x}')
  }_{\sim O(1/|{\bf x}-{\bf x}'|^2)}
$$
d{\"a}r $\varepsilon_0\approx8.854\times10^{12}\ {\rm F}/{\rm m}$ {\"a}r konstanten f{\"o}r den {\it elektriska permittiviteten i vakuum}, eller kort och gott {\it vakuumpermittiviteten}.
\bigskip\centerline{\epsfbox{figs/coulomb.1}}
\medskip
\noindent
Om vi ist{\"a}llet f{\"o}r en enskild punktladdning vid k{\"a}llpunkten betraktar ett system av $N$ laddningar $q'_k$ vid respektive positioner ${\bf x}'_k$ (med prim f{\"o}r konsekvent notation f{\"o}r k{\"a}llpunkter), {\"a}r den totala kraften som verkar p{\aa} laddningen $q$ vid observationspunkten uppenbarligen
$$
  \eqalign{
  {\bf F}&=\sum^N_{k=1} {\bf F}_k\cr
    &={{1}\over{4\pi\varepsilon_0}}\sum^N_{k=1}
     q q'_k {{({\bf x}-{\bf x}'_k)}\over{|{\bf x}-{\bf x}'_k|^3}}\cr
    &=q\bigg({{1}\over{4\pi\varepsilon_0}}\sum^N_{k=1}
     q'_k {{({\bf x}-{\bf x}'_k)}\over{|{\bf x}-{\bf x}'_k|^3}}\bigg)\cr
     &=q{\bf E}({\bf x}),
  }
$$
d{\"a}r vi definierade det elektriska f{\"a}ltet ${\bf E}({\bf x})$
som\numberedfootnote{Griffiths Ekv.~(2.4), sid.~61.}
$$
  {\bf E}({\bf x})\equiv{{1}\over{4\pi\varepsilon_0}}\sum^N_{k=1}
     q'_k {{({\bf x}-{\bf x}'_k)}\over{|{\bf x}-{\bf x}'_k|^3}}.
$$
\bigskip\centerline{\epsfbox{figs/coulombsys.1}}
\medskip
\noindent
Ett par observationer kring det elektriska f{\"a}ltet:
\item{$\bullet$}{Formuleringen av uttrycket f{\"o}r det elektriska f{\"a}ltet {\"a}r {\it helt oberoende av testladdningen} $q$. Detta kan tyckas sj{\"a}lvklart, men har en fundamental betydelse n{\"a}r vi alldeles strax kommer att generalisera f{\"a}ltbeskrivningen som Gauss lag. Specifikt, s{\aa} kan vi till ett elektriskt f{\"a}lt associerat med en viss grupp av laddningar addera ett annat elektriskt f{\"a}lt, det senare associerat till en annan grupp av laddningar, exempelvis genom att betrakta ett totalt f{\"a}lt uppbyggt dels av k{\"a}ll-laddningarna $q'_k$ dels av f{\"a}ltet som {\"a}r associerat till testladdningen $q$.}
\item{$\bullet$}{Summeringen av alla delbidrag vilar p{\aa} att vi kan betrakta varje laddning som oberoende av alla andra laddningar. I grund och botten antar vi att detta {\"a}r ett {\it linj{\"a}rt} problem, d{\"a}r ekvationen f{\"o}r hur ......}
\item{$\bullet$}{Denna m{\"o}jlighet att addera individuella del-l{\"o}sningar till en l{\"o}sning f{\"o}r det totala problemet brukar vi beteckna med {\it superpositionsprincipen}, vilken generellt endast {\"a}r giltig f{\"o}r {\it linj{\"a}ra problem}.}
\item{$\bullet$}{I detta antagande ligger implicit antagandet om att samtliga laddningar i problemet har fixa positioner som inte {\"a}ndras genom n{\"a}rvaro av andra laddningar. Vi kommer senare i kursen att se hur exempelvis ytladdningar p{\aa} ledande material justeras utifr{\aa}n elektrostatiken till att bilda f{\"o}rdelningar beroende p{\aa} externa faktorer (externa laddningar); i dessa fall {\"a}r dock den station{\"a}ra l{\"o}sningen i {\it steady-state} fortfarande giltig under superpositionsprincipen.}


\section{Superpositionsprincipen}
Att separat framtagna elektriska f{\"a}lt kan ses som och adderas som komponenter av ett totalt elektriskt f{\"a}lt {\"a}r basen i vad vi kallar {\it superpositionsprincipen}\numberedfootnote{Generellt g{\"a}ller det under superposition att vi rent matematiskt har att g{\"o}ra med en funktion $F$ som har egenskaperna att $F(x+y)=F(x)+F(y)$ (additivitet) samt att $F(ax)=aF(x)$ (homogenitet). Ordet {\it superposition} h{\"a}rr{\"o}r fr{\aa}n senlatinska {\it superpositionem}, med betydelse {\it att placera {\"o}ver}. Ordet kommer sig av {\it super} (``ovanf{\"o}r'') och {\it ponere} (``att placera'').}.
Vi kan illustrera detta genom att godtyckligt dela upp det elektriska f{\"a}lt vi nyss tog fram i tv{\aa} delar, med tv{\aa} separata grupper av de $N$ k{\"a}lladdningarna, enligt
$$
  {\bf E}({\bf x})\equiv
  \underbrace{{{1}\over{4\pi\varepsilon_0}}\sum^M_{k=1}
     q'_k {{({\bf x}-{\bf x}'_k)}\over{|{\bf x}-{\bf x}'_k|^3}}}_{\vbox{\hbox{Grupp 1 $\rightarrow\ {\bf E}_1({\bf x})$}\hbox{($M$ laddningar)}}}
  +\underbrace{{{1}\over{4\pi\varepsilon_0}}\sum^{N}_{k=M+1}
     q'_k {{({\bf x}-{\bf x}'_k)}\over{|{\bf x}-{\bf x}'_k|^3}}}_{\vbox{\hbox{Grupp 2 $\rightarrow\ {\bf E}_2({\bf x})$}\hbox{($N-M$ laddningar)}}}={\bf E}_1({\bf x})+{\bf E}_2({\bf x}).
$$
\bigskip\centerline{\epsfbox{figs/coulombsyspart.1}}
\medskip
\noindent
Med andra ord, s{\aa} l{\"a}nge som inte de inb{\"o}rdes positionerna eller laddningarna p{\aa}verkas av varandra (inga fria r{\"o}relser eller tillf{\"o}rsel av str{\"o}mmar) och om vi r{\aa}kar ha en geometri som p{\aa} n{\aa}got s{\"a}tt gynnar framtagandet av komponenter f{\"o}r det totala elektriska f{\"a}ltet, s{\aa} st{\aa}r det oss fritt att ber{\"a}kna komponenterna separat och d{\"a}refter sammanfoga dessa till en total l{\"o}sning f{\"o}r det elektriska f{\"a}ltet. Denna superpositionsprincip g{\"a}ller generellt f{\"o}r s{\aa} kallade {\it linj{\"a}ra problem}.

\section{Vad {\"a}r egentligen po{\"a}ngen med att anv{\"a}nda elektriska f{\"a}lt?}
N{\"a}r vi nu framst{\"a}llt tre olika representationer f{\"o}r att behandla elektrostatiska f{\"a}ltproblem, s{\aa} infinner sig naturligtvis fr{\aa}gan varf{\"o}r vi inte bara kan n{\"o}ja oss med Coulombs lag? Denna ger ju direkt kraften, s{\aa} vad {\"a}r {\"o}verhuvud vitsen med att (till synes) bara komplicera saker och ting genom att inf{\"o}ra elektriska f{\"a}lt och potentialer?
\vfill\eject

\section{Gauss lag h{\"a}rledd i fyra steg}
Vi skulle h{\"a}r i princip bara kunna h{\"a}nvisa till den generella formen av Gauss lag, men skulle d{\aa} riskera att missa n{\aa}gra intressanta po{\"a}nger. L{\aa}t oss d{\"a}rf{\"o}r ta detta fr{\aa}n grunden, med ut\-g{\aa}ngs\-punkt i Coulombs lag, i fyra enkla steg som f{\"o}rhoppningsvis ger oss en djupare fysikalisk f{\"o}rst{\aa}else f{\"o}r vad Gauss lag inneb{\"a}r.

\subsection{Definition: Elektriskt fl{\"o}de}
F{\"o}r att diskutera resultaten i dessa fyra steg kommer vi att anv{\"a}nda konceptet {\it elektriskt fl{\"o}de} $\Phi_{\rm E}$ (enhet: ${\rm V}\cdot{\rm m}$), som definieras som den integrerade normalkomponenten av den elektriska f{\"a}ltstyrkan {\"o}ver en yta~$A$,
\bigskip\centerline{\epsfbox{figs/elecflow.1}}
\medskip
\noindent
Notera att det {\it inte finns n{\aa}got fysikaliskt fl{\"o}de associerat med ett elektriskt f{\"a}lt}, men att vi i analogi med andra vektorf{\"a}lt inom ``riktiga fl{\"o}den'' hos gaser eller v{\"a}tskor t{\"a}nker oss ett fl{\"o}de {\"a}ven f{\"o}r det elektriska f{\"a}ltet.\numberedfootnote{Notera {\"a}ven den lite olyckliga associationen man l{\"a}tt g{\"o}r till ``elektriskt fl{\"o}de'' som en slags str{\"o}m; det elektriska f{\"a}ltet i sig inneb{\"a}r ju dock ej n{\aa}gon explicit transport av laddning, vilket i s{\aa} fall skulle betecknas som en elektrisk str{\"o}m. F{\"o}rst i n{\"a}rvaro av fria laddningar har vi ett fysikaliskt fl{\"o}de i form av en str{\"o}m associerad med det elektriska fl{\"o}det.}
Vi kan lite handviftande s{\"a}ga att det elektriska fl{\"o}det {\"a}r ett m{\aa}tt p{\aa} ``hur m{\aa}nga elektriska f{\"a}ltlinjer som passerar ut genom ytan'', r{\"a}knat med tecken utifr{\aa}n ytans normalvektor~${\bf n}$.
\vfill\eject

\subsection{Steg 1: Sf{\"a}risk symmetrisk omslutande yta och punktladdning}
Antag att vi har en punktladdning $q'$ placerad i position vid ortsvektorn ${\bf x}'$. D{\aa} vi ber{\"a}knar det elektriska fl{\"o}det ut fr{\aa}n denna laddning, s{\aa} kan vi se det som att fl{\"o}det h{\"a}rr{\"o}r fr{\aa}n en {\it k{\"a}lla}, och vi kommer fram{\"o}ver i kursen ofta att relatera till ``k{\"a}lladdningar'' som ger upphov till elektriska f{\"a}lt och fl{\"o}den.\numberedfootnote{Genomg{\aa}ende kommer vi i kursen att s{\"a}tta ett prim ($'$) p{\aa} de objekt som vi betraktar som k{\"a}llor, som ${\bf x}'$, $dA'$ eller $dV'$, och l{\aa}ta observationspunkt och m{\"a}tetal vid denna vara oprimmade. Vi kommer {\"a}ven att genomg{\aa}ende explicit anv{\"a}nda Leibniz notation (${{\partial}\over{\partial x}}$, ${{d}\over{dx}}$) f{\"o}r derivator, f{\"o}r att inte r{\aa}ka in i situationer d{\"a}r ett prim kan misstas f{\"o}r en derivata.}

Vi l{\"a}gger en hypotetisk sf{\"a}r $A$ med radien $|{\bf x}-{\bf x}'|=r=\hbox{konstant}$ runt punktladdningen, med syfte att f{\"o}rs{\"o}ka ber{\"a}kna det totala elektriska fl{\"o}det ut genom ytan.
\bigskip\centerline{\epsfbox{figs/gauss-step-1.1}}
\medskip
\noindent
Det elektriska fl{\"o}det ut genom den slutna sf{\"a}ren $A$, r{\"a}knat gentemot ytans normalvektor ${\bf n}$, ges med utnyttjandet av den sf{\"a}riska symmetrin som
$$
  \eqalign{
    \Phi_{\rm E}&=\oiint_A {\bf E}({\bf x})\cdot d{\bf A}\cr
      &=\oiint_A
      \underbrace{
        {{q'({\bf x}-{\bf x}')}\over{4\pi |{\bf x}-{\bf x}'|^3}}
      }_{=E_r(r){\bf e}_r}\cdot
      \underbrace{
        {{({\bf x}-{\bf x}')}\over{|{\bf x}-{\bf x}'|}}dA
      }_{={\bf e}_r dA}
      =\big\{\hbox{Tag $r\equiv|{\bf x}-{\bf x}'|$}\big\}\cr
      &=\oiint_A{{q'}\over{4\pi\varepsilon_0 r^2}}\,dA
      ={{q'}\over{4\pi\varepsilon_0 r^2}}\oiint_A\,dA
      ={{q'}\over{4\pi\varepsilon_0 r^2}}4\pi r^2
      ={{q'}\over{\varepsilon_0}}\cr
  }
$$
Notera att det elektriska fl{\"o}det $\Phi_{\rm E}$ tolkat som ``hur m{\aa}nga f{\"a}ltlinjer som passerar ytan'' {\"a}r obe\-ro\-ende av radien p{\aa} den omslutande sf{\"a}ren, och {\it enbart beror p{\aa} styrkan av den inneslutna laddningen} (som kan vara positivt eller negativt) samt vakuumpermittiviteten $\varepsilon_0$. Vi kommer att utnyttja detta faktum i n{\"a}sta steg.
\vfill\eject

\subsection{Steg 2: Godtycklig omslutande yta och punktladdning}
L{\aa}t oss nu generalisera Steg 1 genom att ers{\"a}tta den sf{\"a}riska referensytan mot en yta av godtycklig form, med det enda kravet att vi fortfarande omsluter punktladdningen $q'$. Denna nya yta beh{\"o}ver inte vara, s{\"a}g, {\"o}verallt konvex, och vi till{\aa}ter {\"a}ven att ytan exempelvis f{\aa}r vika sig runt sig sj{\"a}lv.
\bigskip\centerline{\epsfbox{figs/gauss-step-2.1}}
\medskip
\noindent
Om vi nu betraktar det elektriska fl{\"o}det som m{\aa}ttet p{\aa} ``hur m{\aa}nga fl{\"o}deslinjer som passerar ytan, r{\"a}knat gentemot ytans normalvektor'', s{\aa} ser vi att oavsett hur den omslutande ytan {\"a}r formad s{\aa} kommer exakt lika m{\aa}nga sk{\"a}rningspunkter att erh{\aa}llas som i Steg 1 under sf{\"a}risk symmetri. I de fall d{\"a}r en f{\"a}ltlinje sk{\"a}r en del av den omslutande ytan som {\"a}r vikt omlott, s{\aa} kommer varje sk{\"a}rning in i volymen att exakt motsvaras av en sk{\"a}rning ut ifr{\aa}n ytan, med f{\"o}ljd att samtliga f{\"a}ltlinjer kommer att ha ett resultat av exakt en sk{\"a}rning ut{\aa}t.

Ett s{\"a}tt att bokf{\"o}ringsm{\"a}ssigt s{\"a}tt hantera antalet sk{\"a}rningar mellan varje f{\"a}ltlinje och ytan omslutande punktladdningen {\"a}r att associera varje sk{\"a}rning {\it ut} fr{\aa}n ytan med v{\"a}rdet $+1$ (${\bf n}\cdot{\bf E}>0$) och varje sk{\"a}rning {\it in} mot ytan med v{\"a}rdet $-1$ (${\bf n}\cdot{\bf E}<0$). Om man f{\"o}ljer varje f{\"a}ltlinje fr{\aa}n k{\"a}llan ut mot o{\"a}ndligheten s{\aa} inser man direkt att summan av alla sk{\"a}rningar, och d{\"a}rmed varje f{\"a}ltlinjes bidrag till det totala elektriska fl{\"o}det, {\"a}r {\it exakt en ekvivalent sk{\"a}rning ut{\aa}t genom ytan}.

Notera att laddningar som ligger {\it utanf{\"o}r} volymen alltid kommer att ha f{\"a}ltlinjer som har ett {\it j{\"a}mnt antal sk{\"a}rningar} med den godtyckliga ytan $A$ (inklusive m{\"o}jligheten att en f{\"a}ltlinje inte sk{\"a}r ytan alls). Slutsatsen av detta {\"a}r att {\it laddningar utanf{\"o}r volymen alltid kommer att ha exakt noll i sina bidrag till det totala elektriska fl{\"o}det genom den slutna ytan~$A$}.

Resultatet av detta resonemang {\"a}r att vi fortfarande har exakt samma totala elektriska fl{\"o}de $\Phi_{\rm E}$ ut genom ytan som omsluter punktladdningen $q'$. Med andra ord g{\"a}ller det f{\"o}r en godtycklig omslutande yta $A$ att
$$
  \Phi_{\rm E}=\oiint_A {\bf E}({\bf x})\cdot d{\bf A}={{q'}\over{\varepsilon_0}},
$$
det vill s{\"a}ga att {\it det totala elektriska fl{\"o}det ut genom den slutna ytan enbart best{\"a}ms av v{\"a}rdet p{\aa} den inneslutna punktladdningen}. Vi kommer nu att anv{\"a}nda detta resultat i Steg~3.
\vfill\eject

\subsection{Steg 3: Godtycklig omslutande yta och system av punktladdningar}
Vi kommer nu att ytterligare generalisera f{\"o}reg{\aa}ende resultat genom att betrakta ett system av $N$ statiska punktladdningar, fixerade i rummet och liksom tidigare inneslutna av en hypotetisk godtyckig yta $A$ med samma egenskaper som tidigare.
\bigskip\centerline{\epsfbox{figs/gauss-step-3.1}}
\medskip
\noindent
Vi ser att situationen f{\"o}r varje enskild laddning i sig {\"a}r identisk med situationen som vi analyserade i Steg~2. Varje enskild innesluten punktladdning (``k{\"a}lla'') $q'_k$ skulle d{\"a}rmed ge ett bidrag till det totala elektriska fl{\"o}det som $q'_k/\varepsilon_0$.

Rent formellt {\"a}r det totala elektriska fl{\"o}det ut genom den slutna generella ytan $A$ enligt superpositionsprincipen given som
$$
  \eqalign{
    \Phi_{\rm E}&=\oiint_A {\bf E}({\bf x})\cdot d{\bf A}
      =\big\{\hbox{Superpositionsprincipen}\big\}\cr
      &=\oiint_A \Bigg(\sum^N_{k=1}{\bf E}_k({\bf x})\Bigg)\cdot d{\bf A}
       =\big\{\hbox{Bryt ut summationen}\big\}\cr
      &=\sum^N_{k=1}\underbrace{
             \oiint_A {\bf E}_k({\bf x})\cdot d{\bf A}
           }_{=q'_k/\varepsilon_0}
       ={{1}\over{\varepsilon_0}}\sum^N_{k=1}q'_k
       ={{q'_{\rm tot}}/{\varepsilon_0}}.\cr
  }
$$
Slutsatsen av detta resultat {\"a}r att {\it det totala elektriska fl{\"o}det $\Phi_{\rm E}$ ut genom den slutna ytan fortfarande enbart beror av den inneslutna laddningen $q'_{\rm tot}$.} I det sista steget kommer vi nu att gene\-ra\-li\-sera detta till godtyckliga kontinuerliga laddningsf{\"o}rdelningar.
\vfill\eject

\subsection{Steg 4: Godtycklig omslutande yta och kontinuerlig laddningsf{\"o}rdelning}
Antag att vi nu ist{\"a}llet f{\"o}r diskreta punktladdningar $q'_k$ har en laddningsf{\"o}rdelning $\rho({\bf x})$ (enhet ${\rm C}/{\rm m}^3$) i en sluten (i rummet begr{\"a}nsad) volym $V$. Denna laddningsf{\"o}rdelning kan variera kontinuerligt (j{\"a}mnt) i rummet s{\aa}v{\"a}l som diskontinuerligt (stegvis), och vi l{\"a}mnar {\"a}ven {\"o}ppet f{\"o}r att $\rho({\bf x})$ skall kunna tolkas som en f{\"o}rdelning inneh{\aa}llande (Dirac-)delta-funktioner $\delta({\bf x}-{\bf x}'_k)$, med betydelsen av diskreta punktladdningar placerade vid k{\"a}llpositioner ${\bf x}'_k$.

I termer av laddningsf{\"o}rdelningen $\rho({\bf x})$ uppb{\"a}r d{\aa} varje infinitesimalt {\it k{\"a}llelement} med volymen $dV'$ vid k{\"a}llpositionen ${\bf x}'$ laddningen $dq'=\rho({\bf x}')dV'$, vilken vi kan betrakta som en infinitesimal punktladdning.
\bigskip\centerline{\epsfbox{figs/gauss-step-4.1}}
\medskip
\noindent
Med det tidigare resultatet f{\"o}r framtagandet av det elektriska f{\"a}ltet fr{\aa}n diskreta laddningar s{\aa} f{\"o}ljer det kontinuerliga fallet helt analogt, och med anv{\"a}ndande av superpositionsprincipen f{\aa}r vi direkt att den tidigare summan {\"o}ver diskreta laddningar i rummet ers{\"a}tts av volymsintegralen\numberedfootnote{Notera att Griffiths (sid.~70) olyckligtvis anv{\"a}nder den udda och vilseledande notationen $Q_{\rm enc}=\int_V\rho\,d\tau$ f{\"o}r volymintegralen. Normalt anv{\"a}nder vi $\tau$ som integrationsvariabel i {\it tid}.}
$$
  \eqalign{
    \Phi_{\rm E}=\oiint_A {\bf E}({\bf x})\cdot d{\bf A}
       =\big\{
          \hbox{ Steg 3: ``${{1}\over{\varepsilon_0}}\sum^N_{k=1} dq'_k$'' }
        \big\}
       ={{1}\over{\varepsilon_0}}\iiint_V\rho({\bf x}')\,dV'.
       =q_{\rm tot}/\varepsilon_0.
  }
$$
Vi har d{\"a}rmed kommit fram till den generella formen av Gauss lag p{\aa} integralform, vilken vi sam\-man\-fattar med
$$
  \oiint_A {\bf E}({\bf x})\cdot d{\bf A}
     ={{1}\over{\varepsilon_0}}\iiint_V\rho({\bf x}')\,dV'.
$$
Vi rekapitulerar att Gauss lag har h{\"a}rletts enbart utifr{\aa}n Coulombs klassiska lag f{\"o}r punktladdningar samt superpositionsprincipen.

D{\aa} Coulombs lag bygger p{\aa} ett (f{\"o}r v{\aa}r del) mer eller mindre heuristiskt $1/r^2$-beroende f{\"o}r det elektriska f{\"a}ltets avtagande fr{\aa}n punktladdningen, s{\aa} {\"a}r det l{\"a}tt att tro att detta beroende p{\aa} n{\aa}got s{\"a}tt ocks{\aa} kommer att sl{\aa} in p{\aa} det elektriska f{\"a}ltet fr{\aa}n en godtycklig laddningsf{\"o}rdelning. Detta {\"a}r dock mer komplicerat till sin natur, och som vi senare kommer att se i f{\"o}rel{\"a}sningen kring multipolutveckling av laddningsf{\"o}rdelningar s{\aa} finns det en uppsj{\"o} av olika s{\aa} kallade {\it multipoler} med olika grad av avklingande.
\vfill\eject

\section{Kontinuerliga laddningsf{\"o}rdelningar}
Konceptet kontinuerlig laddningsf{\"o}rdelning kan sj{\"a}lvfallet appliceras p{\aa} {\"a}ven trajektorior i rummet (linjeladdningar), ytor (ytladdningar). I de fall d{\"a}r man har att g{\"o}ra med punktladdningar p{\aa} linjer, ytor eller i volymer, s{\aa} kan dessa modelleras som spatiala delta-pulser $\delta({\bf x}-{\bf x}')$ d{\"a}r ${\bf x}'$ {\"a}r positionen f{\"o}r punktladdningen.
\bigskip\centerline{\epsfbox{figs/chargetypes.1}}
\medskip
\noindent

\section{Fr{\aa}n Gauss lag till Coulombs lag}
En enkel exercis f{\"o}r att f{\aa} en k{\"a}nsla f{\"o}r Gauss lag och vad det inneb{\"a}r {\"a}r att g{\aa} andra v{\"a}gen, och fr{\aa}n v{\aa}rt sista resultat h{\"a}rleda Coulombs lag f{\"o}r v{\"a}xelverkan mellan punktladdningar.
Antag att vi placerat en ``k{\"a}lla'' i form av en punktladdning $q'$ i k{\"a}llpunkten ${\bf x}'$. Sett som en distribution motsvarar detta laddningst{\"a}theten\numberedfootnote{Griffiths anv{\"a}nder notationen $\delta^3({\bf x})$ f{\"o}r den skal{\"a}ra (Dirac-)delta-funktionen i tre dimensioner; exponentl{\"a}gets ``3'' {\"a}r dock on{\"o}digt d{\aa} det utifr{\aa}n argumentet {\"a}r uppenbart att det handlar om just tre dimensioner.}
$$
  \rho({\bf x})=q'\delta({\bf x}-{\bf x}').
$$
Antag vidare att vi l{\"a}gger en hypotetisk sf{\"a}r med radien $r$ centrerad runt denna punktladdning. Gauss lag f{\"o}r laddningsf{\"o}rdelningar ger oss d{\aa} med anv{\"a}ndande av sf{\"a}risk symmetri att
$$
  \underbrace{
    \oiint_A {\bf E}({\bf x})\cdot d{\bf A}
  }_{=E_r(r)4\pi r^2}
  ={{1}\over{\varepsilon_0}}
  \underbrace{
       \iiint_V\underbrace{
           q'\delta({\bf x}-{\bf x}')
       }_{\rho({\bf x})}\,dV'
  }_{=q'}
  \qquad\Rightarrow\qquad
  E_r(r) = {{q'}\over{4\pi\varepsilon_0 r^2}}.
$$
Ut{\"o}ver att visa p{\aa} hur man kan ``g{\aa} bakl{\"a}nges'' fr{\aa}n Gauss lag till Coulombs lag, med det v{\"a}lk{\"a}nda $1/r^2$-beroendet p{\aa} avst{\aa}nd fr{\aa}n punktk{\"a}llan, s{\aa} ger denna h{\"a}rledning ocks{\aa} vid hand hur koefficienten ``$4\pi\varepsilon_0$'' dyker upp. Vi kommer fram{\"o}ver i kursen att se hur denna koefficient dyker upp praktiskt taget {\"o}verallt i elektrostatiken.

\section{Gauss lag p{\aa} differentialform}
S{\aa} som vi formulerat Gauss lag hittills {\"a}r den p{\aa} {\it integralform}. Denna form f{\"o}ljer mer eller mindre intuitivt utifr{\aa}n s{\"a}ttet vi h{\"a}rlett den, genom successiva generaliseringar d{\"a}r vi adderar (integrerar) infinitesimala laddningar och via superpositionsprincipen l{\"a}gger ihop delresultaten f{\"o}r f{\"a}lt och fl{\"o}den till en total l{\"o}sning.
I m{\aa}nga fall {\"a}r det dock anv{\"a}ndbart att ist{\"a}llet ha Gauss lag p{\aa} differentialform till hands, och en f{\"o}rdel med denna form {\"a}r att vi samtidigt enklare ser hur vi kan se laddningst{\"a}theten $\rho({\bf x})$ som en {\it k{\"a}llf{\"o}rdelning} i elektrostatiska (och elektrodynamiska) problem.

Genom att applicera divergensteoremet\numberedfootnote{Se exempelvis innerp{\"a}rmen p{\aa} Griffiths. {\AA}terigen, notera att Griffiths anv{\"a}nder den aningen olyckliga notationen ``$d\tau$'' f{\"o}r volymelement.} p{\aa} det elektriska f{\"a}ltet,
$$
  \iiint_V\nabla\cdot{\bf E}\,dV=\oiint_A{\bf E}\cdot d{\bf A},
$$
Vi har samtidigt att detta uttryck enligt Gauss lag uppenbarligen ges som\numberedfootnote{Vi tar oss h{\"a}r friheten att droppa primmet p{\aa} k{\"a}llorna; det {\"a}r i sammanhanget uppenbart {\"o}ver vilken dom{\"a}n integralerna skall tolkas.}
$$
  \oiint_A {\bf E}\cdot d{\bf A}
     ={{1}\over{\varepsilon_0}}\iiint_V\rho({\bf x})\,dV,
$$
vilket i sin tur betyder att
$$
  \iiint_V\nabla\cdot{\bf E}\,dV
     ={{1}\over{\varepsilon_0}}\iiint_V\rho({\bf x})\,dV.
$$
Eftersom denna relation g{\"a}ller f{\"o}r en godtyckligt vald volym $V$ och f{\"o}r en godtycklig laddnings\-t{\"a}t\-het $\rho({\bf x})$, s{\aa} betyder detta att integranderna i v{\"a}nster- och h{\"o}gerledet m{\aa}ste vara identiska, det vill s{\"a}ga
$$
  \nabla\cdot{\bf E}={{\rho({\bf x})}\over{\varepsilon_0}},
$$
vilket sammanfattar {\it Gauss lag p{\aa} differentialform}.

%\section{Sf{\"a}riska homogena laddningar}
%En sf{\"a}riskt formad homogen laddningsf{\"o}rdelning kan utifr{\aa}n sett betraktas som om den vore en punktladdning placerad i centrum av sf{\"a}ren. Dock kan man st{\"a}lla sig fr{\aa}gan varf{\"o}r det just {\"a}r s{\aa}? Borde inte delar av laddningsf{\"o}rdelningen som {\"a}r lokaliserad n{\"a}rmre en observationspunkt agera kraftigare p{\aa} grund av $1/r^2$-beroendet?

\section{Sammanfattning}
\item{$\bullet$}{I elektrostatiken, och {\"a}ven senare i elektrodynamiken, har vi tre s{\"a}tt att betrakta v{\"a}xelverkan mellan laddningar: Som krafter mellan laddningar, som f{\"a}lt och som potentialer. Mer om potentialer i kommande f{\"o}rel{\"a}sningar.}
\item{$\bullet$}{Superpositionsprincipen inneb{\"a}r att vi kan addera separata l{\"o}sningar f{\"o}r elektriska f{\"a}lt och fl{\"o}den fr{\aa}n separata laddningar och laddningsf{\"o}rdelningar till en l{\"o}sning f{\"o}r det totala f{\"a}ltet och fl{\"o}det. Superpositionsprincipen g{\"a}ller enbart f{\"o}r linj{\"a}ra problem, i vilka inga potenser av det elektriska f{\"a}ltet finns i de grundl{\"a}ggande ekvationerna.}
\item{$\bullet$}{Gauss lag p{\aa} integral- respektive differentialform:
$$
  \oiint_A {\bf E}\cdot d{\bf A}
     ={{1}\over{\varepsilon_0}}\iiint_V\rho({\bf x})\,dV
\qquad\Leftrightarrow\qquad
\nabla\cdot{\bf E}={{\rho({\bf x})}\over{\varepsilon_0}}
$$}

\bye
