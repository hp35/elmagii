%
% File: teach/elmagii/lect-09/lecture-09.tex [plain TeX code]
% Github: https://github.com/elmagii/lect-09/
% Last change: December 2, 2024
%
% Lecture No 9 in the course ``Elektromagnetism II, 1TE626 (2023)'',
% held December 3, 2024, at Uppsala University, Sweden.
%
% Copyright (C) 2024, Fredrik Jonsson, under Gnu General Public License
% (GPL) v3. See the enclosed LICENSE for details.
%
% This program is free software: you can redistribute it and/or modify
% it under the terms of the GNU General Public License as published by
% the Free Software Foundation, either version 3 of the License, or
% (at your option) any later version.
%
% This program is distributed in the hope that it will be useful,
% but WITHOUT ANY WARRANTY; without even the implied warranty of
% MERCHANTABILITY or FITNESS FOR A PARTICULAR PURPOSE.  See the
% GNU General Public License for more details.
%
% You should have received a copy of the GNU General Public License
% along with this program.  If not, see <https://www.gnu.org/licenses/>.
%
\input ../macros/epsf.tex
\input ../macros/eplain.tex
\font\ninerm=cmr9
\font\twelvesc=cmcsc10 at 12 truept
\input amssym % to get the {\Bbb E} font (strikethrough E)
\def\lecture #1 {\hsize=150mm\hoffset=4.6mm\vsize=230mm\voffset=7mm
  \topskip=0pt\baselineskip=12pt\parskip=0pt\leftskip=0pt\parindent=15pt
  \headline={\ifnum\pageno>1\ifodd\pageno\rightheadline\else\leftheadline\fi
    \else\hfill\fi}
  \def\rightheadline{\hfil{\it Elektromagnetism II, 1TE626 (2025)}}
  \def\leftheadline{{\it Elektromagnetism II, 1TE626 (2025)}\hfil}
  \noindent~\vskip-60pt\hskip-40pt{\epsfbox{../macros/UU_logo_color.eps}}
  \vskip-42pt\hfill\vbox{
      \hbox{{\it Elektromagnetism II, 1TE626 (2025)}}
      \hbox{{\it Fredrik Jonsson}}
      \hbox{{\it Document Revision \today}}
      \hbox{{\it https://github.com/hp35/elmagii/}}}\vskip 36pt
    \centerline{\twelvesc #1}
  \vskip 24pt\noindent}
\def\section #1 {\medskip\goodbreak\noindent{\bf #1}
  \par\nobreak\smallskip\noindent}
\def\subsection #1 {\medskip\goodbreak\noindent{\it #1}
  \par\nobreak\smallskip\noindent}
\def\iint{\mathop{\int\kern-8pt\int}}
\def\iiint{\mathop{\int\kern-8pt\int\kern-8pt\int}}
\def\oiint{\mathop{\int\kern-8pt\int\kern-13.2pt{\bigcirc}}}
\def\Re{\mathop{\rm Re}\nolimits} % real part
\def\Im{\mathop{\rm Im}\nolimits} % imaginary part
\def\Tr{\mathop{\rm Tr}\nolimits} % quantum mechanical trace
\def\eqq{\mathop{\vbox{\hbox{\hskip2pt?}\vskip-6pt\hbox{=}}}}
\def\boxit#1{\vbox{\hrule\hbox{\vrule\kern3pt
  \vbox{\kern3pt#1\kern3pt}\kern3pt\vrule}\hrule}}
\def\quote#1{\leftskip=36pt\rightskip=36pt\smallskip\noindent#1\par
  \leftskip=0pt\rightskip=0pt\smallskip}
\def\epsfig#1{\bigskip\centerline{\epsfbox{#1}}\medskip}

\lecture{Tentamensuppgift 1}
\vskip24pt
\noindent
Givet Maxwells ekvationer
$$
  \eqalign{
    \nabla\times{\bf E}
       &=-{{\partial{\bf B}}\over{\partial t}},\hskip4.45em
    \nabla\cdot{\bf D}=\rho,\cr
    \nabla\times{\bf H}&={\bf J}_{\rm f}
       +{{\partial{\bf D}}\over{\partial t}},\hskip3em
       \nabla\cdot{\bf B}=0,
  }
$$
samt de konstitutiva relationerna
$$
  \eqalign{
    {\bf D}&=\varepsilon_0{\bf E}+{\bf P}
            =\varepsilon_0\varepsilon_{\rm r}{\bf E},\cr
    {\bf B}&=\mu_0({\bf H}+{\bf M})
            =\mu_0\mu_{\rm r}{\bf H},\cr
  }
$$
visa att den elektromagnetiska v{\aa}gekvationen i ett generellt medium lyder
$$
  \eqalign{
    \nabla\times\nabla\times{\bf E}
      +\mu_0\varepsilon_0{{\partial^2{\bf E}}\over{\partial t^2}}&=
         -\mu_0{{\partial}\over{\partial t}}
          \underbrace{
             \bigg({\bf J}_{\rm f}
	        +{{\partial{\bf P}}\over{\partial t}}
	        +\nabla\times{\bf M}\bigg)}_{\hbox{gemensam k{\"a}llterm}},\cr
    \nabla\times\nabla\times{\bf B}
      +\mu_0\varepsilon_0{{\partial^2{\bf B}}\over{\partial t^2}}&=
          \mu_0\nabla\times
          \underbrace{
	     \bigg({\bf J}_{\rm f}
	        +{{\partial{\bf P}}\over{\partial t}}
	        +\nabla\times{\bf M}\bigg)}_{\hbox{gemensam k{\"a}llterm}}.\cr
  }
$$
\vfill\eject
\section{L{\"o}sningsf{\"o}rslag}
De ing{\aa}ende f{\"a}lten och deras respektive SI-enheter {\"a}r, f{\"o}r
att rekapitulera,
$$
  \eqalign{
    {\bf E} &= \hbox{Elektrisk f{\"a}ltstyrka (``elektriskt\ f{\"a}lt'')}
               \ ({\rm V}/{\rm m})\cr
    {\bf D} &= \hbox{Elektrisk fl{\"o}dest{\"a}thet}
               \ ({\rm C}/{\rm m}^2)\cr
    {\bf P} &= \hbox{Elektrisk polarisationsdensitet}
               \ ({\rm C}/{\rm m}^2)\cr
    {\bf B} &= \hbox{Magnetisk fl{\"o}dest{\"a}thet (``B-f{\"a}lt'')}
               \ ({\rm T})\cr
    {\bf H} &= \hbox{Magnetisk f{\"a}ltstyrka (``H-f{\"a}lt'')}
               \ ({\rm A}/{\rm m})\cr
    {\bf M} &= \hbox{Magnetisering}
               \ ({\rm A}/{\rm m})\cr
  }
$$
Vi b{\"o}rjar med den elektriska f{\"a}ltstyrkan ${\bf E}$ genom att applicera
$\nabla\times$ (``ta rotationen'') p{\aa} Faradays generella induktionslag,
$$
  \eqalign{
    \nabla\times\nabla\times{\bf E}
      &=\nabla\times\bigg(-{{\partial{\bf B}}\over{\partial t}}\bigg)\cr
      &=\{\ \hbox{Konstitutiv relation}
             \ {\bf B}=\mu_0({\bf H}+{\bf M})\ \}\cr
      &=-{{\partial}\over{\partial t}}
          \nabla\times\bigg(\mu_0({\bf H}+{\bf M})\bigg)\cr
      &=-\mu_0{{\partial}\over{\partial t}}\nabla\times{\bf H}
          -\mu_0{{\partial}\over{\partial t}}\nabla\times{\bf M}\cr
      &=\{\ \hbox{Till{\"a}mpa Amp\`eres lag}\ \}\cr
      &=-\mu_0{{\partial}\over{\partial t}}
          \bigg({\bf J}_{\rm f}+{{\partial{\bf D}}\over{\partial t}}\bigg)
          -\mu_0{{\partial}\over{\partial t}}\nabla\times{\bf M}\cr
      &=\{\ \hbox{Konstitutiv relation}
             \ {\bf D}=\varepsilon_0{\bf E}+{\bf P}\ \}\cr
      &=-\mu_0{{\partial^2}\over{\partial t^2}}
          \big(\varepsilon_0{\bf E}+{\bf P}\big)
          -\mu_0{{\partial}\over{\partial t}}
	     \big({\bf J}_{\rm f}+\nabla\times{\bf M}\big)\cr
      &=\{\ \hbox{Kombinera polarisationsdensiteten in i k{\"a}llterm}\ \}\cr
      &=-\underbrace{\mu_0\varepsilon_0}_{=1/c^2_0}
         {{\partial^2{\bf E}}\over{\partial t^2}}
          -\underbrace{
	     \mu_0{{\partial}\over{\partial t}}
	     \bigg({\bf J}_{\rm f}
	        +{{\partial{\bf P}}\over{\partial t}}
	        +\nabla\times{\bf M}\bigg)}_{\hbox{k{\"a}llterm}}.\cr
  }
$$
Notera att denna ekvation f{\"o}r ${\bf E}$ g{\"a}ller {\it oavsett} eventuella
spatiala variationer hos relativa permittiviteten eller permeabiliteten, det
vill s{\"a}ga oavsett om relationen mellan de exciterande f{\"a}lten och den
resulterande elektriska polarisationsdensiteten eller magnetiseringen
{\"a}ndras.

P{\aa} samma s{\"a}tt har vi f{\"o}r magnetiska fl{\"o}dest{\"a}theten
${\bf B}=\mu_0({\bf H}+{\bf M})$ att
$$
  \eqalign{
    \nabla\times\nabla\times{\bf B}
      &=\mu_0\nabla\times\nabla\times{\bf H}
           +\mu_0\nabla\times\nabla\times{\bf M}\cr
      &=\{\ \hbox{Till{\"a}mpa Amp\`eres lag}\ \}\cr
      &=\mu_0\nabla\times\bigg({\bf J}_{\rm f}
           +{{\partial{\bf D}}\over{\partial t}}\bigg)
           +\mu_0\nabla\times\nabla\times{\bf M}\cr
      &=\mu_0{{\partial}\over{\partial t}}\nabla\times{\bf D}
           +\mu_0\nabla\times\bigg({\bf J}_{\rm f}+\nabla\times{\bf M}\bigg)\cr
      &=\{\ \hbox{Konstitutiv relation}
             \ {\bf D}=\varepsilon_0{\bf E}+{\bf P}\ \}\cr
      &=\mu_0\varepsilon_0{{\partial}\over{\partial t}}\nabla\times{\bf E}
           +\mu_0\nabla\times\bigg({\bf J}_{\rm f}
	   +{{\partial{\bf P}}\over{\partial t}}
	   +\nabla\times{\bf M}\bigg)\cr
      &=\{\ \hbox{Till{\"a}mpa Faradays lag}\ \}\cr
      &=-\underbrace{\mu_0\varepsilon_0}_{=1/c^2_0}
         {{\partial^2{\bf B}}\over{\partial t^2}}
           +\underbrace{\mu_0\nabla\times\bigg({\bf J}_{\rm f}
	   +{{\partial{\bf P}}\over{\partial t}}
	   +\nabla\times{\bf M}\bigg)}_{\hbox{k{\"a}llterm}}.\cr
  }
$$
\bye
