%
% File: teach/elmagii/lect-12/lecture-12.tex [plain TeX code]
% Github: https://github.com/elmagii/lect-12/
% Last change: January 1, 2026
%
% Lecture No 12 in the course ``Elektromagnetism II, 1TE626 (2023)'',
% held December 12, 2025, at Uppsala University, Sweden.
%
% Copyright (C) 2022-2025, Fredrik Jonsson, under Gnu General Public License
% (GPL) v3. See the enclosed LICENSE for details.
%
% This program is free software: you can redistribute it and/or modify
% it under the terms of the GNU General Public License as published by
% the Free Software Foundation, either version 3 of the License, or
% (at your option) any later version.
%
% This program is distributed in the hope that it will be useful,
% but WITHOUT ANY WARRANTY; without even the implied warranty of
% MERCHANTABILITY or FITNESS FOR A PARTICULAR PURPOSE.  See the
% GNU General Public License for more details.
%
% You should have received a copy of the GNU General Public License
% along with this program.  If not, see <https://www.gnu.org/licenses/>.
%
\def\coursename{Elektromagnetism II}
\def\coursecode{1TE626}
\def\courseyear{2025}
\def\courserepo{https://github.com/hp35/elmagii/}
\def\lecturenumber{12}
\def\lecturetitle{Grundl{\"a}ggande antennteori}
\def\lecturetitlesub{}
\def\lectureauthor{Fredrik Jonsson}
\def\lectureplace{Uppsala Universitet}
\def\lecturedate{12 december 2025}
%-------------------- BEGIN OF LOCAL MACROS --------------------
\input macros/epsf.tex
\input macros/eplain.tex
\input amssym % to get the {\Bbb E} font (strikethrough E)
\font\ninerm=cmr9
\font\tenssbx=cmssbx10
\font\twelvesc=cmcsc10 at 12 truept
\newif\ifcolors % creates \ifcolors, \colorstrue, \colorsfalse
\colorstrue     % \colorstrue turns colors on, \colorsfalse turns them off
\ifcolors\input color\fi  % to get colored text output
\def\red#1{\ifcolors{\color{red}#1}\else#1\fi\ignorespaces}
\def\ifempty#1{\ifx\relax#1\relax}
\def\initlecture{
  \hsize=150mm\hoffset=4.6mm\vsize=230mm\voffset=7mm
  \topskip=0pt\baselineskip=12pt\parskip=0pt\leftskip=0pt\parindent=15pt
  \ifcolors
    \voffset=-10.2mm\topskip=0pt
  \fi
  \headline={\ifnum\pageno>1\ifodd\pageno\rightheadline\else\leftheadline\fi
    \else\hfill\fi}
  \def\rightheadline{\tenrm{\it F\"orel\"asning \lecturenumber}
    \hfil{\it \coursename, \coursecode\ (\courseyear)}}
  \def\leftheadline{\tenrm{\it \coursename, \coursecode\ (\courseyear)}
    \hfil{\it F\"orel\"asning \lecturenumber}}
  \noindent~\vskip-60pt\hskip-40pt{\epsfbox{macros/UU_logo_color.eps}}
  \vskip-42pt\hfill\vbox{
      \hbox{{\it \coursename, \coursecode\ (\courseyear)}}
      \hbox{{\it Lecture Notes, \lectureauthor}}
      \hbox{{\it Document Revision \today}}
      \hbox{{\it \courserepo}}}\vskip 36pt
  \centerline{\twelvesc F\"orel\"asning \lecturenumber}
  \vskip 24pt\noindent
  \centerline{\twelvesc\lecturetitle}
  \expandafter\ifempty\expandafter{\lecturetitlesub}%
    \else\centerline{\twelvesc\lecturetitlesub}\fi
  \bigskip
  \centerline{\lectureauthor, \lectureplace, \lecturedate}
  \vskip24pt}
\def\section #1 {\medskip\goodbreak\noindent{\tenssbx #1}
  \par\nobreak\smallskip\noindent}
\def\subsection #1 {\medskip\goodbreak\noindent{\it #1}
  \par\nobreak\smallskip\noindent}
\def\iint{\mathop{\int\kern-8pt\int}}
\def\iiint{\mathop{\int\kern-8pt\int\kern-8pt\int}}
\def\oiint{\mathop{\int\kern-8pt\int\kern-13.2pt{\bigcirc}}}
\def\sgn{\mathop{\rm sgn}\nolimits} % sign
\def\Re{\mathop{\rm Re}\nolimits}   % real part
\def\Im{\mathop{\rm Im}\nolimits}   % imaginary part
\def\Tr{\mathop{\rm Tr}\nolimits}   % quantum mechanical trace
\def\eqq{\mathop{\vbox{\hbox{\hskip2pt?}\vskip-6pt\hbox{=}}}}
\def\quote#1{\par\leftskip=36pt\rightskip=36pt\smallskip\noindent#1\par
  \leftskip=0pt\rightskip=0pt\smallskip}
\long\def\plan#1{\par\leftskip=36pt\rightskip=36pt\bigskip
  \noindent{\it Sammanfattning}\smallskip
  \noindent{\it #1}\par\leftskip=0pt\rightskip=0pt}
\def\threepointsummary#1#2#3{\par\leftskip=36pt\rightskip=36pt\bigskip
  \noindent{\it Tre h{\aa}llpunkter i f{\"o}rel{\"a}sningen}\smallskip
  \leftskip=48pt\rightskip=36pt\hangindent=20pt
  \noindent{\it\hbox to 20pt{1. }#1}\smallskip
  \leftskip=48pt\rightskip=36pt\hangindent=20pt
  \noindent{\it\hbox to 20pt{2. }#2}\smallskip
  \leftskip=48pt\rightskip=36pt\hangindent=20pt
  \noindent{\it\hbox to 20pt{3. }#3}\par
  \leftskip=0pt\rightskip=0pt\vfill\eject}
\def\epsfig#1{\bigskip\centerline{\epsfbox{#1}}\medskip}
\def\captionwide{\advance\leftskip by 60pt
  \advance\rightskip by 60pt}
\newdimen\itemindent \itemindent=18pt
\newdimen\hangitemindent \hangitemindent=38pt
\def\litem[#1]{\smallbreak\noindent%
  \hbox to\itemindent{\hfil}\hbox to\itemindent{#1\hfill}%
  \hangindent\hangitemindent\ignorespaces}
\newif\ifshowindex
\showindextrue  % Use \showindextrue and \showindexfalse to enable/disable index
\def\index{\ifshowindex\vfill\eject\section{Index} \readindexfile{i}\fi}
%--------------------- END OF LOCAL MACROS ---------------------

\initlecture

\plan{I denna f{\"o}rel{\"a}sning knyter vi ihop den potentialteori som vi under kursens g{\aa}ng utvecklat, och applicerar den p{\aa} ett konkret exempel i form av en antenn.}

\threepointsummary{%
}{%
}{%
}
%----------------------- END OF PREAMBLE -----------------------

\section{Retarderade potentialer f{\"o}r antenner}
\sidx{Retarderade potentialer}[f{\"o}r cylindrisk rak antenn]
Vi betraktar en dipolantenn best{\aa}ende av tv{\aa} identiska cylindriska
element av l{\"a}ngd $L$, separerade med ett litet luftgap $g$. Med ``litet''
luftgap menar vi h{\"a}r ett gap som {\"a}r litet i f{\"o}rh{\aa}llande till
antennelementen ($g\ll L$) s{\aa}v{\"a}l som i f{\"o}rh{\aa}llande till
v{\aa}gl{\"a}ngden ($g\ll\lambda$).
Utan att g{\"o}ra ett alltf{\"o}r stort avsteg fr{\aa}n det generella fallet,
kan vi anta att str{\"o}mt{\"a}theten i antennelementen i huvudsak kommer att
l{\"o}pa i elementens mantelyta, och vi kommer att anta att eventuella effekter
fr{\aa}n elementens plana {\"a}ndytor kan f{\"o}rsummas.
I detta avseende kan vi d{\"a}rf{\"o}r modellera antennen som best{\aa}ende av
tv{\aa} cylindriska r{\"o}r, var och ett b{\"a}rande en str{\"o}mt{\"a}thet som
{\"a}r homogen i varje tv{\"a}rsnitt $z$ av antennen.
\epsfig{figs/dipole-antenna.1}\noindent
\sidx{Dipolantenn}

\vfill\eject

\section{Sammanfattning av F{\"o}rel{\"a}sning~12 -- Grundl{\"a}ggande antennteori}
\item{$\bullet$}{}
\item{$\bullet$}{}
\item{$\bullet$}{}
\item{$\bullet$}{}
\index
\bye
