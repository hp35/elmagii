\eqsection{Variabler, konstanter och enheter}
\eqtitle{Beteckningar och SI-enheter}
\halign{#\hfil\ &#\hfil&\ \hfil(#)\cr
${\bf A}$
    & Vektorpotential
    & ${\rm V}\cdot{\rm s}/{\rm m}$\cr
${\bf B}$
    & Magnetisk fl{\"o}dest{\"a}thet ${\bf B}=\mu_0\mu_{\rm r}{\bf H}$
    & ${\rm T}$\cr
$c_0$
    & Ljushastighet i vakuum, $c^{-2}_0=\varepsilon_0\mu_0$
    & ${\rm m}/{\rm s}$\cr
$c$
    & Ljushastighet i medium, $c=c_0/n$
    & ${\rm m}/{\rm s}$\cr
${\bf D}$
    & Elektrisk fl{\"o}dest{\"a}thet
      ${\bf D}=\varepsilon_0\varepsilon_{\rm r}{\bf E}$
    & ${\rm C}/{\rm m}^2$\cr
$\varepsilon_0$
    & Elektrisk permittivitet i vakuum
    & ${\rm F}/{\rm m}$\cr
$\varepsilon_{\rm r}$
    & Relativ elektrisk permittivitet
    & $1$\cr
$e$
    & Elementarladdning
    & ${\rm C}$\cr
${\bf e}_r$
    & Riktning ${\bf e}_r=({\bf x}-{\bf x}')/|{\bf x}-{\bf x}'|$
    & $1$\cr
${\bf E}$
    & Elektrisk f{\"a}ltstyrka
    & ${\rm V}/{\rm m}$\cr
${\cal E}$
    & Elektromotorisk ``kraft'' (EMK)
    & ${\rm V}$\cr
$\phi$
    & Skal{\"a}r potential
    & ${\rm V}$\cr
$\Phi_{\rm E}$
    & Elektriskt fl{\"o}de $\Phi_{\rm E}=\int\kern-5pt\int{\bf E}\cdot d{\bf S}$
    & ${\rm V}\cdot{\rm m}$\cr
$\Phi_{\rm M}$
    & Magnetiskt fl{\"o}de $\Phi_{\rm M}=\int\kern-5pt\int{\bf B}\cdot d{\bf S}$
    & ${\rm T}\cdot{\rm m}^2$\cr
$f$
    & Frekvens
    & ${\rm Hz}$\cr
${\bf F}$
    & Kraft
    & ${\rm N}$\cr
${\bf F}_{\rm mag}$
    & Magnetisk kraft
    & ${\rm N}$\cr
${\bf H}$
    & Magnetiseringsstyrka
    & ${\rm A}/{\rm m}$\cr
${\bf I}$
    & Elektrisk str{\"o}m
    & ${\rm A}$\cr
$I_{\rm enc}$
    & Omsluten elektrisk str{\"o}m
    & ${\rm A}$\cr
${\bf J}$
    & Str{\"o}mt{\"a}thet ${\bf J}={\bf J}_{\rm b}+{\bf J}_{\rm f}$
    & ${\rm A}/{\rm m}^2$\cr
${\bf J}_{\rm b}$
    & Bunden elektrisk str{\"o}mt{\"a}thet
    & ${\rm A}/{\rm m}^2$\cr
${\bf J}_{\rm eff}$
    & Effektiv elektrisk str{\"o}mt{\"a}thet
    & ${\rm A}/{\rm m}^2$\cr
${\bf J}_{\rm f}$
    & Fri elektrisk str{\"o}mt{\"a}thet
    & ${\rm A}/{\rm m}^2$\cr
${\bf k}$
    & V{\aa}gvektor
    & $1/{\rm m}$\cr
$k$
    & V{\aa}gtal, $k=|{\bf k}|=2\pi/\lambda$
    & $1/{\rm m}$\cr
${\bf K}_{\rm b}$
    & Bunden ytstr{\"o}mt{\"a}thet
    & ${\rm A}/{\rm m}$\cr
$\lambda$
    & Linjeladdningst{\"a}thet
    & ${\rm C}/{\rm m}$\cr
$\lambda$
    & V{\aa}gl{\"a}ngd, $\lambda=c/f$
    & ${\rm m}$\cr
$\mu_0$
    & Magnetisk permeabilitet i vakuum
    & ${\rm H}/{\rm m}$\cr
$\mu_{\rm r}$
    & Relativ magnetisk permeabilitet
    & $1$\cr
$m_{\rm e}$
    & Elektronens vilomassa
    & ${\rm kg}$\cr
$m_{\rm p}$
    & Protonens vilomassa
    & ${\rm kg}$\cr
${\bf m}$
    & Magnetiskt dipolmoment
    & ${\rm A}\cdot{\rm m}^2$\cr
${\bf M}$
    & Magnetisering
    & ${\rm A}/{\rm m}$\cr
$n$
    & Brytningsindex $n=\varepsilon^{1/2}_{\rm r}$
    & $1$\cr
$\psi$
    & Gauge-funktion $\psi=\psi({\bf x},t)$
    & ${\rm V}\cdot{\rm s}$\cr
${\bf p}$
    & Elektriskt dipolmoment
    & ${\rm C}\cdot{\rm m}$\cr
${\bf P}$
    & Elektrisk polarisationsdensitet
    & ${\rm C}/{\rm m}^2$\cr
$P$
    & Effekt
    & ${\rm W}$\cr
$q$
    & Elektrisk laddning
    & ${\rm C}$\cr
$r$
    & Radie $r=|{\bf x}-{\bf x}'|$
    & ${\rm m}$\cr
$\rho$
    & Laddningst{\"a}thet $\rho=\rho_{\rm b}+\rho_{\rm f}$
    & ${\rm C}/{\rm m}^3$\cr
$\rho_{\rm b}$
    & Bunden elektrisk laddningst{\"a}thet
    & ${\rm C}/{\rm m}^3$\cr
$\rho_{\rm f}$
    & Fri elektrisk laddningst{\"a}thet
    & ${\rm C}/{\rm m}^3$\cr
$\sigma$
    & Ytladdningst{\"a}thet
    & ${\rm C}/{\rm m}^2$\cr
${\bf S}$
    & Poyntingvektor
    & ${\rm W}/{\rm m}^2$\cr
$\tau$
    & Vridmoment
    & ${\rm N}\cdot{\rm m}^2$\cr
$t$
    & Tid
    & ${\rm s}$\cr
${\bf v}$
    & Hastighet
    & ${\rm m}/{\rm s}$\cr
$v$
    & Fart, $v=|{\bf v}|$
    & ${\rm m}/{\rm s}$\cr
$V$
    & Volym
    & ${\rm m}^3$\cr
$\omega$
    & Vinkelfrekvens, $\omega=2\pi f$
    & $1/{\rm s}$\cr
$W_{\rm e}$
    & Elektrostatisk energi
    & ${\rm J}$\cr
$W_{\rm m}$
    & Magnetostatisk energi
    & ${\rm J}$\cr
${\bf x}$
    & Ortsvektor till observationspunkt
    & ${\rm m}$\cr
${\bf x}'$
    & Ortsvektor till k{\"a}llpunkt
    & ${\rm m}$\cr
$\chi_{\rm e}$
    & Elektrisk susceptibilitet, $\varepsilon_{\rm r}=1+\chi_{\rm e}$
    & $1$\cr
$\chi_{\rm m}$
    & Magnetisk susceptibilitet
    & $1$\cr
}

\eqsection{Naturkonstanter}
\halign{#\hfil\cr
  $\varepsilon_0=8.8541878188(14)\times10^{12}\ {\rm F}/{\rm m}$\cr
  $\mu_0=1.25663706127(20)\times10^{-6}\ {\rm N}/{\rm A}^2$\cr
  $c_0=299\,792\,458\ {\rm m}/{\rm s}$ (exakt per definition)\cr
  $e=1.602176634\times10^{-19}\ {\rm C}$ (exakt per definition)\cr
  $m_{\rm e}=9.1093837139(28)\times10^{-31}\ {\rm kg}$\cr
  $m_{\rm p}=1.67262192595(52)\times10^{-27}\ {\rm kg}$\cr
}

\eqsection{Konstitutiva relationer}
% \eqtitle{Elektrisk polarisationsdensitet ${\bf P}$}
\eqtitle{Elektrisk polarisationsdensitet}
$$
  {\bf P} \equiv \Big\langle{{d{\bf p}}\over{dV}}\Big\rangle
    = \varepsilon_0\chi_{\rm e}{\bf E},\qquad
  \nabla\cdot{\bf P}=-\rho_{\rm b}
$$
% \eqtitle{Elektrisk fl{\"o}dest{\"a}thet ${\bf D}$}
\eqtitle{Elektrisk fl{\"o}dest{\"a}thet}
$$
  {\bf D}\equiv\varepsilon_0{\bf E}+{\bf P}
    =\varepsilon_0\varepsilon_{\rm r}{\bf E}
$$
% \eqtitle{Magnetisering ${\bf M}$}
\eqtitle{Magnetisering}
$$
  {\bf M} \equiv \Big\langle{{d{\bf m}}\over{dV}}\Big\rangle
    = {{1}\over{\mu_0}}\Big(1-{{1}\over{\mu_{\rm r}}}\Big){\bf B}
$$
% \eqtitle{Magnetisk fl{\"o}dest{\"a}thet ${\bf B}$}
\eqtitle{Magnetisk fl{\"o}dest{\"a}thet}
$$
  {\bf B}\equiv\mu_0({\bf H}+{\bf M})
    =\mu_0\mu_{\rm r}{\bf H}
$$
\vfill\eject

\eqsection{Elektrostatik}
\eqtitle{Coulombs kraftlag}
$$
  {\bf F}({\bf x})={{qq'}\over{4\pi\varepsilon_0}}
     {{({\bf x}-{\bf x}')}\over{|{\bf x}-{\bf x}'|^3}}
     =q{\bf E}({\bf x})
$$
\eqtitle{Elektrisk f{\"a}ltstyrka fr{\aa}n punktladdning $q'$}
$$
  {\bf E}({\bf x})={{q'}\over{4\pi\varepsilon_0}}
    {{({\bf x}-{\bf x}')}\over{|{\bf x}-{\bf x}'|^3}}
      ={{q'}\over{4\pi\varepsilon_0 r^2}}{\bf e}_r
$$
\eqtitle{Coulombs generaliserade lag -- ``Coulombintegralen''}
$$
  {\bf E}({\bf x})={{1}\over{4\pi\varepsilon_0}}\iiint_V\rho({\bf x}')
    {{({\bf x}-{\bf x}')}\over{|{\bf x}-{\bf x}'|^3}}\,dV'
$$
\eqtitle{Existens av skal{\"a}r elektrostatisk potential}
Ur Coulombs generaliserade lag f{\"o}r elektrostatiska f{\"a}lt erh{\aa}lls
$\nabla\times{\bf E}={\bf 0}$, vilket visar att den elektriska f{\"a}ltstyrkan
alltid kan skrivas som en gradient av en skal{\"a}r funktion, den skal{\"a}ra
potentialen $\phi$, definierad som
$$
  {\bf E}=-\nabla\phi.
$$
\eqtitle{Elektriskt fl{\"o}de}
$$
  \Phi_{\rm E}=\oiint_S {\bf E}({\bf x})\cdot d{\bf S}
     ={{1}\over{\varepsilon_0}}\iiint_V\rho({\bf x})\,dV
     ={{q_{\rm tot}}\over{\varepsilon_0}}
$$
\eqtitle{Gauss lag f{\"o}r elektrisk fl{\"o}dest{\"a}thet}
$$
  \oiint{\bf D}\cdot d{\bf S}=\iiint\rho\,dV
$$
\eqtitle{Poissons ekvation (Coulombs lag) for skal{\"a}r potential}
$$
  \nabla^2\phi({\bf x})=-\rho({\bf x})/\varepsilon_0
$$
\eqtitle{Skal{\"a}r potential, explicit l{\"o}sning till Poissons ekvation}
$$
  \phi({\bf x})\equiv{{1}\over{4\pi\varepsilon_0}}\iiint_V
      {{\rho({\bf x}')}\over{|{\bf x}-{\bf x}'|}}\,dV'
  %  \ \Rightarrow\ {\bf E}({\bf x})=-\nabla\phi({\bf x})
$$
\eqtitle{Skal{\"a}r elektrostatisk potential fr{\aa}n punktladdning $q'$}
$$
  \phi({\bf x})
    ={{1}\over{4\pi\varepsilon_0}}{{q'}\over{|{\bf x}-{\bf x}'|}},
      \quad{\bf x}\ne{\bf x}'.
$$
\eqtitle{Arbete som tillf{\"o}rs punktladdning $q$ vid f{\"o}rflyttning fr{\aa}n
         $a$ till $b$}
$$
  W_{\rm e}=-\int^{{\bf x}_b}_{{\bf x}_a}{\bf F}({\bf x})\cdot d{\bf l}
          =q\big(\phi({{\bf x}_b})-\phi({{\bf x}_a})\big) % =W_b-W_a
$$

\eqsection{Magnetostatik}
\eqtitle{Lorentz-kraften p{\aa} fri laddning $q$}
$$
  {\bf F}=q\big({\bf E}+{\bf v}\times{\bf B}\big)
$$
\eqtitle{Amp\`eres kraftlag p{\aa} str{\"o}mslinga b{\"a}rande
  str{\"o}mmen $I$}
$$
  {\bf F}_{\rm mag}=\int^{{\bf x}_b}_{{\bf x}_a} ({\bf I}\times{\bf B})\,dl
$$
\eqtitle{Str{\"o}mmen $I$ genom en yta $S$ med str{\"o}mt{\"a}thet}
$$
  I=\iint_S {\bf J}\cdot d{\bf S}
$$
\eqtitle{Str{\"o}mt{\"a}thet i medium med konduktivitet $\sigma$}
$$
  {\bf J}=\sigma{\bf E}
$$
\eqtitle{Lagen om att elektrisk laddning inte kan f{\"o}rsvinna --
         kontinuitetsekvationen}
$$
  \nabla\cdot{\bf J}=-{{d\rho}\over{dt}}.
$$
\eqtitle{Statiska problem}
$$
  {{d\rho}\over{dt}}=0\quad\underline{\hbox{och}}\quad{{d{\bf J}}\over{dt}}=0.
$$
\eqtitle{Divergens av str{\"o}mt{\"a}theten i statiska problem}
$$
  {{d\rho}\over{dt}}=0\quad\Leftrightarrow\quad\nabla\cdot{\bf J}=0
$$
\eqtitle{Biot--Savarts lag f{\"o}r str{\"o}mslingor}
$$
  {\bf B}({\bf x})={{\mu_0}\over{4\pi}}\int^{{\bf x}_b}_{{\bf x}_a}
      {{{\bf I}({\bf x}')\times({\bf x}-{\bf x}')}
        \over{|{\bf x}-{\bf x}'|^3}}\,dl'
$$
\eqtitle{Biot--Savarts generella lag}
$$
  {\bf B}({\bf x})={{\mu_0}\over{4\pi}}\iiint_V
      {{{\bf J}({\bf x}')\times({\bf x}-{\bf x}')}
        \over{|{\bf x}-{\bf x}'|^3}}\,dV'
$$
\eqtitle{Existens av magnetostatisk vektorpotential}
Ur Biot--Savarts generalla lag f{\"o}r magnetostatiska f{\"a}lt erh{\aa}lls
$\nabla\cdot{\bf B}={\bf 0}$, vilket visar att den magnetiska
fl{\"o}dest{\"a}theten alltid kan skrivas som en rotation av en vektorfunktion,
vektorpotentialen ${\bf A}$, definierad som
$$
  {\bf B}=\nabla\times{\bf A}.
$$
\eqtitle{Gauss lag f{\"o}r magnetisk fl{\"o}dest{\"a}thet --
         icke-existens f{\"o}r magnetiska mono\-poler} % existerar inte}
$$
  \nabla\cdot{\bf B}=0\qquad\hbox{(alltid)}
$$
\eqtitle{Amp\`eres statiska lag}
$$
  \oint_{\Gamma}{\bf B}\cdot d{\bf l}
    =\mu_0\iint_S{\bf J}\cdot d{\bf S}
    =\mu_0I_{\rm enc}
  \quad\Leftrightarrow\quad
  \nabla\times{\bf B}=\mu_0{\bf J}
$$
\eqtitle{Bunden elektrisk str{\"o}mt{\"a}thet}
$$
  {\bf J}_{\rm b}=\nabla\times{\bf M}
$$
\eqtitle{Bunden ytstr{\"o}mt{\"a}thet}
$$
  {\bf K}_{\rm b}={\bf M}\times{\bf e}_n
$$
\eqtitle{Poissons ekvation (Amp\`eres lag) f{\"o}r vektorpotentialen}
$$
  \nabla^2{\bf A}=-\mu_0{\bf J},
$$
\eqtitle{Vektorpotential, explicit l{\"o}sning till Poissons ekvation}
$$
  {\bf A}({\bf x})={{\mu_0}\over{4\pi}}\iiint_V
    {{{\bf J}({\bf x}')}\over{|{\bf x}-{\bf x}'|}}\,dV'
$$

\eqsection{Elektrodynamik}
\eqtitle{Magnetiskt fl{\"o}de}
$$
  \Phi_{\rm M}=\iint_S{\bf B}\cdot d{\bf S}
$$
\eqtitle{Elektromotorisk ``kraft'' runt en sluten slinga $\Gamma$}
$$
  {\cal E}=\oint_{\Gamma}\bigg({{{\bf F}}\over{q}}\bigg)\cdot d{\bf l}
    =\oint_{\Gamma}[{\bf E}+({\bf v}\times{\bf B})\big]\cdot d{\bf l}
$$
\eqtitle{Faradays induktionslag}
$$
  {\cal E}=-{{d\Phi_{\rm M}}\over{dt}}
$$
\eqtitle{Lenz lag}
En inducerad str{\"o}m har en riktning som motverkar orsaken till att den
uppkom. Detta inneb{\"a}r att om magnetf{\"a}ltet genom en ledande slinga
{\"o}kar, s{\aa} kommer den i slingan inducerade str{\"o}mmen att ha en
riktning som skapar ett magnetf{\"a}lt som motverkar {\"o}kningen, och vice
versa.
\eqtitle{Faradays induktionslag f{\"o}r spole med $N$ varv}
$$
  {\cal E}=-N{{d\Phi_{\rm M}}\over{dt}}
$$
\eqtitle{Faradays lag p{\aa} differentialform}
$$
  \nabla\times{\bf E}({\bf r},t)
    =-{{\partial{\bf B}({\bf x},t)}\over{\partial t}}
$$
\eqtitle{Faradays lag p{\aa} integralform}
$$
  \oint_{\Gamma}{\bf E}({\bf r},t)\cdot d{\bf l}
    =-{{d}\over{dt}}\iint_S{\bf B}({\bf x},t)\cdot d{\bf S}
$$
\eqtitle{{\"O}msesidig induktans beskrivs av det magnetiska fl{\"o}de
som genereras i en sekund{\"a}rslinga $\Gamma$ fr{\aa}n en
str{\"o}m $I'(t)$ som drivs genom en prim{\"a}rslinga $\Gamma'$,}
$$
  \Phi_{\rm M}=M_{\Gamma\Gamma'} I'(t)
$$
\eqtitle{Neumanns formel f{\"o}r {\"o}msesidig induktans}
$$
  M_{\Gamma\Gamma'}={{\mu_0}\over{4\pi}}\oint_{\Gamma}\oint_{\Gamma'}
        {{d{\bf l}'\cdot d{\bf l}}\over{|{\bf x}-{\bf x}'|}},
  \qquad M_{\Gamma\Gamma'}=M_{\Gamma'\Gamma}
$$
\eqtitle{Maxwells ekvationer p{\aa} differentialform}
$$
  \eqalign{
    \nabla\times{\bf E}
       &=-{{\partial{\bf B}}\over{\partial t}},\hskip53pt
   \nabla\cdot{\bf D}=\rho_{\rm f},\cr
   \nabla\times{\bf H}&={\bf J}_{\rm f}
       +{{\partial{\bf D}}\over{\partial t}},\hskip40pt
    \nabla\cdot{\bf B}=0.\cr
  }
$$
\eqtitle{Maxwells ekvationer p{\aa} integralform}
$$
  \eqalign{
    \oint_{\Gamma}{\bf E}\cdot d{\bf l}
        &=-{{\partial}\over{\partial t}}\iint_S{\bf B}\cdot d{\bf S},\cr
    \oint{\bf H}\cdot d{\bf l} & =\iint{\bf J}_{\rm f}\cdot d{\bf S}
       +{{\partial}\over{\partial t}}\iint{\bf D}\cdot d{\bf S},\cr
    \oiint{\bf D}\cdot d{\bf S} & =\iiint\rho\,dV,\cr
    \oiint{\bf B}\cdot d{\bf S} & =0.\cr
  }
$$
\eqtitle{De elektromagnetiska v{\aa}gekvationerna}
$$
  \eqalign{
    \nabla\times\nabla\times{\bf E}
      +{{1}\over{c^2}}{{\partial^2{\bf E}}\over{\partial t^2}}&=
         -\mu_0{{\partial {\bf J}_{\rm eff}}\over{\partial t}},\cr
    \nabla\times\nabla\times{\bf B}
      +{{1}\over{c^2}}{{\partial^2{\bf B}}\over{\partial t^2}}&=
          \mu_0\nabla\times{\bf J}_{\rm eff},\cr
  }
$$
d{\"a}r den gemensamma k{\"a}lltermen
$$
  {\bf J}_{\rm eff}
    ={\bf J}_{\rm f}
      +{{\partial{\bf P}}\over{\partial t}}
      +\nabla\times{\bf M}
$$
{\"a}r den effektiva str{\"o}mt{\"a}theten, inkluderande den fria
str{\"o}mt{\"a}theten ${\bf J}_{\rm f}$, den elektriska
polarisationsdensi\-teten ${\bf P}$ samt magnetiseringen ${\bf M}$.
% Fr{\aa}n de elektromagnetiska v{\aa}gekvationerna kan samt\-liga
% elektrostatiska, magnetostatiska eller elektrodynamiska fall h{\"a}rledas,
% beroende p{\aa} vilka termer som kan s{\"a}ttas till noll.
\vfill\eject

\eqsection{Elektrodynamik och potentialer}
\eqtitle{Elektrodynamiska potentialer}
$$
  \eqalign{
    {\bf E}({\bf x},t)
      &=-\nabla\phi({\bf x},t)
          -{{\partial{\bf A}({\bf x},t)}\over{\partial t}},\cr
    {\bf B}({\bf x},t)
      &=\nabla\times{\bf A}({\bf x},t).\cr
  }
$$
\eqtitle{Gauge-transformen}
Med den godtyckliga och tv{\aa} g{\aa}nger i rum och tid kontinuerligt
differentierbara gauge-funktionen $\psi({\bf x},t)$, {\"a}r
gauge-transformationen definierad som
$$
  \eqalign{
    {\bf A}'({\bf x},t)
      &={\bf A}({\bf x},t)+\nabla\psi({\bf x},t),\cr
    \phi'({\bf x},t)
      &=\phi({\bf x},t)-{{\partial\psi({\bf x},t)}\over{\partial t}},\cr
  }
$$
vilken l{\"a}mnar de elektromagnetiska f{\"a}lten invarianta.
\eqtitle{Lorenz-villkoret}
Under Lorenz-villkoret v{\"a}ljs gauge-funktionen $\psi$ s{\aa} att
$$
  \nabla\cdot{\bf A}'({\bf x},t)+
    {{1}\over{c^2}}{{\partial\phi'({\bf x},t)}\over{\partial t}} = 0
$$
\eqtitle{Lorenz-villkoret -- V{\aa}gekvationer f{\"o}r potentialer}
Under Lorenz-villkoret frikopplas v{\aa}gekvationerna f{\"o}r den skal{\"a}ra
potentialen och vektorpotentialen, till
$$
  \eqalign{
    \Big(
      \nabla^2-{{1}\over{c^2}}{{\partial^2}\over{\partial t^2}}
    \Big)\phi({\bf x},t)
        &=-{{\rho({\bf x},t)}\over{\varepsilon_0\varepsilon_{\rm r}}},\cr
    \Big(
      \nabla^2-{{1}\over{c^2}}{{\partial^2}\over{\partial t^2}}
    \Big){\bf A}({\bf x},t)
        &=-\mu_0{\bf J}_{\rm f}({\bf x},t).\cr
  }
$$
\eqtitle{Lorenz-villkoret -- Retarderade potentialer}
Under Lorenz-villkoret erh{\aa}lls l{\"o}sningar till v{\aa}g\-ekva\-tion\-erna
f{\"o}r potentialerna som de retarderade potentialerna
$$
  \eqalign{
    \phi({\bf x},t)&={{1}\over{4\pi\varepsilon_0}}\iiint_{{\Bbb R}^3}
      {{\rho({\bf x}',t')}\over{|{\bf x}-{\bf x}'|}}\,dV',\cr
    {\bf A}({\bf x},t)&={{\mu_0}\over{4\pi}}\iiint_{{\Bbb R}^3}
      {{{\bf J}({\bf x}',t')}\over{|{\bf x}-{\bf x}'|}}\,dV',\cr
  }
$$
d{\"a}r $t'=t-|{\bf x}-{\bf x}'|/c$ {\"a}r den {\it retarderade tiden}
fr{\aa}n ${\bf x}'$ till ${\bf x}$.

\eqtitle{Coulomb-villkoret}



\vfill\eject

\eqsection{Vektoridentiteter}
\eqtitle{Notation}
\halign{#\hfil\ &#\hfil&\quad\hfil(#)\cr
  $dl$       & Skal{\"a}rt linjeelement l{\"a}ngs kurva & ${\rm m}$\cr
  $d{\bf l}$ & Riktat linjeelement $d{\bf l}={\bf e}_l dl$
               l{\"a}ngs kurva& ${\rm m}$\cr
  $dS$       & Skal{\"a}rt ytelement & ${\rm m}^2$\cr
  $d{\bf S}$ & Riktat ytelement $d{\bf S}={\bf e}_n dS$ & ${\rm m}^2$\cr
  $dV$       & Volymelement & ${\rm m}^3$\cr
}
\eqtitle{Trippelprodukter}
$$
  \eqalign{
    {\bf a}\cdot({\bf b}\times{\bf c})&=
      {\bf b}\cdot({\bf c}\times{\bf a})=
      {\bf c}\cdot({\bf a}\times{\bf b})\cr
    {\bf a}\times({\bf b}\times{\bf c})&=
      {\bf b}({\bf a}\cdot{\bf c})
        -{\bf c}({\bf a}\cdot{\bf b})
  }
$$
\eqtitle{Produktregler}
$$
  \eqalign{
    \nabla(fg) &= f\nabla g + g\nabla f\cr
    \nabla\cdot(f{\bf a})&=f\nabla\cdot{\bf a}+{\bf a}\cdot\nabla f\cr
    \nabla\times(f{\bf a}) &= f\nabla\times{\bf a} - {\bf a}\times\nabla f\cr
      \nabla({\bf a}\cdot{\bf b})&=
        {\bf a}\times\nabla\times{\bf b}
          +{\bf b}\times\nabla\times{\bf a}\cr
          &\hskip50pt
          +({\bf a}\cdot\nabla){\bf b}
          +({\bf b}\cdot\nabla){\bf a}\cr
    \nabla\cdot({\bf a}\times{\bf b})&=
      {\bf b}\cdot(\nabla\times{\bf a})-{\bf a}\cdot(\nabla\times{\bf b})\cr
      \nabla\times({\bf a}\times{\bf b})&=
        ({\bf b}\cdot\nabla){\bf a}
          -({\bf a}\cdot\nabla){\bf b}\cr
          &\hskip50pt
          +{\bf a}(\nabla\cdot{\bf b})
          -{\bf b}(\nabla\cdot{\bf a})
  }
$$
\eqtitle{``Tricket''}
$$
  \nabla{{1}\over{|{\bf x}-{\bf x}'|}}
    =-\nabla'{{1}\over{|{\bf x}-{\bf x}'|}}
    =-{{({\bf x}-{\bf x}')}\over{|{\bf x}-{\bf x}'|^3}}.
$$
\eqtitle{``Tricket II'' (f{\"o}r att visa att $\nabla\cdot B=0$)\sidx{Tricket
$\displaystyle\nabla{{1}\over{\char124}{\bf x}-{\bf x}'{\char124}}
=-{{({\bf x}-{\bf x}')}\over{{\char124}{\bf x}-{\bf x}'{\char124}^3}}$}}
$$
\nabla\times{{({\bf x}-{\bf x}')}\over{|{\bf x}-{\bf x}'|^3}}=0
$$
\eqtitle{Andraderivator}
$$
  \eqalign{
    \nabla\cdot(\nabla\times{\bf a})&=0\cr
    \nabla\times(\nabla f)&=0\cr
    \nabla\times(\nabla\times{\bf a})
      &=\nabla(\nabla\cdot{\bf a})-\nabla^2{\bf a}\cr
  }
$$
\eqtitle{Gauss teorem (``divergensteoremet'')\sidx{Gauss lag}}
$$
  \int_V(\nabla\cdot{\bf a})\,dV=\oiint_S{\bf a}\cdot d{\bf S}
$$
\eqtitle{Gauss teorem (rotations-versionen)}
$$
  \iiint_V(\nabla\times{\bf a})\,dV=\oiint_S d{\bf S}\times{\bf a}
$$
\eqtitle{Stokes teorem (Kelvin--Stokes teorem, ``rotationsteoremet'')
  \sidx{Stokes teorem}}
$$
  \iint_V(\nabla\times{\bf a})\cdot d{\bf S}=\oint_{\Gamma}{\bf a}\cdot d{\bf l}
$$
\eqtitle{Gradient-teoremet (analysens fundamentalsats)}
$$
  \int^{{\bf x}_b}_{{\bf x}_a}(\nabla f)\cdot d{\bf l}=f({\bf x}_b)-f({\bf x}_a)
$$
\eqtitle{Greens teorem}
$$
  \iiint_V(f\nabla^2g+(\nabla f)\cdot(\nabla g))\,dV
    =\oiint_S(f\nabla g)\cdot d{\bf S}
$$
\eqtitle{Greens andra teorem}
$$
  \iiint_V(f\nabla^2g-g\nabla^2f)\,dV
    =\oiint_S(f\nabla g-g\nabla f)\cdot d{\bf S}
$$
\eqtitle{Diverse anv{\"a}ndbara integraler i vektoranalys}
$$
  \iiint_V(\nabla f)\,dV=\oiint_S f d{\bf S}
$$
$$
  \iint_S(\nabla f)\times d{\bf S}=-\oint_{\Gamma}f\,d{\bf l}
$$
\vfill\eject
% \eqsection{Ofta anv{\"a}ndbara skal{\"a}ra integraler}
\eqsection{Skal{\"a}ra integraler}
$$
  \eqalign{
    \int&x^n\,dx={{1}\over{(n+1)}}x^{n+1},\quad n\ne-1\cr
    \int&{{dx}\over{x}}=\ln x\cr
    \int&\sqrt{x^2+a^2}\,dx={{1}\over{2}}
         [x\sqrt{x^2+a^2}\cr & \hskip80pt+a^2\ln(x+\sqrt{x^2+a^2})]\cr
    \int&{{dx}\over{\sqrt{x^2+a^2}}}=\ln(x+\sqrt{x^2+a^2})\cr
    \int&{{dx}\over{(x^2+a^2)^{3/2}}}={{x}\over{a^2\sqrt{x^2+a^2}}}\cr
    \int&{{dx}\over{\sqrt{a^2-x^2}}}=\arcsin(x/a)\cr
    \int&{{dx}\over{x^2+a^2}}={{1}\over{a}}\arctan(x/a)\cr
    \int&{{dx}\over{\cos^2x}}=\tan(x)\cr
    \int&{{dx}\over{\sin x}}=\ln|\tan(x/2)|\cr
    \int&\ln x\,dx=x\ln x - x\cr
  }
$$
\vfill\eject

\eqsection{Serieutvecklingar}
\eqtitle{Endimensionella Maclaurin-utvecklingar}
$$
  \eqalign{
    \sin(x)&=x-{{x^3}\over{3!}}+{{x^5}\over{5!}}-{{x^7}\over{7!}}+\ldots\cr
%            =\sum^{\infty}_{n=0}(-1)^n{{x^{2n+1}}\over{(2n+1)!}}\cr
    \cos(x)&=1-{{x^2}\over{2!}}+{{x^4}\over{4!}}-{{x^6}\over{6!}}+\ldots\cr
    {{1}\over{1+x}}&=1-x+x^2-x^3+\ldots\cr
    {{1}\over{1-x}}&=1+x+x^2+x^3+\ldots\cr
    \exp(x)&=1+x+{{x^2}\over{2}}+{{x^3}\over{3}}+\ldots\cr
    \ln(1+x)&=x-{{x^2}\over{2}}+{{x^3}\over{3}}-{{x^4}\over{4}}+\ldots\cr
  }
$$
\eqtitle{Tredimensionell Taylor-utveckling runt ${\bf x}={\bf x}_0$}
$$
  \eqalign{
  f({\bf x}) &= \sum^{\infty}_0 {{1}\over{n!}}
    [({\bf x}-{\bf x}_0)\cdot\nabla]^n f({\bf x}_0)\cr
  }
$$
\eqtitle{Tredimensionell Maclaurin-utveckling}
$$
  \eqalign{
  f({\bf x}) &= f({\bf 0})
  +\sum^{3}_{k=1} x'_k
     {{\partial f({\bf x}')}
       \over{\partial x'_k}}\bigg|_{{\bf x}'={\bf 0}}\cr
       &\qquad
  +{{1}\over{2}}\sum^{3}_{j=1}\sum^{3}_{k=1} x'_j x'_k
     {{\partial^2 f({\bf x}')}
       \over{\partial x'_j\partial x'_k}}\bigg|_{{\bf x}'={\bf 0}}
  +\ldots\cr
  }
$$
\vfill\eject

\eqsection{Koordinatsystem}
\eqtitle{Kartesiska koordinater ($x,y,z$)}
\halign{#\hfil\quad & #\hfil\cr
  Ortsvektor   & ${\bf x}={\bf e}_xx+{\bf e}_yy+{\bf e}_zz$ \cr
  Linjeelement & $d{\bf l}={\bf e}_xdx+{\bf e}_ydy+{\bf e}_zdz$ \cr
  Volymelement & $dV=dx\,dy\,dz$ \cr
}
Differentialoperatorer
$$
  \eqalign{
     &\nabla\phi={\bf e}_x{{\partial\phi}\over{\partial x}}
                +{\bf e}_y{{\partial\phi}\over{\partial y}}
                +{\bf e}_z{{\partial\phi}\over{\partial z}}\cr
     &\nabla^2\phi={{\partial^2\phi}\over{\partial x^2}}
                +{{\partial^2\phi}\over{\partial y^2}}
                +{{\partial^2\phi}\over{\partial z^2}}\cr
     &\nabla\cdot{\bf A}={{\partial A_x}\over{\partial x}}
                +{{\partial A_y}\over{\partial y}}
                +{{\partial A_z}\over{\partial z}}\cr
     &\nabla\times{\bf A}=
        {\bf e}_x\Big(
          {{\partial A_z}\over{\partial y}}-{{\partial A_y}\over{\partial z}}
        \Big)
       +{\bf e}_y\Big(
          {{\partial A_x}\over{\partial z}}-{{\partial A_z}\over{\partial x}}
        \Big)
        \cr & \hskip128pt
       +{\bf e}_z\Big(
          {{\partial A_y}\over{\partial x}}-{{\partial A_x}\over{\partial y}}
        \Big)\cr
  }
$$

\eqtitle{Cylindriska koordinater ($r,\varphi,z$)}
\halign{#\hfil\quad & #\hfil\cr
  Ortsvektor    & ${\bf x}={\bf e}_rr+{\bf e}_zz$ \cr
  Enhetsvektorer & ${\bf e}_r={\bf e}_x\cos\varphi+{\bf e}_y\sin\varphi$ \cr
              & ${\bf e}_{\varphi}=-{\bf e}_x\sin\varphi+{\bf e}_y\cos\varphi$ \cr
  Linjeelement  & $d{\bf l}={\bf e}_r\,dr
                     +{\bf e}_{\varphi}r\,d\varphi
                     +{\bf e}_z\,dz$ \cr
  Volymelement  & $dV=r\,dr\,d\varphi\,dz$ \cr
}
Differentialoperatorer
$$
  \eqalign{
     &\nabla\phi={\bf e}_r{{\partial\phi}\over{\partial r}}
                +{\bf e}_{\varphi}{{1}\over{r}}
                  {{\partial\phi}\over{\partial\varphi}}
                +{\bf e}_z{{\partial\phi}\over{\partial z}}\cr
     &\nabla^2\phi={{1}\over{r}}{{\partial}\over{\partial r}}
        \Big(r{{\partial\phi}\over{\partial r}}\Big)
                +{{1}\over{r^2}}{{\partial^2\phi}\over{\partial\varphi^2}}
                +{{\partial^2\phi}\over{\partial z^2}}\cr
     &\nabla\cdot{\bf A}={{1}\over{r}}{{\partial}\over{\partial r}}(rA_r)
                +{{1}\over{r}}{{\partial A_{\varphi}}\over{\partial\varphi}}
                +{{\partial A_z}\over{\partial z}}\cr
     &\nabla\times{\bf A}=
        {\bf e}_r\Big(
          {{1}\over{r}}{{\partial A_z}\over{\partial\varphi}}
            -{{\partial A_{\varphi}}\over{\partial z}}
        \Big)
       +{\bf e}_{\varphi}\Big(
          {{\partial A_r}\over{\partial z}}-{{\partial A_z}\over{\partial r}}
        \Big)
        \cr & \hskip110pt
       +{\bf e}_z{{1}\over{r}}\Big(
          {{\partial}\over{\partial r}}(rA_{\varphi})
            -{{\partial A_r}\over{\partial\varphi}}
        \Big)\cr
  }
$$
\vfill\eject

\eqtitle{Sf{\"a}riska koordinater ($r,\varphi,\theta$)}
\halign{#\hfil\quad & #\hfil\cr
  Ortsvektor    & ${\bf x}={\bf e}_rr$ \cr
}
\halign{#\hfil\quad & #\hfil\cr
  Enhetsvektorer & \cr
  \hskip36pt${\bf e}_r={\bf e}_x\sin\theta\cos\varphi
                        +{\bf e}_y\sin\theta\sin\varphi
                        +{\bf e}_z\cos\theta$ & \cr
  \hskip36pt${\bf e}_{\theta}={\bf e}_x\cos\theta\cos\varphi
                        +{\bf e}_y\cos\theta\sin\varphi
                        -{\bf e}_z\sin\theta$ & \cr
  \hskip36pt${\bf e}_{\varphi}=-{\bf e}_x\sin\varphi+{\bf e}_y\cos\varphi$ & \cr
}
\halign{#\hfil\quad & #\hfil\cr
  Linjeelement  & $d{\bf l}={\bf e}_r\,dr
                    +{\bf e}_{\varphi}r\sin\theta\,d\varphi
                    +{\bf e}_{\vartheta}r\,d\theta$ \cr
  Volymelement  & $dV=r^2\sin\theta\,dr\,d\varphi\,d\theta$ \cr
}
Differentialoperatorer
$$
  \eqalign{
     &\nabla\phi={\bf e}_r{{\partial\phi}\over{\partial r}}
                +{\bf e}_{\varphi}{{1}\over{r\sin\theta}}
                   {{\partial\phi}\over{\partial\varphi}}
                +{\bf e}_{\vartheta}{{1}\over{r}}
                   {{\partial\phi}\over{\partial\theta}}\cr
     &\nabla^2\phi={{1}\over{r^2}}{{\partial}\over{\partial r}}
         \Big(r^2{{\partial\phi}\over{\partial r}}\Big)
       +{{1}\over{r^2\sin^2\theta}}{{\partial^2\phi}\over{\partial\varphi^2}}
          \cr & \hskip90pt
       +{{1}\over{r^2\sin\theta}}{{\partial}\over{\partial\theta}}
           \Big(\sin\theta{{\partial\phi}\over{\partial\theta}}\Big)\cr
     &\nabla\cdot{\bf A}={{1}\over{r^2}}{{\partial}\over{\partial r}}(r^2A_r)
                +{{1}\over{r\sin\theta}}
                   {{\partial A_{\varphi}}\over{\partial\varphi}}
        \cr & \hskip110pt
                +{{1}\over{r\sin\theta}}
                   {{\partial}\over{\partial\theta}}(A_{\theta}\sin\theta)\cr
     &\nabla\times{\bf A}=
        {\bf e}_r{{1}\over{r\sin\theta}}\Big(
          {{\partial}\over{\partial\theta}}(A_{\varphi}\sin\theta)
            -{{\partial A_{\theta}}\over{\partial\varphi}}
        \Big)
        \cr & \hskip90pt
       +{\bf e}_{\varphi}{{1}\over{r}}\Big(
          {{\partial}\over{\partial r}}(rA_{\theta})
              -{{\partial A_r}\over{\partial\theta}}
        \Big)
        \cr & \hskip90pt
       +{\bf e}_{\theta}{{1}\over{r}}\Big(
          {{1}\over{\sin\theta}}{{\partial A_r}\over{\partial\varphi}}
            -{{\partial}\over{\partial r}}(rA_{\varphi})
        \Big)\cr
  }
$$
